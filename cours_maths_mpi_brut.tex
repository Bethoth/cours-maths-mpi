% Set up the document's format to A4 and the font's size to 12pt.
\documentclass[a4paper,12pt]{report}

% Set up the document's title, author and date.
\title{Maths -- MPI}
\author{Romain Bricout}
\date{\today}

% Set up the input's encoding to UTF-8, the document's font and language to T1 (adapted to french) and french (the grammar linter uses this parameter).
\usepackage[utf8]{inputenc}
\usepackage[T1]{fontenc}
\usepackage[frenchb]{babel}

\usepackage[dvipsnames]{xcolor}

% Set up the document's margins.
\usepackage{geometry}
\geometry{hmargin=1.5cm,vmargin=1.5cm}

% The three main maths packages. They are used for a lot of things.
\usepackage{amssymb,amsmath}
\usepackage{mathtools}

% Useful to create nice and easy signs or variations tables.
\usepackage{tkz-tab}

% Useful to create any kind of visual representation (graph functions, illustrate geometry problems, etc)
\usepackage{tikz}
\usetikzlibrary{patterns,angles,quotes,arrows,arrows.meta,bending,matrix,calc}

% Allows to edit the itemize environment's default item document-wide.
\usepackage{enumitem}

% Allows to define \notfoo or \nfoo (not recommended) in order for \not\foo to work as wished.
\usepackage{newtxmath}

\DeclareSymbolFont{CMletters}{OML}{cmm}{m}{it}
\DeclareMathSymbol{\nu}{\mathord}{CMletters}{23}
\DeclareMathSymbol{\delta}{\mathord}{CMletters}{14}
\DeclareMathSymbol{\zeta}{\mathord}{CMletters}{16}
\DeclareMathSymbol{\kappa}{\mathord}{CMletters}{20}
\DeclareMathSymbol{\xi}{\mathord}{CMletters}{24}
\DeclareMathSymbol{\pi}{\mathord}{CMletters}{25}
\DeclareMathSymbol{\upsilon}{\mathord}{CMletters}{29}
\DeclareMathSymbol{\chi}{\mathord}{CMletters}{31}
\DeclareMathSymbol{\omega}{\mathord}{CMletters}{33}
\DeclareMathSymbol{\Phi}{\mathord}{CMletters}{8}
\DeclareMathSymbol{\Omega}{\mathord}{CMletters}{10}
\DeclareMathSymbol{\lambda}{\mathord}{CMletters}{21}

% Makes the table of contents clickable and gives useful commands for links in general.
\usepackage[hypertexnames=false]{hyperref}
\hypersetup{colorlinks=false,linktoc=all}

% Gives the llbracket and rrbracket commands for integer intervals.
\usepackage{stmaryrd}

% Useful to insert nice-looking quotes.
\usepackage{epigraph}

% Allows to insert chapter-specific table of contents.
\usepackage{minitoc}
\mtcselectlanguage{french}
\setcounter{minitocdepth}{6}

% Useful when units are needed.
\usepackage{siunitx}
\sisetup{
locale=FR,
detect-all,
inter-unit-product=\ensuremath{\cdot},
list-final-separator={et},
list-pair-separator={et},
range-phrase={\ensuremath{\xleftrightarrow{}}},
exponent-product=\ensuremath{\cdot},
per-mode=power-positive-first
}

\usepackage[thmmarks,hyperref]{ntheorem}
\makeatletter
\let\old@thm\@thm
\usepackage[lowercase]{theoremref}
\def\@thm#1#2#3{\def\thmref@currname{#3}\old@thm{#1}{#2}{#3}}
\makeatother

% Allows whiteboard digits with \mathds
\usepackage{dsfont}

\usepackage{needspace}

% Useful for better-looking oneline fractions
\usepackage{nicefrac}

% Set up the horizontal space before the first line of a new paragraph to 2em and the vertical space between two paragraphs to 1em.
\setlength{\parindent}{0pt}
\setlength{\parskip}{1em}

% Adds 0.5em to the vertical space between two lines in an align environment. It looks better.
\addtolength{\jot}{0.5em}

% Allows align environment to break if it's too long to fit in the page where it began.
\allowdisplaybreaks[1]

% Trick to make semicolons considered like relation operators (such as =) and therefore being equidistantly spaced from the two numbers around it.
\mathcode`;=\numexpr\mathcode`;-"3000

% Commands for size-adaptative parentheses, brackets, curly brackets, absolute value and magnitude.
\newcommand{\paren}[1]{\left(#1\right)} % (x)
\newcommand{\croch}[1]{\left[#1\right]} % [x]
\newcommand{\accol}[1]{\left\lbrace#1\right\rbrace} % {x}
\newcommand{\abs}[1]{\left\lvert#1\right\rvert} % |x|
\newcommand{\floor}[1]{\left\lfloor#1\right\rfloor} % ⌊x⌋
\newcommand{\ceil}[1]{\left\lceil#1\right\rceil} % ⌈x⌉

% Commands for size-adaptative intervals and integer intervals. The commands' roots are "interv" and "interventier" and the added e or i at the end mean "excluded" and "included" respectively.
\newcommand{\intervii}[2]{\left[#1;#2\right]} % [a;b]
\newcommand{\intervee}[2]{\left]#1;#2\right[} % ]a;b[
\newcommand{\intervie}[2]{\left[#1;#2\right[} % [a;b[
\newcommand{\intervei}[2]{\left]#1;#2\right]} % ]a;b]
\newcommand{\interventierii}[2]{\left\llbracket#1;#2\right\rrbracket} % non-ASCII characters needed
\newcommand{\interventieree}[2]{\left\rrbracket#1;#2\right\llbracket} % non-ASCII characters needed
\newcommand{\interventierie}[2]{\left\llbracket#1;#2\right\llbracket} % non-ASCII characters needed
\newcommand{\interventierei}[2]{\left\rrbracket#1;#2\right\rrbracket} % non-ASCII characters needed

% Commands for usually used sets.
\newcommand{\N}{\mathbb{N}} % natural integers
\newcommand{\Ns}{\mathbb{N}^*}

\newcommand{\Z}{\mathbb{Z}} % relative integers
\newcommand{\Zp}{\mathbb{Z}_+}
\newcommand{\Zs}{\mathbb{Z}^*}
\newcommand{\Zps}{\mathbb{Z}_+^*}

\newcommand{\D}{\mathbb{D}} % decimal numbers
\newcommand{\Dp}{\mathbb{D}_+}
\newcommand{\Dm}{\mathbb{D}_-}
\newcommand{\Ds}{\mathbb{D}^*}
\newcommand{\Dps}{\mathbb{D}_+^*}
\newcommand{\Dms}{\mathbb{D}_-^*}

\newcommand{\Q}{\mathbb{Q}} % rational numbers
\newcommand{\Qp}{\mathbb{Q}_+}
\newcommand{\Qm}{\mathbb{Q}_-}
\newcommand{\Qs}{\mathbb{Q}^*}
\newcommand{\Qps}{\mathbb{Q}_+^*}
\newcommand{\Qms}{\mathbb{Q}_-^*}

\newcommand{\R}{\mathbb{R}} % real numbers
\newcommand{\Rp}{\mathbb{R}_+}
\newcommand{\Rm}{\mathbb{R}_-}
\newcommand{\Rs}{\mathbb{R}^*}
\newcommand{\Rps}{\mathbb{R}_+^*}
\newcommand{\Rms}{\mathbb{R}_-^*}
\newcommand{\Rb}{\overline{\mathbb{R}}}

\newcommand{\C}{\mathbb{C}} % complex numbers
\newcommand{\Cs}{\mathbb{C}^*}

\newcommand{\K}{\mathbb{K}}
\newcommand{\Ks}{\mathbb{K}^*}

\newcommand{\A}{\mathbb{A}}
\renewcommand{\L}[2]{\mathscr{L}\paren{#1,#2}}
\newcommand{\Lc}[2]{\mathscr{L}_c\paren{#1,#2}}
\newcommand{\Lendo}[1]{\mathscr{L}\paren{#1}}

\newcommand{\prem}{\mathbb{P}}

\newcommand{\U}{\mathbb{U}} % complex numbers whose modulus is 1

\renewcommand{\P}[1]{\mathscr{P}\paren{#1}} % subsets of a set
\newcommand{\Pf}[1]{\mathscr{P}_f\paren{#1}} % finite subsets of a set
\newcommand{\F}[2]{\mathscr{F}\paren{#1,#2}} % functions from 1 to 2
\newcommand{\V}[1]{\mathscr{V}\paren{#1}} % neighborhood of a number

% Redefines \Re and \Im to print Re and Im (the same way as ln or lim) instead of fraktur R and I which don't look nice and are less readable.
\renewcommand{\Re}{\operatorname{Re}}
\renewcommand{\Im}{\operatorname{Im}}
\newcommand{\Card}{\operatorname{Card}}

% Command to print an upright e for the exponential instead of a slanted e and put the exponent.
\newcommand{\e}[1]{\mathrm{e}^{#1}}

% Command to print the imaginary i with a little space on the right. This way, the exponents don't look confusing. \i normally prints a dotless i.
\renewcommand{\i}{i\mkern1mu}

% Redefines \vec such that the arrow covers the whole name of the vector.
%\renewcommand{\vec}[1]{\overrightarrow{#1}}

% Commands for 2D and 3D vectors' coordinates
\newcommand{\dcoords}[2]{\begin{pmatrix}#1\\#2\end{pmatrix}}
\newcommand{\tcoords}[3]{\begin{pmatrix}#1\\#2\\#3\end{pmatrix}}

% Redefines binom to print nicer parentheses around the numbers.
\renewcommand{\binom}[2]{\begin{pmatrix}#2\\#1\end{pmatrix}}

% Command for a QED black square. It automatically prints a whitespace before the square such that it looks nice.
\newcommand{\cqfd}{\text{ }\blacksquare}

% Commands with more explicit names for the best way to express divisibility (mid and nmid).
\newcommand{\divise}{\mid}
\newcommand{\notdivise}{\nmid}

% Commands that do the exact same thing but with explicit names for a complex number's conjugate and an event's negation in probability.
\newcommand{\conj}[1]{\overline{#1}}

% Command for a size-adaptative middle bar meaning "such that" (with spacing around it in order to look nice).
\newcommand{\tq}{\;\middle|\;}

% Command with an explicit name for the scalar product.
\newcommand{\scal}{\cdot}
\newcommand{\vecto}{\operatorname{_\wedge}}

% Shortcut for forcing displaystyle in inline mode.
\newcommand{\ds}{\displaystyle}

% Make the not version of implies, impliedby and iff look nice.
\newcommand{\notimp}{\centernot{\imp}}
\newcommand{\notimpr}{\centernot{\impr}}
\newcommand{\notssi}{\centernot{\ssi}}

\renewcommand{\subset}{\subseteq}
\renewcommand{\supset}{\supseteq}
\newcommand{\notsubset}{\centernot{\subset}}
\newcommand{\notsupset}{\centernot{\supset}}

% Shortcut for P(event).
\newcommand{\proba}[1]{\mathbb{P}\paren{#1}}
\newcommand{\probacond}[2]{\mathbb{P}_{#2}\paren{#1}}

% More explicit names for land (logical and) and lor (logical or).
\newcommand{\et}{\land}
\newcommand{\ou}{\lor}
\newcommand{\non}{\lnot}

% Explicitly named environment for tkz-tab tables. Automatically centers the table and handles the tikzpicture environment.
\newenvironment{tkz}
{
\begin{tikzpicture}
}
{
\end{tikzpicture}
}

% More explicitly named commands for the creation of tkz-tab tables.
\newcommand{\tableauinit}[2]{\tkzTabInit{#1}{#2}}
\newcommand{\tableausignes}[1]{\tkzTabLine{#1}}
\newcommand{\tableauvariations}[1]{\tkzTabVar{#1}}

% Shortcut for the curve and the domain of the given function.
\newcommand{\graphe}[1]{\Gamma_{#1}}
\newcommand{\ensembledef}[1]{\mathcal{D}_{#1}}

\renewcommand{\S}[1]{\mathfrak{S}_{#1}}
\newcommand{\frakA}[1]{\mathfrak{A}_{#1}}

\newcommand{\semihrule}{\rule{256.074815pt}{0.4pt}}

% Various environments that create boxes. Each one is one type of thing (example, proof, etc). Each type has its own automatic counter.
\theoremstyle{break}
\theorembodyfont{\upshape}
\theoremheaderfont{\itshape}
\theoremprework{\bigskip\needspace{\baselineskip}\color{green}\hrule\color{black}}
\theorempostwork{\bigskip}
\newtheorem{rem}{Remarque}[chapter]

\theoremstyle{break}
\theorembodyfont{\upshape}
\theoremheaderfont{\itshape}
\theoremprework{\bigskip\needspace{\baselineskip}\color{green}\hrule\color{black}}
\theorempostwork{\bigskip}
\newtheorem{ex}[rem]{Exemple}

\theoremstyle{break}
\theorembodyfont{\upshape}
\theoremheaderfont{\itshape}
\theoremprework{\bigskip\needspace{\baselineskip}\color{green}\hrule\color{black}}
\theorempostwork{\bigskip}
\newtheorem{rappel}[rem]{Rappel}

\theoremstyle{break}
\theorembodyfont{\upshape}
\theoremheaderfont{\itshape}
\theoremprework{\bigskip\needspace{\baselineskip}\color{brown}\hrule\color{black}}
\theorempostwork{\bigskip}
\newtheorem{oubli}[rem]{Oubli}

\theoremstyle{break}
\theorembodyfont{\upshape}
\theoremheaderfont{\normalfont\bfseries}
\theoremprework{\bigskip\needspace{\baselineskip}\color{blue}\hrule\color{black}}
\theorempostwork{\bigskip}
\newtheorem{defi}[rem]{Définition}

\theoremstyle{break}
\theorembodyfont{\upshape}
\theoremheaderfont{\normalfont\bfseries}
\theoremprework{\bigskip\needspace{\baselineskip}\color{blue}\hrule\color{black}}
\theorempostwork{\bigskip}
\newtheorem{reform}[rem]{Reformulation}

\theoremstyle{break}
\theorembodyfont{\upshape}
\theoremheaderfont{\normalfont\bfseries}
\theoremprework{\bigskip\needspace{\baselineskip}\color{magenta}\hrule\color{black}}
\theorempostwork{\bigskip}
\newtheorem{exo}[rem]{Exercice}

\theoremstyle{break}
\theorembodyfont{\upshape}
\theoremheaderfont{\normalfont\bfseries}
\theoremprework{\bigskip\needspace{\baselineskip}\color{magenta}\hrule\color{black}}
\theorempostwork{\bigskip}
\newtheorem{exos}[rem]{\(\star\) Exercice}

\theoremstyle{break}
\theorembodyfont{\upshape}
\theoremheaderfont{\normalfont\bfseries}
\theoremprework{\bigskip\needspace{\baselineskip}\color{magenta}\hrule\color{black}}
\theorempostwork{\bigskip}
\newtheorem{exoss}[rem]{\(\star\star\) Exercice}

\theoremstyle{break}
\theorembodyfont{\upshape}
\theoremheaderfont{\normalfont\bfseries}
\theoremprework{\bigskip\needspace{\baselineskip}\color{magenta}\hrule\color{black}}
\theorempostwork{\bigskip}
\newtheorem{exosss}[rem]{\(\star\star\star\) Exercice}

\theoremstyle{break}
\theorembodyfont{\upshape}
\theoremheaderfont{\normalfont\bfseries}
\theoremprework{\bigskip\needspace{\baselineskip}\color{magenta}\hrule\color{black}}
\theorempostwork{\bigskip}
\newtheorem{exops}[rem]{\(+\star\) Exercice}

\theoremstyle{break}
\theorembodyfont{\upshape}
\theoremheaderfont{\normalfont\bfseries}
\theoremprework{\bigskip\needspace{\baselineskip}\color{magenta}\hrule\color{black}}
\theorempostwork{\bigskip}
\newtheorem{exopss}[rem]{\(+\star\star\) Exercice}

\theoremstyle{break}
\theorembodyfont{\upshape}
\theoremheaderfont{\normalfont\bfseries}
\theoremprework{\bigskip\needspace{\baselineskip}\color{magenta}\hrule\color{black}}
\theorempostwork{\bigskip}
\newtheorem{exopsss}[rem]{\(+\star\star\star\) Exercice}

\theoremstyle{break}
\theorembodyfont{\upshape}
\theoremheaderfont{\normalfont\bfseries}
\theoremprework{\bigskip\needspace{\baselineskip}\color{magenta}\semihrule\color{green}\semihrule\color{black}}
\theorempostwork{\bigskip}
\newtheorem{exoex}[rem]{Exercice/Exemple}

\theoremstyle{break}
\theorembodyfont{\upshape}
\theoremheaderfont{\normalfont\bfseries}
\theoremprework{\bigskip\needspace{\baselineskip}\color{blue}\semihrule\color{red}\semihrule\color{black}}
\theorempostwork{\bigskip}
\newtheorem{defprop}[rem]{Définition/Proposition}

\theoremstyle{break}
\theorembodyfont{\upshape}
\theoremheaderfont{\normalfont\bfseries}
\theoremprework{\bigskip\needspace{\baselineskip}\color{blue}\semihrule\color{red}\semihrule\color{black}}
\theorempostwork{\bigskip}
\newtheorem{deftheo}[rem]{Définition/Théorème}

\theoremstyle{break}
\theorembodyfont{\upshape}
\theoremheaderfont{\normalfont\bfseries}
\theoremprework{\bigskip\needspace{\baselineskip}\color{blue}\hrule\color{black}}
\theorempostwork{\bigskip}
\newtheorem{nota}[rem]{Notation}

\theoremstyle{break}
\theorembodyfont{\upshape}
\theoremheaderfont{\itshape}
\theoremprework{\bigskip\needspace{\baselineskip}\color{blue}\hrule}
\theorempostwork{\hrule\color{black}\needspace{\baselineskip}\bigskip}
\newtheorem*{brouill}{Brouillon}

\theoremstyle{break}
\theorembodyfont{\itshape}
\theoremheaderfont{\normalfont\bfseries}
\theoremprework{\bigskip\needspace{\baselineskip}\color{red}\hrule\color{black}}
\theorempostwork{\bigskip}
\newtheorem{theo}[rem]{Théorème}

\theoremstyle{break}
\theorembodyfont{\itshape}
\theoremheaderfont{\normalfont\bfseries}
\theoremprework{\bigskip\needspace{\baselineskip}\color{red}\hrule\color{black}}
\theorempostwork{\bigskip}
\newtheorem{prop}[rem]{Proposition}

\theoremstyle{break}
\theorembodyfont{\itshape}
\theoremheaderfont{\normalfont\bfseries}
\theoremprework{\bigskip\needspace{\baselineskip}\color{red}\hrule\color{black}}
\theorempostwork{\bigskip}
\newtheorem{cor}[rem]{Corollaire}

\theoremstyle{break}
\theorembodyfont{\itshape}
\theoremheaderfont{\normalfont\bfseries}
\theoremprework{\bigskip\needspace{\baselineskip}\color{red}\hrule\color{black}}
\theorempostwork{\bigskip}
\newtheorem{lem}[rem]{Lemme}

\theoremstyle{break}
\theorembodyfont{\upshape}
\theoremheaderfont{\normalfont\bfseries}
\theoremprework{\bigskip\needspace{\baselineskip}\color{violet}\hrule\color{black}}
\theorempostwork{\bigskip}
\newtheorem{meth}[rem]{Méthode}

\theoremstyle{break}
\theorembodyfont{\upshape}
\theoremheaderfont{\normalfont\bfseries}
\theoremprework{\bigskip\needspace{\baselineskip}\color{violet}\hrule\color{black}}
\theorempostwork{\bigskip}
\newtheorem{appl}[rem]{Application}

\theoremstyle{break}
\theorembodyfont{\upshape}
\theoremheaderfont{\normalfont\bfseries}
\theoremprework{\bigskip\needspace{\baselineskip}\color{violet}\hrule\color{black}}
\theorempostwork{\bigskip}
\newtheorem{abus}[rem]{Abus}

\theoremstyle{break}
\theorembodyfont{\upshape}
\theoremheaderfont{\normalfont\bfseries}
\theoremprework{\bigskip\needspace{\baselineskip}\color{violet}\hrule\color{black}}
\theorempostwork{\bigskip}
\newtheorem{algo}[rem]{Algorithme}

\theoremstyle{break}
\theorembodyfont{\upshape}
\theoremheaderfont{\normalfont\bfseries}
\theoremprework{\bigskip\needspace{\baselineskip}\color{violet}\hrule\color{black}}
\theorempostwork{\bigskip}
\newtheorem{bilan}[rem]{Bilan}

\theoremstyle{break}
\theorembodyfont{\upshape}
\theoremheaderfont{\normalfont\bfseries}
\theoremprework{\bigskip\needspace{\baselineskip}\hrule}
\theorempostwork{\bigskip}
\newtheorem{q}{Question}

\theoremstyle{break}
\theorembodyfont{\upshape}
\theoremheaderfont{\itshape}
\theoremprework{\bigskip\needspace{\baselineskip}\color{BurntOrange}\hrule\color{black}}
\theorempostwork{\bigskip}
\newtheorem{corr}[rem]{Correction}

\theoremstyle{break}
\theorembodyfont{\upshape}
\theoremheaderfont{\itshape}
\theoremsymbol{\ensuremath{\cqfd}}
\theoremprework{\bigskip\needspace{\baselineskip}\color{yellow}\hrule\color{black}}
\theorempostwork{\bigskip}
\newtheorem{dem}[rem]{Démonstration}

% Commands to make proofs easier to write
\newcommand{\impdir}{\fbox{\(\imp\)}~}
\newcommand{\imprec}{\fbox{\(\impr\)}~}
\newcommand{\incdir}{\fbox{\(\subset\)}~}
\newcommand{\increc}{\fbox{\(\supset\)}~}
\newcommand{\leqbox}{\fbox{\(\leq\)}~}
\newcommand{\geqbox}{\fbox{\(\geq\)}~}
\newcommand{\unicite}{\fbox{unicité}~}
\newcommand{\existence}{\fbox{existence}~}
\newcommand{\analyse}{\fbox{analyse}~}
\newcommand{\synthese}{\fbox{synthèse}~}
\newcommand{\conclusion}{\fbox{conclusion}~}

\renewcommand{\to}{\longrightarrow}
\renewcommand{\mapsto}{\longmapsto}

\newcommand{\fonction}[5]{\begin{array}[t]{cccc}#1 : & #2 & \to & #3 \\ & #4 & \mapsto & #5\end{array}}
\newcommand{\fonctionlambda}[4]{\begin{array}[t]{ccc}#1 & \to & #2 \\ #3 & \mapsto & #4\end{array}}

\renewcommand{\leq}{\leqslant}
\renewcommand{\geq}{\geqslant}

\newcommand{\pinf}{+\infty}
\newcommand{\minf}{-\infty}

\newcommand{\id}[1]{\mathrm{id}_{#1}}

\renewcommand{\phi}{\varphi}
\renewcommand{\epsilon}{\varepsilon}

\newcommand{\ind}[1]{\mathds{1}_{#1}}

\newcommand{\iR}{\i\R}

\newcommand{\tcheby}[2]{T_{#1}\paren{#2}}
\newcommand{\utcheby}[2]{U_{#1}\paren{#2}}

\mathcode`l="8000
\begingroup
\makeatletter
\lccode`\~=`\l
\DeclareMathSymbol{\lsb@l}{\mathalpha}{letters}{`l}
\lowercase{\gdef~{\ifnum\the\mathgroup=\m@ne \ell \else \lsb@l \fi}}%
\endgroup

\newcommand{\ensvide}{\varnothing}

\newcommand{\rond}{\circ}

\newcommand{\union}{\cup}
\newcommand{\inter}{\cap}
\newcommand{\bigunion}{\bigcup}
\newcommand{\biginter}{\bigcap}

\newcommand{\ssi}{\iff}
\newcommand{\imp}{\implies}
\newcommand{\impr}{\impliedby}

\newcommand{\excluant}{\setminus}

\newcommand{\littletaller}{\mathchoice{\vphantom{\big|}}{}{}{}}
\newcommand{\restr}[2]{{
\left.\kern-\nulldelimiterspace#1\littletaller\right|_{#2}
}}
\newcommand{\corestr}[2]{{
\left.\kern-\nulldelimiterspace#1\littletaller\right|^{#2}
}}
\newcommand{\restrbar}[1]{{
\left.\kern-\nulldelimiterspace#1\littletaller\right|
}}

\newcommand{\rel}{\mathscr{R}}

\newcommand{\classesdequiv}[1]{\nicefrac{#1}{\sim}}

\newcommand{\majo}[1]{\mathrm{majorants}\paren{#1}}
\newcommand{\mino}[1]{\mathrm{minorants}\paren{#1}}

\newcommand{\ensdiv}[1]{\operatorname{div}\paren{#1}}

\newcommand{\E}[1]{\times 10^{#1}}

\setcounter{secnumdepth}{3}

\newcommand{\guillemets}[1]{\og #1 \fg{}}

\newcommand{\prim}{^{\,\prime}}
\newcommand{\seconde}{^{\,\prime\prime}}

\newcommand{\note}[1]{\textbf{\(\star\star\) #1 \(\star\star\)}}
\newcommand{\cad}{c'est-à-dire }
\newcommand{\Cad}{C'est-à-dire }
\newcommand{\ie}{\textit{i.e.} }
\newcommand{\cf}{\textit{cf.} }
\newcommand{\Cf}{\textit{Cf.} }

\usepackage{xparse}

\NewDocumentCommand{\quantifs}{>{\SplitList{;}}m}{\ProcessList{#1}{\insertquantif}}
\newcommand{\insertquantif}[1]{#1,\;\:}

\DeclareDocumentCommand{\groupe}{m O{+}}{\paren{#1,#2}}
\DeclareDocumentCommand{\anneau}{m O{+} O{\times}}{\paren{#1,#2,#3}}
\DeclareDocumentCommand{\corps}{m O{+} O{\times}}{\paren{#1,#2,#3}}

\DeclareDocumentCommand{\poly}{O{\K} O{X}}{#1\croch{#2}}
\DeclareDocumentCommand{\polydeg}{O{\K} m O{X}}{#1_{#2}\croch{#3}}
\DeclareDocumentCommand{\fracrat}{O{\K} O{X}}{#1\paren{#2}}

\DeclareDocumentCommand{\M}{m O{\K}}{\mathscr{M}_{#1}\paren{#2}}
\DeclareDocumentCommand{\sym}{m O{\K}}{\mathscr{S}_{#1}\paren{#2}}
\DeclareDocumentCommand{\sympos}{m O{\K}}{\mathscr{S}^+_{#1}\paren{#2}}
\DeclareDocumentCommand{\symdefpos}{m O{\K}}{\mathscr{S}^{++}_{#1}\paren{#2}}
\DeclareDocumentCommand{\antisym}{m O{\K}}{\mathscr{A}_{#1}\paren{#2}}
\DeclareDocumentCommand{\GL}{m O{\K}}{\operatorname{GL}_{#1}\paren{#2}}
\DeclareDocumentCommand{\SL}{m O{\K}}{\operatorname{SL}_{#1}\paren{#2}}
\DeclareDocumentCommand{\Mat}{O{\fami{B}} m}{\operatorname*{Mat}_{#1}\paren{#2}}
\newcommand{\pass}[2]{\mathscr{P}_{#1\to#2}}

\DeclareDocumentCommand{\contm}{O{\intervii{a}{b}} O{\K}}{\classe{0}_m\paren{#1,#2}}
\DeclareDocumentCommand{\Esc}{O{\intervii{a}{b}} O{\K}}{\operatorname{Esc}\paren{#1,#2}}

\usepackage{witharrows}

\newcommand{\croix}{^{\times}}

\usepackage{polynom}

\newcommand{\classe}[1]{\mathscr{C}^{#1}}
\newcommand{\ensclasse}[3]{\classe{#1}\paren{#2,#3}}

\newcommand{\deriv}[1]{^{\paren{#1}}}

\usepackage{derivative}
\derivset{\pdv}[delims-eval=.)]
\derivset{\odv}[delims-eval=.)]

\DeclareMathOperator{\Arctan}{Arctan}
\DeclareMathOperator{\Arcsin}{Arcsin}
\DeclareMathOperator{\Arccos}{Arccos}
\DeclareMathOperator{\cotan}{cotan}
\DeclareMathOperator{\sh}{sh}
\DeclareMathOperator{\ch}{ch}
\DeclareMathOperator{\sg}{sg}
\DeclareMathOperator{\supp}{supp}
\DeclareMathOperator{\Supp}{Supp}
\DeclareMathOperator{\rg}{rg}
\DeclareMathOperator{\tr}{tr}

\newcommand{\Hom}[2]{\operatorname{Hom}\paren{#1,#2}}
\newcommand{\Pol}[2]{\operatorname{Pol}\paren{#1,#2}}
\newcommand{\Aut}[1]{\operatorname{Aut}\paren{#1}}
\DeclareDocumentCommand{\Vect}{O{} m}{\operatorname{Vect}_{#1}\paren{#2}}

\newcommand{\diag}[1]{\operatorname{diag}\paren{#1}}

\usepackage{abstract}
\addto\captionsfrench{\renewcommand{\abstractname}{\Large Introduction}}

\newcommand{\inv}{^{-1}}
\newcommand{\etoile}{^{*}}

\newcounter{orcounter}

\newenvironment{orlist}
{
\begin{array}{|l}
\setcounter{orcounter}{0}
}
{
\end{array}
}

\newcommand{\oritem}[1]{%
\ifthenelse{\theorcounter<1}{}{\\ \text{ou} \\}#1\stepcounter{orcounter}
}

\NewDocumentCommand{\orenv}{>{\SplitList{\\}}m}{%
\begin{orlist}\ProcessList{#1}{\oritem}\end{orlist}}

\newcounter{permuitemcounter}

\newcommand{\permuitem}[1]{%
\ifthenelse{\thepermuitemcounter<1}{}{&}#1\stepcounter{permuitemcounter}}

\NewDocumentCommand{\permu}{>{\SplitList{;}}m >{\SplitList{;}}m}{%
\begin{pmatrix}\setcounter{permuitemcounter}{0}\ProcessList{#1}{\permuitem} \\ \setcounter{permuitemcounter}{0}\ProcessList{#2}{\permuitem}\end{pmatrix}}

\NewDocumentCommand{\cycle}{>{\SplitList{;}}m}{%
\begin{pmatrix}\setcounter{permuitemcounter}{0}\ProcessList{#1}{\permuitem}\end{pmatrix}}

\usepackage{pgfplots}

\DeclareDocumentCommand{\pgcd}{o o}{
\IfNoValueTF{#1}{\operatorname{pgcd}}{\operatorname{pgcd}\paren{#1,#2}}
}

\DeclareDocumentCommand{\bezout}{o o}{
\IfNoValueTF{#1}{\operatorname{bezout}}{\operatorname{bezout}\paren{#1,#2}}
}

\usepackage{minted}
\newminted{python}{linenos, breaklines, breakanywhere, breakautoindent,tabsize=4,obeytabs}
\newenvironment{code}{\VerbatimEnvironment\begin{pythoncode}}{\end{pythoncode}}

\newcommand{\valp}[2]{v_{#1}\paren{#2}}

\newcommand{\fami}[1]{\mathscr{#1}}

\newcommand{\echange}{\leftrightarrow}

\newcommand{\trans}[1]{\prescript{t}{}{#1}}

\usepackage{mathdots}

\DeclareDocumentCommand{\detb}{O{\fami{B}}}{{\det}_{#1}}

\usepackage{cancel}

\usepackage{nicematrix}
\NiceMatrixOptions{renew-dots,renew-matrix}

\newcommand{\ps}[2]{\left\langle#1\tq#2\right\rangle}
\newcommand{\ortho}{^{\perp}}

\newcommand{\operp}{\mathrel{%
\begin{tikzpicture}[baseline=-0.25em]
\draw (0,0) circle (0.45em);
\draw (-0.38em,-0.25em) -- (0.38em,-0.25em);
\draw (0,-0.25em) -- (0,0.45em);
\end{tikzpicture}
}%
}

\usepackage{titletoc}
\dottedcontents{section}[5.5em]{}{3.2em}{1pc}

\newcommand{\bouleo}[2]{B\paren{#1,#2}}
\newcommand{\boulef}[2]{\conj{B}\paren{#1,#2}}
\newcommand{\sphere}[2]{S\paren{#1,#2}}

\newcommand{\vdv}[2]{\operatorname{D}_{#1}#2}

\newcommand{\egqd}[1]{\underset{#1}{=}}
\newcommand{\simqd}[1]{\underset{#1}{\sim}}

\newcommand{\tendqd}[1]{\xrightarrow[#1]{}}

\newcommand{\arr}[2]{A_{#2}^{#1}}
\newcommand{\comb}[2]{C_{#2}^{#1}}

\newcommand{\loiuniforme}[1]{\mathscr{U}\paren{#1}}
\newcommand{\loibernoulli}[1]{\mathscr{B}\paren{#1}}
\newcommand{\loibinomiale}[2]{\mathscr{B}\paren{#1,#2}}

\newcommand{\esp}[1]{\operatorname{E}\paren{#1}}
\newcommand{\vari}[1]{\operatorname{V}\paren{#1}}
\newcommand{\cov}[2]{\operatorname{Cov}\paren{#1,#2}}
\newcommand{\ecarttype}[1]{\sigma\paren{#1}}

\renewcommand{\O}[1]{\mathscr{O}\paren{#1}}
\renewcommand{\o}[1]{o\paren{#1}}

\setcounter{MaxMatrixCols}{200}

\newcommand{\Com}[1]{\operatorname{Com}#1}

\usepackage{microtype}

\newcommand{\sig}[1]{\epsilon\paren{#1}}

\newcommand{\legendeexercices}{\(\star\) Exercice proche du cours \\ \(\star\star\) Exercice de difficulté normale \\ \(\star\star\star\) Exercice difficile (voire très difficile) \\ \(+\) Exercice à faire en priorité \\}

\usepackage{accents}

\newcommand{\interieur}[1]{\accentset{\circ}{#1}}

\usepackage{etoolbox}

\DeclareFontFamily{U}{matha}{\hyphenchar\font45}
\DeclareFontShape{U}{matha}{m}{n}{
      <5> <6> <7> <8> <9> <10> gen * matha
      <10.95> matha10 <12> <14.4> <17.28> <20.74> <24.88> matha12
      }{}
\DeclareSymbolFont{matha}{U}{matha}{m}{n}
\DeclareFontSubstitution{U}{matha}{m}{n}

\DeclareFontFamily{U}{mathx}{\hyphenchar\font45}
\DeclareFontShape{U}{mathx}{m}{n}{
      <5> <6> <7> <8> <9> <10>
      <10.95> <12> <14.4> <17.28> <20.74> <24.88>
      mathx10
      }{}
\DeclareSymbolFont{mathx}{U}{mathx}{m}{n}
\DeclareFontSubstitution{U}{mathx}{m}{n}

\DeclareMathDelimiter{\vvvert}{0}{matha}{"7E}{mathx}{"17}
\DeclarePairedDelimiterX{\normesub}[1]
  {\vvvert}
  {\vvvert}
  {\ifblank{#1}{\:\cdot\:}{#1}}

\newcommand{\norme}[1]{\left\|\ifblank{#1}{\:\cdot\:}{#1}\right\|}

\newcommand{\tpt}{\text{pour tout }}
\newcommand{\Tpt}{\text{Pour tout }}

\newcommand{\Ann}[1]{\operatorname{Ann}\paren{#1}}

\DeclareDocumentCommand{\Sp}{O{} m}{\operatorname{Sp}_{#1}\paren{#2}}
\DeclareDocumentCommand{\sep}{O{} m m}{\operatorname{sep}_{#1}\paren{#2,#3}}
\DeclareDocumentCommand{\rac}{O{\K} m}{\operatorname{Z}_{#1}\paren{#2}}

\DeclareDocumentCommand{\diago}{m O{\K}}{\mathscr{D}_{#1}\paren{#2}}
\DeclareDocumentCommand{\Tsup}{m O{\K}}{\mathscr{T}_{#1}^+\paren{#2}}
\DeclareDocumentCommand{\Orth}{m O{\K}}{\mathscr{O}_{#1}\paren{#2}}

\renewcommand{\subsetneq}{\varsubsetneq}

\begin{document}
\renewcommand{\labelitemi}{\(\triangleright\)}
\renewcommand{\labelenumi}{(\arabic{enumi})}

\everymath{\ds}

\maketitle

\begin{abstract}
Ce document réunit l'ensemble de mes cours de Mathématiques de MPI, ainsi que les exercices les accompagnant. Le professeur était M. Walbron. J'ai adapté certaines formulations me paraissant floues ou ne me plaisant pas mais le contenu pur des cours est strictement équivalent.

Les éléments des tables des matières initiale et présentes au début de chaque chapitre sont cliquables (amenant directement à la partie cliquée). C'est également le cas des références à des éléments antérieurs de la forme, par exemple, \guillemets{Démonstration 5.22}.

Cette version ne contient que le contenu des cours imprimés distribués au fil de l'année. Voir l'autre version pour les annotations.
\end{abstract}

\dominitoc\tableofcontents

\part{Cours}

\chapter{Espaces vectoriels normés}

\minitoc

Dans ce chapitre, la lettre \(\K\) désigne \(\R\) ou \(\C\).

\section{Bornes supérieures, bornes inférieures}

\subsection{Borne supérieure d'une partie de \(\R\)}

On rappelle le théorème fondamental, dit \guillemets{théorème (ou axiome) de la borne supérieure}.

\begin{theo}
Toute partie \(A\) de \(\R\), non-vide et majorée, possède une borne supérieure, notée \(\sup A\).

Toute partie \(A\) de \(\R\), non-vide et minorée, possède une borne inférieure, notée \(\inf A\).
\end{theo}

On dispose de caractérisations équivalentes de la borne supérieure.

\begin{prop}\thlabel{prop1}
Soient \(A\) une partie de \(\R\), non-vide et majorée, et \(s\) un réel.

Alors il y a équivalence entre les propositions suivantes :

\begin{itemize}
    \item \(s=\sup A\) \\
    \item \(\begin{dcases}\quantifs{\forall a\in A}a\leq s \\ \quantifs{\forall\epsilon>0;\exists x\in A}s-\epsilon<x\leq s\end{dcases}\) \\
    \item \(\begin{dcases}\quantifs{\forall a\in A}a\leq s \\ \quantifs{\exists\paren{x_n}\in A^\N}x_n\tendqd{n\to\pinf}s\end{dcases}\)
\end{itemize}
\end{prop}

On a évidemment les caractérisations associées à la borne inférieure.

\subsection{Borne supérieure d'une application à valeurs dans \(\R\)}

\begin{defi}
Soient \(X\) un ensemble non-vide et \(f:X\to\R\).

Si \(f\) est majorée sur \(X\), alors on appelle borne supérieure de \(f\) sur \(X\) le réel \(\sup f\paren{X}=\sup_Xf=\sup_{x\in X}f\paren{x}\).

Si \(f\) est minorée sur \(X\), alors on appelle borne inférieure de \(f\) sur \(X\) le réel \(\inf f\paren{X}=\inf_Xf=\inf_{x\in X}f\paren{x}\).
\end{defi}

On déduit de la \thref{prop1} les caractérisations suivantes.

\begin{prop}
Soient \(X\) un ensemble non-vide, \(f:X\to\R\) majorée sur \(X\) et \(s\) un réel.

Alors il y a équivalence entre les propositions suivantes :

\begin{itemize}
    \item \(s=\sup_Xf\) \\
    \item \(\begin{dcases}\quantifs{\forall x\in X}f\paren{x}\leq s \\ \quantifs{\forall\epsilon>0;\exists x\in X}s-\epsilon<f\paren{x}\leq s\end{dcases}\) \\
    \item \(\begin{dcases}\quantifs{\forall x\in X}f\paren{x}\leq s \\ \quantifs{\exists\paren{x_n}\in X^\N}f\paren{x_n}\tendqd{n\to\pinf}s\end{dcases}\)
\end{itemize}
\end{prop}

\subsection{Règles pratiques}

D'abord, des évidences auxquelles on ne pense pas toujours.

\begin{prop}
Soit \(A\) une partie de \(\R\), non-vide et majorée. Alors \(\quantifs{\forall a\in A}a\leq\sup A\).

Soient \(X\) un ensemble non-vide et \(f:X\to\R\) majorée sur \(X\). Alors \(\quantifs{\forall x\in X}f\paren{x}\leq\sup_Xf\).
\end{prop}

En pratique, on n'a pas souvent besoin de connaître la valeur exacte d'une borne supérieure, on a plus souvent besoin de la majorer.

\begin{prop}
\begin{itemize}
    \item Soient \(A\) une partie de \(\R\), non-vide et majorée, et \(M\) un réel. \\ Pour montrer \(\sup A\leq M\), il suffit de montrer \(\quantifs{\forall a\in A}a\leq M\). \\
    \item Soient \(X\) un ensemble non-vide, \(f:X\to\R\) majorée sur \(X\) et \(M\) un réel. \\ Pour montrer \(\sup_Xf\leq M\), il suffit de montrer \(\quantifs{\forall x\in X}f\paren{x}\leq M\).
\end{itemize}
\end{prop}

Multiplication par un réel positif.

\begin{prop}
Soient \(X\) un ensemble non-vide et \(f:X\to\R\) majorée sur \(X\).

Alors \(\quantifs{\tpt\lambda\geq0}\sup_X\paren{\lambda f}=\lambda\sup_Xf\).
\end{prop}

Attention ! C'est bien sûr faux si \(\lambda<0\).

\section{Normes}

\subsection{Définition}

\begin{defi}
Soit \(E\) un \(\K\)-espace vectoriel.

On appelle norme sur \(E\) toute application \(N:E\to\Rp\) telle que :

\begin{itemize}
    \item \(\quantifs{\tpt x\in E}N\paren{x}=0\ssi x=0\qquad\text{(séparation)}\) \\
    \item \(\quantifs{\tpt x\in E;\tpt\lambda\in\K}N\paren{\lambda x}=\abs{\lambda}N\paren{x}\qquad\text{(homogénéité)}\) \\
    \item \(\quantifs{\tpt\paren{x,y}\in E^2}N\paren{x+y}\leq N\paren{x}+N\paren{y}\qquad\text{(inégalité triangulaire)}\).
\end{itemize}

Un espace vectoriel est dit espace vectoriel normé quand on lui associe une norme.
\end{defi}

On déduit de l'inégalité triangulaire une inégalité classique (souvent appelée aussi inégalité triangulaire) : \[\quantifs{\tpt\paren{x,y}\in E^2}\abs{N\paren{x}-N\paren{y}}\leq N\paren{x-y}.\]

Si \(N\) est une norme sur \(E\), alors on peut définir une distance entre deux vecteurs de \(E\) : \(d\paren{u,v}=N\paren{u-v}\).

On définit ainsi une application \(d:E^2\to\Rp\) telle que :

\begin{itemize}
    \item \(\quantifs{\tpt\paren{x,y}\in E^2}d\paren{y,x}=d\paren{x,y}\qquad\text{(symétrie)}\) \\
    \item \(\quantifs{\tpt\paren{x,y}\in E^2}d\paren{x,y}=0\ssi x=y\qquad\text{(sépration)}\) \\
    \item \(\quantifs{\tpt\paren{x,y,z}\in E^3}d\paren{x,z}\leq d\paren{x,y}+d\paren{y,z}\qquad\text{(inégalité triangulaire)}\).
\end{itemize}

\subsection{Exemples fondamentaux}

\begin{itemize}
    \item La valeur absolue dans \(\R\) et le module dans \(\C\) sont des normes. \\
    \item La norme euclidienne habituelle en géométrie plane ou spatiale est une norme. \\
    \item Plus généralement, si \(\ps{\cdot}{\cdot}\) est un produit scalaire sur \(E\), la norme euclidienne associée \(x\mapsto\sqrt{\ps{x}{x}}\) est une norme au sens précédent. \\
    \item Soit \(E\) un \(\K\)-espace vectoriel de dimension finie. On choisit une base de \(E\) \(\fami{B}=\paren{e_1,\dots,e_n}\). Si \(v\) est un vecteur de \(E\), on note \(\paren{v_1,\dots,v_n}\) les coordonnées de \(v\) dans la base \(\fami{B}\). \\ On définit classiquement trois normes sur \(E\) : \[\norme{v}_\infty=\max_{i\in\interventierii{1}{n}}\abs{v_i}\qquad\norme{v}_1=\sum_{i=1}^n\abs{v_i}\qquad\norme{v}_2=\sqrt{\sum_{i=1}^n\abs{v_i}^2}\] appelées respesctivement norme infinie ou norme sup, norme 1 et norme 2. \\ Cas particulier : \(E=\R^n\) muni de la base canonique. \\ Cas particulier : \(E=\M{n,p}\) muni de la base canonique. Si \(A=\paren{a_{ij}}_{\substack{1\leq i\leq n \\ 1\leq j\leq p}}\) est une matrice de \(\M{n,p}\), alors \[\norme{A}_\infty=\max_{\substack{1\leq i\leq n \\ 1\leq j\leq p}}\abs{a_{ij}}\qquad\norme{A}_1=\sum_{\substack{1\leq i\leq n \\ 1\leq j\leq p}}\abs{a_{ij}}\qquad\norme{A}_2=\sqrt{\sum_{\substack{1\leq i\leq n \\ 1\leq j\leq p}}\abs{a_{ij}}^2}\]
    \item Soient \(X\) un ensemble et \(E\) l'ensemble des applications bornées de \(X\) dans \(\K\). La norme sup sur \(E\) est définie par \(\norme{f}_\infty=\sup_{x\in X}\abs{f\paren{x}}\). \\ Cas particulier : si \(X=\N\), \(E\) est l'ensemble des suites bornées et \(\norme{u}_\infty=\sup_{n\in\N}\abs{u_n}\).
\end{itemize}

\begin{prop}
Soient \(E,F\) deux espaces vectoriels normés.

L'application de \(E\times F\) dans \(\Rp\) qui à \(\paren{x,y}\) associe \(\max\paren{\norme{x}_E,\norme{y}_F}\) est une norme.

Autrement dit, le produit de deux espaces vectoriels normés est encore un espace vectoriel normé, résultat qui se généralise par récurrence à un nombre quelconque (fini) d'espaces vectoriels normés.
\end{prop}

\subsection{Normes équivalentes}

\begin{defi}
Soient \(E\) un \(\K\)-espace vectoriel et \(N_1,N_2\) deux normes sur \(E\).

On dit que \(N_1\) et \(N_2\) sont équivalentes quand il existe deux constantes strictement positives \(a,b\) telles que \(\quantifs{\tpt v\in E}aN_1\paren{v}\leq N_2\paren{v}\leq bN_1\paren{v}\).
\end{defi}

\begin{exo}
Montrez que si \(E\) est de dimension finie, les trois normes \(\norme{}_\infty\), \(\norme{}_1\) et \(\norme{}_2\) sont équivalentes.
\end{exo}

\begin{exo}
Soient \(E=\poly[\R]\) et \(P=\sum_{i=0}^na_iX^i\in E\). On pose \(N_1\paren{P}=\sum_{i=0}^n\abs{a_i}\) et \(N_\infty\paren{P}=\max_{0\leq i\leq n}\abs{a_i}\).

Montrez que \(N_1\) et \(N_\infty\) sont des normes sur \(E\).

Montrez qu'elles ne sont pas équivalentes en considérant la suite des polynômes \(P_n=\sum_{i=0}^nX^i\).
\end{exo}

Le résultat suivant est fondamental.

\begin{theo}
Si \(E\) est un \(\K\)-espace vectoriel de dimension finie, alors toutes les normes sur \(E\) sont équivalentes.
\end{theo}

Quand on est en dimension finie, cela signifie que tous les résultats qu'on peut démontrer pour une norme sont à facteurs près valables pour n'importe quelle norme, autrement dit cela nous permettra de choisir la norme que l'on préfère si on ne nous l'impose pas.

Dans toute la suite, \(E\) est un espace vectoriel normé par la norme \(\norme{}\).

\subsection{Boules}

\begin{defi}
Soient \(a\in E\) et \(r\in\Rps\).

On appelle boule ouverte de centre \(a\) et de rayon \(r\) l'ensemble noté \(\bouleo{a}{r}\) défini de la façon suivante : \[\bouleo{a}{r}=\accol{v\in E\tq\norme{v-a}<r}.\]

On appelle boule fermée de centre \(a\) et de rayon \(r\) l'ensemble noté (généralement) \(\boulef{a}{r}\) : \[\boulef{a}{r}=\accol{v\in E\tq\norme{v-a}\leq r}.\]

On appelle sphère de centre \(a\) et de rayon \(r\) l'ensemble (généralement) noté \(\sphere{a}{r}\) : \[\sphere{a}{r}=\accol{v\in E\tq\norme{v-a}=r}.\]
\end{defi}

On appelle boule-unité la boule de centre \(0\) et de rayon \(1\), sphère-unité la sphère de centre \(0\) et de rayon \(1\).

\begin{exo}
Que sont les boules dans \(\R\) ? Que sont les sphères dans \(\R\) ?
\end{exo}

\begin{exo}
On prend \(E=\R^2\) et on définit les normes infinie, 1 et 2 relativement à la base canonique.

Représentez graphiquement les boules-unités pour chacune de ces trois normes.
\end{exo}

\begin{exo}
Montrez que toute boule ouverte est contenue dans une boule fermée et contient une boule fermée de mêmes centres.

Montrez la même chose en inversant les mots \guillemets{ouverte} et \guillemets{fermée}.
\end{exo}

\begin{defi}
Soit \(\paren{x,y}\in E^2\). On note \(\croch{xy}=\accol{tx+\paren{1-t}y\tq t\in\intervii{0}{1}}\), appelé segment (géométrique) d'extrémités \(x\) et \(y\).

Une partie \(A\) de \(E\) est dite convexe quand \(\quantifs{\tpt\paren{x,y}\in A^2}\croch{xy}\subset A\).

On a : \[A\text{ est convexe}\ssi\quantifs{\forall\paren{x,y}\in A^2;\forall t\in\intervii{0}{1}}tx+\paren{1-t}y\in A.\]
\end{defi}

\begin{prop}
Les boules (ouvertes ou fermées) sont des parties convexes.

Les sphères ne sont jamais convexes.

Dans \(\R\), les convexes sont les intervalles.
\end{prop}

\subsection{Parties bornées}

\begin{defi}
On dit qu'une partie \(A\) de \(E\) est bornée quand il existe une boule qui la contient.
\end{defi}

\begin{exo}
Montrez que \(A\) est bornée ssi \(A\) est contenue dans une boule de centre \(0\).

Plus généralement, on choisit arbitrairement un point de \(E\), noté \(x\). Montrez l'équivalence \(A\) est bornée ssi \(A\) est contenue dans une boule de centre \(x\).
\end{exo}

\begin{exo}
Montrez qu'en dimension finie, cette définition ne dépend pas de la norme.
\end{exo}

\begin{prop}
Une partie \(A\) de \(E\) n'est pas bornée ssi il existe une suite \(\paren{v_n}\) à termes dans \(A\) telle que \(\norme{v_n}\tendqd{n\to\pinf}\pinf\).
\end{prop}

\begin{exo}
Dans \(E=\R^2\), on pose \(A=\accol{\paren{x,y}\tq x^4+y^4=20}\) : \(A\) est-elle bornée ? Si oui, pour chacune des normes infinie, 1 et 2, donnez un rayon d'une boule centrée en \(0\) qui contient \(A\).
\end{exo}

\begin{exo}
Même question avec \(E=\C^2\).
\end{exo}

\begin{exo}
Dans \(E=\R^3\), on pose \(B=\accol{\paren{x,y,z}\tq x^2+3y^2+4z^2+2xy+2xz-2yz\leq42}\) : \(B\) est-elle bornée ? Si oui, pour chacune des normes infinie, 1 et 2, donnez un rayon d'une boule centrée en \(0\) qui contient \(B\).
\end{exo}

\begin{exo}
Dans \(\M{2}[\R]\), on note \(\fami{P}\) l'ensemble des matrices de projecteurs : \(\fami{P}\) est-il borné ?
\end{exo}

\begin{defi}
On dit qu'une suite \(v\) à termes dans \(E\) est bornée quand l'ensemble de ses valeurs est borné, autrement dit quand il existe \(M>0\) tel que \(\quantifs{\tpt n\in\N}\norme{v_n}\leq M\).

On dit qu'une fonction \(f\) d'un ensemble \(X\) dans \(E\) est bornée quand l'ensemble de ses valeurs prises sur \(X\) est borné, autrement dit quand il existe \(M>0\) tel que \(\quantifs{\tpt x\in X}\norme{f\paren{x}}\leq M\).
\end{defi}

\begin{exo}
Soit \(u\) une suite complexe arithmético-géométrique de raison \(a\). À quelle condition est-elle bornée ?
\end{exo}

\begin{exo}
Soient \(B,B\prim\) deux boules de \(E\). Si \(\paren{x,x\prim}\in E^2\), on pose \(f\paren{x,x\prim}=d\paren{x,x\prim}\). Montrez que \(f\) est bornée sur \(B\times B\prim\).
\end{exo}

\section{Convergence des suites}

Dans cette section, \(E\) désigne un espace vectoriel normé par la norme \(\norme{}\).

\subsection{Définition}

\begin{defi}
Soient \(u=\paren{u_n}\) une suite à termes dans \(E\) et \(l\in E\).

On dit que la suite \(u\) converge vers \(l\) quand toute boule ouverte de centre \(l\) contient tous les termes de la suite à partir d'un certain rang : \[\quantifs{\forall\epsilon>0;\exists n_0\in\N;\forall n\geq n_0}u_n\in\bouleo{l}{\epsilon}.\]
\end{defi}

\begin{prop}
Dans la définition, on peut remplacer les boules ouvertes par des boules fermées.

On peut réécrire la définition sous deux formes équivalentes : \[\quantifs{\forall\epsilon>0;\exists n_0\in\N;\forall n\geq n_0}\norme{u_n-l}<\epsilon\] ou \[\quantifs{\forall\epsilon>0;\exists n_0\in\N;\forall n\geq n_0}\norme{u_n-l}\leq\epsilon.\]

On peut donc se ramener aux suites réelles positives : la suite vectorielle \(u\) converge vers \(l\) ssi la suite réelle \(\paren{\norme{u_n-l}}\) converge vers \(0\).
\end{prop}

Une suite qui ne converge vers aucun élément de \(E\) est dite divergente.

\subsection{Propriétés usuelles}

\begin{prop}[Unicité de la limite]
Si une suite \(u\in E^\N\) converge vers \(l\in E\), elle ne peut converger vers un autre point de \(E\).

On peut donc noter classiquement \(l=\lim u=\lim_{n\to\pinf}u_n\) ou \(u_n\tendqd{n\to\pinf}l\).
\end{prop}

\begin{prop}
Si une suite \(u\in E^\N\) converge, alors elle est bornée.
\end{prop}

\begin{theo}[Opérations sur les suites convergentes]
Soient \(u,v\in E^\N\) convergeant respectivement vers \(l\) et \(m\) deux éléments de \(E\).

Alors pour tout \(\paren{a,b}\in\K^2\), la suite \(au+bv\) converge vers \(al+bm\).

Soit \(\alpha\in\K^\N\) convergeant vers \(\lambda\in\K\).

Alors la suite \(\alpha u\) converge vers \(\lambda l\).
\end{theo}

\begin{prop}
Toute suite extraite d'une suite convergente converge vers la même limite.

Quasi-réciproque : si \(u\) est une suite telle que les deux suites extraites \(\paren{u_{2n}}\) et \(\paren{u_{2n+1}}\) convergent vers la même limite \(l\), alors \(u\) converge vers \(l\).
\end{prop}

\begin{prop}
Dans un produit de deux espaces vectoriels normés \(E\times F\), une suite \(\paren{u_n}=\paren{\paren{a_n,b_n}}\) converge ssi les suites \(\paren{a_n}\) et \(\paren{b_n}\) convergent dans \(E\), respectivement \(F\).

Dans ce cas, \(\lim\paren{a_n,b_n}=\paren{\lim a_n,\lim b_n}\).

Ce résultat se généralise sans difficulté par récurrence à un nombre quelconque (fini) d'espaces vectoriels normés.
\end{prop}

\subsection{Cas particulier en dimension finie}

Dans cette partie, on suppose que \(E\) est de dimension finie.

\begin{defi}
Soit \(\fami{B}=\paren{e_1,\dots,e_n}\) une base de \(E\).

Pour \(i\in\interventierii{1}{n}\), on appelle \(i\)-ème forme coordonnée (relative à la base \(\fami{B}\)), notée souvent \(d_i\), la forme linéaire qui à un vecteur associe sa \(i\)-ème coordonnée dans la base \(\fami{B}\) : \[\quantifs{\tpt v\in E}v=\sum_{i=1}^nd_i\paren{v}e_i.\]
\end{defi}

\begin{theo}
Soit \(\fami{B}\) une base de \(E\).

Une suite \(u\in E^\N\) converge vers \(l\in E\) ssi pour toute forme coordonnée \(d\) relative à \(\fami{B}\), la suite \(\paren{d\paren{u_n}}\) converge vers \(d\paren{l}\).

Autrement dit, une suite converge ssi ses suites-coordonnées dans n'importe quelle base convergent.

Dans ce cas, la limite de la suite \(u\) est le vecteur \(l\) tel que pour toute forme coordonnée \(d\), \(d\paren{l}=\lim_{n\to\pinf}d\paren{u_n}\).
\end{theo}

\begin{ex}~\\
Si \(M_n=\begin{pmatrix}
1 & \e{-n} \\
\nicefrac{1}{n} & n\sin\paren{\nicefrac{1}{n}}
\end{pmatrix}\), alors la suite de matrices \(\paren{M_n}\) converge vers la matrice \(I_2\).
\end{ex}

\begin{cor}
Si \(E\) est de dimension finie, la convergence d'une suite ne dépend pas du choix de la norme. On peut donc choisir la norme qu'on veut.
\end{cor}

\subsection{Point adhérent à une partie}

\begin{defi}
Soient \(A\) une partie de \(E\) et \(x\in E\).

On dit que \(x\) est un point adhérent à \(A\) quand il existe une suite \(u\in A^\N\) qui converge vers \(x\).

L'adhérence de \(A\) est l'ensemble de ses points adhérents, noté \(\conj{A}\).
\end{defi}

Intuitivement, l'adhérence d'une partie est elle-même à laquelle on ajoute tous les points qui se trouvent sur son bord.

\begin{exo}
Quelle est l'adhérence d'une boule ouverte ?
\end{exo}

\begin{exo}
Quelle est l'adhérence de \(\Z\) dans \(\R\) ?
\end{exo}

\begin{prop}
Soient \(A\) une partie de \(E\) et \(x\in E\).

Alors \(x\) est adhérent à \(A\) ssi toute boule centrée en \(x\) rencontre \(A\).

De manière formalisée : \(x\in\conj{A}\ssi\quantifs{\forall r>0;\exists y\in A}y\in\bouleo{x}{r}\).
\end{prop}

On peut donner la définition de la densité d'une partie.

\begin{defi}
On dit qu'une partie \(A\) est dense dans \(E\) quand \(\conj{A}=E\), \cad qu'on peut trouver des éléments de \(A\) aussi proches de n'importe quel point.
\end{defi}

\begin{ex}
\begin{itemize}
    \item Dans \(\R\), \(\Q\) et \(\R\excluant\Q\) sont denses (\cf cours de première année). \\
    \item \(\GL{n}\) est dense dans \(\M{n}\) (démonstration ultérieure).
\end{itemize}
\end{ex}

\section{Limites de fonctions}

Dans cette section, \(E\) et \(F\) sont deux espaces vectoriels normés par les normes \(\norme{}_E\) et \(\norme{}_F\).

\subsection{Définition}

\begin{defi}
Soient \(f\) une fonction de \(E\) dans \(F\), \(D\) son ensemble de définition, \(a\in\conj{D}\) et \(l\in F\).

On dit que \(f\) a pour limite \(l\) en \(a\) quand \[\quantifs{\forall\epsilon>0;\exists\eta>0;\forall x\in D}\norme{x-a}_E<\eta\imp\norme{f\paren{x}-l}_F<\epsilon.\]
\end{defi}

\begin{rem}
On peut remplacer les inégalités strictes sur les normes par des inégalités larges.

On peut réécrire la définition à l'aide de boules ouvertes (ou fermées) : \[\quantifs{\forall\epsilon>0;\exists\eta>0;\forall x\in D\inter\bouleo{a}{\eta}}f\paren{x}\in\bouleo{l}{\epsilon}.\]

Si \(E\) et \(F\) sont de dimension finie, cette définition ne dépend pas du choix des normes.
\end{rem}

\subsection{Caractérisation séquentielle de la limite}

\begin{theo}
Soient \(f\) une fonction de \(E\) dans \(F\), \(D\) son ensemble de défintion, \(a\in\conj{D}\) et \(l\in F\).

\(f\) a pour limite \(l\) en \(a\) ssi pour toute suite \(u\) à termes dans \(D\) convergeant vers \(a\), la suite \(f\rond u=\paren{f\paren{u_n}}\) converge vers \(l\).
\end{theo}

En pratique, on utilise beaucoup plus souvent le sens direct de l'équivalence précédente.

\subsection{Propriétés usuelles}

\begin{prop}[Unicité de la limite]
Soient \(f\) une fonction de \(E\) dans \(F\), \(D\) son ensemble de définition, \(a\in\conj{D}\) et \(l\in F\).

Si \(f\) a pour limite \(l\) en \(a\), alors elle ne peut avoir d'autre limite que \(l\) en \(a\).

On peut donc noter classiquement \(l=\lim_af=\lim_{x\to a}f\paren{x}\) ou \(f\paren{x}\tendqd{x\to a}l\).
\end{prop}

\begin{prop}
Si \(f\) a pour limite \(l\) en \(a\), alors elle est bornée au voisinage de \(a\).
\end{prop}

\begin{theo}[Opérations sur les limites]
Soient \(f\) et \(g\) deux fonctions de \(E\) dans \(F\), définies sur la même partie \(D\) et ayant respectivement pour limites \(l\) et \(m\) deux éléments de \(F\) en \(a\in\conj{D}\).

Alors pour tout \(\paren{\lambda,\mu}\in\K^2\), la fonction \(\lambda f+\mu g\) a pour limite \(\lambda l+\mu m\) en \(a\).

Soient \(\alpha\) une fonction de \(E\) dans \(\K\) et \(f\) une fonction définie de \(E\) dans \(F\), définies sur la même partie \(D\) et ayant respectivement pour limites \(\beta\in\K\) et \(l\in F\) en \(a\in\conj{D}\).

Alors \(\alpha f\) a pour limite \(\beta l\) en \(a\).
\end{theo}

\begin{prop}
Une fonction \(f=\paren{g,h}\) à valeurs dans un produit d'espaces vectoriels normés a une limite ssi \(g\) et \(h\) ont chacune une limite.

Dans ce cas, \(\lim_af=\paren{\lim_ag,\lim_ah}\).

Ce résultat se généralise sans difficulté par récurrence à un nombre quelconque (fini) d'espaces vectoriels normés.
\end{prop}

\subsection{Cas particulier de la dimension finie}

\begin{theo}
On suppose que \(F\) est de dimension finie. Soit \(\fami{B}\) une base de \(F\).

Soit \(f\) une fonction de \(E\) dans \(F\), \(D\) son ensemble de définition, \(a\in\conj{D}\) et \(l\in F\).

La fonction \(f\) a pour limite \(l\) en \(a\) ssi pour toute forme coordonnée \(d\) relative à \(\fami{B}\), la fonction \(d\rond f\) a pour limite \(d\paren{l}\) en \(a\).

Autrement dit, une fonction a une limite en \(a\) ssi ses fonctions-coordonnées dans n'importe quelle base ont chacune une limite en \(a\).

Dans ce cas, la limite de la fonction \(f\) en \(a\) est le vecteur \(l\) tel que pour tout forme coordonnée \(d\), \(d\paren{l}=\lim_{x\to a}d\paren{f\paren{x}}\).
\end{theo}

\subsection{Composition des limites}

\(G\) désigne un troisième espace vectoriel normé.

\begin{theo}
Soient \(f\) une fonction de \(E\) dans \(F\) et \(D_f\) son ensemble de définition. Soient \(g\) une fonction de \(F\) dans \(G\) et \(D_g\) son ensemble de définition. On suppose que \(f\paren{D_f}\subset D_g\) (condition qui permet de définir la composée \(g\rond f\) sur \(D_f\)).

Soient \(a\in\conj{D_f}\), \(b\in\conj{D_g}\) et \(l\in G\).

Si \(f\) a pour limite \(b\) en \(a\) et \(g\) a pour limite \(l\) en \(b\), alors \(g\rond f\) a pour limite \(l\) en \(a\).

Autrement dit, si \(\begin{dcases}
f\paren{x}\tendqd{x\to a}b \\
g\paren{y}\tendqd{y\to b}l
\end{dcases}\) alors \(g\rond f\paren{x}\tendqd{x\to a}l\).
\end{theo}

\subsection{Extensions des définitions}

D'abord les limites infinies en un point dans le cas où l'espace d'arrivée est \(\R\).

\begin{defi}
Soient \(f\) une fonction de \(E\) dans \(\R\), \(D\) son ensemble de définition et \(a\in\conj{D}\).

On dit que \(f\) a pour limite \(\pinf\) en \(a\) quand \[\quantifs{\forall M>0;\exists\eta>0;\forall x\in D}\norme{x-a}_E\leq\eta\imp f\paren{x}\geq M.\]

On dit que \(f\) a pour limite \(\minf\) en \(a\) quand \[\quantifs{\forall M<0;\exists\eta>0;\forall x\in D}\norme{x-a}_E\leq\eta\imp f\paren{x}\leq M.\]
\end{defi}

Puis les limites en \guillemets{l'infini}.

\begin{defi}
Soient \(f\) une application de \(E\) dans \(F\) et \(l\in F\).

On dit que \(f\) a pour limite \(l\) quand \(\norme{x}\) tend vers l'infini quand \[\quantifs{\forall\epsilon>0;\exists B>0;\forall x\in E}\norme{x}_E\geq B\imp\norme{f\paren{x}-l}_F\leq\epsilon.\]

Dans le cas où \(F=\R\), on dit que \(f\paren{x}\) a pour limite \(\pinf\) quand \(\norme{x}\) tend vers l'infini quand \[\quantifs{\forall M>0;\exists B>0;\forall x\in E}\norme{x}_E\geq B\imp f\paren{x}\geq M.\]

(Définition semblable pour la limite \(\minf\)).
\end{defi}

Enfin, dans le cas où l'espace de départ est \(\R\), on peut parler de limite en l'infini au sens habituel.

\begin{defi}
Soient \(f\) une fonction de \(\R\) dans \(F\), définie sur un ouvert \(\intervee{?}{\pinf}\) et \(l\in F\).

On dit que \(f\paren{x}\) a pour limite \(l\) quand \(x\) tend vers \(\pinf\) quand \[\quantifs{\forall\epsilon>0;\exists B>0;\forall x\geq B}\norme{f\paren{x}-l}\leq\epsilon.\]

(Définition semblable pour la limite \(x\) tend vers \(\minf\)).
\end{defi}

\section{Fonctions continues}

Dans cette section, \(E\) et \(F\) sont des espaces vectoriels normés par les normes \(\norme{}_E\) et \(\norme{}_F\).

\subsection{Continuité en un point}

\begin{prop}
Soient \(f\) une fonction de \(E\) dans \(F\), \(D\) son ensemble de définition, \(a\in\conj{D}\) et \(l\in F\).

Si \(f\) a pour limite \(l\) en \(a\) et si \(a\in D\), alors \(l=f\paren{a}\).
\end{prop}

Dans ce cas, on dit que la fonction \(f\) est continue en \(a\).

\begin{defi}
Soient \(f\) une fonction de \(E\) dans \(F\), \(D\) son ensemble de définition et \(a\in D\).

On dit que \(f\) est continue en \(a\) quand \(f\) a pour limite \(f\paren{a}\) en \(a\).
\end{defi}

On déduit de cette définition et des théorèmes précédents

\begin{itemize}
    \item la caractérisation séquentielle de la continuité en un point ; \\
    \item le fait qu'une fonction continue en un point est bornée au voisinage de ce point ; \\
    \item les théorèmes d'opérations et de compositions des fonctions continues en un point ; \\
    \item l'équivalence entre la continuité d'une fonction et celle de ses fonctions-coordonnées dans une certaine base de \(F\) dans le cas où \(F\) est de dimension finie.
\end{itemize}

\subsection{Continuité sur une partie}

\begin{defi}
Soient \(f\) une fonction de \(E\) dans \(F\), \(D\) son ensemble de définition et \(A\subset D\).

On dit que \(f\) est continue sur \(A\) quand \(f\) est continue en tout point de \(A\).
\end{defi}

On déduit de cette définition et des théorèmes précédents

\begin{itemize}
    \item les théorèmes d'opérations et de compositions des fonctions continues sur une partie ; \\
    \item l'équivalence entre la continuité d'une fonction et celle de ses fonctions-coordonnées dans une certaine base de \(F\) dans le cas où \(F\) est de dimension finie.
\end{itemize}

\begin{prop}
Soient \(f\) et \(g\) deux fonctions de \(E\) dans \(F\) définies sur \(D\) et \(A\subset D\).

Si \(A\) est dense dans \(D\), \(f\) et \(g\) sont continues sur \(D\) et \(f=g\) sur \(A\), alors \(f=g\) sur \(D\).
\end{prop}

\subsection{Cas particulier de la dimension finie}

On suppose que \(E\) et \(F\) sont de dimensions finies.

Dans une base donnée, les formes coordonnées relatives à cette base sont en particulier des applications continues.

Donc toute fonction \(f\) de \(E\) dans \(F\) dont les fonctions-coordonnées \(\paren{f_1,\dots,f_n}\) dans une base de \(F\) sont définies polynomialement à partir des formes coordonnées dans une base de \(E\) est continue.

\begin{ex}
\begin{itemize}
    \item La fonction \(f:\R^2\to\R^2\) telle que \(f\paren{x,y}=\paren{x^2+y^2,xy-\paren{1+x}^3}\) est continue sur \(\R^2\). \\
    \item Les applications trace et déterminant définies sur \(\M{n}\) sont continues.
\end{itemize}
\end{ex}

\begin{exo}
Montrez que l'application \(A\mapsto A^2\) est continue de \(\M{n}\) dans lui-même.
\end{exo}

\begin{exo}
En admettant (momentanément) que \(\GL{n}\) est un ouvert, montrez que l'application \(A\mapsto A\inv\) est continue de \(\GL{n}\) dans lui-même.
\end{exo}

\subsection{Fonctions lipschitziennes}

\begin{defi}
Soient \(f\) une application de \(E\) dans \(F\), \(A\) une partie de \(E\) et \(K\in\Rp\).

On dit que \(f\) est \(K\)-lipschitzienne sur \(A\) (ou lipschitzienne de rapport \(K\)) quand \[\quantifs{\forall\paren{x,y}\in A^2}\norme{f\paren{y}-f\paren{x}}_F\leq K\norme{y-x}_E.\]

On dit que \(f\) est lipschitzienne sur \(A\) quand il existe \(K\in\Rp\) tel que \(f\) soit \(K\)-lipschitzienne sur \(A\).
\end{defi}

\begin{rem}
Si \(f\) est \(K\)-lipschitzienne sur \(A\), alors le rapport \(K\) n'est pas unique, puisque pour tout \(L\geq K\), on a encore \(f\) \(L\)-lipschitzienne sur \(A\).
\end{rem}

\begin{prop}
Toute fonction lipschitzienne est continue.
\end{prop}

Mais la réciproque est fausse (contre-exemple : la fonction \(\sqrt{\cdot}\) sur \(\intervie{0}{\pinf}\)).

Un exemple fondamental : la fonction \(x\mapsto d\paren{x,A}\).

\begin{defi}
Soit \(A\) une partie de \(E\).

Pour \(x\in E\), on appelle distance de \(x\) à \(A\) le réel \(\inf_{a\in A}d\paren{x,a}\).
\end{defi}

\begin{prop}
Pour toute partie \(A\) de \(E\), la fonction \(x\mapsto d\paren{x,A}\) est \(1\)-lipschitzienne.

L'adhérence de \(A\) est l'ensemble des points à distance nulle de \(A\), \ie tels que \(d\paren{x,A}=0\).
\end{prop}

\subsection{Continuité des applications linéaires et \(n\)-linéaires}

\begin{prop}
Soit \(f\in\L{E}{F}\).

Il y a équivalence entre les propositions suivantes :

\begin{itemize}
    \item \(f\) est continue en \(0\) ; \\
    \item \(f\) est continue en un point \(x\) ; \\
    \item \(f\) est continue sur \(E\) ; \\
    \item \(f\) est lipschitzienne sur \(E\) ; \\
    \item il existe \(K\geq 0\) tel que \(\quantifs{\tpt x\in E}\norme{f\paren{x}}_F\leq K\norme{x}_E\) ; \\
    \item \(f\) est bornée sur la boule-unité ; \\
    \item \(f\) est bornée sur une boule.
\end{itemize}
\end{prop}

\begin{exo}
On pose \(E=\ensclasse{0}{\intervii{0}{1}}{\R}\) muni de la norme infinie.

L'application \(f\mapsto\int_0^1f\paren{t}\odif{t}\) est-elle continue sur \(E\) ?
\end{exo}

\begin{exo}
\(E\) désigne le même espace et on pose \(\norme{f}_1=\int_0^1\abs{f\paren{t}}\odif{t}\).

Montrez que \(\norme{}_1\) est une norme sur \(E\).

L'application \(f\mapsto f\paren{1}\) est-elle continue sur \(E\) ?
\end{exo}

\begin{defi}
On note \(\Lc{E}{F}\) l'ensemble des applications linéaires continues de \(E\) dans \(F\).
\end{defi}

\begin{prop}
\(\Lc{E}{F}\) est un sous-espace vectoriel de \(\L{E}{F}\), en général distinct de \(\L{E}{F}\).
\end{prop}

Cas particulier en dimension finie.

\begin{theo}
On suppose que \(E\) est de dimension finie.

Toute application linéaire de \(E\) dans \(F\) est lipschitzienne sur \(E\), donc continue.

Autrement dit, si \(E\) est de dimension finie, alors \(\Lc{E}{F}=\L{E}{F}\).
\end{theo}

\begin{rem}
L'hypothèse de dimension finie de \(E\) est indispensable. Dans le cas contraire, c'est faux en général.
\end{rem}

Le résultat précédent s'étend aux applications multilinéaires.

\begin{theo}
Soient \(E_1,\dots,E_n\) des espaces vectoriels normés de dimensions finies et \(f:E_1\times\dots\times E_n\to F\) une application \(n\)-linéaire.

Il existe alors une constante \(K\geq0\) telle que \[\quantifs{\tpt\paren{x_1,\dots,x_n}\in E_1\times\dots\times E_n}\norme{f\paren{x_1,\dots,x_n}}\leq K\norme{x_1}_{E_1}\dots\norme{x_n}_{E_n}.\]
\end{theo}

\begin{cor}
Soient \(E_1,\dots,E_n\) des espaces vectoriels normés de dimensions finies.

Toute application \(f:E_1\times\dots\times E_n\to F\) qui est \(n\)-linéaire est continue sur \(E_1\times\dots\times E_n\).
\end{cor}

\begin{ex}
\begin{itemize}
    \item Le produit matriciel de \(\M{np}\times\M{pq}\) dans \(\M{nq}\) est bilinéaire, donc continu. \\
    \item Un produit scalaire dans un espace euclidien est bilinéaire, donc continu. \\
    \item Le déterminant dans \(\M{n}\) est \(n\)-linéaire par rapport aux colonnes, donc il est continu.
\end{itemize}
\end{ex}

\subsection{Norme subordonnée}

On définit sur l'espace vectoriel \(\Lc{E}{F}\) des applications linéaires continues de \(E\) dans \(F\) la notion de norme subordonnée (relative aux deux normes sur \(E\) et \(F\)) ou norme triple.

\begin{defi}
Soit \(f\in\Lc{E}{F}\).

On pose \(\normesub{f}=\sup_{x\in\bouleo{0}{1}}\norme{f\paren{x}}\), appelée la norme subordonnée de \(f\).
\end{defi}

\begin{prop}
Soit \(f\in\Lc{E}{F}\).

Alors \(\normesub{f}\) est

\begin{itemize}
    \item égal à \(\sup_{x\not=0}\dfrac{\norme{f\paren{x}}}{\norme{x}}\), mais aussi à \(\sup_{x\in\sphere{0}{1}}\norme{f\paren{x}}\) ; \\
    \item le plus petit réel positif \(M\) tel que \(\quantifs{\tpt x\in E}\norme{f\paren{x}}\leq M\norme{x}\).
\end{itemize}
\end{prop}

\begin{prop}
Les normes subordonnées sont des normes sur les espaces \(\Lc{E}{F}\).

Elles sont dites sous-multiplicatives : pour toutes applications linéaires continues et composables \(f\) et \(g\), \[\normesub{f\rond g}\leq\normesub{f}\times\normesub{g}.\]
\end{prop}

Comme en dimension finie, on peut représenter par choix de bases les applications linéaires par des matrices, on définit de manière semblable la notion de norme sous-multiplicative de matrices (relativement aux normes) ou norme triple.

\begin{defi}
Soit \(\paren{n,p}\in\paren{\Ns}^2\). On choisit deux normes sur \(\K^p\) et \(\K^n\) (espaces identifiés à ceux des matrices-colonnes).

Pour toute matrice \(A\in\M{np}\), on pose \(\normesub{A}=\sup_{\norme{X}=1}\norme{AX}\).
\end{defi}

\begin{prop}
Des normes étant choisies sur les espaces \(\K^p\) et \(\K^n\), les normes subordonnées sont des normes sur tous les espaces \(\M{np}\).

Elles sont dites sous-multiplicatives : pour toutes matrices multipliables \(A\) et \(B\), \[\normesub{AB}\leq\normesub{A}\times\normesub{B}.\]
\end{prop}

\begin{rem}
Dans le cas où un espace vectoriel normé \(E\) est aussi une \(\K\)-algèbre, on dit qu'il est une algèbre normée quand la norme vérifie en plus la propriété de sous-multiplicativité : \(\quantifs{\forall\paren{x,y}\in E^2}\norme{xy}\leq\norme{x}\cdot\norme{y}\).
\end{rem}

\section{Topologie d'un espace vectoriel normé}

Dans cette section, \(E\) est un espace vectoriel normé.

\subsection{Intérieur d'une partie, voisinage d'un point}

\begin{defi}
Soient \(A\) une partie de \(E\) et \(a\in A\).

On dit que \(a\) est un point intérieur à \(A\) quand on peut trouver un rayon \(r>0\) tel que \(\bouleo{a}{r}\) soit incluse dans \(A\). On dit aussi dans ce cas que \(A\) est un voisinage de \(a\).

L'intérieur de \(A\) est l'ensemble de ses points intérieurs, noté \(\interieur{A}\).

On a : \[a\in\interieur{A}\ssi\quantifs{\exists r>0}\bouleo{a}{r}\subset A.\]
\end{defi}

\begin{exo}
Dans \(\R\), quels sont les intérieurs des parties suivantes : \(\intervii{0}{1}\), \(\intervie{0}{\pinf}\), \(\Q\) ?
\end{exo}

\begin{exo}
Quel est l'intérieur d'une boule de centre \(a\) et de rayon \(r>0\) ?
\end{exo}

\begin{rem}
Cette notion dépend a prori de la norme utilisée. En dimension finie, ce n'est pas le cas : l'intérieur d'une partie d'un espace vectoriel normé de dimension finie ne dépend pas du choix de la norme (pourquoi ?).
\end{rem}

\begin{prop}
Soient \(u\in E^\N\) et \(l\in E\).

La suite \(u\) converge vers \(l\) ssi tout voisinage de \(l\) contient tous les termes de la suite à partir d'un certain rang.
\end{prop}

\subsection{Parties ouvertes}

\begin{defi}
On dit qu'une partie \(A\) de \(E\) est ouverte (ou est un ouvert) quand à tout point de \(a\in A\), on peut associer un rayon \(r>0\) tel que la boule de centre \(a\) et de rayon \(r\) soit incluse dans \(A\) : \[\quantifs{\forall a\in A;\exists r>0}\bouleo{a}{r}\subset A.\]

Autrement dit, \(A\) est ouverte quand tout point de \(A\) est intérieur à \(A\) : \(A=\interieur{A}\), ou, autrement dit, quand \(A\) est un voisinage de chacun de ses points.
\end{defi}

\begin{prop}
L'ensemble vide et \(E\) sont des parties ouvertes. Toute boule ouverte est une partie ouverte. Tout produit (fini) de parties ouvertes est ouvert.
\end{prop}

La topologie de \(E\) est l'ensemble de tous les ouverts de \(E\).

\begin{rem}
La topologie dépend a priori de la norme utilisée. En dimension finie, ce n'est pas le cas : dans un espace vectoriel normé de dimension finie, le fait d'être un ouvert ne dépend pas du choix de la norme.
\end{rem}

\subsection{Parties fermées}

On rappelle la notion de point adhérent à une partie.

\begin{defi}
Soient \(A\) une partie de \(E\) et \(x\in E\).

On dit que \(x\) est un point adhérent à \(A\) quand il existe une suite \(u\in A^\N\) qui converge vers \(x\), ou, ce qui revient au même, quand toute boule centrée en \(x\) rencontre \(A\), ou encore quand \(d\paren{x,A}=0\).

L'adhérence de \(A\) est l'ensemble de ses points adhérents, noté \(\conj{A}\).
\end{defi}

On a montré

\begin{defi}
On dit qu'une partie \(A\) de \(E\) est fermée (ou est un fermé) quand tout point adhérent à \(A\) est dans \(A\), autrement dit quand la propriété suivante est vraie : \[\text{si une suite quelconque à termes dans }A\text{ converge vers un point }x\text{ de }E\text{, alors }x\in A.\]

Ou encore : \(A\) est fermée quand \(A=\conj{A}\).
\end{defi}

\begin{prop}
L'ensemble vide et \(E\) sont des parties fermées. Toute boule fermée est une partie fermée. Tout produit (fini) de parties fermées est fermé.
\end{prop}

On note le lien avec les parties ouvertes.

\begin{prop}
Soit \(A\) une partie de \(E\).

Alors \(A\) est une partie ouverte ssi son complémentaire est une partie fermée.
\end{prop}

Encore une fois, le fait d'être un fermé en dimension finie ne dépend pas de la norme.

\begin{prop}
\begin{itemize}
    \item Toute réunion de parties ouvertes est ouverte. Toute intersection finie de parties ouvertes est ouverte. \\
    \item Toute intersection de parties fermées est fermée. Toute réunion finie de parties fermées est fermée.
\end{itemize}
\end{prop}

\begin{exo}
Montrez que pour tout \(a\in E\), \(E\excluant\accol{a}\) est un ouvert. Déduisez-en que si \(A\) est une partie finie de \(E\), alors \(E\excluant A\) est un ouvert.
\end{exo}

\begin{exo}
Quels sont les sous-espaces vectoriels de \(E\) qui sont ouverts ?
\end{exo}

\begin{exo}
Montrez que \(F=\accol{\paren{x,y}\in\R^2\tq x\geq0\text{ et }xy=1}\) est un fermé de \(\R^2\).
\end{exo}

\begin{exo}
On note \(S\) l'ensemble des matrices de \(\M{n}[\R]\) telles que tous les coefficients soient positifs et sur chaque ligne la somme des coefficients vaut \(1\).

Montrez que \(S\) est un fermé.

NB : \(S\) est l'ensemble des matrices dites stochastiques.
\end{exo}

\begin{rem}
A priori, une partie de \(E\) n'est ni ouverte ni fermée : par exemple, dans \(\R\), l'ensemble \(\intervei{0}{1}\) n'est ni ouvert ni fermé.

Donc ne pas confondre \guillemets{complémentaire} et \guillemets{contraire} : on peut dire qu'une partie est un fermé quand son complémentaire est un ouvert, mais pas que le contraire d'être un ouvert c'est être un fermé.
\end{rem}

\begin{rem}
Il est souvent assez facile de montrer qu'une partie est un fermé grâce à la caractérisation séquentielle. Donc pour montrer qu'une partie est un ouvert, on montre souvent de cette façon que son complémentaire est un fermé.

Les fermés sont souvent définis par des égalités ou des inégalités larges. Par complémentaire, les ouverts sont souvent définis par des inégalités strictes ou des différences.
\end{rem}

\subsection{Ouverts ou fermés relatifs à une partie}

Les définitions précédentes parlent d'ouverts et de fermés de \(E\). On peut définir ces notions relativement à une partie.

\begin{defi}
Soient \(A\) une partie de \(E\) et \(U\) un sous-ensemble de \(A\).

On dit que \(U\) est un ouvert de \(A\) quand il existe un ouvert \(V\) de \(E\) tel que \(U=A\inter V\).

On dit que \(U\) est un fermé de \(A\) quand il existe un fermé \(V\) de \(E\) tel que \(U=A\inter V\).
\end{defi}

On remarque que les fermés de \(A\) sont les complémentaires dans \(A\) des ouverts de \(A\). On peut caractériser de même une partie \(U\) fermée de \(A\) par l'égalité entre \(U\) et l'ensemble de ses points adhérents dans \(A\).

\subsection{Image réciproque d'un ouvert ou d'un fermé par une fonction continue}

\begin{rappel}
Si \(f\) est une fonction de \(E\) dans \(F\) définie sur \(D_f\) et \(B\subset F\), l'image réciproque de \(B\) par \(f\) est \[f\inv\paren{B}=\accol{x\in D_f\tq f\paren{x}\in B}.\]
\end{rappel}

\begin{theo}
Soit \(f\) une fonction de \(E\) dans \(F\) définie sur \(D\).

Alors on a équivalence entre les propositions suivantes :

\begin{itemize}
    \item \(f\) est continue sur \(D\) ; \\
    \item pour tout fermé \(B\) de \(F\), son image réciproque \(f\inv\paren{B}\) est un fermé de \(D\) ; \\
    \item pour tout ouvert \(B\) de \(F\), son image réciproque \(f\inv\paren{B}\) est un ouvert de \(D\).
\end{itemize}

Ceci est valable en particulier quand \(f\) est une application continue de \(E\) dans \(F\), auquel cas on peut se passer des notions d'ouvert ou fermé relatif.
\end{theo}

\begin{ex}[Cas particuliers fondamentaux]
Si \(f\) est continue sur \(E\) et à valeurs réelles, alors pour tout \(a\in\R\), les ensembles suivants sont des fermés de \(E\) : \[\accol{x\in E\tq f\paren{x}\geq a}\qquad\accol{x\in E\tq f\paren{x}\leq a}\qquad\accol{x\in E\tq f\paren{x}=a}.\]
\end{ex}

\begin{ex}
\begin{itemize}
    \item Les courbes de fonctions continues de \(\R\) dans \(\R\) sont des fermés de \(\R^2\). \\
    \item L'ensemble des matrices de trace nulle est un fermé de \(\M{n}\).
\end{itemize}
\end{ex}

Par passage au complémentaire, si \(f\) est continue sur \(E\) et à valeurs réelles, alors pour tout \(a\in\R\), les ensembles suivants sont des ouverts de \(E\) : \[\accol{x\in E\tq f\paren{x}<a}\qquad\accol{x\in E\tq f\paren{x}>a}\qquad\accol{x\in E\tq f\paren{x}\not=a}.\]

\begin{ex}
\begin{itemize}
    \item L'ensemble des couples \(\paren{x,y}\in\R^2\) tels que \(x>0\) et \(y>x\) est un ouvert de \(\R^2\). \\
    \item \(\GL{n}\) est un ouvert de \(\M{n}\) : si une matrice \(A\) est inversible, alors toutes les matrices proches de \(A\) le sont aussi.
\end{itemize}
\end{ex}

\subsection{Frontière d'une partie}

\begin{defi}
Soit \(A\) une partie de \(E\). On appelle frontière de \(A\) l'ensemble \(\conj{A}\excluant\interieur{A}\).
\end{defi}

\begin{ex}
\begin{itemize}
    \item Si \(B\) est une boule, alors son intérieur est la boule ouverte de même centre et de même rayon, son adhérence est la boule fermée et sa frontière est la sphère. \\
    \item L'ensemble des rationnels est d'intérieur vide, d'adhérence égale à \(\R\) et donc de frontière \(\R\).
\end{itemize}
\end{ex}

\section{Compacité}

Dans cette section, \(E\) est un espace vectoriel normé.

\subsection{Valeurs d'adhérence d'une suite}

\begin{defi}
Soient \(u=\paren{u_n}\in E^\N\) et \(a\in E\).

On dit que \(a\) est une valeur d'adhérence de la suite \(u\) quand il existe une extractrice \(\phi\) telle que la suite extraite \(\paren{u_{\phi\paren{n}}}\) converge vers \(a\).
\end{defi}

Une suite peut avoir une ou plusieurs valeurs d'adhérence ou ne pas avoir de valeur d'adhérence :

\begin{itemize}
    \item la suite \(\paren{n}_{n\in\N}\) n'a pas de valeur d'adhérence ; \\
    \item toute suite convergente possède une seule valeur d'adhérence : sa limite ; \\
    \item la suite \(u\) définie par \(u_{2n}=\dfrac{1}{n+1}\) et \(u_{2n+1}=1-\dfrac{1}{n+1}\) possède deux valeurs d'adhérence : \(0\) et \(1\) ; \\
    \item il est possible de numéroter les rationnels, autrement dit de créer une suite \(u\) qui prend exactement toutes les valeurs rationnelles dans \(\R\) : cette suite a pour valeurs d'adhérence tous les réels.
\end{itemize}

On peut donner une caractérisation équivalente sans passer par la notion de suite extraite.

\begin{prop}
Soient \(u=\paren{u_n}\in E^\N\) et \(a\in E\).

Alors \(a\) est une valeur d'adhérence de \(u\) ssi \(\quantifs{\tpt\epsilon>0}\accol{n\in\N\tq u_n\in\bouleo{a}{\epsilon}}\text{ est infini}\).
\end{prop}

Ceci peut encore être réécrit de la façon suivante.

\begin{prop}
Soient \(u=\paren{u_n}\in E^\N\) et \(a\in E\).

Alors \(a\) est une valeur d'adhérence de \(u\) ssi \(\quantifs{\forall\epsilon>0;\forall N\in\N;\exists n\geq N}\norme{u_n-a}<\epsilon\).
\end{prop}

\begin{exo}
Soit \(u=\paren{u_n}\in E^\N\). Montrez que l'ensemble \(V\) des valeurs d'adhérence de la suite \(u\) est un fermé de \(E\) en utilisant les ensembles \(U_p=\accol{u_n\tq n\geq p}\).
\end{exo}

\subsection{Théorème de Bolzano-Weierstrass}

\begin{theo}
Si \(E\) est de dimension finie, alors toute suite bornée de \(E\) possède une valeur d'adhérence.
\end{theo}

\begin{rem}
Ce théorème est faux en dimension infinie donc il faut bien mettre en valeur la dimension finie.
\end{rem}

On peut ajouter une précision au théorème précédent.

\begin{prop}
Si \(E\) est de dimension finie, alors toute suite bornée de \(E\) qui ne possède qu'une seule valeur d'adhérence est convergente vers cette valeur d'adhérence.
\end{prop}

\subsection{Parties compactes}

\begin{defi}
Soit \(A\) une partie de \(E\).

On dit que \(A\) est une partie compacte de \(E\) (ou un compact de \(E\)) quand toute suite à termes dans \(A\) possède une valeur d'adhérence dans \(A\) (propriété dite de Bolzano-Weierstrass).
\end{defi}

\begin{ex}
\begin{itemize}
    \item Tout segment \(\intervii{a}{b}\) de \(\R\) est un compact et ce sont les seuls intervalles compacts. \(\intervii{0}{1}\union\intervii{2}{3}\) est compact. \\
    \item Dans \(\K^n\), tout pavé \(\prod_{i=1}^n\intervii{a_i}{b_i}\) est un compact. Plus généralement, un produit (fini) de compacts est compact.
\end{itemize}
\end{ex}

Les parties compactes sont donc celles dont on peut extraire des sous-suites convergentes. Un résultat précédent se généralise alors.

\begin{prop}
Si \(A\) est une partie compacte, alors toute suite de \(A\) qui ne possède qu'une seule valeur d'adhérence est convergente vers cette valeur d'adhérence.
\end{prop}

Un compact étant connu, il est facile d'en construire d'autres.

\begin{prop}
Si \(A\) est une partie compacte de \(E\), alors toute partie \(B\) fermée dans \(A\) est aussi compacte.
\end{prop}

Reconnaître si une partie est compacte n'est pas toujours facile. On dispose d'une condition nécessaire, qui est suffisante en dimension finie.

\begin{prop}
Soit \(A\) une partie de \(E\).

Si \(A\) est compacte, alors \(A\) est une partie fermée et bornée.
\end{prop}

La réciproque est hélas fausse en général. Néanmoins, en dimension finie, elle est vraie.

\begin{prop}
Si \(E\) est de dimension finie, alors une partie de \(E\) est compacte ssi elle est fermée et bornée.
\end{prop}

\begin{rem}
En fait, il n'y a qu'en dimension finie que ce résultat est vrai. Un théorème de Riesz affirme que la boule-unité fermée d'un espace vectoriel normé est compacte ssi l'espace est de dimension finie, ce qui revient à dire que l'équivalence précédente n'est valable que dans un espace de dimension finie.

En dimension infinie, il se passe des choses vraiment étranges : les compacts sont des parties très petites et plates, par exemple, un compact est forcément d'intérieur vide. Heureusement, il est plus courant de travailler à notre niveau en dimension finie.
\end{rem}

\begin{ex}
\begin{itemize}
    \item L'ensemble des matrices stochastiques de \(\M{n}[\R]\) est un compact. \\
    \item La boule-unité fermée de \(E=\ensclasse{0}{\intervii{0}{1}}{\R}\) pour la norme infinie n'est pas compacte, car la suite des fonctions \(\paren{x\mapsto x^n}\) a pour seule valeur d'adhérence possible la fonction \(x\mapsto0\) si \(x\not=1\) et \(1\mapsto1\), qui n'est même pas dans l'espace \(E\).
\end{itemize}
\end{ex}

Une application importante de la notion de compacité est le théorème suivant.

\begin{theo}
Tout sous-espace vectoriel de dimension finie de \(E\) est fermé.
\end{theo}

En dimension infinie, là encore il peut se passer des choses étranges : un sous-espace de \(E\) de dimension infinie peut être dense (et donc non-fermé s'il est différent de \(E\)).

\subsection{Théorème des bornes atteintes}

Le principal intérêt des compacts est de pouvoir généraliser un théorème de première année.

\begin{theo}
Soient \(E,F\) deux espaces vectoriels normés, \(A\) une partie de \(E\) et \(f:A\to F\).

Si \(f\) est continue sur \(A\) et \(A\) est compacte, alors \(f\paren{A}\) est compacte.
\end{theo}

On résume en disant que l'image continue d'un compact est un compact.

En particulier, toute fonction continue sur un compact est donc bornée. Dans le cas des fonctions numériques (\ie à valeurs dans \(\R\)), on peut même être plus précis.

\begin{theo}
Toute fonction continue sur un compact et à valeurs réelles est bornée et atteint ses bornes.

Autrement dit, si \(f:A\to\R\) est continue sur \(A\) et \(A\) est une partie compacte de \(E\), alors il existe \(\paren{a,b}\in A^2\) tel que \(\quantifs{\tpt x\in A}f\paren{a}\leq f\paren{x}\leq f\paren{b}\), ce qui revient à dire que \(f\) possède un minimum et un maximum sur \(A\).
\end{theo}

\begin{rem}
Ce théorème est à rapprocher du théorème vu en première année : toute fonction de \(\R\) dans \(\R\) continue sur un segment est bornée et atteint ses bornes.

Néanmoins, le théorème de l'an dernier donnait un résultat un peu plus précis que celui de cette année car il donnait aussi l'image du segment, en précisant qu'il s'agissait aussi d'un segment, car il faisait aussi intervenir le théorème des valeurs intermédiaires.

Ici, dans la version proposée cette année, on ne peut rien dire de plus.
\end{rem}

\begin{exo}
Un exercice classique, à savoir refaire ! C'est la base de nombreux exercices.

Soient \(E\) de dimension finie et \(f:E\to\R\) continue et telle que \(f\paren{x}\) tende vers \(\pinf\) quand \(\norme{x}\) tend vers \(\pinf\). Montrez que \(f\) possède un minimum.

Exemple : dans le plan euclidien géométrique, on choisit trois points \(A,B,C\) ; montrez alors qu'il existe un point \(M\) du plan tel que la somme \(AM+BM+CM\) soit minimale.
\end{exo}

\begin{exo}
Soit \(f:\paren{x,y}\mapsto xy\sqrt{1-x^2-2y^2}\).

Justifiez que l'ensemble de définition \(D\) de \(f\) est un compact de \(\R^2\).

Déterminez les points critiques de \(f\) dans l'ouvert \(\interieur{D}\), puis les maxima et minima de \(f\).
\end{exo}

On retrouve aussi le théorème de Heine en conséquence de la compacité.

\begin{defi}
Soient \(E,F\) deux espaces vectoriels normés, \(A\) une partie de \(E\) et \(f:A\to F\).

On dit que \(f\) est uniformément continue sur \(A\) quand \[\quantifs{\forall\epsilon>0;\exists\eta>0;\forall\paren{x,y}\in A^2}\norme{x-y}\leq\eta\imp\norme{f\paren{x}-f\paren{y}}\leq\epsilon.\]
\end{defi}

\begin{theo}
Soient \(E,F\) deux espaces vectoriels normés, \(A\) une partie de \(E\) et \(f:A\to F\).

Si \(f\) est continue sur \(A\) et \(A\) est compacte, alors \(f\) est uniformément continue sur \(A\).
\end{theo}

\section{Connexité par arcs}

Dans cette section, \(E\) est un espace vectoriel normé.

\subsection{Chemin}

\begin{defi}\thlabel{defi:chemin}
Soient \(A\) une partie de \(E\) et \(a,b\in A\).

On appelle chemin (ou arc) dans \(A\) de \(a\) à \(b\) toute application continue \(\phi:\intervii{0}{1}\to A\) telle que \(\phi\paren{0}=a\) et \(\phi\paren{1}=b\). Le support du chemin est l'image de \(\phi\).
\end{defi}

On peut définir une relation d'équivalence sur une partie de \(E\) en mettant en relation les points joignables par un chemin.

\begin{defi}
Soient \(A\) une partie de \(E\) et \(a,b\in A\).

On pose \(a\rel b\) quand il existe un chemin dans \(A\) de \(a\) à \(b\).
\end{defi}

\begin{prop}
Avec les notations précédentes, la relation \(\rel\) est une relation d'équivalence sur \(A\).
\end{prop}

\subsection{Parties connexes par arcs}

\begin{defi}\thlabel{defi:connexeParArcs}
Soit \(A\) une partie de \(E\).

On dit que \(A\) est connexe par arcs quand tout couple de points \(\paren{a,b}\in A^2\) est joignable par un chemin.
\end{defi}

\begin{ex}
\begin{itemize}
    \item Les parties convexes de \(E\) sont connexes par arcs. \\
    \item Les parties étoilées de \(E\) sont connexes par arcs. \\
    \item \(\Cs\) et \(\C\excluant D\) où \(D\) est la demi-droite des réels négatifs sont connexes par arcs.
\end{itemize}
\end{ex}

Les classes d'équivalences de la relation notée \(\rel\) précédemment s'appellent les composantes connexes par arcs de \(A\) : ce sont par définition des parties connexes par arcs.

\begin{prop}
Les seules parties connexes par arcs de \(\R\) sont les intervalles.
\end{prop}

\begin{rem}
Il existe une notion plus générale, celle de partie connexe : une partie \(A\) de \(E\) est dite connexe quand les seules parties de \(A\) à la fois ouvertes et fermées sont \(\ensvide\) et \(A\). Elle est plus délicate à aborder et est hors-programme, c'est pourquoi on s'en tient à la notion de connexité par arcs (toute partie connexe par arcs est connexe).
\end{rem}

\subsection{Théorème des valeurs intermédiaires}

Là encore, la notion de connexité par arcs permet de généraliser des résultats de première année.

\begin{theo}
Soient \(E,F\) deux espaces vectoriels normés, \(A\) une partie de \(E\) et \(f:A\to F\).

Si \(f\) est continue par \(A\) et \(A\) est connexe par arcs, alors \(f\paren{A}\) est connexe par arcs.
\end{theo}

On résume en disant que l'image continue d'un connexe par arcs est un connexe par arcs.

Dans le cas des fonctions numériques (\ie à valeurs dans \(\R\)), on peut même être plus précis.

\begin{theo}
Toute fonction continue sur un connexe par arcs et à valeurs réelles vérifie la propriété des valeurs intermédiaires.

Autrement dit, si \(f:A\to F\) est continue sur \(A\) une partie connexe par arcs de \(E\), alors \(f\paren{A}\) est un intervalle.

Ou encore : \[\quantifs{\forall\paren{y,z}\in f\paren{A}^2;\forall w\in\croch{yz};\exists t\in A}f\paren{t}=w.\]
\end{theo}


\chapter{Séries numériques et vectorielles : révisions et compléments}

\minitoc

Dans ce chapitre, \(E\) désigne un espace vectoriel normé (qui peut être \(\R\) ou \(\C\)) et \(\norme{}\) la norme associée (qui est dans ces cas la valeur absolue ou le module).

\section{Rappels}

\subsection{Définitions et notations}

\begin{defi}[Série vectorielle]
Soit \(u\) une suite de \(E\).

On associe à cette suite la suite \(s\) définie de la façon suivante : \(\quantifs{\tpt n\in\N}s_n=\sum_{k=0}^nu_k\).

La suite \(s\) est appelée série de terme général \(u_n\) et notée \(\sum_{n\geq0}u_n\) ou \(\sum u\).

Chaque nombre \(s_n\) est appelé somme partielle d'indice \(n\) de la série.
\end{defi}

L'adjectif \guillemets{numérique} associé au mot \guillemets{série} signifie que les termes généraux de la série sont en fait des nombres réels ou complexes.

\subsection{Convergence d'une série}

\begin{defi}
Soit \(u\) une suite de \(E\).

On dit que la série \(\sum u\) converge ssi la suite des sommes partielles \(\paren{s_n}=\paren{\sum_{k=0}^nu_k}\) converge.

Dans ce cas, si \(l=\lim_{n\to\pinf}s_n\), alors \(l\) est appelée somme de la série \(\sum u\) et on note \(l=\sum_{n=0}^{\pinf}u_n\).

On appelle aussi reste partiel d'indice \(n\) de la série le nombre \(r_n=\sum_{k=n+1}^{\pinf}u_k\), de sorte que \(r_n+s_n=l\).

La suite des restes partiels converge donc vers \(0\).

Dans le cas contraire, on dit que la série \(\sum u\) diverge.
\end{defi}

\begin{ex}
\begin{itemize}
    \item Soit \(x\in\C\). La série \(\sum_{n\geq0}x^n\) converge ssi \(\abs{x}<1\) et, dans ce cas, \(\sum_{n=0}^{\pinf}x^n=\dfrac{1}{1-x}\). Cette série est appelée série géométrique de raison \(x\). \\
    \item Les séries de Riemann : \(\sum_{n\geq1}\dfrac{1}{n^\alpha}\) converge ssi \(\alpha>1\). \\
    \item Pour tout \(z\in\C\), la série \(\sum\dfrac{z^n}{n!}\) converge et \(\sum_{n=0}^{\pinf}\dfrac{z^n}{n!}=\e{z}\).
\end{itemize}
\end{ex}

On peut bien sûr généraliser aux séries quelques théorèmes d'opérations.

\begin{prop}
Soient \(u,v\) deux suites de \(E\) et \(\lambda\) un scalaire.

Si les séries \(\sum u\) et \(\sum v\) convergent, alors la série \(\sum\paren{u+\lambda v}\) converge et \(\sum_{n=0}^{\pinf}\paren{u_n+\lambda v_n}=\sum_{n=0}^{\pinf}u_n+\lambda\sum_{n=0}^{\pinf}v_n\).

Ceci prouve aussi que l'ensemble des séries convergentes est un espace vectoriel.
\end{prop}

\begin{rem}
La somme d'une série divergente et d'une série convergente est une série divergente.

En revanche, il n'y a rien à dire a priori à propos de la somme de deux séries divergentes.
\end{rem}

\subsection{Lien entre convergence de suites et convergence de séries}

\begin{prop}
Soit \(u\) une suite de \(E\).

Si la série \(\sum u\) converge, alors la suite \(u\) converge vers \(0\).
\end{prop}

\begin{rem}
\begin{itemize}
    \item La réciproque est fausse. \\
    \item Par contraposition, si une suite \(u\) ne tend pas vers \(0\), alors la série associée diverge : on dit que la série \(\sum u\) diverge grossièrement.
\end{itemize}
\end{rem}

\begin{ex}
On appelle série harmonique la série \(\sum_{n\geq1}\dfrac{1}{n}\).

Cette série diverge, pourtant son terme général tend vers \(0\).
\end{ex}

\begin{defi}
Soit \(u\) une suite de \(E\). On pose \(v_n=u_{n+1}-u_n\).

La série \(\sum v\) est appelée la série télescopique (ou série domino, ou série différence) associée à \(u\).
\end{defi}

\begin{prop}
Une suite converge ssi sa série télescopique associée converge.
\end{prop}

\begin{exo}~\\
On pose \(u_n=\sum_{k=1}^n\dfrac{1}{k}-\ln n\). Montrez que la suite \(u\) converge.
\end{exo}

\section{Séries réelles à termes positifs}

Dans cette section, on s'intéresse uniquement aux séries dont le terme général est un réel positif.

On appelle un premier théorème issu du cours de première année.

\begin{theo}
Soient \(u\) et \(v\) deux suites réelles positives.

\begin{itemize}
    \item Si \(0\leq u\leq v\) et si la série \(\sum v\) converge, alors la série \(\sum u\) converge. \\
    \item Si \(0\leq u\leq v\) et si la série \(\sum u\) diverge, alors la série \(\sum v\) diverge. \\
    \item Si \(u\sim v\), alors les séries \(\sum u\) et \(\sum v\) sont de même nature.
\end{itemize}
\end{theo}

Une application classique : la règle de d'Alembert.

\begin{prop}
Soit \(u\) une suite réelle strictement positive telle que \(\dfrac{u_{n+1}}{u_n}\tendqd{n\to\pinf}l\). Alors

\begin{itemize}
    \item si \(l<1\), la série \(\sum u\) converge ; \\
    \item si \(l>1\), la série \(\sum u\) diverge ; \\
    \item si \(l=1\), on ne peut rien conclure.
\end{itemize}
\end{prop}

\begin{exo}
Soient \(x,y>0\). Représentez graphiquement l'ensemble des couples \(\paren{x,y}\) tels que la série \(\sum\dfrac{x^n}{y^n+n^x}\) converge.
\end{exo}

\begin{exo}
Montrez que la suite \(\paren{u_n}\) définie par \(u_0\in\intervii{0}{1}\) et \(u_{n+1}=\dfrac{1}{2}\paren{u_n+u_n^2}\) converge vers \(0\) et donnez la nature de la série \(\sum u_n\).
\end{exo}

On donne quelques versions plus élaborées du théorème de comparaison.

\subsection{Théorème de Cesàro}

\begin{theo}
Soit \(u\) une suite numérique qui converge vers \(l\). Alors \(\dfrac{u_0+\dots+u_n}{n}\tendqd{n\to\pinf}l\).
\end{theo}

Dans le cas où \(l\not=0\), la série \(\sum u\) diverge grossièrement et \(\sum_{k=0}^nu_k\simqd{n\to\pinf}nl\).

Dans le cas où \(l=0\), on peut juste dire \(\sum_{k=0}^nu_k=\o{n}\).

\begin{exo}
Soit \(u\) la suite définie par récurrence par \(u_0>0\) et \(\quantifs{\tpt n\in\N}u_{n+1}=u_n+\dfrac{u_n+1}{u_n+2}\).

Étudiez la convergence ou divergence de la suite \(u\), puis donnez un équivalent simple de \(u_n\) quand \(n\) tend vers \(\pinf\).
\end{exo}

\subsection{Théorème de comparaison par domination de séries à termes positifs}

Dans le cas convergent d'abord, les restes partiels suivent la même relation de comparaison.

\begin{theo}
Soient \(u,v\) deux suites réelles positives.

Si \(u=\O{v}\) et la série \(\sum v\) converge, alors la série \(\sum u\) converge. De plus, \(\sum_{k=n+1}^{\pinf}u_k=\O{\sum_{k=n+1}^{\pinf}v_k}\).

Si \(u=\o{v}\) et la série \(\sum v\) converge, alors la série \(\sum u\) converge. De plus, \(\sum_{k=n+1}^{\pinf}u_k=\o{\sum_{k=n+1}^{\pinf}v_k}\).
\end{theo}

Dans le cas divergent ensuite, les sommes partielles suivent aussi la même relation de comparaison.

\begin{theo}
Soient \(u,v\) deux suites réelles positives.

Si \(u=\O{v}\) et la série \(\sum u\) diverge, alors la série \(\sum v\) diverge. De plus, \(\sum_{k=0}^nu_k=\O{\sum_{k=0}^nv_k}\).

Si \(u=\o{v}\) et la série \(\sum u\) diverge, alors la série \(\sum v\) diverge. De plus, \(\sum_{k=0}^nu_k=\o{\sum_{k=0}^nv_k}\).
\end{theo}

\subsection{Théorème de comparaison par équivalence de séries à termes positifs}

\begin{theo}
Soient \(u,v\) deux suites réelles positives.

Si \(u\sim v\), alors les séries \(\sum u\) et \(\sum v\) sont de même nature ; l'une converge ssi l'autre converge.

De plus,

\begin{itemize}
    \item si les séries convergent, alors les restes partiels sont équivalents : \(\sum_{k=n+1}^{\pinf}u_k\simqd{n\to\pinf}\sum_{k=n+1}^{\pinf}v_k\) ; \\
    \item si les séries divergent, alors les sommes partielles divergent vers \(\pinf\) et sont équivalentes : \(\sum_{k=0}^nu_k\simqd{n\to\pinf}\sum_{k=0}^nv_k\).
\end{itemize}
\end{theo}

\begin{exo}
Soit \(a>0\). On pose \(u_n=\sin\dfrac{a^n}{n}\) pour \(n\in\Ns\). Selon la valeur de \(a\), déterminez la nature de la série \(\sum_{n\geq1}u_n\).

Montrez que si \(a=1\), alors \(\sum_{k=1}^nu_k\sim\ln n\) et si \(a<1\), \(\sum_{k=n}^{\pinf}u_k=\o{a^n}\).
\end{exo}

\subsection{Théorème de comparaison série - intégrale}

\begin{prop}
Soit \(f\) une fonction continue, positive et décroissante sur \(\Rp\).

Alors la série de terme général \(f\paren{n}\) et la suite de terme général \(\int_0^nf\) sont de même nature.
\end{prop}

\begin{meth}[À retenir]
La technique d'encadrement des sommes partielles d'une série \(\sum f\paren{n}\) (ou des restes partiels) par des intégrales quand \(f\) est continue, positive et monotone.
\end{meth}

\begin{ex}~\\
\begin{itemize}
    \item \(\sum_{k=1}^n\dfrac{1}{k}\simqd{n\to\pinf}\ln n\) (à connaître). \\
    \item Si \(\alpha>1\), un équivalent simple de \(\sum_{k=n+1}^{\pinf}\dfrac{1}{k^\alpha}\) quand \(n\to\pinf\) est \(\dfrac{1}{\paren{\alpha-1}n^{\alpha-1}}\).
\end{itemize}
\end{ex}

\begin{exo}
Pour \(n\in\Ns\), on pose \(u_n=\sum_{k=n}^{\pinf}\dfrac{\ln k}{k^2}\).

Justifiez l'existence de \(u_n\), puis montrez la divergence de la série \(\sum u_n\).

Montrez que \(\sum_{k=1}^nu_k\simqd{n\to\pinf}\dfrac{\ln^2n}{2}\).
\end{exo}

\section{Séries absolument convergentes}

\begin{defi}
Soit \(u\) une suite de \(E\).

On dit que la série \(\sum u\) est absolument convergente ssi la série à termes positifs \(\sum\norme{u}\) est convergente.
\end{defi}

\subsection{Lien entre absolue convergence et convergence}

\begin{theo}
Si \(E\) est de dimension finie, alors toute série absolument convergente est convergente.
\end{theo}

\begin{rem}
\begin{itemize}
    \item La réciproque est fausse : la série \(\sum_{n\geq1}\dfrac{\paren{-1}^n}{n}\) converge (on l'appelle la série harmonique alternée) mais ne converge pas absolument. \\
    \item L'hypothèse de la dimension finie est indispensable. En dimension infinie, ce résultat est faux en général.
\end{itemize}
\end{rem}

\begin{exo}
Soit \(x>0\). Montrez que les séries suivantes convergent : \[\sum_{n\geq2}\dfrac{\ln\paren{n^2+\paren{-1}^nn}}{n^2+\paren{-1}^nx^n}\qquad\sum_{n\geq0}\sqrt{n}\cos\paren{x}\sin^n\paren{x}\qquad\sum_{n\geq0}\dfrac{\paren{-1}^n\sqrt{n+x}}{x^n+n^{\nicefrac{2}{x}}}.\]
\end{exo}

\subsection{Un exemple fondamental : l'exponentielle de matrice}

Soit \(p\in\Ns\).

On choisit comme norme sur \(E=\M{p}[\C]\) une norme sous-multiplicative.

Alors pour tout \(n\in\Ns\), \(\norme{A^n}\leq\norme{A}^n\), donc \(\norme{\dfrac{A^n}{n!}}\leq\dfrac{\norme{A}^n}{n!}\).

Or la série \(\sum\dfrac{\norme{A}^n}{n!}\) converge (et sa somme vaut \(\exp\norme{A}\)), donc par comparaison de séries à termes positifs, la série \(\sum\dfrac{A^n}{n!}\) est absolument convergente.

On pose alors \(\exp A=\sum_{n=0}^{\pinf}\dfrac{A^n}{n!}\).

\subsection{Extension des résultats par comparaison}

\begin{defi}
Soit \(u\) une suite de \(E\) et \(v\) une suite réelle positive.

On dit que \(u=\O{v}\) quand \(\quantifs{\exists M>0;\exists n_0\in\N;\forall n\geq n_0}\norme{u_n}\leq Mv_n\).

On dit que \(u=\o{v}\) quand \(\quantifs{\forall\epsilon>0;\exists n_0\in\N;\forall n\geq n_0}\norme{u_n}\leq\epsilon v_n\).
\end{defi}

\begin{prop}
Soient \(u\) une suite de \(E\) et \(v\) une suite réelle positive.

Si \(E\) est de dimension finie, \(u_n=\O{v_n}\) quand \(n\) tend vers \(\pinf\) et la série \(\sum v\) converge, alors la série \(\sum u\) est absolument convergente.

De plus, \(\sum_{k=n+1}^{\pinf}u_k=\O{\sum_{k=n+1}^{\pinf}v_k}\).
\end{prop}

Ceci est encore valable si \(u_n=\o{v_n}\).

\begin{prop}
Soient \(u\) une suite de \(E\) et \(v\) une suite réelle positive.

Si \(E\) est de dimension finie, \(u_n=\o{v_n}\) quand \(n\) tend vers \(\pinf\) et la série \(\sum v\) converge, alors la série \(\sum u\) est absolument convergente.

De plus, \(\sum_{k=n+1}^{\pinf}u_k=\o{\sum_{k=n+1}^{\pinf}v_k}\).
\end{prop}

\subsection{Produit de Cauchy de deux séries absolument convergentes}

\begin{defi}
Soient \(E\) une algèbre normée de dimension finie, \(\sum_{n\geq0}a_n\) et \(\sum_{n\geq0}b_n\) deux séries à termes dans \(E\).

On appelle produit de Cauchy des deux séries la série \(\sum_{n\geq0}c_n\) où \(\quantifs{\tpt n\in\N}c_n=\sum_{k=0}^na_kb_{n-k}\).
\end{defi}

\begin{rem}
Quand les séries ne commencent pas à partir du rang \(0\), il faut se méfier ! Une idée simple est de se ramener au cas précédent en décalant les indices.

Exemple très courant : les séries commencent au rang \(1\). Dans ce cas, le produit de Cauchy des séries \(\sum_{n\geq1}a_n\) et \(\sum_{n\geq1}b_n\) est la série \(\sum_{n\geq1}c_n\) où \(\quantifs{\tpt n\in\Ns}c_n=\sum_{k=1}^{n}a_kb_{n+1-k}\).
\end{rem}

\begin{theo}
Avec les mêmes hypothèses sur \(E\).

Si les séries \(\sum_{n\geq0}a_n\) et \(\sum_{n\geq0}b_n\) convergent absolument, alors leur produit de Cauchy est aussi absolument convergent et \[\sum_{n=0}^{\pinf}c_n=\sum_{n=0}^{\pinf}a_n\times\sum_{k=0}^{\pinf}b_n.\]
\end{theo}

\section{Séries alternées}

\begin{defi}
Une série alternée est une série réelle \(\sum u_n\) telle que \(\quantifs{\tpt n\in\N}u_{n+1}\text{ est de signe opposé à }u_n\).
\end{defi}

En général, les séries alternées sont reconnaissables à la présence d'un facteur \(\paren{-1}^n\) dans l'expression du terme général.

On dispose d'une condition suffisante de convergence d'une série alternée qu'on appelle le critère spécial des séries alternées.

\begin{theo}
Soit \(\sum\paren{-1}^nu_n\) une série alternée.

Si la suite \(u\)

\begin{itemize}
    \item est positive, \\
    \item est décroissante, \\
    \item et converge vers \(0\),
\end{itemize}

alors la série \(\sum\paren{-1}^nu_n\) converge.

Dans ce cas, la somme de la série est positive, et si on note \(R_n=\sum_{k=n+1}^{\pinf}\paren{-1}^ku_k\) le reste partiel d'indice \(n\), alors \(\quantifs{\tpt n\in\N}R_n\text{ est du signe de son premier terme (\ie du signe de }\paren{-1}^{n+1}\text{) et }\abs{R_n}\leq u_{n+1}\leq u_n\).
\end{theo}

\begin{ex}
\begin{itemize}
    \item La série harmonique alternée \(\sum_{n\geq1}\dfrac{\paren{-1}^n}{n}\) converge. \\
    \item La série \(\sum_{n\geq2}\dfrac{\paren{-1}^n}{\ln n}\) converge.
\end{itemize}
\end{ex}

\begin{rem}
\begin{itemize}
    \item Si \(\sum_{n\geq n_0}\paren{-1}^nu_n\) est une série alternée convergente, sa somme a le signe du premier terme de la série (ici le signe de \(\paren{-1}^{n_0}u_{n_0}\)). \\
    \item La condition de décroissance de la suite \(u\) est essentielle ! Contre-exemple : la série \(\sum_{n\geq2}\dfrac{\paren{-1}^n}{\paren{-1}^n+\sqrt{n}}\) est une série alternée divergente. \\ De plus, cela fournit un contre-exemple au théorème de comparaison par équivalents si on ne tient pas compte de la condition sur le signe, qui doit être constant.
\end{itemize}
\end{rem}

\begin{exo}
Soit \(\alpha>1\). Pour \(n\in\Ns\), on pose \(u_n=\sum_{k=n}^{\pinf}\dfrac{\paren{-1}^k}{k^\alpha+k}\).

Justifiez l'existence de \(u_n\). Montrez que la série \(\sum_{n\geq1}u_n\) converge.
\end{exo}


\chapter{Familles sommables}

\minitoc

\note{À venir}


\chapter{Rappels et compléments d'algèbre linéaire}

\minitoc

\note{À venir}


\chapter{Réduction des endomorphismes}

\minitoc

Dans ce chapitre, \(\K\) désigne un sous-corps de \(\C\), en général \(\R\) ou \(\C\).

\section{Éléments propres d'un endomorphisme}

Dans cette section, \(E\) est un \(\K\)-espace vectoriel de dimension quelconque, finie ou non.

\subsection{Valeurs propres et vecteurs propres}

\begin{defi}
Soient \(f\in\Lendo{E}\) et \(\lambda\in\K\).

On dit que \(\lambda\) est une valeur propre de \(f\) quand il existe un vecteur \(v\) non-nul tel que \(f\paren{v}=\lambda v\).

Si \(\lambda\) est une valeur propre de \(f\), alors tout vecteur non-nul \(v\) tel que \(f\paren{v}=\lambda v\) est appelé vecteur propre associé à la valeur propre \(\lambda\).
\end{defi}

\begin{ex}
\begin{itemize}
    \item Pour tout \(\alpha\in\K\), \(\alpha\id{E}\) a pour unique valeur propre \(\alpha\) et tout vecteur non-nul de \(E\) est un vecteur propre associé. \\
    \item Si \(p\) est un projecteur non-trivial (\ie \(p\not=0\) et \(p\not=\id{E}\)), alors \(p\) a pour seules valeurs propres \(0\) et \(1\). \\
    \item De même, si \(s\) est une symétrie non-triviale (\ie \(s\not=\id{E}\) et \(s\not=-\id{E}\)), alors les valeurs propres de \(s\) sont \(1\) et \(-1\). \\
    \item L'endomorphisme de \(\poly\) \(P\mapsto XP\) n'a pas de valeur propre.
\end{itemize}
\end{ex}

L'ensemble des valeurs propres d'un endomorphisme \(f\) est appelé le spectre de \(f\) et est noté \(\Sp[\K]{f}\) ou plus simplement \(\Sp{f}\) (en toute rigueur, cette définition est fausse en dimension infinie, mais à notre niveau, cette approximation est acceptable).

\begin{defi}
On appelle droite propre d'un endomorphisme toute droite dirigée par un vecteur propre.
\end{defi}

\begin{prop}
Les droites propres d'un endomorphisme sont exactement les droites stables par cet endomorphisme.
\end{prop}

\begin{exo}
Soit \(f\in\Lendo{\R^\N}\) défini par : si \(\paren{u_n}\in\R^\N\), on pose \(f\paren{u}=\paren{u_{n+1}}\). Quelles sont les valeurs propres de \(f\) et les vecteurs propres associés ?
\end{exo}

\begin{exo}
Même question avec \(d\) l'opérateur de dérivation dans \(\ensclasse{\infty}{\R}{\R}\).
\end{exo}

\begin{exo}
Même question avec \(D\) l'opérateur de dérivation dans \(\poly[\R]\).
\end{exo}

\subsection{Lien avec les polynômes annulateurs}

En dimension quelconque, il est souvent difficile de trouver les valeurs propres d'un endomorphisme. La connaissance d'un polynôme annulateur peut aider.

\begin{lem}
Soient \(f\in\Lendo{E}\) et \(P\in\poly\). Si \(\lambda\) est une valeur propre de \(f\) et \(v\) un vecteur propre associé, alors \(P\paren{f}\paren{v}=P\paren{\lambda}v\).
\end{lem}

Si \(P\in\poly\), on note \(\rac{P}\) l'ensemble des racines de \(P\) dans \(\K\).

\begin{prop}
Soit \(f\in\Lendo{E}\).

Si \(P\) est un polynôme annulateur de \(f\), alors \(\Sp{f}\subset\rac{P}\).
\end{prop}

\begin{rem}
Attention ! La réciproque est fausse. Contre-exemple : le polynôme \(P=X^2-1\) est annulateur de \(\id{E}\) et pourtant \(-1\), qui est racine de \(P\), n'est pas valeur propre de \(\id{E}\).
\end{rem}

\begin{exo}
Soit \(n\geq2\). Pour \(M\in\M{n}\), on pose \(f\paren{M}=M+\trans{M}+\tr\paren{M}I_n\) : \(f\) est clairement un endomorphisme de \(\M{n}\).

Déterminez un polynôme annulateur de \(f\) de degré \(3\) et déduisez-en les valeurs propres de \(f\).
\end{exo}

\subsection{Sous-espaces propres}

\begin{prop}
Soient \(f\in\Lendo{E}\) et \(\lambda\in\K\).

Alors \(\lambda\) est valeur propre de \(f\) ssi \(\ker\paren{f-\lambda\id{E}}\not=\accol{0}\), autrement dit ssi \(f-\lambda\id{E}\) n'est pas injectif.
\end{prop}

\begin{defi}
Soit \(f\in\Lendo{E}\).

Si \(\lambda\in\Sp{f}\), le noyau \(\ker\paren{f-\lambda\id{E}}\) est appelé le sous-espace propre associé à la valeur propre \(\lambda\). Il est souvent noté \(\sep{f}{\lambda}\).
\end{defi}

Par conséquent, \(\sep{f}{\lambda}\) est l'ensemble des vecteurs propres associés à la valeur propre \(\lambda\) auquel on ajoute le vecteur nul.

\begin{rem}
Un cas particulier important : \(0\) est valeur propre ssi \(f\) n'est pas injective.
\end{rem}

\begin{exo}
Soit \(u\) un endomorphisme ayant pour matrice \(M=\begin{pmatrix}
-3 & 4 & -4 \\
4 & -3 & 3 \\
4 & -4 & 4
\end{pmatrix}\) dans une certaine base \(\fami{B}\).

Calculez \(M^3+2M^2-3M\). Déduisez-en les valeurs propres de \(u\) puis déterminez les sous-espaces propres associés.
\end{exo}

\begin{prop}
Tout sous-espace propre d'un endomorphisme est stable par cet endomorphisme. L'endomorphisme induit sur un sous-espace propre est alors une homothétie.
\end{prop}

\begin{theo}
Soient \(f\in\Lendo{E}\) et \(\lambda_1,\dots,\lambda_p\) des valeurs propres distinctes de \(f\).

Alors les sous-espaces propres \(\paren{\sep{f}{\lambda_i}}_{1\leq i\leq p}\) sont en somme directe.

Autrement dit, toute famille de vecteurs propres associés à des valeurs propres distinctes est libre.
\end{theo}

\begin{rem}
Quand on demande de déterminer les éléments propres d'un endomorphisme, on demande de déterminer les valeurs propres et les vecteurs propres associés, \ie les sous-espaces propres.
\end{rem}

\begin{center}
\bfseries
\fbox{À partir de maintenant, il est toujours supposé que \(E\) est de dimension finie \(n\)}
\end{center}

\section{Polynôme caractéristique d'un endomorphisme}

\subsection{Caractérisation des valeurs propres en dimension finie}

\begin{prop}
Soient \(f\in\Lendo{E}\) et \(\lambda\in\K\). Alors \[\lambda\in\Sp{f}\ssi\rg\paren{f-\lambda\id{E}}<n.\]

Dans ce cas, \(\dim\sep{f}{\lambda}=n-\rg\paren{f-\lambda\id{E}}\).
\end{prop}

\subsection{Définition et lien avec les valeurs propres}

\begin{defi}
Soit \(f\in\Lendo{E}\).

On appelle polynôme caractéristique de \(f\) le polynôme \(\chi_f=\det\paren{X\id{E}-f}\).
\end{defi}

La notation \(\chi_f\) est très courante : elle est à connaître.

\begin{theo}\thlabel{theo:chiUnitaireEtRacinesEgalesVP}
Soit \(f\in\Lendo{E}\).

Alors \(\chi_f\) est un polynôme unitaire de degré \(n\) de \(\poly\) et les valeurs propres de \(f\) sont exactement les racines dans \(\K\) de \(\chi_f\) : \(\rac{\chi_f}=\Sp{f}\).

Par conséquent, un endomorphisme d'un espace de dimension \(n\) a au plus \(n\) valeurs propres distinctes.
\end{theo}

\begin{exo}
Montrez que si \(\dim E=2\), alors \(\quantifs{\tpt f\in\Lendo{E}}\chi_f=X^2-\tr\paren{f}X+\det f\).
\end{exo}

\begin{exo}
Calculez le polynôme caractéristique d'un endomorphisme de matrice \(\begin{pmatrix}
1 & 4 & 7 \\
2 & 5 & 8 \\
3 & 6 & 9
\end{pmatrix}\) et donnez ses valeurs propres.
\end{exo}

\begin{exo}
Soient \(\fami{B}=\paren{e_1,\dots,e_n}\) une base de \(E\), \(s=\sum_{i=1}^ne_i\) et \(f\in\Lendo{E}\) tel que \(\quantifs{\tpt j\in\interventierii{1}{n}}f\paren{e_j}=e_j+s\).

Calculez son polynôme caractéristique et ses éléments propres.
\end{exo}

On peut noter un lien avec la trace et le déterminant.

\begin{prop}
Soit \(f\in\Lendo{E}\).

Alors \(\chi_f=X^n-\tr\paren{f}X^{n-1}+\dots+\paren{-1}^n\det f\).
\end{prop}

\subsection{Ordre de multiplicité et dimension du sous-espace propre}

\begin{defi}
Soient \(f\in\Lendo{E}\) et \(\lambda\in\Sp{f}\).

On appelle ordre de multiplicité de la valeur propre \(\lambda\) son ordre de multiplicité en tant que racine de \(\chi_f\).
\end{defi}

\begin{prop}
Soient \(f\in\Lendo{E}\), \(F\) un sous-espace vectoriel de \(E\) stable par \(f\) et \(g\) l'endomorphisme induit par \(f\) dans \(F\).

Alors \(\chi_g\) divise \(\chi_f\).
\end{prop}

Une conséquence très importante de ce résultat est le théorème suivant.

\begin{theo}
Soient \(f\in\Lendo{E}\) et \(\lambda\in\Sp{f}\).

Si \(\lambda\) est une valeur propre d'ordre \(\alpha\), alors \(1\leq\dim\sep{f}{\lambda}\leq\alpha\).
\end{theo}

\begin{exo}
Soit \(f\) un endomorphisme de matrice \(\begin{pmatrix}
3 & -4 & -5 \\
-1 & 3 & 2 \\
1 & -2 & -1
\end{pmatrix}\). Déterminez les valeurs propres de \(f\), leur multiplicité et la dimension des sous-espaces propres associés.
\end{exo}

\subsection{Endomorphisme scindé}

\begin{defi}
On dit qu'un endomorphisme de \(E\) est scindé quand son polynôme caractéristique est scindé dans \(\poly\).
\end{defi}

Dans le cas d'un endomorphisme scindé, on connaît alors la somme et le produit des valeurs propres.

\begin{prop}\thlabel{prop:sommeEtProduitVPSiEndomorphismeScinde}
Si \(f\in\Lendo{E}\) est scindé et a pour valeurs propres \(\lambda_1,\dots,\lambda_p\) avec les ordres de multiplicité \(\alpha_1,\dots,\alpha_p\), alors \[\tr f=\sum_{k=1}^p\alpha_k\lambda_k\qquad\text{et}\qquad\det f=\prod_{k=1}^p\lambda_k^{\alpha_k}.\]
\end{prop}

Si \(\K=\C\) alors on est dans ce cas, car tous les polynômes de \(\poly[\C]\) sont scindés dans \(\poly[\C]\) d'après le théorème de d'Alembert-Gauss.

Mais si \(\K=\R\), alors il faut se méfier des raisonnements hâtifs : comme un \(\R\)-endomorphisme peut ne pas avoir de valeurs propres réelles, la trace et le déterminant peuvent ne pas avoir de rapport avec les valeurs propres.

\begin{exo}
Soit \(f\) un endomorphisme d'un \(\C\)-espace vectoriel de dimension \(n\geq2\) dont la matrice dans une base est remplie par ligne de \(1\), ligne de \(2\), etc. Sans calculer le polynôme caractéristique, déterminez les valeurs propres complexes de \(f\), leur multiplicité et la dimension des sous-espaces propres associés.
\end{exo}

\begin{rem}
Dans le langage courant, on dit souvent que la trace est la somme des valeurs propres. Cette phrase est correcte seulement si l'on sous-entend que l'on parle de la somme des valeurs propres comptées chacune avec son ordre de multiplicité.

On rencontre en fait deux types de résultats à propos des valeurs propres :

\begin{itemize}
    \item ceux où l'on parle des valeurs propres distinctes (comme le \thref{theo:chiUnitaireEtRacinesEgalesVP}) ; \\
    \item ceux où l'on parle des valeurs propres comptées selon leur multiplicité (comme la \thref{prop:sommeEtProduitVPSiEndomorphismeScinde}).
\end{itemize}

Il faut donc être très attentif à la façon dont on considère les valeurs propres.
\end{rem}

\section{Éléments propres d'une matrice carrée}

Soit \(n\in\Ns\). Les matrices-colonnes d'ordre \(n\) sont les matrices de \(\M{n\,1}\), espace souvent identifié avec \(\K^n\).

\subsection{Valeurs propres et vecteurs propres}

\begin{defi}
Soient \(A\in\M{n}\) et \(\lambda\in\K\).

On dit que \(\lambda\) est valeur propre de \(A\) quand il existe une matrice-colonne \(X\) non-nulle telle que \(AX=\lambda X\).

Si \(\lambda\) est une valeur propre de \(A\), alors toute matrice-colonne non-nulle \(X\) telle que \(AX=\lambda X\) est appelée vecteur propre associé à la valeur propre \(\lambda\).
\end{defi}

\begin{ex}
\begin{itemize}
    \item Pour tout \(\alpha\in\K\), \(\alpha I_n\) a pour unique valeur propre \(\alpha\) et toute matrice-colonne non-nulle est un vecteur propre associé. \\
    \item Si \(A\) est une matrice diagonale, alors ses valeurs propres sont les coefficients diagonaux et des vecteurs propres associés sont les colonnes remplies de \(0\) sauf un seul coefficient égal à \(1\).
\end{itemize}
\end{ex}

L'ensemble des valeurs propres d'une matrice \(A\) est appelé le spectre de \(A\) et est noté \(\Sp[\K]{A}\) ou plus simplement \(\Sp{A}\).

Mais comme une matrice à coefficients réels est aussi une matrice à coefficients complexes, il vaut mieux savoir si on parle des valeurs propres réelles ou complexes. Il est donc préférable d'indiquer clairement le corps de base, comme le montre le résultat suivant.

\begin{prop}
Soient \(A\in\M{n}\) et \(\K\prim\) une extension de \(\K\) dans \(\C\).

Alors \(\Sp[\K]{A}\subset\Sp[\K\prim]{A}\).
\end{prop}

\begin{prop}
Soient \(A\in\M{n}\), \(f\in\Lendo{E}\) et \(\fami{B}\) une base de \(E\).

Si \(A=\Mat{f}\), alors \(\Sp[\K]{A}=\Sp{f}\).
\end{prop}

Par conséquent, deux matrices semblables ont les mêmes valeurs propres (mais attention, pas forcément les mêmes vecteurs propres).

\subsection{Lien avec les polynômes annulateurs}

\begin{prop}
Soit \(A\in\M{n}\).

Si \(P\) est un polynôme annulateur de \(A\), alors \(\Sp[\K]{A}\subset\rac{P}\).
\end{prop}

Attention ! La réciproque est fausse. Contre-exemple : le polynôme \(P=X^2-1\) est annulateur de \(I_n\) et pourtant \(-1\), qui est racine de \(P\), n'est pas valeur propre de \(I_n\).

\subsection{Sous-espaces propres}

\begin{prop}
Soient \(A\in\M{n}\) et \(\lambda\in\K\).

Alors \(\lambda\) est valeur propre de \(A\) ssi \(A-\lambda I_n\) n'est pas inversible, autrement dit ssi \(\rg\paren{A-\lambda I_n}<n\) ou \(\det\paren{A-\lambda I_n}=0\).
\end{prop}

Si \(\lambda\in\Sp[\K]{A}\), le sous-espace propre associé à la valeur propre \(\lambda\) est l'ensemble des vecteurs propres associés à la valeur propre \(\lambda\) auquel on ajoute le vecteur nul. Il est souvent noté \(\sep[\K]{A}{\lambda}\) : \[\sep[\K]{A}{\lambda}=\accol{X\in\M{n\,1}\tq AX=\lambda X}.\]

\begin{prop}
Soient \(A\in\M{n}\) et \(\lambda\in\K\). Alors \[\lambda\in\Sp[\K]{A}\ssi\rg\paren{A-\lambda I_n}<n.\]

Dans ce cas, \(\dim\sep[\K]{A}{\lambda}=n-\rg\paren{A-\lambda I_n}\).
\end{prop}

Attention ! Dans la relation \(\dim\sep[\K]{A}{\lambda}=n-\rg\paren{A-\lambda I_n}\), c'est \(n\), pas \(n^2\) ! Il s'agit de la dimension de \(\M{n\,1}\), pas celle de \(\M{n}\).

\begin{rem}
Un cas particulier important : \(0\) est valeur propre ssi \(A\) n'est pas inversible, \cad ssi \(\rg A<n\).
\end{rem}

\begin{theo}
Soient \(A\in\M{n}\) et \(\lambda_1,\dots,\lambda_p\) des valeurs propres distinctes de \(A\).

Alors les sous-espaces propres \(\paren{\sep[\K]{A}{\lambda_i}}_{1\leq i\leq p}\) sont en somme directe.

Autrement dit, toute famille de vecteurs propres associés à des valeurs propres distinctes est libre.
\end{theo}

\begin{rem}
Quand on demande de déterminer les éléments propres d'une matrice, on demande de déterminer les valeurs propres et les vecteurs propres associés, \ie les sous-espaces propres.
\end{rem}

\section{Polynôme caractéristique d'une matrice carrée}

\subsection{Définition et lien avec les valeurs propres}

\begin{defi}
Soit \(A\in\M{n}\).

On appelle polynôme caractéristique de \(A\) le polynôme \(\chi_A=\det\paren{XI_n-A}\).
\end{defi}

\begin{prop}
Soient \(A\in\M{n}\), \(f\in\Lendo{E}\) et \(\fami{B}\) une base de \(E\).

Si \(A=\Mat{f}\), alors \(\chi_A=\chi_f\).
\end{prop}

Par conséquent, deux matrices semblables ont le même polynôme caractéristique.

\begin{theo}
Soit \(A\in\M{n}\).

Alors \(\chi_A\) est un polynôme unitaire de degré \(n\) de \(\poly\) et les valeurs propres de \(A\) sont exactement les racines de \(\chi_A\) dans \(\K\).

Par conséquent, une matrice carrée de taille \(\paren{n,n}\) a au plus \(n\) valeurs propres distinctes.
\end{theo}

\begin{cor}
L'ensemble \(\GL{n}\) est dense dans \(\M{n}\).
\end{cor}

On peut noter un lien avec la trace et le déterminant.

\begin{prop}
Soit \(A\in\M{n}\).

Alors \(\chi_A=X^n-\tr\paren{A}X^{n-1}+\dots+\paren{-1}^n\det A\).
\end{prop}

\subsection{Ordre de multiplicité et dimension du sous-espace propre}

\begin{defi}
Soient \(A\in\M{n}\) et \(\lambda\in\Sp[\K]{A}\).

On appelle ordre de multiplicité de la valeur propre \(\lambda\) son ordre de multiplicité en tant que racine de \(\chi_A\).
\end{defi}

\begin{theo}
Soient \(A\in\M{n}\) et \(\lambda\in\Sp[\K]{A}\).

Si \(\lambda\) est une valeur propre d'ordre \(\alpha\), alors \(1\leq\dim\sep[\K]{A}{\lambda}\leq\alpha\).
\end{theo}

\begin{prop}
Soient \(A\in\M{n}\), \(f\in\Lendo{E}\) et \(\fami{B}\) une base de \(E\).

Si \(A=\Mat{f}\), alors \(\dim\sep[\K]{A}{\lambda}=\dim\sep{f}{\lambda}\).
\end{prop}

Par conséquent, deux matrices semblables ont des sous-espaces propres de même dimension (mais pas les mêmes vecteurs propres).

\subsection{Matrice scindée}

\begin{defi}
On dit qu'une matrice de \(\M{n}\) est scindée quand son polynôme caractéristique est scindé dans \(\poly\).
\end{defi}

Dans le cas d'une matrice scindée, on connaît alors la somme et le produit des valeurs propres.

\begin{prop}
Si \(A\in\M{n}\) est scindée et a pour valeurs propres \(\lambda_1,\dots,\lambda_p\) avec les ordres de multiplicité \(\alpha_1,\dots,\alpha_p\), alors \[\tr A=\sum_{k=1}^p\alpha_k\lambda_k\qquad\text{et}\qquad\det A=\prod_{k=1}^p\lambda_k^{\alpha_k}.\]
\end{prop}

Si \(\K=\C\), alors on est dans ce cas, car tous les polynômes de \(\poly[\C]\) sont scindés dans \(\poly[\C]\) d'après le théorème de d'Alembert-Gauss.

Mais si \(\K=\R\), alors il faut se méfier des raisonnements hâtifs : comme un polynôme à coefficients réels peut ne pas avoir de racines réelles, la trace et le déterminant peuvent ne pas avoir de rapport avec les valeurs propres.

\section{Endomorphismes diagonalisables, matrices diagonalisables}

\subsection{Définition}

\begin{defi}
Soient \(f\in\Lendo{E}\) et \(A\in\M{n}\).

On dit que \(f\) est diagonalisable quand il existe une base de \(E\) constituée de vecteurs propres de \(f\).

On dit que \(A\) est diagonalisable dans \(\M{n}\) (ou \(\K\)-diagonalisable) quand il existe une base de \(\M{n\,1}\) constituée de vecteurs propres de \(A\).
\end{defi}

D'après le lien entre les endomorphismes et les matrices carrées, un endomorphisme est diagonalisable ssi sa matrice dans n'importe quelle base est diagonalisable.

\begin{exo}~\\
La matrice \(\begin{pmatrix}
1 & \sqrt{3} \\
-\sqrt{3} & 1
\end{pmatrix}\) est-elle \(\R\)-diagonalisable ? \(\C\)-diagonalisable ?
\end{exo}

\begin{exo}
Montrez que la matrice \(A=\begin{pmatrix}
5 & -8 & -4 \\
8 & -15 & -8 \\
-10 & 20 & 11
\end{pmatrix}\) est diagonalisable.
\end{exo}

\begin{exo}
Même exercice avec \(B=\begin{pmatrix}
0 & 1 & -1 \\
2 & 1 & 1 \\
4 & -2 & 4
\end{pmatrix}\).
\end{exo}

\begin{exo}
La matrice \(C=\begin{pmatrix}
11 & 7 & -3 \\
11 & 7 & -3 \\
66 & 42 & -18
\end{pmatrix}\) est-elle diagonalisable ?
\end{exo}

\begin{prop}
Si un endomorphisme (une matrice) est diagonalisable, alors il (elle) est scindé(e).
\end{prop}

Mais la réciproque est fausse.

\subsection{Caractérisations équivalentes}

On note \(\diago{n}\) l'ensemble des matrices diagonales de \(\M{n}\).

\begin{prop}
Soient \(f\in\Lendo{E}\) et \(A\in\M{n}\).

\(f\) est diagonalisable ssi il existe une base \(\fami{B}\) de \(E\) telle que \(\Mat{f}\in\diago{n}\). Dans ce cas, les valeurs propres de \(f\) sont les éléments diagonaux de cette matrice.

\(A\) est \(\K\)-diagonalisable ssi elle est \(\K\)-semblable à une matrice diagonale, \ie il existe \(P\in\GL{n}\) et \(D\in\diago{n}\) tel que \(A=PDP\inv\). Dans ce cas, les valeurs propres de \(A\) sont les éléments diagonaux de \(D\).
\end{prop}

\begin{ex}
\begin{itemize}
    \item Toute matrice diagonale est diagonalisable, car elle est semblable à elle-même. \\
    \item Les projecteurs et les symétries sont diagonalisables.
\end{itemize}
\end{ex}

\begin{rem}
Quitte à changer l'ordre des vecteurs dans la base, on peut ranger les valeurs propres sur la diagonale dans l'ordre qu'on veut.
\end{rem}

\begin{ex}~\\
Si \(D=\begin{pmatrix}
1 & 0 & 0 \\
0 & 3 & 0 \\
0 & 0 & 3
\end{pmatrix}\), \(P=\begin{pmatrix}
1 & 1 & 1 \\
0 & 1 & 1 \\
2 & -1 & 3
\end{pmatrix}\) et \(D=P\inv AP\), alors la colonne 1 de \(P\) est un vecteur propre de \(A\) pour la valeur propre \(1\) et les deux autres sont des vecteurs propres pour la valeur propre \(3\), donc en posant \(Q=\begin{pmatrix}
1 & 1 & 1 \\
1 & 1 & 0 \\
3 & -1 & 2
\end{pmatrix}\), on a \(Q\inv AQ=\begin{pmatrix}
3 & 0 & 0 \\
0 & 3 & 0 \\
0 & 0 & 1
\end{pmatrix}\).
\end{ex}

\begin{lem}
Soit \(f\in\Lendo{E}\) diagonalisable : il existe une base de \(E\) dans laquelle la matrice \(D\) de \(f\) est diagonale.

Les valeurs propres de \(f\) sont les éléments diagonaux de \(D\) et si \(\lambda\) est un tel nombre, alors la dimension de \(\sep{f}{\lambda}\) est le nombre d'occurrences de \(\lambda\) dans la diagonale de \(D\).
\end{lem}

On en déduit les théorèmes suivants.

\begin{theo}
Soit \(f\in\Lendo{E}\).

Il y a équivalence entre les propositions suivantes :

\begin{itemize}
    \item \(f\) est diagonalisable \\
    \item les sous-espaces propres de \(f\) sont supplémentaires dans \(E\) \\
    \item \(\sum_{\lambda\in\Sp{f}}\dim\sep{f}{\lambda}=n\)
\end{itemize}
\end{theo}

Et sa version matricielle.

\begin{theo}
Soit \(A\in\M{n}\).

Il y a équivalence entre les propositions suivantes :

\begin{itemize}
    \item \(A\) est diagonalisable dans \(\M{n}\) \\
    \item les sous-espaces propres de \(A\) dans \(\M{n\,1}\) sont supplémentaires dans \(\M{n\,1}\) \\
    \item \(\sum_{\lambda\in\Sp[\K]{A}}\dim\sep[\K]{A}{\lambda}=n\)
\end{itemize}
\end{theo}

\begin{exo}~\\
On pose \(A=\begin{pmatrix}
0 & 1 & -1 \\
2 & 1 & 1 \\
4 & -2 & 4
\end{pmatrix}\). Calculez les valeurs propres de \(A\). Sans calculer les vecteurs propres associés, montrez que \(A\) est diagonalisable.
\end{exo}

\subsection{Lien avec le polynôme caractéristique}

\begin{theo}
Soit \(f\in\Lendo{E}\).

Il y a équivalence entre les propositions suivantes :

\begin{itemize}
    \item \(f\) est diagonalisable \\
    \item \(f\) est scindé et pour tout \(\lambda\in\Sp{f}\), la dimension de \(\sep{f}{\lambda}\) est égale à l'ordre de multiplicité de \(\lambda\)
\end{itemize}
\end{theo}

Et sa version matricielle.

\begin{theo}
Soit \(A\in\M{n}\).

Il y a équivalence entre les propositions suivantes :

\begin{itemize}
    \item \(A\) est diagonalisable dans \(\M{n}\) \\
    \item \(A\) est scindée et pour tout \(\lambda\in\Sp[\K]{A}\), la dimension de \(\sep[\K]{A}{\lambda}\) est égale à l'ordre de multiplicité de \(\lambda\)
\end{itemize}
\end{theo}

Dans le cas où \(\K=\C\), la condition \guillemets{être scindé} est automatiquement satisfaite.

Un cas particulier très courant.

\begin{prop}
Si un endomorphisme de \(E\) possède exactement \(n\) valeurs propres distinctes, alors il est diagonalisable.

Si une matrice de \(\M{n}\) possède exactement \(n\) valeurs propres distinctes dans \(\K\), alors elle est diagonalisable dans \(\M{n}\).
\end{prop}

\begin{exo}
Montrez que la matrice \(\begin{pmatrix}
-4 & 8 & 22 \\
-2 & 3 & 4 \\
-1 & 2 & 7
\end{pmatrix}\) est diagonalisable.
\end{exo}

\section{Lien entre diagonalisabilité et polynômes annulateurs}

\subsection{Racines du polynôme minimal}

\begin{prop}
Soit \(f\in\Lendo{E}\). Les racines de \(\mu_f\) sont exactement les valeurs propres de \(f\) : \(\rac{\mu_f}=\Sp{f}\).

Soit \(A\in\M{n}\). Les racines dans \(\K\) de \(\mu_A\) sont exactement les valeurs propres dans \(\K\) de \(A\) : \(\rac{\mu_A}=\Sp[\K]{A}\).
\end{prop}

\subsection{Lemme des noyaux}

\begin{prop}
Soient \(f\in\Lendo{E}\) et \(P,Q\in\poly\) tels que \(P\et Q=1\).

Alors \(\ker\paren{PQ}\paren{f}=\ker P\paren{f}\oplus\ker Q\paren{f}\).
\end{prop}

\begin{prop}
Soient \(f\in\Lendo{E}\) et \(P_1,\dots,P_k\in\poly\) premiers entre eux deux à deux. On pose \(P=\prod_{i=1}^kP_i\).

Alors \(\ker P\paren{f}=\bigoplus_{i=1}^k\ker P_i\paren{f}\).
\end{prop}

\subsection{Application à la diagonalisabilité}

\begin{defi}
Un polynôme est dit simplement scindé quand il est scindé et à racines simples.
\end{defi}

\begin{theo}
Soit \(f\in\Lendo{E}\).

Il y a équivalence entre les propositions suivantes :

\begin{itemize}
    \item \(f\) est diagonalisable \\
    \item \(\mu_f\) est simplement scindé \\
    \item il existe un polynôme annulateur de \(f\) simplement scindé \\
    \item le polynôme \(\prod_{\lambda\in\Sp{f}}\paren{X-\lambda}\) est un polynôme annulateur de \(f\)
\end{itemize}
\end{theo}

Et sa version matricielle.

\begin{theo}
Soit \(A\in\M{n}\).

Il y a équivalence entre les propositions suivantes :

\begin{itemize}
    \item \(A\) est diagonalisable dans \(\M{n}\) \\
    \item \(\mu_A\) est simplement scindé \\
    \item il existe un polynôme annulateur de \(A\) simplement scindé dans \(\poly\) \\
    \item le polynôme \(\prod_{\lambda\in\Sp[\K]{A}}\paren{X-\lambda}\) est un polynôme annulateur de \(A\)
\end{itemize}
\end{theo}

\begin{exo}~\\
On pose \(A=\begin{pmatrix}
2 & -1 & 2 \\
5 & -3 & 3 \\
-1 & 0 & -2
\end{pmatrix}\). Calculez \(\paren{A+I_3}^3\). \(A\) est-elle diagonalisable ?
\end{exo}

\begin{exo}
Soit \(A\in\M{n}\) telle que \(A^3=I_n\). Selon que \(\K\) soit égal à \(\C\) ou \(\R\), à quelle condition \(A\) est-elle \(\K\)-diagonalisable ?
\end{exo}

\subsection{Diagonalisabilité d'un endomorphisme induit}

\begin{prop}
Soient \(f\in\Lendo{E}\), \(F\) un sous-espace vectoriel de \(E\) stable par \(f\) et \(g\) l'endomorphisme induit par \(f\) dans \(F\).

Alors \(\mu_g\) divise \(\mu_f\).
\end{prop}

\begin{cor}
Soient \(f\in\Lendo{E}\) et \(F\) un sous-espace vectoriel de \(E\) stable par \(f\).

Si \(f\) est diagonalisable, alors l'endomorphisme induit par \(f\) dans \(F\) est aussi diagonalisable.
\end{cor}

\begin{exo}
Soit \(f\) un endomorphisme de matrice \(\begin{pmatrix}
1 & 1 & -1 \\
1 & 1 & 1 \\
1 & 1 & 1
\end{pmatrix}\) dans une base \(\fami{B}=\paren{e_1,e_2,e_3}\). Déterminez les sous-espaces vectoriels de \(E\) stables par \(f\).
\end{exo}

\begin{exo}[Codiagonalisation ou diagonalisation simultanée]
\note{Énoncé étrange, en attente de rectification}

Soient \(A,B\in\M{n}\) diagonalisables et qui commutent.

Alors il existe \(P\in\GL{n}\) telle que \(P\inv AP\) et \(P\inv BP\) sont diagonales.
\end{exo}

\section{Quelques applications de la diagonalisation}

\subsection{Puissances d'une matrice, suites récurrentes linéairement}

Un petit lemme déjà rencontré.

\begin{lem}
Soient \(A,B\in\M{n}\) et \(P\in\GL{n}\) telles que \(A=PBP\inv\).

Alors \(\quantifs{\tpt k\in\N}A^k=PB^kP\inv\).
\end{lem}

Le lemme précédent est particulièrement utile si \(A\) est diagonalisable et si on choisit \(B=D\), matrice diagonale semblable à \(A\), car calculer les puissances d'une matrice diagonale est très facile.

Grâce à la diagonalisation de \(A\), on peut espérer exprimer la forme générale des suites récurrentes linéaires (voir le chapitre précédent, section sur les polynômes annulateurs).

\begin{exo}
Soient \(u,v,w\) les trois suites réelles telles que \(u_0=v_0=w_0=1\) et \[\quantifs{\tpt n\in\N}\begin{dcases}
u_{n+1}=u_n-v_n \\
v_{n+1}=-4u_n+4v_n-6w_n \\
w_{n+1}=-3u_n+3v_n-4w_n
\end{dcases}\]

Déterminez des expressions de \(u_n,v_n,w_n\) en fonction de \(n\).
\end{exo}

Cette technique s'applique en particulier aux suites \(u\) vérifiant une relation de récurrence linéaire de la forme : \(\quantifs{\tpt n\in\N}u_{n+d}=a_{d-1}u_{n+d-1}+\dots+a_2u_{n+2}+a_1u_{n+1}+a_0u_n\).

On pose alors \(X_n=\begin{pmatrix}
u_n \\
u_{n+1} \\
\vdots \\
u_{n+d-1}
\end{pmatrix}\) et \(A=\begin{pmatrix}
0 & 1 & 0 & \dots & 0 \\
0 & 0 & 1 & \ddots & \vdots \\
\vdots & \vdots &  & \ddots & 0 \\
0 & 0 & 0 & \dots & 1 \\
a_0 & a_1 & a_2 & \dots & a_{d-1}
\end{pmatrix}\in\M{d}\).

Alors \(\quantifs{\tpt n\in\N}X_{n+1}=AX_n\) et on est ramené au cas précédent.

La matrice \(A\) s'appelle la matrice-compagnon du polynôme \(P=X^d-a_{d-1}X^{d-1}-\dots-a_1X-a_0\) : elle a la propriété remarquable que son polynôme caractéristique est \(P\), son polynôme minimal est aussi \(P\) et donc que ses valeurs propres sont les racines de \(P\). C'est pourquoi le polynôme \(P\) est appelé polynôme caractéristique associé à la suite \(u\) (cas déjà étudié en première année : \(d=2\)).

On en déduit que \(A\) est diagonalisable ssi \(P\) est simplement scindé et dans ce cas, \(A\) possède \(d\) valeurs propres distinctes. Dans ce cas, en notant \(\lambda_1,\dots,\lambda_p\) les valeurs propres distinctes, la suite \(u\) est combinaison linéaire des suites géométriques \(\paren{\lambda_1^n},\dots,\paren{\lambda_d^n}\).

\begin{exo}
Explicitez l'unique suite \(\paren{u_n}\) vérifiant \[u_0=0,u_1=1,u_2=5\qquad\text{et}\qquad\quantifs{\forall n\in\N}u_{n+3}=6u_{n+2}-11u_{n+1}+6u_n.\]
\end{exo}

\subsection{Systèmes d'équations différentielles}

Ce point sera traité dans le chapitre sur les équations différentielles linéaires.

\section{Endomorphismes trigonalisables, matrices trigonalisables}

\subsection{Définition et propriétés}

\begin{defi}
Un endomorphisme est dit trigonalisable quand il existe une base dans laquelle sa matrice est triangulaire supérieure.

Une matrice carrée de \(\M{n}\) est dite trigonalisable dans \(\M{n}\) quand elle est semblable à une matrice triangulaire dans \(\M{n}\).
\end{defi}

\begin{rem}
\begin{itemize}
    \item Si un endomorphisme (une matrice) est diagonalisable, alors il (elle) est trigonalisable. \\
    \item Si une matrice est trigonalisable, ses valeurs propres sont les nombres sur la diagonale de toute matrice triangulaire semblable.
\end{itemize}
\end{rem}

\begin{exo}
On considère la matrice \(M=\begin{pmatrix}
-2 & -1 & 7 \\
5 & 4 & -8 \\
1 & 1 & 1
\end{pmatrix}\) et \(f\) un endomorphisme de matrice \(M\). Déterminez les éléments propres de \(M\). Est-elle diagonalisable ? En complétant une famille libre de vecteurs propres, déterminez une base \(\fami{B}\) de l'espace où la matrice de \(f\) est triangulaire supérieure, puis trigonalisez \(M\).
\end{exo}

\begin{exo}
Soit \(f\) un endomorphisme de matrice \(A=\begin{pmatrix}
2 & -4 & -5 \\
-1 & 2 & 2 \\
1 & -2 & -2
\end{pmatrix}\). Montrez que \(f\) n'est pas diagonalisable mais est trigonalisable et donnez une base de trigonalisation de \(f\). Donnez une forme générale pour \(A^n\).
\end{exo}

Quand un endomorphisme ou une matrice n'est pas diagonalisable, on peut espérer qu'il ou elle est trigonalisable : faute de grives, on se contente de merles !

\begin{rem}
On ne confondra pas la trigonalisation d'une matrice carrée et la transformation par lignes (ou colonnes) des matrices vue en première année ! Seule la trigonalisation fournit des matrices semblables ! La transformation par lignes ne conserve que le rang !
\end{rem}

\subsection{Caractérisation équivalente}

La trigonalisabilité est beaucoup plus courante que la diagonalisabilité, comme on le voit grâce aux résultats suivants.

\begin{prop}
Un endomorphisme (une matrice) est trigonalisable ssi il (elle) est scindé(e).
\end{prop}

En particulier, quand \(\K=\C\), tous les endomorphismes sont trigonalisables, toutes les matrices de \(\M{n}[\C]\) sont trigonalisables dans \(\M{n}[\C]\).

En pratique, quand on cherche à trigonaliser un endomorphisme, on peut chercher une base dans laquelle la matrice est triangulaire supérieure avec des \(1\) ou des \(0\) sur la sur-diagonale et des \(0\) sur les diagonales partielles encore au-dessus (c'est démontrable, mais c'est difficile à démontrer, cela s'appelle le théorème de Jordan -- hors-programme --).

\begin{theo}
Soit \(f\in\Lendo{E}\).

Il y a équivalence entre les propositions suivantes :

\begin{itemize}
    \item \(f\)  est trigonalisable \\
    \item \(\chi_f\) est scindé \\
    \item \(\mu_f\) est scindé \\
    \item il existe un polynôme annulateur de \(f\) scindé
\end{itemize}
\end{theo}

Et sa version matricielle.

\begin{theo}
Soit \(A\in\M{n}\).

Il y a équivalence entre les propositions suivantes :

\begin{itemize}
    \item \(A\) est trigonalisable dans \(\M{n}\) \\
    \item \(\chi_A\) est scindé \\
    \item \(\mu_A\) est scindé \\
    \item il existe un polynôme annulateur de \(A\) qui est scindé dans \(\poly\)
\end{itemize}
\end{theo}

\begin{exo}~\\
Soit \(A=\begin{pmatrix}
0 & 1 & 1 \\
0 & 0 & 0 \\
0 & 1 & 0
\end{pmatrix}\). Calculez \(A^2\), puis \(A^3\). La matrice \(A\) est-elle diagonalisable ? trigonalisable ? Dans l'affirmative, diagonalisez ou trigonalisez la.
\end{exo}

\subsection{Théorème de Cayley-Hamilton}

\begin{theo}
Le polynôme caractéristique d'un endomorphisme (d'une matrice carrée) est un polynôme annulateur.
\end{theo}

\begin{cor}
Le polynôme minimal divise le polynôme caractéristique. Donc en dimension \(n\), le polynôme minimal est de degré au plus \(n\).
\end{cor}

Les polynômes minimal et caractéristique partagent les mêmes racines dans \(\C\) (en fait dans tout corps \(\K\)) mais pas avec les mêmes ordres de multiplicité : si \(f\) est scindé, alors en notant \(\lambda_1,\dots,\lambda_k\) les \(k\) valeurs propres distinctes de \(f\), on peut écrire \[\chi_f=\prod_{i=1}^k\paren{X-\lambda_i}^{\alpha_i}\qquad\text{et}\qquad\mu_f=\prod_{i=1}^k\paren{X-\lambda_i}^{\beta_i}\] où \(\quantifs{\tpt i\in\interventierii{1}{k}}1\leq\beta_i\leq\alpha_i\).

\subsection{Sous-espaces caractéristiques}

\begin{defi}
Soit \(f\in\Lendo{E}\) un endomorphisme scindé. On écrit \(\chi_f=\prod_{i=1}^k\paren{X-\lambda_i}^{\alpha_i}\) où \(\lambda_1,\dots,\lambda_k\) sont les \(k\) valeurs propres distinctes de \(f\).

Les sous-espaces caractéristiques de \(f\) sont les noyaux \(\ker\paren{f-\lambda_i\id{E}}^{\alpha_i}\).
\end{defi}

\begin{prop}
Les sous-espaces caractéristiques d'un endomorphisme scindé sont supplémentaires et stables par \(f\).
\end{prop}

\begin{theo}
Tout endomorphisme scindé possède une base dans laquelle sa matrice est diagonale par blocs telle que :

\begin{itemize}
    \item il y a autant de blocs que de valeurs propres : à chaque valeur propre, on associe un unique bloc ; \\
    \item chaque bloc est de la forme \(\lambda I_r+U\) où \(\lambda\) est la valeur propre associée au bloc, \(r\) est l'ordre de multiplicité de \(\lambda\) et \(U\) est une matrice strictement triangulaire supérieure de \(\M{r}\)
\end{itemize}

Toute matrice scindée est semblable à une matrice diagonale par blocs vérifiant les conditions précédentes.
\end{theo}

\begin{cor}
La dimension d'un sous-espace caractéristique est l'ordre de multiplicité de la valeur propre associée.
\end{cor}

\section{Endomorphismes nilpotents, matrices nilpotentes}

\subsection{Généralités}

\begin{defi}
Soit \(u\in\Lendo{E}\). On dit que \(u\) est nilpotent quand il existe \(p\in\N\) tel que \(u^p=0\).

Soit \(A\in\M{n}\). On dit que \(A\) est nilpotente quand il existe \(p\in\N\) tel que \(A^p=0\).

Le plus petit indice \(p\) satisfaisant à la condition précédente s'appelle l'indice de nilpotence de \(u\) (de \(A\)).
\end{defi}

\begin{prop}
Toute matrice strictement triangulaire (supérieure ou inférieure) est nilpotente. Par conséquent, les matrices semblables à une matrice strictement triangulaire sont nilpotentes.
\end{prop}

Dans la décomposition en sous-espaces caractéristiques, on a vu apparaître des matrices \(\lambda I_r+U\) : les matrices \(U\) sont nilpotentes.

L'ensemble des matrices nilpotentes n'a pas de structure particulière : en général, la somme et le produit de deux matrices nilpotentes ne sont pas nilpotents. Néanmoins, avec une condition de commutation supplémentaire, on a quelques résultats.

\begin{prop}
Soient \(A,B\in\M{n}\) deux matrices nilpotentes.

Si \(A\) et \(B\) commutent, alors \(A+B\) et \(AB\) sont nilpotentes.
\end{prop}

On a bien sûr les mêmes résultats concernant les endomorphismes nilpotents.

\subsection{Éléments propres d'un nilpotent}

\begin{prop}
Un endomorphisme en dimension \(n\) est nilpotent ssi son polynôme caractéristique est \(X^n\), \ie s'il est scindé et admet \(0\) comme unique valeur propre.

Une matrice de \(\M{n}\) est nilpotente ssi son polynôme caractéristique est \(X^n\), \ie si elle est scindée et admet \(0\) comme unique valeur propre.

L'indice de nilpotence dans ces deux cas est alors le degré du polynôme minimal ; il est donc inférieur ou égal à \(n\).
\end{prop}

Mis à part la matrice nulle, aucune matrice nilpotente n'est diagonalisable : c'est une idée parfois utile pour prouver qu'une matrice est nulle (diagonalisable et nilpotente implique nulle).

\begin{prop}
Tout endomorphisme nilpotent est trigonalisable : il existe une base dans laquelle sa matrice est triangulaire supérieure stricte. Réciproquement, si un endomorphisme est trigonalisable et n'a que \(0\) pour valeur propre, alors il est nilpotent.

Toute matrice nilpotente est trigonalisable : elle est semblable à une matrice triangulaire supérieure stricte. La réciproque est vraie.
\end{prop}

\subsection{Application aux sous-espaces caractéristiques d'un endomorphisme}

\begin{prop}
Soit \(f\in\Lendo{E}\).

Pour toute valeur propre \(\lambda\) de \(f\), si \(\alpha\) est l'ordre de multiplicité de \(\lambda\) dans le polynôme minimal de \(f\), le sous-espace caractéristique associé est aussi le noyau \(\ker\paren{f-\lambda\id{E}}^\alpha\).
\end{prop}

On peut même démontrer mieux.

\begin{prop}
Soient \(f\in\Lendo{E}\), \(\lambda\in\Sp{f}\) et \(\alpha\) l'ordre de multiplicité de \(\lambda\) dans le polynôme minimal de \(f\).

Alors la suite des noyaux \(\paren{\ker\paren{f-\lambda\id{E}}^k}_{k\in\N}\) est strictement croissante jusqu'au rang \(\alpha\), puis constante à partir du rang \(\alpha\) : \[\accol{0}\subsetneq\ker\paren{f-\lambda\id{E}}\subsetneq\ker\paren{f-\lambda\id{E}}^2\subsetneq\dots\subsetneq\ker\paren{f-\lambda\id{E}}^\alpha=\ker\paren{f-\lambda\id{E}}^{\alpha+1}=\dots\]
\end{prop}


\chapter{Intégrales généralisées}

\minitoc

Dans ce chapitre, \(\K\) désigne l'ensemble \(\R\) ou \(\C\). Les fonctions considérées dans ce chapitre sont à valeurs dans \(\K\).

On suppose connue la notion d'intégrale sur un segment d'une fonction continue par morceaux à valeurs réelles ou complexes (\cf cours de première année).

Si \(f\) est une fonction continue sur un segment \(\intervii{a}{b}\) (ou \(\intervii{b}{a}\)), on note \(\int_a^bf=\int_a^bf\paren{t}\odif{t}\) (ou toute autre lettre à la place de \(t\)) l'intégrale de \(f\) entre \(a\) et \(b\) : quand on a pas besoin de nommer la variable d'intégration, on ne la note pas, mais si on la note alors on n'oublie pas l'élément différentiel \(\odif{}\).

En préambule, on généralise la notion de fonction continue par morceaux sur un intervalle quelconque.

\section*{Fonctions continues par morceaux sur un intervalle}
\addcontentsline{toc}{section}{Fonctions continues par morceaux sur un intervalle}

\begin{defi}
Soit \(I\) un intervalle quelconque.

On dit qu'une fonction est continue par morceaux sur \(I\) quand elle est continue par morceaux sur tout segment inclus dans \(I\).
\end{defi}

\begin{ex}
\begin{itemize}
    \item La fonction \(t\mapsto\begin{dcases}
        -\ln t &\text{si }t\in\intervei{0}{1} \\
        \e{-t} &\text{si }t\in\intervie{1}{\pinf}
    \end{dcases}\) est continue par morceaux sur \(\intervei{0}{\pinf}\). \\
    \item La fonction \(t\mapsto\floor{\dfrac{1}{t}}\) est continue par morceaux sur \(\intervei{0}{\pinf}\).
\end{itemize}
\end{ex}

Dans toute la suite, on note \(\contm[I]\) l'ensemble des fonctions continues par morceaux sur \(I\) et à valeurs dans \(\K\).

\begin{prop}
L'ensemble \(\contm[I]\) est une \(\K\)-algèbre.
\end{prop}

\section{Intégrales généralisées sur \(\intervie{a}{\pinf}\)}

Dans cette section, \(a\) est un réel.

\subsection{Définition et exemples fondamentaux}

\begin{defi}
Soit \(f\in\contm[\intervie{a}{\pinf}]\).

On dit que l'intégrale \(\int_a^{\pinf}f\) converge (ou qu'elle est convergente, ou simplement qu'elle existe) quand \(\int_a^xf\) a une limite finie quand \(x\) tend vers \(\pinf\).

Dans ce cas, on pose \(\int_a^{\pinf}=\lim_{x\to\pinf}\int_a^xf\).

Dans le cas contraire, on dit que l'intégrale \(\int_a^{\pinf}\) diverge (ou qu'elle est divergente).
\end{defi}

\begin{rem}
Une intégrale généralisée \(\int_a^{\pinf}\) est une limite et une limite n'existe pas toujours. Avant d'utiliser une telle intégrale dans un raisonnement ou un calcul, on doit donc toujours justifier son existence !
\end{rem}

Les résultats ci-dessous sont à connaître.

\begin{ex}
\begin{itemize}
    \item Soit \(\alpha\) un réel. L'intégrale \(\int_0^{\pinf}\e{-\alpha t}\odif{t}\) converge ssi \(\alpha>0\). \\
    \item Soit \(\alpha\) un réel. L'intégrale \(\int_1^{\pinf}\dfrac{1}{t^\alpha}\odif{t}\) converge ssi \(\alpha>1\) (intégrale dite de Riemann).
\end{itemize}
\end{ex}

\begin{exo}
Quelle est la nature de l'intégrale \(\int_0^{\pinf}\dfrac{1}{1+t^2}\odif{t}\) ?
\end{exo}

\begin{exo}
Quelle est la nature de l'intégrale \(\int_1^{\pinf}\dfrac{\ln t}{t}\odif{t}\).
\end{exo}

\subsection{Propriétés}

La convergence de l'intégrale ne dépend pas de la borne \(a\), ce qui généralise la relation de Chasles.

\begin{prop}
Soient \(f\in\contm[\intervie{a}{\pinf}]\) et \(b\in\intervie{a}{\pinf}\).

Alors l'intégrale \(\int_a^{\pinf}f\) converge ssi \(\int_b^{\pinf}f\) converge.

Dans ce cas, on a \(\int_a^{\pinf}f=\int_a^bf+\int_b^{\pinf}f\).
\end{prop}

Dans le cas convergent, on retrouve la linéarité.

\begin{prop}
Soient \(\paren{f,g}\in\contm[\intervie{a}{\pinf}]^2\) et \(\paren{\lambda,\mu}\in\K^2\).

Si les intégrales \(\int_a^{\pinf}f\) et \(\int_a^{\pinf}g\) convergent, alors \(\int_a^{\pinf}\paren{\lambda f+\mu g}\) converge.

Dans ce cas, on a \(\int_a^{\pinf}\paren{\lambda f+\mu g}=\lambda\int_a^{\pinf}f+\mu\int_a^{\pinf}g\).
\end{prop}

\begin{rem}
\begin{itemize}
    \item \guillemets{La somme d'une intégrale convergente et d'une divergente est divergente}. \\
    \item Il n'y a rien à dire \textit{a priori} sur la \guillemets{somme de deux intégrales divergentes}.
\end{itemize}
\end{rem}

\subsection{Cas des fonctions réelles positives}

Quand une fonction \(f\) est positive et continue par morceaux, dans le cas où l'intégrale \(\int_a^{\pinf}f\) diverge, on pose par convention \(\int_a^{\pinf}f=\pinf\), ce qui permet de donner un sens à toutes les intégrales de fonctions positives.

\begin{prop}
Soit \(f\in\contm[\intervie{a}{\pinf}][\R]\).

Si \(f\geq0\), alors \(\int_a^{\pinf}f\geq0\).

Si, de plus, \(f\) est continue et prend au moins une valeur strictement positive, alors \(\int_a^{\pinf}f>0\).

Ceci est vrai en particulier quand \(f\) est continue et strictement positive sur \(\intervie{a}{\pinf}\).
\end{prop}

On en déduit la propriété de croissance des intégrales.

\begin{prop}
Soit \(\paren{f,g}\in\contm[\intervie{a}{\pinf}][\R]^2\) tel que \(f\) et \(g\) soient positives.

Si \(f\leq g\), alors \(\int_a^{\pinf}f\leq\int_a^{\pinf}g\).
\end{prop}

\subsection{Théorème de comparaison entre fonctions positives}

D'abord une condition nécessaire et suffisante de convergence dans le cas d'une fonction positive.

\begin{prop}
Soit \(f\in\contm[\intervie{a}{\pinf}][\R]\).

Si \(f\geq0\), alors l'intégrale \(\int_a^{\pinf}f\) converge ssi la fonction \(x\mapsto\int_a^xf\) est majorée.
\end{prop}

On en déduit un théorème de comparaison du même type que celui sur les séries.

\begin{theo}
Soit \(\paren{f,g}\in\contm[\intervie{a}{\pinf}][\R]^2\) tel que \(f\) et \(g\) soient positives.

\begin{itemize}
    \item Si \(f\leq g\) et \(\int_a^{\pinf}g\) converge, alors \(\int_a^{\pinf}f\) converge. \\
    \item Si \(f\leq g\) et \(\int_a^{\pinf}f\) diverge, alors \(\int_a^{\pinf}g\) diverge. \\
    \item Si \(f\simqd{\pinf}g\) alors les intégrales \(\int_a^{\pinf}f\) et \(\int_a^{\pinf}g\) sont de même nature : l'une converge ssi l'autre converge.
\end{itemize}
\end{theo}

\begin{rem}
\begin{itemize}
    \item Dans ce théorème, il suffit que les inégalités soient vraies au voisinage de \(\pinf\) seulement. \\
    \item Si les fonctions sont à valeurs négatives, on se ramène à ce théorème en travaillant avec les fonctions opposées. Ce qui compte est donc qu'elles soient de signe constant. \\
    \item Avec des fonctions dont le signe n'est pas constant, ce théorème est faux. Il faut donc bien s'assurer et mettre en valeur que les fonctions sont positives (ou négatives). \\
    \item On compare les fonctions, pas les intégrales ! N'écrivez pas des symboles \(\int_a^{\pinf}\cdots\) partout.
\end{itemize}
\end{rem}

\begin{exo}~\\
Montrez que \(\int_1^{\pinf}\dfrac{\e{-t}}{t}\odif{t}\) converge.
\end{exo}

\begin{exo}~\\
Montrez que \(\int_2^{\pinf}\dfrac{1}{\ln t}\odif{t}\) diverge.
\end{exo}

\begin{exo}
Montrez que \(\quantifs{\tpt\alpha>1}\int_1^{\pinf}\dfrac{\ln t}{t^\alpha}\odif{t}\) converge.
\end{exo}

\subsection{Lien avec les séries}

Le théorème de comparaison série-intégrale peut se réécrire comme suit.

\begin{prop}
Soit \(f\in\contm[\Rp][\R]\) positive et décroissante.

La série \(\sum f\paren{n}\) et l'intégrale \(\int_0^{\pinf}f\) sont de même nature.
\end{prop}

Dans le même genre, on peut étudier la convergence d'une intégrale d'une fonction positive par l'intermédiaire d'une série.

\begin{prop}
Soient \(f\in\contm[\Rp][\R]\) positive et \(u\) une suite positive et strictement croissante qui diverge vers \(\pinf\).

L'intégrale \(\int_0^{\pinf}f\) converge ssi la série \(\sum_{n\geq0}\int_{u_n}^{u_{n+1}}f\) converge.
\end{prop}

\begin{exo}
Montrez que l'intégrale \(\int_1^{\pinf}\dfrac{\abs{\sin t}}{t}\odif{t}\) diverge.
\end{exo}

\begin{exo}
En utilisant l'inégalité \(\sin t\geq\dfrac{2}{\pi}t\), valable pour tout \(t\in\intervii{0}{\dfrac{\pi}{2}}\), montrez que l'intégrale \(\int_0^{\pinf}\dfrac{1}{1+\e{x}\abs{\sin x}}\odif{x}\) converge.
\end{exo}

\section{Intégrales généralisées sur d'autres types d'intervalles}

\subsection{Intégrales généralisées sur \(\intervie{a}{b}\)}

Dans cette partie, \(a\) est un réel et \(b\) est un réel ou \(\pinf\), de sorte que \(a<b\).

\begin{defi}
Soit \(f\in\contm[\intervie{a}{b}]\).

On dit que l'intégrale \(\int_a^bf\) converge quand \(\int_a^xf\) a une limite finie quand \(x\) tend vers \(b^-\).

Dans ce cas, on pose \(\int_a^bf=\lim_{x\to b^-}\int_a^xf\).

Dans le cas contraire, on dit que l'intégrale \(\int_a^bf\) diverge (ou qu'elle est divergente).
\end{defi}

\subsection{Intégrales généralisées sur \(\intervei{a}{b}\)}

Dans cette partie, \(a\) est un réel ou \(\minf\) et \(b\) un réel, de sorte que \(a<b\).

\begin{defi}
Soit \(f\in\contm[\intervei{a}{b}]\).

On dit que l'intégrale \(\int_a^bf\) converge quand \(\int_x^bf\) a une limite finie quand \(x\) tend vers \(a^+\).

Dans ce cas, on pose \(\int_a^bf=\lim_{x\to a^+}\int_x^bf\).

Dans le cas contraire, on dit que l'intégrale \(\int_a^bf\) diverge (ou qu'elle est divergente).
\end{defi}

Les résultats suivants sont à connaître.

\begin{ex}~\\
\begin{itemize}
    \item \(\int_0^1\ln\paren{t}\odif{t}\) converge. \\
    \item Soit \(\alpha\) un réel. L'intégrale \(\int_0^1\dfrac{1}{t^\alpha}\odif{t}\) converge ssi \(\alpha<1\) (intégrale dite de Riemann).
\end{itemize}
\end{ex}

\begin{exo}
Montrez que l'intégrale \(\int_1^2\dfrac{1}{\ln t}\odif{t}\) diverge.

À quelle condition sur \(\alpha\) l'intégrale \(\int_1^2\dfrac{\paren{t-1}^\alpha}{\ln t}\odif{t}\) converge-t-elle ?
\end{exo}

On peut remarquer que par changement de variable \(x\mapsto-x\), l'intégrale \(\int_a^bf\paren{t}\odif{t}\) est de même nature (et, dans le cas convergent, a la même valeur) que \(\int_{-b}^{-a}f\paren{-u}\odif{u}\). Les résultats valables en un point réel ne dépendent donc pas du côté du point où on se place.

\begin{ex}
\begin{itemize}
    \item Si \(a\) est un réel, alors \(\int_a^{a+1}\dfrac{1}{\paren{t-a}^\alpha}\odif{t}\) converge ssi \(\alpha<1\). \\
    \item Si \(a\) est un réel, alors \(\int_{a-1}^a\dfrac{1}{\paren{a-t}^\alpha}\odif{t}\) converge ssi \(\alpha<1\). \\
    \item \(\int_0^1\ln\paren{1-t}\odif{t}\) converge.
\end{itemize}
\end{ex}

\subsection{Intégrales généralisées sur \(\intervee{a}{b}\)}

Dans cette partie, \(a\) et \(b\) sont des réels ou des infinis tels que \(a<b\).

\begin{defi}
Soit \(f\in\contm[\intervee{a}{b}]\).

On dit que l'intégrale \(\int_a^bf\) converge quand il existe \(c\in\intervee{a}{b}\) tel que \(\int_a^cf\) et \(\int_c^bf\) convergent.

Dans ce cas, on pose \(\int_a^bf=\int_a^cf+\int_c^bf=\lim_{x\to a^+}\int_x^cf+\lim_{y\to b^-}\int_c^yf\).

Dans le cas contraire, on dit que l'intégrale \(\int_a^bf\) diverge (ou qu'elle est divergente).
\end{defi}

\begin{rem}
Grâce à la relation de Chasles, on constate que la valeur de \(c\) n'est finalement pas importante : si ça marche pour un certain réel \(c\in\intervee{a}{b}\), alors ça marche pour toute autre valeur prise dans \(\intervee{a}{b}\).
\end{rem}

\begin{exo}
Montrez que l'intégrale de Gauss \(\int_{\minf}^{\pinf}\e{-t^2}\odif{t}\) converge.
\end{exo}

\begin{exo}
Même chose avec l'intégrale \(\int_0^1\dfrac{1}{\sqrt{t\paren{1-t}}}\odif{t}\).
\end{exo}

\begin{exo}
Même chose avec l'intégrale \(\int_0^{\pinf}\dfrac{\e{-t}}{\sqrt{t}}\odif{t}\).
\end{exo}

\begin{exo}
Montrez que \(\quantifs{\tpt\alpha>0}\int_0^{\pinf}\dfrac{1}{t^\alpha}\odif{t}\) diverge.
\end{exo}

\subsection{Propriétés communes à toutes ces intégrales}

Toutes les propriétés vues dans la première section sont préservées : linéarité, positivité, croissance, relation de Chasles.

En voici trois autres.

\subsubsection{Changement de variable}

\begin{prop}
Soient \(a,b,\alpha,\beta\) des réels ou des infinis tels que \(a<b\) et \(\alpha<\beta\) et \(f\in\contm[\intervee{a}{b}]\).

Si \(\phi\) est une bijection de classe \(\classe{1}\) et strictement croissante de \(\intervee{\alpha}{\beta}\) dans \(\intervee{a}{b}\), alors les intégrales \(\int_a^bf\paren{t}\odif{t}\) et \(\int_\alpha^\beta f\rond\phi\paren{u}\phi\prim\paren{u}\odif{u}\) sont de même nature et, si elles sont convergentes, sont égales.
\end{prop}

\begin{rem}
On a évidemment un résultat analogue avec un changement de variable strictement décroissant et des bornes inversées.

Comme une bijection de classe \(\classe{1}\) entre deux intervalles est forcément strictement monotone, l'hypothèse de stricte monotonie est redondante ; mais comme elle est explicitement dans le programme de MPI, il vaut mieux la préciser (de toute façon, elle sera évidente dans les cas pratiques et ne nécessitera pas de longues preuves).
\end{rem}

\begin{exo}~\\
Montrez que \(\int_0^{\nicefrac{\pi}{2}}\sqrt{\tan t}\odif{t}\) converge et qu'on a l'égalité \(\int_0^{\nicefrac{\pi}{2}}\sqrt{\tan t}\odif{t}=\int_0^{\pinf}\dfrac{\sqrt{u}}{1+u^2}\odif{u}\).
\end{exo}

\begin{exo}~\\
Montrez que \(\int_0^{\pinf}\dfrac{\ln t}{1+t^2}\odif{t}\) converge et vaut \(0\).

Déduisez-en la valeur de \(\int_0^{\pinf}\dfrac{\ln t}{a^2+t^2}\odif{t}\) où \(a>0\).
\end{exo}

\subsubsection{Intégration par parties}

Si \(f\) est une fonction définie sur \(\intervee{a}{b}\) et a des limites finies en \(a^+\) et en \(b^-\), on note \(\croch{f}_a^b=\lim_{b^-}f-\lim_{a^+}f\).

\begin{prop}
Soient \(a,b\) des réels ou des infinis tels que \(a<b\) et \(\paren{f,g}\in\ensclasse{1}{\intervee{a}{b}}{\K}\).

Si parmi les trois quantités suivantes \[\int_a^bf\prim g\qquad\int_a^bfg\prim\qquad\croch{fg}_a^b\] deux existent, alors la troisième existe aussi et, dans ce cas, on a l'égalité habituelle \[\int_a^bf\prim g=\croch{fg}_a^b-\int_a^bfg\prim\].
\end{prop}

En pratique, pour éviter d'écrire des choses qui n'ont pas de sens, il vaut mieux revenir à une vraie intégration par parties sur un segment \(\intervii{x}{y}\subset\intervee{a}{b}\), s'assurer qu'on peut faire tendre \(x\) vers \(a\) et \(y\) vers \(b\), puis le faire effectivement pour obtenir la relation entre les intégrales.

\begin{exo}
Pour \(n\in\N\), on pose \(u_n=\int_0^{\pinf}t^n\e{-t}\odif{t}\).

Montrez que les intégrales \(u_n\) convergent, donnez une relation de récurrence simple entre \(u_n\) et \(u_{n+1}\), puis donnez la valeur de \(u_n\) en fonction de \(n\).
\end{exo}

\begin{exo}
Montrez que l'intégrale \(\int_0^{\pinf}\dfrac{1-\cos t}{t^2}\odif{t}\) converge, puis déduisez-en que l'intégrale \(\int_0^{\pinf}\dfrac{\sin t}{t}\odif{t}\) converge.
\end{exo}

\subsubsection{Primitives}

\begin{prop}
Soient \(f\in\contm[\intervee{a}{b}]\) et \(c\in\intervee{a}{b}\) tel que l'intégrale \(\int_a^cf\) converge.

Alors la fonction \(x\mapsto\int_a^xf\) est l'unique primitive de \(f\) sur \(\intervee{a}{b}\) qui a pour limite \(0\) en \(a^+\).
\end{prop}

\begin{prop}
Soient \(f\in\contm[\intervee{a}{b}]\) et \(c\in\intervee{a}{b}\) tel que \(\int_c^bf\) converge.

Alors la fonction \(x\mapsto\int_x^bf\) est l'opposée de l'unique primitive de \(f\) sur \(\intervee{a}{b}\) qui a pour limite \(0\) en \(b^-\).
\end{prop}

\begin{ex}
\begin{itemize}
    \item La fonction \(x\mapsto\int_0^x\ln\paren{t}\odif{t}\) est la primitive de \(\ln\) qui a pour limite \(0\) en \(0\). \\
    \item La fonction \(x\mapsto\int_x^{\pinf}\dfrac{\e{\i t}}{t^2}\odif{t}\) est définie sur \(\intervee{0}{\pinf}\), de classe \(\classe{1}\) sur cet intervalle, et sa dérivée est la fonction \(x\mapsto-\dfrac{\e{\i x}}{x^2}\).
\end{itemize}
\end{ex}

\section{Résumé pour étudier la convergence d'une intégrale}

On veut savoir si une intégrale \(\int_a^bf\) existe, où \(a\) et \(b\) sont des réels ou des infinis tels que \(a<b\).

D'abord, on détermine le plus grand sous-ensemble de \(\intervii{a}{b}\) sur lequel \(f\) est continue par morceaux :

\begin{itemize}
    \item si c'est \(\intervii{a}{b}\), alors il n'y a aucun problème d'existence de l'intégrale : c'est une bête intégrale classique ; \\
    \item si c'est \(\intervie{a}{b}\) (avec \(a\) réel) ou \(\intervei{a}{b}\) (avec \(b\) réel), alors il faut étudier le comportement de \(f\) au voisinage du point ouvert ; \\
    \item si c'est \(\intervee{a}{b}\), alors on choisit arbitrairement un point \(c\in\intervee{a}{b}\) et on se ramène deux fois au cas précédent.
\end{itemize}

Un petit résultat qui supprime parfois le problème en un point ouvert réel : pensez à étudier la limite de la fonction : si elle est réelle, c'est réglé. On dit qu'on a une fausse singularité en ce point réel.

\begin{prop}
Soient \(a\) et \(b\) deux réels tels que \(a<b\) et \(f\in\contm[\intervei{a}{b}]\).

Si \(f\) a une limite réelle en \(a\) par valeurs supérieures, alors on peut prolonger \(f\) par continuité en \(a\), le prolongement \(\conj{f}\) est une fonction continue par morceaux sur \(\intervii{a}{b}\) et l'intégrale \(\int_a^bf\) converge et vaut \(\int_a^b\conj{f}\).
\end{prop}

\begin{exo}
Justifiez que l'intégrale \(\int_0^{\pinf}\dfrac{\sin^2t}{t^2}\odif{t}\) converge.
\end{exo}

\begin{exo}
Montrez que l'intégrale \(\int_0^{\pinf}\dfrac{t\ln t}{1+t^3}\odif{t}\) converge.
\end{exo}

\section{Fonctions intégrables sur un intervalle}

Dans cette section, \(a\) et \(b\) sont des réels ou des infinis tels que \(a<b\). On note \(I=\intervee{a}{b}\).

\subsection{Intégrales absolument convergentes}

\begin{defi}
Soit \(f\in\contm[I]\).

On dit que l'intégrale \(\int_a^bf\) converge absolument (ou est absolument convergente) quand l'intégrale \(\int_a^b\abs{f}\) converge.
\end{defi}

Le théorème suivant est primordial pour la suite du cours.

\begin{theo}
Soit \(f\in\contm[I]\).

Si l'intégrale \(\int_a^bf\) converge absolument, alors l'intégrale \(\int_a^bf\) converge.

Dans ce cas, on a \(\abs{\int_a^bf}\leq\int_a^b\abs{f}\).
\end{theo}

\begin{exo}
Montrez que l'intégrale \(\int_0^{\pinf}\dfrac{\sin^3t}{t^3}\odif{t}\) est absolument convergente et donc convergente.
\end{exo}

\begin{exo}
Montrez que si \(m\) est un complexe de partie réelle strictement positive, alors l'intégrale \(\int_0^{\pinf}\e{-mt}\odif{t}\) converge et donnez sa valeur.

Déduisez-en l'existence et la valeur des intégrales \(\int_0^{\pinf}\sin\paren{t}\e{-t}\odif{t}\) et \(\int_0^{\pinf}\cos\paren{t}\e{-t}\odif{t}\).
\end{exo}

\begin{rem}
La réciproque est fausse ! On a montré que \(\int_0^{\pinf}\dfrac{\sin t}{t}\odif{t}\) converge et qu'elle ne converge pas absolument.
\end{rem}

\subsection{Fonctions intégrables}

\begin{defi}
Soit \(f\in\contm[I]\).

On dit que \(f\) est intégrable sur \(I\) quand l'intégrale \(\int_a^bf\) converge absolument.

On note alors aussi \(\int_If=\int_If\paren{t}\odif{t}=\int_a^bf\).
\end{defi}

L'ensemble des fonctions intégrables sur \(I\) est souvent noté \(\integ{1}{I}{\K}\), L comme Lebesgue, mathématicien français de la fin du 19ème et début du 20ème siècle. Par abus de notation, on écrit parfois \guillemets{\(f\) est \(\integ{1}{}{}\)} pour \guillemets{\(f\) est intégrable sur \(I\)}.

\begin{ex}
Soient \(a\in\R\) et \(\epsilon\in\Rps\).

\begin{itemize}
    \item La fonction \(t\mapsto\e{-t}\) est intégrable sur \(\intervie{\epsilon}{\pinf}\). \\ Plus généralement, la fonction \(t\mapsto t^a\e{-t}\) est intégrable sur \(\intervie{\epsilon}{\pinf}\) (voire \(\intervie{0}{\pinf}\) si \(a>-1\)). \\
    \item La fonction \(t\mapsto\dfrac{1}{t^\alpha}\) est intégrable sur \(\intervie{\epsilon}{\pinf}\) ssi \(\alpha>1\) et sur \(\intervei{0}{\epsilon}\) ssi \(\alpha<1\). \\
    \item La fonction \(\ln\) est intégrable sur \(\intervei{0}{\epsilon}\).
\end{itemize}
\end{ex}

\begin{prop}
\(\integ{1}{I}{\K}\) est un \(\K\)-espace vectoriel.
\end{prop}

\begin{prop}[Stricte positivité de l'intégrale]~\\
Si \(f\) est continue, intégrable sur \(I\) et \(\int_I\abs{f}=0\), alors \(f=0\).

Par contraposée, si \(f\) est continue, intégrable sur \(I\) et \(f\not=0\), alors \(\int_I\abs{f}>0\).
\end{prop}

\subsection{Théorème de comparaison des fonctions intégrables}

\begin{rappel}
Soient \(f\) et \(g\) deux fonctions définies au voisinage d'un point \(p\in\R\union\accol{\minf,\pinf}\).

\begin{itemize}
    \item \(f=\O{g}\) au voisinage de \(p\) signifie qu'il existe \(K>0\) et \(V\) un voisinage de \(p\) tels que \[\quantifs{\forall x\in V}\abs{f\paren{x}}\leq K\abs{g\paren{x}}.\]
    \item \(f=\o{g}\) au voisinage de \(p\) signifie qu'il existe une fonction \(\epsilon>0\) et \(V\) un voisinage de \(p\) tels que \[\quantifs{\forall x\in V}\abs{f\paren{x}}\leq\epsilon\paren{x}\abs{g\paren{x}}\qquad\text{et}\qquad\lim_{x\to p}\epsilon\paren{x}=0.\]
\end{itemize}

Dans le cas où \(g\) ne s'annule pas (ce qui, en pratique, est toujours le cas) :

\begin{itemize}
    \item \(f=\O{g}\) au voisinage de \(p\) signifie que \(\dfrac{f}{g}\) est bornée au voisinage de \(p\). \\
    \item \(f=\o{g}\) au voisinage de \(p\) signifie que \(\dfrac{f}{g}\) a pour limite \(0\) en \(p\).
\end{itemize}
\end{rappel}

\begin{theo}
Soit \(\paren{f,g}\in\contm[I]^2\).

\begin{itemize}
    \item Si \(\abs{f}\leq\abs{g}\) sur \(I\) et si \(g\) est intégrable, alors \(f\) est intégrable. \\
    \item Si \(f=\O{g}\) au voisinage des bornes ouvertes de \(I\) et \(g\) est intégrable, alors \(f\) est intégrable. C'est vrai en particulier si \(f=\o{g}\). \\
    \item Si \(f\sim g\) au voisinage des bornes ouvertes de \(I\), alors il y a équivalence entre l'intégrabilité de \(f\) et l'intégrabilité de \(g\).
\end{itemize}
\end{theo}

\begin{exo}
Montrez que la fonction \(t\mapsto\dfrac{\cos t}{\sqrt{t}}\e{-t}\) est intégrable sur \(\intervee{0}{\pinf}\).
\end{exo}

\begin{exo}
La fonction \(t\mapsto\dfrac{\ln t}{\cos t+t^2}\) est-elle intégrable sur \(\intervee{0}{\pinf}\) ?
\end{exo}

\section{Intégration des relations de comparaison}

Les résultats présentés portent sur des fonctions intégrables sur \(\intervie{a}{\pinf}\). On obtient évidemment des résultats analogues sur les autres types d'intervalles.

\subsection{Théorème de comparaison par domination}

Dans le cas convergent d'abord, les \guillemets{restes partiels} suivent la même relation de comparaison.

\begin{theo}
Soient \(f,g\) deux fonctions définies sur \(\intervie{a}{\pinf}\) avec \(g\) à valeurs dans \(\Rp\).

Si \(f=\O{g}\) et \(g\) est intégrable sur \(\intervie{a}{\pinf}\), alors \(f\) est intégrable sur \(\intervie{a}{\pinf}\).

De plus, \(\int_x^{\pinf}f=\O{\int_x^{\pinf}g}\) quand \(x\to\pinf\).

Si \(f=\o{g}\) et \(g\) est intégrable sur \(\intervie{a}{\pinf}\), alors \(f\) est intégrable sur \(\intervie{a}{\pinf}\).

De plus, \(\int_x^{\pinf}f=\o{\int_x^{\pinf}g}\) quand \(x\to\pinf\).
\end{theo}

Dans le cas divergent ensuite, les \guillemets{intégrales partielles} suivent aussi la même relation de comparaison.

\begin{theo}
Soient \(f,g\) deux fonctions définies sur \(\intervie{a}{\pinf}\) avec \(g\) à valeurs dans \(\Rp\).

Si \(f=\O{g}\) et \(f\) n'est pas intégrable sur \(\intervie{a}{\pinf}\), alors \(g\) n'est pas intégrable sur \(\intervie{a}{\pinf}\).

De plus, \(\int_a^xf=\O{\int_a^xg}\) quand \(x\to\pinf\).

Si \(f=\o{g}\) et \(f\) n'est pas intégrable sur \(\intervie{a}{\pinf}\), alors \(g\) n'est pas intégrable sur \(\intervie{a}{\pinf}\).

De plus, \(\int_a^xf=\o{\int_a^xg}\) quand \(x\to\pinf\).
\end{theo}

\subsection{Théorème de comparaison par équivalence}

\begin{theo}
Soient \(f,g\) deux fonctions définies sur \(\intervie{a}{\pinf}\) avec \(g\) à valeurs dans \(\Rp\).

Si \(f\sim g\), alors l'intégrabilité de \(f\) est équivalente à l'intégrabilité de \(g\).

De plus :

\begin{itemize}
    \item si les fonctions sont intégrables, alors les restes partiels sont équivalents : \[\int_x^{\pinf}f\simqd{x\to\pinf}\int_x^{\pinf}g\]
    \item si les fonctions ne sont pas intégrables, alors les intégrales partielles divergent et sont équivalentes : \[\int_a^xf\simqd{x\to\pinf}\int_a^xg.\]
\end{itemize}
\end{theo}


\chapter{Intégrales à paramètre}

\minitoc

Dans ce chapitre, \(\K\) désigne l'ensemble \(\R\) ou \(\C\). Les fonctions dans ce chapitre sont à valeurs dans \(\K\).

On considère dans ce chapitre des intégrales de la forme \(\int_a^bf\paren{p,t}\odif{t}\) où \(f\paren{p,t}\) est une expression qui dépend de deux variables \(p\) et \(t\), \(p\) pouvant être de n'importe quel type mais \(t\) bien sûr réelle. Par habitude, on distingue dans le vocabulaire ces deux variables : \(t\) est appelée la variable d'intégration (notez le \(\odif{t}\) qui le signale) et \(p\) est appelée le paramètre.

L'intégrale \(\int_a^bf\paren{p,t}\odif{t}\) est donc une intégrale qui dépend du paramètre \(p\) (mais qui ne dépend bien entendu pas de \(t\)) et l'objet de ce chapitre est d'étudier des résultats concernant cette dépendance vis-à-vis de \(p\) ; en somme, d'étudier des propriétés de l'application \(p\mapsto\int_a^bf\paren{p,t}\odif{t}\).

Les sections 2 et 3 étudient surtout le cas où \(p\) est un paramètre entier naturel, les suivantes le cas où \(p\) est un paramètre réel.

\section{Introduction}

Pour commencer, un exercice d'intervertion de symboles, qui marque le début de l'étude de ce problème général et qui va nous occuper durant quelques chapitres.

\begin{exo}
Pour \(n\in\Ns\), on pose \(f_n\paren{x}=n^2\e{-nx}\paren{1-\e{-x}}\).

Pour \(x\geq0\), que vaut \(\lim_{n\to\pinf}f_n\paren{x}\) ?

Montrez la convergence de l'intégrale \(\int_0^{\pinf}f_n\paren{x}\odif{x}\) et donnez sa valeur en fonction de \(n\).

Comparez \(\lim_{n\to\pinf}\int_0^{\pinf}f_n\paren{x}\odif{x}\) et \(\int_0^{\pinf}\paren{\lim_{n\to\pinf}f_n\paren{x}}\odif{x}\).

Quelle conclusion peut-on en tirer ?
\end{exo}

\section{Convergence simple}

\subsection{Convergence simple d'une suite de fonctions}

\begin{defi}
Soient \(A\) une partie de \(\R\) et \(\paren{f_n}\) une suite de fonctions définies sur \(A\).

On dit que la suite \(\paren{f_n}\) converge simplement sur \(A\) quand pour tout \(x\in A\), la suite numérique \(\paren{f_n\paren{x}}\) converge.

Dans ce cas, on peut définir une fonction \(f\) sur \(A\) en posant, \(\quantifs{\tpt x\in A}f\paren{x}=\lim_{n\to\pinf}f_n\paren{x}\).

La fonction \(f\) est alors appelée limite simple sur \(A\) de la suite \(\paren{f_n}\) et on dit que la suite \(\paren{f_n}\) converge simplement vers \(f\) sur \(A\).
\end{defi}

\begin{exo}
Étudiez, pour \(n\in\Ns\), la convergence simple de la suite de fonctions \(f_n:x\mapsto\dfrac{n\e{-x}+x^2}{n+x}\) sur \(\intervie{0}{\pinf}\)
\end{exo}

\begin{exo}
Même question avec la suite de fonctions \(f_n:x\mapsto\dfrac{x^n}{1+x^n}\) sur \(\intervie{0}{\pinf}\).
\end{exo}

\begin{exo}
Même question avec la suite de fonctions \(f_n:x\mapsto n^\alpha x^n\paren{1-x}\) où \(\alpha\) est un réel strictement positif.
\end{exo}

\subsection{Convergence simple d'une série de fonctions}

\begin{defi}
Soient \(A\) une partie de \(\R\) et \(\paren{f_n}\) une suite de fonctions définies sur \(A\).

On dit que la série \(\sum_{n\geq0}f_n\) converge simplement sur \(A\) quand pour tout \(x\in A\), la série numérique \(\sum_{n\geq0}f_n\paren{x}\) converge.

Autrement dit, la série de fonctions \(\sum_{n\geq0}f_n\) converge simplement sur \(A\) quand la suite des sommes partielles \(\paren{\sum_{k=0}^nf_k}\) converge simplement sur \(A\).

Dans ce cas, on peut définir une fonction \(f\) sur \(A\) en posant, \(\quantifs{\tpt x\in A}f\paren{x}=\sum_{n=0}^{\pinf}f_n\paren{x}\).

La fonction \(f\) est alors appelée (fonction) somme sur \(A\) de la série \(\sum_{n\geq0}f_n\).
\end{defi}

\begin{exo}
Étudiez, pour \(n\in\Ns\), la convergence simple de la série de fonctions \(f_n:x\mapsto\dfrac{nx^2}{n^3+x^2}\) sur \(\intervie{0}{\pinf}\).
\end{exo}

\begin{exo}
Même question avec la série de fonctions \(f_n:x\mapsto\dfrac{x^n}{1+x^n}\) sur \(\intervie{0}{\pinf}\).
\end{exo}

\begin{exo}
Même question avec la série de fonctions \(f_n:x\mapsto\dfrac{\sin\paren{nx}}{n^3+x^3}\) sur \(\intervie{0}{\pinf}\).
\end{exo}

\section{Suites et séries de fonctions intégrables}

Dans cette section, tous les théorèmes sont admis (démonstrations très difficiles !).

\subsection{Théorème de convergence dominée}

\begin{theo}
Soient \(I\) un intervalle et \(\paren{f_n}\) une suite de fonctions continues par morceaux sur \(I\).

Si

\begin{itemize}
    \item la suite \(\paren{f_n}\) converge simplement sur \(I\) vers une fonction \(f\) qui est continue par morceaux sur \(I\) \\
    \item il existe une fonction \(\phi\), intégrable sur \(I\) et à valeurs positives, telle que \[\quantifs{\tpt n\in\N}\abs{f_n}\leq\phi\text{ sur \(I\) (hypothèse de domination)}\]
\end{itemize}

alors les fonctions \(f\) et \(f_n\) sont toutes intégrables sur \(I\) et \(\int_If_n\tendqd{n\to\pinf}\int_If\).
\end{theo}

L'hypothèse de domination est essentielle ! Il s'agit donc de trouver une fonction \(\phi\) (dont on dit qu'elle domine la suite \(\paren{f_n}\)) intégrable et, surtout, qui ne dépend pas de \(n\) !

\begin{exo}
Montrez que la suite d'intégrales \(\paren{\int_0^{\pinf}\cos\paren{t}\e{-nt^2}\odif{t}}_{n\geq1}\) est bien définie et qu'elle converge vers \(0\).
\end{exo}

\begin{exo}
Montrez que la suite d'intégrales \(\paren{\int_0^{\nicefrac{\pi}{2}}\sin^nt\odif{t}}\) converge vers \(0\).
\end{exo}

\begin{exo}
Montrez que \(\quantifs{\tpt n\geq2}t\mapsto\dfrac{1}{1+t^n}\) est intégrable sur \(\intervie{0}{\pinf}\), puis donnez la limite des intégrales quand \(n\to\pinf\).
\end{exo}

Quitte à utiliser la caractérisation séquentielle de la limite, on peut étendre le théorème précédent à des fonctions paramétrées par un réel.

\begin{theo}
Soient \(I,A\) deux intervalles, \(\alpha\in\conj{A}\) et \(\paren{f_a}_{a\in A}\) une famille de fonctions continues par morceaux sur \(I\).

Si

\begin{itemize}
    \item \(\quantifs{\tpt x\in I}f_a\paren{x}\tendqd{a\to\alpha}f\paren{x}\) où \(f\) est une fonction continue par morceaux sur \(I\) \\
    \item il existe une fonction \(\phi\) intégrable sur \(I\) et à valeurs positives, telle que \[\quantifs{\tpt a\in A}\abs{f_a}\leq\phi\text{ sur \(I\) (hypothèse de domination)}\]
\end{itemize}

alors les fonctions \(f\) et \(f_a\) sont toutes intégrables sur \(I\) et \(\int_If_a\tendqd{a\to\alpha}\int_If\).
\end{theo}

\subsection{Théorème d'intégration terme à terme}

\begin{theo}
Soient \(I\) un intervalle et \(\paren{f_n}\) une suite de fonctions continues par morceaux sur \(I\).

Si

\begin{itemize}
    \item la série \(\sum_{n\geq0}f_n\) converge simplement sur \(I\) vers une fonction continue par morceaux sur \(I\) \\
    \item \(\quantifs{\tpt n\in\N}f_n\) est intégrable sur \(I\) \\
    \item la série numérique \(\sum_{n\geq0}\int_I\abs{f_n}\) converge
\end{itemize}

alors la fonction somme \(\sum_{n=0}^{\pinf}f_n\) est intégrable sur \(I\) et \(\int_I\sum_{n=0}^{\pinf}f_n=\sum_{n=0}^{\pinf}\int_If_n\).
\end{theo}

L'hypothèse de convergence de la série des intégrales est essentielle, mais hélas très contraignante. Il arrive souvent qu'il soit plus facile d'utiliser le théorème de convergence dominée sur les sommes partielles de la série de fonctions.

\begin{exo}
Justifiez l'existence et calculez \(\int_0^{\pinf}\dfrac{t}{\e{t}-1}\odif{t}\).
\end{exo}

\section{Fonctions définies par une intégrale à paramètre}

On s'intéresse aux propriétés des fonctions définies par des intégrales du type \(x\mapsto\int_If\paren{x,t}\odif{t}\). On dit que \(x\) est un paramètre de l'intégrale \(\int_If\paren{x,t}\odif{t}\).

\subsection{Continuité}

\begin{theo}
Soient \(A,I\) deux intervalles de \(\R\) et \(f:\paren{x,t}\mapsto f\paren{x,t}\) une fonction définie sur \(A\times I\).

Si

\begin{itemize}
    \item \(\quantifs{\tpt x\in A}t\mapsto f\paren{x,t}\) est continue par morceaux sur \(I\) \\
    \item \(\quantifs{\tpt t\in I}x\mapsto f\paren{x,t}\) est continue sur \(A\) \\
    \item il existe une fonction \(\phi\) intégrable sur \(I\) et à valeurs positives, telle que \[\quantifs{\tpt\paren{x,t}\in A\times I}\abs{f\paren{x,t}}\leq\phi\paren{t}\text{ (hypothèse de domination)}\]
\end{itemize}

alors \(\quantifs{\tpt x\in A}t\mapsto f\paren{x,t}\) est intégrable sur \(I\) et la fonction \(x\mapsto\int_If\paren{x,t}\odif{t}\) est continue sur \(A\).
\end{theo}

\begin{exo}
Montrez que la fonction \(g:x\mapsto\int_0^{\pinf}\dfrac{\cos\paren{xt^2}}{1+t^2}\odif{t}\) est définie et continue sur \(\R\).
\end{exo}

\begin{exo}
Montrez que la fonction \(h:u\mapsto\int_0^1\Arctan\paren{u+x\ln x}\odif{x}\) est définie et continue sur \(\R\).
\end{exo}

\subsection{Dérivabilité}

\begin{theo}
Soient \(A,I\) deux intervalles de \(\R\) et \(f:\paren{x,t}\mapsto f\paren{x,t}\) une fonction définie sur \(A\times I\).

Si

\begin{itemize}
    \item \(\quantifs{\tpt x\in A}t\mapsto f\paren{x,t}\) est continue par morceaux et intégrable sur \(I\) \\
    \item \(\quantifs{\tpt t\in I}x\mapsto f\paren{x,t}\) est de classe \(\classe{1}\) sur \(A\) \\
    \item \(\quantifs{\tpt x\in A}t\mapsto\pdv{f}{x}\paren{x,t}\) est continue par morceaux sur \(I\) \\
    \item il existe une fonction \(\phi\) intégrable et à valeurs positives, telle que \[\quantifs{\tpt\paren{x,t}\in A\times I}\abs{\pdv{f}{x}\paren{x,t}}\leq\phi\paren{t}\text{ (hypothèse de domination)}\]
\end{itemize}

alors la fonction \(g:x\mapsto\int_If\paren{x,t}\odif{t}\) est de classe \(\classe{1}\) sur \(A\) et \(\quantifs{\tpt x\in A}g\prim\paren{x}=\int_I\pdv{f}{x}\paren{x,t}\odif{t}\).
\end{theo}

\begin{exo}
Montrez que la fonction \(g:x\mapsto\int_0^\pi\cos\paren{x\sin t}\odif{t}\) est définie et de classe \(\classe{1}\) sur \(\R\).
\end{exo}

\begin{exo}
Montrez que la fonction \(g:x\mapsto\int_0^{\pinf}\cos\paren{xt^2}\e{-t}\odif{t}\) est définie et de classe \(\classe{1}\) sur \(\R\).
\end{exo}

Ce théorème est généralisable pour des dérivations d'ordre plus élevé.

\begin{theo}
Soient \(A,I\) deux intervalles de \(\R\) et \(f:\paren{x,t}\mapsto f\paren{x,t}\) une fonction définie sur \(A\times I\).

Si

\begin{itemize}
    \item \(\quantifs{\tpt x\in A}t\mapsto f\paren{x,t}\) est continue par morceaux et intégrable sur \(I\) \\
    \item \(\quantifs{\tpt t\in I}x\mapsto f\paren{x,t}\) est de classe \(\classe{k}\) sur \(A\) \\
    \item \(\quantifs{\tpt x\in A;\tpt j\in\interventierii{1}{k-1}}t\mapsto\pdv[order=j]{f}{x}\paren{x,t}\) est continue par morceaux et intégrable sur \(I\) \\
    \item il existe une fonction \(\phi\) intégrable sur \(I\) et à valeurs positives, telle que \[\quantifs{\tpt\paren{x,t}\in A\times I}\abs{\pdv[order=k]{f}{x}\paren{x,t}}\leq\phi\paren{t}\text{ (hypothèse de domination)}\]
\end{itemize}

alors la fonction \(g:x\mapsto\int_If\paren{x,t}\odif{t}\) est de classe \(\classe{k}\) sur \(A\) et \(\quantifs{\tpt x\in A;\tpt j\in\interventierii{1}{k}}g\deriv{j}\paren{x}=\int_I\pdv[order=j]{f}{x}\paren{x,t}\odif{t}\).
\end{theo}

\begin{exo}
Montrez que la fonction \(g:x\mapsto\int_0^{\pinf}\cos\paren{xt^2}\e{-t}\odif{t}\) est définie et de classe \(\classe{\infty}\) sur \(\R\).
\end{exo}

\section{Domination sur des sous-intervalles}

La continuité étant une propriété locale, il est souvent inutile d'avoir une domination globale sur \(A\) pour conclure. En général, on peut se contenter de domination sur des parties plus petites que \(A\), en général les segments inclus dans \(A\), ou toute famille recouvrante de parties de \(A\).

\begin{defi}
Soit \(A\) un intervalle.

Une famille \(\fami{F}\) de parties de \(A\) est dite recouvrante quand sa réunion est \(A\) : \(A=\bigunion_{X\in\fami{F}}X\).
\end{defi}

\begin{prop}
Soit \(A\) un intervalle.

La famille des segments inclus dans \(A\) est recouvrante : \(A=\bigunion_{\paren{a,b}\in A^2}\intervii{a}{b}\).
\end{prop}

On en déduit alors le théorème suivant, dont il vaut mieux à mon avis, sur chaque exercice, présenter le détail des idées.

\begin{theo}
Soient \(A,I\) deux intervalles de \(\R\) et \(f:\paren{x,t}\mapsto f\paren{x,t}\) une fonction définie sur \(A\times I\).

Soit \(\fami{F}\) une famille recouvrante de parties de \(A\).

Si

\begin{itemize}
    \item \(\quantifs{\tpt x\in A}t\mapsto f\paren{x,t}\) est continue par morceaux sur \(I\) \\
    \item \(\quantifs{\tpt t\in I}x\mapsto f\paren{x,t}\) est continue sur \(A\) \\
    \item pour toute partie \(F\) de \(\fami{F}\), il existe une fonction \(\phi_F\) intégrable sur \(I\) et à valeurs positives, telle que \[\quantifs{\tpt\paren{x,t}\in F\times I}\abs{f\paren{x,t}}\leq\phi_F\paren{t}\text{ (hypothèse de domination)}\]
\end{itemize}

alors \(\quantifs{\tpt x\in A}t\mapsto f\paren{x,t}\) est intégrable sur \(I\) et la fonction \(x\mapsto\int_If\paren{x,t}\odif{t}\) est continue sur \(A\).
\end{theo}

On a de même une version locale des théorèmes de dérivation sous le signe intégrale.

Autrement dit, au lieu de chercher à appliquer les théorèmes précédents directement sur \(A\), on trouve une famille recouvrante de sous-intervalles sur chacun desquels on peut appliquer les théorèmes précédents, conclure à la continuité ou dérivabilité sur chaque sous-intervalle, puis signaler que par réunion, la propriété reste valable sur \(A\).

\begin{exo}
Montrez que la fonction \(f:x\mapsto\int_0^{\pinf}\dfrac{\e{-xt}}{\sqrt{t}}\odif{t}\) est définie et continue sur \(\intervee{0}{\pinf}\).
\end{exo}

\begin{exo}
Montrez que la fonction \(f:x\mapsto\int_1^{\pinf}\dfrac{\ln\paren{1+xt}}{t^2}\odif{t}\) est définie et continue sur \(\intervie{0}{\pinf}\) et qu'elle est de classe \(\classe{1}\) sur \(\intervee{0}{\pinf}\). Donnez une expression simple de \(f\prim\paren{x}\) pour tout \(x>0\).
\end{exo}

\section{Complément : la fonction \(\Gamma\) d'Euler}

Pour \(x\in\R\), on pose, quand cela a un sens \[\Gamma\paren{x}=\int_0^{\pinf}t^{x-1}\e{-t}\odif{t}.\]

Cette fonction très courante a les propriétés suivantes :

\begin{itemize}
    \item \(\Gamma\) est définie sur \(\intervee{0}{\pinf}\) \\
    \item \(\Gamma\) est de classe \(\classe{\infty}\) sur \(\intervee{0}{\pinf}\) \\
    \item \(\quantifs{\tpt x>0}\Gamma\paren{x+1}=x\Gamma\paren{x}\) \\
    \item \(\quantifs{\tpt n\in\Ns}\Gamma\paren{n}=\paren{n-1}!\) \\
    \item \(\Gamma\paren{\dfrac{1}{2}}=\sqrt{\pi}\) \\
    \item il existe un unique \(\alpha\in\intervee{1}{2}\) tel que \(\Gamma\prim\paren{\alpha}=0\) et \(\Gamma\) est strictement décroissante sur \(\intervei{0}{\alpha}\) et strictement croissante sur \(\intervee{\alpha}{\pinf}\) \\
    \item \(\Gamma\) est convexe sur \(\intervee{0}{\pinf}\) \\
    \item \(\Gamma\) a des limites infinies en \(0\) et en \(\pinf\).
\end{itemize}


\chapter{Espaces préhilbertiens réels}

\minitoc

Dans tout ce chapitre, \(E\) désigne un \(\R\)-espace vectoriel.

\section{Généralités}

\subsection{Produit scalaire}

\begin{defi}
On appelle produit scalaire sur \(E\) toute application \(\phi\) de \(E^2\) dans \(\R\) qui est

\begin{itemize}
    \item bilinéaire (linéaire par rapport à chacune de ses deux variables) \\
    \item symétrique : \(\quantifs{\tpt\paren{x,y}\in E^2}\phi\paren{x,y}=\phi\paren{y,x}\) \\
    \item définie-positive : \(\quantifs{\tpt x\in E}\phi\paren{x,x}\geq0\) et \(\phi\paren{x,x}=0\ssi x=0\).
\end{itemize}
\end{defi}

\begin{rem}
Pour montrer que \(\phi\) est un produit scalaire, on montre en général d'abord que \(\phi\) est symétrique, puis qu'elle est linéaire à gauche, la linéarité à droite découlant alors de la symétrie.
\end{rem}

\begin{defi}
Quand \(E\) est muni d'un produit scalaire, on dit que \(E\) est un espace préhilbertien. Quand, de plus, \(E\) est de dimension finie, on dit que \(E\) est un espace euclidien.

En général, on note \(\ps{}{}\) les produits scalaires.
\end{defi}

\subsection{Exemples fondamentaux}

\begin{itemize}
    \item Le produit scalaire de la géométrie vérifie toutes ces propriétés. \\
    \item Si \(E=\R^n\), soit \(x=\paren{x_1,\dots,x_n}\) et \(y=\paren{y_1,\dots,y_n}\), on pose \(\phi\paren{x,y}=\sum_{i=1}^nx_iy_i\) : \(\phi\) est appelé le produit scalaire canonique sur \(\R^n\). \\
    \item Plus généralement, si \(E\) est un \(\R\)-espace vectoriel de dimension \(n\), alors à toute base \(\fami{B}\) de \(E\), on peut associer un produit scalaire : si \(x\) et \(y\) sont deux vecteurs de coordonnées \(X=\tcoords{x_1}{\vdots}{x_n}_{\fami{B}}\) et \(Y=\tcoords{y_1}{\vdots}{y_n}_{\fami{B}}\), on pose \(\phi\paren{x,y}=\sum_{i=1}^nx_iy_i\). L'expression matricielle du produit scalaire est alors \(\phi\paren{x,y}=\trans{X}Y\). \\
    \item Si \(a,b\) sont deux réels tels que \(a<b\), \(I=\intervii{a}{b}\) et \(E=\ensclasse{0}{I}{\R}\), alors pour \(f,g\) deux éléments de \(E\), on pose \(\phi\paren{f,g}=\int_a^bfg\) : \(\phi\) est un produit scalaire sur \(E\). \\
    \item Si \(I\) est un intervalle et \(E=\ensclasse{0}{I}{\R}\inter\integ{2}{I}{\R}\), ensemble des fonctions \(f\) à valeurs réelles, continues sur \(I\) et telles que \(f^2\) soit intégrables sur \(I\), alors pour \(f,g\) deux éléments de \(E\), on pose \(\phi\paren{f,g}=\int_Ifg\) : \(\phi\) est un produit scalaire sur \(E\). \\
    \item Dans \(\M{n}[\R]\), l'application \(\paren{A,B}\mapsto\tr\paren{\trans{A}B}\) est un produit scalaire, c'est même le produit scalaire canonique.
\end{itemize}

\subsection{Norme euclidienne}

\begin{defi}
Soit \(E\) un espace préhilbertien. On note \(\ps{}{}\) le produit scalaire sur \(E\).

On appelle norme euclidienne assoicée au produit scalaire l'application de \(E\) dans \(\Rp\) définie par \[\quantifs{\forall x\in E}\norme{x}=\sqrt{\ps{x}{x}}.\]
\end{defi}

\begin{rem}
Cette définition a bien un sens, car d'après les propriétés d'un produit scalaire, \(\quantifs{\tpt x\in E}\ps{x}{x}\geq0\) donc \(\sqrt{\ps{x}{x}}\) existe.
\end{rem}

On vérifie alors les résultats suivants, inspirés par la géométrie habituelle dans un triangle ou un parallélogramme.

\begin{prop}
Avec les mêmes notations, pour tout \(\paren{x,y}\in E^2\),

\begin{itemize}
    \item \(\norme{x+y}^2=\norme{x}^2+\norme{y}^2+2\ps{x}{y}\) (égalité d'Al-Kashi) \\
    \item \(\norme{x-y}^2=\norme{x}^2+\norme{y}^2-2\ps{x}{y}\) (égalité d'Al-Kashi) \\
    \item \(\norme{x+y}^2+\norme{x-y}^2=2\norme{x}^2+2\norme{y}^2\) (identité du parallélogramme) \\
    \item \(\norme{x+y}^2-\norme{x-y}^2=4\ps{x}{y}\) (identité de polarisation).
\end{itemize}
\end{prop}

Et encore

\begin{prop}
Avec les mêmes notations,

\begin{itemize}
    \item \(\abs{\ps{x}{y}}\leq\norme{x}\norme{y}\) (inégalité de Cauchy-Schwarz) \\
    \item \(\norme{x+y}\leq\norme{x}+\norme{y}\) (inégalité triangulaire) \\
    \item \(\quantifs{\tpt\lambda\in\R}\norme{\lambda x}=\abs{\lambda}\norme{x}\) \\
    \item \(\norme{x}=0\ssi x=0\).
\end{itemize}
\end{prop}

\begin{rem}
Il y a égalité dans l'inégalité de Cauchy-Schwarz ssi \(x\) et \(y\) sont colinéaires.

Il y a égalité dans l'inégalité triangulaire ssi \(x\) et \(y\) sont colinéaires de même sens.
\end{rem}

On dit qu'un vecteur de \(E\) est unitaire (ou normalisé) si sa norme vaut \(1\). À tout vecteur \(x\in E\excluant\accol{0}\), on associe deux vecteurs unitaires : \(\dfrac{x}{\norme{x}}\) et \(-\dfrac{x}{\norme{x}}\).

\begin{exo}
Soit \(\paren{a_1,\dots,a_n,b_1,\dots,b_n}\in\R^{2n}\).

Donnez une inégalité liant \(\sum_{k=1}^na_kb_k\), \(\sum_{k=1}^na_k^2\) et \(\sum_{k=1}^nb_k^2\).
\end{exo}

\begin{exo}
Soit \(f\in\ensclasse{0}{\intervii{a}{b}}{\Rps}\). Montrez que \[\paren{b-a}^2\leq\paren{\int_a^bf}\paren{\int_a^b\dfrac{1}{f}}.\]
\end{exo}

\subsection{Vecteurs orthogonaux}

\begin{defi}
Soit \(E\) un espace préhilbertien. On note \(\ps{}{}\) le produit scalaire sur \(E\).

On dit que deux vecteurs \(x,y\) sont orthogonaux (pour ce produit scalaire) quand \(\ps{x}{y}=0\).

On peut alors noter \(x\perp y\) pour signifier que \(x\) et \(y\) sont orthogonaux.

Plus généralement, si \(x_1,\dots,x_n\) sont \(n\) vecteurs de \(E\), on dit que la famille \(\paren{x_1,\dots,x_n}\) est une famille orthogonale quand \[\quantifs{\tpt\paren{i,j}\in\interventierii{1}{n}^2\text{ tel que }i\not=j}\ps{x_i}{x_j}=0.\]
\end{defi}

On retrouve alors le célèbre théorème de Pythagore.

\begin{prop}
Avec les mêmes notations, \[x\perp y\ssi\norme{x+y}^2=\norme{x}^2+\norme{y}^2.\]
\end{prop}

\begin{exo}
Soient \(E\) un \(\R\)-espace vectoriel de dimension au moins \(2\) et \(u,v\) deux vecteurs non-colinéaires de \(E\).

Montrez qu'il existe un produit scalaire sur \(E\) pour lequel \(u\) et \(v\) sont orthogonaux.
\end{exo}

\section{Bases orthonormées}

\subsection{Familles orthonormées}

\begin{defi}
Soit \(E\) un espace préhilbertien.

Une famille de vecteurs de \(E\) est dite orthonormée (ou orthonormale) quand elle est orthogonale et ses vecteurs sont unitaires.
\end{defi}

\begin{prop}
Une famille orthogonale sans vecteur nul est libre. En particulier, une famille orthonormée est libre.

Une famille orthonormée génératrice de \(E\) est donc une base orthonormée de \(E\).
\end{prop}

\begin{exo}
Généralisez l'exercice précédent.
\end{exo}

\subsection{Existence de bases orthonormées}

\begin{theo}
Soit \(E\) un espace euclidien.

Il existe dans \(E\) des bases orthonormées.

De plus, pour toute base \(\paren{v_1,\dots,v_n}\) de \(E\), il existe une base orthonormée \(\paren{e_1,\dots,e_n}\) de \(E\) telle que \[\quantifs{\tpt k\in\interventierii{1}{n}}\Vect{v_1,\dots,v_k}=\Vect{e_1,\dots,e_k}.\]
\end{theo}

La démonstration repose sur l'algorithme d'orthogonalisation/orthonormalisation de Schmidt.

On en déduit le théorème de la base orthonormée incomplète.

\begin{theo}
Soit \(E\) un espace euclidien.

Toute famille orthonormée de \(E\) peut être complétée en une base orthonormée de \(E\).
\end{theo}

\begin{exo}
Dans \(\R^n\) muni du produit scalaire canonique, on pose \(u=\paren{1,\dots,n}\).

Complétez la famille \(u\) en une base orthonormée de \(\R^n\).
\end{exo}

\subsection{Calculs en base orthonormée}

Soient \(E\) un espace euclidien et \(\fami{B}=\paren{e_1,\dots,e_n}\) une base orthonormée de \(E\).

Soient \(x,y\in E\), de coordonnées \(X=\tcoords{x_1}{\vdots}{x_n}_{\fami{B}}\) et \(Y=\tcoords{y_1}{\vdots}{y_n}_{\fami{B}}\).

Alors \[\ps{x}{y}=\sum_{i=1}^nx_iy_i=\trans{X}Y\qquad\norme{x}=\sqrt{\sum_{i=1}^nx_i^2}=\sqrt{\trans{X}X}\qquad\quantifs{\forall i\in\interventierii{1}{n}}x_i=\ps{x}{e_i}.\]

\section{Sous-espaces orthogonaux}

\subsection{Orthogonalité de deux sous-espaces vectoriels}

\begin{defi}
Soient \(E\) un espace préhilbertien, \(F,G\) deux sous-espaces vectoriels de \(E\) et \(u\in E\).

On dit que \(u\) est orthogonal (ou normal) à \(F\) quand \(u\) est orthogonal à tous les vecteurs de \(F\).

On dit que \(F\) et \(G\) sont orthogonaux quand tout vecteur de \(F\) et tout vecteur de \(G\) sont orthogonaux, autrement dit quand \[\quantifs{\tpt\paren{x,y}\in F\times G}\ps{x}{y}=0.\]
\end{defi}

\begin{prop}
Si \(F\) est de dimension finie et a pour famille génératrice \(\paren{v_1,\dots,v_k}\), alors \(u\) est orthogonal à \(F\) ssi \(\quantifs{\tpt i\in\interventierii{1}{k}}\ps{u}{v_i}=0\).
\end{prop}

\begin{prop}
Si \(F\) et \(G\) sont orthogonaux, alors ils sont en somme directe : \(F\inter G=\accol{0}\).
\end{prop}

\subsection{Orthogonal d'un sous-espace vectoriel}

\begin{defi}
Soient \(E\) un espace préhilbertien et \(F\) un sous-espace vectoriel de \(E\).

On note \(F\ortho\) l'ensemble des vecteurs normaux à \(F\) : \[F\ortho=\accol{v\in E\tq\quantifs{\forall x\in F}\ps{v}{x}=0}.\]
\end{defi}

Avec cette notation, on a clairement l'équivalence : \[F\text{ et }G\text{ sont orthogonaux}\ssi F\subset G\ortho\text{ ou, ce qui revient au même, }G\subset F\ortho.\]

\begin{theo}
Soient \(E\) un espace préhilbertien et \(F\) un sous-espace vectoriel de \(E\).

Alors \(F\ortho\) est un sous-espace vectoriel de \(E\), orthogonal à \(F\) et donc en somme directe avec \(F\).
\end{theo}

\begin{prop}
Soient \(E\) un espace préhilbertien et \(F\) un sous-espace vectoriel de \(E\).

Alors \(F\subset\paren{F\ortho}\ortho\).
\end{prop}

\begin{rem}
En général, \(F\ortho\) n'est pas supplémentaire à \(F\) et \(F\) n'est pas égal à \(\paren{F\ortho}\ortho\).
\end{rem}

\begin{rem}
Dans le cas où \(F\) est une droite vectorielle dirigée par un vecteur \(u\), on note plutôt \(G=u\ortho\) l'orthogonal de \(F\). Dans ce cas, \(u\ortho\) est un hyperplan et on dit alors que \(u\) est un vecteur normal à \(G\).
\end{rem}

\section{Projection orthogonale sur un sous-espace vectoriel de dimension finie}

\subsection{Projection orthogonale}

\begin{defi}
Soient \(E\) un espace préhilbertien et \(F\) un sous-espace vectoriel de \(E\) de dimension finie.

Alors \(F\ortho\) est un supplémentaire de \(F\), appelé le supplémentaire orthogonal de \(F\).

Le projecteur sur \(F\) parallèlement à \(F\ortho\) est appelé le projecteur orthogonal sur \(F\).

La symétrie orthogonale par rapport à \(F\) est la symétrie par rapport à \(F\) parallèlement à \(F\ortho\).
\end{defi}

Si on connaît une base orthonormée \(\paren{e_1,\dots,e_p}\) de \(F\), alors il est facile de calculer la projection orthogonale de \(x\) sur \(F\) : \[p_F\paren{x}=\sum_{i=1}^p\ps{x}{e_i}e_i.\]

On en déduit l'inégalité de Bessel.

\begin{prop}
Soient \(E\) un espace préhilbertien et \(F\) un sous-espace vectoriel de \(E\) de dimension finie.

Si \(p\) est le projecteur orthogonal sur \(F\), alors \(\quantifs{\tpt x\in E}\norme{p\paren{x}}\leq\norme{x}\).
\end{prop}

\subsection{Distance à un sous-espace vectoriel}

\begin{prop}
Soient \(E\) un espace préhilbertien, \(F\) un sous-espace vectoriel de \(E\) de dimension finie et \(x\in E\).

Soit \(y\) la projection orthogonale de \(x\) sur \(F\).

\(\quantifs{\Tpt z\in F}\norme{x-y}\leq\norme{x-z}\), avec égalité ssi \(z=y\).
\end{prop}

Autrement dit, le projeté orthogonal de \(x\) sur \(F\) est l'unique vecteur de \(F\) qui minimise la distance entre \(x\) et un point de \(F\).

\(\norme{x-y}\) est appelé la distance de \(x\) à \(F\), c'est la plus petite des distances entre \(x\) et un élément de \(F\), notée \(d\paren{x,F}\).

\begin{rem}
Tout ce qui précède est évidemment valable si \(E\) est de dimension finie.

Dans ce cas, pour tout sous-espace vectoriel \(F\) de \(E\), \(F\ortho\) est un supplémentaire de \(F\) dans \(E\).

Par conséquent, \(\dim F\ortho=\dim E-\dim F\).
\end{rem}


\chapter{Endomorphismes dans un espace euclidien}

\minitoc

Dans tout ce chapitre, \(E\) désigne un espace euclidien de dimension \(n\), muni du produit scalaire \(\ps{}{}\).

\section{Adjoint d'un endomorphisme}

\subsection{Représentation des formes linéaires}

Le théorème suivant est parfois appelé théorème de représentation de Riesz.

\begin{prop}
Soit \(\phi\) une forme linéaire sur \(E\).

Il existe un unique vecteur \(v\in E\) tel que \(\quantifs{\tpt x\in E}\phi\paren{x}=\ps{v}{x}\).
\end{prop}

\subsection{Adjoint}

\begin{prop}
Soit \(f\in\Lendo{E}\).

Il existe un unique endomorphisme \(g\in\Lendo{E}\) tel que \(\quantifs{\tpt\paren{x,y}\in E^2}\ps{f\paren{x}}{y}=\ps{x}{g\paren{y}}\).
\end{prop}

\begin{defi}
L'endomorphisme \(g\) est appelé l'adjoint de \(f\) est est noté \(f\adj\).
\end{defi}

Par bilinéarité et symétrie du produit scalaire, on en déduit les propriétés élémentaires de l'adjonction.

\begin{prop}
\begin{itemize}
    \item L'application \(f\mapsto f\adj\) est linéaire. \\
    \item \(\quantifs{\Tpt f\in\Lendo{E}}\paren{f\adj}\adj=f\). \\
    \item \(\quantifs{\Tpt\paren{f,g}\in\Lendo{E}^2}\paren{f\rond g}\adj=g\adj\rond f\adj\).
\end{itemize}
\end{prop}

\begin{exo}
Montrez que si \(f\) est un projecteur orthogonal, alors \(f\adj=f\).
\end{exo}

\begin{exo}
Soit \(f\in\Lendo{E}\).

Montrez que \(\Im f\adj=\paren{\ker f}\ortho\) et \(\ker f\adj=\paren{\Im f}\ortho\).

Comparez \(\rg f\) et \(\rg f\adj\).
\end{exo}

\begin{exo}
Soit \(f\in\Lendo{E}\).

Montrez que \(\rg f=\rg\paren{f\adj\rond f}\).
\end{exo}

\subsection{Matrice de l'adjoint}

\begin{prop}
Soient \(\fami{B}\) une base orthonormée de \(E\) et \(\paren{f,g}\in\Lendo{E}^2\).

On a \(g=f\adj\ssi\Mat{g}=\trans{\paren{\Mat{f}}}\).
\end{prop}

\begin{rem}
Attention, ceci n'est valable qu'en base orthonormée. En base quelconque, c'est plus compliqué.
\end{rem}

\begin{exo}
Soit \(f\in\Lendo{E}\) diagonalisable.

Montrez l'équivalence \[f\adj=f^2\ssi f\text{ est un projecteur orthogonal}.\]
\end{exo}

\subsection{Stabilité de sous-espaces vectoriels}

Une propriété remarquable et utile pour la suite du cours.

\begin{prop}
Soient \(f\in\Lendo{E}\) et \(F\) un sous-espace vectoriel de \(E\).

Si \(F\) est stable par \(f\), alors \(F\ortho\) est stable par \(f\adj\).
\end{prop}

\section{Orientation d'un \(\R\)-espace vectoriel de dimension finie}

Soit \(E\) un \(\R\)-espace vectoriel de dimension finie \(n\geq1\).

\begin{defi}
On dit que deux bases \(\fami{B},\fami{B}\prim\) de \(E\) ont la même orientation quand \(\detb\fami{B}\prim>0\), sinon on dit qu'elles sont d'orientations contraires.
\end{defi}

Orienter \(E\), c'est choisir une base de référence et déclarer directes toutes les bases qui ont la même orientation que cette base de référence. Les bases de l'autre classe d'équivalence sont dites indirectes (ou rétrogrades).

En géométrie classique, dans le plan ou l'espace, on convient systématiquement d'une orientation.

Dans toute la suite, \(E\) désigne un espace euclidien de dimension \(n\). On suppose aussi que \(E\) est orienté.

\section{Isométries vectorielles}

\begin{defi}
On appelle isométrie vectorielle (ou automorphisme orthogonal) tout endomorphisme de \(E\) qui conserve la norme : \(\quantifs{\tpt x\in E}\norme{f\paren{x}}=\norme{x}\).
\end{defi}

\begin{rem}
L'appellation \guillemets{automorphisme} n'est pas usurpée.
\end{rem}

L'ensemble des isométries vectorielles de \(E\) est noté \(\Orth{}[E]\).

\begin{prop}
\(\Orth{}[E]\) est un sous-groupe de \(\groupe{\GL{}[E]}[\rond]\).
\end{prop}

Les symétries orthogonales sont des isométries vectorielles. Parmi celles-ci, on distingue les symétries orthogonales par rapport à un hyperplan : on les appelle les réflexions.

On peut caractériser les isométries vectorielles de diverses façons.

\begin{prop}
Soit \(f\in\Lendo{E}\). Les propositions suivantes sont équivalentes :

\begin{itemize}
    \item \(f\) est une isométrie vectorielle \\
    \item \(f\) conserve le produit scalaire : \(\quantifs{\tpt\paren{x,y}\in E^2}\ps{f\paren{x}}{f\paren{y}}=\ps{x}{y}\) \\
    \item \(f\) transforme toute base orthonormée en base orthonormée \\
    \item \(f\) est un automorphisme et \(f\adj=f\inv\), ou, ce qui revient au même : \(f\adj\rond f=\id{E}\).
\end{itemize}
\end{prop}

\begin{exo}
Soient \(E\) un espace euclidien, \(a\in E\excluant\accol{0}\) et \(k\in\R\). On pose \(f:x\mapsto x+k\ps{x}{a}a\).

Montrez que \(f\) est linéaire, puis déterminez les conditions sur \(a\) et \(k\) pour que \(f\) soit une isométrie vectorielle.

Dans ce cas, reconnaissez-la.
\end{exo}

\section{Matrices orthogonales}

\begin{prop}
Soient \(\fami{B}\) une base orthonormée de \(E\) et \(f\in\Lendo{E}\). On pose \(A=\Mat{f}\).

On a \(f\in\Orth{}[E]\ssi\trans{A}A=I_n\).
\end{prop}

Attention ! Ceci n'est valable que si la base \(\fami{B}\) est orthonormée.

\begin{defi}
Une matrice carrée \(A\) est dite orthogonale quand \(\trans{A}A=I_n\), ce qui est équivalent à \(A\trans{A}=I_n\) ou \(A\) est inversible et \(A\inv=\trans{A}\).
\end{defi}

\begin{prop}
Une matrice de \(\M{n}[\R]\) est orthogonale quand ses colonnes sont de norme \(1\) et deux à deux orthogonales pour le produit scalaire canonique de \(\M{n\,1}[\R]\).
\end{prop}

Cela est également valable pour les lignes de la matrice.

\begin{exo}
Vérifiez que la matrice \(M=\dfrac{1}{7}\begin{pmatrix}
-2 & 6 & -3 \\
6 & 3 & 2 \\
-3 & 2 & 6
\end{pmatrix}\) est une matrice orthogonale, puis montrez qu'elle est la matrice d'une symétrie orthogonale donc vous préciserez les éléments caractéristiques.
\end{exo}

\begin{exo}
Déterminez les réels \(a\) et \(b\) tels que la matrice \(A=\begin{pmatrix}
a & b & b \\
b & a & b \\
b & b & a
\end{pmatrix}\) soit orthogonale. Reconnaissez la nature de l'isométrie vectorielle de matrice \(A\) dans une base orthonormée \(\fami{B}\).
\end{exo}

L'ensemble des matrices orthogonales est noté \(\Orth{n}[\R]\).

\begin{prop}
\(\Orth{n}[\R]\) est un sous-groupe compact de \(\groupe{\GL{n}[\R]}[\times]\).
\end{prop}

\subsection{Déterminant d'une isométrie vectorielle}

\begin{prop}
Si \(f\in\Orth{}[E]\), alors \(\det f\in\accol{-1,1}\).
\end{prop}

La réciproque est bien sûr fausse.

Les isométries vectorielles de déterminant \(1\) sont celles qui conservent l'orientation : les transforment les bases orthonormées directes en bases orthonormées directes. On les appelle les isométries vectorielles directes ou positives.

On note \(\SO{n}[\R]\) l'ensemble des matrices orthogonales de déterminant \(1\) et \(\SO{}[E]\) l'ensemble des isométries vectorielles positives.

\begin{prop}
\(\SO{}[E]\) est un sous-groupe de \(\Orth{}[E]\), appelé groupe spécial orthogonal de \(E\).

\(\SO{n}[\R]\) est un sous-groupe de \(\Orth{n}[\R]\), appelé groupe spécial orthogonal d'ordre \(n\).
\end{prop}

Les réflexions sont des isométries négatives.

\subsection{Changements de bases orthonormées}

\begin{prop}
Soient \(\fami{B}\) et \(\fami{B}\prim\) deux bases orthonormées de \(E\).

La matrice de passage de \(\fami{B}\) à \(\fami{B}\prim\) est une matrice orthogonale.
\end{prop}

L'intérêt de ce genre de changement de bases est que la difficulté liée au calcul de l'inverse de la matrice de passage disparaît :

\(X=PX\prim\) est équivalent à \(X\prim=\trans{P}X\) donc \(A\prim=P\inv AP\) devient \(A\prim=\trans{P}AP\).

\subsection{Produit mixte}

\begin{prop}
Soit \(\paren{v_1,\dots,v_n}\) une famille de \(n\) vecteurs de \(E\).

Le déterminant de \(\paren{v_1,\dots,v_n}\) dans une base orthonormée directe ne dépend pas du choix de cette base.
\end{prop}

Dans ce cas, on appelle produit mixte de \(\paren{v_1,\dots,v_n}\) le déterminant de cette famille dans n'importe quelle base orthonormée directe : il est noté habituellement \(\Det\paren{v_1,\dots,v_n}\) ou \(\croch{v_1,\dots,v_n}\).

Une conséquence directe de la définition du produit mixte est la caractérisation des bases directes.

\begin{prop}
Soit \(\paren{v_1,\dots,v_n}\) une famille de \(n\) vecteurs de \(E\).

La famille \(\paren{v_1,\dots,v_n}\) est une base directe de \(E\) ssi \(\croch{v_1,\dots,v_n}>0\).
\end{prop}

\subsection{Produit vectoriel en dimension 3}

Dans ce paragraphe, \(n=3\).

\begin{prop}
Soit \(\paren{u,v}\in E^2\).

Il existe un unique vecteur \(w\in E\) tel que \(\quantifs{\tpt x\in E}\croch{u,v,x}=\ps{w}{x}\).

Ce vecteur est appelé le produit vectoriel de \(u\) et \(v\) et est noté \(u\vecto v\) ou \(u\times v\).
\end{prop}

En base orthonormée directe, les coordonnées du produit vectoriel se calculent facilement. En base quelconque, c'est beaucoup plus pénible.

Notons quelques propriétés algébriques et géométriques du produit vectoriel.

\begin{prop}
\begin{itemize}
    \item L'application \(\vecto\) est bilinéaire et antisymétrique. \\
    \item \(u\vecto v=0\) ssi \(u\) et \(v\) sont colinéaires. \\
    \item Si \(u\) et \(v\) ne sont pas colinéaires, alors \(u\vecto v\) est un vecteur normal au plan \(\Vect{u,v}\) et la famille \(\paren{u,v,u\vecto v}\) est une base directe de \(E\). \\
    \item Si \(u\) et \(v\) sont unitaires et orthogonaux, alors la famille \(\paren{u,v,u\vecto v}\) est une base orthonormée directe de \(E\).
\end{itemize}
\end{prop}

\section{Étude en dimension 2}

\begin{prop}
\(\Orth{2}[\R]\) contient exclusivement les matrices suivantes :

\begin{itemize}
    \item les matrices de rotation \(R\paren{\theta}=\begin{pmatrix}
        \cos\theta & -\sin\theta \\
        \sin\theta & \cos\theta
    \end{pmatrix}\) \\
    \item les matrices de réflexions \(\begin{pmatrix}
        \cos\theta & \sin\theta \\
        \sin\theta & -\cos\theta
    \end{pmatrix}\)
\end{itemize}

où \(\theta\) est un réel quelconque.
\end{prop}

L'ensemble des matrices de rotation forme le sous-groupe \(\SO{2}[\R]\) : c'est l'ensemble des matrices orthogonales de déterminant \(1\).

Il est remarquable que ce groupe est commutatif, car en dimension \(n\geq3\), ce n'est plus le cas. En effet, il est facile de constater que l'application \(\theta\mapsto R\paren{\theta}\) est un morphisme surjectif de groupes de \(\groupe{\R}\) dans \(\groupe{\SO{2}[\R]}[\times]\) (dont le noyau est le sous-groupe \(2\pi\Z\) de \(\groupe{\R}\)).

Autrement dit, l'application \(\fonctionlambda{\U}{\SO{2}[\R]}{\e{\i\theta}}{R\paren{\theta}}\) est un isomorphisme de groupes.

\begin{prop}
En dimension \(2\), les isométries vectorielles sont :

\begin{itemize}
    \item les rotations vectorielles \\
    \item les réflexions vectorielles.
\end{itemize}
\end{prop}

\section{Réduction des isométries vectorielles ou des matrices orthogonales}

\subsection{Réduction des isométries vectorielles}

D'abord, deux résultats généraux sur les isométries vectorielles.

\begin{prop}
Soit \(f\in\Orth{}[E]\). On a :

\begin{itemize}
    \item \(\Sp{f}\subset\accol{-1,1}\) \\
    \item Si \(F\) est un sous-espace vectoriel de \(E\) stable par \(f\), alors \(F\ortho\) est aussi un sous-espace vectoriel de \(E\) stable par \(f\).
\end{itemize}
\end{prop}

De ces propriétés découlent le théorème suivant.

\begin{theo}
Soit \(f\in\Orth{}[E]\).

Il existe une base orthonormée de \(E\) dans laquelle la matrice de \(f\) est diagonale par blocs, les blocs étant des scalaires \(1\) ou \(-1\) ou des matrices \(\paren{2,2}\) de rotation.
\end{theo}

Les matrices diagonales par blocs sont donc du type suivant : \[D=\begin{pmatrix}
1 &  &  &  &  &  &  &  &  &  &  \\
& \ddots &  &  &  &  &  &  &  &  &  \\
&  & 1 &  &  &  &  &  &  &  &  \\
&  &  & -1 &  &  &  &  &  &  &  \\
&  &  &  & \ddots &  &  &  &  &  &  \\
&  &  &  &  & -1 &  &  &  &  &  \\
&  &  &  &  &  & \cos\theta_1 & -\sin\theta_1 &  &  &  \\
&  &  &  &  &  & \sin\theta_1 & \cos\theta_1 &  &  &  \\
&  &  &  &  &  &  &  & \ddots &  &  \\
&  &  &  &  &  &  &  &  & \cos\theta_k & -\sin\theta_k \\
&  &  &  &  &  &  &  &  & \sin\theta_k & \cos\theta_k \\
\end{pmatrix}.\]

\subsection{Réduction des matrices orthogonales}

\begin{defi}
Soient \(A,B\) deux matrices de \(\M{n}[\R]\).

On dit que \(A\) et \(B\) sont orthogonalement semblables (ou orthosemblables) quand il existe \(P\in\Orth{n}[\R]\) telle que \(B=P\inv AP=\trans{P}AP\).
\end{defi}

Deux matrices sont orthosemblables quand elles représentent le même endomorphisme dans des bases orthonormées différentes.

Le théorème de réduction précédent a une traduction matricielle.

\begin{theo}
Toute matrice orthogonale est orthosemblable à une matrice diagonale par blocs du type ci-dessus.

Pour tout \(A\in\Orth{n}[\R]\), il existe \(P\in\Orth{n}[\R]\) et \(D\) diagonale par blocs comme ci-dessus telles que \(A=\trans{P}DP\).
\end{theo}

\subsection{Étude en dimension 3}

À l'aide de ce résultat, on peut classifier les isométries vectorielles de \(E\) en dimension \(3\). Seule la réduction des rotations est au programme.

Dans la suite de cette section, \(E\) est un espace euclidien de dimension \(3\) et orienté.

\begin{prop}
Soit \(f\in\Orth{}[E]\). On pose \(F=\ker\paren{f-\id{E}}\). Alors

\begin{itemize}
    \item si \(\dim F=3\), alors \(f=\id{E}\) \\
    \item si \(\dim F=2\), alors \(f\) est la réflexion par rapport à \(F\) \\
    \item si \(\dim F=1\), alors \(f\) est une rotation d'axe \(F\) \\
    \item si \(\dim F=0\), alors \(f\) est une antirotation, \cad la composée d'une rotation et d'une réflexion dont l'axe et le plan de base respectifs sont orthogonaux.
\end{itemize}
\end{prop}

En étudiant les différents cas, on constate un lien entre le type de \(f\) et son déterminant.

\begin{prop}
Soit \(f\in\Orth{}[E]\) telle que \(f\not=\id{E}\).

\(f\) est une rotation ssi \(\det f=1\).
\end{prop}

Dans le cas où \(\det f=-1\), cette information ne suffit pas à connaître le type de \(f\). Cependant, si on connaît la matrice \(A\) de \(f\) dans une base orthonormée, alors on peut distinguer les cas 1 et 3.

\begin{prop}
Soient \(f\in\Lendo{E}\) et \(A\) la matrice de \(f\) dans une base orthonormée.

Si \(A\) est une matrice orthogonale et symétrique, alors \(f\) est une symétrie orthogonale.

\begin{itemize}
    \item Si \(\det f=1\), alors \(A\) est un demi-tour (une rotation d'angle \(\pi\)). \\
    \item Si \(\det f=-1\), alors \(A\) est une réflexion.
\end{itemize}
\end{prop}

Donc, si \(A\) est orthogonale de déterminant \(-1\) et non-symétrique, alors \(f\) est une antirotation.

\begin{exo}
Reconnaissez la nature de l'endomorphisme dont la matrice dans une base orthonormée \(\fami{B}\) est \(\dfrac{1}{15}\begin{pmatrix}
-11 & 10 & 2 \\
-2 & -5 & 14 \\
10 & 10 & 5
\end{pmatrix}\) et précisez ses éléments caractéristiques.
\end{exo}

\begin{exo}
Même exercice avec la matrice \(\dfrac{1}{7}\begin{pmatrix}
6 & -2 & -3 \\
3 & 6 & 2 \\
2 & -3 & 6
\end{pmatrix}\).
\end{exo}

\begin{exo}
Même exercice avec la matrice \(\dfrac{1}{7}\begin{pmatrix}
6 & -3 & -2 \\
3 & 2 & 6 \\
2 & 6 & -3
\end{pmatrix}\).
\end{exo}

\begin{exo}
Soit \(\fami{B}=\paren{i,j,k}\) une base orthonormée directe de \(E\) de dimension \(3\).

Déterminez la matrice dans la base \(\fami{B}\) de la rotation d'axe orienté par \(i+j+k\) et d'angle \(\dfrac{\pi}{3}\).
\end{exo}

\section{Endomorphismes auto-adjoints}

\subsection{Définition et propriétés}

\begin{defi}
On dit qu'un endomorphisme \(f\) de \(E\) est auto-adjoint quand \(f\adj=f\), autrement dit quand \[\quantifs{\forall\paren{x,y}\in E^2}\ps{f\paren{x}}{y}=\ps{x}{f\paren{y}}.\]
\end{defi}

On rencontre encore très souvent le mot \guillemets{symétrique} pour \guillemets{auto-adjoint}.

\begin{ex}
\begin{itemize}
    \item Les projecteurs orthogonaux sont des endomorphismes auto-adjoints (mais pas des endomorphismes orthogonaux !). \\
    \item Les symétries orthogonales sont aussi des endomorphismes auto-adjoints.
\end{itemize}
\end{ex}

\begin{prop}
Soient \(\fami{B}\) une base orthonormée de \(E\) et \(f\in\Lendo{E}\).

\(f\) est un endomorphisme auto-adjoint ssi sa matrice dans la base \(\fami{B}\) est symétrique.
\end{prop}

\begin{cor}
L'ensemble des endomorphismes auto-adjoints est un sous-espace vectoriel de \(\Lendo{E}\), de dimension \(\dfrac{n\paren{n+1}}{2}\).
\end{cor}

Il est noté \(\sym{}[E]\).

\subsection{Théorème spectral}

Il y a essentiellement un seul résultat à connaître sur les endomorphismes auto-adjoints ! On commence par deux lemmes.

\begin{lem}
Le polynôme caractéristique d'un endomorphisme auto-adjoint est scindé sur \(\R\).
\end{lem}

\begin{lem}
Si un sous-espace vectoriel \(F\) est stable par un endomorphisme auto-adjoint, alors \(F\ortho\) l'est aussi.
\end{lem}

\begin{theo}
Les sous-espaces propres d'un endomorphisme auto-adjoint sont deux à deux orthogonaux et leur somme directe est \(E\).

Autrement dit, tout endomorphisme auto-adjoint est diagonalisable en base orthonormée, \cad qu'il existe une base orthonormée de vecteurs propres.
\end{theo}

On dit que les endomorphismes auto-adjoints sont orthodiagonalisables.

\begin{rem}
La réciproque est vraie et presque évidente : si un endomorphisme est orthodiagonalisable, alors il est auto-adjoint.
\end{rem}

\begin{exo}[Un grand classique à savoir refaire]
Soit \(u\) un endomorphisme auto-adjoint de \(E\), \(B\) la boule-unité fermée de \(E\) et \(S\) la sphère-unité de \(E\).

On pose \(\alpha\) la plus petite des valeurs propres de \(u\) et \(\beta\) la plus grande.

Montrez que \(\inf_{x\in S}\ps{x}{u\paren{x}}=\alpha\) et \(\sup_{x\in B}\ps{x}{u\paren{x}}=\sup_{x\in S}\ps{x}{u\paren{x}}=\beta\).
\end{exo}

\begin{exo}[Un prolongement de l'exercice précédent]
Montrez que l'application \(N:\sym{}[E]\to\Rp\) définie par \(N\paren{u}=\sup_{x\in B}\abs{\ps{x}{u\paren{x}}}\) est une norme sur \(\sym{}[E]\).
\end{exo}

Le théorème précédent a une version matricielle.

\begin{theo}
Une matrice réelle est orthosemblable à une matrice diagonale ssi elle est symétrique.
\end{theo}

On dit que les matrices symétriques réelles sont orthodiagonalisables.

\begin{exo}
Orthodiagonalisez la matrice suivante : \[A=\begin{pmatrix}
0 & 1 & -1 \\
1 & 0 & -1 \\
-1 & -1 & 0
\end{pmatrix}.\]
\end{exo}

\begin{rem}
La condition \guillemets{réelle} est indispensable dans le théorème spectral !
\end{rem}

\section{Endomorphismes auto-adjoints positifs, définis-positifs}

\subsection{Endomorphismes auto-adjoints positifs}

\begin{defi}
Soit \(f\in\sym{}[E]\).

On dit que \(f\) est un endomorphisme auto-adjoint positif quand \(\quantifs{\tpt x\in E}\ps{f\paren{x}}{x}\geq0\).

On dit que \(f\) est un endomorphisme auto-adjoint défini-positif quand \(\quantifs{\tpt x\in E\excluant\accol{0}}\ps{f\paren{x}}{x}>0\).
\end{defi}

On note \(\sympos{}[E]\) l'ensemble des endomorphismes auto-adjoints positifs et \(\symdefpos{}[E]\) celui des endomorphismes auto-adjoints définis positifs. Attention, ces deux ensembles ne sont pas des espaces vectoriels et ne sont pas stables par composition.

Ces endomorphismes sont couramment présents dans les théories physiques et sont l'objet de propriétés spécifiques.

On donne par exemple une caractérisation simple à l'aide de valeurs propres.

\begin{prop}
Soit \(f\in\sym{}[E]\).

On a \(f\in\sympos{}[E]\) ssi les valeurs propres de \(f\) sont positives.

De même, \(f\in\symdefpos{}[E]\) ssi les valeurs propres de \(f\) sont strictement positives.
\end{prop}

En particulier, \(\symdefpos{}[E]=\sympos{}[E]\inter\GL{}[E]\).

\subsection{Matrices symétriques positives}

\begin{defi}
Soit \(A\in\sym{n}[\R]\).

On dit que \(A\) est une matrice symétrique positive quand \(\quantifs{\tpt X\in\R^n}\trans{X}AX\geq0\).

On dit que \(A\) est une matrice symétrique définie-positive quand \(\quantifs{\tpt X\in\R^n\excluant\accol{0}}\trans{X}AX>0\).
\end{defi}

Les matrices symétriques positives (respectivement définies-positives) sont donc les matrices dans des bases orthonormées des endomorphismes auto-adjoints positifs (respectivement définis-positifs).

On note \(\sympos{n}[\R]\) l'ensemble des matrices symétriques positives et \(\symdefpos{n}[\R]\) celui des matrices symétriques définies-positives. Attention, ces deux ensembles ne sont pas des espaces vectoriels et ne sont pas stables par produit.

\begin{prop}
Soit \(A\in\sym{n}[\R]\).

On a \(A\in\sympos{n}[\R]\) ssi les valeurs propres de \(A\) sont positives.

De même, \(A\in\symdefpos{n}[\R]\) ssi les valeurs propres de \(A\) sont strictement positives.
\end{prop}


\chapter{Fonctions vectorielles}

\minitoc

Dans tout le chapitre, \(I\) et \(J\) désignent des intervalles de \(\R\) contenant au moins deux points et \(n\) désigne un entier naturel non-nul.

\(E\) et \(F\) désignent des espaces vectoriels normés de dimension finie. Par choix d'une base et sachant que les normes sont équivalentes, on peut se ramener à une étude sur \(\R^n\).

\section{Dérivée en un point}

\subsection{Dérivabilité en un point}

\begin{defi}
Soient \(f:I\to E\) et \(a\in I\).

On appelle (fonction) taux d'accroissement de \(f\) en \(a\) l'application \[\fonction{\tau_a}{I\excluant\accol{a}}{\R}{x}{\dfrac{f\paren{x}-f\paren{a}}{x-a}}.\]

On dit que \(f\) est dérivable en \(a\) quand \(\dfrac{f\paren{x}-f\paren{a}}{x-a}\) tend vers une limite \(l\in E\) quand \(x\to a\).

Si \(f\) est dérivable en \(a\), on appelle dérivée de \(f\) en \(a\) le vecteur \[f\prim\paren{a}=l=\lim_{x\to a}\dfrac{f\paren{x}-f\paren{a}}{x-a}.\]
\end{defi}

\begin{rem}
Lorsqu'on étudie \guillemets{à la main} la limite du taux d'accroissement, on effectue très souvent le changement d'origine \(h=x-a\) et on étudie \(\dfrac{f\paren{a+h}-f\paren{a}}{h}\) quand \(h\to0\) (ce qui permet d'utiliser les équivalents ou les développements limités usuels).
\end{rem}

\subsection{Interprétation géométrique, développement limité d'ordre 1, continuité}

\begin{prop}[Développement limité d'ordre 1 d'une fonction dérivable]
Soient \(f:I\to E\) et \(a\in I\).

\(f\) est dérivable en \(a\) ssi il existe \(m\in E\) tel que \(f\paren{x}\egqd{x\to a}f\paren{a}+\paren{x-a}m+\o{x-a}\).

Lorsque ces énoncés sont vrais, on a \(f\prim\paren{a}=m\).
\end{prop}

On peut utiliser un développement limité à un ordre au moins \(1\) en \(a\) pour montrer que la fonction est dérivable en \(a\).

\begin{prop}
Si une fonction est dérivable en \(a\), alors elle est continue en \(a\).
\end{prop}

\begin{rem}
Évidemment, la réciproque est fausse !
\end{rem}

Dans le cas où \(f\prim\paren{a}\not=0\) : un vecteur directeur de la droite passant par \(f\paren{a}\) et \(f\paren{x}\) est par exemple \(f\paren{x}-f\paren{a}\), mais aussi \(\dfrac{f\paren{x}-f\paren{a}}{x-a}\) quand \(x\not=a\).

Autrement dit, \(f\) est dérivable en \(a\) quand ce vecteur directeur de la droite passant par \(f\paren{a}\) et \(f\paren{x}\) a une limite dans \(E\), autrement dit, quand la courbe décrite par \(f\) a une tangente en \(a\) : c'est la droite passant par \(f\paren{a}\) et dirigée par \(f\prim\paren{a}\).

Si \(x\) représente une variable de temps, le vecteur dérivée \(f\prim\paren{a}\) est le vecteur vitesse instantanée au point \(a\). Son sens donne le sens de parcours de la courbe.

\subsection{Dérivées à gauche, dérivées à droite}

\begin{defi}
Soient \(f:I\to E\) et \(a\in I\).

On dit que \(f\) est dérivable à droite (respectivement à gauche) en \(a\) quand \(\dfrac{f\paren{x}-f\paren{a}}{x-a}\) admet une limite à droite (respectivement à gauche) dans \(\R^n\) quand \(x\) tend vers \(a\).

Lorsque \(f\) est dérivable à droite (respectivement à gauche) en \(a\), on appelle cette limite la dérivée à droite (respectivement à gauche) de \(f\) en \(a\) et on la note \(f_d\prim\paren{a}\) (respectivement \(f_g\prim\paren{a}\)).
\end{defi}

\begin{prop}
Si \(f\) est dérivable à droite (respectivement à gauche) en \(a\), alors \(f\) est continue à droite (respectivement à gauche) en \(a\).

En outre, \(f\) est dérivable en \(a\) ssi \(f\) est dérivable à droite et à gauche en \(a\) et \(f_d\prim\paren{a}=f_g\prim\paren{a}\).

Lorsque c'est le cas, \(f\prim\paren{a}\) est égale à la valeur commune de \(f_d\prim\paren{a}\) et \(f_g\prim\paren{a}\).
\end{prop}

On peut parler, dans le cas de vecteurs non-nuls, de demi-tangentes à gaucne ou à droite.

\subsection{Lien avec les coordonnées}

\(E\) étant de dimension finie, on choisit une base \(\fami{B}=\paren{e_1,\dots,e_n}\) de \(E\). Soit \(f:I\to E\).

On a \(\quantifs{\tpt t\in I}f\paren{t}=\sum_{i=1}^nf_i\paren{t}e_i\).

On associe ainsi à \(f\) ses \(n\) fonctions-coordonnées dans la base \(\fami{B}\) (qui sont des fonctions à valeurs dans \(\R\)).

\begin{prop}
Avec les mêmes notations, il y a équivalence entre \guillemets{\(f\) est dérivable en \(a\)} et \guillemets{les fonctions \(f_i\) sont dérivables en \(a\)}.

Dans ce cas, on a \(f\prim\paren{a}=\sum_{i=1}^nf_i\prim\paren{a}e_i\).
\end{prop}

Autrement dit, travailler avec une fonction à valeurs dans \(E\) revient à travailler avec \(n\) fonctions numériques simultanément.

\subsection{Théorèmes opératoires}

\begin{prop}
Soient \(f:I\to E\), \(g:I\to E\), \(a\in I\) et \(\lambda\in\K\).

Si \(f\) et \(g\) sont dérivables en \(a\), alors \(f+g\) et \(\lambda f\) sont dérivables en \(a\) et \[\paren{f+g}\prim\paren{a}=f\prim\paren{a}+g\prim\paren{a}\qquad\paren{\lambda f}\prim\paren{a}=\lambda f\prim\paren{a}.\]
\end{prop}

\begin{prop}
Soient \(\phi:I\to\R\) et \(f:J\to E\) telles que \(\phi\paren{I}\subset J\) et \(a\in I\).

Si \(\phi\) est dérivable en \(a\) et \(f\) est dérivable en \(\phi\paren{a}\), alors \(f\rond\phi\) est dérivable en \(a\) et \[\paren{f\rond\phi}\prim\paren{a}=\phi\prim\paren{a}f\prim\paren{\phi\paren{a}}.\]
\end{prop}

\begin{prop}
Soient \(f:I\to E\), \(L\in\L{E}{F}\) et \(a\in I\).

Si \(f\) est dérivable en \(a\), alors \(L\rond f\) l'est aussi et \[\paren{L\rond f}\prim\paren{a}=L\rond f\prim\paren{a}.\]
\end{prop}

\begin{ex}
Si \(A\in\M{n}[\R]\) et \(X\) est une fonction de \(I\) dans \(\M{n\,1}[\R]\) dérivable en \(a\), alors \(Y:t\mapsto AX\paren{t}\) est dérivable en \(a\) et \(Y\prim\paren{a}=AX\prim\paren{a}\).
\end{ex}

\begin{prop}
Soient \(f,g:I\to E\) et \(B:E^2\to F\) bilinéaire.

Si \(f\) et \(g\) sont dérivables en \(a\), alors \(\paren{f,g}\) l'est aussi et \(\paren{B\paren{f,g}}\prim\paren{a}=B\paren{f\prim,g}\paren{a}+B\paren{f,g\prim}\paren{a}\).
\end{prop}

\begin{ex}
\begin{itemize}
    \item Si \(A\) et \(B\) sont deux fonctions de \(I\) dans \(\M{n}[\R]\) dérivables en \(a\), alors \(M:t\mapsto A\paren{t}B\paren{t}\) est aussi dérivable en \(a\) et \(M\prim\paren{a}=A\prim\paren{a}B\paren{a}+A\paren{a}B\prim\paren{a}\). \\
    \item Si \(u\) et \(v\) sont deux fonctions à valeurs dans \(E\), espace euclidien, et dérivables en \(a\), alors \(p:t\mapsto\ps{u\paren{t}}{v\paren{t}}\) est dérivable en \(a\) et \(p\prim\paren{a}=\ps{u\prim\paren{a}}{v\paren{a}}+\ps{u\paren{a}}{v\prim\paren{a}}\).
\end{itemize}
\end{ex}

\begin{prop}
Soient \(f_1,\dots,f_p:I\to E\), \(a\in I\) et \(M\) une application \(p\)-linéaire de \(E^p\) dans \(F\).

Si \(f_1,\dots,f_p\) sont dérivables en \(a\), alors \(M\paren{f_1,\dots,f_p}\) l'est aussi et \[\paren{M\paren{f_1,\dots,f_p}}\prim\paren{a}=\sum_{i=1}^pM\paren{f_1,\dots,f_i\prim,\dots,f_p}\paren{a}.\]
\end{prop}

\begin{ex}
Si \(M\) est une fonction de \(I\) dans \(\M{n}[\R]\) dérivable en \(a\in I\), alors \(d:t\mapsto\det M\paren{t}\) est aussi dérivable en \(a\) et \(d\prim\paren{a}=\sum_{k=1}^n\det M_k^c\paren{a}\) où \(M_k^c\paren{a}\) est la matrice obtenue à partir de \(M\paren{a}\) en remplaçant sa \(k\)-ème colonne par sa dérivée en \(a\).

On a aussi \(d\prim\paren{a}=\sum_{k=1}^n\det M_k^l\paren{a}\) où \(M_k^l\paren{a}\) est la matrice obtenue à partir de \(M\paren{a}\) en remplaçant sa \(k\)-ème ligne par sa dérivée en \(a\).
\end{ex}

\section{Fonction dérivée}

\begin{defi}
Soit \(f:I\to E\).

\begin{itemize}
    \item On dit que \(f\) est dérivable sur \(I\) si \(f\) est dérivable en tout point \(a\in I\), ce qui revient à dire, après choix d'une base, que les fonctions coordonnées de \(f\) sont dérivables sur \(I\). \\
    \item Si \(f\) est dérivable sur \(I\), on définit sa fonction dérivée par \[\fonction{f\prim}{I}{\R}{x}{f\prim\paren{x}}\]
\end{itemize}
\end{defi}

\begin{prop}[Théorèmes d'opérations sur les fonctions dérivables]
Soient \(f,g\) deux fonctions définies sur \(I\) et \(\lambda\in\K\).

Si \(f\) et \(g\) sont dérivables sur \(I\), alors \(f+g\) et \(\lambda f\) sont dérivables sur \(I\) et \[\paren{f+g}\prim=f\prim+g\prim\qquad\paren{\lambda f}\prim=\lambda f\prim.\]
\end{prop}

\begin{prop}
Soient \(\phi:I\to\R\) et \(f:J\to E\) telles que \(\phi\paren{I}\subset J\).

Si \(\phi\) est dérivable sur \(I\) et \(f\) est dérivable sur \(J\), alors \(f\rond\phi\) est dérivable sur \(I\) et \[\paren{f\rond\phi}\prim=\phi\prim\times f\prim\rond\phi.\]
\end{prop}

\begin{prop}
Soient \(f:I\to E\) et \(L\in\L{E}{F}\).

Si \(f\) est dérivable sur \(I\), alors \(L\rond f\) l'est aussi et \(\paren{L\rond f}\prim=L\rond f\prim\).
\end{prop}

\begin{prop}
Soient \(f,g:I\to E\) et \(B:E^2\to F\) bilinéaire.

Si \(f\) et \(g\) sont dérivables sur \(I\), alors \(B\paren{f,g}\) l'est aussi et \(\paren{B\paren{f,g}}\prim=B\paren{f\prim,g}+B\paren{f,g\prim}\).
\end{prop}

\begin{prop}
Soient \(f_1,\dots,f_p:I\to E\) et \(M:E^p\to F\) \(p\)-linéaire.

Si \(f_1,\dots,f_p\) sont dérivables sur \(I\), alors \(M\paren{f_1,\dots,f_p}\) l'est aussi et \[\paren{M\paren{f_1,\dots,f_p}}\prim=\sum_{k=1}^pM\paren{f_1,\dots,f_k\prim,\dots,f_p}.\]
\end{prop}

\begin{exo}
Montrez que si un point \(M\) se déplace sur une sphère de centre \(A\), sa vitesse est toujours orthogonale au vecteur \(\overrightarrow{AM}\).

Montrez que la réciproque est vraie : si \(A\) est un point fixe et si la vitesse de \(M\) est toujours orthogonale au vecteur \(\overrightarrow{AM}\), alors \(M\) se déplace sur une sphère.
\end{exo}

\begin{exo}
Soit \(S:I\to\M{n}[\R]\) dérivable sur \(I\) telle que \(\quantifs{\tpt t\in I}S\paren{t}\) est une matrice de symétrie.

Montrez que \(\quantifs{\tpt t\in I}\tr\paren{S\paren{t}S\prim\paren{t}}=0\).
\end{exo}

\begin{exo}
Soit \(A:I\to\GL{n}[\R]\) dérivable sur \(I\).

Montrez que \(B:t\mapsto A\inv\paren{t}\) est dérivable sur \(I\) et calculez sa dérivée en fonction de celle de \(A\).
\end{exo}

\begin{exo}
Soit \(M:I\to\M{n}[\R]\) dérivable sur \(I\), intervalle contenant \(0\). On pose \(f:t\mapsto\det\paren{I_n+tM\paren{t}}\).

Justifiez que \(f\) est dérivable sur \(I\) et calculez \(f\prim\paren{0}\).
\end{exo}

\section{Dérivées successives}

\subsection{Définitions et exemples}

Si \(f\) est dérivable sur \(I\), \(f\prim\) est une fonction définie sur \(I\). On peut donc essayer de la dériver : quand c'est possible, on obtient la dérivée seconde \(f\seconde\), et ainsi de suite...

\begin{defi}
Soit \(f:I\to E\). On définit, par récurrence, les notions suivantes :

\begin{itemize}
    \item Par convention, on dit que \(f\) est toujours dérivable \(0\) fois sur \(I\) et on définit la dérivée d'ordre \(0\) de \(f\) par \(f\deriv{0}=f\). \\
    \item Soit \(k\in\Ns\). \\\\ On dit que \(f\) est dérivable \(k\) fois sur \(I\) si elle est dérivable \(\paren{k-1}\) fois sur \(I\) et que sa dérivée d'ordre \(\paren{k-1}\), la fonction \(f\deriv{k-1}\), est dérivable sur \(I\). \\\\ On définit alors la dérivée d'ordre \(k\) par \[f\deriv{k}=\paren{f\deriv{k-1}}\prim.\]
\end{itemize}
\end{defi}

\begin{rem}
\(f\deriv{0}\) désigne \(f\), \(f\deriv{1}\) désigne \(f\prim\) et \(f\deriv{2}\) est aussi notée \(f\seconde\). À partir de trois dérivations, on n'utilise plus de primes.
\end{rem}

La dérivée d'ordre \(k\) de \(f\) est également notée \(\odv[order=k]{f}{t}\).

Enfin, il est facile de montrer que \(f\) est \(\paren{p+q}\) fois dérivable sur \(I\) ssi \(f\) est \(p\) fois dérivable et \(f\deriv{p}\) est \(q\) fois dérivable sur \(I\). Dans ce cas, on a l'égalité \[\paren{f\deriv{p}}\deriv{q}=f\deriv{p+q}.\]

\begin{defi}
Soit \(k\in\N\).

On dit que \(f\) est de classe \(\classe{k}\) sur \(I\) si \(f\) est dérivable \(k\) fois sur \(I\) et que \(f\deriv{k}\) est une fonction continue sur \(I\).

On dit que \(f\) est de classe \(\classe{\infty}\) sur \(I\) si \(f\) est dérivable \(k\) fois sur \(I\) quel que soit \(k\in\N\).

Pour tout \(k\in\N\union\accol{\infty}\), on notera \(\ensclasse{k}{I}{E}\) l'ensemble des fonctions de classe \(\classe{k}\) sur \(I\).
\end{defi}

\begin{rem}
\guillemets{\(f\) est de classe \(\classe{0}\) sur \(I\)} signifie \guillemets{\(f\) est continue sur \(I\)}.

\guillemets{\(f\) est de classe \(\classe{1}\) sur \(I\)} signifie \guillemets{\(f\) est dérivable sur \(I\) et \(f\prim\) est continue sur \(I\)}.
\end{rem}

\begin{prop}
Soient \(f\) une fonction définie sur \(I\) et \(k\in\Ns\).

\begin{itemize}
    \item Si \(f\) est de classe \(\classe{k}\) sur \(I\), alors \(\quantifs{\tpt p\in\interventierii{0}{k}}f\) est de classe \(\classe{p}\) sur \(I\). \\
    \item \(f\) est de classe \(\classe{k}\) sur \(I\) ssi \(f\) est dérivable sur \(I\) et \(f\prim\) est de classe \(\classe{k-1}\) sur \(I\), ou, ce qui revient au même, \(f\) est de classe \(\classe{k-1}\) sur \(I\) et \(f\deriv{k-1}\) est de classe \(\classe{1}\) sur \(I\).
\end{itemize}
\end{prop}

Les ensembles \(\ensclasse{k}{I}{E}\) forment donc une chaîne d'inclusions : \[\ensclasse{\infty}{I}{E}\subset\dots\subset\ensclasse{k+1}{I}{E}\subset\ensclasse{k}{I}{E}\subset\dots\subset\ensclasse{1}{I}{E}\subset\ensclasse{0}{I}{E}.\]

\subsection{Théorèmes opératoires pour les dérivées successives}

\begin{prop}
Soient \(k\in\N\union\accol{\infty}\), \(f,g:I\to E\) deux fonctions de classe \(\classe{k}\) sur \(I\) et \(\lambda\in\K\).

On a :

\begin{itemize}
    \item \(f+g\) est de classe \(\classe{k}\) sur \(I\) et \(\paren{f+g}\deriv{k}=f\deriv{k}+g\deriv{k}\) \\
    \item \(\lambda f\) est de classe \(\classe{k}\) sur \(I\) et \(\paren{\lambda f}\deriv{k}=\lambda f\deriv{k}\).
\end{itemize}
\end{prop}

\begin{prop}
Soient \(k\in\N\union\accol{\infty}\), \(\phi:I\to\R\) et \(f:J\to E\).

Si \(\phi\paren{I}\subset J\) et \(\phi\) et \(f\) sont de classe \(\classe{k}\) sur \(I\) et \(J\) respectivement, alors \(f\rond\phi\) est de classe \(\classe{k}\) sur \(I\).
\end{prop}

\begin{prop}
Soient \(k\in\N\union\accol{\infty}\), \(f:I\to E\) et \(L\in\L{E}{F}\).

Si \(f\) est une fonction de classe \(\classe{k}\) sur \(I\), alors \(L\rond f\) l'est aussi et \(\paren{L\rond f}\deriv{k}=L\rond f\deriv{k}\).
\end{prop}

\begin{prop}
Soient \(k\in\N\union\accol{\infty}\), \(f,g:I\to E\) et \(B:E^2\to F\) bilinéaire.

Si \(f\) et \(g\) sont de classe \(\classe{k}\) sur \(I\), alors \(B\paren{f,g}\) l'est aussi et, d'après la formule de Leibniz : \[\paren{B\paren{f,g}}\deriv{k}=\sum_{i=0}^k\binom{i}{k}B\paren{f\deriv{i},g\deriv{k-i}}.\]
\end{prop}

\section{Intégrales}

\subsection{Définition}

\begin{defi}
Soit \(f:\intervii{a}{b}\to E\) une fonction définie sur le segment \(\intervii{a}{b}\).

On dit que \(f\) est continue par morceaux sur \(\intervii{a}{b}\) quand il existe une subdivision \(\paren{c_0,\dots,c_n}\) de \(\intervii{a}{b}\) telle que :

\begin{itemize}
    \item \(\quantifs{\tpt i\in\interventierii{0}{n-1}}f\) est continue sur l'intervalle ouvert \(\intervee{c_i}{c_{i+1}}\) \\
    \item \(f\) a une limite réelle en \(a\) à droite, en \(b\) à gauche et des limites réelles à gauche et à droite en chaque point \(c_i\) tel que \(1\leq i\leq n-1\).
\end{itemize}

Toute subdivision qui convient dans cette définition est dite adaptée à \(f\).
\end{defi}

On choisit une base \(\fami{B}=\paren{e_1,\dots,e_n}\) de \(E\). Il est alors évident qu'une fonction à valeurs dans \(E\) est continue par morceaux ssi ses fonctions coordonnées dans la base \(\fami{B}\) le sont aussi.

\begin{defi}
Soit \(f:\intervii{a}{b}\to E\) continue par morceaux sur \(\intervii{a}{b}\).

On note \(f_1,\dots,f_n\) ses fonctions coordonnées dans la base \(\fami{B}\), \ie \(f=\sum_{i=1}^nf_ie_i\).

On pose alors \[\int_{\intervii{a}{b}}f=\sum_{i=1}^n\paren{\int_{\intervii{a}{b}}f_i}e_i.\]
\end{defi}

Bien sûr, les notations classiques pour les intégrales sont conservées : \(\int_a^bf\), \(\int_a^bf\paren{t}\odif{t}\), etc.

On pose encore \(\int_b^af=-\int_a^bf\).

Cette définition est \textit{a priori} ambiguë car elle dépend de la base \(\fami{B}\) choisie. On montre aisément qu'en fait ce n'est pas le cas : on obtient toujours le même vecteur intégrale, indépendamment de la base choisie.

\subsection{Propriétés}

En se ramenant aux coordonnées dans une base, on retrouve les propriétés essentielles de l'intégrale.

\begin{prop}
L'application \(\fonctionlambda{\classe{0}_m\paren{\intervii{a}{b}}}{E}{f}{\int_a^bf}\) est linéaire.
\end{prop}

\begin{prop}
La relation de Chasles reste valable : pour tout \(\paren{a,b,c}\in I^3\), si \(f\) est continue par morceaux sur \(I\), alors \(\int_a^bf=\int_a^cf+\int_c^bf\).
\end{prop}

On retrouve une inégalité triangulaire avec la norme (n'importe laquelle !).

\begin{prop}
Soit \(f:\intervii{a}{b}\to E\) continue par morceaux sur \(\intervii{a}{b}\).

La fonction \(t\mapsto\norme{f\paren{t}}\) est continue par morceaux sur \(\intervii{a}{b}\) et à valeurs réelles.

De plus, on a \[\norme{\int_a^bf}\leq\int_a^b\norme{f}.\]
\end{prop}

On retrouve la notion de sommes de Riemann.

\begin{prop}
Soit \(f:\intervii{a}{b}\to E\) continue par morceaux.

Pour tout \(n\in\Ns\), on introduit les sommes de Riemann associées à la fonction \(f\) sur \(\intervii{a}{b}\) : pour \(k\in\interventierii{0}{n}\), on pose \(c_k=a+k\dfrac{b-a}{n}\) et \[S_n=\dfrac{b-a}{n}\sum_{k=0}^{n-1}f\paren{c_k}\qquad S_n\prim=\dfrac{b-a}{n}\sum_{k=1}^nf\paren{c_k}\qquad S_n\seconde=\dfrac{b-a}{n}\sum_{k=0}^nf\paren{c_k}.\]

Les suites \(\paren{S_n}\), \(\paren{S_n\prim}\) et \(\paren{S_n\seconde}\) tendent toutes les trois vers \(\int_{\intervii{a}{b}}f\paren{x}\odif{x}\).
\end{prop}

Enfin, une petite nouveauté.

\begin{prop}
Soient \(f:\intervii{a}{b}\to E\) continue par morceaux et \(L\in\L{E}{F}\).

La fonction \(L\rond f=L\paren{f}\) est continue par morceaux et \(\int_a^bL\paren{f}=L\paren{\int_a^bf}\).
\end{prop}

\subsection{Primitives d'une fonction continue}

\begin{prop}
Soient \(f:I\to E\) continue et \(a\in I\). On pose \(\Phi:x\mapsto\int_a^xf\paren{t}\odif{t}\).

\(\Phi\) est l'unique primitive de \(f\) sur \(I\) qui s'annule en \(a\).
\end{prop}

\begin{rem}
On a donc montré que si \(f\) est continue sur \(I\), alors la fonction \(\Phi:x\mapsto\int_a^xf\) est de classe \(\classe{1}\) sur \(I\), car \(\quantifs{\tpt x\in I}\Phi\prim\paren{x}=f\paren{x}\), et non pas \(\Phi\prim\paren{x}=f\paren{x}-f\paren{a}\) !
\end{rem}

\begin{cor}
Toute fonction continue sur un intervalle y admet des primitives.
\end{cor}

On en déduit l'inégalité des accroissements finis pour une fonction de classe \(\classe{1}\) (dérivable ne suffit pas).

\begin{prop}
Soit \(f:I\to E\) une fonction de classe \(\classe{1}\) sur \(I\).

\(\quantifs{\Tpt\paren{a,b}\in I^2}\norme{f\paren{b}-f\paren{a}}\leq\abs{b-a}\sup_{\intervii{a}{b}}\norme{f\prim}\).
\end{prop}

\subsection{Formules de Taylor}

On retrouve encore par utilisation des fonctions coordonnées les formules de Taylor usuelles.

La formule de Taylor avec reste intégral.

\begin{prop}
Soient \(f\) une fonction de classe \(\classe{n+1}\) sur un intervalle \(I\) et \(a,b\in I\).

On a \[f\paren{b}=f\paren{a}+f\prim\paren{a}\paren{b-a}+\dfrac{f\seconde\paren{a}}{2}\paren{b-a}^2+\dots+\dfrac{f\deriv{n}\paren{a}}{n!}\paren{b-a}^n+\int_a^b\dfrac{\paren{b-t}^n}{n!}f\deriv{n+1}\paren{t}\odif{t}.\]
\end{prop}

L'inégalité de Taylor-Lagrange qui s'en déduit.

\begin{prop}
Soient \(f\) une fonction de classe \(\classe{n+1}\) sur un intervalle \(I\) et \(a,b\in I\).

On a \[\norme{f\paren{b}-f\paren{a}-f\prim\paren{a}\paren{b-a}-\dfrac{f\seconde\paren{a}}{2}\paren{b-a}^2-\dots-\dfrac{f\deriv{n}\paren{a}}{n!}\paren{b-a}^n}\leq\dfrac{\abs{b-a}^{n+1}}{\paren{n+1}!}\sup_{\intervii{a}{b}}\norme{f\deriv{n+1}}.\]
\end{prop}

Et la formule de Taylor-Young.

\begin{prop}
Soit \(f:I\to E\) de classe \(\classe{n}\) sur \(I\).

Pour tout \(t_0\in I\), \(f\) possède un développement limité en \(t_0\) à l'ordre \(n\) : il existe une fonction \(\epsilon\) définie sur \(I\) et à valeurs dans \(E\) telle que \[\begin{dcases}
\quantifs{\tpt t\in I}f\paren{t}=f\paren{t_0}+f\prim\paren{t_0}\paren{t-t_0}+\dfrac{f\seconde\paren{t_0}}{2}\paren{t-t_0}^2+\dots+\dfrac{f\deriv{n}\paren{t_0}}{n!}\paren{t-t_0}^n+\paren{t-t_0}^n\epsilon\paren{t} \\
\epsilon\paren{t}\tendqd{t\to t_0}0
\end{dcases}\]
\end{prop}


\part{Exercices}

\setcounter{chapter}{0}

\chapter{Espaces vectoriels normés}

\legendeexercices

\begin{exopss}[Exercice 1]
Soit \(u\) une suite réelle bornée. Pour \(n\in\N\), on pose \(U_n=\accol{u_p\tq p\geq n}\), \(a_n=\inf U_n\) et \(b_n=\sup U_n\).

\begin{enumerate}
    \item Justifiez l'existence des suites \(a\) et \(b\) et étudiez leur monotonie, ainsi que leur convergence. \\
    \item Montrez que \(u\) converge ssi \(a\) et \(b\) ont la même limite.
\end{enumerate}

Note culturelle : la limite de \(a\) s'appelle la limite inférieure de \(u\), notée \(\underline{\lim}\,u\) et celle de \(b\) est la limite supérieure, notée \(\overline{\lim}\,u\).
\end{exopss}

\begin{exops}[Exercice 2]\label{exo:1.2}
Montrez que les applications \(N\) introduites ci-dessous sont des normes :

\begin{enumerate}
    \item si \(A\in\GL{n}[\R]\), l'application \(N\) définie sur \(\R^n\) par \(N\paren{X}=\norme{AX}_2\) ; \\
    \item sur \(E=\ensclasse{1}{\intervii{a}{b}}{\K}\), \(N\paren{f}=\abs{f\paren{a}}+\int_a^b\abs{f\prim\paren{t}}\odif{t}\) ; \\
    \item pour des réels \(\alpha_0<\dots<\alpha_n\) fixés, l'application \(N\) définie sur \(\polydeg[\R]{n}\) par \(N\paren{P}=\max_{k\in\interventierii{0}{n}}\abs{P\paren{\alpha_k}}\).
\end{enumerate}
\end{exops}

\begin{exos}[Exercice 3]
Sur \(E=\R^2\), on définit \[\norme{\paren{x,y}}=\max\paren{\abs{x},\abs{x+2y}}.\]

Démontrez que \(\norme{}\) définit une norme sur \(\R^2\). Représentez graphiquement la boule-unité.
\end{exos}

\begin{exopss}[Exercice 4]
Les normes définies dans l'\hyperref[exo:1.2]{exercice 2} sont-elles équivalentes à la norme \(\norme{}_\infty\), la norme \(\norme{}_1\), la norme \(\norme{}_2\) ?
\end{exopss}

\begin{exops}[Exercice 5]
Soit \(\theta\not\equiv0\croch{\pi}\). Montrez que \(\paren{\sin\paren{n\theta}}\) et \(\paren{\cos\paren{n\theta}}\) divergent.
\end{exops}

\begin{exops}[Exercice 6]
Soit \(A\in\M{n}\). On suppose que la suite \(\paren{A^k}\) converge. Montrez que sa limite est la matrice d'un projecteur.
\end{exops}



\begin{exoss}[Exercice 7]
Cet exercice prolonge le premier.

Soit \(u\) une suite vérifiant la propriété \[\quantifs{\forall\epsilon>0;\exists N\in\N;\forall\paren{n,p}\in\N^2}p\geq n\geq N\imp\abs{u_n-u_p}\leq\epsilon.\]

Montrez que \(u\) est bornée, puis en vous servant des suites \(a\) et \(b\) définies comme ci-dessus, montrez que \(u\) converge.

Note culturelle : on dit que \(u\) est une suite de Cauchy et on a donc montré que toute suite de Cauchy converge.
\end{exoss}



\begin{exopss}[Exercice 8]
On note \(l_1\) l'ensemble des suites réelles \(\paren{u_n}\) telles que la série \(\sum u_k\) soit absolument convergente, et on pose : \[\quantifs{\forall u=\paren{u_k}\in l_1}N\paren{u}=\sum_{k=0}^{\pinf}\abs{u_k}.\]

\begin{enumerate}
    \item Justifiez que \(l_1\) est un \(\R\)-espace vectoriel. \\
    \item Montrez que \(N\) est une norme sur \(l_1\). On notera désormais \(\norme{u}_1\) pour \(N\paren{u}\). \\
    \item Justifiez que, si \(\paren{u\deriv{n}}_{n\in\N}\) est une suite d'éléments de \(l_1\) convergeant vers la suite \(a\in l_1\) pour la norme 1, alors \[\quantifs{\forall k\in\N}u_k\deriv{n}\tendqd{n\to\pinf}a_k.\]
    \item Montrez que la réciproque est fausse. \\ \textit{Indication :} on pourra étudier l'exemple où, \(\quantifs{\tpt\paren{n,k}\in\N^2}u_k\deriv{n}=\exp\paren{\dfrac{-k}{n+1}}\).
\end{enumerate}
\end{exopss}



\begin{exops}[Exercice 9]
Soit \(E\) le \(\C\)-espace vectoriel des suites bornées muni de la norme infinie.

Montrez que les applications \(u\mapsto\paren{u_{n+1}-u_n}\) et \(u\mapsto\dfrac{u_0+\dots+u_n}{n+1}\) sont des applications continues de \(E\) dans \(E\).
\end{exops}



\begin{exopss}[Exercice 10]
Soit \(E=\M{n}\). Pour \(A=\paren{a_{ij}}\) on pose \(\norme{A}=\max_{1\leq i\leq n}\sum_{j=1}^n\abs{a_{ij}}\).

\begin{enumerate}
    \item Montrez que \(\norme{}\) est une norme sur \(E\). \\
    \item Montrez que l'application \(A\mapsto\trans{A}\) est un endomorphisme continu et déterminez sa norme subordonnée.
\end{enumerate}
\end{exopss}



\begin{exoss}[Exercice 11]
Soit \(E=\ensclasse{0}{\intervii{0}{1}}{\R}\). On pose \(A=\accol{f\in E\tq f\geq0}\) et pour \(f\in E\), \(\norme{f}_1=\int_0^1\abs{f}\).

\begin{enumerate}
    \item Montrez que \(\norme{}_1\) est une norme sur \(E\). \\
    \item Déterminez \(\interieur{A}\) dans \(\groupe{E}[\norme{}_\infty]\), puis dans \(\groupe{E}[\norme{}_1]\). \\
    \item On pose \(D=\fami{D}^1\paren{\intervii{0}{1},\R}\) le sous-espace des fonctions dérivables et \(P\) le sous-espace des fonctions polynômes. \\ Déterminez les intérieurs de \(P\) et \(D\) dans \(\groupe{E}[\norme{}_\infty]\).
\end{enumerate}
\end{exoss}



\begin{exoss}[Exercice 12]
Cet exercice prolonge le précédent, les notations sont reprises.

Soit \(u:E\to E\) qui à toute fonction \(f\) de \(E\) associe sa primitive qui s'annule en \(0\). Vérifiez que \(u\) est un endomorphisme de \(E\).

Est-il continu de \(\groupe{E}[\norme{}_?]\) dans \(\groupe{E}[\norme{}_?]\) (vous étudierez les quatre possibilités) ? Quand c'est le cas, déterminez la norme subordonnée de \(u\).
\end{exoss}



\begin{exopss}[Exercice 13]
Soit \(E=\accol{f\in\ensclasse{1}{\intervii{0}{1}}{\R}\tq f\paren{0}=f\paren{1}=0}\). Pour \(f\in E\), on pose \(\norme{f}=\sup_{\intervii{0}{1}}\abs{f\prim}\).

\begin{enumerate}
    \item Montrez que \(\norme{}\) est une norme sur \(E\). \\
    \item Soit \(\phi:E\to\R\) définie par \(\phi\paren{f}=\int_0^1f\). Montrez que \(\phi\) est continue et déterminez \(\normesub{f}\).
\end{enumerate}
\end{exopss}



\begin{exopss}[Exercice 14]
Soient \(E\) un espace vectoriel normé et \(F\) un sous-espace vectoriel de \(E\).

\begin{enumerate}
    \item Montrez que si \(F\not=E\), alors \(\interieur{F}=\ensvide\) et \(\conj{F}\) est un sous-espace vectoriel de \(E\). \\
    \item Montrez que si \(F\) est un hyperplan, alors \(F\) est fermé ou dense dans \(E\).
\end{enumerate}
\end{exopss}



\begin{exops}[Exercice 15]
Soient \(E\) un espace vectoriel normé et \(A\) une partie de \(E\).

Montrez que \(\interieur{A}\) est le plus grand ouvert inclus dans \(A\) et \(\conj{A}\) est le plus petit fermé contenant \(A\). Montrez que la frontière de \(A\) est un fermé d'intérieur vide.
\end{exops}



\begin{exos}[Exercice 16]
Une intersection d'ouverts est-elle toujours un ouvert ? Une réunion de fermés est-elle toujours un fermé ?
\end{exos}



\begin{exops}[Exercice 17]
Montrez que si \(A\) est une partie convexe d'un espace vectoriel normé \(E\), alors il en est de même pour \(\conj{A}\) et \(\interieur{A}\).
\end{exops}



\begin{exoss}[Exercice 18]
Soient \(E\) un espace vectoriel normé, \(A\) une partie de \(E\) et \(x\in E\). On dit que \(x\) est un point d'accumulation de \(A\) quand il existe une suite injective de \(A^\N\) qui converge vers \(x\). On dit que \(x\) est un point isolé de \(A\) quand il existe \(r>0\) tel que \(\bouleo{x}{r}\inter A=\accol{x}\).

\begin{enumerate}
    \item Exemples. On pose \(A=\accol{\dfrac{1}{n}\tq n\in\Ns}\) dans \(\R\) : montrez que tous les points de \(A\) sont isolés, que le seul point d'accumulation de \(A\) est \(0\) et que \(A\) n'est pas fermé. On pose \(B=\accol{\dfrac{1}{n}+\dfrac{1}{p}\tq\paren{n,p}\in\paren{\Ns}^2}\) : quels sont les points d'accumulation de \(B\) ? \\
    \item Montrez que \(x\) est un point d'accumulation ssi \(\quantifs{\tpt r>0}\bouleo{x}{r}\inter A\text{ est un ensemble infini}\). \\
    \item On note \(A\prim\) l'ensemble des points d'accumulation de \(A\) et \(A^d\) l'ensemble des points isolés dans \(A\). Montrez que \(\conj{A}=A\prim\sqcup A^d\). \\
    \item Montrez que \(A\prim\) est un fermé.
\end{enumerate}
\end{exoss}



\begin{exoss}[Exercice 19]
Soient \(E,F\) deux espaces vectoriels normés et \(f:E\to F\). Montrez l'équivalence entre les propositions :

\begin{enumerate}
    \item \(f\) est continue \\
    \item \(\quantifs{\forall A\in\P{E}}f\paren{\conj{A}}\subset\conj{f\paren{A}}\) \\
    \item \(\quantifs{\forall B\in\P{F}}f\inv\paren{\interieur{B}}\subset\interieur{\overbrace{f\inv\paren{B}}}\)
\end{enumerate}
\end{exoss}



\begin{exopss}[Exercice 20]
Soient \(A,B\) deux fermés disjoints d'un espace vectoriel normé \(E\).

\begin{enumerate}
    \item Montrez que \(\accol{x\in E\tq d\paren{x,A}>d\paren{x,B}}\) est un ouvert. \\
    \item Montrez qu'il existe deux ouverts disjoints \(U\) et \(V\) tels que \(A\subset U\) et \(B\subset V\).
\end{enumerate}
\end{exopss}



\begin{exopss}[Exercice 21]
Soit \(A\) une partie d'un espace vectoriel normé \(E\). Pour \(r>0\), on pose \(V\paren{A,r}=\accol{x\in E\tq d\paren{x,A}<r}\).

Montrez que \(V\paren{A,r}\) est un ouvert de \(E\) et \(\biginter_{r>0}V\paren{A,r}=\conj{A}\).
\end{exopss}



\begin{exopss}[Exercice 22]
Soient \(E\) un espace vectoriel normé, \(K\) un compact de \(E\), \(k\in\intervie{0}{1}\) et \(f:K\to K\) telle que \[\quantifs{\forall\paren{x,y}\in K^2}\norme{f\paren{x}-f\paren{y}}\leq k\norme{x-y}.\]

Soit \(u\) la suite définie par \(u_0\) quelconque dans \(K\) et \(u_{n+1}=f\paren{u_n}\).

Montrez que \(u\) converge et que sa limite est l'unique point fixe de \(f\).
\end{exopss}



\begin{exoss}[Exercice 23]
Soient \(E\) un espace vectoriel normé et \(A\) une partie non-vide de \(E\). On appelle diamètre de \(A\), noté \(\delta\paren{A}\), la borne supérieure dans \(\Rp\union\accol{\pinf}\) des \(\norme{x-y}\) quand \(\paren{x,y}\in A^2\).

\begin{enumerate}
    \item Montrez que \(\delta\paren{A}<\pinf\) ssi \(A\) est bornée. \\
    \item Quel est le diamètre d'une boule ? \\
    \item Montrez que si \(A\) est compacte, alors il existe \(\paren{a,b}\in A^2\) tel que \(\delta\paren{A}=\norme{a-b}\). Est-ce encore vrai si on suppose seulement \(A\) bornée ? \(A\) fermée ?
\end{enumerate}
\end{exoss}



\begin{exopss}[Exercice 24]
Soient \(E\) un espace vectoriel normé et \(A,B\) deux parties non-vides de \(E\). On pose \(d\paren{A,B}=\inf_{\paren{a,b}\in A\times B}\norme{a-b}\), appelé distance de \(A\) à \(B\).

\begin{enumerate}
    \item Montrez que si \(d\paren{A,B}>0\), alors \(A\) et \(B\) sont disjointes, mais que la réciproque est fausse. \\
    \item Montrez que si \(A\) et \(B\) sont compactes, alors \(d\paren{A,B}\) est en fait un minimum plutôt qu'une borne inférieure. \\
    \item Montrez que ce résultat reste vrai si \(E\) est de dimension finie, l'une des deux parties est compacte et l'autre fermée. \\
    \item Est-ce encore vrai si on suppose seulement \(A\) et \(B\) fermées ?
\end{enumerate}
\end{exopss}



\begin{exopss}[Exercice 25]
Soient \(E\) un espace vectoriel normé de dimension finie et \(\paren{B_n=\boulef{a_n}{r_n}}\) une suite de boules fermées, décroissantes pour l'inclusion, telle que \(r_n\to0\).

\begin{enumerate}
    \item Montrez que la suite \(\paren{a_n}\) admet une sous-suite convergeant vers un vecteur \(a\). \\
    \item Montrez que \(a_n\to a\). \\
    \item Montrez que \(\biginter_{n\in\N}B_n=\accol{a}\).
\end{enumerate}
\end{exopss}



\begin{exoss}[Exercice 26]
Soit \(E\) un \(\K\)-espace vectoriel de dimension finie. Montrez que l'ensemble des projecteurs est fermé dans \(\Lendo{E}\). Est-il borné ? Compact ? Connexe par arcs ?
\end{exoss}



\begin{exoss}[Exercice 27]
Soient \(f:\R\to\R\) continue et \(\epsilon>0\). On pose \(E=\accol{\paren{x,y}\in\R^2\tq\abs{f\paren{x}-y}\leq\epsilon}\).

\begin{enumerate}
    \item Montrez que \(E\) est connexe par arcs. \\
    \item Montrez que si \(f\) est une fonction affine, alors \(E\) est une partie convexe. \\
    \item Montrez que la réciproque est vraie.
\end{enumerate}
\end{exoss}



\begin{exoss}[Exercice 28]
Soient \(E\) un \(\R\)-espace vectoriel de dimension finie et \(H\) un hyperplan de \(E\).

\begin{enumerate}
    \item Montrez que \(E\excluant H\) possède deux composantes connexes par arcs qui sont ouvertes. \\
    \item Soit \(B\) une partie de \(H\) telle que \(H\not=B\). Montrez que \(E\excluant B\) est connexe par arcs.
\end{enumerate}
\end{exoss}



\begin{exoss}[Exercice 29]
Soient \(E\) un espace vectoriel normé et \(A,B\) deux parties de \(E\) telles que \(B\) est connexe par arcs et \(B\) rencontre à la fois \(A\) et \(E\excluant A\).

Montrez que \(B\) rencontre la frontière de \(A\).
\end{exoss}



\begin{exoss}[Exercice 30]
Deux parties d'un espace vectoriel normé sont dites homéomorphes quand il existe une bijection continue de l'une dans l'autre telle que la réciproque soit aussi continue.

\begin{enumerate}
    \item Montrez que tout intervalle ouvert est homéomorphe à \(\R\). \\
    \item Montrez qu'un intervalle qui contient l'une de ses bornes réelles ne peut pas être homéomorphe à \(\R\). \\
    \item Montrez que toute boule ouverte d'un espace vectoriel normé \(E\) est homéomorphe à \(E\). \\
    \item Montrez qu'aucune boule fermée de \(E\) n'est homéomorphe à \(E\).
\end{enumerate}
\end{exoss}



\begin{exosss}[Exercice 31]
Soit \(G\) un sous-groupe de \(\groupe{\R}\), autre que \(\accol{0}\).

\begin{enumerate}
    \item Montrez que \(a=\inf G\inter\Rps\) existe. \\
    \item Montrez que \(G=a\Z\) si \(a>0\) ou \(G\) est dense dans \(\R\) si \(a=0\). \\
    \item On pose \(G=\Z+\sqrt{2}\Z\) (\(G\) est le sous-groupe de \(\groupe{\R}\) engendré par \(1\) et \(\sqrt{2}\)) et \(r=\sqrt{2}-1\). En considérant la suite \(\paren{r^n}\), montrez que \(G\) est dense dans \(\R\). \\ Soit \(f:\R\to\R\) continue et périodique de périodes \(1\) et \(\sqrt{2}\). Que peut-on dire de \(f\) ? \\
    \item Soient \(a,b\) deux réels distincts et non-nuls, on pose \(G=a\Z+b\Z\). Montrez que \(G\) est un sous-groupe de \(\groupe{\R}\), puis que \(G\) est dense dans \(\R\) ssi \(\dfrac{a}{b}\) est un rationnel. \\ Application : montrez que les ensembles \(\accol{\cos n\tq n\in\N}\) et \(\accol{\sin n\tq n\in\N}\) sont denses dans \(\intervii{-1}{1}\).
\end{enumerate}
\end{exosss}



\begin{exosss}[Exercice 32]
Soit \(E=\ensclasse{0}{\intervii{0}{1}}{\R}\) muni de la norme infinie \(\norme{f}=\sup_{\intervii{0}{1}}\abs{f}\). On note \(\boulef{0}{1}\) la boule-unité fermée.

Soit \(\paren{t_n}\) une suite injective à valeurs dans \(\intervii{0}{1}\). Pour \(f\in E\), on pose \(L\paren{f}=\sum_{n=1}^{\pinf}\paren{-1}^n\dfrac{f\paren{t_n}}{2^n}\).

\begin{enumerate}
    \item Montrez que \(L\) est une forme linéaire continue sur \(E\). \\
    \item Déterminez \(K=\sup_{f\in\boulef{0}{1}}\abs{L\paren{f}}\). \\
    \item Montrez que si la suite \(\paren{t_n}\) converge ou si elle est dense dans \(\intervii{0}{1}\), \(K\) n'est pas atteinte. \\
    \item Donnez un exemple de suite \(\paren{t_n}\) pour laquelle \(K\) est atteinte. Donnez une condition nécessaire et suffisante sur la suite \(t\) pour que \(K\) soit atteinte.
\end{enumerate}
\end{exosss}



\begin{exosss}[Exercice 33]
Soient \(E\) un espace vectoriel normé et \(u\) une forme linéaire non-nulle et continue sur \(E\). On pose \(H=\ker u\) et \(K=\sup_{x\in E\excluant\accol{0}}\dfrac{\abs{u\paren{x}}}{\norme{x}}\).

\begin{enumerate}
    \item Justifiez l'existence de \(K\). \\
    \item Montrez que \(\quantifs{\tpt a\in E}d\paren{a,H}=\dfrac{\abs{u\paren{a}}}{K}\).
\end{enumerate}
\end{exosss}



\begin{exo}[Oral CCMP, 1]
Soient \(E\) un espace vectoriel normé réel et \(B\) sa boule-unité ouverte. Montrez que \(E\) et \(B\) sont homéomorphes (\ie il existe une bijection de \(E\) dans \(B\) qui est continue et dont la réciproque est aussi continue).
\end{exo}



\begin{exo}[Oral CCMP, 2]
Soient \(E\) un espace vectoriel normé réel et \(C,D\) deux parties de \(E\) telles que \(C\subset D\subset\conj{C}\) et \(C\) convexe. Montrez que \(D\) est connexe par arcs.
\end{exo}



\begin{exo}[Oral CCMP, 3]
Soient \(E\) un espace vectoriel normé réel, \(K\) un compact de \(E\) et \(f:K\to K\) telle que \[\quantifs{\forall\paren{x,y}\in K^2}x\not=y\imp\norme{f\paren{x}-f\paren{y}}<\norme{x-y}.\]

\begin{enumerate}
    \item Montrez que \(f\) possède un unique point fixe. \\
    \item Soit \(u\) la suite définie par \(u_0\) quelconque dans \(K\) et \(u_{n+1}=f\paren{u_n}\). Montrez que \(u\) converge vers le point fixe de \(f\).
\end{enumerate}
\end{exo}



\begin{exo}[Oral CCMP, 4]
Soit \(u\) une suite réelle bornée telle que \(u_{n+1}-u_n\tendqd{n\to\pinf}0\). Montrez que l'ensemble de ses valeurs d'adhérence est un intervalle.
\end{exo}



\begin{exo}[Oral Centrale, 5]
Soit \(G\) un sous-groupe de \(\groupe{\Cs}[\times]\) tel que pour tout \(g\in G\) il existe un voisinage \(V\) de \(g\) tel que \(G\inter V=\accol{g}\).

\begin{enumerate}
    \item Montrez que pour tout compact \(K\) de \(\Cs\), \(G\inter K\) est fini. \\
    \item Montrez que \(G\inter\U\) est un groupe cyclique. \\
    \item On suppose que \(G\) n'est pas contenu dans \(\U\). Soit \(A=\accol{\abs{x}\tq x\in G\text{ et }\abs{x}>1}\). Montrez que \(A\) possède un plus petit élément. Déduisez-en \(G\).
\end{enumerate}
\end{exo}



\begin{exo}[Oral Centrale, 6]
Soient \(E,F\) deux espaces vectoriels normés de dimensions finies et \(f:E\to F\). On dit que \(f\) est propre quand pour tout compact \(K\) de \(F\), \(f\inv\paren{K}\) est un compact de \(E\).

\begin{enumerate}
    \item Montrez que si \(f\) est propre, alors l'image d'un fermé de \(E\) est un fermé de \(F\). \\
    \item Montrez que \(f\) est propre ssi \(\norme{f\paren{x}}\tendqd{\norme{x}\to\pinf}\pinf\).
\end{enumerate}
\end{exo}



\begin{exo}[Oral X, 7]
Soient \(n\in\Ns\), \(U_n\) l'ensemble des polynômes réels unitaires de degré \(n\) et \(A_n\) l'ensemble des polynômes de \(U_n\) qui sont simplement scindés (\ie ayant \(n\) racines réelles distinctes).

\begin{enumerate}
    \item Montrez que \(A_n\) est un ouvert de \(U_n\). \\
    \item Déterminez l'adhérence de \(A_n\).
\end{enumerate}
\end{exo}



\chapter{Séries numériques et vectorielles : révisions et compléments}

\legendeexercices

\begin{exos}[Exercice 1]
\begin{enumerate}
    \item Montrez que la série de terme général \(u_n=\Arctan\dfrac{1}{2n^2}\) est convergente. \\
    \item Montrez que \(\quantifs{\tpt n\geq1}u_n=\Arctan\dfrac{n}{n+1}-\Arctan\dfrac{n-1}{n}\). \\
    \item Déduisez-en la somme \(\sum_{n=1}^{\pinf}u_n\).
\end{enumerate}
\end{exos}

\begin{corr}
\note{À venir}
\end{corr}

\begin{exos}[Exercice 2]~\\
Justifiez que la série \(\sum_{n\geq2}\dfrac{2n-1}{n^3-n}\) converge et déterminez sa somme (indication : décomposition en éléments simples).
\end{exos}

\begin{corr}
\note{À venir}
\end{corr}

\begin{exoss}[Exercice 3]
Donnez la nature des séries suivantes (\(\alpha\) désigne une constante strictement positive, \(x\) un réel dans \(\intervee{-1}{1}\)) :

\begin{enumerate}
    \item \(\sum_{n\geq2}\dfrac{\ln^nn}{n^{\ln n}}\) \\
    \item \(\sum_{n\geq2}\dfrac{1}{\paren{\ln n}^{\ln n}}\) \\
    \item \(\sum\dfrac{\paren{-1}^n}{n^\alpha}\) \\
    \item \(\sum_{n\geq1}\dfrac{\paren{-1}^nn^\alpha}{n!}\) \\
    \item \(\sum\dfrac{\paren{-1}^n\ln n}{n^\alpha}\) \\
    \item \(\sum\ln\paren{1+x^n}\) \\
    \item \(\sum\dfrac{\sin n}{2^n}\) \\
    \item \(\sum2\ln\paren{n^3+1}-3\ln\paren{n^2+1}\) \\
    \item \(\sum\dfrac{\sin n\sin\frac{1}{n}}{n}\) \\
    \item \(\sum_{n\geq2}\dfrac{\paren{-1}^n}{\paren{-1}^n+n}\) \\
    \item \(\sum_{n\geq0}\sin\paren{\pi\sqrt{n^2+1}}\) \\
    \item \(\sum\sqrt[n+1]{n+1}-\sqrt[n]{n}\) \\
    \item \(\sum_{n\geq1}\dfrac{\paren{-1}^n}{\ln n+\sin\frac{2\pi n}{3}}\) \\
    \item \(\sum_{n\geq1}\paren{\paren{1+\dfrac{1}{n}}^n-\e{}}\) \\
    \item \(\sum_{n\geq1}\dfrac{1}{\ln n\ln\paren{\ch n}}\) \\
    \item \(\sum\int_0^{\nicefrac{\pi}{2}}\dfrac{\cos^2t}{n^2+\cos^2t}\odif{t}\)
\end{enumerate}
\end{exoss}

\begin{corr}
\note{À venir}
\end{corr}

\begin{exoss}[Exercice 4]
Soit \(\paren{a,b}\in\R^2\).

\begin{enumerate}
    \item Déterminez \(a\) et \(b\) pour que la série de terme général \(\ln n+a\ln\paren{n+1}+b\ln\paren{n+2}\) converge. Dans ce cas, donnez la valeur de sa somme. \\
    \item Faites de même avec la série de terme général \(\sqrt{n}+a\sqrt{n+1}+b\sqrt{n+2}\).
\end{enumerate}
\end{exoss}

\begin{corr}
\note{À venir}
\end{corr}

\begin{exoss}[Exercice 5]
Pour quelles valeurs de \(\alpha>0\) la série de terme général \(u_n=\ch^\alpha n-\sh^\alpha n\) converge-t-elle ? Dans ce cas, donnez un équivalent de \(\sum_{k=n}^{\pinf}u_k\).
\end{exoss}

\begin{corr}
\note{À venir}
\end{corr}

\begin{exoss}[Exercice 6]~\\
On pose \(u_n=\dfrac{1}{n^\alpha}\sum_{k=1}^nk^{\nicefrac{3}{2}}\). Pour quelles valeurs de \(\alpha>0\) la série \(\sum_{n\geq1}u_n\) converge-t-elle ? Dans ce cas, donnez un équivalent de \(\sum_{k=n+1}^{\pinf}u_k\).
\end{exoss}

\begin{corr}
\note{À venir}
\end{corr}

\begin{exoss}[Exercice 7, séries associées à des suites définies par récurrence]~\\
\begin{enumerate}
    \item Soit \(u\) la suite définie par récurrence par \(u_1\in\R\) et \(\quantifs{\tpt n\geq1}u_{n+1}=\dfrac{1}{n\e{u_n}}\). \\ Quelle est la nature de la série \(\sum_{n\geq1}u_n\) ? \\
    \item Soit \(u\) la suite définie par récurrence par \(u_0\in\Rps\) et \(\quantifs{\tpt n\geq0}u_{n+1}=\ln\paren{1+u_n}\). \\ Quelle est la nature de la série \(\sum\paren{\dfrac{1}{u_{n+1}}-\dfrac{1}{u_n}}\) ? Puis celle de \(\sum u_n\) ? Donnez un équivalent de \(\sum_{k=0}^nu_k\). \\
    \item Soit \(u\) la suite définie par récurrence par \(u_0\in\intervee{0}{\pi}\) et \(\quantifs{\tpt n\geq0}u_{n+1}=\sin u_n\). \\ Quelle est la nature de la série \(\sum\paren{\dfrac{1}{u_{n+1}^2}-\dfrac{1}{u_n^2}}\) ? Puis celle de \(\sum u_n\) ? Donnez un équivalent de \(\sum_{k=0}^n\).
\end{enumerate}
\end{exoss}

\begin{corr}
\note{À venir}
\end{corr}

\begin{exoss}[Exercice 8]
Soit \(P\in\poly[\R]\) tel que \(\quantifs{\tpt x\in\Rp}P\paren{x}\geq0\). Pour \(n\in\Ns\), on pose \(u_n=\sum_{k=n}^{\pinf}\dfrac{\paren{-1}^k}{\e{k}+P\paren{k}}\).

\begin{enumerate}
    \item Justifiez l'existence de \(u_n\). \\
    \item Montrez que la série \(\sum_{n\geq1}u_n\) converge.
\end{enumerate}
\end{exoss}

\begin{corr}
\note{À venir}
\end{corr}

\begin{exoss}[Exercice 9]~\\
Pour \(n\geq1\), on pose \(u_n=\sum_{k=n}^{\pinf}\dfrac{\paren{-1}^k}{\sqrt{k}}\).

\begin{enumerate}
    \item Justifiez l'existence de \(u_n\). \\
    \item Montrez que \(\dfrac{u_n+u_{n+1}}{2}\) est le reste d'une série alternée absolument convergente. \\
    \item Déduisez-en la nature de la série \(\sum u_n\).
\end{enumerate}
\end{exoss}

\begin{corr}
\note{À venir}
\end{corr}

\begin{exoss}[Exercice 10, utilisation de développements limités ou asymptotiques]~\\
\begin{enumerate}
    \item Montrez que la série \(\sum_{n\geq2}\dfrac{\paren{-1}^n}{n-\sqrt{n}}\) converge. \\
    \item Montrez que la série \(\sum_{n\geq2}\ln\paren{1+\dfrac{\paren{-1}^n}{n}}\) converge. \\
    \item Montrez que la série \(\sum_{n\geq2}\dfrac{1}{1+\paren{-1}^nn}\) converge. \\
    \item Montrez que la série \(\sum_{n\geq0}\sin\paren{\pi\sqrt{n^2+1}}\) converge. \\
    \item Déterminez la nature de la série \(\sum_{n\geq2}\paren{-1}^n\dfrac{\ln\paren{n+\paren{-1}^n\sqrt{n}}}{n}\). \\
    \item Déterminez la nature de la série \(\sum_{n\geq2}\dfrac{1}{1+\paren{-1}^n\sqrt{n}}\).
\end{enumerate}
\end{exoss}

\begin{corr}
\note{À venir}
\end{corr}

\begin{exoss}[Exercice 11]
Pour quelles valeurs de \(\alpha>0\) la série \(\sum_{n\geq2}\dfrac{\paren{-1}^n}{\sqrt{\paren{-1}^n+n^\alpha}}\) converge-t-elle ?
\end{exoss}

\begin{corr}
\note{À venir}
\end{corr}

\begin{exoss}[Exercice 12, formule de Stirling]~\\
Montrez que la suite de terme général \(\dfrac{n!}{\sqrt{n}}\paren{\dfrac{\e{}}{n}}^n\) converge vers un réel strictement positif \(L\) (indication : passer au logarithme et penser à une série).

Soit \(u_n=\int_0^{\nicefrac{\pi}{2}}\cos^nt\odif{t}\). On montre que \(u_n\simqd{n\to\pinf}\sqrt{\dfrac{\pi}{2n}}\) et que \(u_{2n}=\dfrac{\pi\paren{2n}!}{2^{2n+1}\paren{n!}^2}\). En admettant ces résultats, montrez la formule de Stirling : \[n!\simqd{n\to\pinf}\sqrt{2\pi n}\paren{\dfrac{n}{\e{}}}^n.\]
\end{exoss}

\begin{corr}
\note{À venir}
\end{corr}

\begin{exoss}[Exercice 13]~\\
Soient \(a\in\R\) et \(u_n=\dfrac{\paren{-1}^{\frac{n\paren{n+1}}{2}}}{n^a}\).

\begin{enumerate}
    \item Dans le cas où \(a\leq0\) ou \(a>1\), quelle est la nature de la série \(\sum_{n\geq1}u_n\) ? \\
    \item On suppose désormais que \(0<a\leq1\) et on pose \(v_n=u_{2n-1}+u_{2n}\). Montrez que la série \(\sum v_n\) converge. Déduisez-en la nature de la série \(\sum u_n\).
\end{enumerate}
\end{exoss}

\begin{corr}
\note{À venir}
\end{corr}

\begin{exoss}[Exercice 14]
Soient \(u\) une suite strictement positive et \(\alpha>0\).

Montrez que les séries de termes généraux \(u_n\), \(v_n=\dfrac{u_n}{1+u_n}\), \(w_n=\ln\paren{1+u_n}\) et \(x_n=\int_0^{u_n}\dfrac{1}{1+x^\alpha}\odif{x}\) sont de même nature.
\end{exoss}

\begin{corr}
\note{À venir}
\end{corr}

\begin{exoss}[Exercice 15]
Soit \(u\) une suite réelle qui ne s'annule pas telle que \(\dfrac{u_{2n+1}}{u_{2n}}\tendqd{n\to\pinf}a\) et \(\dfrac{u_{2n}}{u_{2n-1}}\tendqd{n\to\pinf}b\).

Montrez que si \(\abs{ab}<1\), alors la série \(\sum u_n\) converge.
\end{exoss}

\begin{corr}
\note{À venir}
\end{corr}

\begin{exoss}[Exercice 16]
Soit \(u\) une suite réelle positive décroissante.

Montrez que si \(\sum u_n\) converge, alors \(u_n=\o{\dfrac{1}{n}}\) quand \(n\) tend vers \(\pinf\).

La réciproque est-elle vraie ?
\end{exoss}

\begin{corr}
\note{À venir}
\end{corr}

\begin{exoss}[Exercice 17]
Soit \(\paren{u_n}_{n\geq1}\) une suite réelle strictement positive et bornée telle que la série \(\sum_{n\geq1}u_n\) diverge. Pour \(n\in\Ns\), on pose \(S_n\) la somme partielle d'indice \(n\) de la série \(\sum_{n\geq1}u_n\).

\begin{enumerate}
    \item Montrez que \(\dfrac{u_n}{S_n}\sim\ln\dfrac{S_n}{S_{n+1}}\). Déduisez-en la nature de la série \(\sum_{n\geq1}\dfrac{u_n}{S_n}\). \\
    \item Étudiez la série \(\sum_{n\geq1}\dfrac{u_n}{S_n^\alpha}\) quand \(\alpha\in\intervee{0}{1}\). \\
    \item Soit \(\alpha>1\). Montrez que \(\dfrac{u_n}{S_n^\alpha}\leq\int_{S_{n-1}}^{S_n}\dfrac{1}{x^\alpha}\odif{x}\). Déduisez-en la nature de la série \(\sum_{n\geq1}\dfrac{u_n}{S_n^\alpha}\).
\end{enumerate}
\end{exoss}

\begin{corr}
\note{À venir}
\end{corr}

\begin{exoss}[Exercice 18]
Soit \(u\) une suite strictement positive. Pour \(n\in\N\), on pose \(s_n=\sum_{k=0}^nu_k\).

On suppose que \(u_ns_n\) tend vers \(1\) quand \(n\) tend vers \(\pinf\). Déterminez un équivalent simple de \(u_n\).
\end{exoss}

\begin{corr}
\note{À venir}
\end{corr}

\begin{exoss}[Exercice 19]
Soit \(u\) la suite définie par récurrence par \(u_0>0\) et \(\quantifs{\tpt n\in\N}u_{n+1}=u_n+u_n^2\). On pose \(v_n=\dfrac{1}{2^n}\ln u_n\), puis \(w_n=v_{n+1}-v_n\).

\begin{enumerate}
    \item Montrez que la suite \(\paren{u_n}\) diverge vers \(\pinf\). \\
    \item Montrez que la suite \(\paren{v_n}\) converge vers un réel \(l>0\). On pose alors \(A=\e{l}>1\). \\
    \item Montrez que \(u_n\sim A^{2^n}\).
\end{enumerate}
\end{exoss}

\begin{corr}
\note{À venir}
\end{corr}

\begin{exosss}[Exercice 20, transformation d'Abel]
Soient \(u\) une suite réelle et \(v\) une suite complexe. Pour \(n\in\N\), on pose \(V_n=\sum_{k=0}^nv_k\).

On suppose que la suite \(u\) est positive et décroissante de limite nulle et que la suite \(V\) est bornée.

\begin{enumerate}[series=transformationAbel]
    \item Montrez que \(\quantifs{\tpt n\in\N}\sum_{k=0}^nu_kv_k=u_nV_n-\sum_{k=0}^{n-1}\paren{u_{k+1}-u_k}V_k\). \\
    \item Déduisez-en que la série \(\sum u_nv_n\) converge.
\end{enumerate}

Applications :

\begin{enumerate}[resume=transformationAbel]
    \item Soit \(w\) une suite complexe telle que \(\sum w_n\) converge. Montrez que \(\quantifs{\tpt a>0}\sum\dfrac{w_n}{n^a}\) converge aussi. \\
    \item Soient \(a>0\) et \(\theta\in\R\). Donnez la nature des séries \(\sum\dfrac{\e{\i n\theta}}{n^a}\), \(\sum\dfrac{\cos\paren{n\theta}}{n^a}\) et \(\sum\dfrac{\sin\paren{n\theta}}{n^a}\). \\
    \item Montrez que \(\quantifs{\tpt x\in\R}\abs{\sin x}\geq\dfrac{1-\cos\paren{2x}}{2}\). Déterminez la nature des séries \(\sum\dfrac{\abs{\cos\paren{n\theta}}}{n^a}\) et \(\sum\dfrac{\abs{\sin\paren{n\theta}}}{n^a}\).
\end{enumerate}
\end{exosss}

\begin{corr}
\note{À venir}
\end{corr}

\begin{exosss}[Exercice 21]
Soit \(u\) une suite positive de limite nulle. On appelle \(U_n\) la somme partielle d'indice \(n\) de la série \(\sum u_n\) et on suppose qu'il existe une constante \(M>0\) telle que \(\quantifs{\tpt n\in\N}\abs{U_n-nu_n}\leq M\).

\begin{enumerate}
    \item Montrez que \(\quantifs{\tpt n\geq2}\abs{\dfrac{U_n}{n}-\dfrac{U_{n-1}}{n-1}}\leq M\paren{\dfrac{1}{n-1}-\dfrac{1}{n}}\). \\
    \item Montrez que la série \(\sum u_n\) converge.
\end{enumerate}
\end{exosss}

\begin{corr}
\note{À venir}
\end{corr}

\begin{exosss}[Exercice 22]
Soit \(\sum_{n\geq1}u_n\) une série convergente à termes positifs.

\begin{enumerate}
    \item Montrez que \(\dfrac{u_1+2u_2+\dots+nu_n}{n}\) tend vers \(0\) quand \(n\) tend vers \(\pinf\). \\
    \item Montrez que la série \(\sum\dfrac{u_1+2u_2+\dots+nu_n}{n\paren{n+1}}\) converge et montrez que sa somme est la même que celle de la série \(\sum_{n\geq1}u_n\).
\end{enumerate}
\end{exosss}

\begin{corr}
\note{À venir}
\end{corr}


\chapter{Familles sommables}

\legendeexercices

\begin{exoss}[Exercice 23]~\\
La famille \(\paren{\dfrac{1}{pq\paren{p+q}}}_{p,q\geq1}\) est-elle sommable ?
\end{exoss}



\begin{exoss}[Exercice 24]
\begin{enumerate}
    \item Soit \(\alpha>0\). Montrez que la série \(\sum_{n\geq0}\dfrac{n^\alpha}{2^n}\) est convergente. On note \(S\paren{\alpha}\) sa somme. \\
    \item Dans cette question, on pose \(\alpha=1\) et on note \(s=S\paren{1}=\sum_{n=0}^{\pinf}\dfrac{n}{2^n}\). En effectuant le changement d'indice \(m=n-1\), montrez que \(s=2\paren{s-\sum_{m=0}^{\pinf}\dfrac{1}{2^m}}\) et donnez la valeur de \(S\paren{1}\). \\
    \item En vous inspirant de ce qui précède, donnez une expression de \(S\paren{2}\) en fonction de \(S\paren{1}\) et \(S\paren{0}\), puis sa valeur. \\
    \item Montrez que la famille \(\paren{\dfrac{\paren{-1}^{m+n}m}{2^{m+n}}}_{m,n\geq0}\) est sommable et calculez sa somme.
\end{enumerate}
\end{exoss}



\begin{exoss}[Exercice 25]
Soit \(a\) un complexe tel que \(\abs{a}<1\).

En utilisant un produit de Cauchy, montrez que \(\sum_{n=0}^{\pinf}\paren{n+1}a^n=\paren{\dfrac{1}{1-a}}^2\).
\end{exoss}



\begin{exoss}[Exercice 26]
\begin{enumerate}
    \item Pour \(z\in\C\) tel que \(\abs{z}<1\) et \(N\in\N\), que vaut \(\sum_{n=N}^{\pinf}z^n\) ? \\
    \item Soit \(x\in\C\) tel que \(\abs{x}<1\). Montrez que \(\sum_{n=1}^{\pinf}\dfrac{x^n}{1-x^{2n}}=\sum_{p=0}^{\pinf}\dfrac{x^{2p+1}}{1-x^{2p+1}}\).
\end{enumerate}
\end{exoss}



\begin{exoss}[Exercice 27]~\\
Pour \(n\in\Ns\), on pose \(H_n=\sum_{k=1}^{n}\dfrac{1}{k}\). On rappelle que \(H_n=\ln n+\gamma+\o{1}\).

\begin{enumerate}
    \item Soit \(m\in\Ns\). Montrez que \(\sum_{n=1}^{\pinf}\dfrac{1}{n\paren{n+m}}\simqd{m\to\pinf}\dfrac{\ln m}{m}\). \\
    \item Montrez que la famille \(\paren{\dfrac{\paren{-1}^m}{m\paren{m+n^2}}}_{m,n\geq1}\) est sommable. \\
    \item Montrez que la famille \(\paren{\dfrac{\paren{-1}^m}{\paren{m+n}\paren{m+n-1}}}_{m,n\geq1}\) est sommable et donnez la valeur de sa somme.
\end{enumerate}
\end{exoss}



\begin{exoss}[Exercice 28]
Pour \(n\geq2\), on note \(P\paren{n}\) le plus grand diviseur premier de \(n\). On note \(p_1=2\), \(p_2=3\), \(\dots\) la suite croissante des nombres premiers.

\begin{enumerate}
    \item Montrez que \(\quantifs{\tpt k\geq3}p_{k-1}\leq p_k-2\), puis \(\dfrac{p_k}{p_k-1}\leq\sqrt{\dfrac{p_k}{p_{k-1}}}\). \\
    \item Montrez que la série \(\sum_{n\geq2}\dfrac{1}{nP\paren{n}}\) converge (indication : pensez à une sommation par paquets).
\end{enumerate}
\end{exoss}



\begin{exoss}[Exercice 29]
Soit \(u\) une suite complexe.

Pour \(x\in\R\), on pose \(H_x=\accol{z\in\C\tq\Re z>x}\), son adhérence est \(\conj{H_x}=\accol{z\in\C\tq\Re z\geq x}\).

\begin{enumerate}
    \item Montrez que s'il existe \(s_0\in\C\) tel que la famille \(\paren{\dfrac{u_n}{n^{s_0}}}_{n\geq1}\) est sommable, alors pour tout \(s\in\C\) tel que \(\Re s>\Re s_0\), la famille \(\paren{\dfrac{u_n}{n^s}}_{n\geq1}\) est sommable. \\
    \item Quand la famille \(\paren{\dfrac{u_n}{n^s}}_{n\geq1}\) est sommable, on pose \(f_u\paren{s}=\sum_{n\in\Ns}\dfrac{u_n}{n^s}\). \\ Montrez que l'ensemble de définition de \(f_u\) est, s'il est non-vide, \(\C\), un ensemble \(H_x\) ou un ensemble \(\conj{H_x}\). \\
    \item Pour \(n\in\Ns\), on pose \(D_n=\accol{\paren{d,d\prim}\in\paren{\Ns}^2\tq dd\prim=n}\). Montrez que \(\paren{\Ns}^2=\bigsqcup_{n\in\Ns}D_n\). \\
    \item Soient \(\paren{a_n},\paren{b_n}\) deux suites complexes et \(s\in\C\) tels que les familles \(\paren{\dfrac{a_n}{n^s}}_{n\geq1}\) et \(\paren{\dfrac{b_n}{n^s}}_{n\geq1}\) soient sommables. \\ On pose, pour \(n\in\Ns\), \(c_n=\sum_{d\divise n}a_db_{\nicefrac{n}{d}}\). Montrez que la famille \(\paren{\dfrac{c_n}{n^s}}_{n\geq1}\) est sommable et que \(f_c\paren{s}=f_a\paren{s}\times f_b\paren{s}\).
\end{enumerate}
\end{exoss}



\begin{exoss}[Exercice 30]
Cet exercice prolonge le précédent.

On rappelle la définition de l'indicatrice d'Euler : pour \(n\in\Ns\), \(\phi_n\) est le cardinal de l'ensemble \(\accol{k\in\interventierii{1}{n}\tq k\et n=1}\).

On définit par récurrence la suite de Möbius : \(\mu_1=1\) et \(\quantifs{\tpt n\geq2}\mu_n=-\sum_{\substack{d\divise n \\ d<n}}\mu_d\).

Enfin, on note \(\delta_n\) le nombre de diviseurs de \(n\) et \(\sigma_n\) la somme des diviseurs de \(n\).

On pose \(\zeta=f_1\), \(\xi=f_\phi\) et \(M\paren{s}=f_\mu\).

\begin{enumerate}
    \item Montrez que l'ensemble de définition (au sens précédent) de \(\zeta\) est \(H_1\). Montrez que \(\xi\) est définie sur \(H_2\). \\
    \item On admet la relation suivante : \(\quantifs{\tpt n\in\Ns}n=\sum_{d\divise n}\phi_d\). Donnez une relation valable sur \(H_2\) liant les fonctions \(\xi\) et \(\zeta\). Justifiez alors que l'ensemble de définition de \(\xi\) est \(H_2\). \\
    \item On admet que \(\quantifs{\tpt n\in\Ns}\abs{\mu_n}\leq1\). Donnez une relation entre \(M\) et \(\zeta\) et précisez l'ensemble de définition de \(M\). \\
    \item Déduisez-en la relation : \(\quantifs{\tpt n\in\Ns}\dfrac{\phi_n}{n}=\sum_{d\divise n}\dfrac{\mu_d}{d}\) en admettant l'unicité des coefficients \(u_n\) d'une fonction \(f_u\). \\
    \item Exprimez \(f_\delta\) et \(f_\sigma\) en fonction de \(\zeta\) et précisez leurs ensembles de définition.
\end{enumerate}
\end{exoss}




\chapter{Rappels et compléments d'algèbre linéaire}

\legendeexercices

\begin{exopss}[Exercice 1]
Soient \(E\) un \(\K\)-espace vectoriel et \(\paren{u_1,\dots,u_n}\) une famille libre de \(E\).

\begin{enumerate}
    \item Pour \(i\in\interventierii{1}{n}\), on pose \(c_i=\sum_{\substack{1\leq k\leq n \\ k\not=i}}u_k\). Montrez que la famille \(\paren{c_i}_{1\leq i\leq n}\) est libre. \\
    \item Soit \(\paren{\lambda_1,\dots,\lambda_n}\in\K^n\). On pose \(s=\sum_{i=1}^n\lambda_iu_i\) et \(\quantifs{\tpt i\in\interventierii{1}{n}}v_i=s+u_i\). Montrez que la famille \(\paren{v_1,\dots,v_n}\) est liée ssi \(\sum_{i=1}^n\lambda_i=-1\). \\
    \item Soit \(\lambda\in\K\). On pose \(s=\sum_{i=1}^nu_i\) et \(\quantifs{\tpt i\in\interventierii{1}{n}}v_i=s+\lambda u_i\). Montrez qu'il existe exactement deux valeurs de \(\lambda\) pour lesquelles la famille \(\paren{v_1,\dots,v_n}\) est liée.
\end{enumerate}
\end{exopss}



\begin{exopss}[Exercice 2]
Soit \(n\in\N\). Pour \(k\in\interventierii{0}{n}\), on pose \(P_k=X^k\paren{1-X}^{n-k}\). Montrez que la famille \(\fami{F}_n=\paren{P_0,\dots,P_n}\) est une base de \(\polydeg[\C]{n}\).
\end{exopss}



\begin{exopss}[Exercice 3]
Soit \(f\) une application d'un ensemble \(\Omega\) dans \(\C\) qui prend une infinité de valeurs.

Montrez que \(\quantifs{\tpt n\in\N}\text{la famille }\paren{1,f,f^2,\dots,f^n}\text{ est libre dans l'espace }\F{\Omega}{\C}\).
\end{exopss}



\begin{exopss}[Exercice 4]
Soit \(n\in\N\). Pour \(k\in\interventierii{0}{n}\), on pose \(P_k=\paren{X+k}^n\). Montrez que la famille \(\fami{F}_n=\paren{P_0,\dots,P_n}\) est libre.
\end{exopss}



\begin{exos}[Exercice 5]
Soit \(a\in\Cs\). \(\C\) est vu comme un \(\R\)-espace vectoriel. Soit \(f\) l'application de \(\C\) dans \(\C\) qui à \(z\) associe \(z+a\conj{z}\).

\begin{enumerate}
    \item Montrez que \(f\) est linéaire. \\
    \item Montrez que si \(\abs{a}\not=1\), alors \(f\) est un automorphisme de \(\C\). \\
    \item Déterminez le noyau et l'image de \(f\) dans le cas où \(a=\e{\i\alpha}\) (on pourra utiliser l'écriture trigonométrique des complexes).
\end{enumerate}
\end{exos}



\begin{exops}[Exercice 6]
Soit \(f\) un endomorphisme de \(\poly\) qui conserve le degré : \(\quantifs{\tpt P\in\poly}\deg f\paren{P}=\deg P\).

Montrez que \(f\) est un automorphisme de \(\poly\) (on pourra étudier les restrictions de \(f\) à \(\polydeg{n}\)).
\end{exops}



\begin{exopss}[Exercice 7]
Soit \(n\in\N\). Pour \(P\in\polydeg{n}\), on pose \(f\paren{P}=\paren{X^2-1}P\prim-nXP\).

\begin{enumerate}
    \item Montrez que \(f\) est un endomorphisme de \(\polydeg{n}\). \\
    \item Montrez que si \(P\in\ker f\), alors \(X^2-1\) divise \(P\), puis justifiez qu'il existe \(\alpha\in\Ns\) et \(Q\in\poly\) tels que \(P=\paren{X^2-1}^\alpha Q\) et \(\paren{Q\paren{1}\not=0\text{ ou }Q\paren{-1}\not=0}\). \\
    \item Montrez que si \(n\) est impair, alors \(f\) est un automorphisme. \\
    \item Montrez que si \(n\) est pair, alors \(\ker f\) est une droite vectorielle. Déduisez-en la dimension de \(\Im f\).
\end{enumerate}
\end{exopss}



\begin{exoss}[Exercice 8]
On pose \(E=\poly[\C]\). Pour \(P\in E\), on pose \(f\paren{P}=\paren{X^2-1}P\prim-2XP\paren{X}\).

\begin{enumerate}
    \item Montrez que \(f\in\Lendo{E}\). \\
    \item Déterminez \(\ker f\). \\
    \item Montrez que \(\Im f=\accol{Q\in\poly[C]\tq Q\prim\paren{1}=Q\paren{1}\text{ et }Q\prim\paren{-1}=-Q\paren{-1}}\). Indication : restreindre à \(\polydeg[\C]{n}\).
\end{enumerate}
\end{exoss}



\begin{exos}[Exercice 9]
Soient \(E,F,G\) trois \(\K\)-espaces vectoriels, \(f\in\L{E}{F}\) et \(g\in\L{F}{G}\).

Montrez que \(f\paren{\ker\paren{g\rond f}}=\ker g\inter\Im f\).
\end{exos}



\begin{exopss}[Exercice 10]
Soient \(E,F,G\) trois \(\K\)-espaces vectoriels et \(\paren{f,g}\in\L{E}{F}\times\L{F}{G}\).

\begin{enumerate}
    \item Montrez que \(\ker\paren{g\rond f}=\ker f\ssi\ker g\inter\Im f=\accol{0}\). \\
    \item Montrez que \(\Im\paren{g\rond f}=\Im g\ssi\ker g+\Im f=F\).
\end{enumerate}
\end{exopss}



\begin{exosss}[Exercice 11]
Soient \(E,F\) deux \(\K\)-espaces vectoriels et \(f\in\L{E}{F}\). On dit que \(g\in\L{F}{E}\) est un inverse à droite de \(f\) quand \(f\rond g=\id{F}\).

\begin{enumerate}
    \item Montrez que si \(f\) possède deux inverses à droite différents, alors \(f\) en possède une infinité. \\
    \item Montrez que si \(f\) possède un unique inverse à droite, alors \(f\) est un isomorphisme (vous admettrez l'existence d'un supplémentaire de tout sous-espace vectoriel).
\end{enumerate}
\end{exosss}



\begin{exopss}[Exercice 12]
Soient \(E\) un \(\K\)-espace vectoriel et \(p,q\) deux projecteurs de \(E\).

\begin{enumerate}
    \item Montrez que \(p+q\) est un projecteur ssi \(p\rond q=q\rond p=0\). \\
    \item Dans ce cas, montrez alors que \(\Im\paren{p+q}=\Im p\oplus\Im q\) et \(\ker\paren{p+q}=\ker p\inter\ker q\).
\end{enumerate}
\end{exopss}



\begin{exoss}[Exercice 13]
Soient \(E\) un \(\K\)-espace vectoriel et \(p,q\) deux projecteurs de \(E\) tels que \(p\rond q=0\). Soit \(r=p+q-q\rond p\).

Montrez que \(r\) est un projecteur et précisez ses éléments caractéristiques.
\end{exoss}



\begin{exoss}[Exercice 14]
Soit \(E\) un \(\K\)-espace vectoriel. On suppose qu'il existe \(\paren{f,g}\in\Lendo{E}^2\) tel que \(f\rond g-g\rond f=\id{E}\).

\begin{enumerate}
    \item Démontrez que \(\quantifs{\tpt n\in\Ns}f\rond g^n-g^n\rond f=ng^{n-1}\). \\
    \item Déduisez-en que \(\quantifs{\tpt n\in\Ns}\text{la famille }\paren{g^k}_{0\leq k\leq n}\text{ est libre dans }\Lendo{E}\). \\
    \item Si \(E\) est de dimension finie \(p\geq1\), que pouvez-vous conclure ?
\end{enumerate}
\end{exoss}



\begin{exoss}[Exercice 15]
Soient \(E\) un \(\R\)-espace vectoriel et \(f\) un endomorphisme de \(E\) tel que \(f^3=\id{E}\). Pour \(a\in\R\) et \(u\in E\), on veut résoudre l'équation \(x+af\paren{x}=u\) d'inconnue \(x\in E\).

\begin{enumerate}
    \item Montrez que pour toutes les valeurs de \(a\), sauf une seule \(a_0\), l'équation a une unique solution que vous calculerez. \\
    \item Dans le cas où \(a=a_0\), donnez une condition nécessaire sur \(u\) pour qu'il existe une solution, puis si cette condition est satisfaite, déterminez une solution particulière de l'équation qui soit combinaison linéaire de \(u\) et \(f\paren{u}\). Concluez.
\end{enumerate}
\end{exoss}



\begin{exopss}[Exercice 16]
Soient \(E\) un espace vectoriel de dimension finie \(n\) et \(f\in\Lendo{E}\).

\begin{enumerate}
    \item Montrez que \(\Im f=\ker f\) et, ssi \(n\) est pair, \(\rg f=\dfrac{n}{2}\) et \(f^2=0\). \\
    \item Donnez un exemple d'une telle application linéaire \(f\). \\
    \item Si les conditions de la question (1)  sont satisfaites, alors on pose \(r=\dfrac{n}{2}\) : montrez qu'il existe une base de \(E\) dans laquelle la matrice de \(f\) est \(\begin{pmatrix}0 & 0 \\ I_r & 0\end{pmatrix}\) (matrice par blocs).
\end{enumerate}
\end{exopss}



\begin{exopss}[Exercice 17]
Soient \(E\) un \(\K\)-espace vectoriel de dimension finie et \(f\in\Lendo{E}\). Montrez l'équivalence \[E=\ker f\oplus\Im f\ssi\ker f^2=\ker f\ssi\Im f=\Im f^2.\]
\end{exopss}



\begin{exos}[Exercice 18]
Soient \(E,F\) deux \(\K\)-espaces vectoriels de dimensions finies et \(\paren{f,g}\in\L{E}{F}^2\). Montrez que \(\abs{\rg f-\rg g}\leq\rg\paren{f+g}\leq\rg f+\rg g\).
\end{exos}



\begin{exoss}[Exercice 19]
Soient \(E\) un \(\K\)-espace vectoriel de dimension finie et \(\paren{u,v}\in\Lendo{E}^2\). On suppose que \(u\rond v=0\) et \(u+v\in\GL{}[E]\). Démontrez que \(\rg u+\rg v=\dim E\).
\end{exoss}



\begin{exoss}[Exercice 20]
Soient \(E,F\) deux \(\K\)-espaces vectoriels de dimensions finies.

\begin{enumerate}
    \item Soit \(U\) un sous-espace vectoriel de \(E\). On pose \(\fami{A}=\accol{u\in\L{E}{F}\tq U\subset\ker u}\). Montrez que \(\fami{A}\) est un sous-espace vectoriel de \(\L{E}{F}\) tel que \(\quantifs{\tpt\paren{f,u}\in\Lendo{E}\times\fami{A}}fu\in\fami{A}\) et calculez sa dimension. \\
    \item Montrez que la réciproque est vraie : si \(\fami{A}\) est un sous-espace vectoriel de \(\L{E}{F}\) tel que \(\quantifs{\tpt\paren{f,u}\in\Lendo{E}\times\fami{A}}fu\in\fami{A}\), alors il existe \(U\) un sous-espace vectoriel de \(E\) tel que \(\fami{A}=\accol{u\in\L{E}{F}\tq U\subset\ker u}\).
\end{enumerate}
\end{exoss}



\begin{exops}[Exercice 21]
Soit \(A\in\M{n}\) une matrice de rang \(1\). Montrez que \(A\) peut s'écrire comme le produit d'une matrice colonne par une matrice ligne. Déduisez-en \(A^2=\tr\paren{A}A\).
\end{exops}



\begin{exopss}[Exercice 22]
Soit \(f\) une forme linéaire sur \(\M{n}[\R]\) telle que \(\quantifs{\tpt\paren{A,B}\in\M{n}[\R]^2}f\paren{AB}=f\paren{BA}\). Montrez que \(f\) est proportionnelle à la trace (indication : faire intervenir la base canonique de \(\M{n}[\R]\)).
\end{exopss}



\begin{exosss}[Exercice 23]
Soient \(E\) un \(\K\)-espace vectoriel de dimension finie et \(G\) un sous-groupe de cardinal \(n\) de \(\GL{}[E]\).

On pose \(F=\accol{x\in E\tq\quantifs{\forall g\in G}g\paren{x}=x}\).

Montrez que \(\dim F=\dfrac{1}{n}\sum_{g\in G}\tr g\). Indication : on pourra utiliser \(p=\dfrac{1}{n}\sum_{g\in G}g\).
\end{exosss}



\begin{exos}[Exercice 24]
Montrez qu'il n'existe pas de couple de matrices \(\paren{A,B}\in\M{n}^2\) tels que \(AB-BA=I_n\).
\end{exos}



\begin{exops}[Exercice 25]
Soient \(A\) et \(B\) deux matrices carrées d'ordre \(n\). On suppose que \(\quantifs{\tpt X\in\M{n}}\tr\paren{AX}=\tr\paren{BX}\). Montrez que \(A=B\).
\end{exops}



\begin{exos}[Exercice 26]
Soient \(u,v\) les deux suites réelles telles que \(u_0=1\), \(v_0=-1\) et \(\quantifs{\tpt n\in\N}u_{n+1}=-3u_n+10v_n\) et \(v_{n+1}=-3u_n+8v_n\).

Donnez des expressions de \(u_n\) et \(v_n\) en fonction de \(n\).
\end{exos}



\begin{exoss}[Exercice 27]
Soient \(u,v,w\) les trois suites réelles telles que \(u_0=v_0=w_0=1\) et \[\quantifs{\tpt n\in\N}\begin{dcases}
u_{n+1}=\dfrac{3}{2}u_n-\dfrac{1}{2}v_n+\dfrac{1}{2}w_n \\
v_{n+1}=5u_n-\dfrac{5}{2}v_n+2w_n \\
w_{n+1}=4u_n-\dfrac{5}{2}v_n+2w_n
\end{dcases}\]

Donnez des expressions de \(u_n\), \(v_n\) et \(w_n\) en fonction de \(n\) et leurs limites quand \(n\) tend vers \(\pinf\).
\end{exoss}



\begin{exopss}[Exercice 28]
Soient \(A\in\M{p}\), \(B\in\M{q}\) et \(C\in\M{pq}\). On suppose connaître \(P\) un polynôme annulateur de \(A\) et \(Q\) un polynôme annulateur de \(B\).

On pose \(M=\begin{pmatrix}A & C \\ 0 & B\end{pmatrix}\in\M{p+q}\).

\begin{enumerate}
    \item Montrez que \(PQ\) est un polynôme annulateur de \(M\) dans le cas où \(C=0\). \\
    \item Montrez que ce résultat reste vrai même si \(C\) n'est pas nulle. Indication : penser à un produit par blocs.
\end{enumerate}
\end{exopss}



\begin{exopss}[Exercice 29]
Soient \(A\in\M{n}\) et \(P\) un polynôme annulateur de \(A\). On définit \(B\in\M{2n}\) par blocs : \(B=\begin{pmatrix}
0 & I_n \\
A & 0
\end{pmatrix}\).

\begin{enumerate}
    \item Calculez pour \(k\in\N\) la matrice \(B^k\) en distinguant les cas \(k\) pair et \(k\) impair. \\
    \item On pose \(Q\paren{X}=P\paren{X^2}\). Montrez que \(Q\) est un polynôme annulateur de \(B\).
\end{enumerate}
\end{exopss}



\begin{exopss}[Exercice 30]
Soit \(A\in\M{n}\). On définit \(B\in\M{2n}\) par blocs : \(B=\begin{pmatrix}
A & A \\
A & A
\end{pmatrix}\).

\begin{enumerate}
    \item Calculez pour \(k\in\N\) la matrice \(B^k\). \\
    \item Déterminez un polynôme annulateur de \(B\).
\end{enumerate}
\end{exopss}



\begin{exopss}[Exercice 31]
Soient \(n\in\Ns\) et \(E=\polydeg{n}\). On pose \(d:P\mapsto P\prim\), endomorphisme de \(E\), et \(f:P\mapsto P+P\prim\).

\begin{enumerate}
    \item Donnez un polynôme annulateur de \(d\). Déduisez-en un polynôme annulateur de \(f\). \\
    \item Montrez que \(f\) est un automorphisme, puis exprimez son inverse à l'aide de \(f\). \\
    \item Vérifiez que \(f\inv=\sum_{k=0}^{n-1}\paren{-1}^kd^k\).
\end{enumerate}
\end{exopss}



\begin{exopss}[Exercice 32]
Soit \(A\in\M{n}\). Pour tout \(M\in\M{n}\), on pose \(f\paren{M}=M+\tr\paren{M}A\).

\begin{enumerate}
    \item Montrez que \(f\) est un endomorphisme de \(\M{n}\). \\
    \item Déterminez un polynôme annulateur de \(f\) de degré \(2\). \\
    \item Dans quels cas \(f\) est-il un automorphisme ? Calculez \(f\inv\) quand c'est possible. \\
    \item Dans le cas contraire, vérifiez que \(f\) est un projecteur et déterminez ses éléments caractéristiques.
\end{enumerate}
\end{exopss}



\begin{exopss}[Exercice 33]
Soient \(E\) un espace vectoriel réel de dimension finie et \(f\in\Lendo{E}\).

On suppose que \(f\) possède un polynôme annulateur \(P\) vérifiant \(P\paren{0}=0\) et \(P\prim\paren{0}\not=0\).

Montrez qu'on a alors \(\Im f\oplus\ker f=E\).
\end{exopss}



\begin{exo}[Oral CCINP, 1]
Soient \(E\) un espace vectoriel de dimension finie et \(p,q\) deux endomorphismes de \(E\).

On suppose que \(p+q=\id{E}\) et \(\rg p+\rg q\leq\dim E\).

Montrez que \(p\) et \(q\) sont deux projecteurs.
\end{exo}



\begin{exo}[Oral IMT, 2]
Soient \(E\) un espace vectoriel de dimension supérieure à \(2\) et \(f,g\) deux endomorphismes de \(E\) tels que \(f\rond g\rond f=f\).

\begin{enumerate}
    \item Montrez que \(f\rond g\) et \(g\rond f\) sont des projecteurs. \\
    \item Que peut-on dire des rangs de \(f\), \(f\rond g\) et \(g\rond f\) ? \\
    \item Montrez que \(f\rond g\) est un projecteur sur \(\Im f\), parallèlement à un sous-espace contenant \(\ker g\). \\
    \item On suppose désormais qu'on a aussi \(g\rond f\rond g=g\). Que dire des rangs de \(f\) et \(g\) ? \\
    \item Montrez que \(E=\Im f\oplus\ker g\).
\end{enumerate}
\end{exo}



\begin{exo}[Oral TPE, 3]
Soient \(E\) un espace vectoriel de dimension \(n\) et \(f\in\Lendo{E}\).

\begin{enumerate}
    \item Montrez que \(\quantifs{\tpt k\in\N}\Im f^{k+1}\subset\Im f^k\). \\
    \item Montrez que s'il existe un entier \(p\) tel que \(\Im f^{p+1}=\Im f^p\), alors \(\quantifs{\tpt k\in\N}\Im f^{p+k}\subset\Im f^p\). \\
    \item Déduisez-en que \(\Im f^{n+1}=\Im f^n\).
\end{enumerate}
\end{exo}



\begin{exo}[Oral TPE, 4]
Montrez que \(P\mapsto P-P\prim\) est un automorphisme de \(\poly[\R]\) et donnez son endomorphisme réciproque.
\end{exo}



\begin{exo}[Oral Centrale, 5]
Soient \(E\) un espace vectoriel de dimension finie non-nulle et \(u,v\) deux endomorphismes de \(E\).

\begin{enumerate}
    \item Montrez que \(\abs{\rg u-\rg v}\leq\rg\paren{u+v}\leq\rg u+\rg v\). \\
    \item Soient \(F\) un sous-espace de \(E\) et \(G,H\) deux supplémentaires de \(F\) dans \(E\). On pose \(p\) le projecteur sur \(F\) parallèlement à \(G\) et \(q\) celui sur \(H\) parallèlement à \(F\). Montrez que \(\rg\paren{p+q}=\rg p+\rg q\).
\end{enumerate}
\end{exo}



\begin{exo}[Oral Centrale, 6]
Soient \(E\) un \(\K\)-espace vectoriel de dimension finie, \(\paren{\phi_1,\dots,\phi_p}\) une famille libre de \(E\etoile\) (note : \(E\etoile\) est le \(\K\)-espace vectoriel des formes linéaires sur \(E\)) et \(\psi\in E\etoile\).

\begin{enumerate}
    \item Montrez que \(\psi\in\Vect{\phi_1,\dots,\phi_p}\ssi\biginter_{i=1}^p\ker\phi_i\subset\ker\psi\). \\
    \item On suppose que \(\K=\R\) ou \(\K=\C\). Montrez que les conditions de la question précédente sont encore équivalentes à l'existence d'un réel \(M>0\) tel que \(\quantifs{\forall x\in E}\abs{\psi\paren{x}}\leq M\max_{1\leq i\leq p}\abs{\phi_i\paren{x}}\).
\end{enumerate}
\end{exo}



\begin{exo}[Oral CCMP, 7]
Soient \(n,k\) deux entiers tels que \(2\leq k\leq n\). On pose \(A_k=\paren{a_{i\,j}\deriv{k}}_{1\leq i,j\leq n}\in\M{n}[\R]\) telle que \(a_{i\,j}\deriv{k}=1\) si \(i-j=k-1\), les autres coefficients étant nuls.

\begin{enumerate}[series=oral4.7]
    \item Calculez \(\trans{A_k}A_k\).
\end{enumerate}

Soit \(p\) un projecteur de \(\R^n\) tel que \(p\not=\id{}\).

\begin{enumerate}[resume=oral4.7]
    \item Justifiez que \(\rg p<n\). \\
    \item Montrez que \(p\) est la composée de deux endomorphismes nilpotents.
\end{enumerate}
\end{exo}



\begin{exo}[Oral CCMP, 8]
Soient \(E\) un \(\K\)-espace vectoriel de dimension finie et \(f,g\) deux endomorphismes de \(E\).

On suppose \(f\) inversible et \(g\) de rang \(1\).

Montrez que \(f+g\) est inversible ssi \(\tr\paren{g\rond f\inv}\not=-1\).
\end{exo}



\begin{exo}[Oral CCMP, 9]
Soient \(\paren{n,p}\in\paren{\Ns}^2\), \(A\in\M{np}[\R]\) et \(B\in\M{pn}[\R]\).

Montrez que \(p+\rg\paren{I_n+AB}=n+\rg\paren{I_p+BA}\).
\end{exo}



\begin{exo}[Oral CCMP, 10]
Soient \(n\in\Ns\) et \(A,B\in\M{n}[\R]\). On pose \(E=\accol{M\in\M{n}[\R]\tq AMB=0}\).

Montrez que \(E\) est un sous-espace vectoriel de \(\M{n}[\R]\) et donnez sa dimension.
\end{exo}



\begin{exo}[Oral CCMP, 11]
Soit \(n\in\Ns\). Résolvez dans \(\M{n}[\C]\) l'équation \(M=\Com{M}\).
\end{exo}




\chapter{Réduction des endomorphismes}

\minitoc

\section*{Exercices}
\addcontentsline{toc}{section}{Exercices}

\legendeexercices

\begin{exoss}[Exercice 1]
\begin{enumerate}
    \item Soit \(P\in\poly[\C]\) tel que \(P\) ne soit pas constant. On note \(a_1,\dots,a_m\) les racines distinctes de \(P\) et \(\alpha_1,\dots,\alpha_m\) leurs ordres de multiplicité. Rappelez la décomposition en éléments simples de \(\dfrac{P\prim}{P}\). \\
    \item Application : déterminez les éléments propres de l'endomorphisme de \(\poly[\C]\) défini par \(f\paren{P}=\paren{2X+1}P-\paren{X^2-1}P\prim\). On rappelle que la décomposition en éléments simples est unique. \\
    \item Même question avec \(f\paren{P}=\paren{X^3+X}P\prim-\paren{3X^2-1}P\).
\end{enumerate}
\end{exoss}



\begin{exoss}[Exercice 2]
Soit \(E=\ensclasse{0}{\intervii{0}{1}}{\R}\). Pour \(f\in E\), on définit l'application \(\phi\paren{f}:\intervii{0}{1}\to\R\) de la façon suivante : \(\phi\paren{f}\paren{0}=f\paren{0}\) et \(\quantifs{\tpt x\in\intervei{0}{1}}\phi\paren{f}\paren{x}=\dfrac{1}{x}\int_0^xf\paren{t}\odif{t}\).

\begin{enumerate}
    \item Montrez que \(\phi\) est un endomorphisme de \(E\). \\
    \item Montrez que \(0\) n'est pas valeur propre de \(\phi\). \\
    \item Montrez que \(1\) est valeur propre de \(\phi\) et donnez le sous-espace propre associé. \\
    \item Déterminez les autres valeurs propres de \(\phi\).
\end{enumerate}
\end{exoss}



\begin{exoss}[Exercice 3]
Soient \(n\in\Ns\), \(A\in\M{n}\excluant\accol{0}\) et \(\lambda\in\K\). Pour \(M\in\M{n}\), on pose \(f\paren{M}=\lambda M+\tr\paren{M}A\).

\begin{enumerate}
    \item Justifiez rapidement que \(f\) est un endomorphisme de \(\M{n}\). \\
    \item Déterminez un polynôme annulateur de \(f\) de degré \(2\). \\
    \item Déduisez-en les éléments propres de \(f\). \\
    \item À quelle condition l'endomorphisme \(f\) est-il inversible ?
\end{enumerate}
\end{exoss}



\begin{exoss}[Exercice 4]
Soit \(A\in\M{n}\). Pour \(M\in\M{n}\), on pose \(f\paren{M}=AM\).

Montrez que \(\Sp{f}=\Sp{A}\). Pour tout \(\lambda\) valeur propre de \(f\), donnez une relation entre les dimensions des sous-espaces propres \(\sep{f}{\lambda}\) et \(\sep{A}{\lambda}\).
\end{exoss}



\begin{exoss}[Exercice 5, matrices stochastiques]
Soit \(A=\paren{a_{i\,j}}\in\M{n}[\R]\) une matrice stochastique, \cad vérifiant : \[\quantifs{\forall\paren{i,j}\in\interventierii{1}{n}^2}a_{i\,j}\geq0\text{ et }\quantifs{\forall i\in\interventierii{1}{n}}\sum_{j=1}^na_{i\,j}=1.\]

\begin{enumerate}
    \item Montrez que \(1\) est valeur propre de \(A\). \\
    \item Montrez que toutes les valeurs propres complexes de \(A\) sont de module inférieur ou égal à \(1\). \\
    \item On suppose que \(A\) est strictement stochastique, \cad que \(A\) est stochastique et que ses coefficients sont strictement positifs. Montrez que \(1\) est la seule valeur propre de \(A\) de module \(1\).
\end{enumerate}
\end{exoss}



\begin{exoss}[Exercice 6]
Soient \(\paren{a,b}\in\R^2\) tel que \(a\not=b\) et \(A=\begin{pmatrix}
0 & a & \dots & a \\
b & \ddots & \ddots & \vdots \\
\vdots & \ddots & \ddots & a \\
b & \dots & b & 0
\end{pmatrix}\).

\begin{enumerate}
    \item On pose \(J\) la matrice remplie de \(1\). Montrez que \(t\mapsto\det\paren{XI_n-A-tJ}\) est une fonction polynôme en \(t\) de degré au plus \(1\), puis calculez \(\det\paren{XI_n-A-tJ}\) en fonction de \(t\). \\
    \item Montrez que le polynôme caractéristique de \(A\) est \(\dfrac{1}{b-a}\paren{b\paren{X+a}^n-a\paren{X+b}^n}\). \\
    \item Montrez que si \(b=-a\), alors les images dans \(\C\) des valeurs propres de \(A\) sont sur une droite que vous préciserez, sinon elles sont sur un cercle.
\end{enumerate}
\end{exoss}



\begin{exoss}[Exercice 7]
Soient \(n\geq2\), \(U\in\M{n\,1}[\R]\) et \(A=U\trans{U}\). On note \(\trans{U}U=\begin{pmatrix}
s
\end{pmatrix}\) qui est une matrice carrée à un seul élément.

\begin{enumerate}
    \item Montrez que le polynôme \(X^2-sX\) est annulateur de \(A\). Déduisez-en que \(A\) a au plus deux valeurs propres. \\
    \item Quel est le rang de \(A\) ? Précisez les valeurs propres de \(A\). \\
    \item Déterminez le polynôme caractéristique de \(A\). \\
    \item Déterminez les sous-espaces propres de \(A\).
\end{enumerate}
\end{exoss}



\begin{exos}[Exercice 8]~\\
Soit \(A=\begin{pmatrix}
-1 & 2 & -1 \\
-3 & 4 & -3 \\
-4 & 4 & -4
\end{pmatrix}\).

\begin{enumerate}
    \item Déterminez le polynôme caractéristique de \(A\). \\
    \item Déduisez-en sans calcul supplémentaire que \(A\) est diagonalisable. \\
    \item Déterminez les sous-espaces propres de \(A\). \\
    \item Diagonalisez \(A\), \cad explicitez une matrice \(P\) inversible et une matrice \(D\) diagonale telles que \(A=PDP\inv\). \\
    \item Calculez \(A^n\) pour tout \(n\in\N\).
\end{enumerate}
\end{exos}



\begin{exos}[Exercice 9, réduction en élégance]~\\
Soit \(B=\begin{pmatrix}
1 & 2 & -3 \\
2 & 4 & -6 \\
4 & 8 & -12
\end{pmatrix}\).

\begin{enumerate}
    \item Calculez efficacement le rang de \(B\). Déduisez-en une valeur propre de \(B\) ainsi que la dimension du sous-espace propre associé. \\
    \item Démontrez sans calculer le polynôme caractéristique que \(B\) admet une autre valeur propre et qu'elle est simple. \\
    \item Déduisez-en le polynôme caractéristique de \(B\). Vérifiez le résultat en le calculant grâce à sa définition. \\
    \item La matrice \(B\) est-elle diagonalisable ?
\end{enumerate}
\end{exos}



\begin{exos}[Exercice 10]
Les matrices suivantes sont-elles diagonalisables ?

\[A=\begin{pmatrix}
0 & 1 & 0 \\
0 & 0 & 1 \\
1 & 0 & 0
\end{pmatrix}\qquad B=\begin{pmatrix}
1 & 0 & 1 \\
-1 & 3 & 1 \\
1 & -1 & 2
\end{pmatrix}\qquad C=\begin{pmatrix}
1 & 1 & 1 & 1 \\
1 & 0 & 0 & 1 \\
1 & 0 & 0 & 1 \\
1 & 1 & 1 & 1
\end{pmatrix}\qquad D=\begin{pmatrix}
1 & 1 & 1 & 1 \\
0 & 0 & 1 & 0 \\
0 & 1 & 0 & 0 \\
1 & 1 & 1 & 1
\end{pmatrix}\qquad E=\begin{pmatrix}
5 & 1 & 1 & 1 \\
2 & 6 & 2 & 2 \\
3 & 3 & 7 & 3 \\
4 & 4 & 4 & 8
\end{pmatrix}\]
\end{exos}



\begin{exos}[Exercice 11]
Les matrices suivantes sont-elles diagonalisables ?

\[A=\begin{pmatrix}
0 & 0 & 4 \\
1 & 0 & -8 \\
0 & 1 & 5
\end{pmatrix}\qquad B=\begin{pmatrix}
2 & 1 & 1 \\
0 & 0 & -2 \\
0 & 1 & 3
\end{pmatrix}\]
\end{exos}



\begin{exos}[Exercice 12]
Déterminez une condition nécessaire et suffisante sur le triplet \(\paren{a,b,c}\in\R^3\) pour que la matrice \(M=\begin{pmatrix}
1 & a & 1 \\
0 & 1 & b \\
0 & 0 & c
\end{pmatrix}\) soit diagonalisable.
\end{exos}



\begin{exos}[Exercice 13]~\\
Soit \(A=\begin{pmatrix}
a & b \\
c & d
\end{pmatrix}\) une matrice de \(\M{2}[\R]\) qui n'est pas de la forme \(\lambda I_2\).

Montrez que \(A\) est diagonalisable dans \(\M{2}[\C]\) ssi \(\paren{a-d}^2+4bc\not=0\).

À quelle condition \(A\) est-elle diagonalisable dans \(\M{2}[\R]\) ?
\end{exos}



\begin{exoss}[Exercice 14]~\\
Soit \(A=\begin{pmatrix}
11 & -5 & -5 \\
-5 & 3 & 3 \\
-5 & 3 & 3
\end{pmatrix}\). On recherche les éventuelles racines carrées de \(A\), \cad les matrices \(R\in\M{3}[\R]\) telles que \(R^2=A\).

\begin{enumerate}
    \item Montrez que \(A\) est diagonalisable et déterminez une matrice \(D\) diagonale semblable à \(A\) avec le moins de calculs possible. \\
    \item Soit \(S\in\M{3}[\R]\) une racine carrée de \(D\). Montrez que \(S\) et \(D\) commutent puis montrez que \(S\) est diagonale. \\
    \item Déterminez les racines carrées \(S\) de \(D\). \\
    \item Déduisez-en toutes les racines carrées \(R\) de \(A\). Combien y en a-t-il ? Pourquoi ? \\
    \item Énoncez des conjectures quant au nombre de racines carrées d'une matrice \(A\in\M{n}[\C]\) plus générale, en discutant selon la nature de ses éléments propres.
\end{enumerate}
\end{exoss}



\begin{exoss}[Exercice 15]
Soit \(u\) l'endomorphisme canoniquement associé à la matrice \(A=\begin{pmatrix}
0 & 1 & -1 \\
2 & 1 & 1 \\
-2 & -1 & -1
\end{pmatrix}\).

\begin{enumerate}
    \item Diagonalisez la matrice \(A\). \\
    \item On suppose que \(F\) est un sous-espace de \(\R^3\) stable par \(u\). Montrez que \(F\) est engendré par une famille de vecteurs propres de \(u\). \\
    \item Déterminez tous les sous-espaces de \(\R^3\) stables par \(u\).
\end{enumerate}
\end{exoss}



\begin{exos}[Exercice 16]~\\
Même exercice avec \(A=\begin{pmatrix}
-1 & -6 & -2 \\
2 & 7 & 2 \\
-6 & -18 & -5
\end{pmatrix}\).
\end{exos}



\begin{exos}[Exercice 17]
Montrez que les suites \(\paren{u_n}\in\R^\N\) vérifiant la relation de récurrence suivante sont combinaisons linéaires de trois suites géométriques réelles \(\paren{\alpha^n}\), \(\paren{\beta^n}\) et \(\paren{\gamma^n}\) : \(\quantifs{\forall n\in\N}u_{n+3}=-\dfrac{5}{12}u_{n+2}+\dfrac{1}{2}u_{n+1}-\dfrac{1}{12}u_n\).

Déterminez à quelle condition sur \(\paren{u_0,u_1,u_2}\) ces suites sont convergentes.
\end{exos}



\begin{exos}[Exercice 18]
Même exercice avec \(\quantifs{\forall n\in\N}u_{n+3}=\dfrac{3}{2}u_{n+2}-\dfrac{3}{4}u_{n+1}+\dfrac{1}{4}u_n\) en prenant des suites géométriques complexes.

Montrez que les suites \(\paren{u_n}\) sont combinaisons linéaires de trois suites réelles simples que l'on précisera. Déterminez à quelle condition sur \(\paren{u_0,u_1,u_2}\) ces suites sont convergentes.
\end{exos}



\begin{exos}[Exercice 19]
Les matrices suivantes sont-elles trigonalisables ? Donnez alors, quand c'est possible, une matrice triangulaire supérieure semblable.

\[A=\begin{pmatrix}
1 & 4 & -2 \\
0 & 6 & -3 \\
-1 & 4 & 0
\end{pmatrix}\qquad B=\begin{pmatrix}
2 & -1 & -1 \\
2 & 1 & -2 \\
3 & -1 & -2
\end{pmatrix}\]
\end{exos}



\begin{exos}[Exercice 20]~\\
Soient \(C=\begin{pmatrix}
2 & -3 & -1 \\
1 & -2 & -1 \\
-2 & 6 & 3
\end{pmatrix}\) et \(T=\begin{pmatrix}
1 & 0 & -2 \\
0 & 1 & 1 \\
0 & 0 & 1
\end{pmatrix}\).

\begin{enumerate}
    \item Déterminez le polynôme caractéristique de \(C\). Est-elle diagonalisable ? Trigonalisable ? \\
    \item Déterminez les éléments propres de \(C\). \\
    \item Montrez que la matrice \(C\) est semblable à la matrice \(T\). On pourra considérer l'endomorphisme \(u\) canoniquement associé à \(C\) et construire, par analyse-synthèse, une base \(\fami{B}\) où la matrice de \(u\) est \(T\). \\
    \item Montrez que \(T\) peut s'écrire \(D+N\) où \(D\) est diagonale, \(N\) est nilpotente et \(D\) et \(N\) commutent. Déduisez-en, pour tout \(n\in\N\), \(T^n\) puis \(C^n\). \\
    \item On considère trois suites \(\paren{x_n},\paren{y_n},\paren{z_n}\in\R^\N\) telles que \[\quantifs{\forall n\in\N}\begin{dcases}
        x_{n+1}=2x_n-3y_n-z_n \\
        y_{n+1}=x_n-2y_n-z_n \\
        z_{n+1}=-2x_n+6y_n+3z_n
    \end{dcases}\] Explicitez \(x_n,y_n,z_n\) en fonction de \(n\) et de \(x_0,y_0,z_0\).
\end{enumerate}
\end{exos}



\begin{exoss}[Exercice 21]
Soit \(u\in\Lendo{\R^n}\) tel que \(u^2+u+\id{\R^n}=0\).

\begin{enumerate}
    \item Soient \(F\) un sous-espace stable par \(u\) et \(x\not\in F\). Montrez que \(\Pi_x=\Vect{x,u\paren{x}}\) est un plan, qu'il est stable par \(u\) et qu'il est en somme directe avec \(F\). \\
    \item Montrez qu'il existe une base de \(\R^n\) dans laquelle la matrice de \(u\) est diagonale par blocs, de blocs \(R=\begin{pmatrix}
        0 & -1 \\
        1 & -1
    \end{pmatrix}\). Qu'en déduisez-vous concernant \(n\) ? \\
    \item Soit \(A\in\M{n}[\R]\) la matrice de \(u\) dans la base canonique. Réduisez \(A\) dans \(\C\). \\
    \item Retrouvez que \(A\) est \(\R\)-semblable à une matrice diagonale par blocs, de blocs \(R\).
\end{enumerate}
\end{exoss}



\begin{exoss}[Exercice 22]
Soient \(p\in\Ns\), \(E\) un \(\C\)-espace vectoriel de dimension finie et \(f\in\GL{}[E]\).

Montrez que \(f\) est diagonalisable ssi \(f^p\) est diagonalisable.

Est-ce encore vrai si on remplace \(\C\) par \(\R\) ?
\end{exoss}



\begin{exoss}[Exercice 23]
Soient \(E\) un \(\C\)-espace vectoriel de dimension finie et \(f\in\Lendo{E}\).

\begin{enumerate}
    \item Montrez que si \(f\) est diagonalisable, alors \(f^2\) est diagonalisable et \(\rg f^2=\rg f\). \\
    \item Montrez que si \(\lambda\) est une valeur propre de \(f^2\) non-nulle et \(\mu\) est une racine carrée de \(\lambda\), alors \(\sep{f^2}{\lambda}=\ker\paren{f-\mu\id{}}\oplus\ker\paren{f+\mu\id{}}\). \\
    \item Montrez que la réciproque de la proposition de la question (1) est vraie.
\end{enumerate}
\end{exoss}



\begin{exoss}[Exercice 24, spectre du polynôme d'une matrice]
Soient \(M\in\M{n}\) diagonalisable et \(Q\in\poly\).

\begin{enumerate}
    \item Montrez que la matrice \(Q\paren{M}\) est également diagonalisable. Exprimez le spectre de \(Q\paren{M}\) en fonction des valeurs propres de \(M\). \\
    \item Précisez les sous-espaces propres de \(Q\paren{M}\). \\
    \item Ces résultats restent-ils valables si \(M\) est seulement trigonalisable ? Sans hypothèse sur \(M\) ?
\end{enumerate}
\end{exoss}



\begin{exosss}[Exercice 25]
Soit \(A\in\M{2}[\Z]\). On suppose qu'il existe \(p\in\Ns\) tel que \(A^p=I_2\). Montrez que \(A^{12}=I_2\).
\end{exosss}



\begin{exoss}[Exercice 26]
Soit \(A\in\M{n}[\C]\). Montrez que \(A\) est nilpotente ssi \(\quantifs{\tpt k\in\Ns}\tr A^k=0\).
\end{exoss}



\begin{exoss}[Exercice 27]
Soit \(f\) un endomorphisme diagonalisable d'un \(\K\)-espace vectoriel \(E\) de dimension finie. On note \(C\paren{f}\) le sous-espace vectoriel des endomorphismes de \(E\) qui commutent avec \(f\).

\begin{enumerate}
    \item Démontrez que \(g\in C\paren{f}\) ssi les sous-espaces propres de \(f\) sont stables par \(g\). \\
    \item Déduisez-en que \(\dim C\paren{f}=\sum_{\lambda\in\Sp{f}}\omega_\lambda^2\) où \(\omega_\lambda\) désigne la multiplicité de la valeur propre \(\lambda\). \\
    \item On suppose que les valeurs propres de \(f\) sont simples. Démontrez que \(\paren{\id{},f,\dots,f^{n-1}}\) est une base de \(C\paren{f}\).
\end{enumerate}
\end{exoss}



\begin{exoss}[Exercice 28]
\begin{enumerate}
    \item Montrez que \(\GL{n}\) est un ouvert dense de \(\M{n}\). \\
    \item Montrez que \(\GL{n}[\C]\) est connexe par arcs et que \(\GL{n}[\R]\) ne l'est pas. Déterminez alors les composantes connexes par arcs de \(\GL{n}[\R]\). \\
    \item Montrez que l'ensemble \(D_n\paren{\C}\) des matrices diagonalisables de \(\M{n}[\C]\) est dense dans \(\M{n}[\C]\). Montrez la même chose avec l'ensemble \(D_n^+\paren{\C}\) des matrices de \(\M{n}[\C]\) dont toutes les valeurs propres sont distinctes. Montrez que l'intérieur de \(D_n\paren{\C}\) est \(D_n^+\paren{\C}\). \\
    \item Montrez que \(D_2\paren{\R}\) n'est pas dense dans \(\M{2}[\R]\). \\
    \item Montrez que l'ensemble des matrices trigonalisables de \(\M{n}[\R]\) est un fermé et qu'il est l'adhérence de \(D_n\paren{\R}\). \\
    \item Soit \(r\in\interventierii{1}{n}\). Montrez que l'ensemble des matrices de rang inférieur à \(r\) est un fermé et qu'il est l'adhérence de l'ensemble des matrices de rang \(r\).
\end{enumerate}
\end{exoss}



\begin{exosss}[Exercice 29]
Soient \(A,B\in\M{n}\).

\begin{enumerate}
    \item Montrez que si l'une des deux matrices est inversible, alors \(AB\) et \(BA\) ont les mêmes valeurs propres avec les mêmes ordres de multiplicité. \\
    \item Montrez que ce résultat reste vrai même sans suppose l'une des deux matrices inversible.
\end{enumerate}
\end{exosss}



\begin{exo}[Oral IMT, 1]
Soient \(u,v,w\) trois suites vérifiant \[\quantifs{\forall n\in\N}\begin{dcases}
u_{n+1}=4u_n-3v_n-3w_n \\
v_{n+1}=3u_n-2v_n-3w_n \\
w_{n+1}=3u_n-3v_n-2w_n
\end{dcases}\]

Exprimez \(u_n,v_n,w_n\) en fonction de \(n,u_0,v_0,w_0\).
\end{exo}



\begin{exo}[Oral IMT, 2]
Déterminez les matrices \(M\in\M{3}[\R]\) telles que \(M^2=\begin{pmatrix}
0 & 1 & 1 \\
1 & 0 & 1 \\
1 & 1 & 0
\end{pmatrix}\).
\end{exo}



\begin{exo}[Oral St-Cyr, 3]~\\
Montrez que la matrice \(A=\begin{pmatrix}
a & 0 & b \\
0 & a+b & 0 \\
b & 0 & a
\end{pmatrix}\) est diagonalisable et donnez ses éléments propres.
\end{exo}



\begin{exo}[Oral TPE, 4]~\\
Soient \(\paren{a,b,c}\in\R^3\) et \(A=\begin{pmatrix}
0 & -a & -b \\
a & 0 & -c \\
b & c & 0
\end{pmatrix}\).

\begin{enumerate}
    \item Déterminez un polynôme annulateur de degré \(3\) de \(A\). \\
    \item La matrice \(A\) est-elle inversible ? \\
    \item Est-elle diagonalisable ? \\
    \item Montrez que les valeurs propres de \(A^2\) sont négatives ou nulles.
\end{enumerate}
\end{exo}



\begin{exo}[Oral TPE, 5]
Montrez de deux façons différentes que \(\begin{pmatrix}
0 & 1 & 2 \\
1 & 0 & 1 \\
1 & 0 & 0
\end{pmatrix}\) et \(\begin{pmatrix}
0 & 1 & 1 \\
1 & 0 & 2 \\
0 & 1 & 0
\end{pmatrix}\) sont semblables.
\end{exo}



\begin{exo}[Oral IMT, 6]
Soit \(\alpha\in\R\). Déterminez la limite quand \(n\to\pinf\) de \(\begin{pmatrix}
\cos\nicefrac{\alpha}{n} & \sin\nicefrac{\alpha}{n} \\
\sin\nicefrac{\alpha}{n} & \cos\nicefrac{\alpha}{n}
\end{pmatrix}^n\).
\end{exo}



\begin{exo}[Oral IMT, 7]
Soit \(A\in\M{n}[\C]\) telle que \(A^2-2A\) soit diagonalisable et \(1\) n'est pas valeur propre de \(A\).

Montrez que \(A\) est diagonalisable.
\end{exo}



\begin{exo}[Oral CCINP, 8]
\begin{enumerate}
    \item Soient \(M\in\M{n}[\C]\) et \(\lambda_1,\dots,\lambda_n\) ses valeurs propres. Montrez que \(\tr M^2=\sum_{i=1}^n\lambda_i^2\). \\
    \item Pour \(n\geq3\), on pose \(A\) la matrice dont tous les coefficients sont nuls, sauf ceux situés sur les quatre bords, égaux à \(1\). Déterminez les valeurs propres et les vecteurs propres de \(A\).
\end{enumerate}
\end{exo}



\begin{exo}[Oral CCINP, 9]
Soient \(\alpha\in\C\) et \(A=\paren{\alpha^{i+j-2}}_{1\leq i,j\leq n}\in\M{n}[\C]\).

\begin{enumerate}
    \item Calculez le rang de \(A\). Déduisez-en ses valeurs propres. \\
    \item À quelle condition sur \(\alpha\) la matrice \(A\) est-elle diagonalisable ?
\end{enumerate}
\end{exo}



\begin{exo}[Oral CCINP, 10]
Soient \(n\in\Ns\) et \(f\) l'endomorphisme de \(\M{n}[\R]\) tel que \(f\paren{M}=M+2\trans{M}\).

\begin{enumerate}
    \item Déterminez les valeurs et vecteurs propres de \(f\). \\
    \item L'endomorphisme \(f\) est-il diagonalisable ? Calculez sa trace et son déterminant.
\end{enumerate}
\end{exo}



\begin{exo}[Oral Navale, 11]
Soit \(M\in\M{n}[\C]\). Montrez que \(1\) est la seule valeur propre de \(M\) ssi \(\quantifs{\tpt k\in\Ns}\tr M^k=n\).
\end{exo}



\begin{exo}[Oral Navale, 12]~\\
\begin{enumerate}
    \item Diagonalisez la matrice \(\begin{pmatrix}
        1 & 4 \\
        1 & 1
    \end{pmatrix}\). \\
    \item Soient \(A\in\M{n}[\C]\), \(B=\begin{pmatrix}
        A & 4A \\
        A & A
    \end{pmatrix}\) et \(C=\begin{pmatrix}
        -A & 0 \\
        0 & 3A
    \end{pmatrix}\). Montrez que \(B\) et \(C\) sont semblables.
\end{enumerate}
\end{exo}



\begin{exo}[Oral CCINP, 13]~\\
Soient \(A\in\M{n}[\C]\) et \(B=\begin{pmatrix}
0 & A \\
I_n & 0
\end{pmatrix}\).

\begin{enumerate}
    \item Exprimez le rang de \(B\) en fonction du rang de \(A\). \\
    \item Trouvez une relation entre \(\chi_A\) et \(\chi_B\). Déduisez-en le spectre de \(B\) en fonction de celui de \(A\). \\
    \item Déterminez les dimensions des sous-espaces propres de \(B\) en fonction de celles des sous-espaces propres de \(A\). \\
    \item Montrez que \(B\) est diagonalisable ssi \(A\) est diagonalisable et inversible.
\end{enumerate}
\end{exo}



\begin{exo}[Oral IMT, 14]
Soient \(E\) un espace muni d'une base \(\paren{e_1,\dots,e_n}\), \(v\) un vecteur de \(E\) et \(f\) l'endomorphisme de \(E\) tel que \(f\paren{e_1}=\dots=f\paren{e_n}=v\).

\begin{enumerate}
    \item Quel est le rang de \(f\) ? \\
    \item Discutez de la diagonalisabilité de \(f\) en fonction du vecteur \(v\).
\end{enumerate}
\end{exo}



\begin{exo}[Oral CCINP, 15]
\begin{enumerate}[series=oral5.15]
    \item Montrez que si deux matrices \(U\) et \(V\) sont semblables, alors pour tout polynôme \(R\), \(R\paren{U}\) et \(R\paren{V}\) sont semblables.
\end{enumerate}

Soient \(A,B\) deux matrices de \(\M{n}[\C]\) telles que \(AB=BA\). On pose \(M=\begin{pmatrix}
A & B \\
0 & A
\end{pmatrix}\).

\begin{enumerate}[resume=oral5.15]
    \item Pour \(P\in\poly[\C]\), exprimez \(P\paren{M}\) en fonction de \(P\paren{A}\), \(P\prim\paren{A}\) et \(B\). \\
    \item Montrez que si \(A\) est diagonalisable et \(B\) est nulle, alors \(M\) est diagonalisable. \\
    \item Montrez la réciproque.
\end{enumerate}
\end{exo}



\begin{exo}[Oral CCMP, 16]
\begin{enumerate}
    \item Soient \(A\) et \(B\) deux matrices de \(\M{n}[\R]\) telles que \(A^2=B^2=0\) et \(\rg A=\rg B\). Montrez que \(A\) et \(B\) sont semblables. \\
    \item Le résultat subsiste-t-il avec les hypothèses \(A^3=B^3=0\) et \(\rg A=\rg B\) ?
\end{enumerate}
\end{exo}



\begin{exo}[Oral CCMP, 17]
Soient \(n\in\Ns\) et \(A,B\in\M{n}[\R]\) tels que \(\paren{AB}^n=0\). Montrez que \(\paren{BA}^n=0\).
\end{exo}



\begin{exo}[Oral CCMP, 18]
Soient \(E\) un \(\K\)-espace vectoriel de dimension finie et \(u,v\in\Lendo{E}\).

\begin{enumerate}
    \item On suppose que \(\Vect{u,v}\) contient un élément inversible. Montrez que \(\ker u\inter\ker v=\accol{0}\). \\
    \item Montrez que la réciproque est fausse. \\
    \item Montrez que si \(u\) et \(v\) commutent, alors la réciproque est vraie.
\end{enumerate}
\end{exo}



\begin{exo}[Oral CCMP, 19]
Soient \(u\in\Lendo{\R^3}\) non-nul tel que \(u^3=u^2\) et \(C\paren{u}=\accol{v\in\Lendo{\R^3}\tq u\rond v=v\rond u}\).

Montrez que \(C\paren{u}\) est un sous-espace vectoriel de \(\Lendo{\R^3}\) et déterminez sa dimension.
\end{exo}



\begin{exo}[Oral CCMP, 20]
Soient \(A\in\M{n}[\C]\) et \(\phi\) l'endomorphisme de \(\M{n}[\C]\) défini par \(\phi:M\mapsto M+\tr\paren{AM}A\).

\begin{enumerate}
    \item Étudiez la diagonalisabilité de \(\phi\). \\
    \item Calculez \(\tr\phi\) et \(\det\phi\).
\end{enumerate}
\end{exo}



\begin{exo}[Oral CCMP, 21]~\\
Pour \(c\in\R\), on pose \(A\paren{c}=\begin{pmatrix}
-c & -1 & c \\
-1 & 1-c & 1 \\
c & -1 & -c
\end{pmatrix}\).

\begin{enumerate}
    \item Déterminez les réels \(c\) tels que \(A\paren{c}\) ne pas diagonalisable. \\
    \item Soit \(d\) la plus petite de ces valeurs. Trouvez \(P\) inversible telle que \(P\inv A\paren{d}P\) soit triangulaire.
\end{enumerate}
\end{exo}



\begin{exo}[Oral CCMP, 22]~\\
\begin{enumerate}
    \item Soit \(x=\cos\dfrac{2\pi}{5}\). Déterminez une équation du second degré à coefficients rationnels dont \(x\) est racine, puis donnez les valeurs de \(x\) et \(\cos\dfrac{4\pi}{5}\). \\
    \item Soit \(A\in\M{n}[\R]\) telle que \(A^4+A^3+A^2+A+I_n=0\). On suppose que la trace de \(A\) est un rationnel. Montrez que \(4\) divise \(n\).
\end{enumerate}
\end{exo}



\begin{exo}[Oral CCMP, 23]
Soit \(A\in\M{n}[\R]\) telle que \(A^3=A+I_n\). Montrez que \(\det A>0\).
\end{exo}



\begin{exo}[Oral CCMP, 24]~\\
Soient \(A\in\M{n}\) et \(B=\begin{pmatrix}
A & A \\
0 & A
\end{pmatrix}\). Déterminez une condition nécessaire et suffisante sur \(A\) pour que \(B\) soit diagonalisable.
\end{exo}



\begin{exo}[Oral CCMP, 25]~\\
Soient \(A\in\M{n}\) et \(B=\begin{pmatrix}
I_n & 0 \\
A & A
\end{pmatrix}\). On suppose que \(B\) est diagonalisable. Montrez que \(A\) est diagonalisable et que \(I_n-A\) est inversible.
\end{exo}



\begin{exo}[Oral CCMP, 26]
Soit \(A\in\M{n}[\C]\). On suppose qu'il existe \(P\in\poly[\C]\) tel que \(P\paren{A}\) est diagonalisable et \(P\prim\paren{A}\) est inversible. Montrez que \(A\) est diagonalisable.
\end{exo}



\begin{exo}[Oral CCMP, 27]
Soit \(A=\paren{a_{i\,j}}\in\M{n}[\R]\) définie par \(a_{i\,i}=0\) et \(a_{i\,j}=i\) si \(i\not=j\).

\begin{enumerate}
    \item Montrez qu'un réel \(\lambda\) est valeur propre de \(A\) ssi \(\sum_{k=1}^n\dfrac{k}{k+\lambda}=1\). \\
    \item Montrez que \(A\) est diagonalisable. Listez ses valeurs propres avec un encadrement le plus précis possible. \\
    \item Déterminez la somme des valeurs propres de \(A\). On note \(\mu_n\) la plus grande d'entre elles. Trouvez \(C\in\R\) tel que \(\mu_n\sim Cn^2\) quand \(n\) tend vers l'infini.
\end{enumerate}
\end{exo}



\begin{exo}[Oral Centrale, 28]~\\
À quelles conditions \(A=\begin{pmatrix}
1 & 0 & 0 & 0 \\
a & 1 & 0 & 0 \\
b & c & 2 & 0 \\
d & e & f & 2
\end{pmatrix}\) est-elle diagonalisable ? Le cas échéant, diagonalisez \(A\).
\end{exo}



\begin{exo}[Oral Centrale, 29]~\\
Soient \(a,b,c\in\R\) et \(A=\begin{pmatrix}
0 & a & b & c \\
a & 0 & c & b \\
b & c & 0 & a \\
c & b & a & 0
\end{pmatrix}\).

\begin{enumerate}
    \item Donnez un vecteur propre évident de \(A\). \\
    \item Calculez le polynôme caractéristique de \(A\) et donnez son spectre. Justifiez que \(A\) est diagonalisable. \\
    \item Exprimez, lorsqu'elle existe, la matrice inverse \(A\inv\) en fonction de \(I_4\), \(A\), \(A^2\) et \(A^3\).
\end{enumerate}
\end{exo}



\begin{exo}[Oral Centrale, 30]
Soient \(A,B\in\M{n}[\C]\) telles que \(\quantifs{\tpt M\in\M{n}[\C]}\chi_{AM+B}=\chi_{AM}\).

\begin{enumerate}
    \item Montrez qu'il existe un entier \(p\in\Ns\) tel que \(B^p=0\). \\
    \item Montrez que \(BA=0\). \\
    \item Réciproquement, soient \(A,B\in\M{n}[\C]\) telles que \(B^n=BA=0\). Montrez que \(\quantifs{\tpt M\in\M{n}[\C]}\chi_{AM+B}=\chi_{AM}\).
\end{enumerate}
\end{exo}



\begin{exo}[Oral Centrale, 31]
Soient \(\theta_1,\dots,\theta_p\) des réels distincts modulo \(2\pi\) et \(m_1,\dots,m_p\) des complexes non-nuls. Le but de l'exercice est de montrer que la suite \(\paren{m_1\e{\i\theta_1n}+\dots+m_p\e{\i\theta_pn}}\) ne converge pas vers \(0\).

Par l'absurde, on suppose que \(m_1\e{\i\theta_1n}+\dots+m_p\e{\i\theta_pn}\tendqd{n\to\pinf}0\).

\begin{enumerate}
    \item On note \(M_n=\begin{pmatrix}
        \e{\i\theta_1n} & \dots & \e{\i\theta_pn} \\
        \vdots &  & \vdots \\
        \e{\i\theta_1\paren{n+p-1}} & \dots & \e{\i\theta_p\paren{n+p-1}}
    \end{pmatrix}\). Montrez que \(Y_n=M_n\tcoords{m_1}{\vdots}{m_p}\tendqd{n\to\pinf}0\). \\
    \item Montrez que \(\abs{\det M_n}\) est une constante non-nulle. \\
    \item À l'aide du théorème de Cayley-Hamilton, exprimez \(M\inv\) et trouvez une contradiction.
\end{enumerate}
\end{exo}



\begin{exo}[Oral ENS, 32]
On considère un automorphisme \(\alpha\) de \(\M{n}[\C]\) qui conserve le produit matriciel, \cad tel que \(\quantifs{\tpt A,B\in\M{n}[\C]}\alpha\paren{AB}=\alpha\paren{A}\alpha\paren{B}\).

\begin{enumerate}
    \item Montrez que \(\alpha\paren{I_n}=I_n\). \\
    \item Soit \(A\in\M{n}[\C]\) diagonalisable. Montrez que \(\alpha\paren{A}\) l'est aussi. \\
    \item On suppose que \(A\) est semblable à une matrice diagonale \(D\) à coefficients diagonaux tous distincts. Montrez que \(\alpha\paren{A}\) est, elle aussi, semblable à \(D\). \\
    \item Justifiez l'existence de \(P\in\GL{n}[\C]\) telle que \(\alpha\paren{D}=PDP\inv\) puis montrez que \(\alpha\paren{E}=PEP\inv\) pour toute matrice diagonale \(E\). \\
    \item Déterminer \(\alpha\).
\end{enumerate}
\end{exo}



\section*{Problème -- Matrices réelles sans valeur propre réelle}
\addcontentsline{toc}{section}{Problème -- Matrices réelles sans valeur propre réelle}

Dans ce problème, \(n\) désigne un entier naturel non-nul. Pour alléger les notations, on pose \(E_n=\M{n\,1}[\C]\). La lettre \(X\) est utilisée ici pour désigner une matrice-colonne, on évitera donc de l'utiliser pour désigner l'indéterminée des polynômes qui sera pour une fois notée \(x\).

Si \(\paren{a,b}\in\R^2\), on pose \(S\paren{a,b}=\begin{pmatrix}
a & -b \\
b & a
\end{pmatrix}\).

Toute matrice \(Z\) à coefficients complexes de \(\M{p\,q}[\C]\) peut s'écrire de manière unique sous la forme \(X+\i Y\) où \(X,Y\) sont deux matrices réelles de \(\M{p\,q}[\R]\). On appelle conjuguée de \(Z\) la matrice \(\conj{Z}=X-\i Y\).

D'après les propriétés de la conjugaison dans \(\C\), on en déduit que les règles de calcul sont les mêmes : \(\conj{Z_1+Z_2}=\conj{Z_1}+\conj{Z_2}\), \(\conj{\lambda Z}=\lambda\conj{Z}\) et \(\conj{Z_1.Z_2}=\conj{Z_1}.\conj{Z_2}\).

Soit \(M\in\M{n}[\R]\). On veut montrer l'équivalence entre les propositions suivante :

\begin{enumerate}
    \item[\(\paren{\alpha}\)] il existe \(\paren{a_1,b_1,\dots,a_m,b_m}\in\R^{2m}\) tel que \(M\) soit semblable dans \(\M{n}[\R]\) à la matrice diagonale par blocs \[\diag{S\paren{a_1,b_1},\dots,S\paren{a_m,b_m}}=\begin{pmatrix}
        S\paren{a_1,b_1} & 0 & \dots & 0 \\
        0 & \ddots & \ddots & \vdots \\
        \vdots & \ddots & \ddots & 0 \\
        0 & \dots & 0 & S\paren{a_m,b_m}
    \end{pmatrix}\]
    \item[\(\paren{\beta}\)] il existe un polynôme de \(\poly[\R][x]\) annulateur de \(M\) à racines simples non-réelles.
\end{enumerate}

\subsection*{Un cas particulier simple}
\addcontentsline{toc}{subsection}{Un cas particulier simple}

Soit \(\omega\) un complexe non-réel. On pose \(f:\C\to\C\) définie par \(f\paren{z}=\omega z\).

Vérifiez que \(f\) est un endomorphisme du \(\R\)-espace vectoriel \(\C\), que \(f\) a un polynôme annulateur réel à racines simples non-réelles et que sa matrice dans une base de \(\C\) bien choisie est une matrice \(S\paren{a,b}\).



\subsection*{\(\paren{\beta}\imp\paren{\alpha}\)}
\addcontentsline{toc}{subsection}{\(\paren{\beta}\imp\paren{\alpha}\)}

\begin{q}
Soit \(\paren{a,b}\in\R^2\) tel que \(b\not=0\). Montrez qu'il existe un polynôme de \(\poly[\R][x]\) annulateur de \(S\paren{a,b}\) à racines simples non-réelles.
\end{q}



\begin{q}
Montrez l'implication \(\paren{\beta}\imp\paren{\alpha}\).
\end{q}



\subsection*{\(\paren{\alpha}\imp\paren{\beta}\)}
\addcontentsline{toc}{subsection}{\(\paren{\alpha}\imp\paren{\beta}\)}

Dans cette partie, \(M\) est une matrice de \(\M{n}[\R]\) qui annule un polynôme réel à racines simples non-réelles. On appelle \(f\) l'endomorphisme de \(E_n\) de matrice \(M\) dans la base canonique de \(E_n\) (autrement dit, \(\quantifs{\tpt Z\in E_n}f\paren{Z}=MZ\)).

\begin{q}
Soient \(z\in\Cs\) et \(\lambda\in\C\) tel que \(\conj{z}=\lambda z\). Montrez qu'il existe \(\theta\in\R\) tel que \(\lambda=\e{2\i\theta}\) et \(\e{\i\theta}z\) est un réel.
\end{q}



\begin{q}
Montrez que \(n\) est pair. On note \(n=2m\).
\end{q}



\begin{q}
Montrez que si \(\lambda\) est une valeur propre de \(M\), alors \(\conj{\lambda}\) en est une aussi, puis que les sous-espaces propres \(\sep{M}{\lambda}\) et \(\sep{M}{\conj{\lambda}}\) ont la même dimension.
\end{q}



\begin{q}
Soient \(\lambda\in\C\) une valeur propre de \(M\) et \(Z\) un vecteur propre de \(M\).

\begin{enumerate}
    \item[(a)] Montrez que \(\Vect{Z,\conj{Z}}\) est un plan stable par \(f\). \\
    \item[(b)] On pose \(Z=X+\i Y\) où \(X,Y\) sont des matrices réelles. Montrez que \(\paren{X,Y}\) est une base de ce plan. \\
    \item[(c)] Montrez qu'il existe \(\paren{a,b}\in\R^2\) tel que \(b\not=0\) et l'endomorphisme induit par \(f\) dans ce plan a pour matrice \(S\paren{a,b}\) dans la base \(\paren{X,Y}\).
\end{enumerate}
\end{q}



\begin{q}
Soit \(\lambda\in\C\) une valeur propre de \(M\). On choisit une base \(\paren{Z_1,\dots,Z_p}\) de \(\sep{M}{\lambda}\) et on écrit chaque vecteur sous la forme \(Z_k=X_k+\i Y_k\) où \(X_k,Y_k\) sont réelles.

Montrez que les plans \(\paren{\Vect{X_k,Y_k}}_{1\leq k\leq p}\) sont en somme directe et que \(\sep{M}{\lambda}\oplus\sep{M}{\conj{\lambda}}=\bigoplus_{k=1}^p\Vect{X_k,Y_k}\).
\end{q}



\begin{q}
Montrez qu'il existe une base de \(E_n\) constituée de vecteurs réels dans laquelle la matrice de \(f\) est égale à une matrice diagonale par blocs \(D=\diag{S\paren{a_1,b_1},\dots,S\paren{a_m,b_m}}\).

Montrez enfin qu'il existe une matrice \(P\in\GL{n}[\R]\) telle que \(M=PDP\inv\).
\end{q}



\subsection*{Un exemple}
\addcontentsline{toc}{subsection}{Un exemple}

On pose \(M=\begin{pmatrix}
-1 & -2 & 4 & 0 \\
1 & -3 & 0 & 4 \\
-2 & 0 & 5 & -2 \\
0 & -2 & 1 & 3
\end{pmatrix}\).

\begin{q}
On écrit \(xI_4-M\) par blocs \(\paren{2,2}\) : \(xI_4-M=\begin{pmatrix}
A & -4I_2 \\
2I_2 & B
\end{pmatrix}\).

\begin{enumerate}
    \item Calculez le produit par blocs \(\begin{pmatrix}-4I_2 & A \\ B & 2I_2\end{pmatrix}\begin{pmatrix}-2I_2 & 0 \\ B & I_2\end{pmatrix}\). Déduisez-en que \(\det\paren{xI_4-M}=\det\paren{AB+8I_2}\). \\
    \item Explicitez le polynôme caractéristique de \(M\). Déterminez ses racines imaginaires pures, puis les autres racines.
\end{enumerate}
\end{q}



\begin{q}
Montrez que \(M\) annule un polynôme réel à racines simples non-réelles.

Vérifiez que les vecteurs-colonnes \(\begin{pmatrix}2 \\ 1+\i \\ 2 \\ 1+\i\end{pmatrix}\) et \(\begin{pmatrix}4 \\ 2-2\i \\ 2 \\ 1-\i\end{pmatrix}\) sont des vecteurs propres de \(M\), puis donnez une matrice \(P\in\GL{4}[\R]\) et une matrice réelle diagonale par blocs \(D\) telle que \(M=PDP\inv\).
\end{q}



\subsection*{Application}
\addcontentsline{toc}{subsection}{Application}

Dans cette partie, on suppose que \(M\in\M{n}[\R]\) annule un polynôme de \(\poly[\R][x]\) à racines simples dans \(\C\). On garde les notations introduites précédemment.

On note \(\Sp[\R]{M}\) l'ensemble des valeurs propres réelles et \(\Sp[\C\excluant\R]{M}\) l'ensemble des valeurs propres complexes non-réelles de \(M\).

\begin{q}
On pose \(G=\bigoplus_{\lambda\in\Sp[\R]{M}}\sep{M}{\lambda}\) et \(H=\bigoplus_{\lambda\in\Sp[\C\excluant\R]{M}}\sep{M}{\lambda}\). Par convention, si \(M\) n'a pas de valeur propre réelle, alors \(G=\accol{0}\) et si \(M\) n'a que des valeurs propres réelles, alors \(H=\accol{0}\).

Montrez que \(G\) et \(H\) sont supplémentaires dans \(E_n\) et qu'ils sont stables par \(f\).

On note \(g\) et \(h\) les endomorphismes induits par \(f\) dans \(G\) et \(H\) respectivement.
\end{q}



\begin{q}
Montrez qu'il existe une base \(\fami{B}\) de \(E_n\) telle que \(\Mat{f}=R=\begin{pmatrix}D & 0 \\ 0 & S\end{pmatrix}\) où \(D\) est une matrice diagonale réelle et \(S\) une matrice diagonale par blocs \(\diag{S\paren{a_1,b_1},\dots,S\paren{a_m,b_m}}\). On remarque que \(R\) est une matrice réelle.
\end{q}



\begin{q}
Un exercice classique : soit \(\paren{U,V}\in\M{n}[\R]^2\).

\begin{enumerate}
    \item Justifiez que la fonction \(x\mapsto\det\paren{U+xV}\) est une fonction polynôme. \\
    \item Montrez que s'il existe \(z\in\C\) tel que \(\det\paren{U+zV}\not=0\), alors il existe \(x\in\R\) tel que \(\det\paren{U+xV}\not=0\).
\end{enumerate}

Application : soit \(\paren{A,B}\in\M{n}[\R]^2\) tel que \(A\) et \(B\) soient semblables dans \(\M{n}[\C]\) ; montrez qu'elles sont alors semblables dans \(\M{n}[\R]\).
\end{q}



\begin{q}
Montrez finalement que \(M\) et \(R\) sont semblables dans \(\M{n}[\R]\).
\end{q}




\chapter{Intégrales généralisées}

\minitoc

\section*{Exercices}
\addcontentsline{toc}{section}{Exercices}

\legendeexercices

\begin{exos}
Montrez la convergence des intégrales suivantes et calculez leurs valeurs.

\begin{enumerate}
    \item \(\int_0^{\pinf}\dfrac{1}{\paren{x+1}\paren{x+2}\paren{x+3}}\odif{x}\) \\
    \item \(\int_0^{\pinf}\dfrac{x}{x^3+x^2+x+1}\odif{x}\) \\
    \item \(\int_0^{\pinf}\dfrac{1}{x^3+1}\odif{x}\) \\
    \item \(\int_0^{\pinf}\dfrac{x^5}{x^{12}+1}\odif{x}\) \\
    \item \(\int_0^{\pinf}\dfrac{1}{\paren{1+t^2}^2}\odif{t}\) \\
    \item \(\int_0^{\pinf}\paren{\Arctan\paren{t+1}-\Arctan t}\odif{t}\)
\end{enumerate}
\end{exos}

\begin{exos}
Justifiez la convergence des intégrales \(\int_1^{\pinf}\dfrac{1}{\sh x}\odif{x}\) et \(\int_0^{\pinf}\dfrac{1}{\ch x}\odif{x}\) et donnez leurs valeurs en vous servant de l'égalité \(\ch^2x-\sh^2x=1\).
\end{exos}

\begin{exos}
Soit \(n\in\N\).

Montrez que l'intégrale \(\int_0^{\pinf}\dfrac{1}{\paren{1+x^2}\paren{1+x^n}}\odif{x}\) converge et donnez sa valeur en faisant le changement de variable \(y=\dfrac{1}{x}\).
\end{exos}

\begin{exoss}
Montrez la convergence des intégrales suivantes et calculez leurs valeurs.

\begin{enumerate}
    \item \(\int_0^{\pinf}\ln\paren{1+\dfrac{1}{x^2}}\odif{x}\) \\
    \item \(\int_0^{\pinf}\dfrac{\ln\paren{1+x}}{x\sqrt{x}}\odif{x}\) \\
    \item \(\int_0^{\pinf}\dfrac{\ln x}{\paren{x+a}^2}\odif{x}\) (où \(a>0\)) \\
    \item \(\int_0^{\pinf}\dfrac{x\ln x}{\paren{1+x^2}^3}\odif{x}\) \\
    \item \(\int_0^1\dfrac{\ln\paren{1+3x^2}}{x^2}\odif{x}\) \\
    \item \(\int_0^1\dfrac{\ln\paren{1-x^2}}{x^2}\odif{x}\)
\end{enumerate}
\end{exoss}

\begin{exoss}
On considère l'intégrale \(I=\int_0^{\pinf}\dfrac{\sin^3t}{t^2}\odif{t}\).

\begin{enumerate}
    \item Justifiez que \(I\) est convergente. \\
    \item Démontrez que \(\quantifs{\forall t\in\intervii{0}{\dfrac{\pi}{2}}}t-\dfrac{t^3}{6}\leq\sin t\leq t\). \\ Déduisez-en \(\lim_{x\to0^+}\int_x^{3x}\dfrac{\sin t}{t^2}\odif{t}\). \\
    \item Montrez que \(\lim_{y\to\pinf}\int_y^{3y}\dfrac{\sin t}{t^2}\odif{t}=0\). \\
    \item Linéarisez \(\sin^3t\) et déduisez de tout ce qui précède la valeur de \(I\).
\end{enumerate}
\end{exoss}

\begin{exos}
Soit \(\paren{a,b}\in\R^2\) tels que \(a<b\).

\begin{enumerate}
    \item Déterminez une application affine \(\phi\) envoyant l'intervalle \(\intervee{-1}{1}\) sur \(\intervee{a}{b}\). \\
    \item Déduisez-en la nature et la valeur éventuelle de l'intégrale \(\int_a^b\dfrac{\odif{t}}{\sqrt{\paren{t-a}\paren{b-t}}}\).
\end{enumerate}
\end{exos}

\begin{exoss}~\\
On pose \(C=\int_0^{\nicefrac{\pi}{2}}\ln\paren{\cos t}\odif{t}\), \(S=\int_0^{\nicefrac{\pi}{2}}\ln\paren{\sin t}\odif{t}\) et \(I=\int_0^\pi\ln\paren{\sin t}\odif{t}\).

\begin{enumerate}
    \item Justifiez l'existence de ces trois intégrales. \\
    \item Montrez que \(C=S\). \\
    \item Montrez que \(I=2S\) et donnez une autre relation liant \(C+S\) et \(I\). \\
    \item Donnez la valeur de ces trois intégrales.
\end{enumerate}
\end{exoss}

\begin{exoss}
Justifiez l'existence des intégrales suivantes, puis par le changement de variable \(x=\sin t\), montrez que \[\int_{-1}^1\dfrac{1}{\paren{2-x^2}\sqrt{1-x^2}}\odif{x}=\int_{-\nicefrac{\pi}{2}}^{\nicefrac{\pi}{2}}\dfrac{1}{1+\cos^2t}\odif{t}.\]

Effectuez le changement de variable \(u=\tan t\) dans la dernière intégrale afin d'obtenir sa valeur.
\end{exoss}

\begin{exoss}
\begin{enumerate}
    \item Soit \(a>0\). Montrez que \(\int_0^{\pinf}\dfrac{\sin t}{t^a}\odif{t}\) converge ssi \(a<2\). \\
    \item Montrez que les intégrales suivantes convergent : \(\int_0^1\sin\paren{\dfrac{1}{t}}\odif{t}\), \(\int_0^{\pinf}\sin\paren{t^2}\odif{t}\) et \(\int_0^{\pinf}\sin\paren{\e{x}}\odif{x}\).
\end{enumerate}
\end{exoss}

\begin{exoss}
Soient \(f:\Rp\to\R\) continue telle que \(\int_1^{\pinf}\dfrac{f\paren{x}}{x}\odif{x}\) converge et \(\paren{a,b}\in\R^2\) tel que \(0<a<b\).

\begin{enumerate}
    \item Montrez que \(\quantifs{\tpt\epsilon>0}\int_\epsilon^{\pinf}\dfrac{f\paren{ax}-f\paren{bx}}{x}\odif{x}=\int_{a\epsilon}^{b\epsilon}\dfrac{f\paren{x}}{x}\odif{x}\). \\
    \item Déduisez-en que \(\int_0^{\pinf}\dfrac{f\paren{ax}-f\paren{bx}}{x}\odif{x}\) converge et vaut \(f\paren{0}\ln\dfrac{b}{a}\).
\end{enumerate}
\end{exoss}

\begin{exoss}
Soient \(f:\Rp\to\R\) continue et ayant une limite réelle \(l\) en \(\pinf\) et \(a>0\).

Montrez que \(\int_0^{\pinf}\dfrac{f\paren{x}-f\paren{ax}}{x}\odif{x}\) converge et donnez sa valeur en fonction de \(l\), \(f\paren{0}\) et \(a\).
\end{exoss}

\begin{exoss}[Fonction \(\Gamma\) d'Euler]~\\
Pour \(x>0\), on pose \(\Gamma\paren{x}=\int_0^{\pinf}t^{x-1}\e{-t}\odif{t}\).

\begin{enumerate}
    \item Montrez que \(\Gamma\) est bien définie sur \(\intervee{0}{\pinf}\). \\
    \item Donnez une relation de récurrence entre \(\Gamma\paren{x}\) et \(\Gamma\paren{x+1}\). \\ Déduisez-en la valeur de \(\Gamma\paren{n}\) pour tout \(n\in\Ns\).
\end{enumerate}
\end{exoss}

\begin{exoss}
Une idée fausse : beaucoup pensent que si \(f\) est positive et intégrable sur \(\intervie{0}{\pinf}\), alors \(f\) a pour limite \(0\) en \(\pinf\).

Soit \(f\) la fonction définie sur \(\intervie{0}{\pinf}\) de la façon suivante : pour \(n\in\Ns\), on pose \(a_n=\dfrac{1}{2n^2}\) puis, pour \(x\geq0\), \begin{itemize}
    \item s'il existe \(n\in\Ns\) tel que \(n-a_n\leq x\leq n\), alors \(f\paren{x}=2n^2\paren{x-n}+1\)
    \item s'il existe \(n\in\Ns\) tel que \(n\leq x\leq n+a_n\), alors \(f\paren{x}=-2n^2\paren{x-n}+1\)
    \item sinon, \(f\paren{x}=0\).
\end{itemize}

Représentez la courbe de \(f\) sur \(\intervii{0}{\num{4.5}}\) et montrez que \(f\) est intégrable sur \(\intervie{0}{\pinf}\) mais que \(f\) n'a pas de limite en \(\pinf\).
\end{exoss}

\begin{exoss}
Déterminez la nature des intégrales suivantes (où \(\alpha>0\)).

\begin{enumerate}
    \item \(\int_0^{\pinf}\dfrac{1}{1+\abs{\sin x}}\odif{x}\) \\
    \item \(\int_1^{\pinf}\dfrac{1}{x^\alpha\paren{1+\sin^2x}}\odif{x}\) \\
    \item \(\int_0^{\pinf}\e{-\sqrt{x}}\odif{x}\) \\
    \item \(\int_0^1\e{-\nicefrac{1}{t}}\odif{t}\) \\
    \item \(\int_0^{\pinf}\dfrac{\ln x}{x+\e{-x}}\odif{x}\) \\
    \item \(\int_0^{\pinf}\e{-\ln^2x}\odif{x}\) \\
    \item \(\int_0^1\sin\paren{\ln t}\odif{t}\) \\
    \item \(\int_0^1\dfrac{\ln t}{\paren{1-t}^{\nicefrac{3}{2}}}\odif{t}\) \\
    \item \(\int_0^{\pinf}\dfrac{\Arctan\sqrt{t}}{t\paren{1+t^2}}\odif{t}\) \\
    \item \(\int_0^{\pinf}\dfrac{\sin x}{\e{x}-1}\odif{x}\) \\
    \item \(\int_0^{\pinf}\sin\paren{t^2}\odif{t}\) \\
    \item \(\int_0^{\pinf}\sin\paren{\dfrac{1}{t}}\odif{t}\) \\
    \item \(\int_0^{\pinf}\dfrac{\e{-t}}{\sqrt{t}}\odif{t}\)
\end{enumerate}
\end{exoss}

\begin{exoss}
Montrez les comparaisons suivantes en justifiant l'existence des intégrales.

\begin{enumerate}
    \item \(\int_0^X\dfrac{\ln x}{x+\e{-x}}\odif{x}\simqd{X\to\pinf}\dfrac{1}{2}\ln^2X\) \\
    \item \(\int_x^{\pinf}\dfrac{\Arctan\sqrt{t}}{t\paren{1+t^2}}\odif{t}\simqd{x\to\pinf}\dfrac{\pi}{4x^2}\) \\
    \item \(\int_x^1\dfrac{\e{t}}{\sin t}\odif{t}\simqd{x\to0}-\ln x\) \\
    \item \(\int_0^x\dfrac{\e{-t}}{\sqrt{t}}\odif{t}\simqd{x\to0}2\sqrt{x}\) \\
    \item \(\int_x^{\pinf}\dfrac{\e{-t}}{t}\odif{t}\egqd{x\to\pinf}\o{\e{-x}}\) \\
    \item \(\int_x^{\pinf}\dfrac{\e{-t}}{t}\odif{t}\simqd{x\to0}-\ln x\) \\
    \item \(\int_1^x\dfrac{\e{t}}{t}\odif{t}\egqd{x\to\pinf}\o{\e{\nicefrac{x}{2}}}\) \\
    \item \(\int_0^x\dfrac{\ln\paren{1+t^2}}{1+\sqrt{t}}\odif{t}\simqd{x\to\pinf}2\sqrt{x}\ln x\)
\end{enumerate}
\end{exoss}

\begin{exoss}
Donnez des équivalents simples aux points indiqués des intégrales suivantes en justifiant leurs existences.

\begin{enumerate}
    \item \(\int_0^x\dfrac{\sin t}{t}\odif{t}\simqd{x\to0}?\) \\
    \item \(\int_0^x\dfrac{\sin t}{t^{\nicefrac{3}{2}}}\odif{t}\simqd{x\to0}?\) \\
    \item \(\int_x^{\pinf}\dfrac{t^2\e{-t}}{t^2+1}\odif{t}\simqd{x\to\pinf}?\) \\
    \item \(\int_0^x\ln\paren{t^2+\sin t}\odif{t}\simqd{x\to\pinf}?\)
\end{enumerate}
\end{exoss}

\begin{exoss}
Soient \(a,\alpha\) deux réels strictement positifs.

\begin{enumerate}
    \item En effectuant le changement de variable \(t=\tan x\), montrez que \[\int_0^{\nicefrac{\pi}{2}}\dfrac{1}{1+a\sin^2x}\odif{x}=\int_0^{\pinf}\dfrac{1}{1+\paren{1+a}t^2}\odif{t}\] puis calculez ces intégrales et donnez la valeur de \(\int_0^\pi\dfrac{1}{1+a\sin^2x}\odif{x}\). \\
    \item Donnez la nature de la série de terme général \(\int_0^\pi\dfrac{1}{1+\paren{t+n\pi}^\alpha\sin^2t}\odif{t}\) selon la valeur de \(\alpha\). \\
    \item Étudiez la convergence de l'intégrale \(\int_0^{\pinf}\dfrac{1}{1+x^\alpha\sin^2x}\odif{x}\).
\end{enumerate}
\end{exoss}

\begin{exoss}
Discutez, selon \(\alpha\) et \(\beta\) réels, de la nature de l'intégrale \(\int_0^{\pinf}\dfrac{\paren{t+1}^\alpha-t^\alpha}{t^\beta}\odif{t}\).
\end{exoss}

\begin{exoss}
Soit \(f\) une fonction de classe \(\classe{1}\) sur \(\intervie{0}{\pinf}\), à valeurs réelles et telle que les fonctions \(t\mapsto t^2f^2\paren{t}\) et \(t\mapsto{f\prim}^2\paren{t}\) soient intégrables sur \(\intervie{0}{\pinf}\).

\begin{enumerate}
    \item Montrez que la fonction \(t\mapsto tf\paren{t}f\prim\paren{t}\) est intégrable sur \(\intervie{0}{\pinf}\). \\
    \item Montrez que \(\quantifs{\tpt x>0}xf^2\paren{x}=\int_0^xf^2\paren{t}\odif{t}+2\int_0^xtf\paren{t}f\prim\paren{t}\odif{t}\). \\ Déduisez-en que \(xf^2\paren{x}\tendqd{x\to\pinf}0\). \\
    \item Montrez que \(t\mapsto f^2\paren{t}\) est intégrable sur \(\intervie{0}{\pinf}\). \\
    \item Démontrez que \[\paren{\int_0^{\pinf}f^2\paren{t}\odif{t}}^2\leq4\paren{\int_0^{\pinf}t^2f^2\paren{t}\odif{t}}\paren{\int_0^{\pinf}{f\prim}^2\paren{t}\odif{t}}.\]
\end{enumerate}
\end{exoss}

\begin{exo}[Oral IMT, 1]~\\
Justifiez l'existence de \(\int_0^1\dfrac{\ln\paren{t}\ln\paren{1-t}}{t}\odif{t}\).
\end{exo}

\begin{exo}[Oral IMT, 2]~\\
Soit \(a\in\R\). L'intégrale \(\int_0^{\pinf}x^a\ln\paren{1+\e{ax}}\odif{x}\) est-elle convergente ?
\end{exo}

\begin{exo}[Oral IMT, 3]~\\
Justifiez l'existence de \(\int_0^{\pinf}\paren{1-t\Arctan\dfrac{1}{t}}\odif{t}\) et donnez sa valeur.
\end{exo}

\begin{exo}[Oral CCINP, 4]~\\
Justifiez l'existence de \(\int_0^{\pinf}\dfrac{x^2}{\e{x}-1}\odif{x}\).
\end{exo}

\begin{exo}[Oral CCINP, 5]
Pour quelles valeurs de \(n\in\N\) l'intégrale \(\int_1^{\pinf}\dfrac{1}{\paren{x+\sqrt{x^2-1}}^n}\odif{x}\) converge-t-elle ? Donnez alors sa valeur.
\end{exo}

\begin{exo}[Oral IMT, 6]
Soit \(a>0\). Donnez la nature de \(\int_0^{\pinf}\dfrac{\sin^2x}{x^a}\odif{x}\).
\end{exo}

\begin{exo}[Oral CCINP, 7]~\\
Pour \(k\in\Ns\), on pose \(I_k=\int_k^{k+1}\dfrac{x-\nicefrac{1}{2}-\floor{x}}{x}\odif{x}\).

\begin{enumerate}
    \item Calculez \(I_k\). \\
    \item On pose \(J_n=\int_1^n\dfrac{x-\nicefrac{1}{2}-\floor{x}}{x}\odif{x}\). Montrez que \(J_n=n+\paren{n+\dfrac{1}{2}}\ln\paren{n+1}-\ln n!\). \\
    \item Montrez que \(\ln n!=n\ln n-n+\ln n+\dfrac{1}{2}\ln\paren{2\pi}+\o{1}\). \\
    \item Montrez que la suite \(\paren{J_n}_{n\geq1}\) converge et donnez sa valeur. \\
    \item Montrez que \(\int_1^{\pinf}\dfrac{x-\nicefrac{1}{2}-\floor{x}}{x}\odif{x}\) converge et donnez sa valeur.
\end{enumerate}
\end{exo}

\begin{exo}[Oral CCINP, 8]~\\
Pour \(n\in\Ns\), on pose \(u_n=\paren{1+\dfrac{1}{\sqrt{n}}}^{-n}\), \(v_n=\e{-\sqrt{n}}\) et \(I_n=\int_n^{\pinf}\e{\sqrt{t}}\odif{t}\).

\begin{enumerate}
    \item Montrez que la série \(\sum_nv_n\) converge. \\
    \item Montrez que \(I_n\) existe et que \(I_n=2\paren{1+\sqrt{n}}v_n\). \\
    \item On pose \(R_n=\sum_{k=n+1}^{\pinf}v_k\). Montrez que \(I_{n+1}\leq R_n\leq I_n\) et donnez un équivalent de \(R_n\). \\
    \item On pose \(T_n=\sum_{k=n+1}^{\pinf}u_k\). Montrez que \(T_n\simqd{n\to\pinf}\dfrac{R_n}{\sqrt{\e{}}}\).
\end{enumerate}
\end{exo}

\begin{exo}[Oral Centrale, 9]~\\
Soit \(I_n=\int_0^{\pinf}\dfrac{1}{\paren{1+t^3}^n}\odif{t}\) pour \(n\geq1\).

\begin{enumerate}
    \item On définit les suites \(\paren{u_n}_{n\geq1}\) et \(\paren{v_n}_{n\geq1}\) par \(u_n=\sum_{k=1}^n\dfrac{1}{k}-\ln n\) et \(v_n=\sum_{k=1}^n\dfrac{1}{k}-\ln\paren{n+1}\). \\ Montrez qu'elles convergent vers la même limite. \\
    \item Montrez que \(\quantifs{\tpt n\geq1}I_{n+1}=\paren{1-\dfrac{1}{3n}}I_n\). \\
    \item Montrez qu'il existe des réels \(a\) et \(b\) tels que \(\ln I_n=a\ln n+b+\o{1}\). \\
    \item Montrez que la série de terme général \(I_n\) converge.
\end{enumerate}
\end{exo}

\begin{exo}[Oral CCMP, 10]
Soit \(y\in\R\).

Calculez \(\int_{\minf}^{\pinf}\dfrac{1}{\paren{1+x^2}\paren{1+x^2y^2}}\odif{x}\).
\end{exo}

\begin{exo}[Oral CCMP, 11]
Soit \(y\in\R\).

Justifiez l'existence et calculez \(\int_{\minf}^{\pinf}\dfrac{1}{\paren{1+x^2}\paren{1+\i xy}}\odif{x}\).
\end{exo}

\begin{exo}[Oral CCMP, 12]
Soit \(\alpha>0\).

Étudiez la convergence de \(\int_0^{\pinf}\paren{\exp\paren{\dfrac{\sin^2t}{t^\alpha}}-1}\odif{t}\).
\end{exo}

\begin{exo}[Oral CCMP, 13]
Soient \(a\in\intervie{0}{1}\) et \(f:\Rp\to\Rps\) continue par morceaux telle que \(\dfrac{f\paren{x+1}}{f\paren{x}}\tendqd{x\to\pinf}a\).

Montrez que \(f\) est intégrable sur \(\Rp\).
\end{exo}

\section*{Problème 1 -- Calculs d'intégrales}
\addcontentsline{toc}{section}{Problème 1 -- Calculs d'intégrales}

\subsection*{Intégrale de Gauss}
\addcontentsline{toc}{subsection}{Intégrale de Gauss}

L'intégrale de Gauss est l'intégrale généralisée \(\int_{\minf}^{\pinf}\e{-t^2}\odif{t}\).

On rappelle un résultat à propos des intégrales de Wallis : en notant \(W_n=\int_0^{\nicefrac{\pi}{2}}\paren{\cos\theta}^n\odif{\theta}\), on a \(W_n\simqd{n\to\pinf}\sqrt{\dfrac{\pi}{2n}}\).

\setcounter{q}{0}
\begin{q}
Justifiez que l'intégrale \(\int_{\minf}^{\pinf}\e{-t^2}\odif{t}\) est convergente.
\end{q}

\begin{q}
Montrez que \(\quantifs{\forall n\in\Ns;\forall t\in\intervii{0}{\sqrt{n}}}\paren{1-\dfrac{t^2}{n}}^n\leq\e{-t^2}\leq\paren{1+\dfrac{t^2}{n}}^{-n}\).
\end{q}

\begin{q}
Soit \(n\in\Ns\).

À l'aide du changement de variable \(t=\sqrt{n}\sin\theta\), exprimez l'intégrale \(\int_0^{\sqrt{n}}\paren{1-\dfrac{t^2}{n}}^{n}\odif{t}\) en fonction d'une intégrale de Wallis.
\end{q}

\begin{q}
Avec le changement de variable \(t=\sqrt{n}\tan\theta\), établissez que \[\quantifs{\forall n\in\Ns}\int_0^{\sqrt{n}}\paren{1+\dfrac{t^2}{n}}^{-n}\odif{t}\leq\sqrt{n}W_{2n-2}.\]

Déduisez-en que \(\quantifs{\forall n\in\Ns}\sqrt{n}W_{2n+1}\leq\int_0^{\sqrt{n}}\e{-t^2}\odif{t}\leq\sqrt{n}W_{2n-2}\).
\end{q}

\begin{q}
Déterminez enfin la valeur de l'intégrale de Gauss.
\end{q}

\subsection*{Des intégrales avec des logarithmes}
\addcontentsline{toc}{subsection}{Des intégrales avec des logarithmes}

Pour \(n\in\N\), on pose \(u_n=\int_0^1t^n\ln\paren{t}\odif{t}\) et \(v_n=\int_0^1t^n\dfrac{\ln t}{1+t}\odif{t}\).

\begin{q}
\begin{enumerate}
    \item Justifiez l'existence de \(u_n\) et \(v_n\), puis montrez que \(u_n=\dfrac{-1}{\paren{n+1}^2}\). \\
    \item Montrez que \(\abs{v_n}\leq-u_n\) et déduisez-en la limite de \(v_n\) quand \(n\) tend vers \(\pinf\).
\end{enumerate}
\end{q}

\begin{q}
\begin{enumerate}
    \item Montrez que la série \(\sum_{n\geq0}\paren{-1}^nu_n\) converge, puis que \(\sum_{k=0}^n\paren{-1}^ku_k=v_0+\paren{-1}^nv_{n+1}\). \\
    \item On rappelle que \(\sum_{k=1}^{\pinf}\dfrac{1}{k^2}=\dfrac{\pi^2}{6}\). Montrez que \(v_0=-\dfrac{\pi^2}{12}\).
\end{enumerate}
\end{q}

\begin{q}
Montrez que l'intégrale \(\int_0^1\dfrac{\ln\paren{1+t}}{t}\odif{t}\) converge et donnez sa valeur grâce à ce qui précède.
\end{q}

\section*{Problème 2 -- Des inégalités entre intégrales}
\addcontentsline{toc}{section}{Problème 2 -- Des inégalités entre intégrales}

\subsection*{Partie 1}
\addcontentsline{toc}{subsection}{Partie 1}

Soit \(f:\intervii{0}{1}\to\R\) de classe \(\classe{1}\) telle que \(f\paren{0}=f\paren{1}=0\).

\setcounter{q}{0}
\begin{q}
Montrez l'existence des intégrales suivantes et justifiez l'égalité \[\int_0^1f\paren{x}f\prim\paren{x}\cotan\paren{\pi x}\odif{x}=\dfrac{\pi}{2}\int_0^1\dfrac{f^2\paren{x}}{\sin^2\paren{\pi x}}\odif{x}.\]
\end{q}

\begin{q}
Déduisez-en \(\int_0^1{f\prim}^2-\pi^2\int_0^1f^2=\int_0^1\paren{f\prim\paren{x}-\pi f\paren{x}\cotan\paren{\pi x}}^2\odif{x}\).
\end{q}

\begin{q}
Concluez : \(\int_0^1{f\prim}^2\geq\pi^2\int_0^1f^2\). Dans quel cas y a-t-il égalité ?
\end{q}

\subsection*{Partie 2}
\addcontentsline{toc}{subsection}{Partie 2}

Soit \(f:\Rp\to\R\) de classe \(\classe{2}\) telle que \(f\paren{0}=0\) et l'intégrale \(\int_0^{\pinf}{f\prim}^2\) converge.

On pose \(g:x\mapsto\dfrac{f\paren{x}}{\sqrt{x}}\).

\begin{q}
\begin{enumerate}
    \item Montrez que \(g\) est prolongeable par continuité en \(0\). \\
    \item Justifiez que \(g\) est de classe \(\classe{2}\) sur \(\intervee{0}{\pinf}\) et montrez que \[\quantifs{\forall x>0}{f\prim}^2\paren{x}=x{g\prim}^2\paren{x}+g\paren{x}g\prim\paren{x}+\dfrac{1}{4x}g^2\paren{x}.\]
\end{enumerate}
\end{q}

\begin{q}
Montrez que les intégrales \(\int_0^{\pinf}x{g\prim}^2\paren{x}\odif{x}\) et \(\int_0^{\pinf}\dfrac{f^2\paren{x}}{x^2}\odif{x}\) sont convergentes et que la fonction \(g^2\) a une limite réelle en \(\pinf\).
\end{q}

\begin{q}
Montrez que cette limite est nulle et déduisez-en l'inégalité \[\int_0^{\pinf}\paren{\dfrac{f\paren{x}}{x}}^2\odif{x}\leq4\int_0^{\pinf}{f\prim}^2\paren{x}\odif{x}.\]
\end{q}

\begin{q}
Étudiez le cas d'égalité.
\end{q}

\subsection*{Partie 3}
\addcontentsline{toc}{subsection}{Partie 3}

Soit \(f:\R\to\R\) de classe \(\classe{2}\) telle que les intégrales \(\int_{\minf}^{\pinf}f^2\) et \(\int_{\minf}^{\pinf}{f\seconde}^2\) convergent.

On rappelle l'inégalité de Cauchy-Schwarz : si \(u,v\) sont deux fonctions continues sur \(\intervii{a}{b}\), alors \[\paren{\int_a^buv}^2\leq\int_a^bu^2\times\int_a^bv^2.\]

\begin{q}
\begin{enumerate}
    \item Montrez que l'intégrale \(\int_{\minf}^{\pinf}ff\seconde\) est absolument convergente. \\
    \item Pour \(\paren{a,b}\in\R^2\), donnez une relation entre \(\int_a^b{f\prim}^2\) et \(\int_a^bff\seconde\).
\end{enumerate}
\end{q}

\begin{q}
On suppose que l'intégrale \(\int_0^{\pinf}{f\prim}^2\) diverge.

\begin{enumerate}
    \item Montrez que \(f\paren{x}f\prim\paren{x}\tendqd{x\to\pinf}\pinf\). Déduisez-en que \(f^2\paren{x}\tendqd{x\to\pinf}\pinf\). \\
    \item Est-ce possible ? Que pensez-vous de la nature de l'intégrale \(\int_{\minf}^0{f\prim}^2\) .
\end{enumerate}
\end{q}

\begin{q}
\begin{enumerate}
    \item Montrez que la fonction \(ff\prim\) a des limites réelles a des limites réelles en \(\pinf\) et en \(\minf\), puis que ces limites sont nulles. \\
    \item Justifiez finalement l'inégalité \[\paren{\int_{\minf}^{\pinf}{f\prim}^2}^2\leq\int_{\minf}^{\pinf}f^2\times\int_{\minf}^{\pinf}{f\seconde}^2.\]
\end{enumerate}
\end{q}

\begin{q}
Étudiez le cas d'égalité dans l'inégalité précédente.
\end{q}


\chapter{Intégrales à paramètre}

\minitoc

\section*{Exercices}
\addcontentsline{toc}{section}{Exercices}

\begin{exos}
Déterminez les limites suivantes :

\begin{enumerate}
    \item \(\lim_{n\to\pinf}\int_0^{\pinf}\dfrac{1}{\paren{1+t^2}^n}\odif{t}\) \\
    \item \(\lim_{n\to\pinf}\int_0^{\pinf}\dfrac{t^n}{1+t^{n+2}}\odif{t}\) \\
    \item \(\lim_{n\to\pinf}\int_0^{\pinf}\dfrac{\odif{x}}{x^n+\e{x}}\)
\end{enumerate}
\end{exos}

\begin{exoss}
Déterminez les limites suivantes :

\begin{enumerate}
    \item \(\lim_{n\to\pinf}\int_0^{\pinf}\paren{1+\dfrac{t}{2n}}^n\e{-t}\odif{t}\) \\
    \item \(\lim_{n\to\pinf}\int_{\minf}^{\pinf}\paren{1+\dfrac{t^2}{n}}^{-n}\odif{t}\) \\
    \item \(\lim_{n\to\pinf}\int_0^1\dfrac{1+nx}{\paren{1+x}^n}\odif{x}\)
\end{enumerate}
\end{exoss}

\begin{exoss}
Pour tout \(n\in\N\), on pose \(I_n=\int_0^1\sin\paren{t^n}\odif{t}\).

\begin{enumerate}
    \item Établissez l'inégalité suivante : \(\quantifs{\forall u\in\R}\abs{\sin u}\leq\abs{u}\). \\
    \item Déduisez-en la limite de la suite \(\paren{I_n}_{n\geq0}\) puis retrouvez ce résultat par une autre méthode. \\
    \item À l'aide du changement de variable \(t=u^{\nicefrac{1}{n}}\), montrez que \[I_n\simqd{n\to\pinf}\dfrac{J}{n}\] où \(J\) est une intégrale qu'on ne cherchera pas à calculer.
\end{enumerate}
\end{exoss}

\begin{exoss}
\begin{enumerate}
    \item Soit \(f:\intervii{0}{1}\to\R\) continue. Montrez que \(\int_0^1f\paren{t^n}\odif{t}\tendqd{n\to\pinf}f\paren{0}\). \\
    \item Cherchez un équivalent de \(\int_0^1\dfrac{t^n}{1+t^n}\odif{t}\). \\
    \item Cherchez un équivalent de \(-1+\int_0^1\sqrt{1+t^n}\odif{t}\).
\end{enumerate}
\end{exoss}

\begin{exoss}
Soit \(x\in\intervii{0}{n}\).

Montrez que \(\paren{1-\dfrac{x}{n}}^n\leq\e{-x}\).

Déduisez-en \(\lim_{n\to\pinf}\int_0^n\paren{1-\dfrac{x}{n}}^nx^p\odif{x}\) pour \(p\in\N\).
\end{exoss}

\begin{exoss}
En utilisant notamment un changement de variable et le théorème de convergence dominée, démontrez que \[\int_1^{\pinf}\e{-x^n}\odif{x}\simqd{n\to\pinf}\dfrac{K}{n}\] où \(K=\int_1^{\pinf}\dfrac{\e{-t}}{t}\odif{t}\).
\end{exoss}

\begin{exos}
En développant en série \(\dfrac{1}{1-t}\), démontrez que \[\int_0^1\dfrac{\ln t}{1-t}\odif{t}=-\sum_{n=1}^{\pinf}\dfrac{1}{n^2}.\]
\end{exos}

\begin{exoss}~\\
Montrez que \(\int_0^{\pinf}\dfrac{\sin t}{\e{t}-1}\odif{t}=\sum_{n=1}^{\pinf}\dfrac{1}{n^2+1}\).
\end{exoss}

\begin{exoss}
Montrez que la série \(\sum_n\paren{-1}^n\int_0^{\nicefrac{\pi}{2}}\cos\paren{t}^n\odif{t}\) converge et donnez la valeur de sa somme.
\end{exoss}

\begin{exoss}
Soit \(\paren{a_n}\) une suite telle que la série \(\sum_na_n\) converge absolument.

\begin{enumerate}
    \item Montrez que pour tout \(x\in\R\), la série \(\sum_n\dfrac{a_n}{n!}x^n\) est convergente. \\
    \item On pose \(f\paren{x}=\sum_{n=0}^{\pinf}\dfrac{a_n}{n!}x^n\e{-x}\). On admettra momentanément que \(f\) est continue sur \(\R\). \\ Montrez que \(f\) est intégrable sur \(\intervie{0}{\pinf}\) et que \(\int_0^{\pinf}f\paren{x}\odif{x}=\sum_{n=0}^{\pinf}a_n\).
\end{enumerate}
\end{exoss}

\begin{exoss}
Montrez que la fonction \(S:x\mapsto\sum_{n=1}^{\pinf}\dfrac{\paren{-1}^n}{1+n^2x^2}\) est définie sur \(\Rps\).

On admettra momentanément que \(S\) est continue sur \(\R\).

Prouvez l'existence de \(\int_0^{\pinf}S\paren{x}\odif{x}\) et calculez la.
\end{exoss}

\begin{exoss}~\\
On admet que \(\int_0^{\pinf}\e{-t^2}\odif{t}=\dfrac{\sqrt{\pi}}{2}\) (voir un exercice ci-dessous).

Montrez que \(\quantifs{\tpt x\in\intervee{-1}{1}}\dfrac{2x}{\sqrt{\pi}}\int_0^{\pinf}\dfrac{1}{\e{t^2}-x}\odif{t}=\sum_{n=1}^{\pinf}\dfrac{x^n}{\sqrt{n}}\).
\end{exoss}

\begin{exoss}
Existence et calcul de \(\int_0^{\pinf}\dfrac{x}{\sh x}\odif{x}\) (on admet que \(\sum_{n=1}^{\pinf}\dfrac{1}{n^2}=\dfrac{\pi^2}{6}\)).
\end{exoss}

\begin{exoss}[Intégrale de Gauss]
Le but de cet exercice est de déterminer la valeur de l'intégrale de Gauss \[I=\int_0^{\pinf}\e{-t^2}\odif{t}.\]

Pour tout \(x\in\Rp\), on pose \[g\paren{x}=\int_0^1\dfrac{\e{-x^2\paren{1+t^2}}}{1+t^2}\odif{t}\qquad\text{et}\qquad h\paren{x}=\int_0^\pi\e{-t^2}\odif{t}.\]

\begin{enumerate}
    \item Montrez que \(g\) et \(h\) sont de classe \(\classe{1}\) sur \(\Rp\) et calculez leurs dérivées. \\
    \item Montrez que \(g+h^2\) est une fonction constante sur \(\Rp\), constante que l'on déterminera. \\
    \item Montrez que \(\lim_{x\to\pinf}g\paren{x}=0\). \\
    \item En déduire que \(I=\dfrac{\sqrt{\pi}}{2}\).
\end{enumerate}
\end{exoss}

\begin{exoss}[Fonction caractéristique de la loi normale centrée réduite]
On cherche à calculer explicitement la fonction \(\phialt\) définie par \[\quantifs{\forall t\in\R}\phialt\paren{t}=\dfrac{1}{\sqrt{2\pi}}\int_{\minf}^{\pinf}\e{\i tx}\e{-\nicefrac{x^2}{2}}\odif{x}.\]

NB : il s'agit de la fonction caractéristique pour la loi normale centrée réduite. Pour les variables aléatoires continues, la fonction caractéristique joue un rôle analogue à la fonction génératrice pour les variables discrètes.

\begin{enumerate}
    \item Montrez que \(\phialt\) est de classe \(\classe{1}\) sur \(\R\) et donnez une expression de sa dérivée. \\
    \item Montrez que \(\phialt\) est solution d'une équation différentielle linéaire d'ordre 1 que l'on précisera. \\
    \item À l'aide de l'intégrale de Gauss, calculez \(\phialt\paren{0}\). \\
    \item Montrez enfin que \(\quantifs{\forall t\in\R}\phialt\paren{t}=\e{-\nicefrac{t^2}{2}}\).
\end{enumerate}
\end{exoss}

\begin{exoss}
On pose, pour tout réel \(a\) : \(F\paren{a}=\int_0^1\dfrac{t^2}{\sqrt{1+a^4t^2}}\odif{t}\).

\begin{enumerate}
    \item Montrez que la fonction \(F\) est bien définie et continue sur \(\R\). \\
    \item Déduisez-en la valeur de \(\lim_{a\to0}\int_0^1\dfrac{t^2}{\sqrt{1+a^4t^2}}\odif{t}\).
\end{enumerate}
\end{exoss}

\begin{exoss}
On considère la fonction \(F:x\mapsto\int_0^{\pinf}\dfrac{\sin\paren{xt}}{t}\e{-t}\odif{t}\).

\begin{enumerate}
    \item Déterminez le domaine de définition de \(F\). \\
    \item Montrez que \(F\) est de classe \(\classe{1}\) sur son domaine de définition. \\
    \item Calculez \(F\prim\) et déduisez-en \(F\).
\end{enumerate}
\end{exoss}

\begin{exoss}
Démontrez que la fonction \(F\) définie par \[\quantifs{\forall p\geq0}F\paren{p}=\int_0^{\pinf}\dfrac{\e{-pt}}{1+t^2}\odif{t}\] est continue sur \(\intervie{0}{\pinf}\) et de classe \(\classe{\infty}\) sur \(\intervee{0}{\pinf}\).
\end{exoss}

\begin{exoss}
Soient \(\paren{a,b}\in\R^2\) tel que \(a>0\). On pose \[g\paren{x}=\int_0^{\pinf}\dfrac{\e{-xt}-\e{-at}}{t}\cos\paren{bt}\odif{t}.\]

\begin{enumerate}
    \item Justifiez que \(g\) est bien définie sur \(\Rps\). \\
    \item Justifiez que \(g\) est de classe \(\classe{1}\) sur \(\Rps\). \\
    \item Calculez \(g\prim\) puis \(g\).
\end{enumerate}
\end{exoss}

\begin{exoss}[Compléments sur la fonction Gamma d'Euler]
On rappelle la définition de la fonction Gamma d'Euler : \[\quantifs{\forall x>0}\Gamma\paren{x}=\int_0^{\pinf}t^{x-1}\e{-t}\odif{t}.\]

Démontrez la relation suivante : \[\quantifs{\forall x>1}\Gamma\paren{x}\zeta\paren{x}=\int_0^{\pinf}\dfrac{t^{x-1}}{\e{t}-1}\odif{t},\] où \(\zeta\) est la fonction zêta de Riemann vérifiant \(\zeta\paren{x}=\sum_{n=1}^{\pinf}n^{-x}\) pour tout réel \(x>1\).
\end{exoss}

\begin{exoss}[Transformée de Fourier]
Soit \(f:\R\to\C\) une fonction continue par morceaux et intégrable sur \(\R\). On définit la transformée de Fourier de \(f\), notée \(\hat{f}\), par : \[\quantifs{\forall\omega\in\R}\hat{f}\paren{\omega}=\int_{\minf}^{\pinf}f\paren{x}\e{-\i\omega x}\odif{x}.\]

\begin{enumerate}
    \item Montrez que la fonction \(\hat{f}\) est bien définie et continue sur \(\R\). \\
    \item On suppose de plus que \(f\) est de classe \(\classe{1}\) et que \(f\prim\) est intégrable sur \(\R\). \\ \begin{enumerate}
        \item Montrez que \(\lim_{\pm\infty}f=0\). \\
        \item Montrez que \(\quantifs{\forall\omega\in\R}\widehat{f\prim}\paren{\omega}=\i\omega\hat{f}\paren{\omega}\).
    \end{enumerate}
\end{enumerate}
\end{exoss}

\begin{exoss}~\\
On pose \(g\paren{x}=\int_0^1\dfrac{t^x-1}{\ln t}\odif{t}\).

\begin{enumerate}
    \item \begin{enumerate}
        \item Déterminez le domaine de définition \(\fami{D}\) de \(g\). \\
        \item Montrez que \(g\) est de classe \(\classe{1}\) sur \(\fami{D}\) et calculez sa dérivée. \\
        \item Déduisez-en l'expression de \(g\). \\
    \end{enumerate}
    \item On pose, pour tout \(\paren{\alpha,\beta}\in\R^2\) : \(I\paren{\alpha,\beta}=\int_0^1\dfrac{t^\alpha-t^\beta}{\ln t}\odif{t}\). \\ \begin{enumerate}
        \item Pour quelles valeurs de \(\paren{\alpha,\beta}\) cette intégrale est-elle convergente ? \\
        \item Lorsqu'elle converge, pratiquez le changement de variable \(u\mapsto u^\gamma\) dans cette intégrale. Déduisez-en la valeur de \(I\paren{\alpha,\beta}\).
    \end{enumerate}
\end{enumerate}
\end{exoss}

\begin{exoss}
On s'intéresse au comportement de la fonction \(F\) de l'exercice 18 au voisinage de \(0\).

\begin{enumerate}
    \item Montrez que pour tout \(p>0\), \(F\prim\paren{p}\) se met sous la forme \[F\prim\paren{p}=\int_0^{\pinf}G\paren{p,u}\e{-u}\odif{u},\] où \(G\paren{p,u}\) est une fonction rationnelle des variables \(p\) et \(u\) que l'on précisera. \\
    \item Conjecturez la limite quand \(p\to0^+\) de \(F\prim\paren{p}\) puis démontrez la. \\
    \item La fonction \(F\) est-elle dérivable en \(0\) ? Donnez une interprétation graphique du résultat.
\end{enumerate}
\end{exoss}

\begin{exoss}[Prolongement \(\classe{\infty}\)]
Soit \(f\) une fonction de classe \(\classe{\infty}\) sur \(\R\) telle que \(f\paren{0}=0\).

\begin{enumerate}
    \item Montrez que \(\quantifs{\forall x\in\Rs}\dfrac{f\paren{x}}{x}=\int_0^1f\prim\paren{tx}\odif{t}\). \\
    \item Déduisez-en que \(g:x\mapsto\dfrac{f\paren{x}}{x}\) se prolonge en une fonction de classe \(\classe{\infty}\) sur \(\R\).
\end{enumerate}
\end{exoss}

\begin{exoss}
On donne la valeur de l'intégrale de Dirichlet : \[\int_0^{\pinf}\dfrac{\sin u}{u}\odif{u}=\dfrac{\pi}{2}.\]

Calculez, pour tout réel \(x\) positif : \[g\paren{x}=\int_0^{\pinf}\dfrac{\sin\paren{xt}}{t\paren{1+t^2}}\odif{t}.\]
\end{exoss}

\begin{exoss}
Soit \(f\in\ensclasse{0}{\intervii{0}{1}}{\R}\).Démontrez que \[\lim_{h\to0^+}\int_0^1\dfrac{h}{h^2+x^2}f\paren{x}\odif{x}=\dfrac{\pi}{2}f\paren{0}.\]

\textit{Indication : on pourra effectuer le changement de variable \(x=ht\).}
\end{exoss}

\begin{exo}[Oral CCINP, 1]
\begin{enumerate}
    \item Montrez que \(\quantifs{\tpt n\in\Ns}f_n:x\mapsto\paren{\ch x}^{-n}\) est intégrable sur \(\Rp\). \\
    \item Calculez la limite de \(I_n=\int_0^{\pinf}f_n\) quand \(n\to\pinf\). \\
    \item Donnez la nature des séries \(\sum_n\paren{-1}^nI_n\) et \(\sum_nI_n\) \textit{(indication : montrez que \(\quantifs{\tpt x\in\Rp}\ch x\geq\sh x\)).} \\
    \item Donnez le rayon de convergence de la série entière \(\sum_nI_nx^n\).
\end{enumerate}
\end{exo}

\begin{exo}[Oral Navale, 2]~\\
On pose, pour \(n\geq2\) : \(I_n=\int_0^{\pinf}\dfrac{1+t^n}{\sqrt{t}+t^{2n}}\odif{t}\).

Montrez que \(\paren{I_n}\) est bien définie et calculez sa limite.
\end{exo}

\begin{exo}[Oral CCINP, 3]~\\
Pour \(n\in\N\), on pose \(I_n=\int_0^1\dfrac{1}{1+t^n}\odif{t}\) et \(J_n=\int_0^1\ln\paren{1+t^n}\odif{t}\).

\begin{enumerate}
    \item Donnez une relation entre \(I_n\) et \(J_n\) (on pourra calculer \(n\paren{1-I_n}\)). \\
    \item Déduisez-en un développement asymptotique de \(I_n\) avec une précision de \(\dfrac{1}{n}\). \\
    \item Montrez que l'application \(F:u\in\intervii{0}{1}\mapsto\int_0^u\dfrac{\ln\paren{1+t}}{t}\odif{t}\) est bien définie, puis montrez que \(\int_0^1F\paren{t^n}\odif{t}\tendqd{n\to\pinf}0\). \\
    \item Déduisez-en un développement asymptotique de \(J_n\) avec une précision \(\dfrac{1}{n}\), puis un développement de \(I_n\) en \(\dfrac{1}{n^2}\).
\end{enumerate}
\end{exo}

\begin{exo}[Oral ENSEA, 4]~\\
Soit \(\Phi:t\mapsto\int_0^\pi\cos\paren{t\sin\theta}\odif{\theta}\).

Montrez que \(\Phi\) admet une unique racine \(z\in\intervii{0}{\pi}\) et que \(z>\dfrac{\pi}{2}\).
\end{exo}

\begin{exo}[Oral IMT, 5]~\\
Soit \(F:x\mapsto\int_0^{\pinf}\dfrac{\e{-xt^2}}{1+t}\odif{t}\).

\begin{enumerate}
    \item Montrez que pour tout \(x>0\), l'intégrale \(F\paren{x}\) est convergente. \\
    \item Étudiez les variations de \(F\). \\
    \item Montrez que \(F\) est de classe \(\classe{1}\) sur \(\intervee{0}{\pinf}\). \\
    \item Montrez que \(\quantifs{\tpt x>0}F\paren{x}\geq\dfrac{1}{\e{}}\int_0^{\nicefrac{1}{\sqrt{x}}}\dfrac{\odif{t}}{1+t}\). Déduisez-en la limite de \(F\) en \(0\).
\end{enumerate}
\end{exo}

\begin{exo}[Oral ENSEA, 6]
\begin{enumerate}
    \item Déterminez le domaine de définition de \(f:x\mapsto\int_0^{\pinf}\dfrac{\sh t}{t}\e{-xt}\odif{t}\). \\
    \item Calculez \(f\prim\). \\
    \item Déterminez la limite de \(f\) en \(\pinf\). Déduisez-en \(f\).
\end{enumerate}
\end{exo}

\begin{exo}[Oral CCINP, 7]~\\
Soit \(f:x\mapsto\int_0^{\pinf}\dfrac{t\e{-xt}}{\e{t}-1}\odif{t}\).

\begin{enumerate}
    \item Donnez le domaine de définition de \(f\). \\
    \item Calculez la limite de \(f\) en \(\pinf\). \\
    \item Pour \(x>0\), calculez \(f\paren{x-1}-f\paren{x}\). \\
    \item Déterminez une expression de \(f\paren{x}\) sous forme de série. \\
    \item Quelle autre méthode aurait-on pu utiliser pour trouver cette expression de \(f\paren{x}\) ?
\end{enumerate}
\end{exo}

\begin{exo}[Oral CCINP, 8]~\\
Soit \(f\paren{x}=\int_0^{\pinf}\dfrac{1-\cos t}{t^2}\e{-xt}\odif{t}\).

\begin{enumerate}
    \item Montrez que \(f\paren{x}\) existe pour \(x\geq0\). \\
    \item Montrez que \(f\) est continue sur \(\Rp\). \\
    \item Montrez que \(f\) est de classe \(\classe{2}\) sur \(\intervee{0}{\pinf}\). \\
    \item Déterminez les limites de \(f\) et \(f\prim\) en \(\pinf\). \\
    \item Calculez \(f\prim\paren{x}\) et \(f\paren{x}\). \\
    \item Justifiez l'existence et calculez \(\int_0^{\pinf}\dfrac{\sin t}{t}\odif{t}\).
\end{enumerate}
\end{exo}

\begin{exo}[Oral CCINP, 9]~\\
On pose \(f:x\mapsto\int_0^{\pinf}t^x\e{-t}\odif{t}\).

\begin{enumerate}
    \item Montrez que \(f\) est définie et continue sur \(\Rp\). \\
    \item Pour \(n\in\Ns\), on pose \(u_n=\int_{n-1}^n\ln\paren{f\paren{t}}\odif{t}\). \\ Déterminez la nature de la série \(\sum_n\paren{-1}^nu_n\).
\end{enumerate}
\end{exo}

\begin{exo}[Oral CCMP, 10]
Soit \(f\) une fonction de classe \(\classe{1}\) de \(\R\) dans \(\R\). On pose \(g\paren{x}=\int_0^1f\paren{xt}\ln\paren{t}\odif{t}\).

\begin{enumerate}
    \item Montrez que \(g\) est définie sur \(\R\) et calculez \(g\paren{0}\). \\
    \item Montrez que \(g\) est de classe \(\classe{1}\) sur \(\R\) et calculez \(g\prim\paren{0}\).
\end{enumerate}
\end{exo}

\begin{exo}[Oral CCMP, 11]~\\
Soit \(f:x\mapsto\int_0^{\pinf}\dfrac{1}{\paren{1+t^2}\paren{1+t^x}}\odif{t}\).

\begin{enumerate}
    \item Montrez que \(f\) est définie sur \(\Rp\). Calculez \(f\paren{0}\) et \(\lim_{\pinf}f\). \\
    \item Soit \(x\in\Rp\). Calculez \(f\paren{x}\).
\end{enumerate}
\end{exo}

\begin{exo}[Oral CCMP, 12]~\\
Soit \(f:x\mapsto\int_0^{\pinf}\e{-t^2}\cos\paren{xt}\odif{t}\).

\begin{enumerate}
    \item Montrez que \(f\) est définie sur \(\R\). \\
    \item Montrez que \(f\) est solution d'une équation différentielle. \\
    \item Déduisez-en \(f\).
\end{enumerate}
\end{exo}

\begin{exo}[Oral CCMP, 13]~\\
Soit \(f:x\mapsto\int_0^1\ln\paren{t}\ln\paren{1-t^x}\odif{t}\).

\begin{enumerate}
    \item Déterminez le domaine de définition de \(f\). \\
    \item Écrivez \(f\) comme somme d'une série de fonctions. \\
    \item Déterminez la limite de \(f\) en \(0\).
\end{enumerate}
\end{exo}

\begin{exo}[Oral CCMP, 14]~\\
Soit \(f:x\mapsto\dfrac{1}{x}\int_0^{\pinf}\dfrac{1-\e{-tx}}{1+t^2}\odif{t}\).

\begin{enumerate}
    \item Montrez que \(f\) est de classe \(\classe{2}\) sur \(\Rp\). \\
    \item Trouvez un équivalent simple de \(f\) en \(\pinf\). \\
    \item Trouvez un équivalent simple de \(f\) en \(0\).
\end{enumerate}
\end{exo}

\begin{exo}[Oral CCMP, 15]
Soit \(\Gamma\) la fonction de la variable réelle \(x\) définie par \(\Gamma\paren{x}=\int_0^{\pinf}t^{x-1}\e{-t}\odif{t}\).

\begin{enumerate}
    \item Déterminez le domaine de définition de \(\Gamma\). \\
    \item Donnez un équivalent en \(\pinf\) de \(f\paren{x}=\int_x^{x+1}\ln\paren{\Gamma\paren{u}}\odif{u}\). \\
    \item Déduisez-en un équivalent en \(\pinf\) de \(\ln\paren{\Gamma\paren{x}}\).
\end{enumerate}
\end{exo}

\begin{exo}[Oral CCMP, 16]~\\
Soit \(f:x\mapsto\int_0^1\e{t^x\ln t}\odif{t}\).

\begin{enumerate}
    \item Montrez que \(f\) est définie sur \(\R\). \\
    \item Montrez que \(f\) est croissante et continue sur \(\R\). \\
    \item Donnez une expression de \(f\paren{x}\) comme somme de série pour \(x>0\).
\end{enumerate}
\end{exo}

\begin{exo}[Oral CCMP, 17]
Soit \(\paren{a_n}\) une suite de réels strictement positifs, croissante et de limite \(\pinf\).

Montrez l'égalité \[\int_0^{\pinf}\sum_{n=0}^{\pinf}\paren{-1}^n\e{-a_nt}\odif{t}=\sum_{n=0}^{\pinf}\paren{-1}^n\dfrac{1}{a_n}.\]
\end{exo}

\begin{exo}[Oral CCMP, 18]
Soit \(f:\intervee{-1}{\pinf}\to\R\) définie par \(f\paren{x}=\int_0^1\dfrac{1-\paren{1-t^x}}{t}\odif{t}\).

Justifiez la définition de \(f\) et donnez une expression de \(f\paren{x}\) comme somme d'une série.
\end{exo}

\begin{exo}[Oral CCMP, 19]~\\
Soit \(T:x\mapsto\int_0^{\pinf}\dfrac{\e{\i tx}-1}{t}\e{-t}\odif{t}\).

Montrez que \(T\) est définie sur \(\R\) et calculez \(T\paren{x}\) pour tout \(x\in\R\).
\end{exo}

\section*{Problème 1 -- CCINP 2013 -- PC}
\addcontentsline{toc}{section}{Problème 1 -- CCINP 2013 -- PC}

On s'intéresse ici à des suites et séries de fonctions en liaison avec des intégrales.

Les deux premières parties sont indépendantes.

On admet les résultats suivants : \[K=\int_0^{\pinf}\dfrac{\sin t}{t}\odif{t}=\int_0^{\pinf}\dfrac{1-\cos t}{t^2}\odif{t}=\dfrac{\pi}{2}\qquad\text{et}\qquad\int_0^1\dfrac{\ln u}{u-1}\odif{u}=\dfrac{\pi^2}{6}.\]

\subsection*{Étude de quelques suites d'intégrales}
\addcontentsline{toc}{subsection}{Étude de quelques suites d'intégrales}

\setcounter{q}{0}
\begin{q}
\begin{enumerate}
    \item On considère ici une application continue \(f:\intervie{0}{\pinf}\to\R\). \\ Pour tout \(n\in\N\), on pose \(I_n=\int_0^1f\paren{t^n}\odif{t}\). \\ Déterminez \(\lim_{n\to\pinf}I_n\). \\
    \item On suppose ici de plus que \(u\mapsto\dfrac{f\paren{u}}{u}\) est intégrable sur \(\intervei{0}{1}\). \\ Déterminez \(\lim_{n\to\pinf}nI_n\). \\ \textit{On pourra transformer \(nI_n\) grâce à un changement de variable.} \\
    \item Application : déterminez un équivalent quand \(n\to\pinf\) de \(\int_0^1\sin\paren{t^n}\odif{t}\) (grâce à une intégrale).
\end{enumerate}
\end{q}

\begin{q}
On considère maintenant que \(f:\intervie{0}{\pinf}\to\R\) est une application continue et intégrable sur \(\Rp\).

\begin{enumerate}
    \item Soit \(n\in\Ns\). Grâce à un changement de variable approprié, justifiez l'existence de \(A_n=\int_1^{\pinf}f\paren{t^n}\odif{t}\). \\
    \item Déterminez \(\lim_{n\to\pinf}nA_n\) (grâce à une intégrale qu'on ne cherchera pas à calculer).
\end{enumerate}
\end{q}

\begin{q}
\begin{enumerate}
    \item Pour tout \(n\in\N\) tel que \(n\geq2\) et tout \(A>1\), on pose \(C_n\paren{A}=\int_1^A\sin\paren{t^n}\odif{t}\). \\ Grâce à un changement de variable et une intégration par parties, exprimez \(C_n\paren{A}\) en fonction de \(\int_1^{A^n}\dfrac{1-\cos u}{u^2}u^{\nicefrac{1}{n}}\odif{u}\) et de \(A\). \\
    \item Déduisez-en que \(C_n\paren{A}\) a une limite quand \(A\to\pinf\), prouvant l'existence de \(\int_1^{\pinf}\sin\paren{t^n}\odif{t}\) pour tout \(n\in\N\) tel que \(n\geq2\). \\
    \item Application : déterminez \(\lim_{n\to\pinf}n\int_0^{\pinf}\sin\paren{t^n}\odif{t}\) grâce à \(K\), admise en préambule.
\end{enumerate}
\end{q}

\subsection*{Étude de séries de fonctions}
\addcontentsline{toc}{subsection}{Étude de séries de fonctions}

\note{Ne pas traiter les questions 4 et 5}

\begin{q}[Un premier exemple]
\begin{enumerate}
    \item Pour tout \(x\in\intervee{-1}{1}\), calculez \(F\paren{x}=\sum_{n=1}^{\pinf}x^n\) ainsi que \(F\prim\paren{x}\). \\
    \item Déterminez \(\lim_{x\to1^-}F\paren{x}\), \(\lim_{x\to1^-}\paren{1-x}F\paren{x}\), \(\lim_{x\to1^-}\paren{1-x}F\prim\paren{x}\) et \(\lim_{x\to1^-}\paren{1-x}^2F\prim\paren{x}\).
\end{enumerate}
\end{q}

\begin{q}[Un deuxième exemple]
Dans cette question pour tout \(x\in\intervee{-1}{1}\), on pose cette fois \(F\paren{x}=\sum_{n=1}^{\pinf}\dfrac{x^n}{1-x^n}\).

\begin{enumerate}
    \item Soit \(a\in\intervee{0}{1}\). Prouvez la convergence normale de cette série de fonctions sur le segment \(\intervii{-a}{a}\). \\ Déduisez-en que \(F\) est définie et continue sur \(\intervee{-1}{1}\). \\
    \item Montrez que, pour tout \(x\in\intervee{0}{1}\) et pour tout \(n\in\Ns\), on a \(\dfrac{1-x^n}{1-x}\leq n\). \\ Déduisez-en \(\lim_{x\to1^-}F\paren{x}\) et \(\lim_{x\to1^-}\paren{1-x}F\paren{x}\).
\end{enumerate}
\end{q}

\begin{q}
Dans cette question, \(f\) est une application réelle continue et croissante sur \(\intervie{0}{1}\) avec \(f\paren{0}=0\) et telle que \(u\mapsto\dfrac{f\paren{u}}{u}\) soit intégrable sur \(\intervee{0}{1}\).

Soit \(x\in\intervee{0}{1}\).

\begin{enumerate}
    \item Justifiez l'existence de \(G\paren{x}=\int_0^{\pinf}f\paren{x^t}\odif{t}\) et l'égalité \(G\paren{x}=-\dfrac{1}{\ln x}\int_0^1\dfrac{f\paren{u}}{u}\odif{u}\). \\
    \item Pour tout \(n\in\Ns\), justifiez l'encadrement \[\int_n^{n+1}f\paren{x^t}\odif{t}\leq f\paren{x^n}\leq\int_{n-1}^nf\paren{x^t}\odif{t}.\] \\
    \item Déduisez-en l'existence de \(F\paren{x}=\sum_{n=1}^{\pinf}f\paren{x^n}\), ainsi qu'un encadrement de \(F\paren{x}\) par deux intégrales dépendant de \(x\). \\
    \item Concluez avec soin que \(\lim_{x\to1^-}\paren{1-x}F\paren{x}=\int_0^1\dfrac{f\paren{u}}{u}\odif{u}\).
\end{enumerate}
\end{q}

\begin{q}[Un dernier exemple]
\note{Ne pas traiter la sous-question 1}

Pour tout \(x\in\intervee{-1}{1}\), on pose cette fois \(F\paren{x}=-\sum_{n=1}^{\pinf}\ln\paren{1-x^n}\).

\begin{enumerate}
    \item Montrez que \(F\) est définie et de classe \(\classe{1}\) sur \(\intervee{-1}{1}\) et exprimez sa dérivée sous la forme d'une série de fonctions. \\
    \item Grâce à la question 6.4, montrez que \(\lim_{x\to1^-}\paren{1-x}F\paren{x}=\int_0^1\dfrac{\ln u}{u-1}\odif{u}\) (intégrale donnée en préambule). \\
    \item Par une méthode similaire à celle de la question 6, montrez que \[\lim_{x\to1^-}\paren{\paren{1-x}^2\sum_{n=1}^{\pinf}\dfrac{nx^n}{1-x^n}}=\int_0^1\dfrac{\ln u}{u-1}\odif{u}.\] \\ Déduisez-en \(\lim_{x\to1^-}\paren{1-x}^2F\prim\paren{x}\).
\end{enumerate}
\end{q}

\section*{Problème 2 -- CCINP 2022 -- MP -- Math 1}
\addcontentsline{toc}{section}{Problème 2 -- CCINP 2022 -- MP -- Math 1}

Dans ce problème, on étudie certaines intégrales et séries numériques reliées aux intégrales dites de Fresnel. Augustin Fresnel (1788-1827) démontra le caractère ondulatoire de la lumière et, pour cette raison, il est considéré comme un des fondateurs de l'optique moderne.

\subsection*{Intégrales fonctions de leur borne}
\addcontentsline{toc}{subsection}{Intégrales fonctions de leur borne}

\note{Ne pas traiter les questions 3, 7 et 8}

Dans cette partie, on définit la fonction \(H\) par l'expression \(H\paren{x}=\int_0^x\e{\i t^2}\odif{t}\), où \(\e{\i t^2}\) signifie \(\exp\paren{\i t^2}\).

\setcounter{q}{0}
\begin{q}
Démontrez que \(H\) est définie et de classe \(\classe{\infty}\) sur \(\R\). Donnez une expression de \(H\prim\paren{x}\).
\end{q}

\begin{q}
Étudiez la parité de la fonction \(H\).
\end{q}

\begin{q}
Démontrez que la fonction \(t\mapsto\e{\i t^2}\) est développable en série entière au voisinage de \(0\). Déduisez-en un développement en série entière de la fonction \(H\) au voisinage de \(0\), en précisant l'intervalle sur lequel ce développement est valable.
\end{q}

\begin{q}
Si \(x>0\), démontrez que \[H\paren{x}=\dfrac{1}{2}\int_0^{x^2}\dfrac{\e{\i u}}{\sqrt{u}}\odif{u}.\]
\end{q}

\begin{q}
Pour \(x>\sqrt{2\pi}\), déduisez-en que \[H\paren{x}-H\paren{\sqrt{2\pi}}=-\i\dfrac{\e{\i x^2}}{2x}+\dfrac{\i}{2\sqrt{2\pi}}-\dfrac{\i}{4}\int_{2\pi}^{x^2}\dfrac{\e{\i u}}{u^{\nicefrac{3}{2}}}\odif{u}.\]
\end{q}

\begin{q}
Déduisez-en que l'intégrale généralisée \(\int_0^{\pinf}\e{\i t^2}\odif{t}\) converge.
\end{q}

\begin{q}[Informatique Pour Tous]
Proposez, en langage Python, une fonction \verb|I(f, a, b, n)| qui prend en entrée une fonction \verb|f| à valeurs réelles ou complexes, deux réels \verb|a| et \verb|b| et un entier naturel \verb|n| et qui renvoie une valeur approchée avec la méthode des rectangles de \(\int_a^bf\paren{t}\odif{t}\) calculée avec \(n\) rectangles.
\end{q}

\begin{q}[Informatique Pour Tous]
Proposez, en langage Python, une fonction \verb|H(x, n)| qui prend en entrée un réel \verb|x| et un entier naturel \verb|n| et qui renvoie une valeur approchée de \(H\paren{x}\) calculée avec la fonction de la question précédente. On rappelle que le code Python pour \(\e{\i t^2}\) est \verb|exp(1j * t ** 2)|.
\end{q}

\subsection*{Calcul des intégrales de Fresnel}
\addcontentsline{toc}{subsection}{Calcul des intégrales de Fresnel}

Dans cette partie, on étudie la fonction \(g\) d'expression \[g\paren{x}=\int_{\minf}^{\pinf}\dfrac{\e{-x^2\paren{t^2-\i}}}{t^2-\i}\odif{t}.\]

Pour cela, on pose \(f\paren{x,t}=\dfrac{\e{-x^2\paren{t^2-\i}}}{t^2-\i}\).

\begin{q}
Si \(\paren{x,t}\in\R^2\), déterminez les modules des nombres complexes \(\e{-x^2\paren{t^2-\i}}\) et \(t^2-\i\).
\end{q}

\begin{q}
Démontrez que \(g\) est définie et continue sur \(\R\) (on pourra utiliser un argument de parité).
\end{q}

\begin{q}
Soit \(\paren{x_n}_n\) une suite divergente vers \(\pinf\).

À l'aide du théorème de convergence dominée, démontrez que \(\lim_{n\to\pinf}g\paren{x_n}=0\).

Déduisez-en la limite de \(g\) en \(\pinf\) et en \(\minf\).
\end{q}

\begin{q}
Démontrez que \(g\) est de classe \(\classe{1}\) sur \(\Rs\).
\end{q}

\begin{q}
On admet dans cette question que l'intégrale \(\int_{\minf}^{\pinf}\e{-t^2}\odif{t}\) converge et est égale à \(\sqrt{\pi}\).

Vérifiez que \[\quantifs{\forall x>0}g\prim\paren{x}=-2\sqrt{\pi}\e{\i x^2}.\]
\end{q}

\begin{q}
Décomposez dans \(\poly[\C]\) la fraction rationnelle \(\dfrac{1}{X^2-\i}\).

On admet ensuite que \[\dfrac{1}{X^2-\i}=\dfrac{1-\i}{4}\paren{\dfrac{\sqrt{2}}{2}\times\dfrac{2X-\sqrt{2}}{X^2-X\sqrt{2}+1}+\dfrac{\i}{X^2-X\sqrt{2}+1}-\dfrac{\sqrt{2}}{2}\times\dfrac{2X+\sqrt{2}}{X^2+X\sqrt{2}+1}+\dfrac{\i}{X^2+X\sqrt{2}+1}}.\]

Démontrez que \(\int_{\minf}^{\pinf}\dfrac{1}{t^2-\sqrt{2}t+1}\odif{t}=\pi\sqrt{2}\).

Donnez la valeur de \(\int_{\minf}^{\pinf}\dfrac{1}{t^2+\sqrt{2}t+1}\odif{t}\) puis déterminez la valeur de \(g\paren{0}\).
\end{q}

\begin{q}
Déduisez-en que \[\quantifs{\forall x>0}g\paren{x}=\dfrac{\paren{1+\i}\pi}{\sqrt{2}}-2\sqrt{\pi}H\paren{x}.\]

Donnez ensuite les valeurs de \(\int_0^{\pinf}\e{\i t^2}\odif{t}\), de \(\int_0^{\pinf}\cos\paren{t^2}\odif{t}\) et de \(\int_0^{\pinf}\sin\paren{t^2}\odif{t}\).
\end{q}

\subsection*{Étude d'une série de fonctions}
\addcontentsline{toc}{subsection}{Étude d'une série de fonctions}

Dans cette partie, on étudie la fonction \(S\) d'expression \[S\paren{x}=\sum_{n=1}^{\pinf}\dfrac{\e{\i nx}}{\sqrt{n}}.\]

Pour tout entier naturel \(n\) non-nul, on note \(f_n\) la fonction d'expression \(f_n\paren{x}=\dfrac{\e{\i nx}}{\sqrt{n}}\).

\begin{q}
On suppose que \(\paren{a_n}_{n\in\Ns}\) est une suite réelle positive décroissante de limite nulle et que \(\paren{b_n}_{n\in\N}\) est une suite bornée.

En admettant l'identité suivante : \[\quantifs{\forall N\in\Ns}\sum_{n=1}^Na_n\paren{b_n-b_{n-1}}=\sum_{n=1}^N\paren{a_n-a_{n+1}}b_n+a_{N+1}b_N-a_1b_0,\] démontrez que la série \(\sum_na_n\paren{b_n-b_{n-1}}\) converge.
\end{q}

\begin{q}
Soient \(x\in\intervee{0}{2\pi}\) et \(n\in\Ns\).

Démontrez que \[\sum_{k=1}^n\e{\i kx}=\e{\frac{\i\paren{n+1}x}{2}}\times\dfrac{\sin\paren{\nicefrac{nx}{2}}}{\sin\paren{\nicefrac{x}{2}}}.\]
\end{q}

\begin{q}
À l'aide des deux questions précédentes, démontrez que \(S\) est définie sur \(\intervee{0}{2\pi}\).
\end{q}

\begin{q}
On admet dans cette question que si \(k\in\Ns\) et \(x\in\intervee{0}{2\pi}\), alors \[\abs{\dfrac{\e{\i\paren{k+1}x}-\e{\i kx}}{\i x\sqrt{k}}-\int_k^{k+1}\dfrac{\e{\i tx}}{\sqrt{t}}\odif{t}}\leq\dfrac{1}{4k^{\nicefrac{3}{2}}}.\]

Démontrez qu'il existe une constante \(C>0\) telle que pour tout \(x\in\intervee{0}{2\pi}\) : \[\abs{\dfrac{\e{\i x}-1}{\i x}S\paren{x}-\int_1^{\pinf}\dfrac{\e{\i tx}}{\sqrt{t}}\odif{t}}\leq C.\]
\end{q}

\begin{q}
Démontrez la convergence de l'intégrale \(J\paren{x}=\int_1^{\pinf}\dfrac{\e{\i tx}}{\sqrt{t}}\odif{t}\).

Déterminez la limite, quand \(x\) tend vers \(0^+\), de \(I\paren{x}=\sqrt{x}J\paren{x}\).
\end{q}

\begin{q}
Déterminez la limite en \(0^+\) de la fonction \(x\mapsto\dfrac{\e{\i x}-1}{\i x}\).

Donnez un équivalent de \(S\paren{x}\) quand \(x\) tend vers \(0^+\).
\end{q}


\chapter{Espaces préhilbertiens réels}

\minitoc

\Cf \hyperref[exos_ee]{chapitre suivant}.


\chapter{Endomorphismes dans un espace euclidien}

\minitoc

\section*{Exercices}
\addcontentsline{toc}{section}{Exercices}

\legendeexercices

\begin{exos}
Soit \(E=\polydeg[\R]{n}\). On définit \(\ps{P}{Q}=\sum_{k=0}^nP\paren{k}Q\paren{k}\).

\begin{enumerate}
    \item Montrez que \(\ps{}{}\) est un produit scalaire. \\
    \item Dans le cas où \(n=3\), donnez une base orthonormée de \(\polydeg[\R]{2}\). \\
    \item Calculez la distance de \(X^3\) à \(\polydeg[\R]{2}\).
\end{enumerate}
\end{exos}

\begin{exos}
Soit \(E=\M{n}[\R]\) muni du produit scalaire classique \(\paren{A,B}\mapsto\tr\paren{\trans{A}B}\).

\begin{enumerate}
    \item Montrez que si \(A\) est symétrique et \(B\) est antisymétrique, alors \(A\) et \(B\) sont orthogonales pour ce produit scalaire. \\
    \item Déterminez \(\sym{n}[\R]\ortho\). \\
    \item Dans le cas où \(n=2\), calculez la distance de la matrice \(A=\begin{pmatrix}
        1 & 0 \\
        -1 & 2
    \end{pmatrix}\) au sous-espace vectoriel des matrices triangulaires supérieures.
\end{enumerate}
\end{exos}

\begin{exoss}
Soit \(E=\poly[\R]\). On définit \(\ps{P}{Q}=\sum_{k=0}^{\pinf}\dfrac{P\paren{k}Q\paren{k}}{2^k}\).

\begin{enumerate}
    \item Montrez que \(\ps{}{}\) est un produit scalaire. \\
    \item Donnez une base orthonormée du sous-espace \(\polydeg[\R]{2}\). \\
    \item Calculez la distance de \(X^3\) à \(\Vect{1,X,X^2}\).
\end{enumerate}
\end{exoss}

\begin{exoss}
Même exercice avec \(\ps{P}{Q}=\int_0^{\pinf}P\paren{t}Q\paren{t}\e{-t}\odif{t}\).
\end{exoss}

\begin{exoss}
\begin{enumerate}
    \item Montrez que pour tout \(n\in\N\), il existe un unique polynôme \(T_n\) tel que \(\quantifs{\tpt t\in\R}T_n\paren{\cos t}=\cos\paren{nt}\). \\
    \item Calculez \(T_0\) et \(T_1\) puis montrez que \(\quantifs{\tpt n\in\N}T_{n+2}=2XT_{n+1}-T_n\). En déduire le degré et le coefficient dominant de \(T_n\). \\
    \item Montrez que l'application \(\paren{P,Q}\mapsto\int_{-1}^1\dfrac{P\paren{t}Q\paren{t}}{\sqrt{1-t^2}}\odif{t}\) est un produit scalaire sur \(\poly[\R]\). \\
    \item Montrez que la famille \(\paren{T_n}\) est orthogonale pour ce produit scalaire.
\end{enumerate}
\end{exoss}

\begin{exoss}
Soit \(E\) l'ensemble des suites réelles \(x\) telles que la série \(\sum_nx_n^2\) converge.

\begin{enumerate}
    \item Vérifiez rapidement que \(\quantifs{\tpt\paren{a,b}\in\R^2}\abs{ab}\leq\dfrac{a^2+b^2}{2}\). \\
    \item Soient \(x,y\in E\). Montrez que la série \(\sum_nx_ny_n\) est absolument convergente. \\
    \item Montrez que \(E\) est un \(\R\)-espace vectoriel. \\
    \item On pose \(\ps{x}{y}=\sum_{n=0}^{\pinf}x_ny_n\). Montrez qu'on définit ainsi un produit scalaire sur \(E\).
\end{enumerate}
\end{exoss}

\begin{exos}
Soit \(E\) un espace euclidien de dimension \(3\) muni d'une base orthonormée dans laquelle toutes les coordonnées sont exprimées. Soient \(u\) de coordonnées \(\paren{1,3,1}\), \(v\) de coordonnées \(\paren{5,2,-1}\) et \(w\) de coordonnées \(\paren{1,1,1}\).

\begin{enumerate}
    \item Donnez un système d'équation et une base d'un supplémentaire de chacun des sous-espaces vectoriels \(\Vect{u}\) et \(\Vect{u,v}\). \\
    \item Déterminez la projection orthogonale de \(v\) sur la droite \(\Vect{u}\). \\
    \item Déterminez la projection orthogonale de \(w\) sur le plan \(\Vect{u,v}\). \\
    \item Déterminez la matrice du projecteur orthogonal sur le plan précédent.
\end{enumerate}
\end{exos}

\begin{exos}
Soient \(E\) un espace euclidien de dimension \(4\) muni d'une base orthonormée dans laquelle toutes les coordonnées sont exprimées et \(F=\accol{x\in E\tq x_1+x_2+x_3+x_4=0\text{ et }x_1-x_2+x_3-x_4=0}\).

\begin{enumerate}
    \item Justifiez que \(F\) est un sous-espace vectoriel de \(E\). Donnez une base orthonormée de \(F\). \\
    \item Soit \(v\) de coordonnées \(\paren{2,3,1,-1}\). Déterminez sa projection orthogonale sur \(F\). \\
    \item Déterminez la matrice de la symétrie orthogonale par rapport à \(F\).
\end{enumerate}
\end{exos}

\begin{exoss}
Soit \(E=\Vect{\id{\intervii{0}{\pi}},\sin,\cos}\) muni du produit scalaire \(\ps{f}{g}=\int_0^\pi f\paren{t}g\paren{t}\odif{t}\).

Déterminez \(\inf_{\paren{a,b}\in\R^2}\int_0^\pi\paren{x-\paren{a\cos x+b\sin x}}^2\odif{x}\).
\end{exoss}

\begin{exoss}
\begin{enumerate}
    \item Déterminez \(\inf_{x\in\R}\paren{\paren{2x-b_1}^2+\paren{3x-b_2}^2+\paren{4x-b_3}^2}\) où \(b_1,b_2,b_3\) sont fixés dans \(\R\). \\
    \item Déterminez \(\inf_{\paren{a,b}\in\R^2}\sum_{k=1}^n\paren{k^2-ak-b}^2\). On admet \(\sum_{k=1}^nk^3=\dfrac{n^2\paren{n+1}^2}{4}\) et \\ \(\sum_{k=1}^nk^4=\dfrac{n\paren{n+1}\paren{2n+1}\paren{3n^2+3n-1}}{30}\). \\
    \item Déterminez \(\inf_{\paren{a,b}\in\R^2}\int_0^1\paren{x^2-ax-b}^2x^2\odif{x}\). \\
    \item Déterminez \(\inf_{\paren{a,b,c}\in\R^3}\int_{-\pi}^\pi\paren{t-a\sin t-b\cos t-c}^2\odif{t}\).
\end{enumerate}
\end{exoss}

\begin{exoss}
Soient \(E\) un espace euclidien et \(\paren{e_1,\dots,e_n}\in E^n\) des vecteurs unitaires tels que \[\quantifs{\forall x\in E}\norme{x}^2=\sum_{k=1}^n\ps{x}{e_k}^2.\]

Montrez que \(\paren{e_1,\dots,e_n}\) est une base orthonormale de \(E\).
\end{exoss}

\begin{exoss}
Soient \(E\) un espace euclidien et \(F,G\) deux sous-espaces vectoriels de \(E\).

Montrez que \(\paren{F\ortho}\ortho=F\), \(\paren{F+G}\ortho=F\ortho\inter G\ortho\) et \(\paren{F\inter G}\ortho=F\ortho+G\ortho\).
\end{exoss}

\begin{exosss}
On munit \(E=\ensclasse{0}{\intervii{0}{1}}{\R}\) du produit scalaire \(\ps{f}{g}=\int_0^1f\paren{t}g\paren{t}\odif{t}\).

Soit \(F=\accol{f\in E\tq f\paren{0}=0}\).

A-t-on \(E=F\oplus F\ortho\) ? A-t-on \(\paren{F\ortho}\ortho=F\) ?

\textit{Indication : si \(f\in F\ortho\), considérez la fonction \(t\mapsto tf\paren{t}\).}
\end{exosss}

\begin{exoss}
Soient \(E\) un espace euclidien et \(p\in\Lendo{E}\) un projecteur.

On sait d'après le cours que si \(p\) est un projecteur orthogonal, alors \(\quantifs{\tpt x\in E}\norme{p\paren{x}}\leq\norme{x}\).

Montrez que la réciproque est vraie, en utilisant le vecteur \(y+\lambda x\), où \(x\in\Im p\), \(y\in\ker p\) et \(\lambda\in\R\).
\end{exoss}

\begin{exoss}
Soient \(E\) un espace euclidien de dimension \(3\) rapporté à une base orthonormée \(\fami{B}\) et \(f\in\Lendo{E}\) tel que \(\Mat{f}=A=\dfrac{1}{15}\begin{pmatrix}
8 & 6 & -10 \\
-10 & 5 & 0 \\
6 & -8 & 5
\end{pmatrix}\).

\begin{enumerate}
    \item Donnez une base du noyau et de l'image de \(f\). Vérifiez que ces deux sous-espaces de \(E\) sont supplémentaires orthogonaux. \\
    \item Donnez une base orthonormée directe \(\fami{B}\prim\) de \(E\) dont les vecteurs sont choisis dans le noyau ou l'image de \(f\) (dans cet ordre). \\
    \item Montrez que la matrice de \(f\) dans cette nouvelle base est de la forme \(\begin{pmatrix}
        0 & 0 & 0 \\
        0 & \cos\alpha & -\sin\alpha \\
        0 & \sin\alpha & \cos\alpha
    \end{pmatrix}\). \\
    \item Donnez une interprétation géométrique de l'application \(f\).
\end{enumerate}
\end{exoss}

\begin{exos}
Soit \(E\) un espace euclidien orienté de dimension \(3\) rapporté à une base orthonormée directe.

Déterminez la nature et les éléments caractéristiques de chacune des applications linéaires dont les matrices sont : \[A=\dfrac{1}{3}\begin{pmatrix}
-2 & -1 & 2 \\
2 & -2 & 1 \\
1 & 2 & 2
\end{pmatrix}\qquad B=\dfrac{1}{9}\begin{pmatrix}
7 & 4 & 4 \\
-4 & 8 & -1 \\
4 & 1 & -8
\end{pmatrix}\qquad C=\dfrac{1}{9}\begin{pmatrix}
-8 & 4 & 1 \\
4 & 7 & 4 \\
1 & 4 & -8
\end{pmatrix}\]

\[D=\dfrac{1}{4}\begin{pmatrix}
3 & 1 & \sqrt{6} \\
1 & 3 & -\sqrt{6} \\
-\sqrt{6} & \sqrt{6} & 2
\end{pmatrix}\qquad E=\dfrac{1}{4}\begin{pmatrix}
3 & 1 & -\sqrt{6} \\
1 & 3 & \sqrt{6} \\
-\sqrt{6} & \sqrt{6} & -2
\end{pmatrix}.\]
\end{exos}

\begin{exoss}
Soit \(E\) un espace euclidien orienté de dimension \(3\) rapporté à une base orthonormée directe.

Déterminez la matrice de la rotation d'axe orienté par \(u\) de coordonnées \(\paren{1,1,-1}\) et d'angle \(\dfrac{\pi}{2}\).
\end{exoss}

\begin{exoss}
Soient \(E\) un espace euclidien orienté de dimension \(3\) rapporté à une base orthonormée directe, \(u\) un vecteur unitaire, \(\theta\) un réel et \(f\) la rotation d'axe orienté par \(u\) et d'angle \(\theta\).

Montrez que \(\quantifs{\tpt x\in E}f\paren{x}=\paren{1-\cos\theta}\ps{x}{u}u+\cos\paren{\theta}x+\sin\paren{\theta}u\vecto x\).
\end{exoss}

\begin{exoss}
Déterminez \(\Card\paren{\Orth{n}[\R]\inter\M{n}[\Z]}\).
\end{exoss}

\begin{exoss}
Soit \(E\) un espace euclidien et \(u\) un automorphisme orthogonal de \(E\). On pose \(v=\id{E}-u\).

\begin{enumerate}
    \item Montrez que \(\ker v=\paren{\Im v}\ortho\). \\
    \item Soit \(p\) la projection orthogonale sur \(\ker v\). Montrez que \[\quantifs{\forall x\in E}\dfrac{1}{n}\sum_{k=0}^{n-1}u^k\paren{x}\tendqd{n\to\pinf}p\paren{x}.\]
\end{enumerate}
\end{exoss}

\begin{exos}
Soit \(A\in\M{n}[\R]\) telle que \(\trans{A}=A\).

\begin{enumerate}
    \item Montrez que s'il existe \(k\in\N\) tel que \(A^k=I_n\), alors \(A^2=I_n\). Que peut-on dire de mieux si \(k\) est impair ? \\
    \item Montrez que si \(A\) est nilpotente, alors \(A=0\).
\end{enumerate}
\end{exos}

\begin{exos}
Soient \(S=\paren{s_{i\,j}}\) une matrice symétrique de \(\M{n}[\R]\) et \(\lambda_1,\dots,\lambda_n\) ses valeurs propres.

Montrez que \(\sum_{1\leq i,j\leq n}s_{i\,j}^2=\sum_{i=1}^n\lambda_i^2\).
\end{exos}

\begin{exoss}
Soient \(n\geq3\) et \(A=\begin{pmatrix}
1 & 1 & \dots & 1 & 1 \\
1 & 0 & \dots & 0 & 1 \\
\vdots & \vdots & 0 & \vdots & \vdots \\
1 & 0 & \dots & 0 & 1 \\
1 & 1 & \dots & 1 & 1
\end{pmatrix}\in\M{n}[\R]\).

\begin{enumerate}
    \item Justifiez que \(A\) est diagonalisable, puis que \(A\) possède au plus trois valeurs propres distinctes dont l'une est \(0\). \\
    \item En notant \(\lambda\) et \(\mu\) les deux autres valeurs propres (pouvant être égales), donnez deux équations reliant \(\lambda\) et \(\mu\), et déduisez-en le spectre de \(A\).
\end{enumerate}
\end{exoss}

\begin{exoss}
Soit \(A\in\M{n}[\R]\).

Montrez que \(A\trans{A}\) et \(\trans{A}A\) sont orthosemblables.
\end{exoss}

\begin{exoss}
Déterminez toutes les matrices \(A\in\M{n}[\R]\) telles que \[A\trans{A}A=I_n.\]
\end{exoss}

\begin{exoss}
Soit \(n\in\Ns\). On munit \(E=\polydeg[\R]{n}\) du produit scalaire \[\ps{P}{Q}=\int_{-1}^1P\paren{t}Q\paren{t}\odif{t}.\]

\begin{enumerate}
    \item Montrez que l'application \(u:P\mapsto\paren{1-X^2}P\seconde-2XP\prim\) est un endomorphisme auto-adjoint de \(E\). \\
    \item Montrez que les valeurs propres de \(u\) sont négatives.
\end{enumerate}
\end{exoss}

\begin{exoss}
Soient \(E\) un espace euclidien, \(u_1,\dots,u_p\in E\) et \(f\) l'application de \(E\) dans \(E\) telle que \[f:x\mapsto\sum_{k=1}^p\ps{u_k}{x}u_k.\]

\begin{enumerate}
    \item Quelle est l'application \(f\) lorsque \(\paren{u_1,\dots,u_p}\) est une famille orthonormée ? \\
    \item Dans le cas général, montrez que \(f\) est un endomorphisme auto-adjoint positif de \(E\). \\
    \item Déterminez \(\ker f\) et \(\Im f\).
\end{enumerate}
\end{exoss}

\begin{exoss}
Rappel : une matrice symétrique \(A\in\sym{n}[\R]\) est dite positive quand \(\quantifs{\forall X\in\M{n\,1}[\R]}\trans{X}AX\geq0\).

Soit \(A\in\sym{n}[\R]\). Montrez l'équivalence des propriétés suivantes :

\begin{enumerate}
    \item[(\(\alpha\))] \(A\) est positive \\
    \item[(\(\beta\))] toutes les valeurs propres de \(A\) sont positives \\
    \item[(\(\gamma\))] il existe \(M\in\M{n}[\R]\) telle que \(A=\trans{M}M\) \\
    \item[(\(\delta\))] il existe \(S\in\sym{n}[\R]\) telle que \(A=S^2\).
\end{enumerate}

Modifiez les propositions précédentes pour caractériser les matrices \(A\) définies-positives.
\end{exoss}

\begin{exosss}[Racine carrée]
\begin{enumerate}
    \item Montrez que \(\fonction{r}{\symdefpos{n}[\R]}{\symdefpos{n}[\R]}{M}{M^2}\) est une surjection. \\
    \item Soit \(A\in\symdefpos{n}[\R]\) et \(M\) un antécédent de \(A\) par \(r\). Montrez qu'il existe un polynôme \(P\in\poly[\R]\) tel que \(M=P\paren{A}\). \\
    \item Soient \(A\in\symdefpos{n}[\R]\) et \(M,N\) deux antécédents de \(A\) par \(r\). Montrez que \(\paren{M-N}\paren{M+N}=0\) et que \(M+N\) est inversible. Déduisez-en que l'application \(r\) est injective.
\end{enumerate}
\end{exosss}

\begin{exosss}[Décomposition polaire]
Montrez que pour tout \(A\in\GL{n}[\R]\), il existe un unique couple \(\paren{Q,S}\in\Orth{n}[\R]\times\symdefpos{n}[\R]\) tel que \(A=QS\).
\end{exosss}

\begin{exoss}
Soient \(f\) et \(g\) deux endomorphismes auto-adjoints d'un espace euclidien.

Montrez que \(f\rond g\) est auto-adjoint ssi \(f\) et \(g\) commutent.
\end{exoss}

\begin{exoss}
\begin{enumerate}
    \item Soit \(S\in\sym{n}[\R]\). Montrez qu'il existe des réels \(\lambda_1,\dots,\lambda_n\) et des matrices colonnes \(U_1,\dots,U_n\) de \(\M{n\,1}[\R]\) tels que \[\quantifs{\forall\paren{i,j}\in\interventierii{1}{n}^2}\trans{U_i}U_j=\delta_{i\,j}\qquad\text{et}\qquad S=\sum_{i=1}^n\lambda_iU_i\trans{U_i}.\] \\
    \item Soit \(u\) un endomorphisme auto-adjoint d'un espace euclidien \(E\). Montrez que \(u\) est combinaison linéaire de projections orthogonales sur des droites et que ces projections commutent entre elles.
\end{enumerate}
\end{exoss}

\begin{exoss}
Une matrice \(M\in\M{n}[\R]\) vérifie \(M^2+4I_n=0\) et \(\trans{M}M=M\trans{M}=S\).

\begin{enumerate}
    \item Trouvez un polynôme annulateur de \(S\) de degré \(2\). \\
    \item Déduisez-en que \(\dfrac{1}{2}M\) est orthogonale. \\
    \item Dans le cas \(n=2\), déterminez toutes les matrices de \(\M{2}[\R]\) qui vérifient les conditions de l'exercice.
\end{enumerate}
\end{exoss}

\begin{exoss}
Soient \(E\) un espace euclidien orienté de dimension \(3\) et \(u\) un vecteur non-nul de \(E\).

On pose \(f\) l'endomorphisme de \(E\) défini par \(f\paren{x}=u\vecto\paren{u\vecto x}\).

Montrez que \(f\) est diagonalisable et diagonalisez-le.
\end{exoss}

\begin{exoss}
Soient \(A\in\M{n}[\R]\), \(S=\dfrac{1}{2}\paren{A+\trans{A}}\) et \(a,b\) les plus petite et plus grande valeurs propres de \(S\).

Montrez que toute valeur propre réelle de \(A\) est comprise entre \(a\) et \(b\).
\end{exoss}

\begin{exoss}
Soit \(E\) un espace euclidien. On dit qu'un endomorphisme \(f\) de \(E\) est anti-adjoint quand \[\quantifs{\forall u,v\in E}\ps{f\paren{u}}{v}=-\ps{u}{f\paren{v}}.\]

Soit \(f\) un endomorphisme anti-adjoint de \(E\).

\begin{enumerate}
    \item Montrez que \(\quantifs{\forall u\in E}f\paren{u}\perp u\). \\
    \item Montrez que la seule valeur propre possible de \(f\) est \(0\). L'endomorphisme \(f\) peut-il être diagonalisable ? \\
    \item Montrez que \(f\rond f\) est un endomorphisme auto-adjoint de \(E\) et que ses valeurs propres sont négatives ou nulles. \\
    \item Soit \(x\) un vecteur propre de \(f\rond f\) pour la valeur propre \(-\lambda\not=0\). \\ \begin{enumerate}
        \item Montrez que \(P_x=\Vect{x,f\paren{x}}\) est un plan vectoriel stable par \(f\). \\
        \item Montrez qu'il existe une base orthonormée \(\paren{a,b}\) de \(P_x\) où la matrice de l'endomorphisme induit \(f_{P_x}\) est de la forme \[R_\mu=\begin{pmatrix}
            0 & -\mu \\
            \mu & 0
        \end{pmatrix},\] où \(\mu\) est un réel strictement positif que l'on précisera. \\
    \end{enumerate}
    \item Montrez qu'il existe une base orthonormée de \(E\) où la matrice de \(f\) est diagonale par blocs, avec des blocs de la forme \(R_\mu\) d'abord puis des blocs \(\paren{0}\) de taille \(1\times1\).
\end{enumerate}
\end{exoss}

\begin{exo}[Oral CCINP, 1]~\\
\begin{enumerate}
    \item Calculez, pour \(n\in\N\), \(I_n=\dfrac{1}{\sqrt{\pi}}\int_{\minf}^{\pinf}\e{-t^2}t^n\odif{t}\) (distinguer les cas \(n\) pair et \(n\) impair ; on donne de plus \(I_0=1\)). \\
    \item Montrez que l'application \(\paren{P,Q}\mapsto\int_{\minf}^{\pinf}P\paren{t}Q\paren{t}\e{-t^2}\odif{t}\) est un produit scalaire sur \(\poly[\R]\). \\
    \item Calculez la distance de \(X^3\) à \(\polydeg[\R]{2}\).
\end{enumerate}
\end{exo}

\begin{exo}[Oral CCINP, 2]
Soit \(E=\ensclasse{0}{\intervii{0}{1}}{\R}\). On pose pour \(\paren{f,g}\in E^2\) : \(\ps{f}{g}=\int_0^1f\paren{t}g\paren{t}t^2\odif{t}\).

\begin{enumerate}
    \item Montrez que \(\ps{}{}\) est un produit scalaire. \\
    \item Calculez \(\int_0^1t^n\ln\paren{t}\odif{t}\) pour tout \(n\in\N\). \\
    \item Soit \(F=\accol{x\mapsto ax+b\tq\paren{a,b}\in\R^2}\) et \(u\in E\) telle que \(u\paren{x}=x\ln x\) pour tout \(x\in\intervei{0}{1}\). Déterminez le projeté orthogonal de \(u\) sur \(F\). \\
    \item Déterminez \(\inf_{\paren{a,b}\in\R^2}\int_0^1\paren{at+b-t\ln t}^2t^2\odif{t}\).
\end{enumerate}
\end{exo}

\begin{exo}[Oral CCINP, 3]
Soit \(E=\ensclasse{2}{\intervii{0}{1}}{\R}\). On pose pour \(\paren{f,g}\in E^2\) : \(\ps{f}{g}=\int_0^1\paren{fg+f\prim g\prim}\).

On considère les sous-ensembles \[V=\accol{f\in E\tq f\seconde=f}\qquad G=\accol{f\in E\tq f\paren{0}=f\paren{1}=0}\qquad H=\accol{f\in E\tq f\paren{0}=\ch1\text{ et }f\paren{1}=1}.\]

\begin{enumerate}
    \item Montrez que la famille \(\paren{\ch,\sh}\) est une base de \(V\). \\
    \item Soient \(f\in V\) et \(g\in E\). Montrez que \(\ps{f}{g}=f\prim\paren{1}g\paren{1}-f\prim\paren{0}g\paren{0}\). Calculez \(\ps{\ch}{\sh}\), \(\norme{\ch}^2\) et \(\norme{\sh}^2\). \\
    \item Soient \(f\in V\) et \(g\in G\). Montrez que \(\ps{f}{g}=0\). \\
    \item Soit \(f\in H\). Calculez \(\ps{f}{\ch}\) et \(\ps{f}{\sh}\). Déduisez-en le projeté orthogonal de \(f\) sur \(V\). \\
    \item Calculez \(\inf_{f\in H}\int_0^1\paren{f^2+{f\prim}^2}\).
\end{enumerate}
\end{exo}

\begin{exo}[Oral CCINP, 4]
Soient \(E\) un espace euclidien, \(\paren{e_1,\dots,e_n}\) une base orthonormée de \(E\) et \(\paren{u_1,\dots,u_n}\) une famille de vecteurs de \(E\).

\begin{enumerate}
    \item Soit \(\paren{\lambda_1,\dots,\lambda_n}\in\R^n\). Montrez que \(\norme{\sum_{i=1}^n\lambda_iu_i}^2\leq\sum_{i=1}^n\lambda_i^2\times\sum_{i=1}^n\norme{u_i}^2\). \\
    \item Déduisez-en que si \(\sum_{i=1}^n\norme{u_i}^2<1\), alors la famille \(\paren{e_i+u_i}_{1\leq i\leq n}\) est une base de \(E\).
\end{enumerate}
\end{exo}

\begin{exo}[Oral CCINP, 5]
Soient \(E\) un espace euclidien et \(\fami{B}=\paren{e_1,\dots,e_n}\) une base orthonormée de \(E\). On note \(D\) la droite engendrée par le vecteur \(u=\sum_{k=1}^nke_k\).

\begin{enumerate}
    \item Donnez la matrice dans la base \(\fami{B}\) du projecteur orthogonal sur \(D\), noté \(p\). \\
    \item Donnez le polynôme caractéristique et le spectre de \(p\). \\
    \item Calculez la distance de \(v=\sum_{k=1}^ne_k\) à \(D\).
\end{enumerate}
\end{exo}

\begin{exo}[Oral CCINP, 6]
Soient \(A\in\sym{n}[\R]\) une matrice définie-positive et \(B\in\M{n\,m}[\R]\) une matrice de rang \(m\).

\begin{enumerate}
    \item Montrez que \(n\geq m\). \\
    \item Montrez que \(C=\begin{pmatrix}
        A & B \\
        \trans{B} & 0
    \end{pmatrix}\) est inversible.
\end{enumerate}
\end{exo}

\begin{exo}[Oral IMT, 7]
Soient \(E\) un espace euclidien et \(f\) un endomorphisme auto-adjoint défini-positif de \(E\).

\begin{enumerate}
    \item Montrez que \(\paren{x,y}\mapsto\ps{f\paren{x}}{y}\) est un produit scalaire sur \(E\). \\
    \item Montrez qu'il existe un endomorphisme auto-adjoint défini-positif \(g\) tel que \(g^2=f\).
\end{enumerate}
\end{exo}

\begin{exo}[Oral IMT, 8]
Soient \(A\in\M{n}[\R]\) telle que \(A\trans{A}=\trans{A}A\) et \(p\geq2\) tel que \(A^p=0\).

Montrez que \(A=0\).
\end{exo}

\begin{exo}[Oral CCINP, 9]
Soient \(A,B\in\Orth{n}[\R]\) telles que \(M=\dfrac{1}{3}\paren{2A+B}\in\Orth{n}[\R]\).

\begin{enumerate}
    \item Calculez \(A\trans{B}+B\trans{A}\) en calculant \(M\trans{M}\). \\
    \item Montrez que \(A=B\).
\end{enumerate}
\end{exo}

\begin{exo}[Oral CCINP, 10]~\\
Soient \(C\in\M{n\,1}[\R]\) et \(M=\begin{pmatrix}
1 & -\trans{C} \\
C & I_n
\end{pmatrix}\in\M{n+1}[\R]\).

\begin{enumerate}
    \item Calculez \(\trans{M}M\), déduisez-en que \(M\) est inversible. \\
    \item Montrez que \(M\inv\trans{M}\) est une matrice orthogonale.
\end{enumerate}
\end{exo}

\begin{exo}[Oral CCINP, 11]
Soit \(M\in\M{n}[\R]\) telle que \(M\not=I_n\), \(M^3=I_n\) et \(M\trans{M}=\trans{M}M\).

\begin{enumerate}
    \item Montrez que \(M\in\Orth{n}[\R]\). \\
    \item Dans le cas où \(n=3\), déterminez les matrices \(M\) vérifiant les conditions de l'énoncé.
\end{enumerate}
\end{exo}

\begin{exo}[Oral CCMP, 12]~\\
\begin{enumerate}
    \item Montrez que \(\dim\sym{n}[\R]=\dfrac{n\paren{n+1}}{2}\). \\
    \item Est-ce que l'ensemble des matrices nilpotentes de \(\M{n}[\R]\) est un espace vectoriel ? \\
    \item Soit \(\fami{N}\) un sous-espace de \(\M{n}[\R]\) constitué de matrices nilpotentes. Montrez que \(\dim\fami{N}\leq\dfrac{n\paren{n-1}}{2}\). \\
    \item Peut-on avoir \(\dim\fami{N}=\dfrac{n\paren{n-1}}{2}\) ?
\end{enumerate}
\end{exo}

\begin{exo}[Oral Centrale, 13]
Soit \(A\in\M{n}[\R]\). On dit que \(A\) est une matrice normale quand \(A\trans{A}=\trans{A}A\).

\begin{enumerate}
    \item Déterminez les matrices normales de \(\M{2}[\R]\). \\
    \item Montrez que toute matrice normale stabilise un espace de dimension \(1\) ou \(2\). \\
    \item Montrez que si \(A\) est normale, alors il existe \(P\in\Orth{n}[\R]\) telle que \(\trans{P}AP\) soit diagonale par blocs, chaque bloc diagonal étant soit un bloc \(\paren{a}\), soit un bloc \(\begin{pmatrix}
        a & c \\
        -c & b
    \end{pmatrix}\).
\end{enumerate}
\end{exo}

\begin{exo}[Oral Centrale, 14]
Soit \(A\in\M{n\,p}[\R]\). On munit \(\R^n\) et \(\R^p\) de leurs produits scalaires canoniques.

\begin{enumerate}
    \item Montrez que \(\ker A=\ker\trans{A}A\). Déduisez-en que \(\rg A=\rg\trans{A}A\) (noté \(r\) dans la suite). \\
    \item Montrez qu'il existe une famille orthonormée \(\paren{y_1,\dots,y_r}\) de \(\R^r\) telle que la matrice \(Y\) de colonnes \(y_1,\dots,y_r\) vérifie \(\trans{Y}\trans{A}AY=D\), où \(D\) est diagonale à coefficients diagonaux strictement positifs. \\
    \item Montrez qu'il existe \(U,V\) orthogonales et \(\Delta\) diagonale telles que \(A=U\Delta V\).
\end{enumerate}
\end{exo}

\begin{exo}[Oral CCMP, 15]
Soit \(A=\paren{a_{i\,j}}\in\symdefpos{n}[\R]\).

\begin{enumerate}
    \item Montrez que \(\quantifs{\tpt\paren{i,j}\in\interventierii{1}{n}^2}a_{i\,j}>0\text{ et }a_{i\,j}^2\leq a_{i\,i}a_{j\,j}\). \\
    \item Montrez que \(\max_{\paren{i,j}\in\interventierii{1}{n}^2}\abs{a_{i\,j}}=\max_{k\in\interventierii{1}{n}}a_{k\,k}\).
\end{enumerate}
\end{exo}

\begin{exo}[Oral CCMP, 16]
\begin{enumerate}
    \item Montrez que si \(A,B\in\sympos{n}[\R]\), alors \(\tr\paren{AB}\geq0\). \\
    \item Déterminez les matrices \(M\in\M{n}[\R]\) telles que \(\quantifs{\forall S\in\sympos{n}[\R]}\tr\paren{MS}\geq0\).
\end{enumerate}
\end{exo}

\begin{exo}[Oral CCMP, 17]
Soient \(A\in\M{n}[\R]\) et \(k\geq2\) tels que \(A^k=\trans{A}\).

\begin{enumerate}
    \item Montrez que \(\ker A\) et \(\Im A\) sont supplémentaires orthogonaux dans \(\R^n\). \\
    \item Montrez que \(B=A^{k+1}\) est une matrice de projecteur orthogonal. \\
    \item Montrez que \(A\) induit une isométrie sur \(\Im A\). \\
    \item Déduisez-en \(A\).
\end{enumerate}
\end{exo}

\begin{exo}[Oral CCMP, 18]
Soient \(S\in\symdefpos{n}[\R]\) et \(A\in\antisym{n}[\R]\).

\begin{enumerate}
    \item Montrez que le spectre complexe de \(A\) est inclus dans \(\iR\). \\
    \item Montrez qu'il existe \(S\prim\in\symdefpos{n}[\R]\) telle que \({S\prim}^2=S\). \\
    \item Montrez que \(\det S\leq\det\paren{S+A}\).
\end{enumerate}
\end{exo}

\section*{Problème}
\addcontentsline{toc}{section}{Problème}

\subsection*{Notations}
\addcontentsline{toc}{subsection}{Notations}

Pour ce problème, on désigne par \(n\) un entier naturel non-nul.

On note \(\sym{n}[\R]=\accol{A\in\M{n}[\R]\tq\trans{A}=A}\) le sous-espace vectoriel de \(\M{n}[\R]\) formé des matrices symétriques.

Un vecteur de \(\R^n\) est noté \(x=\paren{x_j}_{1\leq j\leq n}=\tcoords{x_1}{\vdots}{x_n}\) en identifiant les matrices-colonnes de \(\M{n\,1}[\R]\) avec les éléments de \(\R^n\).

Une matrice \(A\) de \(\M{n}[\R]\) est notée \(A=\paren{a_{i\,j}}_{1\leq i,j\leq n}\) où \(a_{i\,j}\) est le coefficient de \(A\) situé en ligne \(i\) et colonne \(j\).

\(\Sp{A}\) est le spectre de \(A\), ensemble de ses valeurs propres.

L'espace vectoriel \(\R^n\) est muni du produit scalaire canonique défini par \[\quantifs{\forall\paren{x,y}\in\R^n\times\R^n}\ps{x}{y}=\trans{x}y=\sum_{k=1}^nx_ky_k\] et \(x\mapsto\norme{x}=\sqrt{\ps{x}{x}}\) est la norme euclidienne associée.

La sphère unité de \(R^n\) est notée \(\Omega_n=\accol{x\in\R^n\tq\norme{x}=1}\).

À toute matrice \(A\in\M{n}[\R]\), on associe la fonction \(q_A:\R^n\to\R\) définie par \[\quantifs{\forall x\in\R^n}q_A\paren{x}=\ps{Ax}{x}.\]

\subsection*{Une norme sur \(\sym{n}[\R]\)}
\addcontentsline{toc}{subsection}{Une norme sur \(\sym{n}[\R]\)}

\setcounter{q}{0}

\begin{q}
Soit \(A\in\M{n}[\R]\).

\begin{enumerate}
    \item Montrez que la fonction \(q_A\) est bornée et atteint ses bornes sur la sphère unité \(\Omega_n\). On note alors \[m_A=\min_{x\in\Omega_n}q_A\paren{x}\qquad\text{et}\qquad M_A=\max_{x\in\Omega_n}q_A\paren{x}.\] \\
    \item Démontrez que toute valeur propre réelle de \(A\) se trouve dans l'intervalle \(\intervii{m_A}{M_A}\). \\
    \item Explicitez \(\Sp{A}\), \(m_A\) et \(M_A\) pour la matrice \[A=\begin{pmatrix}
        2 & -1 \\
        0 & 2
    \end{pmatrix}.\]
\end{enumerate}
\end{q}

\begin{q}
Soit \(A\in\M{n}[\R]\). On suppose que \(q_A\paren{x}=0\) pour tout \(x\in\Omega_n\).

\begin{enumerate}
    \item Montrez que \(q_A\paren{y}=0\) pour tout \(y\in\R^n\). \\
    \item Si \(\paren{y,z}\in\R^n\times\R^n\), exprimez \(q_A\paren{y+z}\) (qui est nul d'après la question 2a) en fonction de \(\ps{Ay}{z}\) et \(\ps{Az}{y}\). \\
    \item Montrez que la matrice \(A\) est anti-symétrique (c'est à dire que \(\trans{A}=-A\)).
\end{enumerate}
\end{q}

\begin{q}
Soit \(A\in\sym{n}[\R]\). Montrez que \[\croch{\quantifs{\forall x\in\Omega_n}q_A\paren{x}=0}\ssi A=0_n.\]
\end{q}

\begin{q}
Montrez que l'application \(N:\sym{n}[\R]\to\R\) définie par \[\quantifs{\forall A\in\sym{n}[\R]}N\paren{A}=\sup_{x\in\Omega_n}\abs{q_A\paren{x}}\] est une norme.
\end{q}

\begin{q}
On considère \(A\in\sym{n}[\R]\) et \(u\) l'endomorphisme de \(\R^n\) canoniquement associé à \(A\).

\begin{enumerate}
    \item Justifiez qu'il existe \(n\) nombres réels \[\lambda_1\leq\dots\leq\lambda_n\] et une base orthonormée \(\paren{e_k}_{1\leq k\leq n}\) de \(\R^n\) tels que \[\quantifs{\forall k\in\interventierii{1}{n}}u\paren{e_k}=Ae_k=\lambda_ke_k.\] \\
    \item Précisez \(q_A\paren{e_k}\) pour tout \(k\in\interventierii{1}{n}\). \\
    \item Soit \(x=\sum_{k=1}^nx_k\prim e_k\in\Omega_n\). Exprimez \(q_A\paren{x}\) en fonction des valeurs propres \(\lambda_k\) de \(A\) et des composantes \(x_k\prim\) de \(x\). \\
    \item Redémontrez le résultat de la question 1 : la fonction \(q_A\) possède un minimum \(m_A\) et un maximum \(M_A\) sur la sphère unité \(\Omega_n\). Explicitez \(m_A\) et \(M_A\) en fonction des valeurs propres de \(A\). \\
    \item Montrez que \[N\paren{A}=\sup_{x\in\Omega_n}\abs{q_A\paren{x}}=\max_{\lambda\in\Sp{A}}\abs{\lambda}.\] Établissez une inégalité entre \(\abs{\det A}\) et \(\paren{N\paren{A}}^n\). \\
    \item Exemple : si \(A=\begin{pmatrix}
        1 & \nicefrac{1}{2} \\
        \nicefrac{1}{2} & \nicefrac{1}{3}
    \end{pmatrix}\), calculez \(\det A\) et \(N\paren{A}\).
\end{enumerate}
\end{q}

Dans toute la suite du problème, pour tout entier \(n\geq2\), on désigne par \(H_n\) la matrice de Hilbert d'ordre \(n\) définie par \[H_n=\paren{\dfrac{1}{i+j-1}}_{1\leq i,j\leq n}=\begin{pmatrix}
1 & \nicefrac{1}{2} & \dots & \nicefrac{1}{n} \\
\nicefrac{1}{2} & \nicefrac{1}{3} & \dots & \nicefrac{1}{n+1} \\
\vdots & \vdots &  & \vdots \\
\nicefrac{1}{n} & \nicefrac{1}{n+1} & \dots & \nicefrac{1}{2n-1}
\end{pmatrix}\] ou encore \(H_n=\paren{a_{i\,j}}_{1\leq i,j\leq n}\) avec \(a_{i\,j}=\dfrac{1}{i+j-1}\).

Pour simplifier, on notera \(q_n\) la fonction \(q_{H_n}:\R^n\to\R\) : \(\quantifs{\forall x\in\R^n}q_n\paren{x}=q_{H_n}\paren{x}=\ps{H_nx}{x}\).

\subsection*{Sur les valeurs propres de \(H_n\)}
\addcontentsline{toc}{subsection}{Sur les valeurs propres de \(H_n\)}

\begin{q}
Soit \(x=\paren{x_j}_{1\leq j\leq n}\in\R^n\).

\begin{enumerate}
    \item Montrez que \[q_n\paren{x}=\ps{H_nx}{x}=\sum_{i=1}^n\paren{\sum_{j=1}^n\dfrac{x_j}{i+j-1}}x_i=\sum_{1\leq i,j\leq n}\dfrac{x_ix_j}{i+j-1}.\] \\
    \item Développez \(\paren{\sum_{i=1}^nx_it^{i-1}}\paren{\sum_{j=1}^nx_jt^{j-1}}\) où \(t\) est une variable réelle. \\
    \item Montrez que \(q_n\paren{x}=\int_0^1\paren{\sum_{j=1}^nx_jt^{j-1}}^2\odif{t}\). \\
    \item Montrez que \[\quantifs{\forall x\in\R^n}q_n\paren{x}\geq0\] et \[\quantifs{\forall x\in\R^n}q_n\paren{x}=0\ssi x=0.\] Que peut-on en déduire concernant les valeurs propres de \(H_n\) ?
\end{enumerate}
\end{q}

\begin{q}
\begin{enumerate}
    \item Soit \(P\paren{t}=\sum_{k=0}^ma_kt^k\) un polynôme à coefficients complexes. \\ Montrez que \(\int_{-1}^1P\paren{t}\odif{t}=-\i\int_0^\pi P\paren{\e{\i\theta}}\e{\i\theta}\odif{\theta}\). \\ \textit{On pourra expliciter \(\int_{-1}^1t^k\odif{t}\) et \(-\i\int_0^\pi\e{\i k\theta}\e{\i\theta}\odif{\theta}\).} \\
    \item En gardant les notations introduites dans la question 6 et en notant \(Q\paren{t}=\sum_{k=1}^nx_kt^{k-1}\), montrez que pour tout \(x\in\R^n\), on a \[0\leq q_n\paren{x}=\int_0^1Q^2\paren{t}\odif{t}\leq\int_0^\pi\abs{\sum_{k=1}^nx_k\e{\i\paren{k-1}\theta}}^2\odif{\theta}.\]
\end{enumerate}
\end{q}

\begin{q}
\begin{enumerate}
    \item Soit \(f\) une fonction continue sur \(\intervii{0}{\pi}\) à valeurs complexes telle que \(\int_0^\pi\abs{f\paren{\theta}}\odif{\theta}=\abs{\int_0^\pi f\paren{\theta}\odif{\theta}}\). \\ Montrez que \(f\) est à valeurs dans \(\Rp\). \\
    \item Soit \(P\paren{t}=\sum_{k=0}^ma_kt^k\) un polynôme à coefficients complexes. \\ Calculez, pour \(k\in\interventierii{0}{n}\), \(\int_0^{2\pi}P\paren{\e{\i\theta}}\e{-\i k\theta}\odif{\theta}\). \\ Déduisez-en que s'il existe \(\alpha\in\R\) tel que \(\quantifs{\tpt\theta}P\paren{\e{\i\theta}}\e{-\i\alpha}\in\Rp\), alors \(P\) est constant. \\
    \item Montrez que les inégalités de la question 7b sont strictes pour \(x\not=0\).
\end{enumerate}
\end{q}

\begin{q}
Montrez que \[\quantifs{\forall x\in\R^n}0\leq q_n\paren{x}\leq\pi\norme{x}^2,\] les inégalités étant strictes pour \(x\not=0\).
\end{q}

\begin{q}
Pour tout entier \(n\geq2\), on note \[\mu_n=\min\Sp{H_n}\qquad\text{et}\qquad\rho_n=\max\Sp{H_n}.\]

\begin{enumerate}
    \item Explicitez \(\mu_2\) et \(\rho_2\). Montrez que pour tout \(n\geq2\), on a \[0<\mu_n<\rho_n<\pi.\] \\
    \item Montrez que \(q_n\paren{\Omega_n}=\intervii{\mu_n}{\rho_n}\). \\
    \item Calculez \(\ps{H_n\epsilon_n}{\epsilon_n}\) où \(\epsilon_n\) désigne le dernier vecteur de la base canonique de \(\M{n\,1}[\R]\) : \(\epsilon_n=\begin{pmatrix}
        0 \\
        \vdots \\
        0 \\
        1
    \end{pmatrix}\). \\ Déduisez-en la limite de \(\mu_n\) lorsque \(n\to\pinf\).
\end{enumerate}
\end{q}

\subsection*{Sur le déterminant de \(H_n\)}
\addcontentsline{toc}{subsection}{Sur le déterminant de \(H_n\)}

\(H_n\) désigne toujours la matrice de Hilbert d'ordre \(n\), pour \(n\geq2\).

\begin{q}[Une fraction rationnelle]
On considère la fraction rationnelle \[R_n\paren{x}=\dfrac{\ds\prod_{k=1}^n\paren{x-k}}{\ds\prod_{k=0}^n\paren{x+k}}.\]

On admettra qu'il existe des réels \(\lambda_{0\,n},\dots,\lambda_{n\,n}\) tels que \[\quantifs{\forall x\in\R\excluant\accol{0,\dots,-n}}R_n\paren{x}=\sum_{k=0}^n\dfrac{\lambda_{k\,n}}{x+k},\] cette décomposition (en éléments simples) de \(R_n\) étant unique.

Exprimez le coefficient \(\lambda_{n\,n}\) de \(\dfrac{1}{x+n}\) à l'aide de \(\paren{2n}!\) et de \(n!\).
\end{q}

\begin{q}[Matrice \(A_n\)]
Pour \(n\geq2\), on considère la matrice \(A_n\) définie par \(A_n=\paren{a_{i\,j}}_{1\leq i,j\leq n}\) avec \[a_{i\,j}=\begin{dcases}
\dfrac{1}{i+j-1} &\text{pour }1\leq i\leq n-1\text{ et }1\leq j\leq n \\
R_{n-1}\paren{j} &\text{pour }i=n\text{ et }1\leq j\leq n
\end{dcases}\] où les \(R_p\) ont été définis plus haut.

\begin{enumerate}
    \item Montrez que pour \(1\leq i\leq n\), on a \[R_{n-1}\paren{i}=\sum_{j=1}^n\lambda_{j-1\,n-1}h_{i\,j}\] puis déduisez-en que \(\det A_n=\binom{n-1}{2\paren{n-1}}\det H_n\). \\
    \item Montrez que \[\det A_n=\dfrac{\det H_{n-1}}{\paren{2n-1}\binom{n-1}{2\paren{n-1}}}.\] Déduisez-en l'expression de \(\det H_n\) en fonction de \(\det H_{n-1}\). \\
    \item Montrez que pour tout \(n\geq2\), \[\det H_n\not=0,\] puis que \[\dfrac{1}{\det H_n}\in\Ns.\]
\end{enumerate}
\end{q}

\begin{q}[Calcul de \(\det H_n\)]
En notant, pour tout \(n\in\Ns\) : \(\Phi_n=\prod_{k=1}^nk!\), montrez que \[\quantifs{\forall n\geq2}\det H_n=\dfrac{\Phi_{n-1}^4}{\Phi_{2n-1}}.\]
\end{q}


\chapter{Fonctions vectorielles}

\minitoc

\legendeexercices

\begin{exoss}
Pour \(x\in\R\) et \(n\in\Ns\), on pose \[D_n=\begin{vmatrix}
x & 1 & 0 & 0 & \dots & 0 \\
\frac{x^2}{2!} & x & 1 & 0 & \dots & 0 \\
\frac{x^3}{3!} & \frac{x^2}{2!} & x & 1 & \ddots & \vdots \\
\vdots & \vdots & \ddots & \ddots & \ddots & 0 \\
\frac{x^{n-1}}{\paren{n-1}!} & \frac{x^{n-2}}{\paren{n-2}!} & \dots & \frac{x^2}{2!} & x & 1 \\
\frac{x^n}{n!} & \frac{x^{n-1}}{\paren{n-1}!} & \frac{x^{n-2}}{\paren{n-2}!} & \dots & \frac{x^2}{2!} & x
\end{vmatrix}.\]

\begin{enumerate}
    \item Justifiez que \(D_n\) est dérivable et calculez \(D_n\prim\). \\
    \item Donnez une expression générale de \(D_n\paren{x}\).
\end{enumerate}
\end{exoss}

\begin{exoss}
Soient \(a>0\) et \(f:\intervii{0}{a}\to\R\) dérivable sur \(\intervii{0}{a}\) et telle que \(f\paren{0}=f\prim\paren{0}=f\paren{a}=0\).

Montrez qu'il existe \(c\in\intervee{0}{a}\) tel que la tangente au point d'abscisse \(c\) passe par l'origine du repère.
\end{exoss}

\begin{exoss}
Soient \(f,g:\intervii{a}{b}\to\R\) deux fonctions continues sur \(\intervii{a}{b}\) et dérivables sur \(\intervee{a}{b}\).

Montrez qu'il existe \(c\in\intervee{a}{b}\) tel que \[\paren{f\paren{b}-f\paren{a}}g\prim\paren{c}=\paren{g\paren{b}-g\paren{a}}f\prim\paren{c}.\]
\end{exoss}

\begin{exoss}
Soient \(f:\R\to\R\) de classe \(\classe{2}\) telle que \(f\paren{x}\tendqd{x\to\pinf}0\) et \(x_0\in\R\) tel que \(f\paren{x_0}f\prim\paren{x_0}\geq0\).

\begin{enumerate}
    \item Montrez qu'il existe \(x_1>x_0\) tel que \(f\prim\paren{x_1}=0\). \\
    \item Montrez qu'il existe \(x_2>x_1\) tel que \(f\seconde\paren{x_2}=0\).
\end{enumerate}
\end{exoss}

\begin{exoss}
Soient \(f\) une fonction \(n\) fois dérivable sur un intervalle \(I\) et \(\paren{x_0,\dots,x_n}\in I^{n+1}\) tel que \(x_0<\dots<x_n\) et \(f\paren{x_0}=\dots=f\paren{x_n}=0\).

Montrez qu'il existe \(c\in I\) tel que \(f\deriv{n}\paren{c}=0\).

Application : soit \(P\) une fonction polynôme de degré \(n\) ; montrez que l'équation \(\e{x}=P\paren{x}\) a au plus \(n+1\) solutions.
\end{exoss}

\begin{exoss}
Soit \(f\) une fonction de classe \(\classe{2}\) sur \(\R\) telle que \(f\) et \(f\seconde\) aient pour limite \(0\) en \(\pinf\).

Montrez que \(f\prim\) a aussi pour limite \(0\) en \(\pinf\).
\end{exoss}

\begin{exoss}
\begin{enumerate}
    \item Soit \(n\in\Ns\). Calculez la dérivée \(n\)-ème de la fonction \(x\mapsto x^{n-1}\ln\paren{1+x}\) définie sur \(\intervee{-1}{\pinf}\). \\
    \item Soient \(n\in\Ns\) et \(f\) une fonction \(n\) fois dérivable sur \(\R\). Calculez la dérivée \(n\)-ème de \[x\mapsto x^{n-1}f\paren{\dfrac{1}{x}}.\] \\
    \item Déduisez-en les dérivées \(n\)-èmes de \[x\mapsto x^{n-1}\ln\paren{1+\dfrac{1}{x}}\qquad\text{et}\qquad x\mapsto x^{n-1}\e{\nicefrac{1}{x}}.\]
\end{enumerate}
\end{exoss}

\begin{exoss}~\\
Soit \(F:t\mapsto\int_t^{t^2}\dfrac{1}{\ln u}\odif{u}\).

\begin{enumerate}
    \item Déterminez l'ensemble de définition de \(F\). \\
    \item Soit \(\phi:u\mapsto\dfrac{1}{\ln u}-\dfrac{1}{u-1}\). Montrez que \(\phi\) est prolongeable en une fonction continue sur \(\intervee{0}{\pinf}\). \\ Déduisez-en que \(\lim_{t\to1}\int_t^{t^2}\dfrac{1}{\ln u}\odif{u}=\ln2\). \\
    \item Montrez que \(\quantifs{\tpt t\in\intervee{1}{\pinf}}\dfrac{t^2-t}{2\ln t}\leq\int_t^{t^2}\dfrac{1}{\ln u}\odif{u}\leq\dfrac{t^2-t}{\ln t}\). Déduisez-en \(\lim_{t\to\pinf}\int_t^{t^2}\dfrac{1}{\ln u}\odif{u}\). \\
    \item Étudiez la fonction \(F\) (on la prolongera par continuité chaque fois que c'est possible) et donnez l'allure de sa courbe représentative. \\
    \item Soit \(f:x\mapsto\dfrac{x-1}{\ln x}\), prolongée par continuité en \(0\) et en \(1\). Montrez que \(\quantifs{\tpt t\in\intervee{0}{1}}\int_0^tf=F\paren{t}\). Déduisez-en la valeur de \(\int_0^1f\).
\end{enumerate}
\end{exoss}

\begin{exoss}
Soit \(\paren{a,b}\in\R^2\) tel que \(a<b\).

Montrez que pour toute application \(f\in\ensclasse{1}{\intervii{a}{b}}{\R}\), on a \[\max_{\intervii{a}{b}}\abs{f}\leq\int_a^b\abs{f\prim}+\dfrac{1}{b-a}\int_a^b\abs{f}.\]

Montrez que cette inégalité est également valable pour toute application \(f\in\ensclasse{1}{\intervii{a}{b}}{\C}\).
\end{exoss}

\begin{exoss}
Soient \(\paren{a,b}\in\R^2\) tel que \(a<b\) et \(f\in\ensclasse{2}{\intervii{a}{b}}{E}\) telle que \(f\paren{a}=f\paren{b}=0\). On pose \(M=\sup_{\intervii{a}{b}}\norme{f\seconde}\).

\begin{enumerate}
    \item Montrez que \(\quantifs{\tpt x\in\intervii{a}{b}}\norme{f\paren{x}}\leq\dfrac{M}{2}\paren{x-a}\paren{b-x}\). \\
    \item Montrez que \(\norme{\int_a^bf}\leq\dfrac{M\paren{b-a}^3}{12}\). \\
    \item Déduisez-en une majoration de l'erreur commise lors du calcul approché de \(\int_a^bf\) par la méthode des trapèzes.
\end{enumerate}
\end{exoss}

\begin{exoss}
Soit \(f\) continue sur \(\intervii{a}{b}\) avec \(a<b\).

Montrez que \(\lim_{n\to\pinf}\paren{\int_a^b\norme{f\paren{t}}^n\odif{t}}^{\nicefrac{1}{n}}=\sup_{\intervii{a}{b}}\norme{f}\).
\end{exoss}

\begin{exoss}
Soit \(a>1\).

Montrez que la fonction \(f:x\mapsto\ln\paren{1-2a\cos x+a^2}\) est continue sur \(\intervii{0}{\pi}\).

Sans chercher à utiliser une primitive, montrez \(\int_0^\pi f=2\pi\ln a\).
\end{exoss}

\begin{exoss}
Soit \(z\in\C\excluant\U\).

Décomposez en éléments simples la fraction \(\dfrac{nX^{n-1}}{X^n-1}\) puis calculez \(\int_0^{2\pi}\dfrac{1}{z-\e{\i t}}\odif{t}\).
\end{exoss}

\begin{exoss}
On considère une application \(f:\R\to\C\) de classe \(\classe{1}\) et \(2\pi\)-périodique telle que \(\quantifs{\tpt t\in\R}f\paren{t}\not=0\).

Montrez que \(\int_0^{2\pi}\dfrac{f\prim}{f}\in2\pi\Z\).
\end{exoss}

\begin{exoss}
Soient \(E\) un espace vectoriel de dimension finie et \(f\in\ensclasse{\infty}{\intervii{a}{b}}{E}\).

On suppose qu'il existe \(P\in\poly[\R]\) de degré impair tel que \(\quantifs{\forall n\in\N;\forall x\in\intervii{a}{b}}\norme{f\deriv{n}\paren{x}}\leq\abs{P\paren{x}}\).

Montrez que \(f=0\).
\end{exoss}

\begin{exoss}[Égalité de la moyenne généralisée]
Soient \(f,g\in\ensclasse{0}{\intervii{a}{b}}{\R}\) telles que \(g\) soit positive.

Montrez qu'il existe \(c\in\intervii{a}{b}\) tel que \[\int_a^bf\paren{t}g\paren{t}\odif{t}=f\paren{c}\int_a^bg\paren{t}\odif{t}.\]
\end{exoss}

\begin{exoss}[Égalité de Taylor-Lagrange]
Soit \(f\in\ensclasse{n+1}{\intervii{a}{b}}{E}\).

Montrez qu'il existe \(c\in\intervee{a}{b}\) tel que \[f\paren{b}=\sum_{k=0}^n\dfrac{\paren{b-a}^k}{k!}f\deriv{k}\paren{a}+\dfrac{\paren{b-a}^{n+1}}{\paren{n+1}!}f\deriv{n+1}\paren{c}.\]
\end{exoss}

\begin{exosss}
Soit \(f\in\ensclasse{2}{\intervii{a}{b}}{\R}\).

Montrez qu'il existe \(c\in\intervee{a}{b}\) tel que \[\dfrac{f\paren{a}+f\paren{b}}{2}=f\paren{\dfrac{a+b}{2}}+\dfrac{\paren{b-a}^2}{8}f\seconde\paren{c}.\]
\end{exosss}

\begin{exosss}
Soit \(f\in\ensclasse{2}{\intervii{a}{b}}{\R}\) telle que \(f\paren{a}=f\prim\paren{a}\) et \(f\paren{b}=f\prim\paren{b}\).

Montrez qu'il existe \(c\in\intervee{a}{b}\) tel que \(f\seconde\paren{c}=f\paren{c}\).
\end{exosss}

\begin{exoss}
On considère l'application \(u:t\mapsto\e{-\nicefrac{t^2}{2}}\) et on pose pour \(n\in\N\), \(H_n\paren{t}=\e{\nicefrac{t^2}{2}}u\deriv{n}\paren{t}\).

\begin{enumerate}
    \item Montrez que pour tout \(n\in\N\), \(H_n\) est un polynôme de degré \(n\). \\
    \item Montrez que si \(f\in\ensclasse{0}{\intervie{a}{\pinf}}{\R}\) est dérivable sur \(\intervie{a}{\pinf}\) et si \(f\paren{t}\tendqd{t\to\pinf}f\paren{a}\), alors il existe \(c\in\intervee{a}{\pinf}\) tel que \(f\prim\paren{c}=0\). \\
    \item Montrez que pour tout \(n\in\N\), \(H_n\) est scindé à racines simples sur \(\R\).
\end{enumerate}
\end{exoss}

\begin{exoss}
Soit \(f\in\ensclasse{2}{\R}{\R}\) telle que \(f\) et \(f\seconde\) soient bornées. On note \(M_0=\sup_\R\abs{f}\) et \(M_2=\sup_\R\abs{f\seconde}\).

\begin{enumerate}
    \item Montrez que \(\quantifs{\tpt\paren{x,h}\in\R^2}\abs{f\paren{x+h}-f\paren{x}-hf\prim\paren{x}}\leq\dfrac{h^2}{2}M_2\) et \\ \(\abs{f\paren{x-h}-f\paren{x}+hf\prim\paren{x}}\leq\dfrac{h^2}{2}M_2\). \\
    \item Déduisez-en que \(\quantifs{\tpt x\in\R;\tpt h>0}\abs{f\prim\paren{x}}\leq\dfrac{M_0}{h}+\dfrac{h}{2}M_2\). \\
    \item Montrez que \(f\prim\) est aussi bornée sur \(\R\) et que \(M_1=\sup_\R\abs{f\prim}\leq\sqrt{2M_0M_2}\).
\end{enumerate}
\end{exoss}

\begin{exosss}
Soit \(f\) une fonction de classe \(\classe{\infty}\) sur \(\R\) telle que

\begin{itemize}
    \item \(\quantifs{\tpt n\in\N}f\deriv{n}\paren{0}=0\) \\
    \item il existe \(a>0\) tel que \(\quantifs{\tpt n\in\N;\tpt t\in\R}\abs{f\deriv{n}\paren{t}}\leq a^nn!\). \\
\end{itemize}

\begin{enumerate}
    \item Montrez que \(f\) est la fonction nulle sur \(\intervii{\dfrac{-1}{a}}{\dfrac{1}{a}}\). \\
    \item Montrez que \(f\) est la fonction nulle sur \(\R\).
\end{enumerate}
\end{exosss}

\begin{exo}[Oral CCINP, 1]
Soit \(f:\R\to\R\) dérivable telle que \(f\prim\paren{x}\tendqd{x\to\pinf}\pinf\).

Montrez que \(f\paren{x}\tendqd{x\to\pinf}\pinf\).
\end{exo}

\begin{exo}[Oral TPE, 2]
Cherchez les \(f:\R\to\R\) dérivables telles que \(f\paren{0}=1\), \(f\prim\paren{0}>1\) et \(\quantifs{\forall x\in\R}f\prim\paren{f\paren{x}}f\prim\paren{x}=1\).
\end{exo}

\begin{exo}[Oral CCINP, 3]
Pour \(n\in\Ns\), on pose \(f_n:x\mapsto x^n\ln x\).

Montrez que \(\lim_{n\to\pinf}f_n\deriv{n}\paren{\dfrac{1}{n}}=\gamma\) où \(\gamma\) est la constante d'Euler-Mascheroni.
\end{exo}

\begin{exo}[Oral CCMP, 4]
Soit \(f:\Rps\to\R\) continue. On pose \(g:x\mapsto xf\paren{x}\) et \(h:x\mapsto f\paren{\dfrac{1}{x}}\).

Montrez que \(g\) est convexe sur \(\Rps\) ssi \(h\) l'est aussi.
\end{exo}

\begin{exo}[Oral CCMP, 5]
Soit \(f:\intervii{0}{1}\to\R\) dérivable telle que \(\quantifs{\tpt x\in\intervii{0}{1}}\paren{f\paren{x},f\prim\paren{x}}\not=\paren{0,0}\).

Montrez que l'ensemble des zéros de \(f\) est fini.
\end{exo}

\begin{exo}[Oral CCMP, 6]
Soient \(\paren{a,b}\in\R^2\) tel que \(a<b\) et \(E\) l'ensemble des fonctions continues sur \(\intervii{a}{b}\) et strictement positives.

Pour \(f\in E\), on pose \(\phi\paren{f}=\int_a^bf\times\int_a^b\dfrac{1}{f}\).

Déterminez \(\phi\paren{E}\).
\end{exo}

\begin{exo}[Oral CCMP, 7]
Soit \(E=\accol{f\in\ensclasse{0}{\intervii{0}{1}}{\Rp}\tq\quantifs{\forall\paren{x,y}\in\intervii{0}{1}^2}\abs{f\paren{x}-f\paren{y}}\geq\abs{x-y}}\).

Déterminez \(\inf_{f\in E}\int_0^1f\).
\end{exo}

\begin{exo}[Oral Centrale, 8]
Soient \(f\in\ensclasse{2}{\Rp}{\R}\) majorée et \(\alpha>0\) tel que \(\quantifs{\tpt x\in\Rp}f\seconde\paren{x}\geq\alpha f\paren{x}\).

\begin{enumerate}
    \item Montrez que \(f\prim\) est croissante et \(\lim_{\pinf}f\prim=0\). \\
    \item Montrez que \(\lim_{\pinf}f=0\). \\
    \item Montrez que \(\quantifs{\tpt x\in\Rp}f\paren{x}\leq f\paren{0}\e{-x\sqrt{\alpha}}\).
\end{enumerate}
\end{exo}

\end{document}