\chapter{Séries entières}

\minitoc

Dans ce chapitre, on s'intéresse aux séries de fonctions de la forme \(x\mapsto a_nx^n\), qu'on appelle des séries entières.

Par abus de notations, on les note sous la forme \(\sum_{n\geq0}a_nx^n\).

Lorsque la variable est complexe, on la note systématiquement \(z\) et on parle des séries \(\sum_{n\geq0}a_nz^n\).

\section{Convergence simple d'une série entière}

\subsection{Rayon de convergence}

On commence par le lemme d'Abel.

\begin{prop}
Soient \(\paren{a_n}\) une suite de complexes et \(z_0\) un complexe.

Si la suite \(\paren{a_nz_0^n}\) est bornée, alors pour \(z\in\C\) tel que \(\abs{z}<\abs{z_0}\), la série numérique \(\sum_{n\geq0}a_nz^n\) converge absolument.
\end{prop}

\begin{defi}
Soit \(\paren{a_n}\) une suite de complexes. On pose \(M=\accol{r\in\Rp\tq\text{la suite }\paren{a_nr^n}\text{ est bornée}}\).

\(M\) est une partie non-vide de \(\R\) car elle contient \(0\).

Si elle est majorée, on appelle rayon de convergence de la série entière \(\sum_{n\geq0}a_nz^n\) le réel \(R=\sup M\).

Sinon, on pose \(R=\pinf\).
\end{defi}

On peut préciser la convergence simple d'une série entière.

\begin{prop}\thlabel{prop:convergenceSimpleSerieEntiere}
Soit \(\paren{a_n}\) une suite de complexes. On pose \(R\) le rayon de convergence de la série entière \(\sum_{n\geq0}a_nz^n\).

Pour tout \(z\in\C\),

\begin{itemize}
    \item si \(\abs{z}<R\), alors la série numérique \(\sum_{n\geq0}a_nz^n\) converge absolument \\
    \item si \(\abs{z}>R\), alors la série numérique \(\sum_{n\geq0}a_nz^n\) diverge grossièrement \\
    \item si \(\abs{z}=R\), alors on ne peut rien dire \textit{a priori}.
\end{itemize}
\end{prop}

\begin{defi}
Soit \(\paren{a_n}\) une suite de complexes. On pose \(R\) le rayon de convergence de la série entière \(\sum_{n\geq0}a_nz^n\).

Si \(R>0\), on appelle disque ouvert de convergence l'ensemble \(\disqueo{0}{R}\subset\C\) et intervalle ouvert de convergence l'ensemble \(\intervee{-R}{R}\subset\R\).
\end{defi}

\begin{ex}
\begin{itemize}
    \item Le rayon de convergence de la série entière \(\sum_{n\geq0}z^n\) est égal à \(1\) et \(\quantifs{\tpt z\in\disqueo{0}{1}}\sum_{n=0}^{\pinf}z^n=\dfrac{1}{1-q}\). \\
    \item Le rayon de convergence de la série entière \(\sum_{n\geq0}\dfrac{z^n}{n!}\) est égal à \(\pinf\) et \(\quantifs{\tpt z\in\C}\sum_{n=0}^{\pinf}\dfrac{z^n}{n!}=\e{z}\).
\end{itemize}
\end{ex}

\subsection{Détermination du rayon de convergence}

La réciproque de la \thref{prop:convergenceSimpleSerieEntiere} est vraie.

\begin{prop}
Soient \(\paren{a_n}\) une suite de complexes et \(R\) un réel strictement positif.

Si

\begin{itemize}
    \item \(\quantifs{\tpt z\in\C\text{ tel que }\abs{z}<R}\text{la série numérique }\sum_{n\geq0}a_nz^n\) converge \\
    \item \(\quantifs{\tpt z\in\C\text{ tel que }\abs{z}>R}\text{la série numérique }\sum_{n\geq0}a_nz^n\) diverge
\end{itemize}

alors \(R\) est le rayon de convergence de la série entière \(\sum_{n\geq0}a_nz^n\).
\end{prop}

De même, on peut donner quelques résultats pratiques découlant de la définition.

\begin{prop}
Soient \(\paren{a_n}\) une suite de complexes et \(R\) un réel strictement positif.

Si

\begin{itemize}
    \item \(\quantifs{\tpt z\in\C\text{ tel que }\abs{z}<R}\text{la série numérique }\sum_{n\geq0}a_nz^n\) converge \\
    \item il existe \(z_0\in\C\) tel que \(\abs{z_0}=R\) et la série \(\sum_{n\geq0}a_nz_0^n\) diverge
\end{itemize}

alors \(R\) est le rayon de convergence de la série entière \(\sum_{n\geq0}a_nz^n\).
\end{prop}

\begin{prop}
Soient \(\paren{a_n}\) une suite de complexes et \(R\) un réel strictement positif.

Si

\begin{itemize}
    \item il existe \(z_0\in\C\) tel que \(\abs{z_0}=R\) et la série \(\sum_{n\geq0}a_nz_0^n\) converge \\
    \item \(\quantifs{\tpt z\in\C\text{ tel que }\abs{z}>R}\text{la série numérique }\sum_{n\geq0}a_nz^n\) diverge \\
\end{itemize}

alors \(R\) est le rayon de convergence de la série entière \(\sum_{n\geq0}a_nz^n\).
\end{prop}

\begin{exo}
Donnez le rayon de convergence des séries entières suivantes : \[\sum_{n\geq0}nz^n\qquad\sum_{n\geq1}\dfrac{z^n}{n}\qquad\sum_{n\geq0}n!\,z^n\qquad\sum_{n\geq0}2^nz^{2n}\qquad\sum_{n\geq0}\cos^2\paren{n}z^n.\]
\end{exo}

\subsection{Comparaison de séries entières}

\begin{prop}
Soient \(\paren{a_n},\paren{b_n}\) deux suites de complexes.

On pose \(R_a,R_b\) les rayons de convergence des séries entières \(\sum_{n\geq0}a_nz^n\) et \(\sum_{n\geq0}b_nz^n\).

Si \(a_n\egqd{n\to\pinf}\O{b_n}\), alors \(R_a\geq R_b\).

En particulier, si \(\abs{a_n}\leq\abs{b_n}\), alors \(R_a\geq R_b\).

Si \(a_n\simqd{n\to\pinf}b_n\), alors \(R_a=R_b\).
\end{prop}

\begin{rem}
Comme le rayon de convergence se calcule en référence à la convergence absolue de la série, il suffit que \(\abs{a_n}\simqd{n\to\pinf}\abs{b_n}\) pour avoir \(R_a=R_b\).
\end{rem}

\begin{exo}
Donnez le rayon de convergence de la série entière \(\sum_{n\geq0}\dfrac{n}{n^2+1}z^n\).
\end{exo}

\begin{exo}
Même question avec la série entière \(\sum_{n\geq0}\paren{1+\dfrac{1}{n}}^{n^2}z^n\).
\end{exo}

Un cas favorable très courant qui permet de calculer le rayon de convergence par utilisation de la règle de D'Alembert, particulièrement utile quand le coefficient \(a_n\) s'écrit à l'aide de produits ou de quotients.

\begin{prop}
Soit \(\paren{a_n}\) une suite de complexes qui ne s'annule pas.

Si la suite \(\paren{\dfrac{\abs{a_{n+1}}}{\abs{a_n}}}\) converge

\begin{itemize}
    \item vers un réel \(l>0\), alors \(R_a=\dfrac{1}{l}\) \\
    \item vers \(0\), alors \(R_a=\pinf\).
\end{itemize}
\end{prop}

\begin{exo}
Donnez le rayon de convergence de la série entière \(\sum_{n\geq0}\dfrac{2^n+1}{n^2+1}z^n\).
\end{exo}

\begin{exo}
Même question avec la série entière \(\sum_{n\geq0}\binom{n}{2n}z^{2n}\).
\end{exo}

\subsection{Opérations sur les séries entières}

\begin{prop}
Soient \(\paren{a_n},\paren{b_n}\) deux suites de complexes et \(\lambda\in\C\).

On pose \(R_a,R_b\) les rayons de convergence des séries entières \(\sum_{n\geq0}a_nz^n\) et \(\sum_{n\geq0}b_nz^n\).

On a \[R_{a+b}\geq\min\paren{R_a,R_b}\] et \[R_{\lambda a}\geq R_a\] avec égalité si \(\lambda\not=0\)
\end{prop}

Par conséquent, l'ensemble des séries entières qui convergent sur un disque \(\disqueo{0}{R}\) est un \(\C\)-espace vectoriel.

\begin{prop}
Soient \(\paren{a_n},\paren{b_n}\) deux suites de complexes.

On pose \(R_a,R_b\) les rayons de convergence des séries entières \(\sum_{n\geq0}a_nz^n\) et \(\sum_{n\geq0}b_nz^n\).

On pose \(\paren{c_n}\) le produit de Cauchy des suites \(\paren{a_n}\) et \(\paren{b_n}\).

Alors la série entière \(\sum_{n\geq0}c_nz^n\) est le produit de Cauchy des séries \(\sum_{n\geq0}a_nz^n\) et \(\sum_{n\geq0}b_nz^n\) et \(R_c\geq\min\paren{R_a,R_b}\).
\end{prop}

\begin{exo}
Calculez le produit de Cauchy de la série \(\sum_{n\geq0}z^n\) avec elle-même.

Donnez la valeur de \(\sum_{n=0}^{\pinf}nz^n\) quand elle a un sens.
\end{exo}

\begin{rem}
Le résultat énoncé ici est utilisable avec des séries \(\sum_{n\geq0}a_nz^n\) : on commence à \(n=0\) !

Dans le cas général, on s'y ramène en ajoutant des termes nuls au début des séries : si on veut faire le produit de Cauchy de deux séries \(\sum_{n\geq n_1}a_nz^n\) et \(\sum_{n\geq n_2}b_nz^n\), on pose \(a_0=\dots=a_{n_1-1}=0\) et \(b_0=\dots=b_{n_2-1}=0\), puis on peut appliquer le résultat précédent : pour tout \(z\in\C\) tel que \(\abs{z}<\min\paren{R_a,R_b}\), \[\sum_{n=n_1}^{\pinf}a_nz^n\times\sum_{n=n_2}^{\pinf}b_nz^n=\sum_{n=0}^{\pinf}a_nz^n\times\sum_{n=0}^{\pinf}b_nz^n=\sum_{n=0}^{\pinf}c_nz^n\] où \(\quantifs{\tpt n\in\N}c_n=\sum_{k=0}^na_kb_{n-k}=a_0b_n+a_1b_{n-1}+\dots+a_{n-1}b_1+a_nb_0\).

Maintenant on peut prendre en compte les termes nuls.

Si \(n<n_1+n_2\), alors dans cette somme qu'on coupe en deux, tous les termes sont nuls : \[c_n=\underbrace{a_0b_n+\dots+a_{n_1-1}b_{n+1-n_1}}_{\quantifs{\forall i\in\interventierii{0}{n_1-1}}a_i=0}+\underbrace{a_{n_1}b_{n-n_1}+\dots+a_nb_0}_{\quantifs{\forall i\in\interventierii{n_1}{n}}n-i\leq n-n_1\text{ donc }b_{n-i}=0}=0\] donc \(\sum_{n=n_1}^{\pinf}a_nz^n\times\sum_{n=n_2}^{\pinf}b_nz^n=\sum_{n=n_1+n_2}^{\pinf}c_nz^n\) où \(\quantifs{\tpt n\geq n_1+n_2}c_n=\sum_{k=n_1}^{n-n_2}a_kb_{n-k}\).

Ne retenons pas ça ! Retenons l'idée et ré-appliquons-la à chaque fois car, en pratique, on a très souvent \(\paren{n_1,n_2}=\paren{1,0}\) ou \(\paren{1,1}\).
\end{rem}

\section{Propriétés de la fonction somme d'une série entière}

\subsection{Convergence uniforme et continuité}

\begin{theo}
Soit \(\sum_{n\geq0}a_nz^n\) une série entière de rayon de converge \(R>0\).

La série de fonctions \(\sum_{n\geq0}a_nz^n\) converge normalement (donc uniformément) sur tout compact inclus dans \(\disqueo{0}{R}\).
\end{theo}

D'après le théorème de continuité des séries de fonctions, on en déduit la continuité de la fonction somme.

\begin{theo}
Soit \(\sum_{n\geq0}a_nz^n\) une série entière de rayon de converge \(R>0\).

La fonction somme \(z\mapsto\sum_{n=0}^{\pinf}a_nz^n\) est continue sur \(\disqueo{0}{R}\).
\end{theo}

\subsection{Primitivation et dérivation}

\begin{lem}
Soit \(\paren{a_n}\) une suite de complexes.

Les séries entières \(\sum_{n\geq0}a_nz^n\) et \(\sum_{n\geq0}na_nz^n\) ont le même rayon de convergence.

Plus généralement, pour tout polynôme \(P\) non-nul, la série entière \(\sum_{n\geq0}P\paren{z}a_nz^n\) a le même rayon de convergence.
\end{lem}

D'après le théorème de dérivabilité des séries de fonctions, on en déduit la dérivabilité de la fonction somme.

\begin{theo}
Soit \(\sum_{n\geq0}a_nz^n\) une série entière de rayon de converge \(R>0\).

La fonction somme \(f:x\mapsto\sum_{n=0}^{\pinf}a_nx^n\) est de classe \(\classe{\infty}\) sur \(\intervee{-R}{R}\).

De plus, on a \[\quantifs{\forall k\in\N;\forall x\in\intervee{-R}{R}}f\deriv{k}\paren{x}=\sum_{n=k}^{\pinf}\dfrac{n!}{\paren{n-k}!}a_nx^{n-k}.\]

On en déduit que \(\quantifs{\tpt k\in\N}a_k=\dfrac{f\deriv{k}\paren{0}}{k!}\).
\end{theo}

On en déduit le théorème de primitivation.

\begin{theo}
Soit \(\sum_{n\geq0}a_nz^n\) une série entière de rayon de converge \(R>0\). On note \(f\) sa somme.

La fonction somme \(x\mapsto\sum_{n=0}^{\pinf}a_n\dfrac{x^{n+1}}{n+1}\) est de classe \(\classe{\infty}\) sur \(\intervee{-R}{R}\) et est une primitive de \(f\) sur \(\intervee{-R}{R}\) (c'est d'ailleurs la primitive de \(f\) qui s'annule en \(0\)).
\end{theo}

\begin{exo}
Donnez le rayon de convergence de la série entière \(\sum_{n\geq0}nx^n\) et retrouvez la valeur de sa somme.
\end{exo}

\begin{exo}
Même exercice avec \(\sum_{n\geq0}n^2x^n\).
\end{exo}

\begin{exo}
Même exercice avec \(\sum_{n\geq1}\dfrac{1}{n}x^n\).
\end{exo}

\subsection{Convergence radiale}

Il existe plusieurs types de résultats concernant ce qui se passe au bord du disque ouvert de convergence. Le programme ne cite que le théorème de convergence radiale d'Abel.

\begin{theo}
Soit \(\sum_{n\geq0}a_nz^n\) une série entière de rayon de converge \(R>0\). On note \(f\) sa somme, définie sur \(\intervie{0}{R}\).

Si la série \(\sum a_nR^n\) converge, alors \(f\) est définie sur \(\intervii{0}{R}\) et est continue à gauche en \(R\), donc sur \(\intervii{0}{R}\) tout entier : \[\lim_{x\to R^-}\sum_{n=0}^{\pinf}a_nx^n=\sum_{n=0}^{\pinf}a_nR^n.\]
\end{theo}

La démonstration de ce théorème est hors-programme. Il y a deux cas particuliers faciles à démontrer : si les coefficients \(a_n\) sont tous positifs ou quand la série \(\sum a_nR^n\) converge absolument. Dans le cas général, c'est plus difficile à justifier.

\section{Fonction développable en série entière}

\subsection{Généralités}

\begin{defi}
Soient \(f:\R\to\C\) et \(r>0\).

On dit que \(f\) est développable en série entière sur \(\intervee{-r}{r}\) quand il existe une suite complexe \(\paren{a_n}\) telle que la série entière \(\sum_{n\geq0}a_nz^n\) soit de rayon de convergence \(R\geq r\) et \(\quantifs{\tpt x\in\intervee{-r}{r}}f\paren{x}=\sum_{n=0}^{\pinf}a_nx^n\).

La somme \(\sum_{n=0}^{\pinf}a_nx^n\) est appelée le développement en série entière de \(f\).
\end{defi}

On dispose d'un théorème d'opérations sur les fonctions développables en série entière.

\begin{theo}
Soient \(f,g\) deux fonctions développables en série entière sur \(\intervee{-r}{r}\), où \(r>0\).

\begin{itemize}
    \item Les fonctions \(f+g\) et \(fg\) sont développables en série entière sur \(\intervee{-r}{r}\). \\
    \item Si \(f\paren{0}=0\), alors \(g\rond f\) est développable en série entière sur un intervalle \(\intervee{-r\prim}{r\prim}\), où \(r\prim>0\). \\
    \item Si \(f\paren{0}\not=0\), alors \(\dfrac{1}{f}\) est développable en série entière sur un intervalle \(\intervee{-r\prim}{r\prim}\) où \(r\prim>0\).
\end{itemize}
\end{theo}

\begin{ex}
Les fonctions rationnelles qui n'ont pas \(0\) comme pôle sont développables en série entière.
\end{ex}

\begin{exo}
Développez en série entière la fonction \(x\mapsto\dfrac{1}{x^2-3x+2}\) et précisez le rayon de convergence.
\end{exo}

\begin{exo}
Même exercice avec la fonction \(x\mapsto\dfrac{1}{1+x+x^2}\).
\end{exo}

\begin{exo}
Même exercice avec la fonction \(x\mapsto\ln\paren{1+x+x^2}\).
\end{exo}

Si une fonction \(f\) est développable en série entière sur un intervalle \(\intervee{-r}{r}\) (où \(r>0\)), alors, d'après ce qui précède, \(f\) est de classe \(\classe{\infty}\) sur \(\intervee{-r}{r}\) et \(\quantifs{\tpt x\in\intervee{-r}{r}}f\paren{x}=\sum_{n=0}^{\pinf}\dfrac{f\deriv{n}\paren{0}}{n!}x^n\).

La réciproque est fausse : une fonction de classe \(\classe{\infty}\) sur un intervalle \(\intervee{-r}{r}\) n'est pas forcément développable en série entière.

De même, si une fonction est développable en série entière sur un intervalle \(\intervee{-r}{r}\) (où \(r>0\)), alors elle possède un développement limité à tout ordre en \(0\), obtenu en tronquant le développement en série entière à l'ordre voulu.

Là encore, la réciproque est fausse.

\subsection{Unicité du développement en série entière}

\begin{prop}
Soient \(\paren{a_n}\) une suite de complexes et \(r>0\).

Si \(\quantifs{\tpt x\in\intervee{-r}{r}}\sum_{n=0}^{\pinf}a_nx^n=0\), alors \(\paren{a_n}=0\).
\end{prop}

Le développement en série entière d'une fonction, s'il existe, est unique. Autrement dit, on peut identifier les coefficients de deux développements en série entière égaux.

\begin{cor}
Soient \(\paren{a_n},\paren{b_n}\) deux suites de complexes.

Si les deux séries entières \(\sum_{n\geq0}a_nz^n\) et \(\sum_{n\geq0}b_nz^n\) ont un rayon de convergence au moins égal à \(r>0\) et si \(\quantifs{\tpt x\in\intervee{-r}{r}}\sum_{n=0}^{\pinf}a_nx^n=\sum_{n=0}^{\pinf}b_nx^n\), alors \(\paren{a_n}=\paren{b_n}\).
\end{cor}

\begin{cor}
Soit \(f\) une fonction développable en série entière sur \(\intervee{-r}{r}\), où \(r>0\).

Si \(f\) est paire, alors les coefficients d'indices impairs du développement en série entière de \(f\) sont nuls.

Si \(f\) est impaire, alors les coefficients d'indices pairs du développement en série entière de \(f\) sont nuls.
\end{cor}

\begin{exo}
On admet momentanément que \(f:x\mapsto\sqrt{1+x}\) est développable en série entière sur \(\intervee{-1}{1}\) et on note alors \(f\paren{x}=\sum_{n=0}^{\pinf}a_nx^n\) pour \(x\in\intervee{-1}{1}\).

Déterminez une équation différentielle linéaire homogène du premier ordre dont \(f\) est solution.

Déduisez-en que \(\quantifs{\tpt n\in\N}a_{n+1}=\dfrac{2n+1}{2n+2}a_n\), puis une expression de \(a_n\) en fonction de \(n\).
\end{exo}

\subsection{Série de Taylor d'une fonction}

\begin{defi}
Soient \(r>0\) et \(f\) une fonction de classe \(\classe{\infty}\) sur \(\intervee{-r}{r}\).

On appelle série de Taylor de \(f\) la série \(\sum_{n\geq0}\dfrac{f\deriv{n}\paren{0}}{n!}x^n\).
\end{defi}

Autrement dit, \(f\) est développable en série entière ssi \(f\) est égale à la somme de sa série de Taylor.

\begin{prop}
Soient \(r>0\) et \(f\) une fonction de classe \(\classe{\infty}\) au voisinage de \(0\).

\(f\) est égale à sa série de Taylor sur \(\intervee{-r}{r}\) ssi \(\quantifs{\forall x\in\intervee{-r}{r}}\lim_{n\to\pinf}\int_0^x\dfrac{\paren{x-t}^n}{n!}f\deriv{n+1}\paren{t}\odif{t}=0\).
\end{prop}

La démonstration est élémentaire et repose sur l'égalité de Taylor avec reste intégral.

Comme l'inégalité de Taylor-Lagrange découle de cette égalité, on a une condition suffisante pour qu'une fonction soit développable en série entière.

\begin{prop}
Soient \(r>0\) et \(f\) une fonction de classe \(\classe{\infty}\) sur \(\intervee{-r}{r}\).

Si \(\lim_{n\to\pinf}\dfrac{r^n}{n!}\sup_{\intervii{-r}{r}}\abs{f\deriv{n}}=0\), alors \(f\) est développable en série entière sur \(\intervee{-r}{r}\).
\end{prop}

\begin{exo}
Soit \(a\in\intervee{0}{1}\). On pose \(f:x\mapsto\paren{1+x}^a\).

Montrez que \(f\) est développable en série entière sur \(\intervee{-\dfrac{1}{2}}{\dfrac{1}{2}}\).
\end{exo}

\subsection{Développements en série entière usuels}

Soient \(a\) un réel et \(n\) un entier naturel.

On pose \(\binom{n}{a}=\dfrac{a\paren{a-1}\paren{a-2}\dots\paren{a-n+1}}{n!}\).

Les fonctions usuelles suivantes sont développables en série entière avec un rayon de convergence \(R>0\) :

\begin{itemize}
    \item \(\e{x}=\sum_{n=0}^{\pinf}\dfrac{1}{n!}x^n\) avec \(R=\pinf\) \\
    \item \(\cos x=\sum_{n=0}^{\pinf}\dfrac{\paren{-1}^n}{\paren{2n}!}x^{2n}\) avec \(R=\pinf\) \\
    \item \(\sin x=\sum_{n=0}^{\pinf}\dfrac{\paren{-1}^n}{\paren{2n+1}!}x^{2n+1}\) avec \(R=\pinf\) \\
    \item \(\ch x=\sum_{n=0}^{\pinf}\dfrac{1}{\paren{2n}!}x^{2n}\) avec \(R=\pinf\) \\
    \item \(\sh x=\sum_{n=0}^{\pinf}\dfrac{1}{\paren{2n+1}!}x^{2n+1}\) avec \(R=\pinf\) \\
    \item \(\Arctan x=\sum_{n=0}^{\pinf}\dfrac{\paren{-1}^n}{2n+1}x^{2n+1}\) avec \(R=1\) \\
    \item \(\ln\paren{1+x}=\sum_{n=0}^{\pinf}\dfrac{\paren{-1}^{n-1}}{n}x^n\) avec \(R=1\) \\
    \item \(\paren{1+x}^a=\sum_{n=0}^{\pinf}\binom{n}{a}x^n\) avec \(R=1\).
\end{itemize}

\begin{exo}
Développez en série entière la fonction \(x\mapsto\ln\dfrac{1-x}{1+x}\).
\end{exo}

\begin{exo}
Même exercice avec \(x\mapsto\paren{x+2}\ch x\).
\end{exo}

\begin{exo}
Même exercice avec \(x\mapsto\sin^2x\).
\end{exo}

\begin{exo}
Même exercice avec \(x\mapsto\e{x}\ln\paren{1+x}\).
\end{exo}
