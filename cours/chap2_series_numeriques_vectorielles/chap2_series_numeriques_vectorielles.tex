\chapter{Séries numériques et vectorielles : révisions et compléments}

\minitoc

Dans ce chapitre, \(E\) désigne un espace vectoriel normé (qui peut être \(\R\) ou \(\C\)) et \(\norme{}\) la norme associée (qui est dans ces cas la valeur absolue ou le module).

\section{Rappels}

\subsection{Définitions et notations}

\begin{defi}[Série vectorielle]
Soit \(u\) une suite de \(E\).

On associe à cette suite la suite \(s\) définie de la façon suivante : \(\quantifs{\tpt n\in\N}s_n=\sum_{k=0}^nu_k\).

La suite \(s\) est appelée série de terme général \(u_n\) et notée \(\sum_{n\geq0}u_n\) ou \(\sum u\).

Chaque nombre \(s_n\) est appelé somme partielle d'indice \(n\) de la série.
\end{defi}

L'adjectif \guillemets{numérique} associé au mot \guillemets{série} signifie que les termes généraux de la série sont en fait des nombres réels ou complexes.

\subsection{Convergence d'une série}

\begin{defi}
Soit \(u\) une suite de \(E\).

On dit que la série \(\sum u\) converge ssi la suite des sommes partielles \(\paren{s_n}=\paren{\sum_{k=0}^nu_k}\) converge.

Dans ce cas, si \(l=\lim_{n\to\pinf}s_n\), alors \(l\) est appelée somme de la série \(\sum u\) et on note \(l=\sum_{n=0}^{\pinf}u_n\).

On appelle aussi reste partiel d'indice \(n\) de la série le nombre \(r_n=\sum_{k=n+1}^{\pinf}u_k\), de sorte que \(r_n+s_n=l\).

La suite des restes partiels converge donc vers \(0\).

Dans le cas contraire, on dit que la série \(\sum u\) diverge.
\end{defi}

\begin{ex}
\begin{itemize}
    \item Soit \(x\in\C\). La série \(\sum_{n\geq0}x^n\) converge ssi \(\abs{x}<1\) et, dans ce cas, \(\sum_{n=0}^{\pinf}x^n=\dfrac{1}{1-x}\). Cette série est appelée série géométrique de raison \(x\). \\
    \item Les séries de Riemann : \(\sum_{n\geq1}\dfrac{1}{n^\alpha}\) converge ssi \(\alpha>1\). \\
    \item Pour tout \(z\in\C\), la série \(\sum\dfrac{z^n}{n!}\) converge et \(\sum_{n=0}^{\pinf}\dfrac{z^n}{n!}=\e{z}\).
\end{itemize}
\end{ex}

On peut bien sûr généraliser aux séries quelques théorèmes d'opérations.

\begin{prop}
Soient \(u,v\) deux suites de \(E\) et \(\lambda\) un scalaire.

Si les séries \(\sum u\) et \(\sum v\) convergent, alors la série \(\sum\paren{u+\lambda v}\) converge et \(\sum_{n=0}^{\pinf}\paren{u_n+\lambda v_n}=\sum_{n=0}^{\pinf}u_n+\lambda\sum_{n=0}^{\pinf}v_n\).

Ceci prouve aussi que l'ensemble des séries convergentes est un espace vectoriel.
\end{prop}

\begin{rem}
La somme d'une série divergente et d'une série convergente est une série divergente.

En revanche, il n'y a rien à dire a priori à propos de la somme de deux séries divergentes.
\end{rem}

\subsection{Lien entre convergence de suites et convergence de séries}

\begin{prop}
Soit \(u\) une suite de \(E\).

Si la série \(\sum u\) converge, alors la suite \(u\) converge vers \(0\).
\end{prop}

\begin{rem}
\begin{itemize}
    \item La réciproque est fausse. \\
    \item Par contraposition, si une suite \(u\) ne tend pas vers \(0\), alors la série associée diverge : on dit que la série \(\sum u\) diverge grossièrement.
\end{itemize}
\end{rem}

\begin{ex}
On appelle série harmonique la série \(\sum_{n\geq1}\dfrac{1}{n}\).

Cette série diverge, pourtant son terme général tend vers \(0\).
\end{ex}

\begin{defi}
Soit \(u\) une suite de \(E\). On pose \(v_n=u_{n+1}-u_n\).

La série \(\sum v\) est appelée la série télescopique (ou série domino, ou série différence) associée à \(u\).
\end{defi}

\begin{prop}
Une suite converge ssi sa série télescopique associée converge.
\end{prop}

\begin{exo}~\\
On pose \(u_n=\sum_{k=1}^n\dfrac{1}{k}-\ln n\). Montrez que la suite \(u\) converge.
\end{exo}

\section{Séries réelles à termes positifs}

Dans cette section, on s'intéresse uniquement aux séries dont le terme général est un réel positif.

On appelle un premier théorème issu du cours de première année.

\begin{theo}
Soient \(u\) et \(v\) deux suites réelles positives.

\begin{itemize}
    \item Si \(0\leq u\leq v\) et si la série \(\sum v\) converge, alors la série \(\sum u\) converge. \\
    \item Si \(0\leq u\leq v\) et si la série \(\sum u\) diverge, alors la série \(\sum v\) diverge. \\
    \item Si \(u\sim v\), alors les séries \(\sum u\) et \(\sum v\) sont de même nature.
\end{itemize}
\end{theo}

Une application classique : la règle de d'Alembert.

\begin{prop}
Soit \(u\) une suite réelle strictement positive telle que \(\dfrac{u_{n+1}}{u_n}\tendqd{n\to\pinf}l\). Alors

\begin{itemize}
    \item si \(l<1\), la série \(\sum u\) converge ; \\
    \item si \(l>1\), la série \(\sum u\) diverge ; \\
    \item si \(l=1\), on ne peut rien conclure.
\end{itemize}
\end{prop}

\begin{exo}
Soient \(x,y>0\). Représentez graphiquement l'ensemble des couples \(\paren{x,y}\) tels que la série \(\sum\dfrac{x^n}{y^n+n^x}\) converge.
\end{exo}

\begin{exo}
Montrez que la suite \(\paren{u_n}\) définie par \(u_0\in\intervii{0}{1}\) et \(u_{n+1}=\dfrac{1}{2}\paren{u_n+u_n^2}\) converge vers \(0\) et donnez la nature de la série \(\sum u_n\).
\end{exo}

On donne quelques versions plus élaborées du théorème de comparaison.

\subsection{Théorème de Cesàro}

\begin{theo}
Soit \(u\) une suite numérique qui converge vers \(l\). Alors \(\dfrac{u_0+\dots+u_n}{n}\tendqd{n\to\pinf}l\).
\end{theo}

Dans le cas où \(l\not=0\), la série \(\sum u\) diverge grossièrement et \(\sum_{k=0}^nu_k\simqd{n\to\pinf}nl\).

Dans le cas où \(l=0\), on peut juste dire \(\sum_{k=0}^nu_k=\o{n}\).

\begin{exo}
Soit \(u\) la suite définie par récurrence par \(u_0>0\) et \(\quantifs{\tpt n\in\N}u_{n+1}=u_n+\dfrac{u_n+1}{u_n+2}\).

Étudiez la convergence ou divergence de la suite \(u\), puis donnez un équivalent simple de \(u_n\) quand \(n\) tend vers \(\pinf\).
\end{exo}

\subsection{Théorème de comparaison par domination de séries à termes positifs}

Dans le cas convergent d'abord, les restes partiels suivent la même relation de comparaison.

\begin{theo}
Soient \(u,v\) deux suites réelles positives.

Si \(u=\O{v}\) et la série \(\sum v\) converge, alors la série \(\sum u\) converge. De plus, \(\sum_{k=n+1}^{\pinf}u_k=\O{\sum_{k=n+1}^{\pinf}v_k}\).

Si \(u=\o{v}\) et la série \(\sum v\) converge, alors la série \(\sum u\) converge. De plus, \(\sum_{k=n+1}^{\pinf}u_k=\o{\sum_{k=n+1}^{\pinf}v_k}\).
\end{theo}

Dans le cas divergent ensuite, les sommes partielles suivent aussi la même relation de comparaison.

\begin{theo}
Soient \(u,v\) deux suites réelles positives.

Si \(u=\O{v}\) et la série \(\sum u\) diverge, alors la série \(\sum v\) diverge. De plus, \(\sum_{k=0}^nu_k=\O{\sum_{k=0}^nv_k}\).

Si \(u=\o{v}\) et la série \(\sum u\) diverge, alors la série \(\sum v\) diverge. De plus, \(\sum_{k=0}^nu_k=\o{\sum_{k=0}^nv_k}\).
\end{theo}

\subsection{Théorème de comparaison par équivalence de séries à termes positifs}

\begin{theo}
Soient \(u,v\) deux suites réelles positives.

Si \(u\sim v\), alors les séries \(\sum u\) et \(\sum v\) sont de même nature ; l'une converge ssi l'autre converge.

De plus,

\begin{itemize}
    \item si les séries convergent, alors les restes partiels sont équivalents : \(\sum_{k=n+1}^{\pinf}u_k\simqd{n\to\pinf}\sum_{k=n+1}^{\pinf}v_k\) ; \\
    \item si les séries divergent, alors les sommes partielles divergent vers \(\pinf\) et sont équivalentes : \(\sum_{k=0}^nu_k\simqd{n\to\pinf}\sum_{k=0}^nv_k\).
\end{itemize}
\end{theo}

\begin{exo}
Soit \(a>0\). On pose \(u_n=\sin\dfrac{a^n}{n}\) pour \(n\in\Ns\). Selon la valeur de \(a\), déterminez la nature de la série \(\sum_{n\geq1}u_n\).

Montrez que si \(a=1\), alors \(\sum_{k=1}^nu_k\sim\ln n\) et si \(a<1\), \(\sum_{k=n}^{\pinf}u_k=\o{a^n}\).
\end{exo}

\subsection{Théorème de comparaison série - intégrale}

\begin{prop}
Soit \(f\) une fonction continue, positive et décroissante sur \(\Rp\).

Alors la série de terme général \(f\paren{n}\) et la suite de terme général \(\int_0^nf\) sont de même nature.
\end{prop}

\begin{meth}[À retenir]
La technique d'encadrement des sommes partielles d'une série \(\sum f\paren{n}\) (ou des restes partiels) par des intégrales quand \(f\) est continue, positive et monotone.
\end{meth}

\begin{ex}~\\
\begin{itemize}
    \item \(\sum_{k=1}^n\dfrac{1}{k}\simqd{n\to\pinf}\ln n\) (à connaître). \\
    \item Si \(\alpha>1\), un équivalent simple de \(\sum_{k=n+1}^{\pinf}\dfrac{1}{k^\alpha}\) quand \(n\to\pinf\) est \(\dfrac{1}{\paren{\alpha-1}n^{\alpha-1}}\).
\end{itemize}
\end{ex}

\begin{exo}
Pour \(n\in\Ns\), on pose \(u_n=\sum_{k=n}^{\pinf}\dfrac{\ln k}{k^2}\).

Justifiez l'existence de \(u_n\), puis montrez la divergence de la série \(\sum u_n\).

Montrez que \(\sum_{k=1}^nu_k\simqd{n\to\pinf}\dfrac{\ln^2n}{2}\).
\end{exo}

\section{Séries absolument convergentes}

\begin{defi}
Soit \(u\) une suite de \(E\).

On dit que la série \(\sum u\) est absolument convergente ssi la série à termes positifs \(\sum\norme{u}\) est convergente.
\end{defi}

\subsection{Lien entre absolue convergence et convergence}

\begin{theo}
Si \(E\) est de dimension finie, alors toute série absolument convergente est convergente.
\end{theo}

\begin{rem}
\begin{itemize}
    \item La réciproque est fausse : la série \(\sum_{n\geq1}\dfrac{\paren{-1}^n}{n}\) converge (on l'appelle la série harmonique alternée) mais ne converge pas absolument. \\
    \item L'hypothèse de la dimension finie est indispensable. En dimension infinie, ce résultat est faux en général.
\end{itemize}
\end{rem}

\begin{exo}
Soit \(x>0\). Montrez que les séries suivantes convergent : \[\sum_{n\geq2}\dfrac{\ln\paren{n^2+\paren{-1}^nn}}{n^2+\paren{-1}^nx^n}\qquad\sum_{n\geq0}\sqrt{n}\cos\paren{x}\sin^n\paren{x}\qquad\sum_{n\geq0}\dfrac{\paren{-1}^n\sqrt{n+x}}{x^n+n^{\nicefrac{2}{x}}}.\]
\end{exo}

\subsection{Un exemple fondamental : l'exponentielle de matrice}

Soit \(p\in\Ns\).

On choisit comme norme sur \(E=\M{p}[\C]\) une norme sous-multiplicative.

Alors pour tout \(n\in\Ns\), \(\norme{A^n}\leq\norme{A}^n\), donc \(\norme{\dfrac{A^n}{n!}}\leq\dfrac{\norme{A}^n}{n!}\).

Or la série \(\sum\dfrac{\norme{A}^n}{n!}\) converge (et sa somme vaut \(\exp\norme{A}\)), donc par comparaison de séries à termes positifs, la série \(\sum\dfrac{A^n}{n!}\) est absolument convergente.

On pose alors \(\exp A=\sum_{n=0}^{\pinf}\dfrac{A^n}{n!}\).

\subsection{Extension des résultats par comparaison}

\begin{defi}
Soit \(u\) une suite de \(E\) et \(v\) une suite réelle positive.

On dit que \(u=\O{v}\) quand \(\quantifs{\exists M>0;\exists n_0\in\N;\forall n\geq n_0}\norme{u_n}\leq Mv_n\).

On dit que \(u=\o{v}\) quand \(\quantifs{\forall\epsilon>0;\exists n_0\in\N;\forall n\geq n_0}\norme{u_n}\leq\epsilon v_n\).
\end{defi}

\begin{prop}
Soient \(u\) une suite de \(E\) et \(v\) une suite réelle positive.

Si \(E\) est de dimension finie, \(u_n=\O{v_n}\) quand \(n\) tend vers \(\pinf\) et la série \(\sum v\) converge, alors la série \(\sum u\) est absolument convergente.

De plus, \(\sum_{k=n+1}^{\pinf}u_k=\O{\sum_{k=n+1}^{\pinf}v_k}\).
\end{prop}

Ceci est encore valable si \(u_n=\o{v_n}\).

\begin{prop}
Soient \(u\) une suite de \(E\) et \(v\) une suite réelle positive.

Si \(E\) est de dimension finie, \(u_n=\o{v_n}\) quand \(n\) tend vers \(\pinf\) et la série \(\sum v\) converge, alors la série \(\sum u\) est absolument convergente.

De plus, \(\sum_{k=n+1}^{\pinf}u_k=\o{\sum_{k=n+1}^{\pinf}v_k}\).
\end{prop}

\subsection{Produit de Cauchy de deux séries absolument convergentes}

\begin{defi}
Soient \(E\) une algèbre normée de dimension finie, \(\sum_{n\geq0}a_n\) et \(\sum_{n\geq0}b_n\) deux séries à termes dans \(E\).

On appelle produit de Cauchy des deux séries la série \(\sum_{n\geq0}c_n\) où \(\quantifs{\tpt n\in\N}c_n=\sum_{k=0}^na_kb_{n-k}\).
\end{defi}

\begin{rem}
Quand les séries ne commencent pas à partir du rang \(0\), il faut se méfier ! Une idée simple est de se ramener au cas précédent en décalant les indices.

Exemple très courant : les séries commencent au rang \(1\). Dans ce cas, le produit de Cauchy des séries \(\sum_{n\geq1}a_n\) et \(\sum_{n\geq1}b_n\) est la série \(\sum_{n\geq1}c_n\) où \(\quantifs{\tpt n\in\Ns}c_n=\sum_{k=1}^{n}a_kb_{n+1-k}\).
\end{rem}

\begin{theo}
Avec les mêmes hypothèses sur \(E\).

Si les séries \(\sum_{n\geq0}a_n\) et \(\sum_{n\geq0}b_n\) convergent absolument, alors leur produit de Cauchy est aussi absolument convergent et \[\sum_{n=0}^{\pinf}c_n=\sum_{n=0}^{\pinf}a_n\times\sum_{k=0}^{\pinf}b_n.\]
\end{theo}

\section{Séries alternées}

\begin{defi}
Une série alternée est une série réelle \(\sum u_n\) telle que \(\quantifs{\tpt n\in\N}u_{n+1}\text{ est de signe opposé à }u_n\).
\end{defi}

En général, les séries alternées sont reconnaissables à la présence d'un facteur \(\paren{-1}^n\) dans l'expression du terme général.

On dispose d'une condition suffisante de convergence d'une série alternée qu'on appelle le critère spécial des séries alternées.

\begin{theo}
Soit \(\sum\paren{-1}^nu_n\) une série alternée.

Si la suite \(u\)

\begin{itemize}
    \item est positive, \\
    \item est décroissante, \\
    \item et converge vers \(0\),
\end{itemize}

alors la série \(\sum\paren{-1}^nu_n\) converge.

Dans ce cas, la somme de la série est positive, et si on note \(R_n=\sum_{k=n+1}^{\pinf}\paren{-1}^ku_k\) le reste partiel d'indice \(n\), alors \(\quantifs{\tpt n\in\N}R_n\text{ est du signe de son premier terme (\ie du signe de }\paren{-1}^{n+1}\text{) et }\abs{R_n}\leq u_{n+1}\leq u_n\).
\end{theo}

\begin{ex}
\begin{itemize}
    \item La série harmonique alternée \(\sum_{n\geq1}\dfrac{\paren{-1}^n}{n}\) converge. \\
    \item La série \(\sum_{n\geq2}\dfrac{\paren{-1}^n}{\ln n}\) converge.
\end{itemize}
\end{ex}

\begin{rem}
\begin{itemize}
    \item Si \(\sum_{n\geq n_0}\paren{-1}^nu_n\) est une série alternée convergente, sa somme a le signe du premier terme de la série (ici le signe de \(\paren{-1}^{n_0}u_{n_0}\)). \\
    \item La condition de décroissance de la suite \(u\) est essentielle ! Contre-exemple : la série \(\sum_{n\geq2}\dfrac{\paren{-1}^n}{\paren{-1}^n+\sqrt{n}}\) est une série alternée divergente. \\ De plus, cela fournit un contre-exemple au théorème de comparaison par équivalents si on ne tient pas compte de la condition sur le signe, qui doit être constant.
\end{itemize}
\end{rem}

\begin{exo}
Soit \(\alpha>1\). Pour \(n\in\Ns\), on pose \(u_n=\sum_{k=n}^{\pinf}\dfrac{\paren{-1}^k}{k^\alpha+k}\).

Justifiez l'existence de \(u_n\). Montrez que la série \(\sum_{n\geq1}u_n\) converge.
\end{exo}
