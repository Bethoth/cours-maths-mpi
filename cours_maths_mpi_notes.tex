% Set up the document's format to A4 and the font's size to 12pt.
\documentclass[a4paper,12pt]{report}

% Set up the document's title, author and date.
\title{Maths -- MPI}
\author{Romain Bricout}
\date{\today}

% Set up the input's encoding to UTF-8, the document's font and language to T1 (adapted to french) and french (the grammar linter uses this parameter).
\usepackage[utf8]{inputenc}
\usepackage[T1]{fontenc}
\usepackage[frenchb]{babel}

\usepackage[dvipsnames]{xcolor}

% Set up the document's margins.
\usepackage{geometry}
\geometry{hmargin=1.5cm,vmargin=1.5cm}

% The three main maths packages. They are used for a lot of things.
\usepackage{amssymb,amsmath}
\usepackage{mathtools}

% Useful to create nice and easy signs or variations tables.
\usepackage{tkz-tab}

% Useful to create any kind of visual representation (graph functions, illustrate geometry problems, etc)
\usepackage{tikz}
\usetikzlibrary{patterns,angles,quotes,arrows,arrows.meta,bending,matrix,calc}

% Allows to edit the itemize environment's default item document-wide.
\usepackage{enumitem}

% Allows to define \notfoo or \nfoo (not recommended) in order for \not\foo to work as wished.
\usepackage{newtxmath}

\DeclareSymbolFont{CMletters}{OML}{cmm}{m}{it}
\DeclareMathSymbol{\nu}{\mathord}{CMletters}{23}
\DeclareMathSymbol{\delta}{\mathord}{CMletters}{14}
\DeclareMathSymbol{\zeta}{\mathord}{CMletters}{16}
\DeclareMathSymbol{\kappa}{\mathord}{CMletters}{20}
\DeclareMathSymbol{\xi}{\mathord}{CMletters}{24}
\DeclareMathSymbol{\pi}{\mathord}{CMletters}{25}
\DeclareMathSymbol{\upsilon}{\mathord}{CMletters}{29}
\DeclareMathSymbol{\chi}{\mathord}{CMletters}{31}
\DeclareMathSymbol{\omega}{\mathord}{CMletters}{33}
\DeclareMathSymbol{\Phi}{\mathord}{CMletters}{8}
\DeclareMathSymbol{\Omega}{\mathord}{CMletters}{10}
\DeclareMathSymbol{\lambda}{\mathord}{CMletters}{21}

% Makes the table of contents clickable and gives useful commands for links in general.
\usepackage[hypertexnames=false]{hyperref}
\hypersetup{colorlinks=false,linktoc=all}

% Gives the llbracket and rrbracket commands for integer intervals.
\usepackage{stmaryrd}

% Useful to insert nice-looking quotes.
\usepackage{epigraph}

% Allows to insert chapter-specific table of contents.
\usepackage{minitoc}
\mtcselectlanguage{french}
\setcounter{minitocdepth}{6}

% Useful when units are needed.
\usepackage{siunitx}
\sisetup{
locale=FR,
detect-all,
inter-unit-product=\ensuremath{\cdot},
list-final-separator={et},
list-pair-separator={et},
range-phrase={\ensuremath{\xleftrightarrow{}}},
exponent-product=\ensuremath{\cdot},
per-mode=power-positive-first
}

\usepackage[thmmarks,hyperref]{ntheorem}
\makeatletter
\let\old@thm\@thm
\usepackage[lowercase]{theoremref}
\def\@thm#1#2#3{\def\thmref@currname{#3}\old@thm{#1}{#2}{#3}}
\makeatother

% Allows whiteboard digits with \mathds
\usepackage{dsfont}

\usepackage{needspace}

% Useful for better-looking oneline fractions
\usepackage{nicefrac}

% Set up the horizontal space before the first line of a new paragraph to 2em and the vertical space between two paragraphs to 1em.
\setlength{\parindent}{0pt}
\setlength{\parskip}{1em}

% Adds 0.5em to the vertical space between two lines in an align environment. It looks better.
\addtolength{\jot}{0.5em}

% Allows align environment to break if it's too long to fit in the page where it began.
\allowdisplaybreaks[1]

% Trick to make semicolons considered like relation operators (such as =) and therefore being equidistantly spaced from the two numbers around it.
\mathcode`;=\numexpr\mathcode`;-"3000

% Commands for size-adaptative parentheses, brackets, curly brackets, absolute value and magnitude.
\newcommand{\paren}[1]{\left(#1\right)} % (x)
\newcommand{\croch}[1]{\left[#1\right]} % [x]
\newcommand{\accol}[1]{\left\lbrace#1\right\rbrace} % {x}
\newcommand{\abs}[1]{\left\lvert#1\right\rvert} % |x|
\newcommand{\floor}[1]{\left\lfloor#1\right\rfloor} % ⌊x⌋
\newcommand{\ceil}[1]{\left\lceil#1\right\rceil} % ⌈x⌉

% Commands for size-adaptative intervals and integer intervals. The commands' roots are "interv" and "interventier" and the added e or i at the end mean "excluded" and "included" respectively.
\newcommand{\intervii}[2]{\left[#1;#2\right]} % [a;b]
\newcommand{\intervee}[2]{\left]#1;#2\right[} % ]a;b[
\newcommand{\intervie}[2]{\left[#1;#2\right[} % [a;b[
\newcommand{\intervei}[2]{\left]#1;#2\right]} % ]a;b]
\newcommand{\interventierii}[2]{\left\llbracket#1;#2\right\rrbracket} % non-ASCII characters needed
\newcommand{\interventieree}[2]{\left\rrbracket#1;#2\right\llbracket} % non-ASCII characters needed
\newcommand{\interventierie}[2]{\left\llbracket#1;#2\right\llbracket} % non-ASCII characters needed
\newcommand{\interventierei}[2]{\left\rrbracket#1;#2\right\rrbracket} % non-ASCII characters needed

% Commands for usually used sets.
\newcommand{\N}{\mathbb{N}} % natural integers
\newcommand{\Ns}{\mathbb{N}^*}

\newcommand{\Z}{\mathbb{Z}} % relative integers
\newcommand{\Zp}{\mathbb{Z}_+}
\newcommand{\Zs}{\mathbb{Z}^*}
\newcommand{\Zps}{\mathbb{Z}_+^*}

\newcommand{\D}{\mathbb{D}} % decimal numbers
\newcommand{\Dp}{\mathbb{D}_+}
\newcommand{\Dm}{\mathbb{D}_-}
\newcommand{\Ds}{\mathbb{D}^*}
\newcommand{\Dps}{\mathbb{D}_+^*}
\newcommand{\Dms}{\mathbb{D}_-^*}

\newcommand{\Q}{\mathbb{Q}} % rational numbers
\newcommand{\Qp}{\mathbb{Q}_+}
\newcommand{\Qm}{\mathbb{Q}_-}
\newcommand{\Qs}{\mathbb{Q}^*}
\newcommand{\Qps}{\mathbb{Q}_+^*}
\newcommand{\Qms}{\mathbb{Q}_-^*}

\newcommand{\R}{\mathbb{R}} % real numbers
\newcommand{\Rp}{\mathbb{R}_+}
\newcommand{\Rm}{\mathbb{R}_-}
\newcommand{\Rs}{\mathbb{R}^*}
\newcommand{\Rps}{\mathbb{R}_+^*}
\newcommand{\Rms}{\mathbb{R}_-^*}
\newcommand{\Rb}{\overline{\mathbb{R}}}

\newcommand{\C}{\mathbb{C}} % complex numbers
\newcommand{\Cs}{\mathbb{C}^*}

\newcommand{\K}{\mathbb{K}}
\newcommand{\Ks}{\mathbb{K}^*}

\newcommand{\A}{\mathbb{A}}
\renewcommand{\L}[2]{\mathscr{L}\paren{#1,#2}}
\newcommand{\Lc}[2]{\mathscr{L}_c\paren{#1,#2}}
\newcommand{\Lendo}[1]{\mathscr{L}\paren{#1}}

\newcommand{\prem}{\mathbb{P}}

\newcommand{\U}{\mathbb{U}} % complex numbers whose modulus is 1

\renewcommand{\P}[1]{\mathscr{P}\paren{#1}} % subsets of a set
\newcommand{\Pf}[1]{\mathscr{P}_f\paren{#1}} % finite subsets of a set
\newcommand{\F}[2]{\mathscr{F}\paren{#1,#2}} % functions from 1 to 2
\newcommand{\V}[1]{\mathscr{V}\paren{#1}} % neighborhood of a number

% Redefines \Re and \Im to print Re and Im (the same way as ln or lim) instead of fraktur R and I which don't look nice and are less readable.
\renewcommand{\Re}{\operatorname{Re}}
\renewcommand{\Im}{\operatorname{Im}}
\newcommand{\Card}{\operatorname{Card}}

% Command to print an upright e for the exponential instead of a slanted e and put the exponent.
\newcommand{\e}[1]{\mathrm{e}^{#1}}

% Command to print the imaginary i with a little space on the right. This way, the exponents don't look confusing. \i normally prints a dotless i.
\renewcommand{\i}{i\mkern1mu}

% Redefines \vec such that the arrow covers the whole name of the vector.
%\renewcommand{\vec}[1]{\overrightarrow{#1}}

% Commands for 2D and 3D vectors' coordinates
\newcommand{\dcoords}[2]{\begin{pmatrix}#1\\#2\end{pmatrix}}
\newcommand{\tcoords}[3]{\begin{pmatrix}#1\\#2\\#3\end{pmatrix}}

% Redefines binom to print nicer parentheses around the numbers.
\renewcommand{\binom}[2]{\begin{pmatrix}#2\\#1\end{pmatrix}}

% Command for a QED black square. It automatically prints a whitespace before the square such that it looks nice.
\newcommand{\cqfd}{\text{ }\blacksquare}

% Commands with more explicit names for the best way to express divisibility (mid and nmid).
\newcommand{\divise}{\mid}
\newcommand{\notdivise}{\nmid}

% Commands that do the exact same thing but with explicit names for a complex number's conjugate and an event's negation in probability.
\newcommand{\conj}[1]{\overline{#1}}

% Command for a size-adaptative middle bar meaning "such that" (with spacing around it in order to look nice).
\newcommand{\tq}{\;\middle|\;}

% Command with an explicit name for the scalar product.
\newcommand{\scal}{\cdot}
\newcommand{\vecto}{\operatorname{_\wedge}}

% Shortcut for forcing displaystyle in inline mode.
\newcommand{\ds}{\displaystyle}

% Make the not version of implies, impliedby and iff look nice.
\newcommand{\notimp}{\centernot{\imp}}
\newcommand{\notimpr}{\centernot{\impr}}
\newcommand{\notssi}{\centernot{\ssi}}

\renewcommand{\subset}{\subseteq}
\renewcommand{\supset}{\supseteq}
\newcommand{\notsubset}{\centernot{\subset}}
\newcommand{\notsupset}{\centernot{\supset}}

% Shortcut for P(event).
\newcommand{\proba}[1]{\mathbb{P}\paren{#1}}
\newcommand{\probacond}[2]{\mathbb{P}_{#2}\paren{#1}}

% More explicit names for land (logical and) and lor (logical or).
\newcommand{\et}{\land}
\newcommand{\ou}{\lor}
\newcommand{\non}{\lnot}

% Explicitly named environment for tkz-tab tables. Automatically centers the table and handles the tikzpicture environment.
\newenvironment{tkz}
{
\begin{tikzpicture}
}
{
\end{tikzpicture}
}

% More explicitly named commands for the creation of tkz-tab tables.
\newcommand{\tableauinit}[2]{\tkzTabInit{#1}{#2}}
\newcommand{\tableausignes}[1]{\tkzTabLine{#1}}
\newcommand{\tableauvariations}[1]{\tkzTabVar{#1}}

% Shortcut for the curve and the domain of the given function.
\newcommand{\graphe}[1]{\Gamma_{#1}}
\newcommand{\ensembledef}[1]{\mathcal{D}_{#1}}

\renewcommand{\S}[1]{\mathfrak{S}_{#1}}
\newcommand{\frakA}[1]{\mathfrak{A}_{#1}}

\newcommand{\semihrule}{\rule{256.074815pt}{0.4pt}}

% Various environments that create boxes. Each one is one type of thing (example, proof, etc). Each type has its own automatic counter.
\theoremstyle{break}
\theorembodyfont{\upshape}
\theoremheaderfont{\itshape}
\theoremprework{\bigskip\needspace{\baselineskip}\color{green}\hrule\color{black}}
\theorempostwork{\bigskip}
\newtheorem{rem}{Remarque}[chapter]

\theoremstyle{break}
\theorembodyfont{\upshape}
\theoremheaderfont{\itshape}
\theoremprework{\bigskip\needspace{\baselineskip}\color{green}\hrule\color{black}}
\theorempostwork{\bigskip}
\newtheorem{ex}[rem]{Exemple}

\theoremstyle{break}
\theorembodyfont{\upshape}
\theoremheaderfont{\itshape}
\theoremprework{\bigskip\needspace{\baselineskip}\color{green}\hrule\color{black}}
\theorempostwork{\bigskip}
\newtheorem{rappel}[rem]{Rappel}

\theoremstyle{break}
\theorembodyfont{\upshape}
\theoremheaderfont{\itshape}
\theoremprework{\bigskip\needspace{\baselineskip}\color{brown}\hrule\color{black}}
\theorempostwork{\bigskip}
\newtheorem{oubli}[rem]{Oubli}

\theoremstyle{break}
\theorembodyfont{\upshape}
\theoremheaderfont{\normalfont\bfseries}
\theoremprework{\bigskip\needspace{\baselineskip}\color{blue}\hrule\color{black}}
\theorempostwork{\bigskip}
\newtheorem{defi}[rem]{Définition}

\theoremstyle{break}
\theorembodyfont{\upshape}
\theoremheaderfont{\normalfont\bfseries}
\theoremprework{\bigskip\needspace{\baselineskip}\color{blue}\hrule\color{black}}
\theorempostwork{\bigskip}
\newtheorem{reform}[rem]{Reformulation}

\theoremstyle{break}
\theorembodyfont{\upshape}
\theoremheaderfont{\normalfont\bfseries}
\theoremprework{\bigskip\needspace{\baselineskip}\color{magenta}\hrule\color{black}}
\theorempostwork{\bigskip}
\newtheorem{exo}[rem]{Exercice}

\theoremstyle{break}
\theorembodyfont{\upshape}
\theoremheaderfont{\normalfont\bfseries}
\theoremprework{\bigskip\needspace{\baselineskip}\color{magenta}\hrule\color{black}}
\theorempostwork{\bigskip}
\newtheorem{exos}[rem]{\(\star\) Exercice}

\theoremstyle{break}
\theorembodyfont{\upshape}
\theoremheaderfont{\normalfont\bfseries}
\theoremprework{\bigskip\needspace{\baselineskip}\color{magenta}\hrule\color{black}}
\theorempostwork{\bigskip}
\newtheorem{exoss}[rem]{\(\star\star\) Exercice}

\theoremstyle{break}
\theorembodyfont{\upshape}
\theoremheaderfont{\normalfont\bfseries}
\theoremprework{\bigskip\needspace{\baselineskip}\color{magenta}\hrule\color{black}}
\theorempostwork{\bigskip}
\newtheorem{exosss}[rem]{\(\star\star\star\) Exercice}

\theoremstyle{break}
\theorembodyfont{\upshape}
\theoremheaderfont{\normalfont\bfseries}
\theoremprework{\bigskip\needspace{\baselineskip}\color{magenta}\hrule\color{black}}
\theorempostwork{\bigskip}
\newtheorem{exops}[rem]{\(+\star\) Exercice}

\theoremstyle{break}
\theorembodyfont{\upshape}
\theoremheaderfont{\normalfont\bfseries}
\theoremprework{\bigskip\needspace{\baselineskip}\color{magenta}\hrule\color{black}}
\theorempostwork{\bigskip}
\newtheorem{exopss}[rem]{\(+\star\star\) Exercice}

\theoremstyle{break}
\theorembodyfont{\upshape}
\theoremheaderfont{\normalfont\bfseries}
\theoremprework{\bigskip\needspace{\baselineskip}\color{magenta}\hrule\color{black}}
\theorempostwork{\bigskip}
\newtheorem{exopsss}[rem]{\(+\star\star\star\) Exercice}

\theoremstyle{break}
\theorembodyfont{\upshape}
\theoremheaderfont{\normalfont\bfseries}
\theoremprework{\bigskip\needspace{\baselineskip}\color{magenta}\semihrule\color{green}\semihrule\color{black}}
\theorempostwork{\bigskip}
\newtheorem{exoex}[rem]{Exercice/Exemple}

\theoremstyle{break}
\theorembodyfont{\upshape}
\theoremheaderfont{\normalfont\bfseries}
\theoremprework{\bigskip\needspace{\baselineskip}\color{blue}\semihrule\color{red}\semihrule\color{black}}
\theorempostwork{\bigskip}
\newtheorem{defprop}[rem]{Définition/Proposition}

\theoremstyle{break}
\theorembodyfont{\upshape}
\theoremheaderfont{\normalfont\bfseries}
\theoremprework{\bigskip\needspace{\baselineskip}\color{blue}\semihrule\color{red}\semihrule\color{black}}
\theorempostwork{\bigskip}
\newtheorem{deftheo}[rem]{Définition/Théorème}

\theoremstyle{break}
\theorembodyfont{\upshape}
\theoremheaderfont{\normalfont\bfseries}
\theoremprework{\bigskip\needspace{\baselineskip}\color{blue}\hrule\color{black}}
\theorempostwork{\bigskip}
\newtheorem{nota}[rem]{Notation}

\theoremstyle{break}
\theorembodyfont{\upshape}
\theoremheaderfont{\itshape}
\theoremprework{\bigskip\needspace{\baselineskip}\color{blue}\hrule}
\theorempostwork{\hrule\color{black}\needspace{\baselineskip}\bigskip}
\newtheorem*{brouill}{Brouillon}

\theoremstyle{break}
\theorembodyfont{\itshape}
\theoremheaderfont{\normalfont\bfseries}
\theoremprework{\bigskip\needspace{\baselineskip}\color{red}\hrule\color{black}}
\theorempostwork{\bigskip}
\newtheorem{theo}[rem]{Théorème}

\theoremstyle{break}
\theorembodyfont{\itshape}
\theoremheaderfont{\normalfont\bfseries}
\theoremprework{\bigskip\needspace{\baselineskip}\color{red}\hrule\color{black}}
\theorempostwork{\bigskip}
\newtheorem{prop}[rem]{Proposition}

\theoremstyle{break}
\theorembodyfont{\itshape}
\theoremheaderfont{\normalfont\bfseries}
\theoremprework{\bigskip\needspace{\baselineskip}\color{red}\hrule\color{black}}
\theorempostwork{\bigskip}
\newtheorem{cor}[rem]{Corollaire}

\theoremstyle{break}
\theorembodyfont{\itshape}
\theoremheaderfont{\normalfont\bfseries}
\theoremprework{\bigskip\needspace{\baselineskip}\color{red}\hrule\color{black}}
\theorempostwork{\bigskip}
\newtheorem{lem}[rem]{Lemme}

\theoremstyle{break}
\theorembodyfont{\upshape}
\theoremheaderfont{\normalfont\bfseries}
\theoremprework{\bigskip\needspace{\baselineskip}\color{violet}\hrule\color{black}}
\theorempostwork{\bigskip}
\newtheorem{meth}[rem]{Méthode}

\theoremstyle{break}
\theorembodyfont{\upshape}
\theoremheaderfont{\normalfont\bfseries}
\theoremprework{\bigskip\needspace{\baselineskip}\color{violet}\hrule\color{black}}
\theorempostwork{\bigskip}
\newtheorem{appl}[rem]{Application}

\theoremstyle{break}
\theorembodyfont{\upshape}
\theoremheaderfont{\normalfont\bfseries}
\theoremprework{\bigskip\needspace{\baselineskip}\color{violet}\hrule\color{black}}
\theorempostwork{\bigskip}
\newtheorem{abus}[rem]{Abus}

\theoremstyle{break}
\theorembodyfont{\upshape}
\theoremheaderfont{\normalfont\bfseries}
\theoremprework{\bigskip\needspace{\baselineskip}\color{violet}\hrule\color{black}}
\theorempostwork{\bigskip}
\newtheorem{algo}[rem]{Algorithme}

\theoremstyle{break}
\theorembodyfont{\upshape}
\theoremheaderfont{\normalfont\bfseries}
\theoremprework{\bigskip\needspace{\baselineskip}\color{violet}\hrule\color{black}}
\theorempostwork{\bigskip}
\newtheorem{bilan}[rem]{Bilan}

\theoremstyle{break}
\theorembodyfont{\upshape}
\theoremheaderfont{\normalfont\bfseries}
\theoremprework{\bigskip\needspace{\baselineskip}\hrule}
\theorempostwork{\bigskip}
\newtheorem{q}{Question}

\theoremstyle{break}
\theorembodyfont{\upshape}
\theoremheaderfont{\itshape}
\theoremprework{\bigskip\needspace{\baselineskip}\color{BurntOrange}\hrule\color{black}}
\theorempostwork{\bigskip}
\newtheorem{corr}[rem]{Correction}

\theoremstyle{break}
\theorembodyfont{\upshape}
\theoremheaderfont{\itshape}
\theoremsymbol{\ensuremath{\cqfd}}
\theoremprework{\bigskip\needspace{\baselineskip}\color{yellow}\hrule\color{black}}
\theorempostwork{\bigskip}
\newtheorem{dem}[rem]{Démonstration}

% Commands to make proofs easier to write
\newcommand{\impdir}{\fbox{\(\imp\)}~}
\newcommand{\imprec}{\fbox{\(\impr\)}~}
\newcommand{\incdir}{\fbox{\(\subset\)}~}
\newcommand{\increc}{\fbox{\(\supset\)}~}
\newcommand{\leqbox}{\fbox{\(\leq\)}~}
\newcommand{\geqbox}{\fbox{\(\geq\)}~}
\newcommand{\unicite}{\fbox{unicité}~}
\newcommand{\existence}{\fbox{existence}~}
\newcommand{\analyse}{\fbox{analyse}~}
\newcommand{\synthese}{\fbox{synthèse}~}
\newcommand{\conclusion}{\fbox{conclusion}~}

\renewcommand{\to}{\longrightarrow}
\renewcommand{\mapsto}{\longmapsto}

\newcommand{\fonction}[5]{\begin{array}[t]{cccc}#1 : & #2 & \to & #3 \\ & #4 & \mapsto & #5\end{array}}
\newcommand{\fonctionlambda}[4]{\begin{array}[t]{ccc}#1 & \to & #2 \\ #3 & \mapsto & #4\end{array}}

\renewcommand{\leq}{\leqslant}
\renewcommand{\geq}{\geqslant}

\newcommand{\pinf}{+\infty}
\newcommand{\minf}{-\infty}

\newcommand{\id}[1]{\mathrm{id}_{#1}}

\renewcommand{\phi}{\varphi}
\renewcommand{\epsilon}{\varepsilon}

\newcommand{\ind}[1]{\mathds{1}_{#1}}

\newcommand{\iR}{\i\R}

\newcommand{\tcheby}[2]{T_{#1}\paren{#2}}
\newcommand{\utcheby}[2]{U_{#1}\paren{#2}}

\mathcode`l="8000
\begingroup
\makeatletter
\lccode`\~=`\l
\DeclareMathSymbol{\lsb@l}{\mathalpha}{letters}{`l}
\lowercase{\gdef~{\ifnum\the\mathgroup=\m@ne \ell \else \lsb@l \fi}}%
\endgroup

\newcommand{\ensvide}{\varnothing}

\newcommand{\rond}{\circ}

\newcommand{\union}{\cup}
\newcommand{\inter}{\cap}
\newcommand{\bigunion}{\bigcup}
\newcommand{\biginter}{\bigcap}

\newcommand{\ssi}{\iff}
\newcommand{\imp}{\implies}
\newcommand{\impr}{\impliedby}

\newcommand{\excluant}{\setminus}

\newcommand{\littletaller}{\mathchoice{\vphantom{\big|}}{}{}{}}
\newcommand{\restr}[2]{{
\left.\kern-\nulldelimiterspace#1\littletaller\right|_{#2}
}}
\newcommand{\corestr}[2]{{
\left.\kern-\nulldelimiterspace#1\littletaller\right|^{#2}
}}
\newcommand{\restrbar}[1]{{
\left.\kern-\nulldelimiterspace#1\littletaller\right|
}}

\newcommand{\rel}{\mathscr{R}}

\newcommand{\classesdequiv}[1]{\nicefrac{#1}{\sim}}

\newcommand{\majo}[1]{\mathrm{majorants}\paren{#1}}
\newcommand{\mino}[1]{\mathrm{minorants}\paren{#1}}

\newcommand{\ensdiv}[1]{\operatorname{div}\paren{#1}}

\newcommand{\E}[1]{\times 10^{#1}}

\setcounter{secnumdepth}{3}

\newcommand{\guillemets}[1]{\og #1 \fg{}}

\newcommand{\prim}{^{\,\prime}}
\newcommand{\seconde}{^{\,\prime\prime}}

\newcommand{\note}[1]{\textbf{\(\star\star\) #1 \(\star\star\)}}
\newcommand{\cad}{c'est-à-dire }
\newcommand{\Cad}{C'est-à-dire }
\newcommand{\ie}{\textit{i.e.} }
\newcommand{\cf}{\textit{cf.} }
\newcommand{\Cf}{\textit{Cf.} }

\usepackage{xparse}

\NewDocumentCommand{\quantifs}{>{\SplitList{;}}m}{\ProcessList{#1}{\insertquantif}}
\newcommand{\insertquantif}[1]{#1,\;\:}

\DeclareDocumentCommand{\groupe}{m O{+}}{\paren{#1,#2}}
\DeclareDocumentCommand{\anneau}{m O{+} O{\times}}{\paren{#1,#2,#3}}
\DeclareDocumentCommand{\corps}{m O{+} O{\times}}{\paren{#1,#2,#3}}

\DeclareDocumentCommand{\poly}{O{\K} O{X}}{#1\croch{#2}}
\DeclareDocumentCommand{\polydeg}{O{\K} m O{X}}{#1_{#2}\croch{#3}}
\DeclareDocumentCommand{\fracrat}{O{\K} O{X}}{#1\paren{#2}}

\DeclareDocumentCommand{\M}{m O{\K}}{\mathscr{M}_{#1}\paren{#2}}
\DeclareDocumentCommand{\sym}{m O{\K}}{\mathscr{S}_{#1}\paren{#2}}
\DeclareDocumentCommand{\antisym}{m O{\K}}{\mathscr{A}_{#1}\paren{#2}}
\DeclareDocumentCommand{\GL}{m O{\K}}{\operatorname{GL}_{#1}\paren{#2}}
\DeclareDocumentCommand{\SL}{m O{\K}}{\operatorname{SL}_{#1}\paren{#2}}
\DeclareDocumentCommand{\Mat}{O{\fami{B}} m}{\operatorname*{Mat}_{#1}\paren{#2}}
\newcommand{\pass}[2]{\mathscr{P}_{#1\to#2}}

\DeclareDocumentCommand{\contm}{O{\intervii{a}{b}} O{\K}}{\classe{0}_m\paren{#1,#2}}
\DeclareDocumentCommand{\Esc}{O{\intervii{a}{b}} O{\K}}{\operatorname{Esc}\paren{#1,#2}}

\usepackage{witharrows}

\newcommand{\croix}{^{\times}}

\usepackage{polynom}

\newcommand{\classe}[1]{\mathscr{C}^{#1}}
\newcommand{\ensclasse}[3]{\classe{#1}\paren{#2,#3}}

\newcommand{\deriv}[1]{^{\paren{#1}}}

\usepackage{derivative}
\derivset{\pdv}[delims-eval=.)]
\derivset{\odv}[delims-eval=.)]

\DeclareMathOperator{\Arctan}{Arctan}
\DeclareMathOperator{\Arcsin}{Arcsin}
\DeclareMathOperator{\Arccos}{Arccos}
\DeclareMathOperator{\cotan}{cotan}
\DeclareMathOperator{\sh}{sh}
\DeclareMathOperator{\ch}{ch}
\DeclareMathOperator{\sg}{sg}
\DeclareMathOperator{\supp}{supp}
\DeclareMathOperator{\Supp}{Supp}
\DeclareMathOperator{\rg}{rg}
\DeclareMathOperator{\tr}{tr}

\newcommand{\Hom}[2]{\operatorname{Hom}\paren{#1,#2}}
\newcommand{\Pol}[2]{\operatorname{Pol}\paren{#1,#2}}
\newcommand{\Aut}[1]{\operatorname{Aut}\paren{#1}}
\DeclareDocumentCommand{\Vect}{O{} m}{\operatorname{Vect}_{#1}\paren{#2}}

\newcommand{\diag}[1]{\operatorname{diag}\paren{#1}}

\usepackage{abstract}
\addto\captionsfrench{\renewcommand{\abstractname}{\Large Introduction}}

\newcommand{\inv}{^{-1}}
\newcommand{\etoile}{^{*}}

\newcounter{orcounter}

\newenvironment{orlist}
{
\begin{array}{|l}
\setcounter{orcounter}{0}
}
{
\end{array}
}

\newcommand{\oritem}[1]{%
\ifthenelse{\theorcounter<1}{}{\\ \text{ou} \\}#1\stepcounter{orcounter}
}

\NewDocumentCommand{\orenv}{>{\SplitList{\\}}m}{%
\begin{orlist}\ProcessList{#1}{\oritem}\end{orlist}}

\newcounter{permuitemcounter}

\newcommand{\permuitem}[1]{%
\ifthenelse{\thepermuitemcounter<1}{}{&}#1\stepcounter{permuitemcounter}}

\NewDocumentCommand{\permu}{>{\SplitList{;}}m >{\SplitList{;}}m}{%
\begin{pmatrix}\setcounter{permuitemcounter}{0}\ProcessList{#1}{\permuitem} \\ \setcounter{permuitemcounter}{0}\ProcessList{#2}{\permuitem}\end{pmatrix}}

\NewDocumentCommand{\cycle}{>{\SplitList{;}}m}{%
\begin{pmatrix}\setcounter{permuitemcounter}{0}\ProcessList{#1}{\permuitem}\end{pmatrix}}

\usepackage{pgfplots}

\DeclareDocumentCommand{\pgcd}{o o}{
\IfNoValueTF{#1}{\operatorname{pgcd}}{\operatorname{pgcd}\paren{#1,#2}}
}

\DeclareDocumentCommand{\bezout}{o o}{
\IfNoValueTF{#1}{\operatorname{bezout}}{\operatorname{bezout}\paren{#1,#2}}
}

\usepackage{minted}
\newminted{python}{linenos, breaklines, breakanywhere, breakautoindent,tabsize=4,obeytabs}
\newenvironment{code}{\VerbatimEnvironment\begin{pythoncode}}{\end{pythoncode}}

\newcommand{\valp}[2]{v_{#1}\paren{#2}}

\newcommand{\fami}[1]{\mathscr{#1}}

\newcommand{\echange}{\leftrightarrow}

\newcommand{\trans}[1]{\prescript{t}{}{#1}}

\usepackage{mathdots}

\DeclareDocumentCommand{\detb}{O{\fami{B}}}{{\det}_{#1}}

\usepackage{cancel}

\usepackage{nicematrix}
\NiceMatrixOptions{renew-dots,renew-matrix}

\newcommand{\ps}[2]{\left\langle#1\tq#2\right\rangle}
\newcommand{\ortho}{^{\perp}}

\newcommand{\operp}{\mathrel{%
\begin{tikzpicture}[baseline=-0.25em]
\draw (0,0) circle (0.45em);
\draw (-0.38em,-0.25em) -- (0.38em,-0.25em);
\draw (0,-0.25em) -- (0,0.45em);
\end{tikzpicture}
}%
}

\usepackage{titletoc}
\dottedcontents{section}[5.5em]{}{3.2em}{1pc}

\newcommand{\bouleo}[2]{B\paren{#1,#2}}
\newcommand{\boulef}[2]{\conj{B}\paren{#1,#2}}
\newcommand{\sphere}[2]{S\paren{#1,#2}}

\newcommand{\vdv}[2]{\operatorname{D}_{#1}#2}

\newcommand{\egqd}[1]{\underset{#1}{=}}
\newcommand{\simqd}[1]{\underset{#1}{\sim}}

\newcommand{\tendqd}[1]{\xrightarrow[#1]{}}

\newcommand{\arr}[2]{A_{#2}^{#1}}
\newcommand{\comb}[2]{C_{#2}^{#1}}

\newcommand{\loiuniforme}[1]{\mathscr{U}\paren{#1}}
\newcommand{\loibernoulli}[1]{\mathscr{B}\paren{#1}}
\newcommand{\loibinomiale}[2]{\mathscr{B}\paren{#1,#2}}

\newcommand{\esp}[1]{\operatorname{E}\paren{#1}}
\newcommand{\vari}[1]{\operatorname{V}\paren{#1}}
\newcommand{\cov}[2]{\operatorname{Cov}\paren{#1,#2}}
\newcommand{\ecarttype}[1]{\sigma\paren{#1}}

\renewcommand{\O}[1]{\mathscr{O}\paren{#1}}
\renewcommand{\o}[1]{o\paren{#1}}

\setcounter{MaxMatrixCols}{200}

\newcommand{\Com}[1]{\operatorname{Com}#1}

\usepackage{microtype}

\newcommand{\sig}[1]{\epsilon\paren{#1}}

\newcommand{\legendeexercices}{\(\star\) Exercice proche du cours \\ \(\star\star\) Exercice de difficulté normale \\ \(\star\star\star\) Exercice difficile (voire très difficile) \\ \(+\) Exercice à faire en priorité \\}

\usepackage{accents}

\newcommand{\interieur}[1]{\accentset{\circ}{#1}}

\usepackage{etoolbox}

\DeclareFontFamily{U}{matha}{\hyphenchar\font45}
\DeclareFontShape{U}{matha}{m}{n}{
      <5> <6> <7> <8> <9> <10> gen * matha
      <10.95> matha10 <12> <14.4> <17.28> <20.74> <24.88> matha12
      }{}
\DeclareSymbolFont{matha}{U}{matha}{m}{n}
\DeclareFontSubstitution{U}{matha}{m}{n}

\DeclareFontFamily{U}{mathx}{\hyphenchar\font45}
\DeclareFontShape{U}{mathx}{m}{n}{
      <5> <6> <7> <8> <9> <10>
      <10.95> <12> <14.4> <17.28> <20.74> <24.88>
      mathx10
      }{}
\DeclareSymbolFont{mathx}{U}{mathx}{m}{n}
\DeclareFontSubstitution{U}{mathx}{m}{n}

\DeclareMathDelimiter{\vvvert}{0}{matha}{"7E}{mathx}{"17}
\DeclarePairedDelimiterX{\normesub}[1]
  {\vvvert}
  {\vvvert}
  {\ifblank{#1}{\:\cdot\:}{#1}}

\newcommand{\norme}[1]{\left\|\ifblank{#1}{\:\cdot\:}{#1}\right\|}

\newcommand{\tpt}{\text{pour tout }}
\newcommand{\Tpt}{\text{Pour tout }}

\newcommand{\Ann}[1]{\operatorname{Ann}\paren{#1}}

\DeclareDocumentCommand{\Sp}{O{} m}{\operatorname{Sp}_{#1}\paren{#2}}
\DeclareDocumentCommand{\sep}{O{} m m}{\operatorname{sep}_{#1}\paren{#2,#3}}
\DeclareDocumentCommand{\rac}{O{\K} m}{\operatorname{Z}_{#1}\paren{#2}}

\DeclareDocumentCommand{\diago}{m O{\K}}{\mathscr{D}_{#1}\paren{#2}}
\DeclareDocumentCommand{\Tsup}{m O{\K}}{\mathscr{T}_{#1}^+\paren{#2}}

\renewcommand{\subsetneq}{\varsubsetneq}

\begin{document}
\renewcommand{\labelitemi}{\(\triangleright\)}
\renewcommand{\labelenumi}{(\arabic{enumi})}

\everymath{\ds}

\maketitle

\begin{abstract}
Ce document réunit l'ensemble de mes cours de Mathématiques de MPI, ainsi que les exercices les accompagnant. Le professeur était M. Walbron. J'ai adapté certaines formulations me paraissant floues ou ne me plaisant pas mais le contenu pur des cours est strictement équivalent.

Les éléments des tables des matières initiale et présentes au début de chaque chapitre sont cliquables (amenant directement à la partie cliquée). C'est également le cas des références à des éléments antérieurs de la forme, par exemple, \guillemets{Démonstration 5.22}.

Cette version contient, en plus des cours imprimés distribués durant l'année, toutes les démonstrations qui vont avec. Voir l'autre version pour n'avoir que les cours bruts.
\end{abstract}

\dominitoc\tableofcontents

\part{Cours}

\chapter{Espaces vectoriels normés}

\minitoc

Dans ce chapitre, la lettre \(\K\) désigne \(\R\) ou \(\C\).

\section{Bornes supérieures, bornes inférieures}

\subsection{Borne supérieure d'une partie de \(\R\)}

On rappelle le théorème fondamental, dit \guillemets{théorème (ou axiome) de la borne supérieure}.

\begin{theo}
Toute partie \(A\) de \(\R\), non-vide et majorée, possède une borne supérieure, notée \(\sup A\).

Toute partie \(A\) de \(\R\), non-vide et minorée, possède une borne inférieure, notée \(\inf A\).
\end{theo}

On dispose de caractérisations équivalentes de la borne supérieure.

\begin{prop}\thlabel{prop1}
Soient \(A\) une partie de \(\R\), non-vide et majorée, et \(s\) un réel.

Alors il y a équivalence entre les propositions suivantes :

\begin{enumerate}
    \item[\(\paren{\alpha}\)] \(s=\sup A\) \\
    \item[\(\paren{\beta}\)] \(\begin{dcases}\quantifs{\forall a\in A}a\leq s \\ \quantifs{\forall\epsilon>0;\exists x\in A}s-\epsilon<x\leq s\end{dcases}\) \\
    \item[\(\paren{\gamma}\)] \(\begin{dcases}\quantifs{\forall a\in A}a\leq s \\ \quantifs{\exists\paren{x_n}\in A^\N}x_n\tendqd{n\to\pinf}s\end{dcases}\)
\end{enumerate}
\end{prop}

\begin{dem}[\(\paren{\alpha}\imp\paren{\beta}\)]
\begin{itemize}
    \item \(s=\sup A\) est le plus petit majorant de \(A\) donc c'est un majorant de \(A\) : \[\quantifs{\forall a\in A}a\leq s.\]
    \item \(s\) est le plus petit majorant de \(A\) donc \[\quantifs{\forall\epsilon>0}s-\epsilon<s\] donc \(s-\epsilon\) n'est pas un majorant de \(A\). \\ Donc il existe \(x\in A\) tel que \(s-\epsilon<x\leq s\).
\end{itemize}
\end{dem}

\begin{dem}[\(\paren{\beta}\imp\paren{\alpha}\)]
\(s\) est un majorant de \(A\) et tout réel strictement inférieur à \(s\) n'est pas un majorant \ie tout majorant est supérieur ou égal à \(s\).

Donc \(s=\sup A\).
\end{dem}

\begin{dem}[\(\paren{\beta}\imp\paren{\gamma}\)]~\\
On spécialise \(\epsilon\gets\dfrac{1}{n+1}\) pour \(n\in\N\). On a \[\quantifs{\forall n\in\N;\exists x_n\in A}s-\dfrac{1}{n+1}<x_n\leq s.\]

De cette façon, on construit une suite \(\paren{x_n}\in A^\N\) telle que \[\quantifs{\forall n\in\N}s-\dfrac{1}{n+1}<x_n\leq s.\]

D'après le théorème des gendarmes, on a \(x_n\tendqd{n\to\pinf}s\).
\end{dem}

\begin{dem}[\(\paren{\gamma}\imp\paren{\beta}\)]
Soit \(\epsilon>0\).

Il existe \(N\in\N\) tel que \[\begin{aligned}
\quantifs{\forall n\geq N}\abs{x_n-s}&\leq\dfrac{\epsilon}{2} \\
\text{donc }s-\dfrac{\epsilon}{2}&\leq x_n.
\end{aligned}\]

Or \(x_n\in A\) donc \(s-\epsilon<s-\dfrac{\epsilon}{2}\leq x_n\leq s\).

D'où \[\quantifs{\forall\epsilon>0;\exists x\in A}s-\epsilon<x\leq s.\]
\end{dem}

On a évidemment les caractérisations associées à la borne inférieure.

\subsection{Borne supérieure d'une application à valeurs dans \(\R\)}

\begin{defi}
Soient \(X\) un ensemble non-vide et \(f:X\to\R\).

Si \(f\) est majorée sur \(X\), alors on appelle borne supérieure de \(f\) sur \(X\) le réel \(\sup f\paren{X}=\sup_Xf=\sup_{x\in X}f\paren{x}\).

Si \(f\) est minorée sur \(X\), alors on appelle borne inférieure de \(f\) sur \(X\) le réel \(\inf f\paren{X}=\inf_Xf=\inf_{x\in X}f\paren{x}\).
\end{defi}

On déduit de la \thref{prop1} les caractérisations suivantes.

\begin{prop}
Soient \(X\) un ensemble non-vide, \(f:X\to\R\) majorée sur \(X\) et \(s\) un réel.

Alors il y a équivalence entre les propositions suivantes :

\begin{itemize}
    \item \(s=\sup_Xf\) \\
    \item \(\begin{dcases}\quantifs{\forall x\in X}f\paren{x}\leq s \\ \quantifs{\forall\epsilon>0;\exists x\in X}s-\epsilon<f\paren{x}\leq s\end{dcases}\) \\
    \item \(\begin{dcases}\quantifs{\forall x\in X}f\paren{x}\leq s \\ \quantifs{\exists\paren{x_n}\in X^\N}f\paren{x_n}\tendqd{n\to\pinf}s\end{dcases}\)
\end{itemize}
\end{prop}

\subsection{Règles pratiques}

D'abord, des évidences auxquelles on ne pense pas toujours.

\begin{prop}
Soit \(A\) une partie de \(\R\), non-vide et majorée. Alors \(\quantifs{\forall a\in A}a\leq\sup A\).

Soient \(X\) un ensemble non-vide et \(f:X\to\R\) majorée sur \(X\). Alors \(\quantifs{\forall x\in X}f\paren{x}\leq\sup_Xf\).
\end{prop}

En pratique, on n'a pas souvent besoin de connaître la valeur exacte d'une borne supérieure, on a plus souvent besoin de la majorer.

\begin{prop}\thlabel{prop1.4}
\begin{itemize}
    \item Soient \(A\) une partie de \(\R\), non-vide et majorée, et \(M\) un réel. \\ Pour montrer \(\sup A\leq M\), il suffit de montrer \(\quantifs{\forall a\in A}a\leq M\). \\
    \item Soient \(X\) un ensemble non-vide, \(f:X\to\R\) majorée sur \(X\) et \(M\) un réel. \\ Pour montrer \(\sup_Xf\leq M\), il suffit de montrer \(\quantifs{\forall x\in X}f\paren{x}\leq M\).
\end{itemize}
\end{prop}

Multiplication par un réel positif.

\begin{prop}\thlabel{prop1.5}
Soient \(X\) un ensemble non-vide et \(f:X\to\R\) majorée sur \(X\).

Alors \(\quantifs{\tpt\lambda\geq0}\sup_X\paren{\lambda f}=\lambda\sup_Xf\).
\end{prop}

\begin{dem}
Soit \(\lambda\geq0\). On pose \(s=\sup_Xf\).

On veut montrer \(\sup_X\paren{\lambda f}=\lambda s\).

On a \(\quantifs{\forall x\in X}f\paren{x}\leq s\) et \(\lambda\geq0\) donc \[\quantifs{\forall x\in X}\lambda f\paren{x}\leq\lambda s.\]

Donc \(\lambda s\) est un majorant de \(\lambda f\).

Comme \(s=\sup_Xf\), il existe \(\paren{x_n}\in X^\N\) telle que \(f\paren{x_n}\tendqd{n\to\pinf}s\).

D'après les théorèmes d'opération sur les limites, on a \[\lambda f\paren{x_n}\tendqd{n\to\pinf}\lambda s.\]

D'où \(\lambda s=\sup_X\paren{\lambda f}\) d'après la \thref{prop1}.
\end{dem}

Attention ! C'est bien sûr faux si \(\lambda<0\).

\section{Normes}

\subsection{Définition}

\begin{defi}
Soit \(E\) un \(\K\)-espace vectoriel.

On appelle norme sur \(E\) toute application \(N:E\to\Rp\) telle que :

\begin{itemize}
    \item \(\quantifs{\tpt x\in E}N\paren{x}=0\ssi x=0\qquad\text{(séparation)}\) \\
    \item \(\quantifs{\tpt x\in E;\tpt\lambda\in\K}N\paren{\lambda x}=\abs{\lambda}N\paren{x}\qquad\text{(homogénéité)}\) \\
    \item \(\quantifs{\tpt\paren{x,y}\in E^2}N\paren{x+y}\leq N\paren{x}+N\paren{y}\qquad\text{(inégalité triangulaire)}\).
\end{itemize}

Un espace vectoriel est dit espace vectoriel normé quand on lui associe une norme.
\end{defi}

On déduit de l'inégalité triangulaire une inégalité classique (souvent appelée aussi inégalité triangulaire) : \[\quantifs{\tpt\paren{x,y}\in E^2}\abs{N\paren{x}-N\paren{y}}\leq N\paren{x-y}.\]

\begin{dem}
Soit \(\paren{x,y}\in E^2\).

D'après l'inégalité triangulaire, on a \[\begin{aligned}
N\paren{x-y+y}&\leq N\paren{x-y}+N\paren{y} \\
N\paren{x}&\leq N\paren{x-y}+N\paren{y} \\
N\paren{x}-N\paren{y}&\leq N\paren{x-y}.
\end{aligned}\]

De même, en échangeant \(x\) et \(y\) : \[N\paren{y}-N\paren{x}\leq N\paren{y-x}.\]

Or \[N\paren{x-y}=N\paren{-\paren{y-x}}=\abs{-1}N\paren{y-x}=N\paren{y-x}.\]

D'où \[\begin{dcases}
N\paren{x}-N\paren{y}\leq N\paren{x-y} \\
N\paren{y}-N\paren{x}\leq N\paren{x-y}
\end{dcases}\] donc \[\abs{N\paren{x}-N\paren{y}}\leq N\paren{x-y}.\]
\end{dem}

Si \(N\) est une norme sur \(E\), alors on peut définir une distance entre deux vecteurs de \(E\) : \(d\paren{u,v}=N\paren{u-v}\).

On définit ainsi une application \(d:E^2\to\Rp\) telle que :

\begin{itemize}
    \item \(\quantifs{\tpt\paren{x,y}\in E^2}d\paren{y,x}=d\paren{x,y}\qquad\text{(symétrie)}\) \\
    \item \(\quantifs{\tpt\paren{x,y}\in E^2}d\paren{x,y}=0\ssi x=y\qquad\text{(séparation)}\) \\
    \item \(\quantifs{\tpt\paren{x,y,z}\in E^3}d\paren{x,z}\leq d\paren{x,y}+d\paren{y,z}\qquad\text{(inégalité triangulaire)}\).
\end{itemize}

\subsection{Exemples fondamentaux}

\begin{itemize}
    \item La valeur absolue dans \(\R\) et le module dans \(\C\) sont des normes. \\
    \item La norme euclidienne habituelle en géométrie plane ou spatiale est une norme. \\
    \item Plus généralement, si \(\ps{\cdot}{\cdot}\) est un produit scalaire sur \(E\), la norme euclidienne associée \(x\mapsto\sqrt{\ps{x}{x}}\) est une norme au sens précédent. \\
    \item Soit \(E\) un \(\K\)-espace vectoriel de dimension finie. On choisit une base de \(E\) \(\fami{B}=\paren{e_1,\dots,e_n}\). Si \(v\) est un vecteur de \(E\), on note \(\paren{v_1,\dots,v_n}\) les coordonnées de \(v\) dans la base \(\fami{B}\). \\ On définit classiquement trois normes sur \(E\) : \[\norme{v}_\infty=\max_{i\in\interventierii{1}{n}}\abs{v_i}\qquad\norme{v}_1=\sum_{i=1}^n\abs{v_i}\qquad\norme{v}_2=\sqrt{\sum_{i=1}^n\abs{v_i}^2}\] appelées respesctivement norme infinie ou norme sup, norme 1 et norme 2. \\ Cas particulier : \(E=\R^n\) muni de la base canonique. \\ Cas particulier : \(E=\M{n,p}\) muni de la base canonique. Si \(A=\paren{a_{ij}}_{\substack{1\leq i\leq n \\ 1\leq j\leq p}}\) est une matrice de \(\M{n,p}\), alors \[\norme{A}_\infty=\max_{\substack{1\leq i\leq n \\ 1\leq j\leq p}}\abs{a_{ij}}\qquad\norme{A}_1=\sum_{\substack{1\leq i\leq n \\ 1\leq j\leq p}}\abs{a_{ij}}\qquad\norme{A}_2=\sqrt{\sum_{\substack{1\leq i\leq n \\ 1\leq j\leq p}}\abs{a_{ij}}^2}\]
    \item Soient \(X\) un ensemble et \(E\) l'ensemble des applications bornées de \(X\) dans \(\K\). La norme sup sur \(E\) est définie par \(\norme{f}_\infty=\sup_{x\in X}\abs{f\paren{x}}\). \\ Cas particulier : si \(X=\N\), \(E\) est l'ensemble des suites bornées et \(\norme{u}_\infty=\sup_{n\in\N}\abs{u_n}\).
\end{itemize}

\begin{dem}[\(\norme{}_\infty\) est une norme sur \(E\)]\thlabel{dem:normeInfEstUneNorme}
\begin{itemize}
    \item Si \(\norme{v}_\infty=0\) alors par définition d'un maximum \[\quantifs{\forall i\in\interventierii{1}{n}}0\leq\abs{v_i}\leq\norme{v}_\infty=0\text{ donc }v_i=0\] donc \(v=0\). \\
    \item Soit \(\lambda\in\K\). On a \[\begin{WithArrows}
        \norme{\lambda v}_\infty&=\sup_{i\in\interventierii{1}{n}}\abs{\lambda v_i} \\
        &=\sup_{i\in\interventierii{1}{n}}\abs{\lambda}\times\abs{v_i} \Arrow{\(\abs{\lambda}\geq0\) (\thref{prop1.5})} \\
        &=\abs{\lambda}\sup_{i\in\interventierii{1}{n}}\abs{v_i} \\
        &=\abs{\lambda}\norme{v}_\infty.
    \end{WithArrows}\]
    \item Soit \(\paren{v,w}\in E^2\). On veut montrer \(\norme{v+w}_\infty\leq\norme{v}_\infty+\norme{w}_\infty\). \\\\ D'après la \thref{prop1.4}, il suffit de montrer \[\quantifs{\forall i\in\interventierii{1}{n}}\abs{v_i+w_i}\leq\norme{v}_\infty+\norme{w}_\infty.\] Pour \(i\in\interventierii{1}{n}\), on a \[\begin{WithArrows}
        \abs{v_i+w_i}&\leq\abs{v_i}+\abs{w_i} \Arrow{\(\begin{dcases}\abs{v_i}\leq\norme{v}_\infty \\ \abs{w_i}\leq\norme{w}_\infty\end{dcases}\)} \\
        &\leq\norme{v}_\infty+\norme{w}_\infty.
    \end{WithArrows}\] Donc \(\norme{v+w}_\infty\leq\norme{v}_\infty+\norme{w}_\infty\).
\end{itemize}
\end{dem}

\begin{dem}[\(\norme{}_1\) est une norme sur \(E\)]\thlabel{dem:norme1EstUneNorme}~\\
\begin{itemize}
    \item Si \(\norme{v}_1=0\) alors \(\sum_{i=1}^n\abs{v_i}=0\). Or une somme de réels positifs est nulle ssi tous les réels sont nuls. Donc \[\quantifs{\forall i\in\interventierii{1}{n}}\abs{v_i}=0\] donc \(v=0\). \\
    \item Soit \(\lambda\in\R\). On a \[\begin{aligned}
        \norme{\lambda v}_1&=\sum_{i=1}^n\abs{\lambda v_i} \\
        &=\sum_{i=1}^n\abs{\lambda}\times\abs{v_i} \\
        &=\abs{\lambda}\sum_{i=1}^n\abs{v_i} \\
        &=\abs{\lambda}\norme{v}_1.
    \end{aligned}\]
    \item Soit \(\paren{v,w}\in E^2\). On a \[\begin{aligned}
        \norme{v+w}_1&=\sum_{i=1}^n\abs{v_i+w_i} \\
        &\leq\sum_{i=1}^n\paren{\abs{v_i}+\abs{w_i}} \\
        &=\norme{v}_1+\norme{w}_1.
    \end{aligned}\]
\end{itemize}
\end{dem}

\begin{prop}
Soient \(E,F\) deux espaces vectoriels normés.

L'application de \(E\times F\) dans \(\Rp\) qui à \(\paren{x,y}\) associe \(\max\paren{\norme{x}_E,\norme{y}_F}\) est une norme.

Autrement dit, le produit de deux espaces vectoriels normés est encore un espace vectoriel normé, résultat qui se généralise par récurrence à un nombre quelconque (fini) d'espaces vectoriels normés.
\end{prop}

\subsection{Normes équivalentes}

\begin{defi}
Soient \(E\) un \(\K\)-espace vectoriel et \(N_1,N_2\) deux normes sur \(E\).

On dit que \(N_1\) et \(N_2\) sont équivalentes quand il existe deux constantes strictement positives \(a,b\) telles que \(\quantifs{\tpt v\in E}aN_1\paren{v}\leq N_2\paren{v}\leq bN_1\paren{v}\).
\end{defi}

\begin{prop}
On note \(\normes{E}\) l'ensemble des normes sur \(E\) et \[\quantifs{\forall\paren{N_1,N_2}\in\normes{E}^2}N_1\sim N_2\ssi N_1\text{ et }N_2\text{ sont équivalentes}.\]

La relation \(\sim\) est alors une relation d'équivalence sur \(\normes{E}\).
\end{prop}

\begin{dem}
\begin{itemize}
    \item Soit \(N\in\normes{E}\). On a \(N\leq N\leq N\) donc \(N\sim N\). \\\\ Donc \(\sim\) est réflexive. \\
    \item Soit \(\paren{N_1,N_2}\in\normes{E}^2\) tel que \(N_1\sim N_2\). \\\\ Il existe \(a,b>0\) tels que \(aN_1\leq N_2\leq bN_1\). \\\\ Donc \(\dfrac{1}{b}N_2\leq N_1\leq\dfrac{1}{a}N_2\) \ie \(N_2\sim N_1\). \\\\ Donc \(\sim\) est symétrique. \\
    \item Soit \(\paren{N_1,N_2,N_3}\in\normes{E}^3\) tel que \(N_1\sim N_2\) et \(N_2\sim N_3\). \\\\ Il existe \(a,b,c,d>0\) tels que \(\begin{dcases}
        aN_1\leq N_2\leq bN_1 \\
        cN_2\leq N_3\leq dN_2
    \end{dcases}\) \\\\ Donc \(acN_1\leq N_3\leq bdN_1\) \ie \(N_1\sim N_3\). \\\\ Donc \(\sim\) est transitive. \\
\end{itemize}

Finalement, \(\sim\) est une relation d'équivalence.
\end{dem}

\begin{exo}\thlabel{exo1.1}
Montrez que si \(E\) est de dimension finie, les trois normes \(\norme{}_\infty\), \(\norme{}_1\) et \(\norme{}_2\) sont équivalentes.
\end{exo}

\begin{corr}
Soient \(E\) un espace vectoriel normé de dimension finie et \(\fami{B}\) une base de \(E\).

Soit \(v=\paren{v_1,\dots,v_n}_\fami{B}\).

On a \[\norme{v}_\infty=\sup_{i\in\interventierii{1}{n}}\abs{v_i}\qquad\norme{v}_1=\sum_{i=1}^n\abs{v_i}\qquad\norme{v}_2=\sqrt{\sum_{i=1}^n\abs{v_i}^2}.\]

\begin{itemize}
    \item On a \(\norme{v}_\infty\leq\norme{v}_1\leq n\norme{v}_\infty\). \\\\ En effet, il existe \(j\in\interventierii{1}{n}\) tel que \(\norme{v}_\infty=\abs{v_j}\) donc \[\norme{v}_\infty=\abs{v_j}\leq\abs{v_1}+\dots+\abs{v_n}=\norme{v}_1.\] De plus, \(\quantifs{\tpt i\in\interventierii{1}{n}}\abs{v_i}\leq\norme{v}_\infty\) donc en additionnant les inégalités : \[\sum_{i=1}^n\abs{v_i}=\norme{v}_1\leq\sum_{i=1}^n\norme{v}_\infty=n\norme{v}_\infty.\] Donc \(\norme{}_\infty\sim\norme{}_1\). \\
    \item En mettant des carrés partout on arrive à \[\norme{v}_\infty\leq\norme{v}_2\leq\sqrt{n}\norme{v}_\infty.\] Donc \(\norme{}_\infty\sim\norme{}_2\). \\
    \item Par transitivité, on a \(\norme{}_1\sim\norme{}_2\).
\end{itemize}
\end{corr}

\begin{exo}
Soient \(E=\poly[\R]\) et \(P=\sum_{i=0}^na_iX^i\in E\). On pose \(N_1\paren{P}=\sum_{i=0}^n\abs{a_i}\) et \(N_\infty\paren{P}=\max_{0\leq i\leq n}\abs{a_i}\).

Montrez que \(N_1\) et \(N_\infty\) sont des normes sur \(E\).

Montrez qu'elles ne sont pas équivalentes en considérant la suite des polynômes \(P_n=\sum_{i=0}^nX^i\).
\end{exo}

\begin{corr}
\begin{itemize}
    \item \(N_1\) et \(N_\infty\) sont clairement des normes (\cf \thref{dem:normeInfEstUneNorme} et \thref{dem:norme1EstUneNorme}). \\
    \item Pour \(n\in\N\), on pose \(P_n=\sum_{i=0}^nX^i=1+X+\dots+X^n\). \\\\ Par l'absurde, on suppose \(N_1\sim N_\infty\). \\\\ Il existe \(a,b>0\) tels que \(aN_1\leq N_\infty\leq bN_1\). \\\\ Donc \[\quantifs{\forall n\in\N}a\underbrace{N_1\paren{P_n}}_{=n+1}\leq \underbrace{N_\infty\paren{P_n}}_{=1}\leq b\underbrace{N_1\paren{P_n}}_{=n+1}\] donc \[\quantifs{\forall n\in\N}a\paren{n+1}\leq1\] or \(\lim_{n\to\pinf}a\paren{n+1}=\pinf\) : contradiction. \\\\ Donc \(N_1\) et \(N_\infty\) ne sont pas équivalentes.
\end{itemize}
\end{corr}

\begin{rem}
Soit \(\paren{N_1,N_2}\in\normes{E}^2\). On a \[N_1\sim N_2\ssi v\mapsto\dfrac{N_1\paren{v}}{N_2\paren{v}}\text{ est bornée sur }E\excluant\accol{0}.\]

Ainsi, pour montrer que \(N_1\) et \(N_2\) ne sont pas équivalentes, on cherche une suite \(\paren{v_n}\in\paren{E\excluant\accol{0}}^\N\) telle que \[\dfrac{N_1\paren{v_n}}{N_2\paren{v_n}}\tendqd{n\to\pinf}\pinf\qquad\text{ou}\qquad\dfrac{N_1\paren{v_n}}{N_2\paren{v_n}}\tendqd{n\to\pinf}0.\]
\end{rem}

Le résultat suivant est fondamental.

\begin{theo}
Si \(E\) est un \(\K\)-espace vectoriel de dimension finie, alors toutes les normes sur \(E\) sont équivalentes.
\end{theo}

Quand on est en dimension finie, cela signifie que tous les résultats qu'on peut démontrer pour une norme sont à facteurs près valables pour n'importe quelle norme, autrement dit cela nous permettra de choisir la norme que l'on préfère si on ne nous l'impose pas.

Dans toute la suite, \(E\) est un espace vectoriel normé par la norme \(\norme{}\).

\subsection{Boules}

\begin{defi}
Soient \(a\in E\) et \(r\in\Rps\).

On appelle boule ouverte de centre \(a\) et de rayon \(r\) l'ensemble noté \(\bouleo{a}{r}\) défini de la façon suivante : \[\bouleo{a}{r}=\accol{v\in E\tq\norme{v-a}<r}.\]

On appelle boule fermée de centre \(a\) et de rayon \(r\) l'ensemble noté (généralement) \(\boulef{a}{r}\) : \[\boulef{a}{r}=\accol{v\in E\tq\norme{v-a}\leq r}.\]

On appelle sphère de centre \(a\) et de rayon \(r\) l'ensemble (généralement) noté \(\sphere{a}{r}\) : \[\sphere{a}{r}=\accol{v\in E\tq\norme{v-a}=r}.\]
\end{defi}

On appelle boule-unité la boule de centre \(0\) et de rayon \(1\), sphère-unité la sphère de centre \(0\) et de rayon \(1\).

\begin{exo}
Que sont les boules dans \(\R\) ? Que sont les sphères dans \(\R\) ?
\end{exo}

\begin{corr}
\(\abs{\cdot}\) est une norme sur \(\R\).

Soient \(a\in\R\) et \(r>0\).

On a \[\bouleo{a}{r}=\intervee{a-r}{a+r}\qquad\boulef{a}{r}=\intervii{a-r}{a+r}\qquad\sphere{a}{r}=\accol{a-r,a+r}.\]
\end{corr}

\begin{exo}
On prend \(E=\R^2\) et on définit les normes infinie, 1 et 2 relativement à la base canonique.

Représentez graphiquement les boules-unités pour chacune de ces trois normes.
\end{exo}

\begin{corr}
On a \[\norme{\paren{x,y}}_\infty=\max\paren{\abs{x},\abs{y}}\qquad\norme{\paren{x,y}}_1=\abs{x}+\abs{y}\qquad\norme{\paren{x,y}}_2=\sqrt{x^2+y^2}.\]

On en déduit les boules suivantes :

\begin{center}
\begin{tikzpicture}
\draw[->,ultra thick] (-2,0) -- ++(4,0) node[right] {\(x\)};
\draw[->,ultra thick] (0,-2) -- ++(0,4) node[above] {\(y\)};
\draw[pattern=north east lines] (-1,-1) -- ++(2,0) node[below right] {\(B_\infty\paren{0,1}\)} -- ++(0,2) -- ++(-2,0) -- ++(0,-2);
\node[below right] at (1,0) {\(1\)};
\node[below left] at (-1,0) {\(-1\)};
\node[above left] at (0,1) {\(1\)};
\node[below left] at (0,-1) {\(-1\)};

\draw[->,ultra thick] (4,0) -- ++(4,0) node[right] {\(x\)};
\draw[->,ultra thick] (6,-2) -- ++(0,4) node[above] {\(y\)};
\draw[pattern=north east lines] (6,0) circle (1);
\node[below right] at (7,0) {\(1\)};
\node[below left] at (5,0) {\(-1\)};
\node[above left] at (6,1) {\(1\)};
\node[below left] at (6,-1) {\(-1\)};
\node[right] at (7,-1) {\(B_2\paren{0,1}\)};

\draw[->,ultra thick] (10,0) -- ++(4,0) node[right] {\(x\)};
\draw[->,ultra thick] (12,-2) -- ++(0,4) node[above] {\(y\)};
\draw[pattern=north east lines] (13,0) -- (12,1) -- (11,0) -- (12,-1) -- (13,0) node[midway,below right] {\(B_1\paren{0,1}\)};
\node[below right] at (13,0) {\(1\)};
\node[below left] at (11,0) {\(-1\)};
\node[above left] at (12,1) {\(1\)};
\node[below left] at (12,-1) {\(-1\)};
\end{tikzpicture}
\end{center}
\end{corr}

\begin{exo}
Montrez que toute boule ouverte est contenue dans une boule fermée et contient une boule fermée de mêmes centres.

Montrez la même chose en inversant les mots \guillemets{ouverte} et \guillemets{fermée}.
\end{exo}

\begin{corr}
Soient \(a\in E\) et \(r>0\).

On a \[\boulef{a}{\dfrac{r}{2}}\subset\bouleo{a}{r}\subset\boulef{a}{r}\] et \[\bouleo{a}{r}\subset\boulef{a}{r}\subset\bouleo{a}{2r}.\]
\end{corr}

\begin{defi}
Soit \(\paren{x,y}\in E^2\). On note \(\croch{xy}=\accol{tx+\paren{1-t}y\tq t\in\intervii{0}{1}}\), appelé segment (géométrique) d'extrémités \(x\) et \(y\).

Une partie \(A\) de \(E\) est dite convexe quand \(\quantifs{\tpt\paren{x,y}\in A^2}\croch{xy}\subset A\).

On a : \[A\text{ est convexe}\ssi\quantifs{\forall\paren{x,y}\in A^2;\forall t\in\intervii{0}{1}}tx+\paren{1-t}y\in A.\]
\end{defi}

\begin{prop}
Les boules (ouvertes ou fermées) sont des parties convexes.

Les sphères ne sont jamais convexes.

Dans \(\R\), les convexes sont les intervalles.
\end{prop}

\begin{dem}[Les boules sont convexes]
Soient \(a\in E\) et \(r>0\).

Soient \(\paren{x,y}\in\bouleo{a}{r}^2\) et \(t\in\intervii{0}{1}\).

On veut montrer que \(tx+\paren{1-t}y\in\bouleo{a}{r}\) \ie \(\norme{tx+\paren{1-t}y-a}<r\).

On a \[\begin{WithArrows}
\norme{tx+\paren{1-t}y-a}&=\norme{t\paren{x-a}+at+\paren{1-t}y-a} \\
&=\norme{t\paren{x-a}+\paren{t-1}a+\paren{1-t}y} \\
&=\norme{t\paren{x-a}+\paren{1-t}\paren{y-a}} \Arrow{inégalité triangulaire} \\
&\leq\norme{t\paren{x-a}}+\norme{\paren{1-t}\paren{y-a}} \Arrow[tikz={text width=5cm}]{homogénéité et \(t\geq0\) et \(1-t\geq0\)} \\
&=t\norme{x-a}+\paren{1-t}\norme{y-a} \Arrow{\(x,y\in\bouleo{a}{r}\)} \\
&<tr+\paren{1-t}r \\
&=r.
\end{WithArrows}\]

Ceci prouve \[\quantifs{\forall\paren{x,y}\in\bouleo{a}{r}}\croch{xy}\subset\bouleo{a}{r}\] \ie \(\bouleo{a}{r}\) est convexe.

De même, \(\boulef{a}{r}\) est convexe.
\end{dem}

\begin{dem}[Les sphères ne sont pas convexes]
Soient \(a\in E\) et \(r>0\).

On veut montrer que \(\sphere{a}{r}\) n'est pas convexe \ie \[\quantifs{\exists\paren{x,y}\in\sphere{a}{r}^2;\exists t\in\intervii{0}{1}}tx+\paren{1-t}y\not\in\sphere{a}{r}.\]

On choisit \(x\in\sphere{a}{r}\) puis \(y=2a-x\) (diamétralement opposé).

On a \[a\in\croch{xy}\] car \(a=\dfrac{1}{2}x+\dfrac{1}{2}y\) et \[a\not\in\sphere{a}{r}.\]

Donc \(\sphere{a}{r}\) n'est pas convexe.
\end{dem}

\subsection{Parties bornées}

\begin{defi}
On dit qu'une partie \(A\) de \(E\) est bornée quand il existe une boule qui la contient.
\end{defi}

\begin{exo}
Montrez que \(A\) est bornée ssi \(A\) est contenue dans une boule de centre \(0\).

Plus généralement, on choisit arbitrairement un point de \(E\), noté \(x\). Montrez l'équivalence \(A\) est bornée ssi \(A\) est contenue dans une boule de centre \(x\).
\end{exo}

\begin{corr}[Première équivalence]
\imprec Immédiat.

\impdir

Soit \(A\) une partie bornée de \(E\).

Il existe \(a\in E\) et \(r>0\) tels que \(A\subset\bouleo{a}{r}\).

On pose \(r\prim=\norme{a}+r>0\).

Soit \(x\in A\).

On a \(x\in\bouleo{a}{r}\) donc \(\norme{x-a}<r\).

Donc \(r>\abs{\norme{x}-\norme{a}}\).

Donc \(r>\norme{x}-\norme{a}\).

Donc \(r+\norme{a}>\norme{x}\).

Donc \(A\subset\bouleo{0}{r\prim}\).
\end{corr}

\begin{corr}[Seconde équivalence]
Plus généralement, soit \(x\in E\).

Avec \(r\seconde=\norme{x-a}+r\), on a de même \(A\subset\bouleo{x}{r\seconde}\).
\end{corr}

\begin{exo}
Montrez qu'en dimension finie, cette définition ne dépend pas de la norme.
\end{exo}

\begin{corr}
On veut montrer que si \(E\) est de dimension finie et \(N_1,N_2\) sont des normes sur \(E\), alors \[A\text{ est bornée pour }N_1\ssi A\text{ est bornée pour }N_2.\]

Soit \(A\) une partie bornée pour \(N_1\).

Alors il existe \(r_1>0\) tel que \(A\subset\bouleo[1]{0}{r_1}\).

Comme \(E\) est de dimension finie, les normes \(N_1\) et \(N_2\) sont équivalentes.

Il existe donc \(a,b>0\) tels que \(aN_2\leq N_1\leq bN_2\).

On pose \(r_2=\dfrac{r_1}{a}>0\).

Soit \(x\in A\).

Alors \(x\in\bouleo[1]{0}{r_1}\) donc \[\begin{aligned}
N_1\paren{x}&<r_1 \\
N_2\paren{x}&\leq\dfrac{1}{a}N_1\paren{x}<r_2.
\end{aligned}\]

Donc \(x\in\bouleo[2]{0}{r_2}\).

Donc \(A\subset\bouleo[2]{0}{r_2}\).

Donc \(A\) est bornée pour \(N_2\).

Et réciproquement.
\end{corr}

\begin{prop}
Une partie \(A\) de \(E\) n'est pas bornée ssi il existe une suite \(\paren{v_n}\) à termes dans \(A\) telle que \(\norme{v_n}\tendqd{n\to\pinf}\pinf\).
\end{prop}

\begin{exo}
Dans \(E=\R^2\), on pose \(A=\accol{\paren{x,y}\tq x^4+y^4=20}\) : \(A\) est-elle bornée ? Si oui, pour chacune des normes infinie, 1 et 2, donnez un rayon d'une boule centrée en \(0\) qui contient \(A\).
\end{exo}

\begin{corr}
\(E\) est de dimension finie donc les normes sur \(E\) sont toutes équivalentes.

On choisit la norme sup.

Soit \(\paren{x,y}\in A\).

On a \(x^4+y^4=20\) donc \(x^4\leq x^4+y^4=20\) car \(x^4\geq0\).

Donc \(\abs{x}\leq\sqrt[4]{20}\) et, de même, \(\abs{y}\leq\sqrt[4]{20}\).

Donc \(\paren{x,y}\in\boulef[\infty]{0}{\sqrt[4]{20}}\).

Donc \(A\subset\boulef[\infty]{0}{\sqrt[4]{20}}\).

D'après l'\thref{exo1.1}, on a \[\norme{}_\infty\leq\norme{}_1\leq2\norme{}_\infty\] et \[\norme{}_\infty\leq\norme{}_2\leq\sqrt{2}\norme{}_\infty.\]

Donc \(A\subset\boulef[1]{0}{2\sqrt[4]{20}}\) et \(A\subset\boulef[2]{0}{\sqrt{2}\sqrt[4]{20}}\).
\end{corr}

\begin{exo}
Même question avec \(E=\C^2\).
\end{exo}

\begin{corr}
\(E\) est de dimension finie donc les normes sur \(E\) sont équivalentes.

\(x\) étant quelconque dans \(\C\), existe-t-il \(y\in\C\) tel que \(x^4+y^4=20\) ?

Dans \(\C\), tout nombre possède une racine quatrième donc en posant \(y\) une racine quatrième de \(20-x^4\), on obtient \(\paren{x,y}\in A\).

On pose \(v_n=\paren{n,y_n}\) où \(y_n\) est une racine quatrième de \(20-n^4\).

On a \(\norme{v_n}_\infty=\max\paren{n,\abs{y_n}}\geq n\) donc \[\norme{v_n}_\infty\tendqd{n\to\pinf}\pinf.\]

Or \(\paren{v_n}\in A^\N\) donc \(A\) n'est pas bornée.
\end{corr}

\begin{exo}
Dans \(E=\R^3\), on pose \(B=\accol{\paren{x,y,z}\tq x^2+3y^2+4z^2+2xy+2xz-2yz\leq42}\) : \(B\) est-elle bornée ? Si oui, pour chacune des normes infinie, 1 et 2, donnez un rayon d'une boule centrée en \(0\) qui contient \(B\).
\end{exo}

\begin{corr}
\(E\) est de dimension finie donc les normes sur \(E\) sont équivalentes.

On a \[\begin{aligned}
\quantifs{\forall\paren{x,y,z}\in E}x^2+3y^2+4z^2+2xy+2xz-2yz&=\paren{x+y+z}^2-y^2-z^2-2yz+3y^2+4z^2-2yz \\
&=\paren{x+y+z}^2+2y^2+3z^2-4yz \\
&=\paren{x+y+z}^2+2\paren{y-z}^2+z^2.
\end{aligned}\]

Si \(\paren{x,y,z}\in B\) alors \(\paren{x+y+z}^2+2\paren{y-z}^2+z^2\leq42\).

Donc \(\begin{dcases}
z^2\leq42 \\
2\paren{y-z}^2\leq42 \\
\paren{x+y+z}^2\leq42
\end{dcases}\)

Donc \(\begin{dcases}
\abs{z}\leq7 \\
\abs{y-z}\leq5 \\
\abs{x+y+z}\leq7
\end{dcases}\)

Donc \(\begin{dcases}
\abs{z}\leq7 \\
\abs{y}=\abs{y-z+z}\leq\abs{y-z}+\abs{z}\leq12 \\
\abs{x}=\abs{x+y+z-y-z}\leq\abs{x+y+z}+\abs{y}+\abs{z}\leq26
\end{dcases}\)

Donc \(\norme{\paren{x,y,z}}_\infty\leq26\) et on a \[B\subset\bouleo[\infty]{0}{26}\qquad B\subset\bouleo[1]{0}{78}\qquad B\subset\bouleo[2]{0}{26\sqrt{3}}.\]
\end{corr}

\begin{exo}
Dans \(\M{2}[\R]\), on note \(\fami{P}\) l'ensemble des matrices de projecteurs : \(\fami{P}\) est-il borné ?
\end{exo}

\begin{corr}
\(\M{2}[\R]\) est de dimension finie donc les normes sont équivalentes.

On choisit la norme sup : \[\norme{\begin{pmatrix}x & z \\
y & t\end{pmatrix}}_\infty=\max\paren{\abs{x},\abs{y},\abs{z},\abs{t}}.\]

On a \(\fami{P}=\accol{M\in\M{2}[\R]\tq M^2=M}\).

On pose \(M=\begin{pmatrix}
x & z \\
y & t
\end{pmatrix}\) et on a \[M=M^2\ssi\begin{dcases}
x^2+yz=x \\
z\paren{x+t}=z \\
y\paren{x+t}=y \\
t^2+yz=t
\end{dcases}\]

On impose \(x+t=1\).

On a \[t-t^2=\paren{1-x}-\paren{1-x}^2=x-x^2.\]

Donc avec \(y\not=0\), on a \(M=\begin{pmatrix}
x & \dfrac{x-x^2}{y} \\
y & 1-x
\end{pmatrix}\).

On en déduit que \(\fami{P}\) contient les matrices \(M_n=\begin{pmatrix}
n & n-n^2 \\
1 & 1-n
\end{pmatrix}\) et on a \(\norme{M_n}_\infty\geq n\).

Donc \(\fami{P}\) n'est pas bornée.
\end{corr}

\begin{defi}
On dit qu'une suite \(v\) à termes dans \(E\) est bornée quand l'ensemble de ses valeurs est borné, autrement dit quand il existe \(M>0\) tel que \(\quantifs{\tpt n\in\N}\norme{v_n}\leq M\).

On dit qu'une fonction \(f\) d'un ensemble \(X\) dans \(E\) est bornée quand l'ensemble de ses valeurs prises sur \(X\) est borné, autrement dit quand il existe \(M>0\) tel que \(\quantifs{\tpt x\in X}\norme{f\paren{x}}\leq M\).
\end{defi}

\begin{exo}
Soit \(u\) une suite complexe arithmético-géométrique de raison \(a\). À quelle condition est-elle bornée ?
\end{exo}

\begin{corr}
Soit \(\paren{u_n}\in\C^\N\) telle que \(\quantifs{\forall n\in\N}u_{n+1}=au_n+b\).

Si \(a\not=1\), \(\paren{u_n}\) est de la forme \(\paren{\lambda a^n+\dfrac{b}{1-a}}_{n\in\N}\) où \(\lambda\in\C\).

Si \(a=1\), \(\paren{u_n}\) est de la forme \(\paren{\lambda+nb}_{n\in\N}\) où \(\lambda\in\C\).

On choisit \(\lambda\not=0\).

On a alors \[\paren{u_n}\text{ est bornée}\ssi\orenv{\abs{a}\leq1\text{ et }a\not=1 \\ a=1\text{ et }b=0}\]
\end{corr}

\begin{exo}
Soient \(B,B\prim\) deux boules de \(E\). Si \(\paren{x,x\prim}\in E^2\), on pose \(f\paren{x,x\prim}=d\paren{x,x\prim}\). Montrez que \(f\) est bornée sur \(B\times B\prim\).
\end{exo}

\begin{corr}
Soient \(\paren{a,b}\in E^2\) et \(r,s>0\) tels que \(B=\bouleo{a}{r}\) et \(B\prim=\bouleo{b}{s}\).

On a \[\quantifs{\forall\paren{x,x\prim}\in B\times B\prim}0\leq f\paren{x,x\prim}\leq\norme{b-a}+r+s\] car \[\begin{aligned}
f\paren{x,x\prim}&=\norme{x-x\prim} \\
&=\norme{x-a+a-b+b-x\prim} \\
&\leq\norme{x-a}+\norme{a-b}+\norme{b-x\prim} \\
&\leq r+s+\norme{b-a}.
\end{aligned}\]
\end{corr}

\section{Convergence des suites}

Dans cette section, \(E\) désigne un espace vectoriel normé par la norme \(\norme{}\).

\subsection{Définition}

\begin{defi}
Soient \(u=\paren{u_n}\) une suite à termes dans \(E\) et \(l\in E\).

On dit que la suite \(u\) converge vers \(l\) quand toute boule ouverte de centre \(l\) contient tous les termes de la suite à partir d'un certain rang : \[\quantifs{\forall\epsilon>0;\exists n_0\in\N;\forall n\geq n_0}u_n\in\bouleo{l}{\epsilon}.\]
\end{defi}

\begin{prop}
Dans la définition, on peut remplacer les boules ouvertes par des boules fermées.

On peut réécrire la définition sous deux formes équivalentes : \[\quantifs{\forall\epsilon>0;\exists n_0\in\N;\forall n\geq n_0}\norme{u_n-l}<\epsilon\] ou \[\quantifs{\forall\epsilon>0;\exists n_0\in\N;\forall n\geq n_0}\norme{u_n-l}\leq\epsilon.\]

On peut donc se ramener aux suites réelles positives : la suite vectorielle \(u\) converge vers \(l\) ssi la suite réelle \(\paren{\norme{u_n-l}}\) converge vers \(0\).
\end{prop}

Une suite qui ne converge vers aucun élément de \(E\) est dite divergente.

\subsection{Propriétés usuelles}

\begin{prop}[Unicité de la limite]
Si une suite \(u\in E^\N\) converge vers \(l\in E\), elle ne peut converger vers un autre point de \(E\).

On peut donc noter classiquement \(l=\lim u=\lim_{n\to\pinf}u_n\) ou \(u_n\tendqd{n\to\pinf}l\).
\end{prop}

\begin{dem}
Par l'absurde, soit \(\paren{l,l\prim}\in E^2\) tel que \(u\) converge vers \(l\) et \(l\prim\), et \(l\not=l\prim\).

Soit \(\epsilon=\dfrac{\norme{l-l\prim}}{2}\).

Il existe \(N_1\in\N\) et \(N_2\in\N\) tels que \[\quantifs{\forall n\geq N_1}\norme{u_n-l}<\epsilon\] et \[\quantifs{\forall n\geq N_2}\norme{u_n-l\prim}<\epsilon.\]

On pose \(N=\max\paren{N_1,N_2}\).

On a alors \[\begin{aligned}
\quantifs{\forall n\geq N}\epsilon&=\dfrac{\norme{l-l\prim}}{2} \\
&=\dfrac{\norme{l-u_n+u_n-l\prim}}{2} \\
&\leq\dfrac{\norme{l-u_n}+\norme{u_n-l\prim}}{2} \\
&<\epsilon.
\end{aligned}\]

Contradiction donc \(l=l\prim\).
\end{dem}

\begin{prop}
Si une suite \(u\in E^\N\) converge, alors elle est bornée.
\end{prop}

\begin{dem}
On pose \(\epsilon=42\).

Il existe \(N\in\N\) tel que \[\quantifs{\forall n\geq N}\norme{u_n-l}\leq42\] \ie \(\quantifs{\forall n\geq N}u_n\in\boulef{l}{42}\).

Parmi les \(N\) premiers termes de la suite \(u\), on détermine le plus lointain de \(l\) : on pose \[r=\max_{0\leq k\leq N-1}\norme{u_k-l}.\]

Puis on pose \(R=\max\paren{42,r}>0\).

On a alors \(\quantifs{\forall n\in\N}\norme{u_n-l}\leq R\).

Donc \(\quantifs{\forall n\in\N}u_n\in\boulef{l}{R}\).

Donc la suite est bornée.
\end{dem}

\begin{theo}[Opérations sur les suites convergentes]
Soient \(u,v\in E^\N\) convergeant respectivement vers \(l\) et \(m\) deux éléments de \(E\).

Alors pour tout \(\paren{a,b}\in\K^2\), la suite \(au+bv\) converge vers \(al+bm\).

Soit \(\alpha\in\K^\N\) convergeant vers \(\lambda\in\K\).

Alors la suite \(\alpha u\) converge vers \(\lambda l\).
\end{theo}

\begin{dem}[\(\alpha u\) converge vers \(\lambda l\)]
On a \[\begin{aligned}
\quantifs{\forall n\in\N}\norme{\alpha_nu_n-\lambda l}&=\norme{\alpha_nu_n-\lambda u_n+\lambda u_n-\lambda l} \\
&=\norme{\paren{\alpha_n-\lambda}u_n+\lambda\paren{u_n-l}} \\
&\leq\norme{\paren{\alpha_n-\lambda}u_n}+\norme{\lambda\paren{u_n-l}} \\
&=\abs{\alpha_n-\lambda}\norme{u_n}+\abs{\lambda}\norme{u_n-l} \\
&\leq\abs{\alpha_n-\lambda}\norme{u_n}+\paren{\abs{\lambda}+1}\norme{u_n-l}.
\end{aligned}\]

Soit \(\epsilon>0\).

Il existe \(n_0\in\N\) tel que \[\quantifs{\forall n\geq n_0}\norme{u_n-l}\leq\dfrac{\epsilon}{\abs{\lambda}+1}.\]

\(u\) converge donc est bornée : il existe \(K\in\Rps\) tel que \[\quantifs{\forall n\in\N}\norme{u_n}\leq K.\]

Donc il existe \(n_1\in\N\) tel que \[\quantifs{\forall n\geq n_1}\abs{\alpha_n-\lambda}\leq\dfrac{\epsilon}{K}.\]

Donc \[\begin{aligned}
\quantifs{\forall n\geq\max\paren{n_0,n_1}}\abs{\alpha_nu_n-\lambda l}&\leq K\dfrac{\epsilon}{K}+\paren{\abs{\lambda}+1}\dfrac{\epsilon}{\abs{\lambda}+1} \\
&=2\epsilon.
\end{aligned}\]

D'où \(\alpha_nu_n\tendqd{n\to\pinf}\lambda l\).
\end{dem}

\begin{prop}
Toute suite extraite d'une suite convergente converge vers la même limite.

Quasi-réciproque : si \(u\) est une suite telle que les deux suites extraites \(\paren{u_{2n}}\) et \(\paren{u_{2n+1}}\) convergent vers la même limite \(l\), alors \(u\) converge vers \(l\).
\end{prop}

\begin{prop}
Dans un produit de deux espaces vectoriels normés \(E\times F\), une suite \(\paren{u_n}=\paren{\paren{a_n,b_n}}\) converge ssi les suites \(\paren{a_n}\) et \(\paren{b_n}\) convergent dans \(E\), respectivement \(F\).

Dans ce cas, \(\lim\paren{a_n,b_n}=\paren{\lim a_n,\lim b_n}\).

Ce résultat se généralise sans difficulté par récurrence à un nombre quelconque (fini) d'espaces vectoriels normés.
\end{prop}

\subsection{Cas particulier en dimension finie}

Dans cette partie, on suppose que \(E\) est de dimension finie.

\begin{defi}
Soit \(\fami{B}=\paren{e_1,\dots,e_n}\) une base de \(E\).

Pour \(i\in\interventierii{1}{n}\), on appelle \(i\)-ème forme coordonnée (relative à la base \(\fami{B}\)), notée souvent \(d_i\), la forme linéaire qui à un vecteur associe sa \(i\)-ème coordonnée dans la base \(\fami{B}\) : \[\quantifs{\tpt v\in E}v=\sum_{i=1}^nd_i\paren{v}e_i.\]
\end{defi}

\begin{theo}
Soit \(\fami{B}\) une base de \(E\).

Une suite \(u\in E^\N\) converge vers \(l\in E\) ssi pour toute forme coordonnée \(d\) relative à \(\fami{B}\), la suite \(\paren{d\paren{u_n}}\) converge vers \(d\paren{l}\).

Autrement dit, une suite converge ssi ses suites-coordonnées dans n'importe quelle base convergent.

Dans ce cas, la limite de la suite \(u\) est le vecteur \(l\) tel que pour toute forme coordonnée \(d\), \(d\paren{l}=\lim_{n\to\pinf}d\paren{u_n}\).
\end{theo}

\begin{dem}
\(E\) est de dimension finie donc toutes les normes sur \(E\) sont équivalentes. On note \(p=\dim E\).

\(\fami{B}\) étant une base de \(E\), on choisit la norme \(\norme{}_\infty\) relative à \(\fami{B}\).

Il existe \(a,b>0\) tels que \(a\norme{}_\infty\leq\norme{}\leq b\norme{}_\infty\).

\impdir

Si \(\paren{u_n}\) converge vers \(l\), alors \(\norme{u_n-l}\tendqd{n\to\pinf}0\) donc d'après l'inégalité ci-dessus, on a \[0\leq\norme{u_n-l}_\infty\leq\dfrac{1}{a}\norme{u_n-l}_\infty.\]

Donc, d'après le théorème des gendarmes, on a \[\norme{u_n-l}_\infty\tendqd{n\to\pinf}0.\]

En notant \(u_n=\paren{u_{n,1},\dots,u_{n,p}}_\fami{B}\) et \(l=\paren{l_1,\dots,l_p}_\fami{B}\), on a \[\norme{u_n-l}_\infty=\max_{k\in\interventierii{1}{p}}\abs{u_{n,k}-l_k}.\]

Donc \(\quantifs{\forall k\in\interventierii{1}{p}}\abs{u_{n,k}-l_k}\leq\norme{u_n-l}_\infty\) et, par théorème des gendarmes, on a \[u_{n,k}\tendqd{n\to\pinf}l_k.\]

\imprec

Si pour tout \(k\in\interventierii{1}{p}\), \(\paren{u_{n,k}}_{n\in\N}\) converge vers \(l_k\), on a \(\norme{u_n-l}_\infty\tendqd{n\to\pinf}0\) et d'après l'inégalité précédente, on a \[\norme{u_n-l}\leq b\norme{u_n-l}_\infty\] donc \(\norme{u_n-l}\tendqd{n\to\pinf}0\) \ie \(\paren{u_n}\) converge vers \(l\).
\end{dem}

\begin{ex}~\\
Si \(M_n=\begin{pmatrix}
1 & \e{-n} \\
\nicefrac{1}{n} & n\sin\paren{\nicefrac{1}{n}}
\end{pmatrix}\), alors la suite de matrices \(\paren{M_n}\) converge vers la matrice \(I_2\).
\end{ex}

\begin{cor}
Si \(E\) est de dimension finie, la convergence d'une suite ne dépend pas du choix de la norme. On peut donc choisir la norme qu'on veut.
\end{cor}

\subsection{Point adhérent à une partie}

\begin{defi}
Soient \(A\) une partie de \(E\) et \(x\in E\).

On dit que \(x\) est un point adhérent à \(A\) quand il existe une suite \(u\in A^\N\) qui converge vers \(x\).

L'adhérence de \(A\) est l'ensemble de ses points adhérents, noté \(\conj{A}\).
\end{defi}

Intuitivement, l'adhérence d'une partie est elle-même à laquelle on ajoute tous les points qui se trouvent sur son bord.

\begin{rem}
On remarque qu'on a \(A\subset\conj{A}\) car pour tout \(a\in A\), la suite constante égale à \(a\) converge vers \(a\) et est à termes dans \(A\).
\end{rem}

\begin{exo}
Quelle est l'adhérence d'une boule ouverte ?
\end{exo}

\begin{corr}
Soient \(a\in E\) et \(r>0\). Montrons que \(\conj{\bouleo{a}{r}}=\boulef{a}{r}\).

\incdir

Soit \(b\in\conj{\bouleo{a}{r}}\).

Il existe \(\paren{u_n}\in\bouleo{a}{r}^\N\) qui converge vers \(b\).

D'après la deuxième inégalité triangulaire, on a \[\begin{aligned}
\abs{\norme{u_n-a}-\norme{b-a}}&\leq\norme{\paren{u_n-a}-\paren{b-a}} \\
&=\norme{u_n-b}.
\end{aligned}\]

Or \(\norme{u_n-b}\tendqd{n\to\pinf}0\) donc d'après le théorème d'encadrement, on a \[\norme{u_n-a}\tendqd{n\to\pinf}\norme{b-a}.\]

Or \(\quantifs{\forall n\in\N}u_n\in\bouleo{a}{r}\) donc \(\norme{u_n-a}<r\).

Par passage à la limite, \(\norme{b-a}\leq r\).

Donc \(b\in\boulef{a}{r}\).

\increc

Soit \(b\in\boulef{a}{r}\).

Alors \(\norme{b-a}\leq r\).

Si \(\norme{b-a}<r\), alors \(b\in\bouleo{a}{r}\subset\conj{\bouleo{a}{r}}\).

Si \(\norme{b-a}=r\), on pose, pour \(n\in\Ns\), \(u_n=\dfrac{1}{n}a+\paren{1-\dfrac{1}{n}}b\).

La suite \(\paren{u_n}\) converge vers \(b\) par opérations sur les limites et on a \[\begin{aligned}
\quantifs{\forall n\in\Ns}\norme{u_n-a}&=\norme{\paren{\dfrac{1}{n}-1}a+\paren{1-\dfrac{1}{n}}b} \\
&=\abs{1-\dfrac{1}{n}}\norme{b-a} \\
&=\dfrac{n-1}{n}r \\
&<r
\end{aligned}\] donc \(\paren{u_n}\in\bouleo{a}{r}^\N\).

On a ainsi trouvé une suite à termes dans \(\bouleo{a}{r}\) qui converge vers \(b\) donc \(b\in\conj{\bouleo{a}{r}}\).
\end{corr}

\begin{rem}
On a montré au passage que si \(u_n\tendqd{n\to\pinf}l\) alors \(\quantifs{\tpt a\in E}\norme{u_n-a}\tendqd{n\to\pinf}\norme{l-a}\).
\end{rem}

\begin{exo}
Quelle est l'adhérence de \(\Z\) dans \(\R\) ?
\end{exo}

\begin{corr}
\(\R\) est muni de la norme \(\abs{\cdot}\).

Montrons que \(\Z=\conj{\Z}\).

\incdir Trivial.

\increc

Soit \(b\in\conj{\Z}\).

Il existe une suite \(\paren{u_n}\in\Z^\N\) qui converge vers \(b\).

On pose \(\epsilon=\dfrac{1}{2}\).

Il existe \(n_0\in\N\) tel que \[\quantifs{\forall n\geq n_0}\abs{u_n-b}<\dfrac{1}{2}\] \ie \(u_n\in\intervee{b-\dfrac{1}{2}}{b+\dfrac{1}{2}}\).

Or l'intervalle \(\intervee{b-\dfrac{1}{2}}{b+\dfrac{1}{2}}\) est de longueur \(1\) et est ouvert donc il contient au plus un entier.

Or il en contient un donc il en contient un et un seul.

On en déduit que \(\paren{u_n}\) est stationnaire en cet entier à partir de \(n_0\) et donc \(\lim u=b\) est un entier.

Donc \(b\in\Z\).
\end{corr}

\begin{prop}
Soient \(A\) une partie de \(E\) et \(x\in E\).

Alors \(x\) est adhérent à \(A\) ssi toute boule centrée en \(x\) rencontre \(A\).

De manière formalisée : \(x\in\conj{A}\ssi\quantifs{\forall r>0;\exists y\in A}y\in\bouleo{x}{r}\).
\end{prop}


\begin{dem}
\imprec

On a \(\quantifs{\tpt n\in\Ns}\bouleo{x}{\dfrac{1}{n}}\inter A\not=\ensvide\).

On peut choisir \(y_n\in\bouleo{x}{\dfrac{1}{n}}\inter A\).

On construit ainsi une suite \(\paren{y_n}\in A^{\Ns}\) telle que \[\quantifs{\forall n\in\Ns}\norme{x-y_n}<\dfrac{1}{n}.\]

D'après le théorème d'encadrement, on a donc \(y_n\tendqd{n\to\pinf}x\).

Donc \(x\in\conj{A}\).

\impdir

Soient \(x\in\conj{A}\) et \(\paren{u_n}\in A^\N\) telle que \(u_n\tendqd{n\to\pinf}x\).

Soit \(r>0\).

Il existe \(n_0\in\N\) tel que \[\quantifs{\forall n\geq n_0}\norme{u_n-x}<r\] donc \(u_{n_0}\in A\inter\bouleo{x}{r}\) donc \(A\inter\bouleo{x}{r}\not=\ensvide\).
\end{dem}

On peut donner la définition de la densité d'une partie.

\begin{defi}
On dit qu'une partie \(A\) est dense dans \(E\) quand \(\conj{A}=E\), \cad qu'on peut trouver des éléments de \(A\) aussi proches de n'importe quel point.
\end{defi}

\begin{ex}
\begin{itemize}
    \item Dans \(\R\), \(\Q\) et \(\R\excluant\Q\) sont denses (\cf cours de première année). \\
    \item \(\GL{n}\) est dense dans \(\M{n}\) (démonstration ultérieure).
\end{itemize}
\end{ex}

\section{Limites de fonctions}

Dans cette section, \(E\) et \(F\) sont deux espaces vectoriels normés par les normes \(\norme{}_E\) et \(\norme{}_F\).

\subsection{Définition}

\begin{defi}
Soient \(f\) une fonction de \(E\) dans \(F\), \(D\) son ensemble de définition, \(a\in\conj{D}\) et \(l\in F\).

On dit que \(f\) a pour limite \(l\) en \(a\) quand \[\quantifs{\forall\epsilon>0;\exists\eta>0;\forall x\in D}\norme{x-a}_E<\eta\imp\norme{f\paren{x}-l}_F<\epsilon.\]
\end{defi}

\begin{rem}
On peut remplacer les inégalités strictes sur les normes par des inégalités larges.

On peut réécrire la définition à l'aide de boules ouvertes (ou fermées) : \[\quantifs{\forall\epsilon>0;\exists\eta>0;\forall x\in D\inter\bouleo{a}{\eta}}f\paren{x}\in\bouleo{l}{\epsilon}.\]

Si \(E\) et \(F\) sont de dimension finie, cette définition ne dépend pas du choix des normes.
\end{rem}

\subsection{Caractérisation séquentielle de la limite}

\begin{theo}
Soient \(f\) une fonction de \(E\) dans \(F\), \(D\) son ensemble de défintion, \(a\in\conj{D}\) et \(l\in F\).

\(f\) a pour limite \(l\) en \(a\) ssi pour toute suite \(u\) à termes dans \(D\) convergeant vers \(a\), la suite \(f\rond u=\paren{f\paren{u_n}}\) converge vers \(l\).
\end{theo}

\begin{dem}
\impdir

Suposons que \(f\) a pour limite \(l\) en \(a\).

Soit \(u\in D^\N\) qui converge vers \(a\).

Pour tout \(\epsilon>0\), il existe \(\eta>0\) tel que \[\quantifs{\forall x\in D}\norme{x-a}_E<\eta\imp\norme{f\paren{x}-l}_F<\epsilon.\]

On a \(\eta>0\) et \(u_n\tendqd{n\to\pinf}a\) donc il existe \(N\in\N\) tel que \[\quantifs{\forall n\geq N}\norme{u_n-a}_E<\eta.\]

Donc pour tout \(n\geq N\), comme \(u_n\in D\) et \(\norme{u_n-a}_E<\eta\), on a \[\norme{f\paren{u_n}-l}_F<\epsilon.\]

Donc \(\paren{f\paren{u_n}}\) converge vers \(l\).

\imprec

Par contraposée, montrons que si \(f\) n'a pas pour limite \(l\) en \(a\) alors il existe \(u\in D^\N\) telle que \(u_n\tendqd{n\to\pinf}a\) et \(f\paren{u_n}\ntendqd{n\to\pinf}l\).

Si \(f\) n'a pas pour limite \(l\) en \(a\), alors il existe \(\epsilon>0\) tel que pour tout \(\eta>0\), il existe \(x\in D\) tel que \(\norme{x-a}_E<\eta\) et \(\norme{f\paren{x}-l}_F\geq\epsilon\).

Donc pour tout \(n\in\N\), il existe \(u_n\in D\) tel que \(\begin{dcases}
\norme{u_n-a}_E<\dfrac{1}{n+1} \\
\norme{f\paren{u_n}-l}_F\geq\epsilon
\end{dcases}\).

On construit ainsi une suite \(\paren{u_n}\in D^\N\) telle que \[\quantifs{\forall n\in\N}\begin{dcases}
\norme{u_n-a}_E<\dfrac{1}{n+1} \\
\norme{f\paren{u_n}-l}_F\geq\epsilon
\end{dcases}\]

Par encadrement, \(\paren{u_n}\) converge vers \(a\) mais \(\paren{f\paren{u_n}}\) ne converge pas vers \(l\).
\end{dem}

En pratique, on utilise beaucoup plus souvent le sens direct de l'équivalence précédente.

\subsection{Propriétés usuelles}

\begin{prop}[Unicité de la limite]
Soient \(f\) une fonction de \(E\) dans \(F\), \(D\) son ensemble de définition, \(a\in\conj{D}\) et \(l\in F\).

Si \(f\) a pour limite \(l\) en \(a\), alors elle ne peut avoir d'autre limite que \(l\) en \(a\).

On peut donc noter classiquement \(l=\lim_af=\lim_{x\to a}f\paren{x}\) ou \(f\paren{x}\tendqd{x\to a}l\).
\end{prop}

\begin{prop}
Si \(f\) a pour limite \(l\) en \(a\), alors elle est bornée au voisinage de \(a\).
\end{prop}

\begin{theo}[Opérations sur les limites]
Soient \(f\) et \(g\) deux fonctions de \(E\) dans \(F\), définies sur la même partie \(D\) et ayant respectivement pour limites \(l\) et \(m\) deux éléments de \(F\) en \(a\in\conj{D}\).

Alors pour tout \(\paren{\lambda,\mu}\in\K^2\), la fonction \(\lambda f+\mu g\) a pour limite \(\lambda l+\mu m\) en \(a\).

Soient \(\alpha\) une fonction de \(E\) dans \(\K\) et \(f\) une fonction définie de \(E\) dans \(F\), définies sur la même partie \(D\) et ayant respectivement pour limites \(\beta\in\K\) et \(l\in F\) en \(a\in\conj{D}\).

Alors \(\alpha f\) a pour limite \(\beta l\) en \(a\).
\end{theo}

\begin{prop}
Une fonction \(f=\paren{g,h}\) à valeurs dans un produit d'espaces vectoriels normés a une limite ssi \(g\) et \(h\) ont chacune une limite.

Dans ce cas, \(\lim_af=\paren{\lim_ag,\lim_ah}\).

Ce résultat se généralise sans difficulté par récurrence à un nombre quelconque (fini) d'espaces vectoriels normés.
\end{prop}

\subsection{Cas particulier de la dimension finie}

\begin{theo}
On suppose que \(F\) est de dimension finie. Soit \(\fami{B}\) une base de \(F\).

Soit \(f\) une fonction de \(E\) dans \(F\), \(D\) son ensemble de définition, \(a\in\conj{D}\) et \(l\in F\).

La fonction \(f\) a pour limite \(l\) en \(a\) ssi pour toute forme coordonnée \(d\) relative à \(\fami{B}\), la fonction \(d\rond f\) a pour limite \(d\paren{l}\) en \(a\).

Autrement dit, une fonction a une limite en \(a\) ssi ses fonctions-coordonnées dans n'importe quelle base ont chacune une limite en \(a\).

Dans ce cas, la limite de la fonction \(f\) en \(a\) est le vecteur \(l\) tel que pour tout forme coordonnée \(d\), \(d\paren{l}=\lim_{x\to a}d\paren{f\paren{x}}\).
\end{theo}

\subsection{Composition des limites}

\(G\) désigne un troisième espace vectoriel normé.

\begin{theo}
Soient \(f\) une fonction de \(E\) dans \(F\) et \(D_f\) son ensemble de définition. Soient \(g\) une fonction de \(F\) dans \(G\) et \(D_g\) son ensemble de définition. On suppose que \(f\paren{D_f}\subset D_g\) (condition qui permet de définir la composée \(g\rond f\) sur \(D_f\)).

Soient \(a\in\conj{D_f}\), \(b\in\conj{D_g}\) et \(l\in G\).

Si \(f\) a pour limite \(b\) en \(a\) et \(g\) a pour limite \(l\) en \(b\), alors \(g\rond f\) a pour limite \(l\) en \(a\).

Autrement dit, si \(\begin{dcases}
f\paren{x}\tendqd{x\to a}b \\
g\paren{y}\tendqd{y\to b}l
\end{dcases}\) alors \(g\rond f\paren{x}\tendqd{x\to a}l\).
\end{theo}

\subsection{Extensions des définitions}

D'abord les limites infinies en un point dans le cas où l'espace d'arrivée est \(\R\).

\begin{defi}
Soient \(f\) une fonction de \(E\) dans \(\R\), \(D\) son ensemble de définition et \(a\in\conj{D}\).

On dit que \(f\) a pour limite \(\pinf\) en \(a\) quand \[\quantifs{\forall M>0;\exists\eta>0;\forall x\in D}\norme{x-a}_E\leq\eta\imp f\paren{x}\geq M.\]

On dit que \(f\) a pour limite \(\minf\) en \(a\) quand \[\quantifs{\forall M<0;\exists\eta>0;\forall x\in D}\norme{x-a}_E\leq\eta\imp f\paren{x}\leq M.\]
\end{defi}

Puis les limites en \guillemets{l'infini}.

\begin{defi}
Soient \(f\) une application de \(E\) dans \(F\) et \(l\in F\).

On dit que \(f\) a pour limite \(l\) quand \(\norme{x}\) tend vers l'infini quand \[\quantifs{\forall\epsilon>0;\exists B>0;\forall x\in E}\norme{x}_E\geq B\imp\norme{f\paren{x}-l}_F\leq\epsilon.\]

Dans le cas où \(F=\R\), on dit que \(f\paren{x}\) a pour limite \(\pinf\) quand \(\norme{x}\) tend vers l'infini quand \[\quantifs{\forall M>0;\exists B>0;\forall x\in E}\norme{x}_E\geq B\imp f\paren{x}\geq M.\]

(Définition semblable pour la limite \(\minf\)).
\end{defi}

Enfin, dans le cas où l'espace de départ est \(\R\), on peut parler de limite en l'infini au sens habituel.

\begin{defi}
Soient \(f\) une fonction de \(\R\) dans \(F\), définie sur un ouvert \(\intervee{?}{\pinf}\) et \(l\in F\).

On dit que \(f\paren{x}\) a pour limite \(l\) quand \(x\) tend vers \(\pinf\) quand \[\quantifs{\forall\epsilon>0;\exists B>0;\forall x\geq B}\norme{f\paren{x}-l}\leq\epsilon.\]

(Définition semblable pour la limite \(x\) tend vers \(\minf\)).
\end{defi}

\section{Fonctions continues}

Dans cette section, \(E\) et \(F\) sont des espaces vectoriels normés par les normes \(\norme{}_E\) et \(\norme{}_F\).

\subsection{Continuité en un point}

\begin{prop}
Soient \(f\) une fonction de \(E\) dans \(F\), \(D\) son ensemble de définition, \(a\in\conj{D}\) et \(l\in F\).

Si \(f\) a pour limite \(l\) en \(a\) et si \(a\in D\), alors \(l=f\paren{a}\).
\end{prop}

Dans ce cas, on dit que la fonction \(f\) est continue en \(a\).

\begin{defi}
Soient \(f\) une fonction de \(E\) dans \(F\), \(D\) son ensemble de définition et \(a\in D\).

On dit que \(f\) est continue en \(a\) quand \(f\) a pour limite \(f\paren{a}\) en \(a\).
\end{defi}

On déduit de cette définition et des théorèmes précédents

\begin{itemize}
    \item la caractérisation séquentielle de la continuité en un point ; \\
    \item le fait qu'une fonction continue en un point est bornée au voisinage de ce point ; \\
    \item les théorèmes d'opérations et de compositions des fonctions continues en un point ; \\
    \item l'équivalence entre la continuité d'une fonction et celle de ses fonctions-coordonnées dans une certaine base de \(F\) dans le cas où \(F\) est de dimension finie.
\end{itemize}

\subsection{Continuité sur une partie}

\begin{defi}
Soient \(f\) une fonction de \(E\) dans \(F\), \(D\) son ensemble de définition et \(A\subset D\).

On dit que \(f\) est continue sur \(A\) quand \(f\) est continue en tout point de \(A\).
\end{defi}

On déduit de cette définition et des théorèmes précédents

\begin{itemize}
    \item les théorèmes d'opérations et de compositions des fonctions continues sur une partie ; \\
    \item l'équivalence entre la continuité d'une fonction et celle de ses fonctions-coordonnées dans une certaine base de \(F\) dans le cas où \(F\) est de dimension finie.
\end{itemize}

\begin{prop}
Soient \(f\) et \(g\) deux fonctions de \(E\) dans \(F\) définies sur \(D\) et \(A\subset D\).

Si \(A\) est dense dans \(D\), \(f\) et \(g\) sont continues sur \(D\) et \(f=g\) sur \(A\), alors \(f=g\) sur \(D\).
\end{prop}

\begin{dem}
On suppose que \(A\) est dense dans \(D\) et que \(f\) et \(g\) sont continues.

Soit \(x\in D\).

Il existe \(\paren{u_n}\in A^\N\) telle que \(u_n\tendqd{n\to\pinf}x\).

\(f\) est continue en \(x\) donc d'après la caractérisation séquentielle de la continuité, on a \[f\paren{u_n}\tendqd{n\to\pinf}f\paren{x}.\]

De même, on a \[g\paren{u_n}\tendqd{n\to\pinf}g\paren{x}.\]

Or \(f=g\) sur \(A\) donc \(\paren{f\paren{u_n}}=\paren{g\paren{u_n}}\) donc \(f\paren{x}=g\paren{x}\) par unicité de la limite.
\end{dem}

\subsection{Cas particulier de la dimension finie}

On suppose que \(E\) et \(F\) sont de dimensions finies.

Dans une base donnée, les formes coordonnées relatives à cette base sont en particulier des applications continues.

Donc toute fonction \(f\) de \(E\) dans \(F\) dont les fonctions-coordonnées \(\paren{f_1,\dots,f_n}\) dans une base de \(F\) sont définies polynomialement à partir des formes coordonnées dans une base de \(E\) est continue.

\begin{ex}
\begin{itemize}
    \item La fonction \(f:\R^2\to\R^2\) telle que \(f\paren{x,y}=\paren{x^2+y^2,xy-\paren{1+x}^3}\) est continue sur \(\R^2\). \\
    \item Les applications trace et déterminant définies sur \(\M{n}\) sont continues.
\end{itemize}
\end{ex}

\begin{exo}
Montrez que l'application \(A\mapsto A^2\) est continue de \(\M{n}\) dans lui-même.
\end{exo}

\begin{corr}~\\
On a \(A^2=\paren{\sum_{k=1}^na_{i,k}a_{k,j}}_{\paren{i,j}\in\interventierii{1}{n}^2}\) donc \(A\mapsto A^2\) est continue.
\end{corr}

\begin{exo}
En admettant (momentanément) que \(\GL{n}\) est un ouvert, montrez que l'application \(A\mapsto A\inv\) est continue de \(\GL{n}\) dans lui-même.
\end{exo}

\begin{corr}
Si \(A\in\GL{n}\), alors \(A\inv=\dfrac{1}{\det A}\trans{\paren{\Com{A}}}\).

Les coefficients de \(\trans{\paren{\Com{A}}}\) sont des déterminants calculés à partir des coefficients de \(A\) donc dépendent polynomialement de ces coefficients, donc \(A\mapsto\trans{\paren{\Com{A}}}\) est continue.

Donc \(A\mapsto A\inv\) est le produit de deux fonctions continues et est donc continue.
\end{corr}

\subsection{Fonctions lipschitziennes}

\begin{defi}
Soient \(f\) une application de \(E\) dans \(F\), \(A\) une partie de \(E\) et \(K\in\Rp\).

On dit que \(f\) est \(K\)-lipschitzienne sur \(A\) (ou lipschitzienne de rapport \(K\)) quand \[\quantifs{\forall\paren{x,y}\in A^2}\norme{f\paren{y}-f\paren{x}}_F\leq K\norme{y-x}_E.\]

On dit que \(f\) est lipschitzienne sur \(A\) quand il existe \(K\in\Rp\) tel que \(f\) soit \(K\)-lipschitzienne sur \(A\).
\end{defi}

\begin{rem}
Si \(f\) est \(K\)-lipschitzienne sur \(A\), alors le rapport \(K\) n'est pas unique, puisque pour tout \(L\geq K\), on a encore \(f\) \(L\)-lipschitzienne sur \(A\).
\end{rem}

\begin{prop}
Toute fonction lipschitzienne est continue.
\end{prop}

Mais la réciproque est fausse (contre-exemple : la fonction \(\sqrt{\cdot}\) sur \(\intervie{0}{\pinf}\)).

Un exemple fondamental : la fonction \(x\mapsto d\paren{x,A}\).

\begin{defi}
Soit \(A\) une partie de \(E\).

Pour \(x\in E\), on appelle distance de \(x\) à \(A\) le réel \(\inf_{a\in A}d\paren{x,a}\).
\end{defi}

\begin{prop}
Pour toute partie \(A\) de \(E\), la fonction \(x\mapsto d\paren{x,A}\) est \(1\)-lipschitzienne.

L'adhérence de \(A\) est l'ensemble des points à distance nulle de \(A\), \ie tels que \(d\paren{x,A}=0\).
\end{prop}

\begin{dem}[\(1\)-lipschitziannité de la fonction]
On veut montrer \[\quantifs{\forall\paren{x,y}\in E^2}\abs{d\paren{x,A}-d\paren{y,A}}\leq\norme{x-y}.\]

On montre d'abord \[\quantifs{\forall\paren{x,y}\in E^2}d\paren{x,A}-d\paren{y,A}\leq\norme{x-y}\] ce qui est équivalent à \[\quantifs{\forall\paren{x,y}\in E^2}d\paren{x,A}\leq d\paren{y,A}+\norme{x-y}.\]

On a \(d\paren{x,A}=\inf_{a\in A}\norme{x-a}\) donc pour tout \(a\in A\), on a \[d\paren{x,A}\leq\norme{x-a}.\]

Or \(\norme{x-a}=\norme{x-y+y-a}\leq\norme{x-y}+\norme{y-a}\) donc \[d\paren{x,A}\leq\norme{x-y}+\norme{y-a}.\]

De plus, on a \(d\paren{y,A}=\inf_{a\in A}\norme{y-a}\) donc il existe \(\paren{u_n}\in A^\N\) telle que \[\norme{y-u_n}\tendqd{n\to\pinf}d\paren{y,A}.\]

Alors \(\quantifs{\tpt n\in\N}d\paren{x,A}\leq\norme{x-y}+\norme{y-u_n}\), donc par passage à la limite quand \(n\to\pinf\) : \[d\paren{x,A}\leq\norme{x-y}+d\paren{y,A}\] donc \(d\paren{x,A}-d\paren{y,A}\leq\norme{x-y}\).

En échangeant \(x\) et \(y\), on obtient \[\quantifs{\forall\paren{x,y}\in E^2}d\paren{y,A}-d\paren{x,A}\leq\norme{y-x}=\norme{x-y}.\]

D'où \[\quantifs{\forall\paren{x,y}\in E^2}\abs{d\paren{x,A}-d\paren{y,A}}\leq\norme{x-y}.\]
\end{dem}

\begin{dem}[\(x\in\conj{A}\ssi d\paren{x,A}=0\)]
\impdir

Si \(x\in\conj{A}\) alors il existe une suite \(\paren{u_n}\in A^\N\) qui converge vers \(x\).

Or \(x\mapsto d\paren{x,A}\) est lipschitzienne donc continue, donc \[d\paren{u_n,A}\tendqd{n\to\pinf}d\paren{x,A}.\]

Or \(\quantifs{\tpt n\in\N}d\paren{u_n,A}=0\) car \(\paren{u_n}\in A^\N\), donc \[d\paren{x,A}=0.\]

\imprec

Si \(d\paren{x,A}=0\), alors \(\inf_{a\in A}\norme{x-a}=0\).

Donc il existe \(\paren{a_n}\in A^\N\) telle que \(\norme{x-a_n}\tendqd{n\to\pinf}0\) \ie \(\paren{a_n}\) converge vers \(x\) \ie \(x\in\conj{A}\).
\end{dem}

\subsection{Continuité des applications linéaires et \(n\)-linéaires}

\begin{prop}\thlabel{prop1.22}
Soit \(f\in\L{E}{F}\).

Il y a équivalence entre les propositions suivantes :

\begin{enumerate}
    \item \(f\) est continue en \(0\) ; \\
    \item \(f\) est continue en un point \(x\) ; \\
    \item \(f\) est continue sur \(E\) ; \\
    \item \(f\) est lipschitzienne sur \(E\) ; \\
    \item il existe \(K\geq 0\) tel que \(\quantifs{\tpt x\in E}\norme{f\paren{x}}_F\leq K\norme{x}_E\) ; \\
    \item \(f\) est bornée sur la boule-unité ; \\
    \item \(f\) est bornée sur une boule.
\end{enumerate}
\end{prop}

\begin{dem}
On a clairement \(\paren{4}\imp\paren{3}\), \(\paren{3}\imp\paren{2}\) et \(\paren{3}\imp\paren{1}\).
\end{dem}

\begin{dem}[\(\paren{1}\imp\paren{2}\)]
Soit \(x\in E\).

\(f\) est continue en \(0\) donc \(\lim_{h\to0}f\paren{h}=f\paren{0}=0\).

Donc \(f\paren{x+h}=f\paren{x}+f\paren{h}\tendqd{h\to0}f\paren{x}\).

Donc \(f\) est continue en \(x\).
\end{dem}

\begin{dem}[\(\paren{2}\imp\paren{3}\)]
Soit \(y\in E\).

\(f\) est continue en \(x\) donc \(\lim_{t\to x}f\paren{t}=f\paren{x}\).

Donc \[\begin{aligned}
f\paren{t}&=f\paren{t-y+x-x+y} \\
&=f\paren{t-y+x}-f\paren{x}+f\paren{y} \\
&\tendqd{t\to y}f\paren{y}.
\end{aligned}\]

Donc \(f\) est continue en \(y\).

Donc \(f\) est continue sur \(E\).
\end{dem}

\begin{dem}[\(\paren{3}\imp\paren{7}\)]
Soit \(x\in E\).

\(f\) est continue en \(x\) donc \(f\) est bornée au voisinage de \(x\) \ie il existe \(r>0\) tel que \(f\) soit bornée sur \(\bouleo{x}{r}\).
\end{dem}

\begin{dem}[\(\paren{7}\imp\paren{6}\)]
\(f\) est bornée sur \(\bouleo{x}{r}\) : il existe \(M>0\) tel que \[\quantifs{\forall t\in\bouleo{x}{r}}\norme{f\paren{t}}\leq M.\]

Pour tout \(y\in\bouleo{0}{r}\), on a \(x+y\in\bouleo{x}{r}\).

Donc \(\norme{f\paren{x+y}}\leq M\).

Donc \(\norme{f\paren{x}+f\paren{y}}\leq M\) donc \(\norme{f\paren{y}}-\norme{f\paren{x}}\leq\norme{f\paren{y}+f\paren{x}}\leq M\).

Donc \(\norme{f\paren{y}}\leq M+\norme{f\paren{x}}=M\prim\).

Pour tout \(z\in\bouleo{0}{1}\), on a \(rz\in\bouleo{0}{r}\).

Donc d'après ce qui précède : \[\begin{aligned}
\norme{f\paren{rz}}&\leq M\prim \\
\norme{rf\paren{z}}&\leq M\prim \\
r\norme{f\paren{z}}&\leq M\prim \\
\norme{f\paren{z}}&\leq\dfrac{M\prim}{r}.
\end{aligned}\]

Donc \(f\) est bornée sur \(\bouleo{0}{1}\).
\end{dem}

\begin{dem}[\(\paren{6}\imp\paren{5}\)]~\\
\(\quantifs{\Tpt x\in E\excluant\accol{0}}\norme{\dfrac{x}{\norme{x}}}=1\).

Donc \(\dfrac{x}{2\norme{x}}\in\bouleo{0}{1}\).

\(f\) est bornée sur \(\bouleo{0}{1}\) : il existe \(M>0\) tel que \[\quantifs{\forall t\in\bouleo{0}{1}}\norme{f\paren{t}}\leq M.\]

Donc \(\quantifs{\tpt x\not=0}\norme{f\paren{\dfrac{x}{2\norme{x}}}}\leq M\).

Donc \(\norme{\dfrac{1}{2\norme{x}}f\paren{x}}\leq M\).

Donc \(\dfrac{1}{2\norme{x}}\norme{f\paren{x}}\leq M\).

Donc \(\norme{f\paren{x}}\leq2M\norme{x}\).

Ceci est vrai aussi pour \(x=0\) : \(\norme{f\paren{0}}=\norme{0}=0\leq2M\norme{0}\).
\end{dem}

\begin{dem}[\(\paren{5}\imp\paren{4}\)]
On a \[\quantifs{\forall\paren{x,y}\in E^2}\norme{f\paren{x}-f\paren{y}}=\norme{f\paren{x-y}}\leq K\norme{x-y}.\]

On a donc montré toutes les équivalences :

\begin{center}
\begin{tikzpicture}
\node[shape=circle,draw=black] (1) at (0,0) {\(\paren{1}\)};
\node[shape=circle,draw=black] (2) at (-2,-2) {\(\paren{2}\)};
\node[shape=circle,draw=black] (3) at (2,-2) {\(\paren{3}\)};
\node[shape=circle,draw=black] (4) at (0,-4) {\(\paren{4}\)};
\node[shape=circle,draw=black] (5) at (-4,-4) {\(\paren{5}\)};
\node[shape=circle,draw=black] (6) at (-2,-6) {\(\paren{6}\)};
\node[shape=circle,draw=black] (7) at (2,-6) {\(\paren{7}\)};
\path[->] (1) edge (2);
\path[->] (2) edge (3);
\path[->] (3) edge (1);
\path[->] (3) edge (2);
\path[->] (3) edge (7);
\path[->] (7) edge (6);
\path[->] (6) edge (5);
\path[->] (5) edge (4);
\path[->] (4) edge (3);
\end{tikzpicture}
\end{center}
\end{dem}

\begin{exo}\thlabel{exo1.18}
On pose \(E=\ensclasse{0}{\intervii{0}{1}}{\R}\) muni de la norme infinie.

L'application \(f\mapsto\int_0^1f\paren{t}\odif{t}\) est-elle continue sur \(E\) ?
\end{exo}

\begin{corr}
Pour \(f\in E\), on a \(\norme{f}_\infty=\sup_{\intervii{0}{1}}\abs{f}\in\R\) car \(\abs{f}\) est continue sur le segment \(\intervii{0}{1}\).

On note \(I:f\mapsto\int_0^1f\paren{t}\odif{t}\). \(I\) est linéaire.

Pour \(f\in E\), on a \(\abs{I\paren{f}}\leq\int_0^1\abs{f}\).

Or \(\quantifs{\forall t\in\intervii{0}{1}}\abs{f\paren{t}}\leq\norme{f}_\infty\).

Donc \[\abs{I\paren{f}}\leq\int_0^1\abs{f}\leq\int_0^1\norme{f}_\infty\odif{t}=\norme{f}_\infty.\]

Donc d'après la \thref{prop1.22}, \(I\) est continue sur \(E\).
\end{corr}

\begin{exo}
\(E\) désigne le même espace et on pose \(\norme{f}_1=\int_0^1\abs{f\paren{t}}\odif{t}\).

Montrez que \(\norme{}_1\) est une norme sur \(E\).

L'application \(f\mapsto f\paren{1}\) est-elle continue sur \(E\) ?
\end{exo}

\begin{corr}[\(\norme{}_1\) est une norme sur \(E\)]~\\
\begin{itemize}
    \item Soit \(f\in E\). Si \(\norme{f}_1=0\) alors \(\int_0^1\abs{f}=0\). Or \(\abs{f}\) est continue et positive donc d'après le théorème de stricte positivité de l'intégrale, \(\abs{f}=0\) donc \(f=0\). \\
    \item Soient \(f\in E\) et \(\lambda\in\R\). On a \[\norme{\lambda f}_1=\int_0^1\abs{\lambda f}=\int_0^1\abs{\lambda}\abs{f}=\abs{\lambda}\int_0^1\abs{f}=\abs{\lambda}\norme{f}_1.\]
    \item Soit \(\paren{f,g}\in E^2\). On a \[\norme{f+g}_1=\int_0^1\abs{f+g}\leq\int_0^1\paren{\abs{f}+\abs{g}}=\int_0^1\abs{f}+\int_0^1\abs{g}=\norme{f}_1+\norme{g}_1.\]
    \item Donc \(\norme{}_1\) est une norme sur \(E\).
\end{itemize}
\end{corr}

\begin{corr}[Continuité de l'application ?]
On pose \(\fonction{V}{E}{\R}{f}{f\paren{1}}\)

Pour \(n\in\Ns\), on pose \(f_n:x\mapsto\begin{dcases}
0 &\text{si }x\in\intervii{0}{1-\dfrac{1}{n}} \\
n^2\paren{x-\paren{1-\dfrac{1}{n}}} &\text{sinon}
\end{dcases}\)

Pour \(n\in\Ns\), on a \(\int_0^1\abs{f}=\int_0^1f_n=\dfrac{1}{2}\) donc \(f_n\in\bouleo{0}{1}\).

Or \(\abs{V\paren{f_n}}=\abs{f_n\paren{1}}=n\tendqd{n\to\pinf}\pinf\).

On a ainsi trouvé une suite \(\paren{f_n}\) à termes dans \(\bouleo{0}{1}\) telle que \(V\paren{f_n}\tendqd{n\to\pinf}\pinf\).

Donc \(V\) n'est pas bornée sur \(\bouleo{0}{1}\).

Donc comme \(V\) est linéaire, \(V\) n'est pas continue sur \(\groupe{E}[\norme{}_1]\).

Remarque : on a \(\quantifs{\forall f\in E}\abs{V\paren{f}}=\abs{f\paren{1}}\leq\norme{f}_\infty\) donc \(V\) est continue sur \(\groupe{E}[\norme{}_\infty]\).
\end{corr}

\begin{defi}
On note \(\Lc{E}{F}\) l'ensemble des applications linéaires continues de \(E\) dans \(F\).
\end{defi}

\begin{prop}
\(\Lc{E}{F}\) est un sous-espace vectoriel de \(\L{E}{F}\), en général distinct de \(\L{E}{F}\).
\end{prop}

Cas particulier en dimension finie.

\begin{theo}
On suppose que \(E\) est de dimension finie.

Toute application linéaire de \(E\) dans \(F\) est lipschitzienne sur \(E\), donc continue.

Autrement dit, si \(E\) est de dimension finie, alors \(\Lc{E}{F}=\L{E}{F}\).
\end{theo}

\begin{dem}
On note \(p=\dim E\) et \(\fami{B}=\paren{e_1,\dots,e_p}\) une base de \(E\).

Pour \(x=\paren{x_1,\dots,x_p}_\fami{B}\), on a \(\norme{x}_\infty=\sup_{1\leq i\leq p}\abs{x_i}=\max_{1\leq i\leq p}\abs{x_i}\).

Soit \(f\in\L{E}{F}\) et \(N\) une norme sur \(F\).

Pour \(x=\paren{x_1,\dots,x_p}_\fami{B}\), on a \(x=\sum_{k=1}^px_ke_k\).

Donc, \(f\) étant linéaire, on a \(f\paren{x}=\sum_{k=1}^px_kf\paren{e_k}\).

Donc \[\begin{aligned}
N\paren{f\paren{x}}&=N\paren{\sum_{k=1}^px_kf\paren{e_k}} \\
&\leq\sum_{k=1}^pN\paren{x_kf\paren{e_k}} \\
&=\sum_{k=1}^p\abs{x_k}N\paren{f\paren{e_k}}.
\end{aligned}\]

De plus, on a \[\begin{aligned}
\quantifs{\forall i\in\interventierii{1}{p}}\abs{x_i}&\leq\norme{x}_\infty \\
\abs{x_i}N\paren{f\paren{e_i}}&\leq\norme{x}_\infty N\paren{f\paren{e_i}}.
\end{aligned}\]

Donc \(N\paren{f\paren{x}}\leq\norme{x}_\infty\underbrace{\sum_{k=1}^pN\paren{f\paren{e_k}}}_{K}\).

Ceci prouve d'après la \thref{prop1.22} que \(f\) est continue de \(\groupe{E}[\norme{}_\infty]\) dans \(\groupe{F}[N]\).

Soit maintenant \(\norme{}\) une norme quelconque sur \(E\).

Comme \(E\) est de dimension finie, toutes les normes sur \(E\) sont équivalentes, donc il existe \(a,b>0\) tels que \(a\norme{}\leq\norme{}_\infty\leq b\norme{}\).

Donc \(\quantifs{\tpt x\in E}N\paren{f\paren{x}}\leq bK\norme{x}\).

Donc \(f\) est continue de \(\groupe{E}[\norme{}]\) dans \(\groupe{F}[N]\).
\end{dem}

\begin{rem}
L'hypothèse de dimension finie de \(E\) est indispensable. Dans le cas contraire, c'est faux en général.
\end{rem}

Le résultat précédent s'étend aux applications multilinéaires.

\begin{theo}
Soient \(E_1,\dots,E_n\) des espaces vectoriels normés de dimensions finies et \(f:E_1\times\dots\times E_n\to F\) une application \(n\)-linéaire.

Il existe alors une constante \(K\geq0\) telle que \[\quantifs{\tpt\paren{x_1,\dots,x_n}\in E_1\times\dots\times E_n}\norme{f\paren{x_1,\dots,x_n}}\leq K\norme{x_1}_{E_1}\dots\norme{x_n}_{E_n}.\]
\end{theo}

\begin{dem}
Pour tout \(i\in\interventierii{1}{n}\), on note \(p_i=\dim E_i\) et \(\fami{B}_i=\paren{e_{i,1},\dots,e_{i,p_i}}\) une base de \(E_i\).

Soit \(\paren{x_1,\dots,x_n}\in E_1\times\dots\times E_n\). Pour tout \(i\in\interventierii{1}{n}\), on note \(x_i=\paren{x_{i,1},\dots,x_{i,p_i}}_{\fami{B}_i}\).

On a \[\begin{aligned}
f\paren{x_1,\dots,x_n}&=f\paren{\sum_{j_1=1}^{p_1}x_{1,j_1}e_{1,j_1},\dots,\sum_{j_n=1}^{p_n}x_{n,j_n}e_{n,j_n}} \\
&=\sum_{j_1=1}^{p_1}\dots\sum_{j_n=1}^{p_n}x_{1,j_1}\dots x_{n,j_n}f\paren{e_{1,j_1},\dots,e_{n,j_n}}.
\end{aligned}\]

Donc \[\begin{aligned}
N\paren{f\paren{x_1,\dots,x_n}}&\leq\sum_{\substack{1\leq j_1\leq p_1 \\ \vdots \\ 1\leq j_n\leq p_n}}\abs{x_{1,j_1}}\dots\abs{x_{n,j_n}}N\paren{f\paren{e_{1,j_1},\dots,e_{n,j_n}}} \\
&\leq\norme{x_1}_{1,\infty}\dots\norme{x_n}_{n,\infty}\underbrace{\sum_{\substack{1\leq j_1\leq p_1 \\ \vdots \\ 1\leq j_n\leq p_n}}N\paren{f\paren{e_{1,j_1},\dots,e_{n,j_n}}}}_{K}.
\end{aligned}\]

On conclut de la même façon que dans la démonstration précédente.
\end{dem}

\begin{cor}
Soient \(E_1,\dots,E_n\) des espaces vectoriels normés de dimensions finies.

Toute application \(f:E_1\times\dots\times E_n\to F\) qui est \(n\)-linéaire est continue sur \(E_1\times\dots\times E_n\).
\end{cor}

\begin{ex}
\begin{itemize}
    \item Le produit matriciel de \(\M{np}\times\M{pq}\) dans \(\M{nq}\) est bilinéaire, donc continu. \\
    \item Un produit scalaire dans un espace euclidien est bilinéaire, donc continu. \\
    \item Le déterminant dans \(\M{n}\) est \(n\)-linéaire par rapport aux colonnes, donc il est continu.
\end{itemize}
\end{ex}

\subsection{Norme subordonnée}

On définit sur l'espace vectoriel \(\Lc{E}{F}\) des applications linéaires continues de \(E\) dans \(F\) la notion de norme subordonnée (relative aux deux normes sur \(E\) et \(F\)) ou norme triple.

\begin{defi}
Soit \(f\in\Lc{E}{F}\).

On pose \(\normesub{f}=\sup_{x\in\bouleo{0}{1}}\norme{f\paren{x}}\), appelée la norme subordonnée de \(f\).
\end{defi}

\begin{rem}
Cette définition a un sens car \(f\) étant linéaire de \(E\) dans \(F\) et continue, elle est bornée sur \(\bouleo{0}{1}\) d'après la \thref{prop1.22}.
\end{rem}

\begin{rem}
On a \[\normesub{f}=\sup_{x\in\bouleo{0}{1}}\norme{f\paren{x}}=\sup_{x\in\sphere{0}{1}}\norme{f\paren{x}}=\sup_{x\in\boulef{0}{1}}\norme{f\paren{x}}.\]
\end{rem}

\begin{prop}
Soit \(f\in\Lc{E}{F}\).

Alors \(\normesub{f}\) est

\begin{itemize}
    \item égal à \(\sup_{x\not=0}\dfrac{\norme{f\paren{x}}}{\norme{x}}\), mais aussi à \(\sup_{x\in\sphere{0}{1}}\norme{f\paren{x}}\) ; \\
    \item le plus petit réel positif \(M\) tel que \(\quantifs{\tpt x\in E}\norme{f\paren{x}}\leq M\norme{x}\).
\end{itemize}
\end{prop}

\begin{dem}
On note \(N_1\paren{f}=\sup_{x\not=0}\dfrac{\norme{f\paren{x}}}{\norme{x}}\) et \(N_2\paren{f}=\sup_{x\in\sphere{0}{1}}\norme{f\paren{x}}\).
\end{dem}

\begin{dem}[\(N_1\paren{f}=N_2\paren{f}\)]~\\
\(\quantifs{\Tpt x\not=0}\dfrac{x}{\norme{x}}\in\sphere{0}{1}\).

Donc \(\norme{f\paren{\dfrac{x}{\norme{x}}}}\leq N_2\paren{f}\).

Donc \(\dfrac{1}{\norme{x}}\norme{f\paren{x}}\leq N_2\paren{f}\) \ie \(N_1\paren{f}\leq N_2\paren{f}\).

De plus, \(\quantifs{\tpt x\in\sphere{0}{1}}\norme{x}=1\) donc \(\dfrac{\norme{f\paren{x}}}{\norme{x}}=\norme{f\paren{x}}\leq N_1\paren{f}\).

Donc \(N_2\paren{f}\leq N_1\paren{f}\).

Finalement, on a \(N_1\paren{f}=N_2\paren{f}\).
\end{dem}

\begin{dem}[\(N_2\paren{f}=\normesub{f}\)]~\\
Pour tout \(x\in\bouleo{0}{1}\excluant\accol{0}\), on a \(\dfrac{x}{\norme{x}}\in\sphere{0}{1}\) donc \[\begin{aligned}
\norme{f\paren{\dfrac{x}{\norme{x}}}}&\leq N_2\paren{f} \\
\dfrac{1}{\norme{x}}\norme{f\paren{x}}&\leq N_2\paren{f}.
\end{aligned}\]

Or \(\norme{x}\leq1\) donc \(\norme{f\paren{x}}\leq\dfrac{1}{\norme{x}}\norme{f\paren{x}}\leq N_2\paren{f}\).

Ceci est encore vrai pour \(x=0\) donc \(\normesub{f}\leq N_2\paren{f}\).

De plus, soient \(x\in\sphere{0}{1}\) et \(\lambda\in\intervie{0}{1}\).

On a \(\norme{\lambda x}=\lambda<1\) donc \(\lambda x\in\bouleo{0}{1}\).

Donc \(\norme{f\paren{\lambda x}}=\lambda\norme{f\paren{x}}\leq\normesub{f}\).

Donc, par passage à la limite quand \(\lambda\to1\) : \[\norme{f\paren{x}}\leq\normesub{f}\] \ie \(N_2\paren{f}\leq\normesub{f}\).

Donc \(N_2\paren{f}=\normesub{f}\).
\end{dem}

\begin{dem}[Second point]
On a \[\begin{aligned}
\normesub{f}&=\sup_{x\not=0}\dfrac{\norme{f\paren{x}}}{\norme{x}} \\
&=\min\accol{K\in\R\tq\quantifs{\forall x\not=0}\dfrac{\norme{f\paren{x}}}{\norme{x}}\leq K} \\
&=\min\accol{K\in\R\tq\quantifs{\forall x\in E}\norme{f\paren{x}}\leq K\norme{x}}.
\end{aligned}\]
\end{dem}

\begin{ex}
Dans l'\thref{exo1.18}, on avait montré \(\quantifs{\forall f\in E}\abs{I\paren{f}}\leq\norme{f}_\infty\).

On a \(\abs{I\paren{1}}=1=1\times\norme{1}_\infty\).

Donc \(\normesub{I}=1\).
\end{ex}

\begin{meth}
Si \[\quantifs{\forall x\in E}\norme{f\paren{x}}\leq K\norme{x}\] et s'il existe \(x_0\in E\) tel que \[\norme{f\paren{x_0}}=K\norme{x_0}\] alors \(\normesub{f}=K\).
\end{meth}

\begin{prop}
Les normes subordonnées sont des normes sur les espaces \(\Lc{E}{F}\).

Elles sont dites sous-multiplicatives : pour toutes applications linéaires continues et composables \(f\) et \(g\), \[\normesub{f\rond g}\leq\normesub{f}\times\normesub{g}.\]
\end{prop}

\begin{dem}[\(\normesub{}\) est une norme sur \(\Lc{E}{F}\)]
\begin{itemize}
    \item Soit \(f\in\Lc{E}{F}\). \\\\ Si \(\normesub{f}=0\) alors \(\quantifs{\forall x\in E}\norme{f\paren{x}}\leq0\times\norme{x}\) donc \(f=0\). \\
    \item Soient \(f\in\Lc{E}{F}\) et \(\lambda\in\K\). \\\\ On a \[\normesub{\lambda f}=\sup_{x\not=0}\dfrac{\norme{\lambda f\paren{x}}}{\norme{x}}=\sup_{x\not=0}\abs{\lambda}\dfrac{\norme{f\paren{x}}}{\norme{x}}=\abs{\lambda}\sup_{x\not=0}\dfrac{\norme{f\paren{x}}}{\norme{x}}=\abs{\lambda}\normesub{f}.\]
    \item Soit \(\paren{f,g}\in\Lc{E}{F}^2\). \\\\ On a \[\begin{aligned}
        \quantifs{\forall x\in\bouleo{0}{1}}\norme{\paren{f+g}\paren{x}}&=\norme{f\paren{x}+g\paren{x}} \\
        &\leq\norme{f\paren{x}}+\norme{g\paren{x}} \\
        &\leq\normesub{f}+\normesub{g}
    \end{aligned}\] donc \(\normesub{f+g}\leq\normesub{f}+\normesub{g}\).
\end{itemize}
\end{dem}

\begin{dem}[Sous-multiplicativité]
On a \[\begin{aligned}
\quantifs{\forall x\in E}\norme{f\rond g\paren{x}}&=\norme{f\paren{g\paren{x}}} \\
&\leq\normesub{f}\norme{g\paren{x}} \\
&\leq\normesub{f}\normesub{g}\norme{x}.
\end{aligned}\]

Donc \(\normesub{f\rond g}\leq\normesub{f}\normesub{g}\).
\end{dem}

Comme en dimension finie, on peut représenter par choix de bases les applications linéaires par des matrices, on définit de manière semblable la notion de norme sous-multiplicative de matrices (relativement aux normes) ou norme triple.

\begin{defi}
Soit \(\paren{n,p}\in\paren{\Ns}^2\). On choisit deux normes sur \(\K^p\) et \(\K^n\) (espaces identifiés à ceux des matrices-colonnes).

Pour toute matrice \(A\in\M{np}\), on pose \(\normesub{A}=\sup_{\norme{X}=1}\norme{AX}\).
\end{defi}

\begin{prop}
Des normes étant choisies sur les espaces \(\K^p\) et \(\K^n\), les normes subordonnées sont des normes sur tous les espaces \(\M{np}\).

Elles sont dites sous-multiplicatives : pour toutes matrices multipliables \(A\) et \(B\), \[\normesub{AB}\leq\normesub{A}\times\normesub{B}.\]
\end{prop}

\begin{rem}
Dans le cas où un espace vectoriel normé \(E\) est aussi une \(\K\)-algèbre, on dit qu'il est une algèbre normée quand la norme vérifie en plus la propriété de sous-multiplicativité : \(\quantifs{\forall\paren{x,y}\in E^2}\norme{xy}\leq\norme{x}\cdot\norme{y}\).
\end{rem}

\begin{rem}
En dimension finie, toute \(\K\)-algèbre \(A\) possède des normes sous-multiplicatives.
\end{rem}

\begin{dem}
Soit \(A\) une \(\K\)-algèbre de dimension finie.

L'application \(\fonctionlambda{A^2}{A}{\paren{a,b}}{ab}\) est bilinéaire donc continue.

Il existe donc \(K>0\) tel que \[\quantifs{\forall\paren{a,b}\in A^2}\norme{ab}\leq K\norme{a}\norme{b}.\]

On pose \(N=K\norme{}\).

On a alors \[\quantifs{\forall\paren{a,b}\in A^2}N\paren{ab}\leq N\paren{a}N\paren{b}\] et \(N\) est une norme sur \(A\).
\end{dem}

\section{Topologie d'un espace vectoriel normé}

Dans cette section, \(E\) est un espace vectoriel normé.

\subsection{Intérieur d'une partie, voisinage d'un point}

\begin{defi}
Soient \(A\) une partie de \(E\) et \(a\in A\).

On dit que \(a\) est un point intérieur à \(A\) quand on peut trouver un rayon \(r>0\) tel que \(\bouleo{a}{r}\) soit incluse dans \(A\). On dit aussi dans ce cas que \(A\) est un voisinage de \(a\).

L'intérieur de \(A\) est l'ensemble de ses points intérieurs, noté \(\interieur{A}\).

On a : \[a\in\interieur{A}\ssi\quantifs{\exists r>0}\bouleo{a}{r}\subset A.\]
\end{defi}

\begin{exo}
Dans \(\R\), quels sont les intérieurs des parties suivantes : \(\intervii{0}{1}\), \(\intervie{0}{\pinf}\), \(\Q\) ?
\end{exo}

\begin{corr}
\begin{itemize}
    \item Si \(A=\intervii{0}{1}\), alors \(\interieur{A}=\intervee{0}{1}\). \\\\ En effet, pour \(x\in\intervee{0}{1}\), on peut poser \(r=\min\paren{\dfrac{x}{2},\dfrac{1-x}{2}}>0\) pour avoir \(\bouleo{x}{r}\subset\intervii{0}{1}\). \\
    \item Si \(A=\intervie{0}{\pinf}\), alors \(\interieur{A}=\intervee{0}{\pinf}\) (même idée). \\
    \item Si \(A=\Q\), alors \(\interieur{A}=\ensvide\). \\\\ En effet, \(\quantifs{\tpt x\in\Q;\tpt r>0}\text{il existe }y\in\R\excluant\Q\) tel que \(\abs{x-y}<r\) \ie \(\bouleo{x}{r}\not\subset\Q\).
\end{itemize}
\end{corr}

\begin{exo}
Quel est l'intérieur d'une boule de centre \(a\) et de rayon \(r>0\) ?
\end{exo}

\begin{corr}
Soient \(a\in E\) et \(r>0\).

Si \(A=\bouleo{a}{r}\), alors \(\interieur{A}=A\).

En effet, pour tout \(x\in\bouleo{a}{r}\), on pose \(p=\dfrac{r-\norme{x-a}}{2}>0\) et on a \[\bouleo{x}{p}\subset A.\]
\end{corr}

\begin{rem}
Cette notion dépend a prori de la norme utilisée. En dimension finie, ce n'est pas le cas : l'intérieur d'une partie d'un espace vectoriel normé de dimension finie ne dépend pas du choix de la norme (pourquoi ?).
\end{rem}

\begin{dem}
Si \(N_1,N_2\) sont deux normes équivalentes sur \(E\), \(A\) est une partie de \(E\) et \(a\in E\), alors \(a\) est intérieur à \(A\) pour \(N_1\) ssi \(a\) est intérieur à \(A\) pour \(N_2\).

Il existe \(\alpha,\beta>0\) tels que \(\alpha N_2\leq N_1\leq\beta N_2\).

Si \(a\) est intérieur à \(A\) pour \(N_1\), alors il existe \(r>0\) tel que \(\bouleo[1]{a}{r}\subset A\).

On pose \(p=\dfrac{r}{\beta}>0\) et on montre \(\bouleo[2]{a}{p}\subset A\).

Soit \(x\in\bouleo[2]{a}{p}\).

On a \(N_2\paren{a-x}<p=\dfrac{r}{\beta}\).

Donc \(N_1\paren{a-x}\leq\beta N_2\paren{a-x}<r\).

Donc \(x\in\bouleo[1]{a}{r}\subset A\).

Donc \(x\) est intérieur à \(A\) pour \(N_2\).

On montre la réciproque de même, en montrant \(\bouleo[1]{a}{\alpha r}\subset\bouleo[2]{a}{r}\).
\end{dem}

\begin{prop}
Soient \(u\in E^\N\) et \(l\in E\).

La suite \(u\) converge vers \(l\) ssi tout voisinage de \(l\) contient tous les termes de la suite à partir d'un certain rang.
\end{prop}

\subsection{Parties ouvertes}

\begin{defi}
On dit qu'une partie \(A\) de \(E\) est ouverte (ou est un ouvert) quand à tout point de \(a\in A\), on peut associer un rayon \(r>0\) tel que la boule de centre \(a\) et de rayon \(r\) soit incluse dans \(A\) : \[\quantifs{\forall a\in A;\exists r>0}\bouleo{a}{r}\subset A.\]

Autrement dit, \(A\) est ouverte quand tout point de \(A\) est intérieur à \(A\) : \(A=\interieur{A}\), ou, autrement dit, quand \(A\) est un voisinage de chacun de ses points.
\end{defi}

\begin{prop}
L'ensemble vide et \(E\) sont des parties ouvertes. Toute boule ouverte est une partie ouverte. Tout produit (fini) de parties ouvertes est ouvert.
\end{prop}

\begin{dem}
Soient \(E,F\) deux esapces vectoriels normés par \(\norme{}_E\) et \(\norme{}_F\).

On pose \(N\paren{x,y}=\max\paren{\norme{x}_E,\norme{y}_F}\) pour obtenir une norme \(N\) sur \(E\times F\).

Montrons que si \(A\) est un ouvert de \(E\) et \(B\) un ouvert de \(F\), alors \(A\times B\) est un ouvert de \(E\times F\).

Soit \(\paren{a,b}\in A\times B\).

\(a\in A\) et \(A\) est un ouvert donc il existe \(r>0\) tel que \(\bouleo[E]{a}{r}\subset A\).

\(b\in B\) et \(B\) est un ouvert donc il existe \(s>0\) tel que \(\bouleo[F]{b}{s}\subset B\).

On pose \(p=\min\paren{r,s}>0\).

Montrons que \(\bouleo[E\times F]{\paren{a,b}}{p}\subset A\times B\).

Soit \(\paren{x,y}\in\bouleo[E\times F]{\paren{a,b}}{p}\).

On a \(N\paren{\paren{x,y}-\paren{a,b}}<p\) \ie \(N\paren{x-a,y-b}<p\).

Donc \(\norme{x-a}_E<p\) et \(\norme{y-b}_F<p\).

Donc \(x\in\bouleo[E]{a}{p}\) et \(y\in\bouleo[F]{b}{p}\).

Or \(p\leq r\) donc \(\bouleo[E]{a}{p}\subset\bouleo[E]{a}{r}\subset A\) et \(p\leq s\) donc \(\bouleo[F]{b}{p}\subset\bouleo[F]{b}{s}\subset B\).

Donc \(\paren{x,y}\in A\times B\).

Donc \(\bouleo[E\times F]{\paren{a,b}}{p}\subset A\times B\).

On généralise à un produit de plusieurs ouverts par récurrence.
\end{dem}

La topologie de \(E\) est l'ensemble de tous les ouverts de \(E\).

\begin{rem}
La topologie dépend a priori de la norme utilisée. En dimension finie, ce n'est pas le cas : dans un espace vectoriel normé de dimension finie, le fait d'être un ouvert ne dépend pas du choix de la norme.
\end{rem}

\subsection{Parties fermées}

On rappelle la notion de point adhérent à une partie.

\begin{defi}
Soient \(A\) une partie de \(E\) et \(x\in E\).

On dit que \(x\) est un point adhérent à \(A\) quand il existe une suite \(u\in A^\N\) qui converge vers \(x\), ou, ce qui revient au même, quand toute boule centrée en \(x\) rencontre \(A\), ou encore quand \(d\paren{x,A}=0\).

L'adhérence de \(A\) est l'ensemble de ses points adhérents, noté \(\conj{A}\).
\end{defi}

On a montré

\begin{defi}
On dit qu'une partie \(A\) de \(E\) est fermée (ou est un fermé) quand tout point adhérent à \(A\) est dans \(A\), autrement dit quand la propriété suivante est vraie : \[\text{si une suite quelconque à termes dans }A\text{ converge vers un point }x\text{ de }E\text{, alors }x\in A.\]

Ou encore : \(A\) est fermée quand \(A=\conj{A}\).
\end{defi}

\begin{prop}
L'ensemble vide et \(E\) sont des parties fermées. Toute boule fermée est une partie fermée. Tout produit (fini) de parties fermées est fermé.
\end{prop}

On note le lien avec les parties ouvertes.

\begin{prop}
Soit \(A\) une partie de \(E\).

Alors \(A\) est une partie ouverte ssi son complémentaire est une partie fermée.
\end{prop}

\begin{dem}
\impdir

On suppose \(A\) ouverte. On veut montrer que \(E\excluant A\) est fermée.

Soit \(\paren{x_n}\in\paren{E\excluant A}^\N\) qui converge vers \(l\).

Par l'absurde, supposons \(l\in A\).

\(A\) est ouverte donc il existe \(\epsilon>0\) tel que \(\bouleo{l}{\epsilon}\subset A\).

Or \(x_n\tendqd{n\to\pinf}l\) donc il existe \(N\in\N\) tel que \[\quantifs{\forall n\geq N}x_n\in\bouleo{l}{\epsilon}.\]

Donc \(\quantifs{\tpt n\geq N}x_n\in A\) : contradiction.

Donc \(l\in E\excluant A\).

Donc \(E\excluant A\) est un fermé.

\imprec

Supposons que \(E\excluant A\) est fermée. On veut montrer que \(A\) est ouverte.

Soit \(a\in A\).

Par l'absurde, on suppose \(\quantifs{\forall r>0;\exists x\in\bouleo{a}{r}}x\not\in A\).

Alors pour tout \(n\in\N\), il existe \(x_n\in\bouleo{a}{\dfrac{1}{n+1}}\) tel que \(x_n\not\in A\).

On a construit une suite \(\paren{x_n}\in\paren{E\excluant A}^\N\) telle que \(\quantifs{\forall n\in\N}\norme{a-x_n}<\dfrac{1}{n+1}\).

Par théorème d'encadrement, on a \(x_n\tendqd{n\to\pinf}a\).

Or \(a\not\in E\excluant A\) : contradiction car \(E\excluant A\) est un fermé.

Donc il existe \(r>0\) tel que \(\bouleo{a}{r}\subset A\).

Donc \(A\) est un ouvert.
\end{dem}

Encore une fois, le fait d'être un fermé en dimension finie ne dépend pas de la norme.

\begin{prop}
\begin{itemize}
    \item Toute réunion de parties ouvertes est ouverte. Toute intersection finie de parties ouvertes est ouverte. \\
    \item Toute intersection de parties fermées est fermée. Toute réunion finie de parties fermées est fermée.
\end{itemize}
\end{prop}

\begin{dem}[Réunion d'ouverts]
Soit \(\paren{A_i}_{i\in I}\) une famille de parties ouvertes.

Montrons que \(\bigunion_{i\in I}A_i\) est ouverte.

Soit \(x\in\bigunion_{i\in I}A_i\).

Il existe \(i\in I\) tel que \(x\in A_i\).

Or \(A_i\) est ouverte donc il existe \(r>0\) tel que \(\bouleo{x}{r}\subset A_i\).

Donc \(\bouleo{x}{r}\subset\bigunion_{i\in I}A_i\).
\end{dem}

\begin{dem}[Intersection finie d'ouverts]
Soient \(A_1,\dots,A_n\) des parties ouvertes.

Montrons que \(\biginter_{i=1}^nA_i\) est ouverte.

Soit \(x\in\biginter_{i=1}^nA_i\).

Pour tout \(i\in\interventierii{1}{n}\), il existe \(r_i>0\) tel que \(\bouleo{x}{r_i}\subset A_i\).

On pose \(r=\min_{1\leq i\leq n}r_i>0\).

\(\quantifs{\Tpt i\in\interventierii{1}{n}}\bouleo{x}{r}\subset\bouleo{x}{r_i}\subset A_i\).

Donc \(\bouleo{x}{r}\subset\biginter_{i=1}^nA_i\).
\end{dem}

\begin{rem}
Si la famille d'ouverts n'est pas finie, on ne peut rien dire sur l'intersection.

Par exemple, pour \(n\in\N\), on pose les ouverts \(A_n=\intervee{\dfrac{-1}{n+1}}{\dfrac{1}{n+1}}\).

Alors \(\biginter_{n\in\N}A_n=\accol{0}\) n'est pas ouverte.
\end{rem}

\begin{exo}
Montrez que pour tout \(a\in E\), \(E\excluant\accol{a}\) est un ouvert. Déduisez-en que si \(A\) est une partie finie de \(E\), alors \(E\excluant A\) est un ouvert.
\end{exo}

\begin{corr}
Pour tout \(x\in E\excluant\accol{a}\), on pose \(r=\dfrac{\norme{x-a}}{2}\).

Alors \(\bouleo{x}{r}\subset E\excluant\accol{a}\).

Donc \(E\excluant\accol{a}\) est un ouvert.

Si \(A=\accol{a_1,\dots,a_n}\), alors \(E\excluant A\) est le complémentaire de \(\bigunion_{i=1}^n\accol{a_i}\), qui est un fermé par union finie de fermés, et est donc un ouvert.
\end{corr}

\begin{exo}
Quels sont les sous-espaces vectoriels de \(E\) qui sont ouverts ?
\end{exo}

\begin{corr}
Soit \(F\) un sous-espace vectoriel de \(E\) ouvert dans \(E\).

\(0\in F\) et \(F\) est un ouvert donc il existe \(r>0\) tel que \(\bouleo{0}{r}\subset F\).

Soit \(x\in E\excluant\accol{0}\).

On a \(\dfrac{r}{2}\dfrac{x}{\norme{x}}\in\bouleo{0}{r}\) donc \[x=\dfrac{2\norme{x}}{r}\paren{\dfrac{r}{2}\dfrac{x}{\norme{x}}}\in F.\]

Donc \(E=F\) : \(E\) est le seul sous-espace vectoriel de \(E\) ouvert dans \(E\).
\end{corr}

\begin{exo}
Montrez que \(F=\accol{\paren{x,y}\in\R^2\tq x\geq0\text{ et }xy=1}\) est un fermé de \(\R^2\).
\end{exo}

\begin{corr}
Soit \(\paren{\paren{x_n,y_n}}\in F^\N\) qui converge vers \(\paren{a,b}\).

Montrons que \(\paren{a,b}\in F\).

On a \(x_n\tendqd{n\to\pinf}a\), \(y_n\tendqd{n\to\pinf}b\) et \(\quantifs{\forall n\in\N}x_ny_n=1\).

Donc par passage à la limite quand \(n\to\pinf\), on a \(a\geq0\) et \(ab=1\).

Donc \(\paren{a,b}\in F\).

Donc \(F\) est un fermé.
\end{corr}

\begin{exo}
On note \(S\) l'ensemble des matrices de \(\M{n}[\R]\) telles que tous les coefficients soient positifs et sur chaque ligne la somme des coefficients vaut \(1\).

Montrez que \(S\) est un fermé.

NB : \(S\) est l'ensemble des matrices dites stochastiques.
\end{exo}

\begin{rem}
A priori, une partie de \(E\) n'est ni ouverte ni fermée : par exemple, dans \(\R\), l'ensemble \(\intervei{0}{1}\) n'est ni ouvert ni fermé.

Donc ne pas confondre \guillemets{complémentaire} et \guillemets{contraire} : on peut dire qu'une partie est un fermé quand son complémentaire est un ouvert, mais pas que le contraire d'être un ouvert c'est être un fermé.
\end{rem}

\begin{rem}
Il est souvent assez facile de montrer qu'une partie est un fermé grâce à la caractérisation séquentielle. Donc pour montrer qu'une partie est un ouvert, on montre souvent de cette façon que son complémentaire est un fermé.

Les fermés sont souvent définis par des égalités ou des inégalités larges. Par complémentaire, les ouverts sont souvent définis par des inégalités strictes ou des différences.
\end{rem}

\subsection{Ouverts ou fermés relatifs à une partie}

Les définitions précédentes parlent d'ouverts et de fermés de \(E\). On peut définir ces notions relativement à une partie.

\begin{defi}
Soient \(A\) une partie de \(E\) et \(U\) un sous-ensemble de \(A\).

On dit que \(U\) est un ouvert de \(A\) quand il existe un ouvert \(V\) de \(E\) tel que \(U=A\inter V\).

On dit que \(U\) est un fermé de \(A\) quand il existe un fermé \(V\) de \(E\) tel que \(U=A\inter V\).
\end{defi}

On remarque que les fermés de \(A\) sont les complémentaires dans \(A\) des ouverts de \(A\). On peut caractériser de même une partie \(U\) fermée de \(A\) par l'égalité entre \(U\) et l'ensemble de ses points adhérents dans \(A\).

\subsection{Image réciproque d'un ouvert ou d'un fermé par une fonction continue}

\begin{rappel}
Si \(f\) est une fonction de \(E\) dans \(F\) définie sur \(D_f\) et \(B\subset F\), l'image réciproque de \(B\) par \(f\) est \[f\inv\paren{B}=\accol{x\in D_f\tq f\paren{x}\in B}.\]
\end{rappel}

\begin{theo}
Soit \(f\) une fonction de \(E\) dans \(F\) définie sur \(D\).

Alors on a équivalence entre les propositions suivantes :

\begin{enumerate}
    \item \(f\) est continue sur \(D\) ; \\
    \item pour tout fermé \(B\) de \(F\), son image réciproque \(f\inv\paren{B}\) est un fermé de \(D\) ; \\
    \item pour tout ouvert \(B\) de \(F\), son image réciproque \(f\inv\paren{B}\) est un ouvert de \(D\).
\end{enumerate}

Ceci est valable en particulier quand \(f\) est une application continue de \(E\) dans \(F\), auquel cas on peut se passer des notions d'ouvert ou fermé relatif.
\end{theo}

\begin{dem}[\(\paren{1}\imp\paren{2}\)]
Soient \(B\) un fermé de \(F\) et \(\paren{u_n}\in f\inv\paren{B}^\N\) telle que \(u_n\tendqd{n\to\pinf}l\in D\).

\(f\) étant continue sur \(D\) et donc en \(l\), on a \(f\paren{u_n}\tendqd{n\to\pinf}f\paren{l}\).

De plus, on a \(\paren{f\paren{u_n}}\in B^\N\) or \(B\) est un fermé donc \(f\paren{l}\in B\).

Donc \(l\in f\inv\paren{B}\).

Donc \(f\inv\paren{B}\) est un fermé de \(D\).
\end{dem}

\begin{dem}[\(\paren{2}\imp\paren{3}\)]
Soit \(A\) un ouvert de \(F\).

Alors \(F\excluant A\) est un fermé de \(F\).

Donc \(f\inv\paren{F\excluant A}\) est un fermé de \(D\).

Or \(f\inv\paren{F\excluant A}=D\excluant f\inv\paren{A}\).

Donc \(f\inv\paren{A}\) est un ouvert de \(D\).
\end{dem}

\begin{dem}[\(\paren{3}\imp\paren{1}\)]
On suppose que \(\quantifs{\tpt A\text{ ouvert de }F}f\inv\paren{A}\) est un ouvert de \(D\).

Soit \(d\in D\). Montrons que \(f\) est continue en \(d\).

Soit \(\epsilon>0\).

La boule \(\bouleo{f\paren{d}}{\epsilon}\) est un ouvert de \(F\).

Donc \(f\inv\paren{\bouleo{f\paren{d}}{\epsilon}}\) est un ouvert de \(D\).

Or \(f\paren{d}\in\bouleo{f\paren{d}}{\epsilon}\) donc \(d\in f\inv\paren{\bouleo{f\paren{d}}{\epsilon}}\).

Donc il existe \(\alpha>0\) tel que \(D\inter\bouleo{d}{\alpha}\subset f\inv\paren{\bouleo{f\paren{d}}{\epsilon}}\).

Donc pour tout \(x\in D\) tel que \(x\in\bouleo{d}{\alpha}\), on a \(f\paren{x}\in\bouleo{f\paren{d}}{\epsilon}\) \ie \[\quantifs{\forall x\in D}\norme{x-d}<\alpha\imp\norme{f\paren{x}-f\paren{d}}<\epsilon.\]

Donc \(f\) est continue en \(d\).
\end{dem}

\begin{ex}[Cas particuliers fondamentaux]
Si \(f\) est continue sur \(E\) et à valeurs réelles, alors pour tout \(a\in\R\), les ensembles suivants sont des fermés de \(E\) : \[\accol{x\in E\tq f\paren{x}\geq a}\qquad\accol{x\in E\tq f\paren{x}\leq a}\qquad\accol{x\in E\tq f\paren{x}=a}.\]
\end{ex}

\begin{ex}
\begin{itemize}
    \item Les courbes de fonctions continues de \(\R\) dans \(\R\) sont des fermés de \(\R^2\). \\
    \item L'ensemble des matrices de trace nulle est un fermé de \(\M{n}\).
\end{itemize}
\end{ex}

\begin{dem}[Courbes des fonctions continues]
Si \(f:\R\to\R\) est continue, on pose \(\graphe{f}=\accol{\paren{x,y}\in\R^2\tq y=f\paren{x}}\).

On a alors \[\graphe{f}=\phi\inv\paren{\accol{0}}\] où \(\fonction{\phi}{\R^2}{\R}{\paren{x,y}}{y-f\paren{x}}\) est continue sur \(\R^2\) car \(f\) est continue sur \(\R\).

Or \(\accol{0}\) est un fermé de \(\R\) donc \(\graphe{f}\) est un fermé de \(\R^2\).
\end{dem}

\begin{dem}[Ensemble des matrices de trace nulle]
L'ensemble des matrices de trace nulle dans \(\M{n}\) est \[T=\accol{M\in\M{n}\tq\tr\paren{M}=0}.\]

Or \(\M{n}\) est de dimension finie et \(\tr\) est linéaire donc \(\tr\) est continue.

Donc \(T\) est l'image réciproque du fermé \(\accol{0}\) par l'application continue \(\tr\).

Donc \(T\) est un fermé de \(\M{n}\).
\end{dem}

Par passage au complémentaire, si \(f\) est continue sur \(E\) et à valeurs réelles, alors pour tout \(a\in\R\), les ensembles suivants sont des ouverts de \(E\) : \[\accol{x\in E\tq f\paren{x}<a}\qquad\accol{x\in E\tq f\paren{x}>a}\qquad\accol{x\in E\tq f\paren{x}\not=a}.\]

\begin{ex}
\begin{itemize}
    \item L'ensemble des couples \(\paren{x,y}\in\R^2\) tels que \(x>0\) et \(y>x\) est un ouvert de \(\R^2\). \\
    \item \(\GL{n}\) est un ouvert de \(\M{n}\) : si une matrice \(A\) est inversible, alors toutes les matrices proches de \(A\) le sont aussi.
\end{itemize}
\end{ex}

\begin{dem}[Ensemble des couples susmentionnés]
On pose \[\begin{aligned}
A&=\accol{\paren{x,y}\in\R^2\tq x>0\text{ et }y>x} \\
&=\accol{\paren{x,y}\in\R^2\tq x\in\intervee{0}{\pinf}}\inter\accol{\paren{x,y}\in\R^2\tq y-x\in\intervee{0}{\pinf}}.
\end{aligned}\]

Or \(\paren{x,y}\mapsto x\) et \(\paren{x,y}\mapsto y-x\) sont continues.

Donc \(A\) est un ouvert de \(\R^2\).
\end{dem}

\begin{dem}[\(\GL{n}\)]
On a \(\GL{n}={\det}\inv\paren{\K\excluant\accol{0}}\).

Or \(\det\) est continue et \(\K\excluant\accol{0}\) est un ouvert de \(\K\) donc \(\GL{n}\) est un ouvert de \(\M{n}\).
\end{dem}

\subsection{Frontière d'une partie}

\begin{defi}
Soit \(A\) une partie de \(E\). On appelle frontière de \(A\) l'ensemble \(\conj{A}\excluant\interieur{A}\).
\end{defi}

\begin{ex}
\begin{itemize}
    \item Si \(B\) est une boule, alors son intérieur est la boule ouverte de même centre et de même rayon, son adhérence est la boule fermée et sa frontière est la sphère. \\
    \item L'ensemble des rationnels est d'intérieur vide, d'adhérence égale à \(\R\) et donc de frontière \(\R\).
\end{itemize}
\end{ex}

\section{Compacité}

Dans cette section, \(E\) est un espace vectoriel normé.

\subsection{Valeurs d'adhérence d'une suite}

\begin{defi}
Soient \(u=\paren{u_n}\in E^\N\) et \(a\in E\).

On dit que \(a\) est une valeur d'adhérence de la suite \(u\) quand il existe une extractrice \(\phi\) telle que la suite extraite \(\paren{u_{\phi\paren{n}}}\) converge vers \(a\).
\end{defi}

Une suite peut avoir une ou plusieurs valeurs d'adhérence ou ne pas avoir de valeur d'adhérence :

\begin{itemize}
    \item la suite \(\paren{n}_{n\in\N}\) n'a pas de valeur d'adhérence ; \\
    \item toute suite convergente possède une seule valeur d'adhérence : sa limite ; \\
    \item la suite \(u\) définie par \(u_{2n}=\dfrac{1}{n+1}\) et \(u_{2n+1}=1-\dfrac{1}{n+1}\) possède deux valeurs d'adhérence : \(0\) et \(1\) ; \\
    \item il est possible de numéroter les rationnels, autrement dit de créer une suite \(u\) qui prend exactement toutes les valeurs rationnelles dans \(\R\) : cette suite a pour valeurs d'adhérence tous les réels.
\end{itemize}

On peut donner une caractérisation équivalente sans passer par la notion de suite extraite.

\begin{prop}\thlabel{prop1.32}
Soient \(u=\paren{u_n}\in E^\N\) et \(a\in E\).

Alors \(a\) est une valeur d'adhérence de \(u\) ssi \(\quantifs{\tpt\epsilon>0}\accol{n\in\N\tq u_n\in\bouleo{a}{\epsilon}}\text{ est infini}\).
\end{prop}

\begin{dem}
\imprec

Supposons que \(\quantifs{\tpt\epsilon>0}\accol{n\in\N\tq u_n\in\bouleo{a}{\epsilon}}\) est infini.

On spécialise \(\epsilon\gets\dfrac{1}{k+1}\) pour \(k\in\N\).

L'ensemble \(\accol{n\in\N\tq u_n\in\bouleo{a}{1}}\) est infini donc non-vide. On choisit \(\phi\paren{0}\) un élément de cet ensemble.

L'ensemble \(\accol{n\in\N\tq u_n\in\bouleo{a}{\dfrac{1}{2}}}\) est infini donc il contient des entiers strictement supérieurs à \(\phi\paren{0}\) ; on en choisit un, qu'on note \(\phi\paren{1}\).

Si on suppose avoir construit \(\phi\paren{0}<\phi\paren{1}<\dots<\phi\paren{k}\) tels que \(u_{\phi\paren{0}}\in\bouleo{a}{1}\), \(u_{\phi\paren{1}}\in\bouleo{a}{\dfrac{1}{2}}\), ..., \(u_{\phi\paren{k}}\in\bouleo{a}{\dfrac{1}{k+1}}\), comme l'ensemble \[\accol{n\in\N\tq u_n\in\bouleo{a}{\dfrac{1}{k+2}}}\] est infini, on peut choisir \(\phi\paren{k+1}\) dans cet ensemble tel que \(\phi\paren{k+1}>\phi\paren{k}\).

Par récurrence, on construit une suite \(\paren{\phi\paren{k}}_{k\in\N}\) strictement croissante d'entiers naturels tels que \[\quantifs{\forall k\in\N}u_{\phi\paren{k}}\in\bouleo{a}{\dfrac{1}{k+1}}\] \ie \(\norme{u_{\phi\paren{k}}-a}<\dfrac{1}{k+1}\).

Par théorème d'encadrement, on a \(u_{\phi\paren{k}}\tendqd{k\to\pinf}a\) : \(a\) est une valeur d'adhérence de la suite \(\paren{u_n}\).

\impdir

Supposons que \(a\) est une valeur d'adhérence de la suite \(\paren{u_n}\).

Il existe alors une extractrice \(\phi\) telle que \(u_{\phi\paren{k}}\tendqd{k\to\pinf}a\).

Donc pour tout \(\epsilon>0\), il existe \(N\in\N\) tel que \[\quantifs{\forall n\geq N}\norme{u_{\phi\paren{n}}-a}<\epsilon.\]

Donc \(\accol{n\in\N\tq u_n\in\bouleo{a}{\epsilon}}\) contient \(\phi\paren{N},\phi\paren{N+1},\dots\) \ie c'est un ensemble infini.
\end{dem}

Ceci peut encore être réécrit de la façon suivante.

\begin{prop}
Soient \(u=\paren{u_n}\in E^\N\) et \(a\in E\).

Alors \(a\) est une valeur d'adhérence de \(u\) ssi \(\quantifs{\forall\epsilon>0;\forall N\in\N;\exists n\geq N}\norme{u_n-a}<\epsilon\).
\end{prop}

\begin{dem}
Soit \(I\) une partie de \(\N\).

On a \[\begin{aligned}
I\text{ est infini}&\ssi I\text{ n'est pas majorée} \\
&\ssi\non\paren{\quantifs{\exists N\in\N;\forall n\in I}n\leq N} \\
&\ssi\quantifs{\forall N\in\N;\exists n\in I}n>N.
\end{aligned}\]

Pour tout \(\epsilon>0\), on pose \(I_\epsilon=\accol{n\in\N\tq u_n\in\bouleo{a}{\epsilon}}\).

On a alors, d'après la \thref{prop1.32} : \[\begin{aligned}
a\text{ est une valeur d'adhérence de }u&\ssi\quantifs{\forall\epsilon>0}I_\epsilon\text{ est infini} \\
&\ssi\quantifs{\forall\epsilon>0;\forall N\in\N;\exists n\in I_\epsilon}n>N \\
&\ssi\quantifs{\forall\epsilon>0;\forall N\in\N;\exists n>N}\norme{u_n-a}<\epsilon.
\end{aligned}\]
\end{dem}

\begin{exo}
Soit \(u=\paren{u_n}\in E^\N\). Montrez que l'ensemble \(V\) des valeurs d'adhérence de la suite \(u\) est un fermé de \(E\) en utilisant les ensembles \(U_p=\accol{u_n\tq n\geq p}\).
\end{exo}

\begin{corr}
Soit \(a\in E\).

On a \[\begin{aligned}
a\in V&\ssi\quantifs{\forall\epsilon>0;\forall N\in\N;\exists x\in U_N}\norme{x-a}<\epsilon \\
&\ssi\quantifs{\forall\epsilon>0;\forall N\in\N}U_N\inter\bouleo{a}{\epsilon}\not=\ensvide \\
&\ssi\quantifs{\forall N\in\N;\forall\epsilon>0}U_N\inter\bouleo{a}{\epsilon}\not=\ensvide \\
&\ssi\quantifs{\forall N\in\N}a\in\conj{U_N} \\
&\ssi a\in\biginter_{N\in\N}\conj{U_N}.
\end{aligned}\]

Donc \(V=\biginter_{N\in\N}\conj{U_N}\).

\(V\) est donc un fermé par intersection de fermés.
\end{corr}

\subsection{Théorème de Bolzano-Weierstrass}

\begin{theo}
Si \(E\) est de dimension finie, alors toute suite bornée de \(E\) possède une valeur d'adhérence.
\end{theo}

\begin{dem}
On note \(\P{k}\) le prédicat \guillemets{si \(E\) est de dimension \(k\), alors toute suite bornée de \(E\) possède une valeur d'adhérence}.

\begin{itemize}
    \item Pour \(k=1\) : \\\\ On pose \(E=\Vect{e_1}\). \\\\ Si \(\paren{u_n}\) est une suite bornée de \(E\), en notant \(\paren{u_n}=\paren{\lambda_ne_1}\) où \(\paren{\lambda_n}\) est une suite bornée de \(\K\), d'après le théorème de Bolzano-Weierstrass dans \(\R\) ou \(\C\), \(\paren{\lambda_n}\) possède une valeur d'adhérence et donc \(\paren{u_n}\) aussi. \\\\ D'où \(\P{1}\). \\
    \item Soit \(k\in\Ns\) tel que \(\P{k}\) soit vraie. \\\\ Soit \(E\) de dimension \(k+1\). \\\\ On choisit une base \(\fami{B}=\paren{e_1,\dots,e_{k+1}}\) de \(E\). \\\\ Soit \(\paren{u_n}\in E^\N\) une suite bornée. \\\\ Alors les suites-coordonnées associées sont bornées. \\\\ Pour tout \(n\in\N\), on note \(u_n=\paren{u_{1,n},\dots,u_{k+1,n}}_\fami{B}\). \\\\ \(\paren{u_{k+1,n}}_{n\in\N}\) est une suite bornée de \(\K\) donc (même théorème) il existe une extractrice \(\phi\) telle que \(\paren{u_{k+1,\phi\paren{n}}}_{n\in\N}\) converge. \\\\ Pour tout \(n\in\N\), on pose \(v_n=\paren{u_{1,n},\dots,u_{k,n},0}_\fami{B}\). \\\\ \(\paren{v_{\phi\paren{n}}}_{n\in\N}\) est une suite de vecteurs de \(\Vect{e_1,\dots,e_k}\) et bornée donc par hypothèse de récurrence, il existe une extractrice \(\psi\) telle que \(\paren{v_{\phi\rond\psi\paren{n}}}_{n\in\N}\) converge. \\\\ De plus, \(\paren{u_{k+1,\phi\rond\psi\paren{n}}}_{n\in\N}\) converge car c'est une suite extraite d'une suite convergente. \\\\ Donc \(\paren{u_{\phi\rond\psi\paren{n}}}_{n\in\N}\) converge. \\\\ Donc \(\P{k+1}\) est vrai. \\
    \item Donc \(\quantifs{\tpt k\in\Ns}\P{k}\) est vrai.
\end{itemize}
\end{dem}

\begin{rem}
Ce théorème est faux en dimension infinie donc il faut bien mettre en valeur la dimension finie.
\end{rem}

On peut ajouter une précision au théorème précédent.

\begin{prop}
Si \(E\) est de dimension finie, alors toute suite bornée de \(E\) qui ne possède qu'une seule valeur d'adhérence est convergente vers cette valeur d'adhérence.
\end{prop}

\begin{dem}
Supposons \(E\) de dimension finie.

Soit \(\paren{u_n}\in E^\N\) une suite bornée qui admet une unique valeur d'adhérence \(l\).

Par l'absurde, on suppose que \(\paren{u_n}\) ne converge pas vers \(l\).

On a \[\quantifs{\exists\epsilon>0}\underbrace{\quantifs{\forall N\in\N;\exists n\geq N}\norme{u_n-l}\geq\epsilon}_{\accol{n\in\N\tq u_n\not\in\bouleo{l}{\epsilon}}\text{ infini}}.\]

En ordonnant les éléments de cet ensemble et en les notant \(\phi\paren{0}<\phi\paren{1}<\dots\), on construit une extractrice \(\phi\) telle que \[\quantifs{\forall n\in\N}u_{\phi\paren{n}}\not\in\bouleo{l}{\epsilon}.\]

Or \(\paren{u_{\phi\paren{n}}}\) est bornée et \(E\) est de dimension finie donc d'après le théorème de Bolzano-Weierstrass, il existe une extractrice \(\psi\) et \(l\prim\in E\) tels que \[u_{\phi\rond\psi\paren{n}}\tendqd{n\to\pinf}l\prim.\]

Or pour tout \(n\in\N\), \(u_{\phi\paren{n}}\) appartient au fermé \(E\excluant\bouleo{l}{\epsilon}\) donc \(l\prim\in E\excluant\bouleo{l}{\epsilon}\).

Donc \(l\prim\not=l\).

Donc \(l\prim=\lim_{n\to\pinf}u_{\phi\rond\psi\paren{n}}\) est une autre valeur d'adhérence de \(\paren{u_n}\) : contradiction.

Donc \(u_n\tendqd{n\to\pinf}l\).
\end{dem}

\subsection{Parties compactes}

\begin{defi}
Soit \(A\) une partie de \(E\).

On dit que \(A\) est une partie compacte de \(E\) (ou un compact de \(E\)) quand toute suite à termes dans \(A\) possède une valeur d'adhérence dans \(A\) (propriété dite de Bolzano-Weierstrass).
\end{defi}

\begin{ex}
\begin{itemize}
    \item Tout segment \(\intervii{a}{b}\) de \(\R\) est un compact et ce sont les seuls intervalles compacts. \(\intervii{0}{1}\union\intervii{2}{3}\) est compact. \\
    \item Dans \(\K^n\), tout pavé \(\prod_{i=1}^n\intervii{a_i}{b_i}\) est un compact. Plus généralement, un produit (fini) de compacts est compact.
\end{itemize}
\end{ex}

Les parties compactes sont donc celles dont on peut extraire des sous-suites convergentes. Un résultat précédent se généralise alors.

\begin{prop}
Si \(A\) est une partie compacte, alors toute suite de \(A\) qui ne possède qu'une seule valeur d'adhérence est convergente vers cette valeur d'adhérence.
\end{prop}

Un compact étant connu, il est facile d'en construire d'autres.

\begin{prop}
Si \(A\) est une partie compacte de \(E\), alors toute partie \(B\) fermée dans \(A\) est aussi compacte.
\end{prop}

\begin{dem}
Soient \(A\) une partie compacte de \(E\), \(B\) un fermé de \(A\) et \(\paren{u_n}\in B^\N\).

Comme \(B\subset A\), on a \(\paren{u_n}\in A^\N\).

\(A\) est compacte donc il existe une extractrice \(\phi\) et \(l\in A\) tels que \(u_{\phi\paren{n}}\tendqd{n\to\pinf}l\).

La suite \(\paren{u_{\phi\paren{n}}}\) est à termes dans \(B\) et converge vers \(l\) donc comme \(B\) est un fermé, on a \(l\in B\).

Ainsi, toute suite de \(B^\N\) possède une valeur d'adhérence dans \(B\) \ie \(B\) est un compact.
\end{dem}

Reconnaître si une partie est compacte n'est pas toujours facile. On dispose d'une condition nécessaire, qui est suffisante en dimension finie.

\begin{prop}
Soit \(A\) une partie de \(E\).

Si \(A\) est compacte, alors \(A\) est une partie fermée et bornée.
\end{prop}

\begin{dem}
\begin{itemize}
    \item Si \(A\) n'est pas bornée, alors pour tout \(n\in\N\), il existe \(a_n\in A\) tel que \(\norme{a_n}\geq n\). \\\\ Si \(\paren{a_n}\) possède une valeur d'adhérence dans \(A\), alors il existe une extractrice \(\phi\) et \(l\in A\) tels que \(a_{\phi\paren{n}}\tendqd{n\to\pinf}l\). \\\\ Alors \(\norme{a_{\phi\paren{n}}}\tendqd{n\to\pinf}\norme{l}\) : contradiction. \\\\ Donc \(A\) n'est pas compacte. \\
    \item Supposons que \(A\) est compacte. \\\\ Soit \(\paren{u_n}\in A^\N\) telle que \(u_n\tendqd{n\to\pinf}l\in E\). \\\\ \(A\) étant compacte, il existe \(\phi\) une extractrice et \(l\prim\in A\) tels que \(u_{\phi\paren{n}}\tendqd{n\to\pinf}l\prim\). \\\\ La suite \(\paren{u_{\phi\paren{n}}}\) est extraite de la suite convergente \(\paren{u_n}\) donc \(u_{\phi\paren{n}}\tendqd{n\to\pinf}l\). \\\\ Donc par unicité de la limite, on a \(l\prim=l\in A\). \\\\ Donc \(A\) est fermée.
\end{itemize}
\end{dem}

La réciproque est hélas fausse en général. Néanmoins, en dimension finie, elle est vraie.

\begin{prop}
Si \(E\) est de dimension finie, alors une partie de \(E\) est compacte ssi elle est fermée et bornée.
\end{prop}

\begin{rem}
En fait, il n'y a qu'en dimension finie que ce résultat est vrai. Un théorème de Riesz affirme que la boule-unité fermée d'un espace vectoriel normé est compacte ssi l'espace est de dimension finie, ce qui revient à dire que l'équivalence précédente n'est valable que dans un espace de dimension finie.

En dimension infinie, il se passe des choses vraiment étranges : les compacts sont des parties très petites et plates, par exemple, un compact est forcément d'intérieur vide. Heureusement, il est plus courant de travailler à notre niveau en dimension finie.
\end{rem}

\begin{ex}
\begin{itemize}
    \item L'ensemble des matrices stochastiques de \(\M{n}[\R]\) est un compact. \\
    \item La boule-unité fermée de \(E=\ensclasse{0}{\intervii{0}{1}}{\R}\) pour la norme infinie n'est pas compacte, car la suite des fonctions \(\paren{x\mapsto x^n}\) a pour seule valeur d'adhérence possible la fonction \(x\mapsto0\) si \(x\not=1\) et \(1\mapsto1\), qui n'est même pas dans l'espace \(E\).
\end{itemize}
\end{ex}

\begin{dem}[Matrices stochastiques]
On note \[S_n=\accol{M=\paren{m_{i,j}}\in\M{n}[\R]\tq\quantifs{\forall\paren{i,j}\in\interventierii{1}{n}^2}m_{i,j}\geq0\text{ et }\quantifs{\forall i\in\interventierii{1}{n}}\sum_{j=1}^nm_{i,j}=1}.\]

Soit \(M=\paren{m_{i,j}}\in S_n\).

\(\quantifs{\Tpt i\in\interventierii{1}{n}}\sum_{j=1}^nm_{i,j}\) est une somme de réels positifs qui vaut \(1\) donc \(\quantifs{\tpt j\in\interventierii{1}{n}}0\leq m_{i,j}\leq1\).

Donc \(\norme{M}_\infty\leq1\).

Donc \(S_n\) est bornée.

Soit \(\paren{M_k}=\paren{\paren{m_{i,j}^k}_{i,j}}_k\) une suite de matrices de \(S_n\) qui converge vers \(A=\paren{a_{i,j}}\in\M{n}[\R]\) : \[\quantifs{\forall\paren{i,j}\in\interventierii{1}{n}^2}m_{i,j}^k\tendqd{k\to\pinf}a_{i,j}.\]

Par passage à la limite quand \(k\to\pinf\) dans les deux conditions qui définissent \(S_n\), on obtient \[\quantifs{\forall\paren{i,j}\in\interventierii{1}{n}^2}a_{i,j}\geq0\qquad\text{et}\qquad\quantifs{\forall i\in\interventierii{1}{n}}\sum_{j=1}^na_{i,j}=1.\]

Donc \(A\in S_n\).

Donc \(S_n\) est fermée.

On aurait aussi pu considérer les fonctions continues sur \(\M{n}[\R]\) \[c_{i,j}:\paren{m_{i,j}}\mapsto m_{i,j}\qquad\text{et}\qquad s_i:\paren{m_{i,j}}\mapsto\sum_{j=1}^nm_{i,j}\] et remarquer que \[S_n=\biginter_{1\leq i,j\leq n}c_{i,j}\inv\paren{\intervie{0}{\pinf}}\inter\biginter_{i=1}^ns_i\inv\paren{\accol{1}}\] ce qui montre que \(S_n\) est un fermé par intersection de fermés.

Alors, comme \(\M{n}[\R]\) est de dimension finie, on en déduit que \(S_n\) est un compact de \(\M{n}[\R]\).
\end{dem}

\begin{dem}[Deuxième point]
On pose \(f_n:x\mapsto x^n\).

\(\quantifs{\Tpt n\in\N}\norme{f}_\infty=1\).

Si \(\paren{f_n}\) a une valeur d'adhérence \(g\in\boulef{0}{1}\), alors il existe une extractrice \(\phi\) telle que \(f_{\phi\paren{n}}\tendqd{n\to\pinf}g\) \ie \(\norme{f_{\phi\paren{n}}-g}_\infty\tendqd{n\to\pinf}0\).

Or \(\quantifs{\tpt x\in\intervii{0}{1}}\abs{f_{\phi\paren{n}}\paren{x}-g\paren{x}}\leq\norme{f_{\phi\paren{n}}-g}_\infty\).

Donc par encadrement, on a \(f_{\phi\paren{n}}\paren{x}\tendqd{n\to\pinf}g\paren{x}\).

Si \(x\in\intervie{0}{1}\), alors \(f_{\phi\paren{n}}\paren{x}=x^{\phi\paren{n}}\tendqd{n\to\pinf}0\).

Si \(x=1\), alors \(f_{\phi\paren{n}}\paren{x}=1\tendqd{n\to\pinf}1\).

Donc \(g:x\mapsto\begin{dcases}
0 &\text{si }x\in\intervie{0}{1} \\
1 &\text{sinon}
\end{dcases}\)

Or \(g\not\in E\) : contradiction de la compacité.
\end{dem}

Une application importante de la notion de compacité est le théorème suivant.

\begin{theo}
Tout sous-espace vectoriel de dimension finie de \(E\) est fermé.
\end{theo}

\begin{dem}
Soit \(F\) un sous-espace vectoriel de \(E\) de dimension finie.

Soit \(\paren{u_n}\in F^\N\) une suite convergente vers \(l\in E\).

Alors \(\paren{u_n}\) est bornée : il existe \(R>0\) tel que \(\quantifs{\forall n\in\N}u_n\in\boulef{0}{R}\).

Donc \(\quantifs{\forall n\in\N}u_n\in\boulef{0}{R}\inter F=\accol{x\in F\tq\norme{x}\leq R}=\boulef[F]{0}{R}\).

Donc \(\boulef[F]{0}{R}\) est un fermé borné de \(F\) et donc un compact de \(F\).

Il existe donc une extractrice \(\phi\) et \(a\in\boulef[F]{0}{R}\) tels que \(u_{\phi\paren{n}}\tendqd{n\to\pinf}a\).

Donc \(l=a\in F\).

Donc \(F\) est fermé.
\end{dem}

En dimension infinie, là encore il peut se passer des choses étranges : un sous-espace de \(E\) de dimension infinie peut être dense (et donc non-fermé s'il est différent de \(E\)).

\subsection{Théorème des bornes atteintes}

Le principal intérêt des compacts est de pouvoir généraliser un théorème de première année.

\begin{theo}
Soient \(E,F\) deux espaces vectoriels normés, \(A\) une partie de \(E\) et \(f:A\to F\).

Si \(f\) est continue sur \(A\) et \(A\) est compacte, alors \(f\paren{A}\) est compacte.
\end{theo}

\begin{dem}
On suppose que \(f\) est continue et que \(A\) est compacte.

Soit \(\paren{u_n}\in f\paren{A}^\N\).

\(\quantifs{\Tpt n\in\N}u_n\in f\paren{A}\) donc il existe \(v_n\in A\) tel que \(u_n=f\paren{v_n}\).

\(\paren{v_n}\in A^\N\) et \(A\) est compacte donc il existe une extractrice \(\phi\) et \(a\in A\) tels que \(v_{\phi\paren{n}}\tendqd{n\to\pinf}a\).

\(f\) est continue en \(a\) donc \(u_{\phi\paren{n}}=f\paren{v_{\phi\paren{n}}}\tendqd{n\to\pinf}f\paren{a}\in f\paren{A}\).

Donc \(f\paren{A}\) est compacte.
\end{dem}

On résume en disant que l'image continue d'un compact est un compact.

En particulier, toute fonction continue sur un compact est donc bornée. Dans le cas des fonctions numériques (\ie à valeurs dans \(\R\)), on peut même être plus précis.

\begin{theo}
Toute fonction continue sur un compact et à valeurs réelles est bornée et atteint ses bornes.

Autrement dit, si \(f:A\to\R\) est continue sur \(A\) et \(A\) est une partie compacte de \(E\), alors il existe \(\paren{a,b}\in A^2\) tel que \(\quantifs{\tpt x\in A}f\paren{a}\leq f\paren{x}\leq f\paren{b}\), ce qui revient à dire que \(f\) possède un minimum et un maximum sur \(A\).
\end{theo}

\begin{dem}
\(f\paren{A}\) est un fermé borné de \(\R\) donc possède un minimum et un maximum.
\end{dem}

\begin{rem}
Pour toute partie \(X\) bornée de \(\R\) non-vide, \(\sup X\) et \(\inf X\) sont dans l'adhérence de \(X\).
\end{rem}

\begin{rem}
Ce théorème est à rapprocher du théorème vu en première année : toute fonction de \(\R\) dans \(\R\) continue sur un segment est bornée et atteint ses bornes.

Néanmoins, le théorème de l'an dernier donnait un résultat un peu plus précis que celui de cette année car il donnait aussi l'image du segment, en précisant qu'il s'agissait aussi d'un segment, car il faisait aussi intervenir le théorème des valeurs intermédiaires.

Ici, dans la version proposée cette année, on ne peut rien dire de plus.
\end{rem}

\begin{exo}
Un exercice classique, à savoir refaire ! C'est la base de nombreux exercices.

Soient \(E\) de dimension finie et \(f:E\to\R\) continue et telle que \(f\paren{x}\) tende vers \(\pinf\) quand \(\norme{x}\) tend vers \(\pinf\). Montrez que \(f\) possède un minimum.

Exemple : dans le plan euclidien géométrique, on choisit trois points \(A,B,C\) ; montrez alors qu'il existe un point \(M\) du plan tel que la somme \(AM+BM+CM\) soit minimale.
\end{exo}

\begin{corr}[Cas général]
On a \(f\paren{0}\in\R\) donc il existe \(A>0\) tel que \(\quantifs{\forall x\in E}\norme{x}>A\imp f\paren{x}\geq f\paren{0}\).

Sur \(\boulef{0}{A}\), fermé borné d'un espace de dimension finie donc un compact, \(f\) est continue et y admet donc un minimum en \(x_0\) d'après le théorème des bornes atteintes.

Pour tout \(x\in E\),

\begin{itemize}
    \item si \(x\not\in\boulef{0}{A}\), alors \(f\paren{x}\geq f\paren{0}\geq f\paren{x_0}\) car \(0\in\boulef{0}{A}\) \\
    \item si \(x\in\boulef{0}{A}\), alors \(f\paren{x}\geq f\paren{x_0}\).
\end{itemize}

Donc \(f\paren{x_0}=\min_Ef\).
\end{corr}

\begin{corr}[Exemple]
On note \(\fami{P}\) le plan considéré.

On pose \(\fonction{f}{\fami{P}}{\R}{M\dcoords{x}{y}}{AM+BM+CM}\) qui est une fonction continue.

Par inégalité triangulaire, on a \(f\paren{M}\geq3OM+\mathrm{cte}\) donc \[f\paren{M}\tendqd{\norme{\overrightarrow{OM}}\to\pinf}\pinf.\]

D'où l'existence d'un minimum d'après la propriété démontrée précédemment.
\end{corr}

\begin{exo}
Soit \(f:\paren{x,y}\mapsto xy\sqrt{1-x^2-2y^2}\).

Justifiez que l'ensemble de définition \(D\) de \(f\) est un compact de \(\R^2\).

Déterminez les points critiques de \(f\) dans l'ouvert \(\interieur{D}\), puis les maxima et minima de \(f\).
\end{exo}

\begin{corr}
On a \(D=\accol{\paren{x,y}\in\R^2\tq x^2+2y^2\leq1}\).

Pour tout \(\paren{x,y}\in D\), on a \[x^2\leq1\text{ donc }\abs{x}\leq1\] et \[2y^2\leq1\text{ donc }\abs{y}\leq\dfrac{1}{\sqrt{2}}\leq1\] donc \(\norme{\paren{x,y}}_\infty\leq1\) donc \(D\) est borné.

De plus, \(D=\phi\inv\paren{\intervei{\minf}{1}}\) où \(\phi:\paren{x,y}\mapsto x^2+2y^2\) continue sur \(\R^2\) donc \(D\) est un fermé.

\(\R^2\) est de dimension finie donc \(D\) est un compact.

\(f\) est continue sur \(D\) donc d'après le théorème des bornes atteintes, \(\max_Df\) et \(\min_Df\) existent.

Sur \(\interieur{D}\), \(f\) est de classe \(\classe{1}\).

\(\paren{x,y}\in\interieur{D}\) est un point critique de \(f\) ssi \(\nabla f\paren{x,y}=0\) \ie \(\begin{dcases}
\pdv{f}{x}\paren{x,y}=0 \\
\pdv{f}{y}\paren{x,y}=0
\end{dcases}\)

Or \[\begin{aligned}
\begin{dcases}
\pdv{f}{x}\paren{x,y}=0 \\
\pdv{f}{y}\paren{x,y}=0
\end{dcases}&\ssi\begin{dcases}
y\sqrt{1-x^2-2y^2}-xy\dfrac{x}{\sqrt{1-x^2-2y^2}}=0 \\
x\sqrt{1-x^2-2y^2}-xy\dfrac{2y}{\sqrt{1-x^2-2y^2}}=0
\end{dcases} \\
&\ssi\begin{dcases}
y\paren{1-x^2-2y^2}-x^2y=0 \\
x\paren{1-x^2-2y^2}-2xy^2=0
\end{dcases} \\
&\ssi\paren{S}~\begin{dcases}
y\paren{1-2x^2-2y^2}=0 \\
x\paren{1-x^2-4y^2}=0
\end{dcases}
\end{aligned}\]

Si \(x=0\) alors \(y=0\) donc une solution : \(\paren{0,0}\).

Si \(x\not=0\), alors \[\begin{aligned}
\paren{S}&\ssi\begin{dcases}
x^2=1-4y^2 \\
y\paren{1-2\paren{1-4y^2}-2y^2}=0
\end{dcases} \\
&\ssi\begin{dcases}
x^2=1-4y^2 \\
y=0\text{ ou }y^2=\dfrac{1}{6}
\end{dcases} \\
&\ssi\begin{dcases}
x^2=\dfrac{1}{3} \\
y^2=\dfrac{1}{6}
\end{dcases}
\end{aligned}\]

On en déduit quatre autres solutions : \(\paren{\dfrac{t}{\sqrt{3}},\dfrac{t}{\sqrt{6}}}\) où \(t\in\accol{-1,1}\).

\note{À finir}
\end{corr}

On retrouve aussi le théorème de Heine en conséquence de la compacité.

\begin{defi}
Soient \(E,F\) deux espaces vectoriels normés, \(A\) une partie de \(E\) et \(f:A\to F\).

On dit que \(f\) est uniformément continue sur \(A\) quand \[\quantifs{\forall\epsilon>0;\exists\eta>0;\forall\paren{x,y}\in A^2}\norme{x-y}\leq\eta\imp\norme{f\paren{x}-f\paren{y}}\leq\epsilon.\]
\end{defi}

\begin{theo}
Soient \(E,F\) deux espaces vectoriels normés, \(A\) une partie de \(E\) et \(f:A\to F\).

Si \(f\) est continue sur \(A\) et \(A\) est compacte, alors \(f\) est uniformément continue sur \(A\).
\end{theo}

\begin{dem}
Par l'absurde, on suppose \[\quantifs{\exists\epsilon>0;\forall\eta>0;\exists\paren{x,y}\in A^2}\begin{dcases}
\norme{x-y}\leq\eta \\
\norme{f\paren{x}-f\paren{y}}>\epsilon.
\end{dcases}\]

On spécialise \(\eta\gets\dfrac{1}{n+1}\) pour \(n\in\N\).

Pour \(n\in\N\), il existe \(\paren{x_n,y_n}\in A^2\) tel que \(\norme{x-y}\leq\dfrac{1}{n+1}\) et \(\norme{f\paren{x}-f\paren{y}}>\epsilon\).

On a ainsi construit deux suites \(\paren{x_n},\paren{y_n}\in A^\N\) telles que \[\quantifs{\forall n\in\N}\norme{x_n-y_n}\leq\dfrac{1}{n+1}\qquad\text{et}\qquad\norme{f\paren{x_n}-f\paren{y_n}}>\epsilon.\]

\(A\) étant compacte, \(\paren{x_n}\) possède une valeur d'adhérence dans \(A\) donc il existe une extractrice \(\phi\) et \(l\in A\) tels que \(x_{\phi\paren{n}}\tendqd{n\to\pinf}l\).

On a \[\begin{aligned}
\quantifs{\forall n\in\N}\norme{y_{\phi\paren{n}}-l}&\leq\norme{y_{\phi\paren{n}}-x_{\phi\paren{n}}}+\norme{x_{\phi\paren{n}}-l} \\
&\leq\dfrac{1}{\phi\paren{n}+1}+\norme{x_{\phi\paren{n}}-l}.
\end{aligned}\]

Or \(\phi\paren{n}\tendqd{n\to\pinf}\pinf\) et \(\norme{x_{\phi\paren{n}}-l}\tendqd{n\to\pinf}0\) donc par encadrement, on a \(y_{\phi\paren{n}}\tendqd{n\to\pinf}l\).

Par continuité de \(f\) en \(l\), on a \(\begin{dcases}
f\paren{x_{\phi\paren{n}}}\tendqd{n\to\pinf}f\paren{l} \\
f\paren{y_{\phi\paren{n}}}\tendqd{n\to\pinf}f\paren{l}
\end{dcases}\)

Donc \(f\paren{x_{\phi\paren{n}}}-f\paren{y_{\phi\paren{n}}}\tendqd{n\to\pinf}0\), ce qui contredit l'inégalité \[\quantifs{\forall n\in\N}\norme{f\paren{x_{\phi\paren{n}}}-f\paren{y_{\phi\paren{n}}}}>\epsilon.\]
\end{dem}

\section{Connexité par arcs}

Dans cette section, \(E\) est un espace vectoriel normé.

\subsection{Chemin}

\begin{defi}\thlabel{defi:chemin}
Soient \(A\) une partie de \(E\) et \(a,b\in A\).

On appelle chemin (ou arc) dans \(A\) de \(a\) à \(b\) toute application continue \(\phi:\intervii{0}{1}\to A\) telle que \(\phi\paren{0}=a\) et \(\phi\paren{1}=b\). Le support du chemin est l'image de \(\phi\).
\end{defi}

On peut définir une relation d'équivalence sur une partie de \(E\) en mettant en relation les points joignables par un chemin.

\begin{defi}
Soient \(A\) une partie de \(E\) et \(a,b\in A\).

On pose \(a\rel b\) quand il existe un chemin dans \(A\) de \(a\) à \(b\).
\end{defi}

\begin{prop}
Avec les notations précédentes, la relation \(\rel\) est une relation d'équivalence sur \(A\).
\end{prop}

\begin{dem}
\begin{itemize}
    \item Soit \(a\in A\). \\\\ La fonction \(\fonction{\phi}{\intervii{0}{1}}{A}{t}{a}\) est continue sur \(\intervii{0}{1}\) et on a \(\phi\paren{0}=a\) et \(\phi\paren{1}=a\). \\\\ Donc \(a\rel a\) : \(\rel\) est réflexive. \\
    \item Soit \(\paren{a,b}\in A^2\) tel que \(a\rel b\). \\\\ Il existe une fonction continue \(\phi:\intervii{0}{1}\to A\) telle que \(\phi\paren{0}=a\) et \(\phi\paren{1}=b\). \\\\ On pose \(\fonction{\psi}{\intervii{0}{1}}{A}{t}{\phi\paren{1-t}}\) \\\\ \(\psi\) est une fonction continue sur \(\intervii{0}{1}\) telle que \(\psi\paren{0}=b\) et \(\psi\paren{1}=a\). \\\\ Donc \(b\rel a\) : \(\rel\) est symétrique. \\
    \item Soit \(\paren{a,b,c}\in A^3\) tel que \(a\rel b\) et \(b\rel c\). \\\\ Il existe \(\paren{\phi,\psi}\in\ensclasse{0}{\intervii{0}{1}}{A}\) tel que \(\phi\paren{0}=a\), \(\phi\paren{1}=b\), \(\psi\paren{0}=b\) et \(\psi\paren{1}=c\). \\\\ On pose \(\fonction{\theta}{\intervii{0}{1}}{A}{x}{\begin{dcases}\phi\paren{2x} &\text{si }0\leq x\leq\dfrac{1}{2} \\ \psi\paren{2x-1} &\text{sinon}\end{dcases}}\) \\\\ \(\theta\) est une fonction continue sur \(\intervii{0}{1}\) telle que \(\theta\paren{0}=a\) et \(\theta\paren{1}=c\). \\\\ Donc \(a\rel c\) : \(\rel\) est transitive. \\
    \item Finalement, \(\rel\) est une relation d'équivalence.
\end{itemize}
\end{dem}

\subsection{Parties connexes par arcs}

\begin{defi}\thlabel{defi:connexeParArcs}
Soit \(A\) une partie de \(E\).

On dit que \(A\) est connexe par arcs quand tout couple de points \(\paren{a,b}\in A^2\) est joignable par un chemin.
\end{defi}

\begin{ex}
\begin{itemize}
    \item Les parties convexes de \(E\) sont connexes par arcs. \\
    \item Les parties étoilées de \(E\) sont connexes par arcs. \\
    \item \(\Cs\) et \(\C\excluant D\) où \(D\) est la demi-droite des réels négatifs sont connexes par arcs.
\end{itemize}
\end{ex}

\begin{dem}
\begin{itemize}
    \item Une partie convexe est une partie dont tous les points sont reliables en ligne droite donc, en particulier, est une partie connexe par arcs. \\
    \item Une partie \(A\) est dite étoilée quand il existe \(c\in A\) tel que \(\quantifs{\tpt b\in A}\croch{cb}\subset A\). Alors \(A\) est clairement connexe par arcs.
\end{itemize}
\end{dem}

Les classes d'équivalences de la relation notée \(\rel\) précédemment s'appellent les composantes connexes par arcs de \(A\) : ce sont par définition des parties connexes par arcs.

\begin{prop}
Les seules parties connexes par arcs de \(\R\) sont les intervalles.
\end{prop}

\begin{rem}
Il existe une notion plus générale, celle de partie connexe : une partie \(A\) de \(E\) est dite connexe quand les seules parties de \(A\) à la fois ouvertes et fermées sont \(\ensvide\) et \(A\). Elle est plus délicate à aborder et est hors-programme, c'est pourquoi on s'en tient à la notion de connexité par arcs (toute partie connexe par arcs est connexe).
\end{rem}

\subsection{Théorème des valeurs intermédiaires}

Là encore, la notion de connexité par arcs permet de généraliser des résultats de première année.

\begin{theo}\thlabel{theo1.16}
Soient \(E,F\) deux espaces vectoriels normés, \(A\) une partie de \(E\) et \(f:A\to F\).

Si \(f\) est continue par \(A\) et \(A\) est connexe par arcs, alors \(f\paren{A}\) est connexe par arcs.
\end{theo}

\begin{dem}
Supposons que \(A\) est connexe par arcs et que \(f\) est continue.

Soit \(\paren{x,y}\in f\paren{A}^2\).

Il existe \(\paren{a,b}\in A^2\) tel que \(f\paren{a}=x\) et \(f\paren{b}=y\).

Or \(A\) est connexe par arcs donc il existe \(\phi:\intervii{0}{1}\to A\) continue telle que \(\phi\paren{0}=a\) et \(\phi\paren{1}=b\).

\(f\rond\phi\) est donc un chemin qui relie \(x\) et \(y\) (par composition de fonctions continues).

Donc \(f\paren{A}\) est connexe par arcs.
\end{dem}

On résume en disant que l'image continue d'un connexe par arcs est un connexe par arcs.

Dans le cas des fonctions numériques (\ie à valeurs dans \(\R\)), on peut même être plus précis.

\begin{theo}
Toute fonction continue sur un connexe par arcs et à valeurs réelles vérifie la propriété des valeurs intermédiaires.

Autrement dit, si \(f:A\to F\) est continue sur \(A\) une partie connexe par arcs de \(E\), alors \(f\paren{A}\) est un intervalle.

Ou encore : \[\quantifs{\forall\paren{y,z}\in f\paren{A}^2;\forall w\in\croch{yz};\exists t\in A}f\paren{t}=w.\]
\end{theo}

\begin{dem}
Évident à partir du \thref{theo1.16}.
\end{dem}


\chapter{Séries numériques et vectorielles : révisions et compléments}

\minitoc

Dans ce chapitre, \(E\) désigne un espace vectoriel normé (qui peut être \(\R\) ou \(\C\)) et \(\norme{}\) la norme associée (qui est dans ces cas la valeur absolue ou le module).

\section{Rappels}

\subsection{Définitions et notations}

\begin{defi}[Série vectorielle]
Soit \(u\) une suite de \(E\).

On associe à cette suite la suite \(s\) définie de la façon suivante : \(\quantifs{\tpt n\in\N}s_n=\sum_{k=0}^nu_k\).

La suite \(s\) est appelée série de terme général \(u_n\) et notée \(\sum_{n\geq0}u_n\) ou \(\sum u\).

Chaque nombre \(s_n\) est appelé somme partielle d'indice \(n\) de la série.
\end{defi}

L'adjectif \guillemets{numérique} associé au mot \guillemets{série} signifie que les termes généraux de la série sont en fait des nombres réels ou complexes.

\subsection{Convergence d'une série}

\begin{defi}
Soit \(u\) une suite de \(E\).

On dit que la série \(\sum u\) converge ssi la suite des sommes partielles \(\paren{s_n}=\paren{\sum_{k=0}^nu_k}\) converge.

Dans ce cas, si \(l=\lim_{n\to\pinf}s_n\), alors \(l\) est appelée somme de la série \(\sum u\) et on note \(l=\sum_{n=0}^{\pinf}u_n\).

On appelle aussi reste partiel d'indice \(n\) de la série le nombre \(r_n=\sum_{k=n+1}^{\pinf}u_k\), de sorte que \(r_n+s_n=l\).

La suite des restes partiels converge donc vers \(0\).

Dans le cas contraire, on dit que la série \(\sum u\) diverge.
\end{defi}

\begin{ex}
\begin{itemize}
    \item Soit \(x\in\C\). La série \(\sum_{n\geq0}x^n\) converge ssi \(\abs{x}<1\) et, dans ce cas, \(\sum_{n=0}^{\pinf}x^n=\dfrac{1}{1-x}\). Cette série est appelée série géométrique de raison \(x\). \\
    \item Les séries de Riemann : \(\sum_{n\geq1}\dfrac{1}{n^\alpha}\) converge ssi \(\alpha>1\). \\
    \item Pour tout \(z\in\C\), la série \(\sum\dfrac{z^n}{n!}\) converge et \(\sum_{n=0}^{\pinf}\dfrac{z^n}{n!}=\e{z}\).
\end{itemize}
\end{ex}

On peut bien sûr généraliser aux séries quelques théorèmes d'opérations.

\begin{prop}
Soient \(u,v\) deux suites de \(E\) et \(\lambda\) un scalaire.

Si les séries \(\sum u\) et \(\sum v\) convergent, alors la série \(\sum\paren{u+\lambda v}\) converge et \(\sum_{n=0}^{\pinf}\paren{u_n+\lambda v_n}=\sum_{n=0}^{\pinf}u_n+\lambda\sum_{n=0}^{\pinf}v_n\).

Ceci prouve aussi que l'ensemble des séries convergentes est un espace vectoriel.
\end{prop}

\begin{rem}
La somme d'une série divergente et d'une série convergente est une série divergente.

En revanche, il n'y a rien à dire a priori à propos de la somme de deux séries divergentes.
\end{rem}

\subsection{Lien entre convergence de suites et convergence de séries}

\begin{prop}
Soit \(u\) une suite de \(E\).

Si la série \(\sum u\) converge, alors la suite \(u\) converge vers \(0\).
\end{prop}

\begin{rem}
\begin{itemize}
    \item La réciproque est fausse. \\
    \item Par contraposition, si une suite \(u\) ne tend pas vers \(0\), alors la série associée diverge : on dit que la série \(\sum u\) diverge grossièrement.
\end{itemize}
\end{rem}

\begin{ex}
On appelle série harmonique la série \(\sum_{n\geq1}\dfrac{1}{n}\).

Cette série diverge, pourtant son terme général tend vers \(0\).
\end{ex}

\begin{defi}
Soit \(u\) une suite de \(E\). On pose \(v_n=u_{n+1}-u_n\).

La série \(\sum v\) est appelée la série télescopique (ou série domino, ou série différence) associée à \(u\).
\end{defi}

\begin{prop}
Une suite converge ssi sa série télescopique associée converge.
\end{prop}

\begin{exo}~\\
On pose \(u_n=\sum_{k=1}^n\dfrac{1}{k}-\ln n\). Montrez que la suite \(u\) converge.
\end{exo}

\begin{corr}
\(u\) converge ssi \(\sum\paren{u_{n+1}-u_n}\) converge.

On a \[\begin{aligned}
u_{n+1}-u_n&=\dfrac{1}{n+1}-\ln\paren{n+1}+\ln n \\
&=\dfrac{1}{n+1}+\ln\dfrac{n}{n+1} \\
&=\dfrac{1}{n+1}+\ln\paren{1-\dfrac{1}{n+1}}.
\end{aligned}\]

Or \(\ln\paren{1+u}\egqd{u\to0}u-\dfrac{u^2}{2}+\o{u^2}\) et \(\dfrac{1}{n+1}\tendqd{n\to\pinf}0\) donc \[\begin{aligned}
u_{n+1}-u_n&\egqd{n\to\pinf}\dfrac{1}{n+1}+\paren{\dfrac{-1}{n+1}+\dfrac{1}{2\paren{n+1}^2}+\o{\dfrac{1}{\paren{n+1}^2}}} \\
&\egqd{n\to\pinf}\dfrac{1}{2\paren{n+1}^2}+\o{\dfrac{1}{\paren{n+1}^2}} \\
&\simqd{n\to\pinf}\dfrac{1}{2n^2}.
\end{aligned}\]

Les séries \(\sum\paren{u_{n+1}-u_n}\) et \(\sum\dfrac{1}{2n^2}\) sont à termes positifs à partir d'un certain rang et \(\sum\dfrac{1}{n^2}\) converge donc d'après le théorème de comparaison des séries à termes positifs, \(\sum\paren{u_{n+1}-u_n}\) converge \ie \(u\) converge.

Note culturelle : on appelle constante d'Euler-Mascheroni la limite de cette suite, notée \(\gamma\) et on a le développement asymptotique \[\sum_{k=1}^n\dfrac{1}{k}\egqd{n\to\pinf}\ln n+\gamma+\o{1}.\]

On remarque que l'on a \[\sum_{k=1}^n\dfrac{1}{k}\simqd{n\to\pinf}\ln n.\]
\end{corr}

\section{Séries réelles à termes positifs}

Dans cette section, on s'intéresse uniquement aux séries dont le terme général est un réel positif.

On appelle un premier théorème issu du cours de première année.

\begin{theo}
Soient \(u\) et \(v\) deux suites réelles positives.

\begin{itemize}
    \item Si \(0\leq u\leq v\) et si la série \(\sum v\) converge, alors la série \(\sum u\) converge. \\
    \item Si \(0\leq u\leq v\) et si la série \(\sum u\) diverge, alors la série \(\sum v\) diverge. \\
    \item Si \(u\sim v\), alors les séries \(\sum u\) et \(\sum v\) sont de même nature.
\end{itemize}
\end{theo}

Une application classique : la règle de d'Alembert.

\begin{prop}
Soit \(u\) une suite réelle strictement positive telle que \(\dfrac{u_{n+1}}{u_n}\tendqd{n\to\pinf}l\). Alors

\begin{itemize}
    \item si \(l<1\), la série \(\sum u\) converge ; \\
    \item si \(l>1\), la série \(\sum u\) diverge ; \\
    \item si \(l=1\), on ne peut rien conclure.
\end{itemize}
\end{prop}

\begin{dem}
\begin{itemize}
    \item Si \(l<1\), on pose \(K=\dfrac{1+l}{2}\). \\\\ Comme \(l<K<1\) et \(\dfrac{u_{n+1}}{u_n}\tendqd{n\to\pinf}l\), il existe \(N\in\N\) tel que \(\quantifs{\tpt n\geq N}0<\dfrac{u_{n+1}}{u_n}\leq K\). \\\\ Par récurrence, on montre \(\quantifs{\forall n\geq N}u_n\leq u_NK^{n-N}=\dfrac{u_N}{K^N}K^n\). \\\\ Comme \(0<K<1\), la série \(\sum K^n\) converge donc d'après le théorème de comparaison des séries à termes positifs, \(\sum u_n\) converge. \\
    \item Si \(l>1\), à partir d'un certain rang, on a \(u_n\geq\paren{\dfrac{1+l}{2}}^n\times\mathrm{cte}\tendqd{n\to\pinf}\pm\infty\). \\\\ Donc \(\sum u_n\) diverge grossièrement. \\
    \item Si \(l=1\), on ne peut rien dire. Deux exemples : \\ \begin{itemize}
        \item En posant \(u_n=\dfrac{1}{n}\), on a \(\dfrac{u_{n+1}}{u_n}\tendqd{n\to\pinf}1\) mais \(\sum u_n\) diverge. \\\\
        \item En posant \(u_n=\dfrac{1}{n^2}\), on a \(\dfrac{u_{n+1}}{u_n}\tendqd{n\to\pinf}1\) mais \(\sum u_n\) converge.
    \end{itemize}
\end{itemize}
\end{dem}

\begin{exo}
Soient \(x,y>0\). Représentez graphiquement l'ensemble des couples \(\paren{x,y}\) tels que la série \(\sum\dfrac{x^n}{y^n+n^x}\) converge.
\end{exo}

\begin{corr}
Pour tout \(n\in\N\), on pose \(u_n=\dfrac{x^n}{y^n+n^x}\geq0\).

\begin{itemize}
    \item Si \(x=1\), alors \(u_n=\dfrac{1}{y^n+n}\tendqd{n\to\pinf}0\). \\ \begin{itemize}
        \item Si \(y>1\) : \\\\ On a \(n\egqd{n\to\pinf}\o{y^n}\) donc \(y^n+n\simqd{n\to\pinf}y^n\). \\\\ Donc \(u_n\simqd{n\to\pinf}\dfrac{1}{y^n}=\paren{\dfrac{1}{y}}^n\). \\\\ Or \(0\leq\dfrac{1}{y}<1\) donc la série géométrique \(\sum\paren{\dfrac{1}{y}}^n\) converge. \\\\ Ainsi, d'après le théorème de comparaison des séries à termes positifs, \(\sum u_n\) converge. \\
        \item Sinon, on a \(y^n\egqd{n\to\pinf}\o{n}\) donc \(u_n\simqd{n\to\pinf}\dfrac{1}{n}\). \\\\ Or \(\sum\dfrac{1}{n}\) diverge donc d'après le théorème de comparaison des séries à termes positifs, \(\sum u_n\) diverge. \\
    \end{itemize}
    \item Si \(y<1\), on a \(n\egqd{n\to\pinf}\o{y^n}\) donc \(y^n+n\simqd{n\to\pinf}y^n\). \\\\ Donc \(u_n\simqd{n\to\pinf}\dfrac{x^n}{y^n}=\paren{\dfrac{x}{y}}^n\). \\\\ Ainsi, d'après le théorème de comparaison des séries à termes positifs, \(\sum u_n\) converge ssi \(\dfrac{x}{y}<1\) \ie \(x<y\). \\
    \item Si \(y\leq1\), on a \(y^n\egqd{n\to\pinf}\o{n^x}\) donc \(u_n\simqd{n\to\pinf}\dfrac{x^n}{n^x}\). \\ \begin{itemize}
        \item Si \(x>1\), on a \(n^x\egqd{n\to\pinf}\o{x^n}\) donc \(\lim_{n\to\pinf}u_n=\pinf\) : \(\sum u_n\) diverge grossièrement. \\
        \item Si \(x<1\), on a \(0\leq\dfrac{x^n}{n^x}\leq x^n\) donc, comme \(0<x<1\), \(\sum x^n\) converge et par théorème de comparaison des séries à termes positifs, \(\sum\dfrac{x^n}{n^x}\) converge, donc \(\sum u_n\) converge.
    \end{itemize}
\end{itemize}

Finalement, on a la représentation graphique suivante :

\begin{center}
\begin{tikzpicture}
\draw[thick,gray,->] (-1,0) -- (5,0) node[below right] {\(x\)};
\draw[thick,gray,->] (0,-1) -- (0,5) node[above left] {\(y\)};
\fill[pattern=north west lines] (0,0) -- (1,0) -- (1,1) -- (5,5) -- (0,5) -- (0,0);
\end{tikzpicture}
\end{center}
\end{corr}

\begin{exo}\thlabel{exo2.3}
Montrez que la suite \(\paren{u_n}\) définie par \(u_0\in\intervii{0}{1}\) et \(u_{n+1}=\dfrac{1}{2}\paren{u_n+u_n^2}\) converge vers \(0\) et donnez la nature de la série \(\sum u_n\).
\end{exo}

\begin{corr}
Localisation : on montre par récurrence triviale que \(\quantifs{\forall n\in\N}u_n\in\intervie{0}{1}\).

Monotonie : pour tout \(n\in\N\), on a \[u_{n+1}-u_n=\dfrac{u_n+u_n^2}{2}-u_n=\dfrac{u_n^2-u_n}{2}\leq\dfrac{u_n\paren{u_n-1}}{2}\leq0\] donc \(\paren{u_n}\) est décroissante.

Or \(\paren{u_n}\) est minorée par \(0\) donc \(\paren{u_n}\) converge d'après le théorème de la limite monotone.

On pose \(l=\lim_nu_n\in\intervie{0}{1}\).

Par opérations sur les limites, on a \[u_{n+1}=\dfrac{u_n+u_n^2}{2}\tendqd{n\to\pinf}\dfrac{l+l^2}{2}.\]

Or \(\paren{u_{n+1}}\) est extraite de \(\paren{u_n}\) donc \(u_{n+1}\tendqd{n\to\pinf}l\).

Par unicité de la limite, on a donc \(l=\dfrac{l+l^2}{2}\).

Donc \(l\paren{l-1}=0\), or \(l\in\intervie{0}{1}\) donc \(l=0\).

Enfin, on a \(u_n\tendqd{n\to\pinf}0\) donc il existe \(N\in\N\) tel que \(\quantifs{\forall n\geq N}u_n^2\leq\dfrac{u_n}{2}\).

Donc pour \(n\geq N\), on a \(u_{n+1}\leq\dfrac{3}{4}u_n\).

Par récurrence, on montre que \(\quantifs{\tpt n\geq N}u_n\leq\paren{\dfrac{3}{4}}^{-N}u_N\times\paren{\dfrac{3}{4}}^n\).

Or la série géométrique \(\sum\paren{\dfrac{3}{4}}^n\) converge donc d'après le théorème de comparaison des séries à termes positifs, \(\sum u_n\) converge.
\end{corr}

On donne quelques versions plus élaborées du théorème de comparaison.

\subsection{Théorème de Cesàro}

\begin{theo}
Soit \(u\) une suite numérique qui converge vers \(l\). Alors \(\dfrac{u_0+\dots+u_n}{n}\tendqd{n\to\pinf}l\).
\end{theo}

\begin{dem}
Soit \(\epsilon>0\). Il existe \(N\in\N\) tel que \(\quantifs{\forall n\geq N}\abs{u_n-l}\leq\epsilon\).

On a \[\begin{aligned}
\dfrac{u_0+\dots+u_n}{n}-l&=\dfrac{u_0+\dots+u_N}{n}+\dfrac{u_{N+1}+\dots+u_n}{n}-l \\
&=\dfrac{u_0+\dots+u_N}{n}+\dfrac{u_{N+1}-l+\dots+u_n-l}{n}+\dfrac{\paren{n-N}l}{n}-l \\
&=\dfrac{u_0+\dots+u_N}{n}-\dfrac{Nl}{n}+\dfrac{u_{N+1}-l+\dots+u_n-l}{n}.
\end{aligned}\]

Donc \[\begin{WithArrows}
\abs{\dfrac{u_0+\dots+u_n}{n}-l}&\leq\dfrac{\overbrace{\abs{u_0+\dots+u_N-Nl}}^{K}}{n}+\dfrac{\abs{u_{N+1}-l}+\dots+\abs{u_n-l}}{n} \\
&\leq\dfrac{K}{n}+\dfrac{\paren{n-N}\epsilon}{n} \Arrow{\(\dfrac{n-N}{n}\leq1\)} \\
&\leq\dfrac{K}{n}+\epsilon.
\end{WithArrows}\]

Or \(\dfrac{K}{n}\tendqd{n\to\pinf}0\) donc il existe \(N\prim\in\N\) tel que \(\quantifs{\forall n\geq N\prim}\dfrac{K}{n}\leq\epsilon\).

On pose \(N\seconde=\max\paren{N,N\prim}\).

Donc \(\quantifs{\tpt n\geq N\seconde}\abs{\dfrac{u_0+\dots+u_n}{n}-l}\leq2\epsilon\).

On a montré \[\quantifs{\forall\epsilon>0;\exists N\in\N;\forall n\geq N}\abs{\dfrac{u_0+\dots+u_n}{n}-l}\leq\epsilon\] donc \(\dfrac{u_0+\dots+u_n}{n}\tendqd{n\to\pinf}l\).
\end{dem}

Dans le cas où \(l\not=0\), la série \(\sum u\) diverge grossièrement et \(\sum_{k=0}^nu_k\simqd{n\to\pinf}nl\).

Dans le cas où \(l=0\), on peut juste dire \(\sum_{k=0}^nu_k=\o{n}\).

\begin{exo}
Soit \(u\) la suite définie par récurrence par \(u_0>0\) et \(\quantifs{\tpt n\in\N}u_{n+1}=u_n+\dfrac{u_n+1}{u_n+2}\).

Étudiez la convergence ou divergence de la suite \(u\), puis donnez un équivalent simple de \(u_n\) quand \(n\) tend vers \(\pinf\).
\end{exo}

\begin{corr}
Par récurrence immédiate, \(\paren{u_n}\) est à termes strictement positifs.

On en déduit que \(\quantifs{\tpt n\in\N}u_{n+1}-u_n=\dfrac{u_n+1}{u_n+2}>0\) donc \(\paren{u_n}\) est croissante.

Par l'absurde, on suppose que \(\paren{u_n}\) converge vers \(l\in\Rps\).

De la même façon que dans l'\thref{exo2.3}, par passage à la limite, on obtient \(l-l=\dfrac{l+1}{l+2}\) donc \(l=-1\) : contradiction.

Donc \(\paren{u_n}\) est croissante et diverge donc elle diverge vers \(\pinf\).

De plus, on a \(\sum_{k=0}^{n-1}\paren{u_{k+1}-u_k}=u_n-u_0=\sum_{k=0}^{n-1}\dfrac{u_n+1}{u_n+2}\).

Or \(\dfrac{u_n+1}{u_n+2}\tendqd{n\to\pinf}1\) donc d'après le théorème de Cesàro, on a \[\dfrac{\ds\sum_{k=0}^{n-1}\dfrac{u_n+1}{u_n+2}}{n}\tendqd{n\to\pinf}1\] \ie \(u_n-u_0\simqd{n\to\pinf}n\).

Donc \(u_n\simqd{n\to\pinf}n\).
\end{corr}

\subsection{Théorème de comparaison par domination de séries à termes positifs}

Dans le cas convergent d'abord, les restes partiels suivent la même relation de comparaison.

\begin{theo}\thlabel{theo2.3}
Soient \(u,v\) deux suites réelles positives.

Si \(u=\O{v}\) et la série \(\sum v\) converge, alors la série \(\sum u\) converge. De plus, \(\sum_{k=n+1}^{\pinf}u_k=\O{\sum_{k=n+1}^{\pinf}v_k}\).

Si \(u=\o{v}\) et la série \(\sum v\) converge, alors la série \(\sum u\) converge. De plus, \(\sum_{k=n+1}^{\pinf}u_k=\o{\sum_{k=n+1}^{\pinf}v_k}\).
\end{theo}

\begin{dem}
\begin{itemize}
    \item Si \(u_n\egqd{n\to\pinf}\O{v_n}\), il existe \(K>0\) et \(N\in\N\) tels que \(\quantifs{\forall n\geq N}0\leq u_n\leq Kv_n\). \\\\ La série \(\sum v_n\) converge donc d'après le théorème de comparaison des séries à termes positifs, \(\sum u_n\) converge. \\\\ Pour \(n\geq N\), on a \(\sum_{k=n+1}^{\pinf}u_k\leq K\sum_{k=n+1}^{\pinf}v_k\). \\\\ D'où \(\sum_{k=n+1}^{\pinf}u_k\egqd{n\to\pinf}\O{\sum_{k=n+1}^{\pinf}v_k}\). \\
    \item Si \(u_n\egqd{n\to\pinf}\o{v_n}\), pour tout \(\epsilon>0\), il existe \(N\in\N\) tel que \(\quantifs{\forall n\geq N}0\leq u_n\leq\epsilon v_n\). \\\\ La série \(\sum v_n\) converge donc d'après le théorème de comparaison des séries à termes positifs, \(\sum u_n\) converge. \\\\ Pour \(n\geq N\), on a \(\sum_{k=n+1}^{\pinf}u_k\leq\epsilon\sum_{k=n+1}^{\pinf}v_k\). \\\\ D'où \(\sum_{k=n+1}^{\pinf}u_k\egqd{n\to\pinf}\o{\sum_{k=n+1}^{\pinf}v_k}\).
\end{itemize}
\end{dem}

Dans le cas divergent ensuite, les sommes partielles suivent aussi la même relation de comparaison.

\begin{theo}
Soient \(u,v\) deux suites réelles positives.

Si \(u=\O{v}\) et la série \(\sum u\) diverge, alors la série \(\sum v\) diverge. De plus, \(\sum_{k=0}^nu_k=\O{\sum_{k=0}^nv_k}\).

Si \(u=\o{v}\) et la série \(\sum u\) diverge, alors la série \(\sum v\) diverge. De plus, \(\sum_{k=0}^nu_k=\o{\sum_{k=0}^nv_k}\).
\end{theo}

\begin{dem}\thlabel{demtheo2.4}
\begin{itemize}
    \item Si \(u_n\egqd{n\to\pinf}\O{v_n}\), il existe \(K>0\) et \(N\in\N\) tels que \(\quantifs{\forall n\geq N}0\leq u_n\leq Kv_n\). \\\\ Par comparaison de séries à termes positifs, \(\sum v_n\) diverge. \\\\ \(u\) et \(v\) étant deux suites positives telles que \(\sum u_n\) et \(\sum v_n\) divergent, on en déduit que \(\sum_{k=0}^nu_k\tendqd{n\to\pinf}\pinf\) et \(\sum_{k=0}^nv_k\tendqd{n\to\pinf}\pinf\). \\\\ Pour \(n\geq N\), on a \[\begin{aligned}
        \sum_{k=0}^nu_k&=\sum_{k=0}^Nu_k+\sum_{k=N+1}^nu_k \\
        &\leq\sum_{k=0}^Nu_k+K\sum_{k=N+1}^nv_k \\
        &\leq\underbrace{\sum_{k=0}^Nu_k-K\sum_{k=0}^Nv_k}_{L}+K\sum_{k=0}^nv_k.
    \end{aligned}\] Or \(\sum_{k=0}^nv_k\tendqd{n\to\pinf}\pinf\) donc il existe \(N\prim\in\N\) tel que \(\quantifs{\forall n\geq N\prim}L\leq\sum_{k=0}^nv_k\). \\\\ Alors, pour \(n\geq\max\paren{N,N\prim}\), on a \(\sum_{k=0}^nu_k\leq\paren{K+1}\sum_{k=0}^nv_k\). \\\\ On a montré : \(\sum_{k=0}^nu_k\egqd{n\to\pinf}\O{\sum_{k=0}^nv_k}\). \\
    \item Idem.
\end{itemize}
\end{dem}

\subsection{Théorème de comparaison par équivalence de séries à termes positifs}

\begin{theo}\thlabel{theo2.5}
Soient \(u,v\) deux suites réelles positives.

Si \(u\sim v\), alors les séries \(\sum u\) et \(\sum v\) sont de même nature ; l'une converge ssi l'autre converge.

De plus,

\begin{itemize}
    \item si les séries convergent, alors les restes partiels sont équivalents : \(\sum_{k=n+1}^{\pinf}u_k\simqd{n\to\pinf}\sum_{k=n+1}^{\pinf}v_k\) ; \\
    \item si les séries divergent, alors les sommes partielles divergent vers \(\pinf\) et sont équivalentes : \(\sum_{k=0}^nu_k\simqd{n\to\pinf}\sum_{k=0}^nv_k\).
\end{itemize}
\end{theo}

\begin{dem}
\begin{itemize}
    \item Si les séries convergent, comme \(u_n\simqd{n\to\pinf}v_n\), on a \[\quantifs{\forall\epsilon>0;\exists N\in\N;\forall n\geq N}\paren{1-\epsilon}v_n\leq u_n\leq\paren{1+\epsilon}v_n\] donc \[\quantifs{\forall n\geq N}\paren{1-\epsilon}\sum_{k=n+1}^{\pinf}v_k\leq\sum_{k=n+1}^{\pinf}u_k\leq\paren{1+\epsilon}\sum_{k=n+1}^{\pinf}v_k.\] Donc \(\sum_{k=n+1}^{\pinf}u_k\simqd{n\to\pinf}\sum_{k=n+1}^{\pinf}v_k\). \\
    \item Si les séries divergent, la démonstration est similaire à la \thref{demtheo2.4}.
\end{itemize}
\end{dem}

\begin{exo}
Soit \(a>0\). On pose \(u_n=\sin\dfrac{a^n}{n}\) pour \(n\in\Ns\). Selon la valeur de \(a\), déterminez la nature de la série \(\sum_{n\geq1}u_n\).

Montrez que si \(a=1\), alors \(\sum_{k=1}^nu_k\sim\ln n\) et si \(a<1\), \(\sum_{k=n}^{\pinf}u_k=\o{a^n}\).
\end{exo}

\begin{corr}
\begin{itemize}
    \item Si \(a=1\), on a \(u_n=\sin\dfrac{1}{n}\simqd{n\to\pinf}\dfrac{1}{n}\). \\\\ Or \(\sum\dfrac{1}{n}\) diverge donc par théorème de comparaison des séries à termes positifs, \(\sum u_n\) diverge. \\
    \item Si \(a<1\), on a \(u_n=\sin\dfrac{a^n}{n}\simqd{n\to\pinf}\dfrac{a^n}{n}\egqd{n\to\pinf}\o{a^n}\). \\\\ Or \(\sum a^n\) converge car \(a<1\) donc par théorème de comparaison des séries à termes positifs, \(\sum u_n\) converge.
\end{itemize}

De plus, d'après le \thref{theo2.5}, si \(a=1\), on a \[\sum_{k=1}^nu_k\simqd{n\to\pinf}\sum_{k=1}^n\dfrac{1}{n}\simqd{n\to\pinf}\ln n\] et d'après le \thref{theo2.3}, si \(a<1\), on a \[\begin{aligned}
\sum_{k=n+1}^{\pinf}u_k&\egqd{n\to\pinf}\o{\sum_{k=n+1}^{\pinf}a^k} \\
&\egqd{n\to\pinf}\o{\dfrac{a^{n+1}}{1-a}} \\
&\egqd{n\to\pinf}\o{a^n}.
\end{aligned}\]
\end{corr}

\subsection{Théorème de comparaison série - intégrale}

\begin{prop}
Soit \(f\) une fonction continue, positive et décroissante sur \(\Rp\).

Alors la série de terme général \(f\paren{n}\) et la suite de terme général \(\int_0^nf\) sont de même nature.
\end{prop}

\begin{dem}
\(f\) étant décroissante et continue sur \(\Rp\), pour \(k\in\Ns\), \(\quantifs{\tpt t\in\intervii{k-1}{k}}f\paren{k}\leq f\paren{t}\).

Donc \(\int_{k-1}^kf\paren{k}\odif{t}\leq\int_{k-1}^kf\paren{t}\odif{t}\) \ie \(f\paren{k}\leq\int_{k-1}^kf\paren{t}\odif{t}\).

De même, \(\int_k^{k+1}f\paren{t}\odif{t}\leq f\paren{k}\).

Donc \(\quantifs{\tpt n\in\Ns}\sum_{k=1}^n\int_k^{k+1}f\paren{t}\odif{t}\leq\sum_{k=1}^nf\paren{k}\leq\sum_{k=1}^n\int_{k-1}^kf\paren{t}\odif{t}\).

Donc \(\int_1^{n+1}f\paren{t}\odif{t}\leq\sum_{k=1}^nf\paren{k}\leq\int_0^nf\paren{t}\odif{t}\).

\(f\) étant positive, les suites \(\paren{\sum_{k=1}^nf\paren{k}}\) et \(\paren{\int_0^nf\paren{t}\odif{t}}\) sont croissantes.

Donc si \(\paren{\int_0^nf}\) converge, elle est majorée, donc \(\paren{\sum_{k=1}^nf\paren{k}}\) est majorée et donc convergente, \ie \(\sum f\paren{n}\) converge.

Réciproquement, si \(\sum f\paren{n}\) converge, \(\paren{\sum_{k=1}^nf\paren{k}}\) est majorée donc \(\paren{\int_1^{n+1}f}\) est majorée et croissante donc convergente. Donc \(\paren{\int_0^nf}\) converge.
\end{dem}

\begin{meth}[À retenir]
La technique d'encadrement des sommes partielles d'une série \(\sum f\paren{n}\) (ou des restes partiels) par des intégrales quand \(f\) est continue, positive et monotone.
\end{meth}

\begin{ex}~\\
\begin{itemize}
    \item \(\sum_{k=1}^n\dfrac{1}{k}\simqd{n\to\pinf}\ln n\) (à connaître). \\
    \item Si \(\alpha>1\), un équivalent simple de \(\sum_{k=n+1}^{\pinf}\dfrac{1}{k^\alpha}\) quand \(n\to\pinf\) est \(\dfrac{1}{\paren{\alpha-1}n^{\alpha-1}}\).
\end{itemize}
\end{ex}

\begin{rappel}
Si \(u_n\simqd{n\to\pinf}v_n\) et \(u_n\tendqd{n\to\pinf}\pinf\) ou \(0\), alors \[\ln u_n\simqd{n\to\pinf}\ln v_n.\]
\end{rappel}

\begin{dem}[Premier point]~\\
La fonction \(f:t\mapsto\dfrac{1}{t}\) est positive, décroissante et continue sur \(\intervie{1}{\pinf}\).

On a donc l'inégalité : \[\underbrace{\int_2^{n+1}\dfrac{1}{t}\odif{t}}_{=\ln\paren{n+1}-\ln2}\leq\sum_{k=2}^n\dfrac{1}{k}\leq\underbrace{\int_1^n\dfrac{1}{t}\odif{t}}_{=\ln n}.\]

Or on a \[\ln\paren{n+1}-\ln2\simqd{n\to\pinf}\ln\paren{n+1}\simqd{n\to\pinf}\ln n\] donc d'après le théorème d'encadrement, on a \[\sum_{k=2}^n\dfrac{1}{k}\simqd{n\to\pinf}\ln n.\]
\end{dem}

\begin{dem}[Deuxième point]
Soit \(\alpha>1\).

La fonction \(f:t\mapsto\dfrac{1}{t^\alpha}\) est continue, décroissante et positive sur \(\intervie{1}{\pinf}\).

On a \[\int_{n+1}^{\pinf}f\paren{t}\odif{t}\leq\sum_{k=n+1}^{\pinf}f\paren{k}\leq\underbrace{\int_n^{\pinf}f\paren{t}\odif{t}}_{=\lim_{X\to\pinf}\int_n^Xf\paren{t}\odif{t}}.\]

Soient \(X\geq n\geq1\).

On a \[\int_{n+1}^{X+1}\dfrac{1}{t^\alpha}\odif{t}\leq\sum_{k=n+1}^X\dfrac{1}{k^\alpha}\leq\int_n^X\dfrac{1}{t^\alpha}\odif{t}.\]

Or \[\int_n^X\dfrac{1}{t^\alpha}\odif{t}=\croch{\dfrac{1}{-\alpha+1}t^{-\alpha+1}}_n^X=\dfrac{1}{1-\alpha}\paren{\dfrac{1}{X^{\alpha-1}}-\dfrac{1}{n^{\alpha-1}}}.\]

Donc \[\dfrac{1}{\alpha-1}\paren{\dfrac{1}{\paren{n+1}^{\alpha-1}}-\dfrac{1}{\paren{X+1}^{\alpha-1}}}\leq\sum_{k=n+1}^X\dfrac{1}{k^\alpha}\leq\dfrac{1}{\alpha-1}\paren{\dfrac{1}{n^{\alpha-1}}-\dfrac{1}{X^{\alpha-1}}}.\]

Or \(\dfrac{1}{X^{\alpha-1}}\tendqd{X\to\pinf}0\) et \(\dfrac{1}{\paren{X+1}^{\alpha-1}}\tendqd{X\to\pinf}0\).

Donc, par passage à la limite quand \(X\to\pinf\), on a \[\dfrac{1}{\paren{\alpha-1}\paren{n+1}^{\alpha-1}}\leq\sum_{k=n+1}^{\pinf}\dfrac{1}{k^\alpha}\leq\dfrac{1}{\paren{\alpha-1}n^{\alpha-1}}.\]

Or \(\dfrac{1}{\paren{n+1}^{\alpha-1}}\simqd{n\to\pinf}\dfrac{1}{n^{\alpha-1}}\) donc par encadrement, on a \[\sum_{k=n+1}^{\pinf}\dfrac{1}{k^\alpha}\simqd{n\to\pinf}\dfrac{1}{\paren{\alpha-1}n^{\alpha-1}}.\]
\end{dem}

\begin{exo}
Pour \(n\in\Ns\), on pose \(u_n=\sum_{k=n}^{\pinf}\dfrac{\ln k}{k^2}\).

Justifiez l'existence de \(u_n\), puis montrez la divergence de la série \(\sum u_n\).

Montrez que \(\sum_{k=1}^nu_k\simqd{n\to\pinf}\dfrac{\ln^2n}{2}\).
\end{exo}

\begin{corr}
Pour \(k\geq1\), on a \[0\leq\dfrac{\ln k}{k^2}=\dfrac{\ln k}{k^{\nicefrac{1}{2}}\times k^{\nicefrac{3}{2}}}=\dfrac{\ln k}{k^{\nicefrac{1}{2}}}\times\dfrac{1}{k^{\nicefrac{3}{2}}}.\]

Or \(\dfrac{\ln k}{k^{\nicefrac{1}{2}}}\tendqd{k\to\pinf}0\) donc \(\dfrac{\ln k}{k^{\nicefrac{1}{2}}}\egqd{k\to\pinf}\o{\dfrac{1}{k^{\nicefrac{3}{2}}}}\).

Or \(\sum\dfrac{1}{k^{\nicefrac{3}{2}}}\) converge donc d'après le théorème de comparaison des séries à termes positifs, \(\sum\dfrac{\ln k}{k^2}\) converge.

Donc \(\quantifs{\tpt n\geq1}\sum_{k=n}^{\pinf}\dfrac{\ln k}{k^2}\) existe.

La fonction \(f:t\mapsto\dfrac{\ln t}{t^2}\) est de classe \(\classe{\infty}\) et positive sur \(\intervie{1}{\pinf}\).

On a \(f\prim:t\mapsto\dfrac{t-2t\ln t}{t^4}\).

Or, pour \(t\geq2\), on a \(t\leq2t\ln t\) et \(t^4\geq0\) donc \[\quantifs{\forall t\in\intervie{2}{\pinf}}f\prim\paren{t}\leq0\] donc \(f\) est décroissante sur \(\intervie{2}{\pinf}\).

Soient \(X\geq n\geq3\).

On a \[\int_n^{X+1}\dfrac{\ln t}{t^2}\odif{t}\leq\sum_{k=n}^X\dfrac{\ln k}{k^2}\leq\int_{n-1}^X\dfrac{\ln t}{t^2}\odif{t}.\]

Or, par intégration par parties, on a \[\int_{n-1}^X\dfrac{\ln t}{t^2}\odif{t}=\dfrac{\ln\paren{n-1}+1}{n-1}-\dfrac{\ln X+1}{X}.\]

Donc \[\dfrac{\ln n+1}{n}-\dfrac{\ln\paren{X+1}+1}{X+1}\leq\sum_{k=n}^X\dfrac{\ln k}{k^2}\leq\dfrac{\ln\paren{n-1}+1}{n-1}-\dfrac{\ln X+1}{X}.\]

Donc, par passage à la limite quand \(X\to\pinf\), on a \[\dfrac{\ln n+1}{n}\leq u_n\leq\dfrac{\ln\paren{n-1}+1}{n-1}.\]

Or \(\ln\paren{n-1}\simqd{n\to\pinf}\ln n\) donc \[\dfrac{\ln\paren{n-1}+1}{n-1}\simqd{n\to\pinf}\dfrac{\ln n+1}{n}\simqd{n\to\pinf}\dfrac{\ln n}{n}.\]

D'où \(u_n\simqd{n\to\pinf}\dfrac{\ln n}{n}\).

Pour \(n\geq3\), on a \(\dfrac{\ln n}{n}\geq\dfrac{1}{n}\geq0\).

Or \(\sum\dfrac{1}{n}\) diverge donc d'après le théorème de comparaison des séries à termes positifs, \(\sum\dfrac{\ln n}{n}\) diverge et donc \(\sum u_n\) diverge.

De plus, d'après le \thref{theo2.5}, on a \[\sum_{k=1}^nu_k\simqd{n\to\pinf}\sum_{k=1}^n\dfrac{\ln k}{k}.\]

La fonction \(g:t\mapsto\dfrac{\ln t}{t}\) est de classe \(\classe{\infty}\), positive et décroissante sur \(\intervie{3}{\pinf}\).

On pose \(n\geq4\). On a \[\int_4^{n+1}\dfrac{\ln t}{t}\odif{t}\leq\sum_{k=4}^ng\paren{k}\leq\int_3^n\dfrac{\ln t}{t}\odif{t}.\]

Or \[\int_3^n\dfrac{\ln t}{t}\odif{t}=\int_3^n\dfrac{1}{t}\ln t\odif{t}=\croch{\dfrac{\ln^2t}{2}}_3^n=\dfrac{\ln^2n}{2}-\dfrac{\ln^23}{2}.\]

Donc \[\underbrace{\dfrac{\ln^2\paren{n+1}}{2}+K_2}_{\simqd{n\to\pinf}\dfrac{\ln^2n}{2}}\leq\sum_{k=4}^n\dfrac{\ln k}{k}\leq\underbrace{\dfrac{\ln^2n}{2}+K_1}_{\simqd{n\to\pinf}\dfrac{\ln^2n}{2}}.\]

Par encadrement, on a \(\sum_{k=4}^nu_k\simqd{n\to\pinf}\dfrac{\ln^2n}{2}\) puis \(\sum_{k=1}^nu_k\simqd{n\to\pinf}\dfrac{\ln^2n}{2}\).
\end{corr}

\section{Séries absolument convergentes}

\begin{defi}
Soit \(u\) une suite de \(E\).

On dit que la série \(\sum u\) est absolument convergente ssi la série à termes positifs \(\sum\norme{u}\) est convergente.
\end{defi}

\subsection{Lien entre absolue convergence et convergence}

\begin{theo}
Si \(E\) est de dimension finie, alors toute série absolument convergente est convergente.
\end{theo}

\begin{dem}
Soient \(E\) un espace vectoriel normé de dimension finie \(p\in\N\) et \(\fami{B}\) une base de \(E\).

On choisit la norme sup \(\norme{}_\infty\) associée à cette base.

Toutes les normes sont équivalentes sur \(E\) donc si \(\norme{}\) est une norme telle que \(\sum\norme{u_n}\) converge, alors \(\sum\norme{u_n}_\infty\) converge d'après le théorème de comparaison des séries à termes positifs.

On note \(\paren{u_{n,1},\dots,u_{n,p}}\) les coordonnées de \(u_n\) dans \(\fami{B}\).

On a \(\quantifs{\forall k\in\interventierii{1}{p}}\abs{u_{n,k}}\leq\norme{u_n}_\infty\) donc d'après le théorème de comparaison des séries à termes positifs, \(\sum_n\abs{u_{n,k}}\) converge, donc \(\sum_nu_{n,k}\) est absolument convergente et donc convergente.

Donc les suites des sommes partielles \(\paren{\sum_{j=0}^nu_j}=\paren{\paren{\sum_{j=0}^nu_{j,1},\dots,\sum_{j=0}^nu_{j,p}}}\) convergent \ie \(\sum u_n\) converge.
\end{dem}

\begin{rem}
\begin{itemize}
    \item La réciproque est fausse : la série \(\sum_{n\geq1}\dfrac{\paren{-1}^n}{n}\) converge (on l'appelle la série harmonique alternée) mais ne converge pas absolument. \\
    \item L'hypothèse de la dimension finie est indispensable. En dimension infinie, ce résultat est faux en général.
\end{itemize}
\end{rem}

\begin{exo}
Soit \(x>0\). Montrez que les séries suivantes convergent : \[\sum_{n\geq2}\dfrac{\ln\paren{n^2+\paren{-1}^nn}}{n^2+\paren{-1}^nx^n}\qquad\sum_{n\geq0}\sqrt{n}\cos\paren{x}\sin^n\paren{x}\qquad\sum_{n\geq0}\dfrac{\paren{-1}^n\sqrt{n+x}}{x^n+n^{\nicefrac{2}{x}}}.\]
\end{exo}

\subsection{Un exemple fondamental : l'exponentielle de matrice}

Soit \(p\in\Ns\).

On choisit comme norme sur \(E=\M{p}[\C]\) une norme sous-multiplicative.

Alors pour tout \(n\in\Ns\), \(\norme{A^n}\leq\norme{A}^n\), donc \(\norme{\dfrac{A^n}{n!}}\leq\dfrac{\norme{A}^n}{n!}\).

Or la série \(\sum\dfrac{\norme{A}^n}{n!}\) converge (et sa somme vaut \(\exp\norme{A}\)), donc par comparaison de séries à termes positifs, la série \(\sum\dfrac{A^n}{n!}\) est absolument convergente.

On pose alors \(\exp A=\sum_{n=0}^{\pinf}\dfrac{A^n}{n!}\).

\subsection{Extension des résultats par comparaison}

\begin{defi}
Soit \(u\) une suite de \(E\) et \(v\) une suite réelle positive.

On dit que \(u=\O{v}\) quand \(\quantifs{\exists M>0;\exists n_0\in\N;\forall n\geq n_0}\norme{u_n}\leq Mv_n\).

On dit que \(u=\o{v}\) quand \(\quantifs{\forall\epsilon>0;\exists n_0\in\N;\forall n\geq n_0}\norme{u_n}\leq\epsilon v_n\).
\end{defi}

\begin{prop}
Soient \(u\) une suite de \(E\) et \(v\) une suite réelle positive.

Si \(E\) est de dimension finie, \(u_n=\O{v_n}\) quand \(n\) tend vers \(\pinf\) et la série \(\sum v\) converge, alors la série \(\sum u\) est absolument convergente.

De plus, \(\sum_{k=n+1}^{\pinf}u_k=\O{\sum_{k=n+1}^{\pinf}v_k}\).
\end{prop}

\begin{dem}
On a \(u_n\egqd{n\to\pinf}\O{v_n}\ssi\norme{u_n}\egqd{n\to\pinf}\O{v_n}\).

Si \(\sum v_n\) converge alors \(\sum\norme{u_n}\) converge d'après le théorème de comparaison des séries à termes positifs donc, comme \(E\) est de dimension finie, \(\sum u_n\) converge.

De plus, on a \(\norme{\sum_{k=n+1}^{\pinf}u_k}\leq\sum_{k=n+1}^{\pinf}\norme{u_k}\) donc d'après le \thref{theo2.3}, on a \[\sum_{k=n+1}^{\pinf}\norme{u_k}\egqd{n\to\pinf}\O{\sum_{k=n+1}^{\pinf}v_k}.\]

Donc \[\norme{\sum_{k=n+1}^{\pinf}u_k}\egqd{n\to\pinf}\O{\sum_{k=n+1}^{\pinf}v_k}.\]
\end{dem}

Ceci est encore valable si \(u_n=\o{v_n}\).

\begin{prop}
Soient \(u\) une suite de \(E\) et \(v\) une suite réelle positive.

Si \(E\) est de dimension finie, \(u_n=\o{v_n}\) quand \(n\) tend vers \(\pinf\) et la série \(\sum v\) converge, alors la série \(\sum u\) est absolument convergente.

De plus, \(\sum_{k=n+1}^{\pinf}u_k=\o{\sum_{k=n+1}^{\pinf}v_k}\).
\end{prop}

\begin{dem}
Idem.
\end{dem}

\subsection{Produit de Cauchy de deux séries absolument convergentes}

\begin{defi}
Une \(\K\)-algèbre est un \(\K\)-espace vectoriel muni d'un produit interne bilinéaire.
\end{defi}

\begin{defi}
Soient \(E\) une algèbre normée de dimension finie, \(\sum_{n\geq0}a_n\) et \(\sum_{n\geq0}b_n\) deux séries à termes dans \(E\).

On appelle produit de Cauchy des deux séries la série \(\sum_{n\geq0}c_n\) où \(\quantifs{\tpt n\in\N}c_n=\sum_{k=0}^na_kb_{n-k}\).
\end{defi}

\begin{rem}
Quand les séries ne commencent pas à partir du rang \(0\), il faut se méfier ! Une idée simple est de se ramener au cas précédent en décalant les indices.

Exemple très courant : les séries commencent au rang \(1\). Dans ce cas, le produit de Cauchy des séries \(\sum_{n\geq1}a_n\) et \(\sum_{n\geq1}b_n\) est la série \(\sum_{n\geq1}c_n\) où \(\quantifs{\tpt n\in\Ns}c_n=\sum_{k=1}^{n}a_kb_{n+1-k}\).
\end{rem}

\begin{theo}
Avec les mêmes hypothèses sur \(E\).

Si les séries \(\sum_{n\geq0}a_n\) et \(\sum_{n\geq0}b_n\) convergent absolument, alors leur produit de Cauchy est aussi absolument convergent et \[\sum_{n=0}^{\pinf}c_n=\sum_{n=0}^{\pinf}a_n\times\sum_{k=0}^{\pinf}b_n.\]
\end{theo}

\begin{dem}[Premier cas : \(E=\R\) et \(\paren{a_n},\paren{b_n}\) positives]~\\
On pose \(A_n=\sum_{k=0}^na_k\), \(B_n=\sum_{k=0}^nb_k\) et \(C_n=\sum_{k=0}^nc_k\).

On a \(A_nB_n=\sum_{k=0}^na_k\times\sum_{k=0}^nb_k=\sum_{\substack{0\leq k\leq n \\ 0\leq l\leq n}}a_kb_l\).

On a \(\paren{\star}~C_n\leq A_nB_n\leq C_{2n}\).

Les séries \(\sum a_n\) et \(\sum b_n\) convergent donc les suites \(\paren{A_n}\) et \(\paren{B_n}\) convergent et sont donc majorées.

Comme ce sont des suites positives, la suite \(\paren{A_nB_n}\) est majorée (produit d'inégalités entre positifs).

Donc \(\paren{C_n}\), qui est croissante, est majorée donc converge \ie \(\sum c_n\) converge.

Donc \(A_n\tendqd{n\to\pinf}\sum_{k=0}^{\pinf}a_k\), \(B_n\tendqd{n\to\pinf}\sum_{k=0}^{\pinf}b_k\), \(C_n\tendqd{n\to\pinf}\sum_{k=0}^{\pinf}c_k\) et \(C_{2n}\tendqd{n\to\pinf}\sum_{k=0}^{\pinf}c_k\).

Donc, par passage à la limite dans \(\paren{\star}\), on a \[\sum_{k=0}^{\pinf}c_k=\sum_{k=0}^{\pinf}a_k\times\sum_{k=0}^{\pinf}b_k.\]
\end{dem}

\begin{dem}[Cas général]
On a \(\dim E\in\N\) donc toutes les normes sont équivalentes sur \(E\) ; on en choisit donc une qui est sous-multiplicative.

Avec les mêmes notations, on a \(A_nB_n-C_n=\sum_{k=0}^n\sum_{j=n-k+1}^nc_j\), d'où \[\begin{aligned}
\norme{A_nB_n-C_n}&\leq\sum_{k=0}^n\sum_{j=n-k+1}^n\norme{c_j} \\
&=\sum_{k=0}^n\sum_{j=n-k+1}\norme{\sum_{l=0}^ja_{j-l}b_l} \\
&\leq\sum_{k=0}^n\sum_{j=n-k+1}^n\sum_{l=0}^j\norme{a_{j-l}b_l} \\
&\leq\sum_{k=0}^n\sum_{j=n-k+1}^n\sum_{l=0}^j\norme{a_{j-l}}\times\norme{b_l}.
\end{aligned}\]

On pose \(A_n\prim=\sum_{k=0}^n\norme{a_k}\), \(B_n\prim=\sum_{k=0}^n\norme{b_k}\) et \(C_n\prim=\sum_{k=0}^nc_k\prim=\sum_{k=0}^n\sum_{j=0}^k\norme{a_{k-j}}\times\norme{b_j}\).

On a de même \(\norme{A_nB_n-C_n}\leq A_n\prim B_n\prim-C_n\prim\).

D'après le premier cas, \(\paren{C_n\prim}\) converge vers \(\sum_{k=0}^{\pinf}\norme{a_k}\times\sum_{k=0}^{\pinf}\norme{b_k}=\lim_nA_n\prim B_n\prim\).

Donc \(A_n\prim B_n\prim-C_n\prim\tendqd{n\to\pinf}0\) donc par encadrement \(A_nB_n-C_n\tendqd{n\to\pinf}0\).

Or \(A_n\tendqd{n\to\pinf}\sum_{k=0}^{\pinf}a_k\) et \(B_n\tendqd{n\to\pinf}\sum_{k=0}^{\pinf}b_k\) donc \[C_n\tendqd{n\to\pinf}\sum_{k=0}^{\pinf}a_k\times\sum_{k=0}^{\pinf}b_k.\]
\end{dem}

\section{Séries alternées}

\begin{defi}
Une série alternée est une série réelle \(\sum u_n\) telle que \(\quantifs{\tpt n\in\N}u_{n+1}\text{ est de signe opposé à }u_n\).
\end{defi}

En général, les séries alternées sont reconnaissables à la présence d'un facteur \(\paren{-1}^n\) dans l'expression du terme général.

On dispose d'une condition suffisante de convergence d'une série alternée qu'on appelle le critère spécial des séries alternées.

\begin{theo}
Soit \(\sum\paren{-1}^nu_n\) une série alternée.

Si la suite \(u\)

\begin{itemize}
    \item est positive, \\
    \item est décroissante, \\
    \item et converge vers \(0\),
\end{itemize}

alors la série \(\sum\paren{-1}^nu_n\) converge.

Dans ce cas, la somme de la série est positive, et si on note \(R_n=\sum_{k=n+1}^{\pinf}\paren{-1}^ku_k\) le reste partiel d'indice \(n\), alors \(\quantifs{\tpt n\in\N}R_n\text{ est du signe de son premier terme (\ie du signe de }\paren{-1}^{n+1}\text{) et }\abs{R_n}\leq u_{n+1}\leq u_n\).
\end{theo}

\begin{ex}
\begin{itemize}
    \item La série harmonique alternée \(\sum_{n\geq1}\dfrac{\paren{-1}^n}{n}\) converge. \\
    \item La série \(\sum_{n\geq2}\dfrac{\paren{-1}^n}{\ln n}\) converge.
\end{itemize}
\end{ex}

\begin{rem}
\begin{itemize}
    \item Si \(\sum_{n\geq n_0}\paren{-1}^nu_n\) est une série alternée convergente, sa somme a le signe du premier terme de la série (ici le signe de \(\paren{-1}^{n_0}u_{n_0}\)). \\
    \item La condition de décroissance de la suite \(u\) est essentielle ! Contre-exemple : la série \(\sum_{n\geq2}\dfrac{\paren{-1}^n}{\paren{-1}^n+\sqrt{n}}\) est une série alternée divergente. \\ De plus, cela fournit un contre-exemple au théorème de comparaison par équivalents si on ne tient pas compte de la condition sur le signe, qui doit être constant.
\end{itemize}
\end{rem}

\begin{exo}
Soit \(\alpha>1\). Pour \(n\in\Ns\), on pose \(u_n=\sum_{k=n}^{\pinf}\dfrac{\paren{-1}^k}{k^\alpha+k}\).

Justifiez l'existence de \(u_n\). Montrez que la série \(\sum_{n\geq1}u_n\) converge.
\end{exo}

\begin{corr}~\\
La suite \(\paren{\dfrac{1}{k^\alpha+k}}_{k\in\Ns}\) est positive, décroissante et converge vers \(0\) donc d'après le critère spécial des séries alternées, \(\sum\dfrac{\paren{-1}^k}{k^\alpha+k}\) converge.

Ainsi, \(u_n=\sum_{k=n}^{\pinf}\dfrac{\paren{-1}^k}{k^\alpha+k}\) existe pour \(n\in\Ns\).

De plus, on a \[\abs{u_n}=\abs{\sum_{k=n}^{\pinf}\dfrac{\paren{-1}^k}{k^\alpha+k}}\leq\dfrac{1}{n^\alpha+n}\leq\dfrac{1}{n^\alpha}.\]

Or \(\alpha>1\) donc d'après le théorème de comparaison des séries à termes positifs, \(\sum\abs{u_n}\) converge et donc \(\sum u_n\) converge.
\end{corr}


\chapter{Familles sommables}

\minitoc

\note{À venir}


\chapter{Rappels et compléments d'algèbre linéaire}

\minitoc

Dans tout ce chapitre, \(\K\) désigne un sous-corps de \(\C\), en général \(\R\) ou \(\C\).

\section{Sommes de sous-espaces vectoriels}

\subsection{Généralités}

\begin{defi}
Soient \(E\) un \(\K\)-espace vectoriel et \(F_1,\dots,F_n\) des sous-espaces vectoriels de \(E\).

On appelle somme de \(F_1,\dots,F_n\) l'ensemble noté \(F_1+\dots+F_n\) : \[F_1+\dots+F_n=\accol{\sum_{i=1}^nx_i\tq\paren{x_1,\dots,x_n}\in F_1\times\dots\times F_n}.\]
\end{defi}

\begin{prop}
Soit \(\paren{u_1,\dots,u_p,u_{p+1},\dots,u_n}\in E^n\).

Alors \(\Vect{u_1,\dots,u_p,u_{p+1},\dots,u_n}=\Vect{u_1,\dots,u_p}+\Vect{u_{p+1},\dots,u_n}\).
\end{prop}

\begin{prop}
Avec les mêmes notations : \(F_1+\dots+F_n\) est un sous-espace vectoriel de \(E\).

De plus, c'est le plus petit sous-espace vectoriel qui contient \(F_1,\dots,F_n\).

Si on connaît des familles génératrices de chacun des sous-espaces vectoriels \(F_1,\dots,F_n\), alors en concaténant ces familles, on obtient une famille génératrice de \(F_1+\dots+F_n\).
\end{prop}

Conséquence : en fractionnant une famille génératrice de \(E\) en sous-familles, on décompose l'espace \(E\) en une somme de sous-espaces vectoriels.

\subsection{Sommes directes}

\begin{defi}
Soient \(E\) un \(\K\)-espace vectoriel et \(F_1,\dots,F_n\) des sous-espaces vectoriels de \(E\).

On dit que la somme \(F_1+\dots+F_n\) est directe quand tout vecteur de \(F_1+\dots+F_n\) a une unique écriture \(\sum_{i=1}^nx_i\) où \(\paren{x_1,\dots,x_n}\in F_1\times\dots\times F_n\).
\end{defi}

On dit aussi que les sous-espaces sont en somme directe. Dans ce cas, quand on veut insister sur cette propriété, on note la somme sous la forme \(F_1\oplus\dots\oplus F_n=\bigoplus_{i=1}^nF_i\).

\begin{prop}
Avec les mêmes hypothèses.

La somme \(F_1+\dots+F_n\) est directe ssi le vecteur nul a une unique décomposition \(\sum_{i=1}^nx_i\) où \(\paren{x_1,\dots,x_n}\in F_1\times\dots\times F_n\), qui est la décomposition triviale.

Autrement dit, la somme \(F_1+\dots+F_n\) est directe ssi la seule solution de l'équation \(\sum_{i=1}^nx_i=0\) d'inconnue \(\paren{x_1,\dots,x_n}\in F_1\times\dots\times F_n\) est le \(n\)-uplet nul.
\end{prop}

\begin{dem}
\impdir

Immédiat : le vecteur nul appartient à \(F_1+\dots+F_n\) et a une unique écriture sous la forme \(\sum_{i=1}^nx_i\) où \(\paren{x_1,\dots,x_n}\in F_1\times\dots\times F_n\).

Or on en connaît une : \(0=0+\dots+0\).

Donc la seule solution à l'équation \(\sum_{i=1}^nx_i=0\) d'inconnue \(\paren{x_1,\dots,x_n}\in F_1\times\dots\times F_n\) est la solution triviale \(\paren{0,\dots,0}\).

\imprec

Soit \(z\in F_1+\dots+F_n\).

On veut montrer que \(z\) a une unique écriture \(z=\sum_{i=1}^nx_i\) où \(\paren{x_1,\dots,x_n}\in F_1\times\dots\times F_n\).

Si \(z=\sum_{i=1}^nx_i=\sum_{i=1}^nx_i\prim\) où \(\paren{x_1,\dots,x_n},\paren{x_1\prim,\dots,x_n\prim}\in F_1\times\dots\times F_n\), alors \(\sum_{i=1}^n\paren{x_i-x_i\prim}=0\).

Or \(\quantifs{\tpt i\in\interventierii{1}{n}}x_i-x_i\prim\in F_i\) car \(F_i\) est un sous-espace vectoriel.

Ainsi, \(\quantifs{\tpt i\in\interventierii{1}{n}}x_i-x_i\prim=0\) \ie \(\paren{x_1,\dots,x_n}=\paren{x_1\prim,\dots,x_n\prim}\).

D'où l'unicité voulue.
\end{dem}

Un exemple fondamental : si \(\paren{v_1,\dots,v_n}\) est une famille libre, alors les droites vectorielles \(\Vect{v_i}\) sont en somme directe.

\begin{prop}
Avec les mêmes hypothèses.

Si la somme \(F_1+\dots+F_n\) est directe, alors en concaténant des familles libres de chacun des sous-espaces vectoriels, on obtient une famille libre.
\end{prop}

\begin{dem}
Si \(F_1+\dots+F_n\) est une somme directe alors soient \(L_1,\dots,L_n\) des familles libres de \(F_1,\dots,F_n\) respectivement.

On note \(L_i=\paren{e_{i\,1},\dots,e_{i\,l_i}}\) où \(i\in\interventierii{1}{n}\).

Soit \(\paren{\lambda_{i\,j}}_{\substack{1\leq i\leq n \\ 1\leq j\leq l_i}}\) une famille de scalaires telle que \(\sum_{\substack{1\leq i\leq n \\ 1\leq j\leq l_i}}\lambda_{i\,j}e_{i\,j}=0\).

Alors \(\sum_{i=1}^n\sum_{j=1}^{l_i}\lambda_{i\,j}e_{i\,j}=0\).

Or \(F_1+\dots+F_n\) est directe donc \(\quantifs{\tpt i\in\interventierii{1}{n}}\sum_{j=1}^{l_i}\lambda_{i\,j}e_{i\,j}=0\).

Or \(L_i\) est libre donc \(\lambda_{i\,1}=\dots=\lambda_{i\,l_i}=0\).

Donc la concaténation de \(L_1,\dots,L_n\) est libre.
\end{dem}

\subsection{Sous-espaces supplémentaires}

\begin{defi}
Soient \(E\) un \(\K\)-espace vectoriel et \(F_1,\dots,F_n\) des sous-espaces vectoriels de \(E\).

On dit que les sous-espaces \(F_1,\dots,F_n\) sont supplémentaires (dans \(E\)) quand \(E=\bigoplus_{i=1}^nF_i\).
\end{defi}

On déduit des deux parties précédentes le résultat de la décomposition d'un vecteur.

\begin{prop}
Soient \(E\) un \(\K\)-espace vectoriel et \(F_1,\dots,F_n\) des sous-espaces vectoriels de \(E\).

Il y a équivalence entre :

\begin{itemize}
    \item les sous-espaces \(F_1,\dots,F_n\) sont supplémentaires \\
    \item tout vecteur de \(E\) peut s'écrire de façon unique comme somme de vecteurs des sous-espaces vectoriels \(F_1,\dots,F_n\) : \[\quantifs{\forall v\in E;\exists!\paren{v_1,\dots,v_n}\in F_1\times\dots\times F_n}v=\sum_{i=1}^nv_i.\]
\end{itemize}
\end{prop}

Dans ce cas, soit \(v\) un vecteur de \(E\). Il existe un unique \(n\)-uplet \(\paren{v_1,\dots,v_n}\in F_1\times\dots\times F_n\) tel que \(v=\sum_{i=1}^nv_i\).

Pour tout \(j\in\interventierii{1}{n}\), on définit \(p_j:E\to E\) en posant \(p_j\paren{v}=v_j\).

Alors les applications \(p_j\) sont des projecteurs qui vérifient les propriétés :

\begin{itemize}
    \item \(\sum_{i=1}^np_i=\id{E}\) \\
    \item \(\quantifs{\tpt\paren{i,j}\in\interventierii{1}{n}^2}p_i\rond p_j=\delta_{i\,j}p_i\) où \(\delta_{x\,y}\) est le symbole de Kronecker.
\end{itemize}

La réciproque est vraie : si \(\paren{p_1,\dots,p_n}\) sont \(n\) projecteurs vérifiant les deux propriétés précédentes, alors les sous-espaces \(\paren{\Im p_i}_{1\leq i\leq n}\) sont supplémentaires.

\begin{dem}
Soient \(F_1,\dots,F_n\) supplémentaires dans \(E\).

\begin{itemize}
    \item \(\quantifs{\Tpt j\in\interventierii{1}{n}}p_j\) est un projecteur : \\\\ Soit \(v\in E\). On écrit \(v=v_1+\dots+v_j+\dots+v_n\). \\\\ Par définition, on a \(p_j\paren{v}=v_j=0+\dots+0+v_j+0+\dots+0\). \\\\ Donc, par définition, on a \(p_j\paren{p_j\paren{v}}=p_j\paren{v_j}=v_j\). \\\\ Donc \(p_j^2=p_j\). \\\\ Donc \(p\) est un projecteur (la linéarité est évidente). \\
    \item \(\quantifs{\Tpt v\in E}v=v_1+\dots+v_n=p_1\paren{v}+\dots+p_n\paren{v}\) donc \(\sum_{i=1}^np_i=\id{E}\). \\
    \item Montrons que \(\quantifs{\tpt\paren{i,j}\in\interventierii{1}{n}^2}p_ip_j=\delta_{i\,j}p_i\). \\\\ Si \(i=j\), c'est vrai car \(p_i\) est un projecteur. \\\\ Si \(i\not=j\), soit \(v\in E\). On écrit \(v=v_1+\dots+v_n\). \\\\ Par définition, \(p_j\paren{v}=v_j=0+\dots+0+v_j+0+\dots+0\). \\\\ D'où \(p_ip_j\paren{v}=0\).
\end{itemize}

Réciproquement, montrons que si \(p_1,\dots,p_n\) sont des projecteurs tels que \(\sum_{i=1}^np_i=\id{E}\) et \(\quantifs{\forall\paren{i,j}\in\interventierii{1}{n}^2}p_ip_j=\delta_{i\,j}p_i\), alors \(E=\bigoplus_{i=1}^n\Im p_i\).

\begin{itemize}
    \item Montrons que \(\Im p_1+\dots+\Im p_n\) est directe. \\\\ Soit \(\paren{v_1,\dots,v_n}\in\Im p_1\times\dots\times\Im p_n\) telle que \(v_1+\dots+v_n=0\). \\\\ Pour \(i\in\interventierii{1}{n}\), on a \[\begin{aligned}
        p_i\paren{0}=0&=p_i\paren{v_1}+\dots+p_i\paren{v_n} \\
        &=p_ip_1\paren{v_1}+\dots+p_ip_i\paren{v_i}+\dots+p_ip_n\paren{v_n} \\
        &=v_i. \\
    \end{aligned}\] Donc \(v_1=\dots=v_n=0\). \\\\ Donc la somme est directe. \\
    \item On a \(\sum_{i=1}^np_i=\id{E}\) donc \(\quantifs{\tpt v\in E}v=\sum_{i=1}^np_i\paren{v}\). \\\\ Donc \(v\in\Im p_1+\dots+\Im p_n\).
\end{itemize}

Finalement, \(E=\bigoplus_{i=1}^n\Im p_i\).
\end{dem}

\begin{exo}
Soit \(E=\F{\C}{\C}\). Pour \(k\in\accol{0,1,2}\), on pose \(E_k=\accol{f\in E\tq\quantifs{\forall x\in\C}f\paren{jx}=j^kf\paren{x}}\).

Montrez que \(E_0,E_1,E_2\) sont trois sous-espaces vectoriels de \(E\) supplémentaires.
\end{exo}

\begin{corr}
\begin{itemize}
    \item Soit \(k\in\accol{0,1,2}\). \(E_k\) contient l'application nulle. \\\\ Soient \(\paren{f,g}\in E_k^2\), \(\lambda\in\C\) et \(x\in\C\). \\\\ On a \[\begin{aligned}
        \paren{\lambda f+g}\paren{jx}&=\lambda f\paren{jx}+g\paren{jx} \\
        &=\lambda j^kf\paren{x}+j^kg\paren{x} \\
        &=j^k\paren{\lambda f\paren{x}+g\paren{x}} \\
        &=j^k\paren{\lambda f+g}\paren{x}. \\
    \end{aligned}\] Donc \(\lambda f+g\in E_k\) \ie \(E_k\) est un sous-espace vectoriel de \(E\). \\
    \item Montrons que \(\quantifs{\forall f\in E;\exists!\paren{f_0,f_1,f_2}\in E_0\times E_1\times E_2}f=f_0+f_1+f_2\). \\\\ \analyse \\\\ Soient \(f\in E\) et \(\paren{f_0,f_1,f_2}\in E_0\times E_1\times E_2\) telles que \(f=f_0+f_1+f_2\). \\\\ Pour tout \(x\in\C\), on a \[f\paren{jx}=f_0\paren{jx}+f_1\paren{jx}+f_2\paren{jx}=f_0\paren{x}+jf_1\paren{x}+j^2f_2\paren{x}\] et \[f\paren{j^2x}=f_0\paren{x}+j^2f_1\paren{x}+jf_2\paren{x}.\] D'où \[\begin{dcases}
        f_0\paren{x}+f_1\paren{x}+f_2\paren{x}=f\paren{x} \\
        f_0\paren{x}+jf_1\paren{x}+j^2f_2\paren{x}=f\paren{jx} \\
        f_0\paren{x}+j^2f_1\paren{x}+jf_2\paren{x}=f\paren{j^2x}
    \end{dcases}\] Or \(\begin{vmatrix}
        1 & 1 & 1 \\
        1 & j & j^2 \\
        1 & j^2 & j^4
    \end{vmatrix}\) est un déterminant de Vandermonde et est donc non-nul. \\\\ En faisant \(L_1+L_2+L_3\), on obtient \[f_0\paren{x}=\dfrac{1}{3}\paren{f\paren{x}+f\paren{jx}+f\paren{j^2x}}.\] En faisant \(L_1+j^2L_2+jL_3\), on obtient \[f_1\paren{x}=\dfrac{1}{3}\paren{f\paren{x}+j^2f\paren{jx}+jf\paren{j^2x}}.\] En faisant \(L_1+jL_2+j^2L_3\), on obtient \[f_2\paren{x}=\dfrac{1}{3}\paren{f\paren{x}+jf\paren{jx}+j^2f\paren{j^2x}}.\] Ceci prouve l'unicité de la décomposition (si elle existe). \\\\ \synthese \\\\ On pose \(f_k:x\mapsto\dfrac{1}{3}\paren{f\paren{x}+j^{2k}f\paren{jx}+j^kf\paren{j^2x}}\) pour \(k\in\accol{0,1,2}\). \\\\ On a \(\quantifs{\forall x\in\C}f\paren{x}=f_0\paren{x}+f_1\paren{x}+f_2\paren{x}\) \ie \(f=f_0+f_1+f_2\). \\\\ De plus, on a \(f_0\paren{jx}=\dfrac{1}{3}\paren{f\paren{jx}+f\paren{j^2x}+f\paren{j^3x}}=f_0\paren{x}\) car \(j^3=1\) donc \(f_0\in E_0\). \\\\ On montre de même que \(f_1\in E_1\) et \(f_2\in E_2\). \\\\ \conclusion On a bien \(E=\bigoplus_{k=0}^2E_k\).
\end{itemize}
\end{corr}

\begin{prop}
Avec les mêmes hypothèses.

Si les sous-espaces \(F_1,\dots,F_n\) sont supplémentaires, alors en concaténant des bases de chacun des sous-espaces vectoriels, on obtient une base de \(E\).
\end{prop}

\subsection{Cas particulier de deux sous-espaces}

\begin{prop}
Soient \(E\) un \(\K\)-espace vectoriel et \(F,G\) deux sous-espaces vectoriels de \(E\).

La somme \(F+G\) est directe ssi \(F\inter G=\accol{0}\).
\end{prop}

Attention ! Il ne faut pas généraliser à trois ou plus sous-espaces. Même si \(F_1\inter F_2\inter F_3\) est le sous-espace nul, on ne peut pas conclure que la somme \(F_1+F_2+F_3\) est directe.

\subsection{Applications linéaires et sommes directes}

\begin{prop}
Soient \(E,F\) deux \(\K\)-espaces vectoriels, \(E_1,\dots,E_n\) des sous-espaces vectoriels supplémentaires dans \(E\) et \(f_1,\dots,f_n\) des applications linéaires de \(E_1,\dots,E_n\) dans \(F\) respectivement.

Alors il existe une unique application linéaire \(f\) de \(E\) dans \(F\) telle que \(\quantifs{\tpt i\in\interventierii{1}{n}}\restr{f}{E_i}=f_i\).
\end{prop}

Autrement dit, pour définir une application linéaire sur une somme directe de sous-espaces vectoriels, il suffit de la définir sur chacun des sous-espaces vectoriels.

\begin{dem}
\analyse

On suppose que \(f\) existe telle que \(\quantifs{\forall i\in\interventierii{1}{n}}\restr{f}{E_i}=f_i\).

Soit \(x\in E\).

Comme \(E=\bigoplus_{i=1}^nE_i\), il existe un unique \(n\)-uplet \(\paren{x_1,\dots,x_n}\in E_1\times\dots\times E_n\) tel que \(x=\sum_{i=1}^nx_i\).

\(f\) étant linéaire, on a \(f\paren{x}=\sum_{i=1}^nf\paren{x_i}\).

Or \(\quantifs{\tpt i\in\interventierii{1}{n}}x_i\in E_i\) et \(\restr{f}{E_i}=f_i\) donc \(f\paren{x_i}=f_i\paren{x_i}\).

Donc \(f\paren{x}=\sum_{i=1}^nf_i\paren{x_i}\).

L'analyse prouve l'unicité de \(f\) (si elle existe).

\synthese

Pour \(x\in E\) qu'on écrit \(\sum_{i=1}^nx_i\) où \(\paren{x_1,\dots,x_n}\in E_1\times\dots\times E_n\), on pose \(f\paren{x}=\sum_{i=1}^nf_i\paren{x_i}\).

\begin{itemize}
    \item Montrons que \(\quantifs{\tpt i\in\interventierii{1}{n}}\restr{f}{E_i}=f_i\). \\\\ Pour \(i\in\interventierii{1}{n}\), pour \(x\in E_i\), on a \(x=0+\dots+0+x+0+\dots+0\). \\\\ Donc \(f\paren{x}=f_1\paren{0}+\dots+f_{i-1}\paren{0}+f_i\paren{x}+f_{i+1}\paren{0}+\dots+f_n\paren{0}=f_i\paren{x}\). \\\\ Donc \(\restr{f}{E_i}=f_i\). \\
    \item Montrons que \(f\) est linéaire. \\\\ Soient \(\paren{x,y}\in E^2\) et \(\lambda\in\K\). \\\\ On écrit \(x=\sum_{i=1}^nx_i\) et \(y=\sum_{i=1}^ny_i\) où \(\paren{x_1,\dots,x_n},\paren{y_1,\dots,y_n}\in E_1\times\dots\times E_n\). \\\\ Alors \[\begin{aligned}
        f\paren{\lambda x+y}&=\sum_{i=1}^nf_i\paren{\lambda x_i+y_i} \\
        &=\sum_{i=1}^n\paren{\lambda f_i\paren{x_i}+f_i\paren{y_i}} \\
        &=\lambda\sum_{i=1}^nf_i\paren{x_i}+\sum_{i=1}^nf_i\paren{y_i} \\
        &=\lambda f\paren{x}+f\paren{y}.
    \end{aligned}\]
\end{itemize}
\end{dem}

\begin{exo}
Soient \(E_1,\dots,E_n\) des sous-espaces vectoriels supplémentaires. Quelles sont les applications qui induisent l'application identité sur un des \(E_i\) et l'application nulle sur les autres \(E_j\) ?
\end{exo}

\begin{corr}
On pose \(f=\id{E_i}\) sur \(E_i\) et \(f=0\) sur \(E_j\) pour \(j\not=i\).

\(f\) est le projecteur sur \(E_i\) parallèlement à \(\bigoplus_{j\not=i}E_j\).
\end{corr}

\section{Somme de sous-espaces vectoriels en dimension finie}

\subsection{Base adaptée à un sous-espace}

\begin{defi}
Soient \(E\) un \(\K\)-espace vectoriel de dimension \(n\) et \(F\) un sous-espace vectoriel de \(E\) de dimension \(p\).

On appelle base de \(E\) adaptée à \(F\) toute base de \(E\) qui contient une base de \(F\). Quitte à changer l'ordre des vecteurs, on peut suppose dans une base adaptée à \(F\) que les \(p\) premiers vecteurs de la base forment une base de \(F\).
\end{defi}

\begin{defi}
Soient \(E\) un \(\K\)-espace vectoriel de dimension \(n\) et \(\paren{F_1,\dots,F_p}\) une famille de sous-espaces vectoriels supplémentaires dans \(E\).

On appelle base adaptée à la somme \(E=\bigoplus_{i=1}^pF_i\) la concaténation de bases de chacun des sous-espaces \(F_1,\dots,F_p\) (dans cet ordre).
\end{defi}

Si \(\fami{B}\) est une base de \(E\), alors en fractionnant la base en sous-familles, les sous-espaces vectoriels engendrés par chacune de ces sous-familles sont supplémentaires et la base est alors adaptée à la somme des sous-espaces.

\subsection{Sommes directes et bases}

On donne un moyen simple de vérifier qu'une somme est directe, voire plus.

\begin{prop}
Soient \(E\) un \(\K\)-espace vectoriel de dimension \(n\) et \(\paren{F_1,\dots,F_p}\) une famille de sous-espaces vectoriels de \(E\).

Si en concaténant des bases de chacun des sous-espaces vectoriels \(F_1,\dots,F_p\) on obtient une famille libre, alors les sous-espaces vectoriels sont en somme directe.

Si en concaténant des bases de chacun des sous-espaces vectoriels \(F_1,\dots,F_p\) on obtient une base de \(E\), alors les sous-espaces vectoriels sont supplémentaires.
\end{prop}

\subsection{Dimension d'une somme de sous-espaces vectoriels}

\begin{prop}
Soient \(E\) un \(\K\)-espace vectoriel de dimension finie et \(F_1,\dots,F_n\) des sous-espaces vectoriels de \(E\).

Alors \(\dim\sum_{i=1}^nF_i\leq\sum_{i=1}^n\dim F_i\).
\end{prop}

\begin{dem}
On note \(p_i=\dim F_i\) pour \(i\in\interventierii{1}{n}\).

Pour tout \(i\in\interventierii{1}{n}\), on choisit une base \(\fami{B}_i\) de \(F_i\) et on a \(F_i=\Vect{\fami{B}_i}\).

Donc \(F_1+\dots+F_n=\Vect{\fami{B}_1\dots\fami{B}_n}\).

Donc \[\begin{aligned}
\dim\paren{F_1+\dots+F_n}&\leq\Card\paren{\fami{B}_1\dots\fami{B}_n} \\
&=\sum_{i=1}^n\dim F_i.
\end{aligned}\]
\end{dem}

\begin{dem}[Autre méthode]\thlabel{dem4.10}
On pose l'application linéaire \[\fonction{\phi}{F_1\times\dots\times F_n}{E}{\paren{x_1,\dots,x_n}}{\sum_{i=1}^nx_i}\]

On a \(\Im\phi=F_1+\dots+F_n\).

Donc \[\begin{aligned}
\dim\paren{F_1+\dots+F_n}&=\dim\Im\phi \\
&=\rg\phi \\
&=\dim\paren{F_1\times\dots\times F_n}-\dim\ker\phi \\
&\leq\dim\paren{F_1\times\dots\times F_n} \\
&=\sum_{i=1}^n\dim F_i.
\end{aligned}\]
\end{dem}

Il y a égalité quand la somme est directe.

\begin{theo}
Soient \(E\) un \(\K\)-espace vectoriel de dimension finie et \(F_1,\dots,F_n\) des sous-espaces vectoriels de \(E\).

Alors \(F_1,\dots,F_n\) sont en somme directe ssi \(\dim\sum_{i=1}^nF_i=\sum_{i=1}^n\dim F_i\).
\end{theo}

\begin{dem}
On reprend la \thref{dem4.10}.

Si \(\dim\sum_{i=1}^nF_i=\sum_{i=1}^n\dim F_i\), alors \(\dim\ker\phi=0\).

Donc \(\ker\phi=\accol{\paren{x_1,\dots,x_n}\in F_1\times\dots\times F_n\tq\sum_{i=1}^nx_i=0}=\accol{\paren{0,\dots,0}}\).

Ainsi, \(F_1+\dots+F_n\) est directe.

Et réciproquement.
\end{dem}

\subsection{Sous-espaces supplémentaires}

En dimension finie, on a une façon plus simple de prouver que des sous-espaces vectoriels sont supplémentaires.

\begin{prop}[Trois pour le prix de deux]
Soient \(E\) un \(\K\)-espace vectoriel de dimension finie et \(F_1,\dots,F_n\) des sous-espaces vectoriels de \(E\).

Quand deux propriétés parmi les trois suivantes sont vraies, alors la troisième l'est aussi :

\begin{itemize}
    \item \(E=\sum_{i=1}^nF_i\) \\
    \item \(\dim\sum_{i=1}^nF_i=\sum_{i=1}^n\dim F_i\) \\
    \item \(\dim E=\sum_{i=1}^n\dim F_i\)
\end{itemize}

Donc dans ce cas, les sous-espaces vectoriels \(F_1,\dots,F_n\) sont supplémentaires.
\end{prop}

En pratique, le cas le plus utile est le suivant : si \(\dim\sum_{i=1}^nF_i=\sum_{i=1}^n\dim F_i=\dim E\), alors les sous-espaces vectoriels \(F_1,\dots,F_n\) sont supplémentaires.

\begin{dem}~\\
Si \(\dim\sum_{i=1}^nF_i=\sum_{i=1}^n\dim F_i\) et \(\dim\sum_{i=1}^nF_i=\dim E\).

Alors \(F_1+\dots+F_n\) est directe et \(\sum_{i=1}^nF_i=E\).

Alors \(E=\bigoplus_{i=1}^nF_i\).
\end{dem}

\begin{exo}
Dans \(\R^4\), soient \(H=\accol{\paren{x,y,z,t}\in\R^4\tq x+z+t=0\text{ et }x-y-z+t=0}\), \(F=\Vect{\paren{-1,0,1,1}}\) et \(G=\Vect{\paren{2,-1,1,0}}\).

\(F\), \(G\) et \(H\) sont-ils supplémentaires ?
\end{exo}

\begin{corr}
\(H\) est l'ensemble des solutions d'un système linéaire homogène : \[S:\begin{dcases}
x+z+t=0 \\
x-y-z+t=0
\end{dcases}\] \ie \(H=\ker f\) où \(f:\paren{x,y,z,t}\mapsto\paren{x+z+t,x-y-z+t}\).

D'après le théorème du rang, on a \(\dim H=\dim\R^4-\rg S=4-2=2\).

\(F\) et \(G\) sont des droites vectorielles donc \(\dim F=\dim G=1\).

On a donc \(\dim\R^4=\dim F+\dim G+\dim H\).

On a \[\begin{aligned}
v=\paren{x,y,z,t}\in H&\ssi\begin{dcases}
x+z+t=0 \\
x-y-z+t=0
\end{dcases} \\
&\ssi\begin{dcases}
x=-z-t \\
y=-2z
\end{dcases} \\
&\ssi v=\paren{-z-t,-2z,z,t} \\
&\ssi v=z\underbrace{\paren{-1,-2,1,0}}_{h_1}+t\underbrace{\paren{-1,0,0,1}}_{h_2}.
\end{aligned}\]

Donc \(H=\Vect{h_1,h_2}\).

On note \(f=\paren{-1,0,1,1}\) et \(g=\paren{2,-1,1,0}\).

On a alors \(F+G+H=\Vect{f,g,h_1,h_2}\).

Donc \(\dim\paren{F+G+H}=\rg\paren{f,g,h_1,h_2}\).

On note \(\fami{B}_0\) la base canonique de \(\R^4\).

On a \[\begin{aligned}
A=\Mat[\fami{B}_0]{f,g,h_1,h_2}&=\begin{pmatrix}
-1 & 2 & -1 & -1 \\
0 & -1 & -2 & 0 \\
1 & 1 & 1 & 0 \\
1 & 0 & 0 & 1
\end{pmatrix} \\
&\simqd{\substack{L_3\gets L_3+L_1 \\ L_4\gets L_4+L_1}}\begin{pmatrix}
-1 & 2 & -1 & -1 \\
0 & -1 & -2 & 0 \\
0 & 3 & 0 & -1 \\
0 & 2 & -1 & 0
\end{pmatrix} \\
&\simqd{\substack{L_3\gets L_3+3L_2 \\ L_4\gets L_4+2L_2}}\begin{pmatrix}
-1 & 2 & -1 & -1 \\
0 & 1 & -2 & 0 \\
0 & 0 & -6 & -1 \\
0 & 0 & -5 & 0
\end{pmatrix}.
\end{aligned}\]

Donc \(\rg A=4\).

Donc \(\dim\paren{F+G+H}=\dim\R^4=\dim F+\dim G+\dim H\).

Finalement, on a \(\R^4=F\oplus G\oplus H\).
\end{corr}

\subsection{Dimension d'une somme de deux sous-espaces vectoriels}

Rappel : la formule de Grassmann.

\begin{prop}
Soient \(E\) un \(\K\)-espace vectoriel de dimension finie et \(F,G\) deux sous-espaces vectoriels de \(E\).

Alors \(\dim\paren{F+G}=\dim F+\dim G-\dim\paren{F\inter G}\).
\end{prop}

\section{Polynômes d'endomorphismes et de matrices}

\subsection{\(\K\)-algèbres}

\subsubsection{Définition}

\begin{defi}
Un ensemble \(A\) est appelé \(\K\)-algèbre quand \(A\) est à la fois un anneau et un \(\K\)-espace vectoriel, dont les multiplications sont compatibles.

Il y a donc trois lois dans une \(\K\)-algèbre :

\begin{itemize}
    \item une addition classique \(+\) ; \\
    \item une multiplication externe \(.\) ; \\
    \item une multiplication interne, compatible avec la précédente : \[\quantifs{\forall\paren{\lambda,a,b}\in\K\times A^2}\lambda.\paren{ab}=\paren{\lambda.a}b=a\paren{\lambda.b}.\]
\end{itemize}
\end{defi}

On qualifie les \(\K\)-algèbres par du vocabulaire des anneaux (algèbres intègres, algèbres principales, etc) ou des espaces vectoriels (algèbres de dimension finie, etc).

\begin{ex}
\begin{itemize}
    \item \(\K\) est lui-même une \(\K\)-algèbre, où les deux multiplications sont confondues ; \(\C\) est aussi une \(\R\)-algèbre de dimension \(2\). \\
    \item \(\poly\) est une \(\K\)-algèbre intègre, commutative et de dimension finie. \\
    \item Si \(I\) est un intervalle, \(\F{I}{\K}\) est une \(\K\)-algèbre commutative, non-intègre et de dimension finie. \\
    \item Si \(n\in\Ns\), \(\M{n}\) est une \(\K\)-algèbre de dimension \(n^2\) qui n'est ni intègre ni commutative. \\
    \item Si \(E\) est un \(\K\)-espace vectoriel, alors \(\Lendo{E}\) est une \(\K\)-algèbre de dimension finie ssi \(E\) l'est aussi qui n'est ni intègre ni commutative.
\end{itemize}
\end{ex}

\subsubsection{Polynômes d'éléments dans une \(\K\)-algèbre}

\begin{prop}\thlabel{prop:polynomeElementAlgebre}
Soient \(A\) une \(\K\)-algèbre et \(a\in A\).

Pour \(P=\sum_{i=0}^nc_iX^i\in\poly\), on pose \(P\paren{a}=\sum_{i=0}^nc_ia^i\).

L'application \(\fonctionlambda{\poly}{A}{P}{P\paren{a}}\) est alors un morphisme d'algèbres : \[\quantifs{\forall\paren{P,Q}\in\poly^2;\forall\lambda\in\K}\begin{dcases}
\paren{P+Q}\paren{a}=P\paren{a}+Q\paren{a} \\
\paren{PQ}\paren{a}=P\paren{a}Q\paren{a} \\
\paren{\lambda P}\paren{a}=\lambda P\paren{a}
\end{dcases}\]

De plus, on note \(\poly[\K][a]\) l'ensemble \(\accol{P\paren{a}\tq P\in\poly}\) : cet ensemble est stable par les lois de \(A\), on dit que c'est une sous-algèbre de \(A\).
\end{prop}

\subsection{Cas particulier des algèbres \(\Lendo{E}\) ou \(\M{n}\)}

\(E\) étant un \(\K\)-espace vectoriel, l'ensemble \(\Lendo{E}\) est une \(\K\)-algèbre. De même, \(n\) étant un entier non-nul, \(\M{n}\) est une \(\K\)-algèbre. Dans ces algèbres, on définit naturellement la notion de polynôme d'endomorphisme ou de matrice. Bien sûr, ces notions sont liées par choix d'une base de l'espace.

\begin{prop}
Soient \(E\) un \(\K\)-espace vectoriel de dimension \(n\), \(\fami{B}\) une base de \(E\) et \(f\in\Lendo{E}\) tel que \(\Mat{f}=A\).

Alors pour tout polynôme \(P\in\poly\), \(P\paren{f}\) a pour matrice \(P\paren{A}\) dans la base \(\fami{B}\).
\end{prop}

\begin{dem}
Comme \(A=\Mat{f}\), \(\quantifs{\tpt k\in\N}A^k=\Mat{f^k}\).

Donc \(\quantifs{\forall\paren{a_0,\dots,a_p}\in\K^p}\Mat{\sum_{i=0}^pa_if^i}=\sum_{i=0}^pa_iA^i\) \cad \[\quantifs{\forall P\in\poly}\Mat{P\paren{f}}=P\paren{A}.\]
\end{dem}

\begin{rem}
\begin{itemize}
    \item La \guillemets{multiplication} dans \(\Lendo{E}\) est la composition \(\rond\). Donc la deuxième propriété de la \thref{prop:polynomeElementAlgebre} doit être comprise comme suit : si \(f\in\Lendo{E}\), alors \(\paren{PQ}\paren{f}=P\paren{f}\rond Q\paren{f}\). \\
    \item Même si les multiplications dans \(\M{n}\) ou \(\Lendo{E}\) ne sont pas commutatives en général, on peut intervertir l'ordre des polynômes car la multiplication dans \(\poly\) est commutative : si \(A\in\M{n}\), \(\paren{PQ}\paren{A}=P\paren{A}Q\paren{A}=Q\paren{A}P\paren{A}\) ; si \(u\in\Lendo{E}\), \(\paren{PQ}\paren{u}=P\paren{u}\rond Q\paren{u}=Q\paren{u}\rond P\paren{u}\). \\
    \item Attention aux notations ! Si \(f\in\Lendo{E}\), \(x\in E\) et \(P\in\poly\), alors l'application de \(P\paren{f}\) au vecteur \(x\) se note \(P\paren{f}\paren{x}\) et pas \(P\paren{f\paren{x}}\), notation qui n'a aucun sens.
\end{itemize}
\end{rem}

\subsection{Polynôme annulateur d'une matrice ou d'un endomorphisme}

\begin{defi}
Soient \(n\in\Ns\) et \(A\in\M{n}\). On appelle polynôme annulateur de \(A\) tout polynôme non-nul \(P\in\poly\) tel que \(P\paren{A}=0\).

Soient \(E\) un \(\K\)-espace vectoriel et \(u\in\Lendo{E}\). On appelle polynôme annulateur de \(u\) tout polynôme non-nul \(P\in\poly\) tel que \(P\paren{u}=0\).
\end{defi}

\begin{rem}
Attention à ne pas confondre les notions : si \(P\) est un polynôme annulateur de la matrice \(A\) (on dit aussi que \(P\) annule \(A\) par abus de langage), on ne dit pas que \(A\) est une racine de \(P\) !

Une racine d'un polynôme est un nombre...

De même, si \(P\paren{u}=0\), on ne dit pas que \(u\) est une racine de \(P\), ça n'a aucun sens.
\end{rem}

\begin{defi}
Si \(A\in\M{n}\), alors l'ensemble \(\Ann{A}=\accol{P\in\poly\tq P\paren{A}=0}\) est appelé idéal annulateur de \(A\).

Si \(u\) est un endomorphisme d'un espace vectoriel \(E\), alors l'ensemble \(\Ann{u}=\accol{P\in\poly\tq P\paren{u}=0}\) est appelé idéal annulateur de \(u\).
\end{defi}

\begin{theo}
Soit \(A\in\M{n}\). Alors \(\Ann{A}\) est un sous-espace vectoriel de \(\poly\) stable par \(\times\). De plus, il existe un unique polynôme unitaire \(\mu_A\) tel que \(\Ann{A}=\mu_A\poly\).

Soient \(E\) un \(\K\)-espace vectoriel de dimension finie et \(u\in\Lendo{E}\). Alors \(\Ann{u}\) est un sous-espace vectoriel de \(\poly\) stable par \(\times\). De plus, il existe un unique polynôme unitaire \(\mu_u\) tel que \(\Ann{u}=\mu_u\poly\).
\end{theo}

\begin{dem}
\begin{itemize}
    \item Soient \(\paren{P,Q}\in\Ann{A}^2\) et \(\lambda\in\K\). \\\\ On a \[\paren{P+Q}\paren{A}=P\paren{A}+Q\paren{A}=0+0=0\] et \[\paren{PQ}\paren{A}=P\paren{A}Q\paren{A}=0\times0=0\] et \[\paren{\lambda P}\paren{A}=\lambda P\paren{A}=\lambda\times0=0.\] Donc \(P+Q\in\Ann{A}\), \(PQ\in\Ann{A}\) et \(\lambda P\in\Ann{A}\). \\\\ Remarque : on a même montré que \(\quantifs{\forall P\in\poly;\forall Q\in\Ann{A}}PQ\in\Ann{A}\). \\
    \item On montre d'abord l'existence d'un polynôme annulateur de \(A\) non-nul. \\\\ On considère la famille \(\paren{I_n,A,\dots,A^{n^2}}\) qui contient \(n^2+1\) éléments de \(\M{n}\), espace de dimension \(n^2\), et est donc liée : il existe \(\paren{a_0,\dots,a_{n^2}}\in\K^{n^2+1}\) tel que \[\paren{a_0,\dots,a_{n^2}}\not=\paren{0,\dots,0}\qquad\text{et}\qquad\sum_{i=0}^{n^2}a_iA^i=0.\] Alors le polynôme \(\sum_{i=0}^{n^2}a_iX^i\) est un polynôme annulateur de \(A\). \\
    \item On pose \(D=\accol{\deg P\tq P\in\Ann{A}\excluant\accol{0}}\). \\\\ D'après ce qui précède, \(D\) est une partie non-vide de \(\N\). \\\\ Donc \(D\) admet un minimum \(d\geq1\) d'après le principe fondamental de \(\N\), associé à un polynôme \(P\not=0\) tel que \(P\paren{A}=0\). \\\\ En notant \(\lambda\) le coefficient dominant de \(P\), le polynôme \(\mu=\dfrac{1}{\lambda}P\) est annulateur de \(A\) et unitaire. \\
    \item Montrons que \(\Ann{A}=\mu\poly\). \\\\ L'inclusion \(\mu\poly\subset\Ann{A}\) est triviale. \\\\ Soit \(P\in\Ann{A}\). On effectue la division euclidienne de \(P\) par \(\mu\) : il existe \(\paren{Q,R}\in\poly^2\) tel que \[P=Q\mu+R\qquad\text{et}\qquad\deg R<\deg\mu=d.\] Donc \(P\paren{A}=Q\paren{A}\mu\paren{A}+R\paren{A}\). \\\\ Or \(P\paren{A}=\mu\paren{A}=0\) donc \(R\paren{A}=0\) \ie \(R\in\Ann{A}\). \\\\ Par définition de \(d\), on a \(R=0\). \\\\ Donc \(P\) est un multiple de \(\mu\). \\\\ D'où \(\Ann{A}\subset\mu\poly\). \\\\ Donc \(\Ann{A}=\mu\poly\). \\
    \item Montrons maintenant l'unicité de \(\mu\). \\\\ Si \(\nu\) est un polynôme annulateur de \(A\) de degré \(d\) et unitaire, alors \(\Ann{A}=\nu\poly=\mu\poly\). \\\\ Donc \(\nu\divise\mu\) et \(\mu\divise\nu\). \\\\ Or \(\mu\) et \(\nu\) sont unitaires donc \(\mu=\nu\).
\end{itemize}
\end{dem}

\begin{rem}
Ce dernier résultat est faux en dimension infinie. Contre-exemple : l'endomorphisme \(u:\poly\to\poly\) défini par \(u\paren{P}=XP\).
\end{rem}

\begin{defi}
On appelle polynôme minimal d'une matrice carrée \(A\) le polynôme \(\mu_A\) précédent (noté aussi parfois \(\pi_A\)). C'est le polynôme unitaire de degré minimal qui annule \(A\).

On appelle polynôme minimal d'un endomorphisme \(u\) en dimension finie le polynôme \(\mu_u\) précédent (noté aussi parfois \(\pi_u\)). C'est le polynôme unitaire de degré minimal qui annule \(u\).
\end{defi}

Autrement dit, on a l'équivalence : pour tout \(P\in\poly\), \[P\paren{u}=0\ssi\mu_u\divise P.\]

De même, on a l'équivalence : pour tout \(P\in\poly\), \[P\paren{A}=0\ssi\mu_A\divise P.\]

Les polynômes annulateurs sont donc les multiples des polynômes minimaux.

On verra plus tard qu'on peut trouver des polynômes annulateurs de plus petits degrés que ceux donnés par le théorème précédent.

En général, il est souvent pénible de calculer le polynôme minimal. En pratique, on se contente de trouver des polynômes annulateurs de degrés pas trop grands (et souvent, il s'agit du polynôme minimal).

\begin{rem}
Si \(A\in\M{n}\) et \(\K\prim\) est un sous-corps de \(\C\) qui contient \(\K\) (on dit que \(\K\prim\) est une extension de \(\K\)), alors le polynôme minimal de \(A\), vue comme matrice de \(\M{n}[\K\prim]\), est a priori différent de celui de \(A\) vue comme matrice de \(\M{n}\). On peut seulement affirmer pour l'instant que \(\mu_{A\,\K\prim}\) divise \(\mu_{A\,\K}\).

En fait, on montre plus loin que le polynôme minimal ne dépend pas du corps \(\K\).
\end{rem}

\subsection{Utilisation pratique d'un polynôme annulateur}

\subsubsection{Calcul de l'inverse}

\begin{prop}\thlabel{prop4.15}
Soient \(n\in\Ns\) et \(A\in\M{n}\).

Alors \(A\) est inversible ssi \(A\) possède un polynôme annulateur \(P\) tel que \(0\) ne soit pas racine de \(P\). Dans ce cas, \(A\inv\) est un polynôme en \(A\).

De même, soient \(E\) un \(\K\)-espace vectoriel et \(f\in\Lendo{E}\).

Alors si \(f\) possède un polynôme annulateur \(P\) tel que \(0\) ne soit pas racine de \(P\), \(f\) est un automorphisme de \(E\). Dans ce cas, \(f\inv\) est un polynôme en \(f\). La réciproque est vraie si \(E\) est de dimension finie.
\end{prop}

\begin{dem}
\imprec

On suppose que \(A\) possède un polynôme annulateur \(P\) tel que \(P\paren{0}\not=0\).

En notant \(P=\sum_{i=0}^pa_iX^i\), on a \(\sum_{i=0}^pa_iA^i=0\) et \(a_0\not=0\).

Donc \(a_0I_n+a_1A+\dots+a_pA^p=0\).

Donc \(a_0I_n=-a_1A-\dots-a_pA^p\).

Donc \(I_n=A\underbrace{\paren{\dfrac{-a_1}{a_0}I_n-\dots-\dfrac{a_p}{a_0}A^{p-1}}}_{B}\).

On a trouvé une matrice \(B\in\M{n}\) telle que \(AB=I_n=BA\).

Donc \(A\) est inversible et \(A\inv=B=Q\paren{A}\) où \(Q=\sum_{i=0}^{p-1}\dfrac{-a_{i+1}}{a_0}X^i\).

\impdir

On suppose \(A\) inversible. On veut montrer que \(\mu_A\paren{0}\not=0\).

Par l'absurde, on suppose \(\mu_A\paren{0}=0\).

On a \(\mu_A=a_1X+\dots+X^p\) et \(\mu_A\paren{A}=0\).

Donc \(a_1A+\dots+A^p=0\).

\(A\) étant inversible, on multiplie par \(A\inv\) et on obtient \(a_1I_n+a_2A+\dots+A^{p-1}=0\).

Donc \(a_1+a_2X+\dots+X^{p-1}\) est annulateur de \(A\) et de degré strictement inférieur à celui de \(\mu_A\) : contradiction.

Donc \(\mu_A\paren{0}\not=0\).
\end{dem}

\begin{exo}~\\
On pose \(A=\begin{pmatrix}
1 & 1 & 1 \\
0 & 1 & 1 \\
0 & 0 & 2
\end{pmatrix}\). Déterminez un polynôme annulateur de \(A\), montrez que \(A\) est inversible et calculez \(A\inv\).
\end{exo}

\begin{corr}~\\
On a \(A^2=\begin{pmatrix}
1 & 2 & 4 \\
0 & 1 & 3 \\
0 & 0 & 4
\end{pmatrix}\) et \(A^3=\begin{pmatrix}
1 & 3 & 11 \\
0 & 1 & 7 \\
0 & 0 & 8
\end{pmatrix}\).

Donc on a \[\begin{aligned}
A^3=xA^2+yA+zI_3&\ssi\begin{dcases}
1=x+y+z \\
8=4x+2y+z \\
7=3x+y \\
11=4x+y \\
3=2x+y
\end{dcases} \\
&\ssi\begin{dcases}
x=4 \\
y=-5 \\
z=2
\end{dcases}
\end{aligned}\]

Donc \(A^3=4A^2-5A+2I_3\).

Donc \(P=X^3-4X^2+5X-2\) est un polynôme annulateur de \(A\).

On a \(P\paren{0}\not=0\) donc \(A\) est inversible.

On obtient \[A\inv=\dfrac{1}{2}\paren{A^2-4A+5I_3}=\begin{pmatrix}
1 & -1 & 0 \\
0 & 1 & -\nicefrac{1}{2} \\
0 & 0 & \nicefrac{1}{2}
\end{pmatrix}.\]
\end{corr}

\begin{exo}
Soient \(E\) un \(\K\)-espace vectoriel et \(p\in\Lendo{E}\) un projecteur. Déterminez un polynôme annulateur de \(p\).

Soit \(\lambda\in\K\). On pose \(f=p-\lambda\id{E}\). Déterminez un polynôme annulateur de \(f\) et vérifiez que \(f\) est un automorphisme pour presque toutes les valeurs de \(\lambda\) ; dans ce cas, calculez son inverse.
\end{exo}

\begin{corr}
\(p\) est un projecteur donc \(p^2=p\) donc \(X^2-X\) est un polynôme annulateur de \(p\).

Donc le polynôme annulateur minimal de \(p\) est un diviseur unitaire de \(X^2-X\) : \(X^2-X\), \(X\) (auquel cas \(p=0\)) ou \(X-1\) (auquel cas \(p=\id{E}\)).

On a \(f=p-\lambda\id{E}\) donc \[\begin{aligned}
f^2&=p^2-2\lambda p+\lambda^2\id{E} \\
&=\paren{1-2\lambda}p+\lambda^2\id{E} \\
&=\paren{1-2\lambda}f+\lambda\paren{1-2\lambda}\id{E}+\lambda^2\id{E} \\
&=\paren{1-2\lambda}f+\paren{-\lambda^2+\lambda}\id{E}.
\end{aligned}\]

Donc \(P=X^2+\paren{2\lambda-1}X+\paren{\lambda^2-\lambda}\) est annulateur de \(f\).

Si \(\lambda\not=0\) et \(\lambda\not=1\) alors \(\lambda^2-\lambda=P\paren{0}=0\).

Donc d'après la \thref{prop4.15}, \(f\in\GL{}[E]\).

De plus, \(f\inv=\dfrac{1}{\lambda-\lambda^2}\paren{f+\paren{2\lambda-1}\id{E}}\).

Si \(\lambda=0\) alors \(f=p\) est un automorphisme ssi \(p=\id{E}\).

Si \(\lambda=1\) alors \(f=p-\id{E}\) est un automorphisme ssi \(p=0\).
\end{corr}

\subsubsection{Calcul de puissances}

\begin{prop}
Soient \(n\in\Ns\) et \(A\in\M{n}\). On choisit un polynôme annulateur \(P\) de la matrice \(A\).

Alors \(\quantifs{\tpt p\in\N}A^p=R_p\paren{A}\) où \(R_p\) est le reste de la division euclidienne de \(X^p\) par \(P\).

De même, soient \(E\) un \(\K\)-espace vectoriel et \(f\in\Lendo{E}\) qui possède un polynôme annulateur \(P\).

Alors \(\quantifs{\tpt p\in\N}f^p=R_p\paren{f}\) où \(R_p\) est le reste de la division euclidienne de \(X^p\) par \(P\).
\end{prop}

\begin{dem}
On effectue la division euclidienne de \(X^p\) par \(P\) : \(X^p=PQ+R_p\) et \(\deg R_p<\deg P\).

Alors \(A^p=P\paren{A}Q\paren{A}+R_p\paren{A}=R_p\paren{A}\) car \(P\paren{A}=0\).
\end{dem}

Conséquence :

\begin{itemize}
    \item si \(A\) possède un polynôme annulateur de degré \(a\), alors \(\poly[\K][A]=\accol{P\paren{A}\tq P\in\polydeg{a-1}}\) ; \\
    \item si \(f\) possède un polynôme annulateur de degré \(a\), alors \(\poly[\K][f]=\accol{P\paren{f}\tq P\in\polydeg{a-1}}\).
\end{itemize}

\begin{prop}
Si \(p\) est le degré du polynôme minimal d'une matrice \(A\), alors \(\dim\poly[\K][A]=p\) et \(\paren{I_n,A,\dots,A^{p-1}}\) est une base de \(\poly[\K][A]\).

Si \(p\) est le degré du polynôme minimal d'un endomorphisme \(f\) d'un espace vectoriel \(E\), alors \(\dim\poly[\K][f]=p\) et \(\paren{\id{E},f,\dots,f^{p-1}}\) est une base de \(\poly[\K][f]\).
\end{prop}

\begin{dem}
On note \(p=\deg\mu_A\).

Soit \(B\in\poly[\K][A]\).

Il existe \(P\in\poly\) tel que \(B=P\paren{A}\).

On effectue la division euclidienne de \(P\) par \(\mu_A\) : \(P=Q\mu_A+R\) avec \(\deg R<p\).

Donc \(P\paren{A}=Q\paren{A}\mu_A\paren{A}+R\paren{A}=R\paren{A}\in\Vect{I_n,A,\dots,A^{p-1}}\).

Donc \(\poly[\K][A]\subset\Vect{I_n,A,\dots,A^{p-1}}\).

L'inclusion réciproque est immédiate.

Donc \(\poly[\K][A]=\Vect{I_n,A,\dots,A^{p-1}}\).

Si \(\paren{I_n,A,\dots,A^{p-1}}\) est liée, alors il existe \(\paren{a_0,\dots,a_{p-1}}\not=\paren{0,\dots,0}\) tel que \(\sum_{i=0}^{p-1}a_iA^i=0\).

En posant \(P=\sum_{i=0}^{p-1}a_iX^i\), on a un polynôme annulateur de \(A\) de degré strictement inférieur à \(p=\deg\mu_A\) : contradiction.

Donc \(\paren{I_n,A,\dots,A^{p-1}}\) est libre et est donc une base de \(\poly[\K][A]\) et \(\dim\poly[\K][A]=p\).
\end{dem}

\begin{exo}
Soient \(E\) un espace vectoriel et \(\paren{p,q}\in\Lendo{E}^2\) deux projecteurs tels que \(p+q=\id{E}\). Vérifiez que \(p\rond q=q\rond p=0\). Déterminez un polynôme annulateur de \(f=2p+3q\). Donnez une expression générale de \(f^k\) en fonction de \(f\) et \(k\).
\end{exo}

\begin{corr}
On a \(p+q=\id{E}\) donc \(p\rond p+p\rond q=p\) donc \(p\rond q=0\) car \(p\rond p=p\).

De même, \(q\rond p=0\).

On a donc \(f^2=4p^2+6pq+6qp+9q^2=4p+9q\).

Donc \[\begin{aligned}
f^2=xf+y\id{E}&\ssi\begin{dcases}
4=2x+y \\
9=3x+y
\end{dcases} \\
&\ssi\begin{dcases}
5=x \\
-6=y
\end{dcases}
\end{aligned}\]

Donc \(P=X^2-5X+6\) est un polynôme annulateur de \(f\) dont les racines sont \(2\) et \(3\).

Soit \(k\in\N\).

On effectue la division euclidienne de \(X^k\) par \(P\) : \[X^k=\paren{X^2-5X+6}Q+a_kX+b_k.\]

En appliquant en \(2\) et en \(3\), on obtient \[\begin{dcases}
2a_k+b_k=2^k \\
3a_k+b_k=3^k
\end{dcases}\qquad\text{donc}\qquad\begin{dcases}
a_k=3^k-2^k \\
b_k=3\times2^k-2\times3^k
\end{dcases}\]

Donc \(f^k=a_kf+b_k\id{E}=\paren{3^k-2^k}f+\paren{3\times2^k-2\times3^k}\id{E}\).
\end{corr}

\begin{rem}
Plus généralement, si \(P\) est annulateur de \(f\) et de degré \(N\) à \(N\) racines simples \(\omega_1,\dots,\omega_N\).

Alors \(X^k=PQ+c_{N-1}X^{N-1}+\dots+c_1X+c_0\).

Alors on a un système linéaire à \(N\) équations et \(N\) inconnues \[\begin{dcases}
c_0+c_1\omega_1+c_2\omega_1^2+\dots+c_{N-1}\omega_1^{N-1}=\omega_1^k \\
\vdots \\
c_0+c_1\omega_N+c_2\omega_N^2+\dots+c_{N-1}\omega_N^{N-1}=\omega_N^k
\end{dcases}\]

Ce système a pour déterminant \[\begin{vmatrix}
1 & \omega_1 & \omega_1^2 & \dots & \omega_1^{N-1} \\
 & & \vdots & & \vdots \\
1 & \omega_N & \omega_N^2 & \dots & \omega_N^{N-1}
\end{vmatrix}=\prod_{1\leq i<j\leq n}\paren{\omega_j-\omega_i}\not=0\] et admet donc une unique solution.
\end{rem}

\begin{exo}~\\
On pose \(A=\begin{pmatrix}
2 & -4 & -5 \\
-1 & 2 & 2 \\
1 & -2 & -2
\end{pmatrix}\). Vérifiez que \(P=X^3-2X^2+X\) est un polynôme annulateur de \(A\). Donnez une expression générale de \(A^p\) en fonction de \(A\) et \(p\).
\end{exo}

\begin{corr}~\\
On a \(A^2=\begin{pmatrix}
3 & -6 & -8 \\
-2 & 4 & 5 \\
2 & -4 & -5
\end{pmatrix}\) et \(A^3=\begin{pmatrix}
4 & -8 & -11 \\
-3 & 6 & 8 \\
3 & -6 & -8
\end{pmatrix}\) donc \(P=X^3-2X^2+X=X\paren{X-1}^2\) est annulateur de \(A\).

Soit \(k\in\N\).

On effectue la division euclidienne de \(X^k\) par \(P\) : \[X^k=X\paren{X-1}^2Q+a_kX^2+b_kX+c_k.\]

En évaluant en \(0\) et \(1\), on obtient \[\begin{dcases}
c_k=0^k \\
a_k+b_k+c_k=1^k \\
2a_k+b_k=k1^k
\end{dcases}\]

Or \(1\) est racine double donc \(P\prim\paren{1}=0\) donc \(2a_k+b_k=k\).

Donc, si \(k\not=0\), on a \[\begin{dcases}
c_k=0 \\
a_k+b_k=1 \\
2a_k+b_k=k
\end{dcases}\qquad\text{donc}\qquad\begin{dcases}
a_k=k-1 \\
b_k=2-k \\
c_k=0
\end{dcases}\]

Donc \(\quantifs{\tpt k\in\Ns}A^k=\paren{k-1}A^2+\paren{2-k}A\).
\end{corr}

\begin{cor}
Si \(\K\prim\) est une extension de \(\K\), alors pour toute matrice \(A\in\M{n}\), ses polynômes minimaux relativement à \(\K\) et \(\K\prim\) sont égaux : \(\mu_{A\,\K}=\mu_{A\,\K\prim}\).

Autrement dit, le polynôme minimal ne dépend pas du corps \(\K\).
\end{cor}

\begin{dem}
Soit \(A\in\M{n}\).

On note \(\Ann[\K]{A}=\accol{P\in\poly\tq P\paren{A}=0}\) et \(\Ann[\K\prim]{A}=\accol{P\in\poly[\K\prim]\tq P\paren{A}=0}\).

Soit \(P\in\Ann[\K]{A}\).

Comme \(\K\subset\K\prim\), on a \(P\in\Ann[\K\prim]{A}\).

Donc \(\mu_{A\,\K}\poly=\Ann[\K]{A}\subset\Ann[\K\prim]{A}=\mu_{A\,\K\prim}\poly[\K\prim]\).

Donc \(\mu_{A\,\K}\in\Ann[\K\prim]{A}\) donc \(\mu_{A\,\K\prim}\divise\mu_{A\,\K}\).

Donc \(p\prim=\deg\mu_{A\,\K\prim}\leq\deg\mu_{A\,\K}=p\).

Or \(\paren{I_n,A,\dots,A^{p-1}}\) est libre dans le \(\K\)-espace vectoriel \(\poly[\K][A]\) et \(\paren{I_n,A,\dots,A^{p\prim-1}}\) est libre dans le \(\K\prim\)-espace vectoriel \(\poly[\K\prim][A]\).

On veut montrer que \(\paren{I_n,A,\dots,A^{p-1}}\) est libre dans le \(\K\prim\)-espace vectoriel \(\poly[\K\prim][A]\).

Par l'absurde, si cette famille est liée, alors il existe \(\paren{\lambda_0,\dots,\lambda_{p-1}}\in\paren{\K\prim}^p\) tel que \[\begin{dcases}
\paren{\lambda_0,\dots,\lambda_{p-1}}\not=\paren{0,\dots,0} \\
\sum_{i=0}^{p-1}\lambda_iA^i=0
\end{dcases}\]

On pose \(q=\max\accol{i\in\interventierii{0}{p-1}\tq\lambda_i\not=0}\).

On a alors \(A^q=-\sum_{i=0}^{q-1}\dfrac{\lambda_i}{\lambda_q}A^i\).

Cette égalité est interprétable en un système linéaire à \(n^2\) équations dont les \guillemets{inconnues} sont \(\dfrac{-\lambda_i}{\lambda_q}\) et les coefficients sont les coefficients de \(A^q,\dots,I_n\) qui sont tous dans \(\K\).

En résolvant ce système linéaire, on obtient des solutions qui sont dans \(\K\).

Donc on obtient une relation de dépendance linéaire entre les matrices \(\paren{I_n,\dots,A^q,\dots,A^{p-1}}\) à coefficients dans \(\K\) : contradiction car \(\paren{I_n,\dots,A^{p-1}}\) est libre.

Donc \(p\leq\dim_{\K\prim}\poly[\K\prim][A]=p\prim\).

D'où \(p=p\prim\).

Donc, comme \(\mu_{A\,\K\prim}\divise\mu_{A\,\K}\) et que ces polynômes sont unitaires, on a \[\mu_{A\,\K\prim}=\mu_{A\,\K}.\]
\end{dem}

\section{Matrices semblables, trace}

\subsection{Trace d'une matrice}

\begin{defi}
Soit \(A\in\M{n}\). On appelle trace de \(A\) la somme de ses coefficients diagonaux : \[\tr A=\sum_{i=1}^na_{i\,i}.\]
\end{defi}

L'application trace vérifie de remarquables propriétés.

\begin{prop}
\begin{itemize}
    \item La trace est une forme linéaire sur \(\M{n}\). \\
    \item \(\quantifs{\Tpt A\in\M{n}}\tr\paren{\trans{A}}=\tr A\). \\
    \item \(\quantifs{\Tpt\paren{A,B}\in\M{n}^2}\tr\paren{AB}=\tr\paren{BA}\).
\end{itemize}
\end{prop}

\subsection{Matrices semblables}

\begin{defi}
Soient \(A,B\in\M{n}\).

On dit que \(A\) et \(B\) sont semblables quand il existe \(P\in\GL{n}\) telle que \(B=P\inv AP\).
\end{defi}

\begin{prop}
La relation de similitude entre matrices de \(\M{n}\) est une relation d'équivalence.
\end{prop}

La relation de similitude est une relation très contraignante. Il n'existe pas de caractérisation simple de la similitude entre deux matrices carrées : savoir si deux matrices sont semblables est un problème difficile.

D'après la formule de changement de base, on a immédiatement le résultat suivant.

\begin{prop}
Deux matrices de \(\M{n}\) sont semblables ssi elle représentent un même endomorphisme dans des bases différentes. La matrice \(P\) est la matrice de passage d'une base à l'autre.
\end{prop}

\begin{cor}
Deux matrices semblables ont même rang, même trace et même déterminant.
\end{cor}

Mais c'est loin d'être suffisant pour être semblables.

\begin{prop}
Si \(A\) et \(B\) sont deux matrices semblables, alors pour tout polynôme \(P\in\poly\), \(P\paren{A}\) et \(P\paren{B}\) sont semblables avec la même matrice de passage.
\end{prop}

\begin{dem}
Si \(A\) et \(B\) sont semblables, il existe \(Q\in\GL{n}\) telle que \(B=Q\inv AQ\).

Alors, par récurrence sur \(k\in\N\), on a \(B^k=Q\inv A^kQ\).

Puis, par combinaison linéaire, \(\quantifs{\tpt P\in\poly}P\paren{B}=Q\inv P\paren{A}Q\).
\end{dem}

\subsection{Trace d'un endomorphisme}

\begin{prop}
Soient \(E\) un \(\K\)-espace vectoriel de dimension finie et \(f\in\Lendo{E}\).

Toutes les matrices carrées représentant \(f\) ont la même trace. Cette trace ne dépend donc pas du choix de la base dans laquelle on écrit la matrice de \(f\), elle ne dépend que de \(f\) : on l'appelle trace de \(f\) et on la note \(\tr f\).
\end{prop}

On peut alors reformuler les résultats sur la trace d'une matrice.

\begin{prop}
\begin{itemize}
    \item La trace est une forme linéaire sur \(\Lendo{E}\). \\
    \item \(\quantifs{\Tpt\paren{u,v}\in\Lendo{E}^2}\tr\paren{u\rond v}=\tr\paren{v\rond u}\).
\end{itemize}
\end{prop}

\section{Opérations par blocs}

\subsection{Cas général}

Soit \(\paren{n,p}\in\paren{\Ns}^2\). On fixe deux entiers \(k,l\) tels que \(1\leq k\leq n-1\) et \(1\leq l\leq p-1\).

À toute matrice \(M\in\M{n\,p}\), on associe quatre matrices obtenues en découpant la matrice en blocs : \[M=\begin{pmatrix}
A & B \\
C & D
\end{pmatrix}\] où \(A=\paren{m_{i\,j}}_{\substack{1\leq i\leq k \\ 1\leq j\leq l}}\), \(B=\paren{m_{i\,j}}_{\substack{1\leq i\leq k \\ l+1\leq j\leq p}}\), \(C=\paren{m_{i\,j}}_{\substack{k+1\leq i\leq n \\ 1\leq j\leq l}}\) et \(D=\paren{m_{i\,j}}_{\substack{k+1\leq i\leq n \\ l+1\leq j\leq p}}\).

Cette décomposition par blocs permet de faire des calculs formellement comme s'il s'agissait de nombres.

\begin{prop}~\\
Soient \(M=\begin{pmatrix}
A & B \\
C & D
\end{pmatrix}\) et \(M\prim=\begin{pmatrix}
A\prim & B\prim \\
C\prim & D\prim
\end{pmatrix}\) deux matrices de même taille décomposées de la même façon en blocs et \(\lambda\in\K\).

Alors \(M+M\prim=\begin{pmatrix}
A+A\prim & B+B\prim \\
C+C\prim & D+D\prim
\end{pmatrix}\) et \(\lambda M=\begin{pmatrix}
\lambda A & \lambda B \\
\lambda C & \lambda D
\end{pmatrix}\).
\end{prop}

\begin{prop}~\\
Soient \(M=\begin{pmatrix}
A & B \\
C & D
\end{pmatrix}\) et \(M\prim=\begin{pmatrix}
A\prim & B\prim \\
C\prim & D\prim
\end{pmatrix}\) deux matrices telles que le produit \(MM\prim\) existe et décomposées en blocs.

Alors, sous réserve que les blocs soient de tailles compatibles pour la multiplication, on a \[MM\prim=\begin{pmatrix}
AA\prim+BC\prim & AB\prim+BD\prim \\
CA\prim+DC\prim & CB\prim+DD\prim
\end{pmatrix}.\]
\end{prop}

\begin{rem}
\begin{itemize}
    \item Comme les symboles mis en jeu ne sont pas des nombres mais des matrices, il est indispensable de respecter l'ordre dans les produits. \\
    \item On peut généraliser à un nombre quelconque de blocs, pas forcément deux en ligne ou en colonne.
\end{itemize}
\end{rem}

\subsection{Cas particuliers des matrices carrées}

Si \(M\) est une matrice de \(\M{n}\), alors avec les mêmes notations, on choisit toujours \(A\) et \(D\) carrées elles aussi. Dans ce paragraphe, on suppose que c'est le cas.

\begin{defi}
On dit que \(M\) est triangulaire supérieure par blocs quand il existe des matrices carrées \(A_1,\dots,A_k\) telles que \(M\) soit de la forme \[M=\begin{pmatrix}
A_1 & ? & ? & \dots & ? \\
0 & A_2 & ? & \dots & ? \\
\vdots &  & \ddots &  & \vdots \\
0 & \dots & 0 & A_{k-1} & ? \\
0 & \dots & 0 & 0 & A_k
\end{pmatrix}.\]
\end{defi}

On définit de même la notion de matrice triangulaire inférieure par blocs.

\begin{defi}
On dit que \(M\) est diagonale par blocs quand il existe des matrices carrées \(A_1,\dots,A_k\) telles que \(M\) soit de la forme \[M=\begin{pmatrix}
A_1 & 0 & 0 & \dots & 0 \\
0 & A_2 & 0 & \dots & 0 \\
\vdots &  & \ddots &  & \vdots \\
0 & \dots & 0 & A_{k-1} & 0 \\
0 & \dots & 0 & 0 & A_k
\end{pmatrix}.\]
\end{defi}

Les résultats sur les matrices triangulaires ou diagonales restent valables par blocs : la somme et le produit de deux matrices triangulaires supérieures par blocs de mêmes tailles l'est encore, et de même pour les matrices triangulaires inférieures par blocs et les matrices diagonales par blocs.

Une conséquence est qu'une matrice \(M\) triangulaire par blocs est inversible ssi tous les blocs diagonaux sont inversibles.

Dans ce cas, l'inverse de \(M\) est triangulaire par blocs et ses blocs diagonaux sont les inverses des blocs diagonaux de \(M\).

En particulier, l'inverse d'une matrice \(M\) diagonale par blocs est la matrice diagonale par blocs dont les blocs diagonaux sont les inverses de ceux de \(M\).

De plus, le déterminant d'une matrice triangulaire par blocs est le produit des déterminants des blocs diagonaux.

\subsection{Interprétation des blocs}

\begin{defi}
Soient \(E\) un \(\K\)-espace vectoriel, \(f\in\Lendo{E}\) et \(F\) un sous-espace vectoriel de \(E\).

On dit que \(F\) est stable par \(f\) quand \(f\paren{F}\subset F\), \ie \(\quantifs{\tpt x\in F}f\paren{x}\in F\).
\end{defi}

Dans ce cas, on peut définir une application \(\phi\) de \(F\) dans \(F\) en posant \(\quantifs{\tpt x\in F}\phi\paren{x}=f\paren{x}\).

Il est facile de vérifier que \(\phi\) est un endomorphisme de \(F\), appelé endomorphisme induit par \(f\) dans \(F\).

\begin{ex}
Si \(g\) est un endomorphisme de \(E\) qui commute avec \(f\) (\ie \(fg=gf\)), alors \(\ker g\) et \(\Im g\) sont stables par \(f\).
\end{ex}

\begin{prop}
Soient \(E\) un \(\K\)-espace vectoriel de dimension \(n\), \(f\in\Lendo{E}\) et \(F\) un sous-espace vectoriel de \(E\) de dimension \(p\).

Si \(F\) est stable par \(f\), alors il existe une base de \(E\) dans laquelle la matrice de \(f\) est triangulaire supérieure par bloc, le premier bloc étant de taille \(\paren{p,p}\) : \[\Mat{f}=\begin{pmatrix}
A & B \\
0 & D
\end{pmatrix}\text{ et }A\in\M{p}.\]

Réciproquement, si \(f\) possède une matrice de cette forme, alors le sous-espace vectoriel engendré par les \(p\) premiers vecteurs est stable par \(f\).
\end{prop}

\begin{dem}
On choisit une base de \(E\) adaptée à \(F\) : \(\fami{B}=\paren{\underbrace{e_1,\dots,e_p}_{\text{base de }F},e_{p+1},\dots,e_n}\).

Pour tout \(j\in\interventierii{1}{p}\), en notant \(f\paren{e_j}=\paren{a_{1\,j},\dots,a_{n\,j}}_{\fami{B}}\), on a \[\begin{aligned}
f\paren{e_j}\in F&\ssi f\paren{e_j}\in\Vect{e_1,\dots,e_p} \\
&\ssi\quantifs{\forall i\geq p+1}a_{i\,j}=0 \\
&\ssi f\paren{e_j}=\paren{a_{1\,j},\dots,a_{p\,j},0,\dots,0}_{\fami{B}}.
\end{aligned}\]

Alors, si \(F\) est stable par \(f\), on a \[\Mat{f}=\begin{pmatrix}
a_{1\,1} & \dots & a_{1\,p} & ? & \dots & ? \\
\vdots &  & \vdots & \vdots &  & \vdots \\
a_{p\,1} & \dots & a_{p\,p} & \vdots &  & \vdots \\
0 & \dots & 0 & \vdots &  & \vdots \\
\vdots &  & \vdots & \vdots &  & \vdots \\
0 & \dots & 0 & ? & \dots & ?
\end{pmatrix}.\]

Et réciproquement.
\end{dem}

\begin{prop}
Soient \(E\) un \(\K\)-espace vectoriel de dimension \(n\) et \(f\in\Lendo{E}\).

Si \(F_1,\dots,F_k\) sont des sous-espaces vectoriels supplémentaires stables par \(f\) de dimensions respectives \(p_1,\dots,p_k\), alors il existe une base de \(E\) dans laquelle la matrice de \(f\) est diagonale par blocs, la taille du \(i\)-ème bloc étant \(\paren{p_i,p_i}\) : \[\Mat{f}=\begin{pmatrix}
A_1 & 0 & 0 & \dots & 0 \\
0 & A_2 & 0 & \dots & 0 \\
\vdots &  & \ddots &  & \vdots \\
0 & \dots & 0 & A_{k-1} & 0 \\
0 & \dots & 0 & 0 & A_k
\end{pmatrix}.\]

Réciproquement, si \(f\) possède une matrice dans une certaine base qui est diagonale par blocs et contenant \(k\) blocs carrés, alors il existe \(k\) sous-espaces vectoriels \(F_1,\dots,F_k\) stables par \(f\) et supplémentaires dans \(E\).
\end{prop}


\chapter{Réduction des endomorphismes}

\minitoc

Dans ce chapitre, \(\K\) désigne un sous-corps de \(\C\), en général \(\R\) ou \(\C\).

\section{Éléments propres d'un endomorphisme}

Dans cette section, \(E\) est un \(\K\)-espace vectoriel de dimension quelconque, finie ou non.

\subsection{Valeurs propres et vecteurs propres}

\begin{defi}
Soient \(f\in\Lendo{E}\) et \(\lambda\in\K\).

On dit que \(\lambda\) est une valeur propre de \(f\) quand il existe un vecteur \(v\) non-nul tel que \(f\paren{v}=\lambda v\).

Si \(\lambda\) est une valeur propre de \(f\), alors tout vecteur non-nul \(v\) tel que \(f\paren{v}=\lambda v\) est appelé vecteur propre associé à la valeur propre \(\lambda\).
\end{defi}

\begin{rem}
Si \(f\paren{v}=\lambda v\) et \(v\not=0\) alors \(\quantifs{\tpt\alpha\not=0}f\paren{\alpha v}=\alpha f\paren{v}=\alpha\paren{\lambda v}=\lambda\paren{\alpha v}\). Donc \(\alpha v\) est un vecteur propre de \(f\) pour la valeur propre \(\lambda\).
\end{rem}

\begin{ex}
\begin{itemize}
    \item Pour tout \(\alpha\in\K\), \(\alpha\id{E}\) a pour unique valeur propre \(\alpha\) et tout vecteur non-nul de \(E\) est un vecteur propre associé. \\
    \item Si \(p\) est un projecteur non-trivial (\ie \(p\not=0\) et \(p\not=\id{E}\)), alors \(p\) a pour seules valeurs propres \(0\) et \(1\). \\
    \item De même, si \(s\) est une symétrie non-triviale (\ie \(s\not=\id{E}\) et \(s\not=-\id{E}\)), alors les valeurs propres de \(s\) sont \(1\) et \(-1\). \\
    \item L'endomorphisme de \(\poly\) \(P\mapsto XP\) n'a pas de valeur propre.
\end{itemize}
\end{ex}

L'ensemble des valeurs propres d'un endomorphisme \(f\) est appelé le spectre de \(f\) et est noté \(\Sp[\K]{f}\) ou plus simplement \(\Sp{f}\) (en toute rigueur, cette définition est fausse en dimension infinie, mais à notre niveau, cette approximation est acceptable).

\begin{defi}
On appelle droite propre d'un endomorphisme toute droite dirigée par un vecteur propre.
\end{defi}

\begin{prop}
Les droites propres d'un endomorphisme sont exactement les droites stables par cet endomorphisme.
\end{prop}

\begin{exo}
Soit \(f\in\Lendo{\R^\N}\) défini par : si \(\paren{u_n}\in\R^\N\), on pose \(f\paren{u}=\paren{u_{n+1}}\). Quelles sont les valeurs propres de \(f\) et les vecteurs propres associés ?
\end{exo}

\begin{exo}
Même question avec \(d\) l'opérateur de dérivation dans \(\ensclasse{\infty}{\R}{\R}\).
\end{exo}

\begin{exo}
Même question avec \(D\) l'opérateur de dérivation dans \(\poly[\R]\).
\end{exo}

\subsection{Lien avec les polynômes annulateurs}

En dimension quelconque, il est souvent difficile de trouver les valeurs propres d'un endomorphisme. La connaissance d'un polynôme annulateur peut aider.

\begin{lem}
Soient \(f\in\Lendo{E}\) et \(P\in\poly\). Si \(\lambda\) est une valeur propre de \(f\) et \(v\) un vecteur propre associé, alors \(P\paren{f}\paren{v}=P\paren{\lambda}v\).
\end{lem}

\begin{dem}
On montre par récurrence la propriété \(\P{k}\) : \guillemets{\(f^k\paren{v}=\lambda^kv\)}.

On a \(f^0\paren{v}=v=\lambda^0v\).

Si \(\P{k}\) est vraie, alors \[\begin{aligned}f^{k+1}\paren{v}&=ff^k\paren{v} \\
&=f\paren{\lambda^kv} \\
&=\lambda^kf\paren{v} \\
&=\lambda^k\lambda v \\
&=\lambda^{k+1}v.
\end{aligned}\]

D'où \(\P{k+1}\).

Par récurrence, \(\quantifs{\tpt k\in\N}\P{k}\text{ est vraie}\).

On écrit \(P=\sum_{i=0}^na_iX^i\).

Alors \(P\paren{f}=\sum_{i=0}^na_if^i\).

Donc \[\begin{aligned}
P\paren{f}\paren{v}&=\sum_{i=0}^na_if^i\paren{v} \\
&=\sum_{i=0}^na_i\paren{\lambda^iv} \\
&=v\sum_{i=0}^na_i\lambda^i \\
&=P\paren{\lambda}v.
\end{aligned}\]
\end{dem}

Si \(P\in\poly\), on note \(\rac{P}\) l'ensemble des racines de \(P\) dans \(\K\).

\begin{prop}\thlabel{prop:vpInclusesDansLesRacinesDesPolynômesAnnulateurs}
Soit \(f\in\Lendo{E}\).

Si \(P\) est un polynôme annulateur de \(f\), alors \(\Sp{f}\subset\rac{P}\).
\end{prop}

\begin{dem}
Il existe \(v\not=0\) tel que \(f\paren{v}=\lambda v\).

D'après le lemme précédent, \(P\paren{f}\paren{v}=P\paren{\lambda}v\).

Or \(P\paren{f}=0\) donc \(P\paren{\lambda}v=0\).

Or \(v\not=0\) donc \(P\paren{\lambda}=0\).

Donc \(\lambda\in\rac{P}\).
\end{dem}

\begin{rem}
Attention ! La réciproque est fausse. Contre-exemple : le polynôme \(P=X^2-1\) est annulateur de \(\id{E}\) et pourtant \(-1\), qui est racine de \(P\), n'est pas valeur propre de \(\id{E}\).
\end{rem}

\begin{exo}
Soit \(n\geq2\). Pour \(M\in\M{n}\), on pose \(f\paren{M}=M+\trans{M}+\tr\paren{M}I_n\) : \(f\) est clairement un endomorphisme de \(\M{n}\).

Déterminez un polynôme annulateur de \(f\) de degré \(3\) et déduisez-en les valeurs propres de \(f\).
\end{exo}

\subsection{Sous-espaces propres}

\begin{prop}\thlabel{prop:vpSsiKerNonNul}
Soient \(f\in\Lendo{E}\) et \(\lambda\in\K\).

Alors \(\lambda\) est valeur propre de \(f\) ssi \(\ker\paren{f-\lambda\id{E}}\not=\accol{0}\), autrement dit ssi \(f-\lambda\id{E}\) n'est pas injectif.
\end{prop}

\begin{dem}
On a \[\begin{aligned}
\lambda\in\Sp{f}&\ssi\quantifs{\exists v\in E}v\not=0\text{ et }f\paren{v}=\lambda v \\
&\ssi\quantifs{\exists v\in E}v\not=0\text{ et }f\paren{v}-\lambda v=0 \\
&\ssi\quantifs{\exists v\in E}v\not=0\text{ et }\paren{f-\lambda\id{E}}\paren{v}=0 \\
&\ssi\quantifs{\exists v\in E}v\not=0\text{ et }v\in\ker\paren{f-\lambda\id{E}} \\
&\ssi\ker\paren{f-\lambda\id{E}}\not=\accol{0} \\
&\ssi f-\lambda\id{E}\text{ non-injective}.
\end{aligned}\]
\end{dem}

\begin{defi}
Soit \(f\in\Lendo{E}\).

Si \(\lambda\in\Sp{f}\), le noyau \(\ker\paren{f-\lambda\id{E}}\) est appelé le sous-espace propre associé à la valeur propre \(\lambda\). Il est souvent noté \(\sep{f}{\lambda}\).
\end{defi}

Par conséquent, \(\sep{f}{\lambda}\) est l'ensemble des vecteurs propres associés à la valeur propre \(\lambda\) auquel on ajoute le vecteur nul.

\begin{rem}
Un cas particulier important : \(0\) est valeur propre ssi \(f\) n'est pas injective.
\end{rem}

\begin{exo}
Soit \(u\) un endomorphisme ayant pour matrice \(M=\begin{pmatrix}
-3 & 4 & -4 \\
4 & -3 & 3 \\
4 & -4 & 4
\end{pmatrix}\) dans une certaine base \(\fami{B}\).

Calculez \(M^3+2M^2-3M\). Déduisez-en les valeurs propres de \(u\) puis déterminez les sous-espaces propres associés.
\end{exo}

\begin{prop}
Tout sous-espace propre d'un endomorphisme est stable par cet endomorphisme. L'endomorphisme induit sur un sous-espace propre est alors une homothétie.
\end{prop}

\begin{dem}
Soit \(v\in\sep{f}{\lambda}\).

On a \(f\paren{v}=\lambda v\).

Donc \(f\paren{f\paren{v}}=\lambda f\paren{v}\).

Donc \(f\paren{v}\in\sep{f}{\lambda}\).

Donc le sous-espace propre \(\sep{f}{\lambda}\) est stable par \(f\).

De plus, l'endomorphisme induit par \(f\) sur ce sous-espace est \[\fonctionlambda{\sep{f}{\lambda}}{\sep{f}{\lambda}}{v}{f\paren{v}=\lambda v}\] \ie l'homothétie de rapport \(\lambda\).
\end{dem}

\begin{theo}
Soient \(f\in\Lendo{E}\) et \(\lambda_1,\dots,\lambda_p\) des valeurs propres distinctes de \(f\).

Alors les sous-espaces propres \(\paren{\sep{f}{\lambda_i}}_{1\leq i\leq p}\) sont en somme directe.

Autrement dit, toute famille de vecteurs propres associés à des valeurs propres distinctes est libre.
\end{theo}

\begin{dem}~\\
Soit \(\paren{v_1,\dots,v_p}\in\prod_{i=1}^p\sep{f}{\lambda_i}\) tel que \(v_1+\dots+v_p=0\quad\text{(1)}\).

On veut montrer que \(v_1=\dots=v_p=0\).

On applique \(f\) à (1) : \(f\paren{v_1}+\dots+f\paren{v_p}=0\) \ie \(\lambda_1v_1+\dots+\lambda_pv_p=0\).

On réitère \(p-2\) fois et on obtient le système suivant : \[\begin{dcases}
v_1+\dots+v_p=0 \\
\lambda_1v_1+\dots+\lambda_pv_p=0 \\
\lambda_1^2v_1+\dots+\lambda_p^2v_p=0 \\
\vdots \\
\lambda_1^{p-1}v_1+\dots+\lambda_p^{p-1}v_p=0
\end{dcases}\]

La matrice de ce système linéaire est \[\begin{pmatrix}
1 & \dots & 1 \\
\lambda_1 & \dots & \lambda_p \\
\lambda_1^2 & \dots & \lambda_p^2 \\
\vdots &  & \vdots \\
\lambda_1^{p-1} & \dots & \lambda_p^{p-1}
\end{pmatrix}\] \ie une matrice de Vandermonde inversible car les \(\lambda_i\) sont distincts donc le système a une unique solution \(\paren{v_1,\dots,v_p}=\paren{0,\dots,0}\).
\end{dem}

\begin{rem}
Quand on demande de déterminer les éléments propres d'un endomorphisme, on demande de déterminer les valeurs propres et les vecteurs propres associés, \ie les sous-espaces propres.
\end{rem}

\begin{center}
\bfseries
\fbox{À partir de maintenant, il est toujours supposé que \(E\) est de dimension finie \(n\)}
\end{center}

\section{Polynôme caractéristique d'un endomorphisme}

\subsection{Caractérisation des valeurs propres en dimension finie}

\begin{prop}
Soient \(f\in\Lendo{E}\) et \(\lambda\in\K\). Alors \[\lambda\in\Sp{f}\ssi\rg\paren{f-\lambda\id{E}}<n.\]

Dans ce cas, \(\dim\sep{f}{\lambda}=n-\rg\paren{f-\lambda\id{E}}\).
\end{prop}

\begin{dem}
D'après le théorème du rang, on a \[n=\underbrace{\dim\ker\paren{f-\lambda\id{E}}}_{=\dim\sep{f}{\lambda}}+\rg\paren{f-\lambda\id{E}}.\]

Donc \(\dim\sep{f}{\lambda}=n-\rg\paren{f-\lambda\id{E}}\).

On obtient l'inégalité voulue grâce à la \thref{prop:vpSsiKerNonNul}.
\end{dem}

\subsection{Définition et lien avec les valeurs propres}

\begin{defi}
Soit \(f\in\Lendo{E}\).

On appelle polynôme caractéristique de \(f\) le polynôme \(\chi_f=\det\paren{X\id{E}-f}\).
\end{defi}

La notation \(\chi_f\) est très courante : elle est à connaître.

\begin{theo}\thlabel{theo:chiUnitaireEtRacinesEgalesVP}
Soit \(f\in\Lendo{E}\).

Alors \(\chi_f\) est un polynôme unitaire de degré \(n\) de \(\poly\) et les valeurs propres de \(f\) sont exactement les racines dans \(\K\) de \(\chi_f\) : \(\rac{\chi_f}=\Sp{f}\).

Par conséquent, un endomorphisme d'un espace de dimension \(n\) a au plus \(n\) valeurs propres distinctes.
\end{theo}

\begin{dem}\thlabel{dem:vpEgalRacinesChi}
On choisit une base \(\fami{B}\) de \(E\) et on pose \(A=\paren{a_{i\,j}}\in\M{n}\) la matrice de \(f\) dans \(\fami{B}\).

On a \[\begin{aligned}
\chi_f&=\det\paren{X\id{E}-f} \\
&=\det\paren{XI_n-A} \\
&=\begin{vmatrix}
X-a_{1\,1} & -a_{1\,2} & \dots & -a_{1\,n} \\
-a_{2\,1} & X - a_{2\,2} & \ddots & \vdots \\
\vdots & \ddots & \ddots & -a_{n-1\,n} \\
-a_{n\,1} & \dots & -a_{n\,n-1} & X-a_{n\,n}
\end{vmatrix}
\end{aligned}\]

On pose \(c_{i\,j}=\begin{dcases}
-a_{i\,j} &\text{si }i\not=j \\
X-a_{i\,i} &\text{sinon}
\end{dcases}\)

Alors \[\begin{aligned}
\chi_f&=\sum_{\sigma\in\S{n}}\sig{\sigma}\prod_{i=1}^nc_{i\,\sigma\paren{i}} \\
&=\sig{\id{}}\prod_{i=1}^nc_{i\,i}+\sum_{\sigma\in\S{n}\excluant\accol{\id{}}}\sig{\sigma}\prod_{i=1}^nc_{i\,\sigma\paren{i}}
\end{aligned}\]

On remarque que si \(\sigma\in\S{n}\) alors \(\sigma\) a \(n\) points fixes si \(\sigma=\id{}\) et \(\sigma\) a moins de \(n-2\) points fixes sinon donc si \(\sigma\not=\id{}\), il existe au moins deux entiers \(i,j\in\interventierii{1}{n}\) tels que \(\sigma\paren{i}\not=i\) et \(\sigma\paren{j}\not=j\).

Donc pour toute permutation \(\sigma\not=\id{}\), parmi les facteurs du produit \(\prod_{i=1}^na_{i\,\sigma\paren{i}}\), il en existe au moins deux qui sont de la forme \(a_{?\,?}\) donc \(\prod_{i=1}^nc_{i\,\sigma\paren{i}}\) est un polynôme de degré au plus \(n-2\).

Donc \(\deg\chi_f=n\) et \(\chi_f\) est unitaire.

De plus, on a \[\begin{WithArrows}
\lambda\in\Sp{f}&\ssi f-\lambda\id{E}\text{ n'est pas injectif} \Arrow{dimension finie} \\
&\ssi f-\lambda\id{E}\text{ n'est pas bijectif} \\
&\ssi\det\paren{f-\lambda\id{E}}=0 \\
&\ssi\det\paren{\lambda\id{E}-f}=0 \\
&\ssi\chi_f\paren{\lambda}=0.
\end{WithArrows}\]
\end{dem}

\begin{exo}
Montrez que si \(\dim E=2\), alors \(\quantifs{\tpt f\in\Lendo{E}}\chi_f=X^2-\tr\paren{f}X+\det f\).
\end{exo}

\begin{exo}
Calculez le polynôme caractéristique d'un endomorphisme de matrice \(\begin{pmatrix}
1 & 4 & 7 \\
2 & 5 & 8 \\
3 & 6 & 9
\end{pmatrix}\) et donnez ses valeurs propres.
\end{exo}

\begin{exo}
Soient \(\fami{B}=\paren{e_1,\dots,e_n}\) une base de \(E\), \(s=\sum_{i=1}^ne_i\) et \(f\in\Lendo{E}\) tel que \(\quantifs{\tpt j\in\interventierii{1}{n}}f\paren{e_j}=e_j+s\).

Calculez son polynôme caractéristique et ses éléments propres.
\end{exo}

On peut noter un lien avec la trace et le déterminant.

\begin{prop}
Soit \(f\in\Lendo{E}\).

Alors \(\chi_f=X^n-\tr\paren{f}X^{n-1}+\dots+\paren{-1}^n\det f\).
\end{prop}

\begin{dem}
\begin{itemize}
    \item On a \(\chi_f=\det\paren{X\id{E}-f}\) donc \[\chi_f\paren{0}=\det\paren{-f}=\paren{-1}^n\det f\] est le coefficient constant de \(\chi_f\). \\
    \item On avait \(\chi_f=\prod_{i=1}^n\paren{X-a_{i\,i}}+Q\) avec \(\deg Q\leq n-2\) (\cf \thref{dem:vpEgalRacinesChi}). \\ Donc le coefficient d'indice \(n-1\) est celui du produit \(\prod_{i=1}^n\paren{X-a_{i\,i}}\). \\ Or on a \[\begin{aligned}
        \paren{X-a_{1\,1}}\dots\paren{X-a_{n\,n}}&=X^n+\paren{-a_{1\,1}-\dots-a_{n\,n}}X^{n-1}+\dots \\
        &=X^n-\tr\paren{f}X^{n-1}+\dots.
    \end{aligned}\]
\end{itemize}
\end{dem}

\subsection{Ordre de multiplicité et dimension du sous-espace propre}

\begin{defi}
Soient \(f\in\Lendo{E}\) et \(\lambda\in\Sp{f}\).

On appelle ordre de multiplicité de la valeur propre \(\lambda\) son ordre de multiplicité en tant que racine de \(\chi_f\).
\end{defi}

\begin{prop}\thlabel{prop:chiEndInduitDiviseChiEnd}
Soient \(f\in\Lendo{E}\), \(F\) un sous-espace vectoriel de \(E\) stable par \(f\) et \(g\) l'endomorphisme induit par \(f\) dans \(F\).

Alors \(\chi_g\) divise \(\chi_f\).
\end{prop}

\begin{dem}
On choisit une base \(\fami{B}=\paren{e_1,\dots,e_p,e_{p+1},\dots,e_n}\) de \(E\) adaptée à \(F\) : \(\paren{e_1,\dots,e_p}\) est une base de \(F\).

Alors \[\Mat{f}=\begin{pmatrix}
A & ? \\
0 & B
\end{pmatrix}\] où \(A\in\M{p}\) et \(B\in\M{n-p}\).

On remarque que \(A=\Mat[\paren{e_1,\dots,e_p}]{g}\).

Alors \[\begin{aligned}
\chi_f&=\begin{vmatrix}
XI_p-A & -? \\
0 & XI_{n-p}-B
\end{vmatrix} \\
&=\underbrace{\det\paren{XI_p-A}}_{=\chi_g}\det\paren{XI_{n-p}-B}.
\end{aligned}\]

Donc \(\chi_g\divise\chi_f\).
\end{dem}

Une conséquence très importante de ce résultat est le théorème suivant.

\begin{theo}\thlabel{theo:dimSepInferieurALaMultiplicite}
Soient \(f\in\Lendo{E}\) et \(\lambda\in\Sp{f}\).

Si \(\lambda\) est une valeur propre d'ordre \(\alpha\), alors \(1\leq\dim\sep{f}{\lambda}\leq\alpha\).
\end{theo}

\begin{dem}
Si \(\lambda\in\Sp{f}\) alors \(\sep{f}{\lambda}\) est stable par \(f\) et l'endomorphisme induit par \(f\) dans \(\sep{f}{\lambda}\) est l'homothétie de rapport \(\lambda\) : \(g=\lambda\id{}\).

On note \(p=\dim\sep{f}{\lambda}\).

On a \[\chi_g=\begin{vmatrix}
X-\lambda & 0 & \dots & 0 \\
0 & \ddots & \ddots & \vdots \\
\vdots & \ddots & \ddots & 0 \\
0 & \dots & 0 & X-\lambda
\end{vmatrix}_{\croch{p}}=\paren{X-\lambda}^p.\]

D'après la \thref{prop:chiEndInduitDiviseChiEnd}, \(\paren{X-\lambda}^p\divise\chi_f\) donc \(p\leq\alpha\).

De plus, on a \(1\leq p\) car \(\sep{f}{\lambda}\not=\accol{0}\).
\end{dem}

\begin{exo}
Soit \(f\) un endomorphisme de matrice \(\begin{pmatrix}
3 & -4 & -5 \\
-1 & 3 & 2 \\
1 & -2 & -1
\end{pmatrix}\). Déterminez les valeurs propres de \(f\), leur multiplicité et la dimension des sous-espaces propres associés.
\end{exo}

\subsection{Endomorphisme scindé}

\begin{defi}
On dit qu'un endomorphisme de \(E\) est scindé quand son polynôme caractéristique est scindé dans \(\poly\).
\end{defi}

Dans le cas d'un endomorphisme scindé, on connaît alors la somme et le produit des valeurs propres.

\begin{prop}\thlabel{prop:sommeEtProduitVPSiEndomorphismeScinde}
Si \(f\in\Lendo{E}\) est scindé et a pour valeurs propres \(\lambda_1,\dots,\lambda_p\) avec les ordres de multiplicité \(\alpha_1,\dots,\alpha_p\), alors \[\tr f=\sum_{k=1}^p\alpha_k\lambda_k\qquad\text{et}\qquad\det f=\prod_{k=1}^p\lambda_k^{\alpha_k}.\]
\end{prop}

\begin{dem}
Relations coefficients/racines.
\end{dem}

Si \(\K=\C\) alors on est dans ce cas, car tous les polynômes de \(\poly[\C]\) sont scindés dans \(\poly[\C]\) d'après le théorème de d'Alembert-Gauss.

Mais si \(\K=\R\), alors il faut se méfier des raisonnements hâtifs : comme un \(\R\)-endomorphisme peut ne pas avoir de valeurs propres réelles, la trace et le déterminant peuvent ne pas avoir de rapport avec les valeurs propres.

\begin{exo}
Soit \(f\) un endomorphisme d'un \(\C\)-espace vectoriel de dimension \(n\geq2\) dont la matrice dans une base est remplie par ligne de \(1\), ligne de \(2\), etc. Sans calculer le polynôme caractéristique, déterminez les valeurs propres complexes de \(f\), leur multiplicité et la dimension des sous-espaces propres associés.
\end{exo}

\begin{rem}
Dans le langage courant, on dit souvent que la trace est la somme des valeurs propres. Cette phrase est correcte seulement si l'on sous-entend que l'on parle de la somme des valeurs propres comptées chacune avec son ordre de multiplicité.

On rencontre en fait deux types de résultats à propos des valeurs propres :

\begin{itemize}
    \item ceux où l'on parle des valeurs propres distinctes (comme le \thref{theo:chiUnitaireEtRacinesEgalesVP}) ; \\
    \item ceux où l'on parle des valeurs propres comptées selon leur multiplicité (comme la \thref{prop:sommeEtProduitVPSiEndomorphismeScinde}).
\end{itemize}

Il faut donc être très attentif à la façon dont on considère les valeurs propres.
\end{rem}

\section{Éléments propres d'une matrice carrée}

Soit \(n\in\Ns\). Les matrices-colonnes d'ordre \(n\) sont les matrices de \(\M{n\,1}\), espace souvent identifié avec \(\K^n\).

\subsection{Valeurs propres et vecteurs propres}

\begin{defi}
Soient \(A\in\M{n}\) et \(\lambda\in\K\).

On dit que \(\lambda\) est valeur propre de \(A\) quand il existe une matrice-colonne \(X\) non-nulle telle que \(AX=\lambda X\).

Si \(\lambda\) est une valeur propre de \(A\), alors toute matrice-colonne non-nulle \(X\) telle que \(AX=\lambda X\) est appelée vecteur propre associé à la valeur propre \(\lambda\).
\end{defi}

\begin{ex}
\begin{itemize}
    \item Pour tout \(\alpha\in\K\), \(\alpha I_n\) a pour unique valeur propre \(\alpha\) et toute matrice-colonne non-nulle est un vecteur propre associé. \\
    \item Si \(A\) est une matrice diagonale, alors ses valeurs propres sont les coefficients diagonaux et des vecteurs propres associés sont les colonnes remplies de \(0\) sauf un seul coefficient égal à \(1\).
\end{itemize}
\end{ex}

L'ensemble des valeurs propres d'une matrice \(A\) est appelé le spectre de \(A\) et est noté \(\Sp[\K]{A}\) ou plus simplement \(\Sp{A}\).

Mais comme une matrice à coefficients réels est aussi une matrice à coefficients complexes, il vaut mieux savoir si on parle des valeurs propres réelles ou complexes. Il est donc préférable d'indiquer clairement le corps de base, comme le montre le résultat suivant.

\begin{prop}
Soient \(A\in\M{n}\) et \(\K\prim\) une extension de \(\K\) dans \(\C\).

Alors \(\Sp[\K]{A}\subset\Sp[\K\prim]{A}\).
\end{prop}

\begin{prop}
Soient \(A\in\M{n}\), \(f\in\Lendo{E}\) et \(\fami{B}\) une base de \(E\).

Si \(A=\Mat{f}\), alors \(\Sp[\K]{A}=\Sp{f}\).
\end{prop}

Par conséquent, deux matrices semblables ont les mêmes valeurs propres (mais attention, pas forcément les mêmes vecteurs propres).

\subsection{Lien avec les polynômes annulateurs}

\begin{prop}
Soit \(A\in\M{n}\).

Si \(P\) est un polynôme annulateur de \(A\), alors \(\Sp[\K]{A}\subset\rac{P}\).
\end{prop}

Attention ! La réciproque est fausse. Contre-exemple : le polynôme \(P=X^2-1\) est annulateur de \(I_n\) et pourtant \(-1\), qui est racine de \(P\), n'est pas valeur propre de \(I_n\).

\subsection{Sous-espaces propres}

\begin{prop}
Soient \(A\in\M{n}\) et \(\lambda\in\K\).

Alors \(\lambda\) est valeur propre de \(A\) ssi \(A-\lambda I_n\) n'est pas inversible, autrement dit ssi \(\rg\paren{A-\lambda I_n}<n\) ou \(\det\paren{A-\lambda I_n}=0\).
\end{prop}

Si \(\lambda\in\Sp[\K]{A}\), le sous-espace propre associé à la valeur propre \(\lambda\) est l'ensemble des vecteurs propres associés à la valeur propre \(\lambda\) auquel on ajoute le vecteur nul. Il est souvent noté \(\sep[\K]{A}{\lambda}\) : \[\sep[\K]{A}{\lambda}=\accol{X\in\M{n\,1}\tq AX=\lambda X}.\]

\begin{prop}
Soient \(A\in\M{n}\) et \(\lambda\in\K\). Alors \[\lambda\in\Sp[\K]{A}\ssi\rg\paren{A-\lambda I_n}<n.\]

Dans ce cas, \(\dim\sep[\K]{A}{\lambda}=n-\rg\paren{A-\lambda I_n}\).
\end{prop}

Attention ! Dans la relation \(\dim\sep[\K]{A}{\lambda}=n-\rg\paren{A-\lambda I_n}\), c'est \(n\), pas \(n^2\) ! Il s'agit de la dimension de \(\M{n\,1}\), pas celle de \(\M{n}\).

\begin{rem}
Un cas particulier important : \(0\) est valeur propre ssi \(A\) n'est pas inversible, \cad ssi \(\rg A<n\).
\end{rem}

\begin{theo}
Soient \(A\in\M{n}\) et \(\lambda_1,\dots,\lambda_p\) des valeurs propres distinctes de \(A\).

Alors les sous-espaces propres \(\paren{\sep[\K]{A}{\lambda_i}}_{1\leq i\leq p}\) sont en somme directe.

Autrement dit, toute famille de vecteurs propres associés à des valeurs propres distinctes est libre.
\end{theo}

\begin{rem}
Quand on demande de déterminer les éléments propres d'une matrice, on demande de déterminer les valeurs propres et les vecteurs propres associés, \ie les sous-espaces propres.
\end{rem}

\section{Polynôme caractéristique d'une matrice carrée}

\subsection{Définition et lien avec les valeurs propres}

\begin{defi}
Soit \(A\in\M{n}\).

On appelle polynôme caractéristique de \(A\) le polynôme \(\chi_A=\det\paren{XI_n-A}\).
\end{defi}

\begin{prop}
Soient \(A\in\M{n}\), \(f\in\Lendo{E}\) et \(\fami{B}\) une base de \(E\).

Si \(A=\Mat{f}\), alors \(\chi_A=\chi_f\).
\end{prop}

Par conséquent, deux matrices semblables ont le même polynôme caractéristique.

\begin{theo}
Soit \(A\in\M{n}\).

Alors \(\chi_A\) est un polynôme unitaire de degré \(n\) de \(\poly\) et les valeurs propres de \(A\) sont exactement les racines de \(\chi_A\) dans \(\K\).

Par conséquent, une matrice carrée de taille \(\paren{n,n}\) a au plus \(n\) valeurs propres distinctes.
\end{theo}

\begin{cor}
L'ensemble \(\GL{n}\) est dense dans \(\M{n}\).
\end{cor}

\begin{dem}
Soit \(A\in\M{n}\).

On veut montrer qu'il existe une suite de matrices inversibles qui converge vers \(A\).

Considérons la suite \(\paren{A+\dfrac{1}{k}I_n}_{k\in\Ns}\).

On a \(\lim_{k\to\pinf}\paren{A+\dfrac{1}{k}I_n}=A\).

Montrons qu'à partir d'un certain rang, cette suite est constituée de matrices inversibles.

\(\quantifs{\Tpt k\in\Ns}A+\dfrac{1}{k}I_n\text{ n'est pas inversible}\ssi\dfrac{-1}{k}\text{ est valeur propre de }A\).

\begin{itemize}
    \item Si \(A\) n'a que des valeurs propres positives ou nulles, alors comme \(\quantifs{\tpt k\in\Ns}\dfrac{-1}{k}<0\), \(\dfrac{-1}{k}\) n'est pas valeur propre. \\
    \item Si \(A\) possède au moins une valeur propre strictement négative, on pose \(r=\min\accol{\abs{\lambda}\tq\lambda\in\Sp{A}\inter\Rms}>0\). \\ Dès que \(\dfrac{1}{k}<r\), il est certain que \(\dfrac{-1}{k}\) n'est pas valeur propre.
\end{itemize}
\end{dem}

On peut noter un lien avec la trace et le déterminant.

\begin{prop}
Soit \(A\in\M{n}\).

Alors \(\chi_A=X^n-\tr\paren{A}X^{n-1}+\dots+\paren{-1}^n\det A\).
\end{prop}

\subsection{Ordre de multiplicité et dimension du sous-espace propre}

\begin{defi}
Soient \(A\in\M{n}\) et \(\lambda\in\Sp[\K]{A}\).

On appelle ordre de multiplicité de la valeur propre \(\lambda\) son ordre de multiplicité en tant que racine de \(\chi_A\).
\end{defi}

\begin{theo}
Soient \(A\in\M{n}\) et \(\lambda\in\Sp[\K]{A}\).

Si \(\lambda\) est une valeur propre d'ordre \(\alpha\), alors \(1\leq\dim\sep[\K]{A}{\lambda}\leq\alpha\).
\end{theo}

\begin{prop}
Soient \(A\in\M{n}\), \(f\in\Lendo{E}\) et \(\fami{B}\) une base de \(E\).

Si \(A=\Mat{f}\), alors \(\dim\sep[\K]{A}{\lambda}=\dim\sep{f}{\lambda}\).
\end{prop}

Par conséquent, deux matrices semblables ont des sous-espaces propres de même dimension (mais pas les mêmes vecteurs propres).

\subsection{Matrice scindée}

\begin{defi}
On dit qu'une matrice de \(\M{n}\) est scindée quand son polynôme caractéristique est scindé dans \(\poly\).
\end{defi}

Dans le cas d'une matrice scindée, on connaît alors la somme et le produit des valeurs propres.

\begin{prop}
Si \(A\in\M{n}\) est scindée et a pour valeurs propres \(\lambda_1,\dots,\lambda_p\) avec les ordres de multiplicité \(\alpha_1,\dots,\alpha_p\), alors \[\tr A=\sum_{k=1}^p\alpha_k\lambda_k\qquad\text{et}\qquad\det A=\prod_{k=1}^p\lambda_k^{\alpha_k}.\]
\end{prop}

Si \(\K=\C\), alors on est dans ce cas, car tous les polynômes de \(\poly[\C]\) sont scindés dans \(\poly[\C]\) d'après le théorème de d'Alembert-Gauss.

Mais si \(\K=\R\), alors il faut se méfier des raisonnements hâtifs : comme un polynôme à coefficients réels peut ne pas avoir de racines réelles, la trace et le déterminant peuvent ne pas avoir de rapport avec les valeurs propres.

\section{Endomorphismes diagonalisables, matrices diagonalisables}

\subsection{Définition}

\begin{defi}
Soient \(f\in\Lendo{E}\) et \(A\in\M{n}\).

On dit que \(f\) est diagonalisable quand il existe une base de \(E\) constituée de vecteurs propres de \(f\).

On dit que \(A\) est diagonalisable dans \(\M{n}\) (ou \(\K\)-diagonalisable) quand il existe une base de \(\M{n\,1}\) constituée de vecteurs propres de \(A\).
\end{defi}

D'après le lien entre les endomorphismes et les matrices carrées, un endomorphisme est diagonalisable ssi sa matrice dans n'importe quelle base est diagonalisable.

\begin{exo}~\\
La matrice \(\begin{pmatrix}
1 & \sqrt{3} \\
-\sqrt{3} & 1
\end{pmatrix}\) est-elle \(\R\)-diagonalisable ? \(\C\)-diagonalisable ?
\end{exo}

\begin{exo}
Montrez que la matrice \(A=\begin{pmatrix}
5 & -8 & -4 \\
8 & -15 & -8 \\
-10 & 20 & 11
\end{pmatrix}\) est diagonalisable.
\end{exo}

\begin{exo}\thlabel{exo:montrezQueBEstDiagonalisable}
Même exercice avec \(B=\begin{pmatrix}
0 & 1 & -1 \\
2 & 1 & 1 \\
4 & -2 & 4
\end{pmatrix}\).
\end{exo}

\begin{exo}
La matrice \(C=\begin{pmatrix}
11 & 7 & -3 \\
11 & 7 & -3 \\
66 & 42 & -18
\end{pmatrix}\) est-elle diagonalisable ?
\end{exo}

\begin{prop}
Si un endomorphisme (une matrice) est diagonalisable, alors il (elle) est scindé(e).
\end{prop}

Mais la réciproque est fausse.

\subsection{Caractérisations équivalentes}

On note \(\diago{n}\) l'ensemble des matrices diagonales de \(\M{n}\).

\begin{prop}
Soient \(f\in\Lendo{E}\) et \(A\in\M{n}\).

\(f\) est diagonalisable ssi il existe une base \(\fami{B}\) de \(E\) telle que \(\Mat{f}\in\diago{n}\). Dans ce cas, les valeurs propres de \(f\) sont les éléments diagonaux de cette matrice.

\(A\) est \(\K\)-diagonalisable ssi elle est \(\K\)-semblable à une matrice diagonale, \ie il existe \(P\in\GL{n}\) et \(D\in\diago{n}\) tel que \(A=PDP\inv\). Dans ce cas, les valeurs propres de \(A\) sont les éléments diagonaux de \(D\).
\end{prop}

\begin{dem}
Si \(f\) est diagonalisable, il existe une base \(\fami{B}=\paren{e_1,\dots,e_n}\) de \(E\) constituée de vecteurs propres, \ie \[\quantifs{\tpt j\in\interventierii{1}{n}}f\paren{e_j}=\lambda_je_j\] où \(\lambda_j\) est la valeur propre associée à \(e_j\).

Donc \[\Mat{f}=\begin{pmatrix}
\lambda_1 & 0 & \dots & 0 \\
0 & \ddots & \ddots & \vdots \\
\vdots & \ddots & \ddots & 0 \\
0 & \dots & 0 & \lambda_n
\end{pmatrix}\in\diago{n}.\]

Et réciproquement.
\end{dem}

\begin{ex}
\begin{itemize}
    \item Toute matrice diagonale est diagonalisable, car elle est semblable à elle-même. \\
    \item Les projecteurs et les symétries sont diagonalisables.
\end{itemize}
\end{ex}

\begin{rem}
Quitte à changer l'ordre des vecteurs dans la base, on peut ranger les valeurs propres sur la diagonale dans l'ordre qu'on veut.
\end{rem}

\begin{ex}~\\
Si \(D=\begin{pmatrix}
1 & 0 & 0 \\
0 & 3 & 0 \\
0 & 0 & 3
\end{pmatrix}\), \(P=\begin{pmatrix}
1 & 1 & 1 \\
0 & 1 & 1 \\
2 & -1 & 3
\end{pmatrix}\) et \(D=P\inv AP\), alors la colonne 1 de \(P\) est un vecteur propre de \(A\) pour la valeur propre \(1\) et les deux autres sont des vecteurs propres pour la valeur propre \(3\), donc en posant \(Q=\begin{pmatrix}
1 & 1 & 1 \\
1 & 1 & 0 \\
3 & -1 & 2
\end{pmatrix}\), on a \(Q\inv AQ=\begin{pmatrix}
3 & 0 & 0 \\
0 & 3 & 0 \\
0 & 0 & 1
\end{pmatrix}\).
\end{ex}

\begin{lem}
Soit \(f\in\Lendo{E}\) diagonalisable : il existe une base de \(E\) dans laquelle la matrice \(D\) de \(f\) est diagonale.

Les valeurs propres de \(f\) sont les éléments diagonaux de \(D\) et si \(\lambda\) est un tel nombre, alors la dimension de \(\sep{f}{\lambda}\) est le nombre d'occurrences de \(\lambda\) dans la diagonale de \(D\).
\end{lem}

On en déduit les théorèmes suivants.

\begin{theo}\thlabel{theo:fDiagonalisableSsiSepSupplémentaires}
Soit \(f\in\Lendo{E}\).

Il y a équivalence entre les propositions suivantes :

\begin{itemize}
    \item \(f\) est diagonalisable \\
    \item les sous-espaces propres de \(f\) sont supplémentaires dans \(E\) \\
    \item \(\sum_{\lambda\in\Sp{f}}\dim\sep{f}{\lambda}=n\)
\end{itemize}
\end{theo}

\begin{dem}
Les sous-espaces propres d'un endomorphisme sont en somme directe donc ils sont supplémentaires ssi la somme de leurs dimensions est celle de l'espace \(E\).
\end{dem}

Et sa version matricielle.

\begin{theo}
Soit \(A\in\M{n}\).

Il y a équivalence entre les propositions suivantes :

\begin{itemize}
    \item \(A\) est diagonalisable dans \(\M{n}\) \\
    \item les sous-espaces propres de \(A\) dans \(\M{n\,1}\) sont supplémentaires dans \(\M{n\,1}\) \\
    \item \(\sum_{\lambda\in\Sp[\K]{A}}\dim\sep[\K]{A}{\lambda}=n\)
\end{itemize}
\end{theo}

\begin{exo}~\\
On pose \(A=\begin{pmatrix}
0 & 1 & -1 \\
2 & 1 & 1 \\
4 & -2 & 4
\end{pmatrix}\). On a vu à l'\thref{exo:montrezQueBEstDiagonalisable} que \(A\) est diagonalisable. Diagonalisez \(A\).
\end{exo}

\subsection{Lien avec le polynôme caractéristique}

\begin{theo}
Soit \(f\in\Lendo{E}\).

Il y a équivalence entre les propositions suivantes :

\begin{itemize}
    \item \(f\) est diagonalisable \\
    \item \(f\) est scindé et pour tout \(\lambda\in\Sp{f}\), la dimension de \(\sep{f}{\lambda}\) est égale à l'ordre de multiplicité de \(\lambda\)
\end{itemize}
\end{theo}

\begin{dem}
Si \(\lambda\in\Sp{f}\), on note \(\omega\paren{\lambda}\) l'ordre de multiplicité de la valeur propre \(\lambda\).

\impdir

Si \(f\) est diagonalisable alors \(\sum_{\lambda\in\Sp{f}}\dim\sep{f}{\lambda}=n=\sum_{\lambda\in\Sp{f}}\omega\paren{\lambda}\).

Donc \(\sum_{\lambda\in\Sp{f}}\underbrace{\paren{\omega\paren{\lambda}-\dim\sep{f}{\lambda}}}_{\geq0\text{ d'après le \thref{theo:dimSepInferieurALaMultiplicite}}}=0\).

Or une somme de réels positifs est nulle ssi tous ces réels sont nuls donc \[\quantifs{\forall\lambda\in\Sp{f}}\omega\paren{\lambda}=\dim\sep{f}{\lambda}.\]

\imprec

Si \(f\) est scindé et \(\quantifs{\forall\lambda\in\Sp{f}}\omega\paren{\lambda}=\dim\sep{f}{\lambda}\), alors \(\chi_f\) est scindé.

Donc \(\sum_{\lambda\in\Sp{f}}\omega\paren{\lambda}=\deg\chi_f=n\).

Donc \(\sum_{\lambda\in\Sp{f}}\dim\sep{f}{\lambda}=n\).

Donc \(f\) est diagonalisable d'après le \thref{theo:fDiagonalisableSsiSepSupplémentaires}.
\end{dem}

Et sa version matricielle.

\begin{theo}
Soit \(A\in\M{n}\).

Il y a équivalence entre les propositions suivantes :

\begin{itemize}
    \item \(A\) est diagonalisable dans \(\M{n}\) \\
    \item \(A\) est scindée et pour tout \(\lambda\in\Sp[\K]{A}\), la dimension de \(\sep[\K]{A}{\lambda}\) est égale à l'ordre de multiplicité de \(\lambda\)
\end{itemize}
\end{theo}

Dans le cas où \(\K=\C\), la condition \guillemets{être scindé} est automatiquement satisfaite.

Un cas particulier très courant.

\begin{prop}
Si un endomorphisme de \(E\) possède exactement \(n\) valeurs propres distinctes, alors il est diagonalisable.

Si une matrice de \(\M{n}\) possède exactement \(n\) valeurs propres distinctes dans \(\K\), alors elle est diagonalisable dans \(\M{n}\).
\end{prop}

\begin{exo}
Montrez que la matrice \(\begin{pmatrix}
-4 & 8 & 22 \\
-2 & 3 & 4 \\
-1 & 2 & 7
\end{pmatrix}\) est diagonalisable.
\end{exo}

\begin{theo}[Théorème spectral]
Si \(A\) est une matrice réelle symétrique, alors \(A\) est diagonalisable.
\end{theo}

\begin{dem}
\note{Admis, sera démontré plus tard (\cf \thref{demtheo9.3})}
\end{dem}

\section{Lien entre diagonalisabilité et polynômes annulateurs}

\subsection{Racines du polynôme minimal}

\begin{prop}
Soit \(f\in\Lendo{E}\). Les racines de \(\mu_f\) sont exactement les valeurs propres de \(f\) : \(\rac{\mu_f}=\Sp{f}\).

Soit \(A\in\M{n}\). Les racines dans \(\K\) de \(\mu_A\) sont exactement les valeurs propres dans \(\K\) de \(A\) : \(\rac{\mu_A}=\Sp[\K]{A}\).
\end{prop}

\begin{dem}
\increc \Cf \thref{prop:vpInclusesDansLesRacinesDesPolynômesAnnulateurs} car \(\mu_f\) est un polynôme annulateur de \(f\).

\incdir

Soit \(\lambda\in\rac{\mu_f}\).

Alors \(X-\lambda\divise\mu_f\), \ie il existe \(Q\in\poly\) tel que \(\mu_f=\paren{X-\lambda}Q\).

Alors \(\mu_f\paren{f}=0=\paren{f-\lambda\id{E}}\rond Q\paren{f}\).

Donc \(\quantifs{\tpt x\in E}\paren{f-\lambda\id{E}}\paren{Q\paren{f}\paren{x}}=0\).

Donc \(\Im Q\paren{f}\subset\ker\paren{f-\lambda\id{E}}\).

Or \(\deg Q<\deg\mu_f\) donc \(Q\) n'est pas annulateur de \(f\), \ie \(Q\paren{f}\not=0\), \ie \(\Im Q\paren{f}\not=\accol{0}\).

Donc \(\ker\paren{f-\lambda\id{E}}\not=\accol{0}\), \ie \(\lambda\in\Sp{f}\).

Donc \(\rac{\mu_f}\subset\Sp{f}\).
\end{dem}

\subsection{Lemme des noyaux}

\begin{prop}\thlabel{prop:lemmeDesNoyauxDeuxPolynômes}
Soient \(f\in\Lendo{E}\) et \(P,Q\in\poly\) tels que \(P\et Q=1\).

Alors \(\ker\paren{PQ}\paren{f}=\ker P\paren{f}\oplus\ker Q\paren{f}\).
\end{prop}

\begin{dem}
D'après le théorème de Bézout, il existe \(\paren{U,V}\in\poly^2\) tel que \(UP+VQ=1\).

Donc \(\paren{UP}\paren{f}+\paren{VQ}\paren{f}=\id{E}\), \ie \[U\paren{f}\rond P\paren{f}+V\paren{f}\rond Q\paren{f}=\id{E}\qquad\text{(1)}\]

Soit \(x\in\ker P\paren{f}\inter\ker Q\paren{f}\).

On a \(P\paren{f}\paren{x}=0\) et \(Q\paren{f}\paren{x}=0\).

Donc, en appliquant (1) sur le vecteur \(x\), on obtient \[\begin{aligned}
x&=U\paren{f}\paren{P\paren{f}\paren{x}}+V\paren{f}\paren{Q\paren{f}\paren{x}} \\
&=U\paren{f}\paren{0}+V\paren{f}\paren{0} \\
&=0.
\end{aligned}\]

Donc \(\ker P\paren{f}\) et \(\ker Q\paren{f}\) sont en somme directe.

\increc

On a \(\paren{PQ}\paren{f}=P\paren{f}\rond Q\paren{f}=Q\paren{f}\rond P\paren{f}\).

Donc \(\ker P\paren{f}\subset\ker\paren{PQ}\paren{f}\) et \(\ker Q\paren{f}\subset\ker\paren{PQ}\paren{f}\).

Donc \(\ker P\paren{f}\oplus\ker Q\paren{f}\subset\ker\paren{PQ}\paren{f}\).

\incdir

Soit \(x\in\ker\paren{PQ}\paren{f}\).

On veut montrer qu'il existe \(\paren{a,b}\in\ker P\paren{f}\times\ker Q\paren{f}\) tel que \(x=a+b\).

On applique (1) sur \(x\) : \[x=U\paren{f}\rond P\paren{f}\paren{x}+V\paren{f}\rond Q\paren{f}\paren{x}.\]

On pose \(a=V\paren{f}\rond Q\paren{f}\paren{x}\).

On a \[P\paren{f}\paren{a}=P\paren{f}\paren{V\paren{f}\rond Q\paren{f}\paren{x}}=\paren{P\paren{f}\rond V\paren{f}\rond Q\paren{f}}\paren{x}.\]

Or \(\poly[\K][f]\) est une algèbre commutative donc \[\begin{aligned}
P\paren{f}\paren{a}&=\paren{V\paren{f}\rond P\paren{f}\rond Q\paren{f}}\paren{x} \\
&=V\paren{f}\paren{P\paren{f}\rond Q\paren{f}\paren{x}} \\
&=V\paren{f}\paren{\paren{PQ}\paren{f}\paren{x}} \\
&=V\paren{f}\paren{0} \\
&=0.
\end{aligned}\]

Donc \(a\in\ker P\paren{f}\).

De même, \(b=U\paren{f}\rond P\paren{f}\paren{x}\in\ker Q\paren{f}\).

Finalement, on a \[\ker\paren{PQ}\paren{f}=\ker P\paren{f}\oplus\ker Q\paren{f}.\]
\end{dem}

\begin{prop}
Soient \(f\in\Lendo{E}\) et \(P_1,\dots,P_k\in\poly\) premiers entre eux deux à deux. On pose \(P=\prod_{i=1}^kP_i\).

Alors \(\ker P\paren{f}=\bigoplus_{i=1}^k\ker P_i\paren{f}\).
\end{prop}

\begin{dem}
On note \(\P{k}\) le prédicat énoncé.

\begin{itemize}
    \item On a clairement \(\P{1}\) et \(\P{2}\) est vraie (\cf \thref{prop:lemmeDesNoyauxDeuxPolynômes}). \\
    \item Soit \(k\in\Ns\) tel que \(\P{k}\) soit vraie. \\\\ Soient \(P_1,\dots,P_{k+1}\in\poly\) premiers entre eux deux à deux. \\\\ On a \(P_1\dots P_k\et P_{k+1}=1\). \\\\ D'après \(\P{2}\), on a \[\ker\paren{P_1\dots P_{k+1}}\paren{f}=\ker\paren{P_1\dots P_k}\paren{f}\oplus\ker P_{k+1}\paren{f}.\] \\\\ Puis, par hypothèse de récurrence, on a \[\ker\paren{P_1\dots P_k}\paren{f}=\bigoplus_{i=1}^k\ker P_i\paren{f}.\] \\\\ Finalement, on a \[\ker\paren{P_1\dots P_{k+1}}\paren{f}=\bigoplus_{i=1}^{k+1}P_i\paren{f}\] d'où \(\P{k+1}\). \\
    \item D'après le principe de récurrence, \(\quantifs{\tpt k\in\Ns}\P{k}\text{ est vraie}\).
\end{itemize}
\end{dem}

\subsection{Application à la diagonalisabilité}

\begin{defi}
Un polynôme est dit simplement scindé quand il est scindé et à racines simples.
\end{defi}

\begin{theo}\thlabel{theo:endDiagonalisableSsiPolynômeMinimalSimplementScindéSsiExistePolyAnnulateurSimplementScindéSsiProdXMoinsVpAnnulateur}
Soit \(f\in\Lendo{E}\).

Il y a équivalence entre les propositions suivantes :

\begin{enumerate}
    \item[(\(\alpha\))] \(f\) est diagonalisable \\
    \item[(\(\beta\))] \(\mu_f\) est simplement scindé \\
    \item[(\(\gamma\))] il existe un polynôme annulateur de \(f\) simplement scindé \\
    \item[(\(\delta\))] le polynôme \(\prod_{\lambda\in\Sp{f}}\paren{X-\lambda}\) est un polynôme annulateur de \(f\)
\end{enumerate}
\end{theo}

\begin{dem}[(\(\beta\)) \(\imp\) (\(\gamma\))]
Immédiat car \(\mu_f\) est annulateur de \(f\).
\end{dem}

\begin{dem}[(\(\gamma\)) \(\imp\) (\(\beta\))]
Si \(P\) est simplement scindé et \(P\paren{f}=0\) alors \(\mu_f\divise P\) donc \(\mu_f\) est simplement scindé.
\end{dem}

\begin{dem}[(\(\beta\)) \(\imp\) (\(\delta\))]
On sait que \(\rac{\mu_f}=\Sp{f}\) donc si \(\mu_f\) est simplement scindé, alors \(\mu_f=\prod_{\lambda\in\Sp{f}}\paren{X-\lambda}\). Or \(\mu_f\) est annulateur de \(f\).
\end{dem}

\begin{dem}[(\(\delta\)) \(\imp\) (\(\gamma\))]
Immédiat.
\end{dem}

\begin{dem}[(\(\alpha\)) \(\imp\) (\(\delta\))]
Supposons \(f\) diagonalisable, \ie il existe une base \(\fami{B}\) de \(E\) telle que \(\Mat{f}\) soit diagonale : \[\Mat{f}=\begin{pmatrix}
\Lambda_1 & 0 & \dots & 0 \\
0 & \ddots & \ddots & \vdots \\
\vdots & \ddots & \ddots & 0 \\
0 & \dots & 0 & \Lambda_k
\end{pmatrix}=D\] où \(\lambda_1,\dots,\lambda_k\) sont les valeurs propres distinctes de \(f\) et \(\quantifs{\tpt i\in\interventierii{1}{k}}\Lambda_i=\begin{pmatrix}
\lambda_i & 0 & \dots & 0 \\
0 & \ddots & \ddots & \vdots \\
\vdots & \ddots & \ddots & 0 \\
0 & \dots & 0 & \lambda_i
\end{pmatrix}\).

On pose \(P=\prod_{i=1}^k\paren{X-\lambda_i}\).

Or on a \(\quantifs{\tpt Q\in\poly}Q\paren{D}=\begin{pmatrix}
Q\paren{\Lambda_1} & 0 & \dots & 0 \\
0 & \ddots & \ddots & \vdots \\
\vdots & \ddots & \ddots & 0 \\
0 & \dots & 0 & Q\paren{\Lambda_k}
\end{pmatrix}\)

et \(\quantifs{\tpt Q\in\poly;\tpt i\in\interventierii{1}{k}}Q\paren{\Lambda_i}=\begin{pmatrix}
Q\paren{\lambda_i} & 0 & \dots & 0 \\
0 & \ddots & \ddots & \vdots \\
\vdots & \ddots & \ddots & 0 \\
0 & \dots & 0 & Q\paren{\lambda_i}
\end{pmatrix}\).

En particulier, \(P\paren{D}=0\) car \(\accol{\lambda_1,\dots,\lambda_k}=\rac{P}\).
\end{dem}

\begin{dem}[(\(\delta\)) \(\imp\) (\(\alpha\))]
On pose \(\Sp{f}=\accol{\lambda_1,\dots,\lambda_k}\) et \(\quantifs{\tpt i\in\interventierii{1}{k}}P_i=X-\lambda_i\).

Les polynômes \(P_1,\dots,P_k\) sont premiers entre eux deux à deux donc d'après le lemme des noyaux, on a \[\ker\underbrace{\paren{P_1\dots P_k}\paren{f}}_{=0}=\bigoplus_{i=1}^k\underbrace{\ker P_i\paren{f}}_{=\sep{f}{\lambda_i}}.\]

D'où \(E=\bigoplus_{i=1}^k\sep{f}{\lambda_i}\).

Donc d'après le \thref{theo:fDiagonalisableSsiSepSupplémentaires}, \(f\) est diagonalisable.
\end{dem}

Et sa version matricielle.

\begin{theo}
Soit \(A\in\M{n}\).

Il y a équivalence entre les propositions suivantes :

\begin{itemize}
    \item \(A\) est diagonalisable dans \(\M{n}\) \\
    \item \(\mu_A\) est simplement scindé \\
    \item il existe un polynôme annulateur de \(A\) simplement scindé dans \(\poly\) \\
    \item le polynôme \(\prod_{\lambda\in\Sp[\K]{A}}\paren{X-\lambda}\) est un polynôme annulateur de \(A\)
\end{itemize}
\end{theo}

\begin{exo}~\\
On pose \(A=\begin{pmatrix}
2 & -1 & 2 \\
5 & -3 & 3 \\
-1 & 0 & -2
\end{pmatrix}\). Calculez \(\paren{A+I_3}^3\). \(A\) est-elle diagonalisable ?
\end{exo}

\begin{exo}
Soit \(A\in\M{n}\) telle que \(A^3=I_n\). Selon que \(\K\) soit égal à \(\C\) ou \(\R\), à quelle condition \(A\) est-elle \(\K\)-diagonalisable ?
\end{exo}

\subsection{Diagonalisabilité d'un endomorphisme induit}

\begin{prop}
Soient \(f\in\Lendo{E}\), \(F\) un sous-espace vectoriel de \(E\) stable par \(f\) et \(g\) l'endomorphisme induit par \(f\) dans \(F\).

Alors \(\mu_g\) divise \(\mu_f\).
\end{prop}

\begin{dem}
On a \(\fonction{g}{F}{F}{x}{f\paren{x}}\)

\(\quantifs{\Tpt x\in F;\tpt P\in\poly}P\paren{g}\paren{x}=P\paren{f}\paren{x}\).

Or \(\mu_f\paren{f}=0\) donc \(\quantifs{\tpt x\in F}\mu_f\paren{g}\paren{x}=0\), \ie \(\mu_f\) est annulateur de \(g\).

Donc \(\mu_g\divise\mu_f\).
\end{dem}

\begin{cor}
Soient \(f\in\Lendo{E}\) et \(F\) un sous-espace vectoriel de \(E\) stable par \(f\).

Si \(f\) est diagonalisable, alors l'endomorphisme induit par \(f\) dans \(F\) est aussi diagonalisable.
\end{cor}

\begin{dem}
Si \(f\) est diagonalisable, d'après le \thref{theo:endDiagonalisableSsiPolynômeMinimalSimplementScindéSsiExistePolyAnnulateurSimplementScindéSsiProdXMoinsVpAnnulateur}, \(\mu_f\) est simplement scindé.

Or \(\mu_g\divise\mu_f\) donc \(\mu_g\) est simplement scindé.

Donc \(g\) est diagonalisable d'après le \thref{theo:endDiagonalisableSsiPolynômeMinimalSimplementScindéSsiExistePolyAnnulateurSimplementScindéSsiProdXMoinsVpAnnulateur}.
\end{dem}

\begin{rem}
On a également :

\begin{itemize}
    \item \(\Sp{g}=\rac{\mu_g}\subset\rac{\mu_f}=\Sp{f}\) \\
    \item si \(x\) est un vecteur propre de \(g\) pour la valeur propre \(\lambda\), \ie \(\begin{dcases}
        x\in F \\
        x\not=0 \\
        g\paren{x}=f\paren{x}=\lambda x
    \end{dcases}\) alors \(x\) est un vecteur propre de \(f\) dans \(F\), et réciproquement. \\\\ On a donc \(\sep{g}{\lambda}=\sep{f}{\lambda}\inter F\).
\end{itemize}
\end{rem}

\begin{exo}
Soit \(f\) un endomorphisme de matrice \(\begin{pmatrix}
1 & 1 & -1 \\
1 & 1 & 1 \\
1 & 1 & 1
\end{pmatrix}\) dans une base \(\fami{B}=\paren{e_1,e_2,e_3}\). Déterminez les sous-espaces vectoriels de \(E\) stables par \(f\).
\end{exo}

\begin{exo}[Codiagonalisation ou diagonalisation simultanée]
Soient \(A,B\in\M{n}\) diagonalisables et qui commutent.

Montez qu'il existe \(P\in\GL{n}\) telle que \(P\inv AP\) et \(P\inv BP\) sont diagonales.
\end{exo}

\section{Quelques applications de la diagonalisation}

\subsection{Puissances d'une matrice, suites récurrentes linéairement}

Un petit lemme déjà rencontré.

\begin{lem}
Soient \(A,B\in\M{n}\) et \(P\in\GL{n}\) telles que \(A=PBP\inv\).

Alors \(\quantifs{\tpt k\in\N}A^k=PB^kP\inv\).
\end{lem}

Le lemme précédent est particulièrement utile si \(A\) est diagonalisable et si on choisit \(B=D\), matrice diagonale semblable à \(A\), car calculer les puissances d'une matrice diagonale est très facile.

Grâce à la diagonalisation de \(A\), on peut espérer exprimer la forme générale des suites récurrentes linéaires (voir le chapitre précédent, section sur les polynômes annulateurs).

\begin{exo}
Soient \(u,v,w\) les trois suites réelles telles que \(u_0=v_0=w_0=1\) et \[\quantifs{\tpt n\in\N}\begin{dcases}
u_{n+1}=u_n-v_n \\
v_{n+1}=-4u_n+4v_n-6w_n \\
w_{n+1}=-3u_n+3v_n-4w_n
\end{dcases}\]

Déterminez des expressions de \(u_n,v_n,w_n\) en fonction de \(n\).
\end{exo}

Cette technique s'applique en particulier aux suites \(u\) vérifiant une relation de récurrence linéaire de la forme : \(\quantifs{\tpt n\in\N}u_{n+d}=a_{d-1}u_{n+d-1}+\dots+a_2u_{n+2}+a_1u_{n+1}+a_0u_n\).

On pose alors \(X_n=\begin{pmatrix}
u_n \\
u_{n+1} \\
\vdots \\
u_{n+d-1}
\end{pmatrix}\) et \(A=\begin{pmatrix}
0 & 1 & 0 & \dots & 0 \\
0 & 0 & 1 & \ddots & \vdots \\
\vdots & \vdots &  & \ddots & 0 \\
0 & 0 & 0 & \dots & 1 \\
a_0 & a_1 & a_2 & \dots & a_{d-1}
\end{pmatrix}\in\M{d}\).

Alors \(\quantifs{\tpt n\in\N}X_{n+1}=AX_n\) et on est ramené au cas précédent.

La matrice \(A\) s'appelle la matrice-compagnon du polynôme \(P=X^d-a_{d-1}X^{d-1}-\dots-a_1X-a_0\) : elle a la propriété remarquable que son polynôme caractéristique est \(P\), son polynôme minimal est aussi \(P\) et donc que ses valeurs propres sont les racines de \(P\). C'est pourquoi le polynôme \(P\) est appelé polynôme caractéristique associé à la suite \(u\) (cas déjà étudié en première année : \(d=2\)).

On en déduit que \(A\) est diagonalisable ssi \(P\) est simplement scindé et dans ce cas, \(A\) possède \(d\) valeurs propres distinctes. Dans ce cas, en notant \(\lambda_1,\dots,\lambda_p\) les valeurs propres distinctes, la suite \(u\) est combinaison linéaire des suites géométriques \(\paren{\lambda_1^n},\dots,\paren{\lambda_d^n}\).

\begin{exo}
Explicitez l'unique suite \(\paren{u_n}\) vérifiant \[u_0=0,u_1=1,u_2=5\qquad\text{et}\qquad\quantifs{\forall n\in\N}u_{n+3}=6u_{n+2}-11u_{n+1}+6u_n.\]
\end{exo}

\subsection{Systèmes d'équations différentielles}

Ce point sera traité dans le chapitre sur les équations différentielles linéaires.

\section{Endomorphismes trigonalisables, matrices trigonalisables}

\subsection{Définition et propriétés}

\begin{defi}
Un endomorphisme est dit trigonalisable quand il existe une base dans laquelle sa matrice est triangulaire supérieure.

Une matrice carrée de \(\M{n}\) est dite trigonalisable dans \(\M{n}\) quand elle est semblable à une matrice triangulaire dans \(\M{n}\).
\end{defi}

\begin{rem}
\begin{itemize}
    \item Si un endomorphisme (une matrice) est diagonalisable, alors il (elle) est trigonalisable. \\
    \item Si une matrice est trigonalisable, ses valeurs propres sont les nombres sur la diagonale de toute matrice triangulaire semblable.
\end{itemize}
\end{rem}

\begin{exo}
On considère la matrice \(M=\begin{pmatrix}
-2 & -1 & 7 \\
5 & 4 & -8 \\
1 & 1 & 1
\end{pmatrix}\) et \(f\) un endomorphisme de matrice \(M\). Déterminez les éléments propres de \(M\). Est-elle diagonalisable ? En complétant une famille libre de vecteurs propres, déterminez une base \(\fami{B}\) de l'espace où la matrice de \(f\) est triangulaire supérieure, puis trigonalisez \(M\).
\end{exo}

\begin{exo}
Soit \(f\) un endomorphisme de matrice \(A=\begin{pmatrix}
2 & -4 & -5 \\
-1 & 2 & 2 \\
1 & -2 & -2
\end{pmatrix}\). Montrez que \(f\) n'est pas diagonalisable mais est trigonalisable et donnez une base de trigonalisation de \(f\). Donnez une forme générale pour \(A^n\).
\end{exo}

Quand un endomorphisme ou une matrice n'est pas diagonalisable, on peut espérer qu'il ou elle est trigonalisable : faute de grives, on se contente de merles !

\begin{rem}
On ne confondra pas la trigonalisation d'une matrice carrée et la transformation par lignes (ou colonnes) des matrices vue en première année ! Seule la trigonalisation fournit des matrices semblables ! La transformation par lignes ne conserve que le rang !
\end{rem}

\subsection{Caractérisation équivalente}

La trigonalisabilité est beaucoup plus courante que la diagonalisabilité, comme on le voit grâce aux résultats suivants.

\begin{prop}
Un endomorphisme (une matrice) est trigonalisable ssi il (elle) est scindé(e).
\end{prop}

\begin{dem}
On pose \(\P{n}\) : \guillemets{si \(f\) est un endomorphisme d'un espace de dimension \(n\) et si \(\chi_f\) est scindé, alors \(f\) est trigonalisable}.

\begin{itemize}
    \item \(\P{1}\) est vraie car tout endomorphisme en dimension \(1\) est trigonalisable. \\
    \item Supposons \(\P{n-1}\). \\\\ Soient \(E\) un espace de dimension \(n\) et \(f\in\Lendo{E}\) tel que \(\chi_f\) soit scindé. \\\\ Comme \(\chi_f\) est scindé, il existe \(\lambda\in\K\) tel que \(\lambda\) soit racine de \(\chi_f\) et donc une valeur propre de \(f\) à laquelle on associe un vecteur propre \(u_1\). \\\\ Comme \(u_1\not=0\), d'après le théorème de la base incomplète, il existe \(\paren{u_2,\dots,u_n}\in E^{n-1}\) tel que \(\fami{B}_0=\paren{u_1,\dots,u_n}\) soit une base de \(E\). \\\\ On a \[\Mat[\fami{B}_0]{f}=\begin{pNiceArray}{c|cw{c}{1cm}c}[margin]
        \lambda & \Block{1-3}<\large>{L} & & \\
        \hline
        0 & \Block{3-3}<\large>{B} & & \\
        \Vdots & & & \\
        0 & & &
    \end{pNiceArray}.\] \\\\ Donc \[\chi_f=\begin{vNiceArray}{c|cw{c}{1.5cm}c}[margin]
        X-\lambda & \Block{1-3}<\large>{-L} & & \\
        \hline
        0 & \Block{3-3}<\large>{XI_{n-1}-B} & & \\
        \Vdots & & & \\
        0 & & &
    \end{vNiceArray}=\paren{X-\lambda}\chi_B.\] \\\\ On pose \(F=\Vect{u_2,\dots,u_n}\). \\\\ Soit \(g\in\Lendo{F}\) tel que \(\Mat[\paren{u_2,\dots,u_n}]{g}=B\). \\\\ On a \(\chi_g=\chi_B\) scindé donc par hypothèse de récurrence, \(g\) est trigonalisable : il existe une base \(\paren{u_2\prim,\dots,u_n\prim}\) de \(F\) telle que \(\Mat[\paren{u_2\prim,\dots,u_n\prim}]{g}=\begin{pmatrix}
        t_{2\,2} & t_{2\,3} & \dots & t_{2\,n} \\
        0 & \ddots & \ddots & \vdots \\
        \vdots & \ddots & \ddots & t_{n-1\,n} \\
        0 & \dots & 0 & t_{n\,n}
    \end{pmatrix}=T\). \\\\ La famille \(\fami{B}=\paren{u_1,u_2\prim,\dots,u_n\prim}\) est une base de \(E\). \\\\ On veut montrer que \[\Mat{f}=\begin{pmatrix}
        \lambda & \alpha_2 & \dots & & \alpha_n \\
        0 & t_{2\,2} & t_{2\,3} & \dots & t_{2\,n} \\
        \vdots & 0 & \ddots & \ddots & \vdots \\
        & \vdots & \ddots & \ddots & t_{n-1\,n} \\
        0 & 0 & \dots & 0 & t_{n\,n}
    \end{pmatrix}.\] \\\\ On a \(g=p\rond\restr{f}{F}\) où \(p\) est le projecteur sur \(F\) parallèlement à \(\Vect{u_1}\). \\\\ Donc \(\quantifs{\tpt x\in F}f\paren{x}=\underbrace{g\paren{x}}_{\in F}+\alpha u_1\) où \(\alpha\in\K\). \\\\ De plus, \[\begin{aligned}
        \quantifs{\tpt j\in\interventierii{2}{n}}f\paren{u_j\prim}&=g\paren{u_j\prim}+\alpha_ju_1 \\
        &=\sum_{i=2}^jt_{i\,j}u_i\prim+\alpha_ju_1.
    \end{aligned}\] \\\\ D'où \(\P{n}\). \\
    \item Par récurrence, \(\quantifs{\tpt n\in\Ns}\P{n}\text{ est vraie}\).
\end{itemize}
\end{dem}

En particulier, quand \(\K=\C\), tous les endomorphismes sont trigonalisables, toutes les matrices de \(\M{n}[\C]\) sont trigonalisables dans \(\M{n}[\C]\).

En pratique, quand on cherche à trigonaliser un endomorphisme, on peut chercher une base dans laquelle la matrice est triangulaire supérieure avec des \(1\) ou des \(0\) sur la sur-diagonale et des \(0\) sur les diagonales partielles encore au-dessus (c'est démontrable, mais c'est difficile à démontrer, cela s'appelle le théorème de Jordan -- hors-programme --).

\begin{theo}
Soit \(f\in\Lendo{E}\).

Il y a équivalence entre les propositions suivantes :

\begin{itemize}
    \item \(f\)  est trigonalisable \\
    \item \(\chi_f\) est scindé \\
    \item \(\mu_f\) est scindé \\
    \item il existe un polynôme annulateur de \(f\) scindé
\end{itemize}
\end{theo}

Et sa version matricielle.

\begin{theo}
Soit \(A\in\M{n}\).

Il y a équivalence entre les propositions suivantes :

\begin{itemize}
    \item \(A\) est trigonalisable dans \(\M{n}\) \\
    \item \(\chi_A\) est scindé \\
    \item \(\mu_A\) est scindé \\
    \item il existe un polynôme annulateur de \(A\) qui est scindé dans \(\poly\)
\end{itemize}
\end{theo}

\begin{exo}~\\
Soit \(A=\begin{pmatrix}
0 & 1 & 1 \\
0 & 0 & 0 \\
0 & 1 & 0
\end{pmatrix}\). Calculez \(A^2\), puis \(A^3\). La matrice \(A\) est-elle diagonalisable ? trigonalisable ? Dans l'affirmative, diagonalisez ou trigonalisez la.
\end{exo}

\subsection{Théorème de Cayley-Hamilton}

\begin{theo}
Le polynôme caractéristique d'un endomorphisme (d'une matrice carrée) est un polynôme annulateur.
\end{theo}

\begin{dem}
On pose \(\P{n}\) : \guillemets{si \(A\in\M{n}[\C]\), alors \(\chi_A\paren{A}=0\)}.

\begin{itemize}
    \item Si \(n=1\) : on pose \(A=\paren{a}\). \\\\ On a \(\chi_A=X-a\) donc \(\chi_A\paren{A}=A-aI_1=\paren{a}-\paren{a}=0\). \\\\ D'où \(\P{1}\). \\
    \item Supposons \(\P{n-1}\). \\\\ Soit \(A\in\M{n}[\C]\). \\\\ Le polynôme \(\chi_A\) est scindé donc \(A\) est trigonalisable dans \(\M{n}[\C]\), \ie il existe \(P\in\GL{n}[\C]\) et \(T\in\Tsup{n}[\C]\) telles que \(A=PTP\inv\), avec \(T=\begin{pmatrix}
        \lambda_1 & ? & \dots & ? \\
        0 & \ddots & \ddots & \vdots \\
        \vdots & \ddots & \ddots & ? \\
        0 & \dots & 0 & \lambda_n
    \end{pmatrix}\). \\\\ On a \(\chi_A=\chi_T=\prod_{i=1}^n\paren{X-\lambda_i}\). \\\\ On peut écrire \[T=\begin{pmatrix}
        \lambda_1 & ? & \dots & & ? \\
        0 & \lambda_2 & ? & \dots & ? \\
        \vdots & 0 & \ddots & \ddots & \vdots \\
        & \vdots & \ddots & \ddots & ? \\
        0 & 0 & \dots & 0 & \lambda_n
    \end{pmatrix}=\begin{pmatrix}
        \lambda_1 & ? & \dots & ? \\
        0 & \Block{3-3}<\Large>{U} & & \\
        \vdots & & & \\
        0 & & &
    \end{pmatrix}\] où \(U\in\M{n-1}[\C]\). \\\\ On a \(\chi_U=\prod_{i=2}^n\paren{X-\lambda_i}\) donc \(\chi_A=\chi_T=\paren{X-\lambda_1}\chi_U\). \\\\ Donc \[\begin{WithArrows}
        \chi_A\paren{A}&=\paren{A-\lambda_1I_n}\chi_U\paren{A} \\
        &=\paren{PTP\inv-\lambda_1I_n}\chi_U\paren{PTP\inv} \Arrow{\(\paren{PTP\inv}^k=PT^kP\inv\)} \\
        &=P\paren{T-\lambda_1I_n}\chi_U\paren{T}P\inv.
    \end{WithArrows}\] \\\\ Or on a \[\underbrace{\begin{pmatrix}
        0 & ? & \dots & & ? \\
        0 & \lambda_2-\lambda_1 & ? & \dots & ? \\
        \vdots & 0 & \ddots & \ddots & \vdots \\
        & \vdots & \ddots & \ddots & ? \\
        0 & 0 & \dots & 0 & \lambda_n-\lambda_1
    \end{pmatrix}}_{T-\lambda_1I_n}\underbrace{\begin{pmatrix}
        \chi_U\paren{\lambda_1} & ? & \dots & ? \\
        0 & \Block{3-3}<\Large>{\chi_U\paren{U}} & & \\
        \vdots & & & \\
        0 & & &
    \end{pmatrix}}_{\chi_U\paren{T}}=\begin{pmatrix}
        0 & 0 & \dots & 0 \\
        0 & \Block{3-3}<\Large>{0} & & \\
        \vdots & & & \\
        0 & & &
    \end{pmatrix}\] car \(\chi_U\paren{U}=0\). \\\\ Donc \(\chi_A\paren{A}=0\), d'où \(\P{n}\). \\
    \item Par récurrence, \(\quantifs{\tpt n\in\Ns}\P{n}\text{ est vraie}\).
\end{itemize}
\end{dem}

\begin{cor}\thlabel{cor:polynomeMinimalDeDegreAuPlusN}
Le polynôme minimal divise le polynôme caractéristique. Donc en dimension \(n\), le polynôme minimal est de degré au plus \(n\).
\end{cor}

Les polynômes minimal et caractéristique partagent les mêmes racines dans \(\C\) (en fait dans tout corps \(\K\)) mais pas avec les mêmes ordres de multiplicité : si \(f\) est scindé, alors en notant \(\lambda_1,\dots,\lambda_k\) les \(k\) valeurs propres distinctes de \(f\), on peut écrire \[\chi_f=\prod_{i=1}^k\paren{X-\lambda_i}^{\alpha_i}\qquad\text{et}\qquad\mu_f=\prod_{i=1}^k\paren{X-\lambda_i}^{\beta_i}\] où \(\quantifs{\tpt i\in\interventierii{1}{k}}1\leq\beta_i\leq\alpha_i\).

\subsection{Sous-espaces caractéristiques}

\begin{defi}
Soit \(f\in\Lendo{E}\) un endomorphisme scindé. On écrit \(\chi_f=\prod_{i=1}^k\paren{X-\lambda_i}^{\alpha_i}\) où \(\lambda_1,\dots,\lambda_k\) sont les \(k\) valeurs propres distinctes de \(f\).

Les sous-espaces caractéristiques de \(f\) sont les noyaux \(\ker\paren{f-\lambda_i\id{E}}^{\alpha_i}\).
\end{defi}

\begin{prop}
Les sous-espaces caractéristiques d'un endomorphisme scindé sont supplémentaires et stables par \(f\).
\end{prop}

\begin{dem}
\begin{itemize}
    \item Soient \(\lambda\in\Sp{f}\) et \(\alpha\) l'ordre de multiplicité de \(\lambda\). \\\\ Soit \(x\in\ker\paren{f-\lambda\id{E}}^\alpha\). \\\\ On a \[\begin{WithArrows}
        \paren{f-\lambda\id{E}}^\alpha\paren{f\paren{x}}&=\paren{\paren{f-\lambda\id{E}}^\alpha\rond f}\paren{x} \Arrow[tikz={text width=4cm}]{composée de deux polynômes en \(f\) donc commutative} \\
        &=\paren{f\rond\paren{f-\lambda\id{E}}^\alpha}\paren{x} \\
        &=f\paren{0} \\
        &=0.
    \end{WithArrows}\] \\\\ Donc \(f\paren{x}\in\ker\paren{f-\lambda\id{E}}^\alpha\). \\\\ Donc \(\ker\paren{f-\lambda\id{E}}^\alpha\) est stable par \(f\). \\
    \item On a \(\chi_f=\prod_{i=1}^k\paren{X-\lambda_i}^{\alpha_i}\) : produit de polynômes premiers entre eux deux à deux. \\\\ D'après le lemme des noyaux, on a \[\ker\chi_f\paren{f}=\bigoplus_{i=1}^k\ker\paren{f-\lambda_i\id{E}}^{\alpha_i}.\] \\\\ Or \(\chi_f\paren{f}=0\) d'après le théorème de Cayley-Hamilton donc \[E=\bigoplus_{i=1}^k\ker\paren{f-\lambda_i\id{E}}^{\alpha_i}.\]
\end{itemize}
\end{dem}

\begin{theo}
Tout endomorphisme scindé possède une base dans laquelle sa matrice est diagonale par blocs telle que :

\begin{itemize}
    \item il y a autant de blocs que de valeurs propres : à chaque valeur propre, on associe un unique bloc ; \\
    \item chaque bloc est de la forme \(\lambda I_r+U\) où \(\lambda\) est la valeur propre associée au bloc, \(r\) est l'ordre de multiplicité de \(\lambda\) et \(U\) est une matrice strictement triangulaire supérieure de \(\M{r}\)
\end{itemize}

Toute matrice scindée est semblable à une matrice diagonale par blocs vérifiant les conditions précédentes.
\end{theo}

\begin{dem}
Soient \(\lambda\in\Sp{f}\) et \(\alpha\) son ordre de multiplicité.

Sur \(F=\ker\paren{f-\lambda\id{E}}^\alpha\), \(f\) induit un endomorphisme \(\tilde{f}\) tel que \(\paren{\tilde{f}-\lambda\id{F}}^\alpha=0\).

Donc \(\paren{X-\lambda}^\alpha\) est un polynôme annulateur de \(\tilde{f}\) qui est scindé donc \(\tilde{f}\) a pour unique valeur propre \(\lambda\) et est trigonalisable.

Donc il existe une base \(\fami{B}_\lambda\) de \(F\) telle que \[\Mat[\fami{B}_\lambda]{\tilde{f}}=\begin{pmatrix}
\lambda & ? & \dots & ? \\
0 & \ddots & \ddots & \vdots \\
\vdots & \ddots & \ddots & ? \\
0 & \dots & 0 & \lambda
\end{pmatrix}=\lambda I_\alpha+U.\]

Comme \(E=\bigoplus_{i=1}^k\ker\paren{f-\lambda_i\id{E}}^{\alpha_i}\), en concaténant de telles bases, on obtient une base de \(E\) dans laquelle la matrice de \(f\) est \[\begin{pmatrix}
\lambda_1I_{\alpha_1}+U_1 & 0 & \dots & 0 \\
0 & \ddots & \ddots & \vdots \\
\vdots & \ddots & \ddots & 0 \\
0 & \dots & 0 & \lambda_kI_{\alpha_k}+U_k
\end{pmatrix}.\]
\end{dem}

\begin{cor}
La dimension d'un sous-espace caractéristique est l'ordre de multiplicité de la valeur propre associée.
\end{cor}

\section{Endomorphismes nilpotents, matrices nilpotentes}

\subsection{Généralités}

\begin{defi}
Soit \(u\in\Lendo{E}\). On dit que \(u\) est nilpotent quand il existe \(p\in\N\) tel que \(u^p=0\).

Soit \(A\in\M{n}\). On dit que \(A\) est nilpotente quand il existe \(p\in\N\) tel que \(A^p=0\).

Le plus petit indice \(p\) satisfaisant à la condition précédente s'appelle l'indice de nilpotence de \(u\) (de \(A\)).
\end{defi}

\begin{prop}
Toute matrice strictement triangulaire (supérieure ou inférieure) est nilpotente. Par conséquent, les matrices semblables à une matrice strictement triangulaire sont nilpotentes.
\end{prop}

\begin{dem}
Soit \(A\in\M{n}\) une matrice strictement triangulaire : \[A=\begin{pmatrix}
0 & ? & \dots & ? \\
0 & \ddots & \ddots & \vdots \\
\vdots & \ddots & \ddots & ? \\
0 & \dots & 0 & 0
\end{pmatrix}.\]

On a \(\chi_A=X^n\) et \(\chi_A\paren{A}=0\) donc \(A^n=0\).
\end{dem}

Dans la décomposition en sous-espaces caractéristiques, on a vu apparaître des matrices \(\lambda I_r+U\) : les matrices \(U\) sont nilpotentes.

L'ensemble des matrices nilpotentes n'a pas de structure particulière : en général, la somme et le produit de deux matrices nilpotentes ne sont pas nilpotents. Néanmoins, avec une condition de commutation supplémentaire, on a quelques résultats.

\begin{prop}
Soient \(A,B\in\M{n}\) deux matrices nilpotentes.

Si \(A\) et \(B\) commutent, alors \(A+B\) et \(AB\) sont nilpotentes.
\end{prop}

\begin{dem}
Soit \(\paren{k,l}\in\N^2\) tel que \(A^k=0\) et \(B^l=0\).

Supposons \(AB=BA\).

On a \[\begin{aligned}
\paren{AB}^{\min\paren{k,l}}&=A^{\min\paren{k,l}}B^{\min\paren{k,l}} \\
&=0
\end{aligned}\] et \[\begin{aligned}
\paren{A+B}^{k+l}&=\sum_{i=0}^{k+l}\binom{i}{k+l}A^iB^{k+l-i} \\
&=\sum_{i=0}^k\binom{i}{k+l}A^i\underbrace{B^{k+l-i}}_{=0}+\sum_{i=k+1}^{k+l}\binom{i}{k+l}\underbrace{A^i}_{=0}B^{k+l-i} \\
&=0.
\end{aligned}\]
\end{dem}

On a bien sûr les mêmes résultats concernant les endomorphismes nilpotents.

\subsection{Éléments propres d'un nilpotent}

\begin{prop}
Un endomorphisme en dimension \(n\) est nilpotent ssi son polynôme caractéristique est \(X^n\), \ie s'il est scindé et admet \(0\) comme unique valeur propre.

Une matrice de \(\M{n}\) est nilpotente ssi son polynôme caractéristique est \(X^n\), \ie si elle est scindée et admet \(0\) comme unique valeur propre.

L'indice de nilpotence dans ces deux cas est alors le degré du polynôme minimal ; il est donc inférieur ou égal à \(n\).
\end{prop}

\begin{dem}
Si \(f\) est nilpotent alors il existe \(k\in\N\) tel que \(f^k=0\) donc \(X^k\) est annulateur de \(f\) donc \(\Sp{f}=\accol{0}\) donc \(\chi_f=X^n\).

Réciproquement, si \(\chi_f=X^n\), d'après le théorème de Cayley-Hamilton, \(f^n=0\) donc \(f\) est nilpotent.

Or \(\mu_f\divise\chi_f\) donc \(\mu_f\) est de la forme \(X^l\) avec \(l\leq n\) et par définition de \(\mu_f\), \(l\) est l'indice de nilpotence de \(f\).
\end{dem}

Mis à part la matrice nulle, aucune matrice nilpotente n'est diagonalisable : c'est une idée parfois utile pour prouver qu'une matrice est nulle (diagonalisable et nilpotente implique nulle).

\begin{prop}
Tout endomorphisme nilpotent est trigonalisable : il existe une base dans laquelle sa matrice est triangulaire supérieure stricte. Réciproquement, si un endomorphisme est trigonalisable et n'a que \(0\) pour valeur propre, alors il est nilpotent.

Toute matrice nilpotente est trigonalisable : elle est semblable à une matrice triangulaire supérieure stricte. La réciproque est vraie.
\end{prop}

\subsection{Application aux sous-espaces caractéristiques d'un endomorphisme}

\begin{prop}\thlabel{prop:ordreDeMultiplicitéDansChiEstLeMêmeQueLaPuissanceDansLeSousEspaceCaractéristique}
Soit \(f\in\Lendo{E}\).

Pour toute valeur propre \(\lambda\) de \(f\), si \(\alpha\) est l'ordre de multiplicité de \(\lambda\) dans le polynôme minimal de \(f\), le sous-espace caractéristique associé est aussi le noyau \(\ker\paren{f-\lambda\id{E}}^\alpha\).
\end{prop}

\begin{lem}\thlabel{lem:sommesDirectesEgalesEtUneInclusionDonneEgalite}
Si \(F_1,\dots,F_k,G_1,\dots,G_k\) vérifient \(\bigoplus_{i=1}^kF_i=\bigoplus_{i=1}^kG_i\) et \(\quantifs{\tpt i\in\interventierii{1}{k}}F_i\subset G_i\), alors \(\quantifs{\tpt i\in\interventierii{1}{k}}F_i=G_i\).
\end{lem}

\begin{dem}
Soient \(i\in\interventierii{1}{k}\) et \(x\in G_i\).

On a \(x\in\bigoplus_{j=1}^kG_j=\bigoplus_{j=1}^kF_j\).

Donc il existe \(\paren{y_1,\dots,y_k}\in F_1\times\dots\times F_k\) tel que \[\underbrace{x}_{\in G_i}=\underbrace{y_1}_{\in F_1\subset G_1}+\dots+\underbrace{y_k}_{\in F_k\subset G_k}.\]

Or la somme \(\bigoplus_{j=1}^kG_j\) est directe donc par unicité \[\begin{dcases}
y_1=0 \\
\vdots \\
y_{i-1}=0 \\
y_i=x \\
y_{i+1}=0 \\
\vdots \\
y_k=0
\end{dcases}\]

Donc \(x=y_i\in F_i\).

Donc \(F_i\subset G_i\).

Donc \(F_i=G_i\).
\end{dem}

\begin{dem}[de la \thref{prop:ordreDeMultiplicitéDansChiEstLeMêmeQueLaPuissanceDansLeSousEspaceCaractéristique}]~\\
On note \(\chi_f=\prod_{i=1}^k\paren{X-\lambda_i}^{\alpha_i}\) et \(\mu_f=\prod_{i=1}^k\paren{X-\lambda_i}^{\beta_i}\) où \(\quantifs{\tpt i\in\interventierii{1}{k}}\alpha_i\geq\beta_i\geq1\).

On veut montrer que \(\quantifs{\tpt i\in\interventierii{1}{k}}\ker\paren{f-\lambda_i\id{E}}^{\alpha_i}=\ker\paren{f-\lambda_i\id{E}}^{\beta_i}\).

Comme \(\beta_i\leq\alpha_i\), on a immédiatement \(\quantifs{\tpt i\in\interventierii{1}{k}}\ker\paren{f-\lambda_i\id{E}}^{\beta_i}\subset\ker\paren{f-\lambda_i\id{E}}^{\alpha_i}\).

Comme \(\mu_f\paren{f}=\chi_f\paren{f}=0\), d'après le lemme des noyaux : \[E=\bigoplus_{i=1}^k\ker\paren{f-\lambda_i\id{E}}^{\beta_i}=\bigoplus_{i=1}^k\ker\paren{f-\lambda_i\id{E}}^{\alpha_i}.\]

D'après le \thref{lem:sommesDirectesEgalesEtUneInclusionDonneEgalite}, on en déduit \[\quantifs{\forall i\in\interventierii{1}{k}}\ker\paren{f-\lambda_i\id{E}}^{\alpha_i}=\ker\paren{f-\lambda_i\id{E}}^{\beta_i}.\]
\end{dem}

On peut même démontrer mieux.

\begin{prop}
Soient \(f\in\Lendo{E}\), \(\lambda\in\Sp{f}\) et \(\alpha\) l'ordre de multiplicité de \(\lambda\) dans le polynôme minimal de \(f\).

Alors la suite des noyaux \(\paren{\ker\paren{f-\lambda\id{E}}^k}_{k\in\N}\) est strictement croissante jusqu'au rang \(\alpha\), puis constante à partir du rang \(\alpha\) : \[\accol{0}\subsetneq\ker\paren{f-\lambda\id{E}}\subsetneq\ker\paren{f-\lambda\id{E}}^2\subsetneq\dots\subsetneq\ker\paren{f-\lambda\id{E}}^\alpha=\ker\paren{f-\lambda\id{E}}^{\alpha+1}=\dots\]
\end{prop}


\chapter{Intégrales généralisées}

\minitoc

Dans ce chapitre, \(\K\) désigne l'ensemble \(\R\) ou \(\C\). Les fonctions considérées dans ce chapitre sont à valeurs dans \(\K\).

On suppose connue la notion d'intégrale sur un segment d'une fonction continue par morceaux à valeurs réelles ou complexes (\cf cours de première année).

Si \(f\) est une fonction continue sur un segment \(\intervii{a}{b}\) (ou \(\intervii{b}{a}\)), on note \(\int_a^bf=\int_a^bf\paren{t}\odif{t}\) (ou toute autre lettre à la place de \(t\)) l'intégrale de \(f\) entre \(a\) et \(b\) : quand on a pas besoin de nommer la variable d'intégration, on ne la note pas, mais si on la note alors on n'oublie pas l'élément différentiel \(\odif{}\).

En préambule, on généralise la notion de fonction continue par morceaux sur un intervalle quelconque.

\section*{Fonctions continues par morceaux sur un intervalle}
\addcontentsline{toc}{section}{Fonctions continues par morceaux sur un intervalle}

\begin{defi}
Soit \(I\) un intervalle quelconque.

On dit qu'une fonction est continue par morceaux sur \(I\) quand elle est continue par morceaux sur tout segment inclus dans \(I\).
\end{defi}

\begin{ex}
\begin{itemize}
    \item La fonction \(t\mapsto\begin{dcases}
        -\ln t &\text{si }t\in\intervei{0}{1} \\
        \e{-t} &\text{si }t\in\intervie{1}{\pinf}
    \end{dcases}\) est continue par morceaux sur \(\intervei{0}{\pinf}\). \\
    \item La fonction \(t\mapsto\floor{\dfrac{1}{t}}\) est continue par morceaux sur \(\intervei{0}{\pinf}\).
\end{itemize}
\end{ex}

Dans toute la suite, on note \(\contm[I]\) l'ensemble des fonctions continues par morceaux sur \(I\) et à valeurs dans \(\K\).

\begin{prop}
L'ensemble \(\contm[I]\) est une \(\K\)-algèbre.
\end{prop}

\section{Intégrales généralisées sur \(\intervie{a}{\pinf}\)}

Dans cette section, \(a\) est un réel.

\subsection{Définition et exemples fondamentaux}

\begin{defi}
Soit \(f\in\contm[\intervie{a}{\pinf}]\).

On dit que l'intégrale \(\int_a^{\pinf}f\) converge (ou qu'elle est convergente, ou simplement qu'elle existe) quand \(\int_a^xf\) a une limite finie quand \(x\) tend vers \(\pinf\).

Dans ce cas, on pose \(\int_a^{\pinf}=\lim_{x\to\pinf}\int_a^xf\).

Dans le cas contraire, on dit que l'intégrale \(\int_a^{\pinf}\) diverge (ou qu'elle est divergente).
\end{defi}

\begin{rem}
Une intégrale généralisée \(\int_a^{\pinf}\) est une limite et une limite n'existe pas toujours. Avant d'utiliser une telle intégrale dans un raisonnement ou un calcul, on doit donc toujours justifier son existence !
\end{rem}

Les résultats ci-dessous sont à connaître.

\begin{ex}
\begin{itemize}
    \item Soit \(\alpha\) un réel. L'intégrale \(\int_0^{\pinf}\e{-\alpha t}\odif{t}\) converge ssi \(\alpha>0\). \\
    \item Soit \(\alpha\) un réel. L'intégrale \(\int_1^{\pinf}\dfrac{1}{t^\alpha}\odif{t}\) converge ssi \(\alpha>1\) (intégrale dite de Riemann).
\end{itemize}
\end{ex}

\begin{dem}
\begin{itemize}
    \item La fonction \(t\mapsto\e{-\alpha t}\) est continue sur \(\intervie{0}{\pinf}\). \\\\ Pour \(x\geq0\), si \(\alpha\not=0\), on a \[\int_0^x\e{-\alpha t}\odif{t}=\croch{\dfrac{1}{-\alpha}\e{-\alpha t}}_0^x=\dfrac{1}{\alpha}\paren{1-\e{-\alpha x}}.\] Donc si \(\alpha>0\), on a \[\int_0^x\e{-\alpha t}\odif{t}\tendqd{x\to\pinf}\dfrac{1}{\alpha}\] et si \(\alpha<0\), on a \[\int_0^x\e{-\alpha t}\odif{t}\tendqd{x\to\pinf}\pinf.\] De plus, si \(\alpha=0\), on a \[\int_0^x\e{-\alpha t}\odif{t}=x\tendqd{x\to\pinf}\pinf.\] Donc \(\int_0^{\pinf}\e{-\alpha t}\odif{t}\) converge ssi \(\alpha>0\) et, dans ce cas, \(\int_0^{\pinf}\e{-\alpha t}\odif{t}=\dfrac{1}{\alpha}\). \\
    \item La fonction \(t\mapsto\dfrac{1}{t^\alpha}\) est continue sur \(\intervie{1}{\pinf}\). \\\\ Pour \(x\geq1\), si \(\alpha\not=1\), on a \[\int_1^x\dfrac{1}{t^\alpha}\odif{t}=\int_1^xt^{-\alpha}\odif{t}=\croch{\dfrac{1}{-\alpha+1}t^{-\alpha+1}}_1^x=\dfrac{1}{\alpha-1}\paren{1-x^{-\alpha+1}}.\] Donc si \(\alpha>1\), on a \[\int_1^x\dfrac{1}{t^\alpha}\odif{t}\tendqd{x\to\pinf}\dfrac{1}{\alpha-1}\] et si \(\alpha<1\), on a \[\int_1^x\dfrac{1}{t^\alpha}\odif{t}\tendqd{x\to\pinf}\pinf.\] De plus, si \(\alpha=1\) alors \[\int_1^x\dfrac{1}{t^\alpha}\odif{t}=\ln x\tendqd{x\to\pinf}\pinf.\] Donc \(\int_1^{\pinf}\dfrac{1}{t^\alpha}\odif{t}\) converge ssi \(\alpha>1\) et, dans ce cas, \(\int_1^{\pinf}\dfrac{1}{t^\alpha}\odif{t}=\dfrac{1}{\alpha-1}\).
\end{itemize}
\end{dem}

\begin{exo}
Quelle est la nature de l'intégrale \(\int_0^{\pinf}\dfrac{1}{1+t^2}\odif{t}\) ?
\end{exo}

\begin{corr}
La fonction \(t\mapsto\dfrac{1}{1+t^2}\) est continue sur \(\intervie{0}{\pinf}\).

Pour \(x\geq0\), on a \[\int_0^x\dfrac{1}{1+t^2}\odif{t}=\croch{\Arctan t}_0^x=\Arctan x\tendqd{x\to\pinf}\dfrac{\pi}{2}\].

Donc \(\int_0^{\pinf}\dfrac{1}{1+t^2}\odif{t}\) converge et vaut \(\dfrac{\pi}{2}\).
\end{corr}

\begin{exo}
Quelle est la nature de l'intégrale \(\int_1^{\pinf}\dfrac{\ln t}{t}\odif{t}\).
\end{exo}

\begin{corr}
La fonction \(t\mapsto\dfrac{\ln t}{t}\) est continue sur \(\intervie{1}{\pinf}\).

Pour \(x\geq1\), on a \[\int_1^x\dfrac{\ln t}{t}\odif{t}=\croch{\dfrac{\ln^2t}{2}}_1^x=\dfrac{\ln^2x}{2}\tendqd{x\to\pinf}\pinf.\]

Donc \(\int_1^{\pinf}\dfrac{\ln t}{t}\odif{t}\) diverge.
\end{corr}

\subsection{Propriétés}

La convergence de l'intégrale ne dépend pas de la borne \(a\), ce qui généralise la relation de Chasles.

\begin{prop}
Soient \(f\in\contm[\intervie{a}{\pinf}]\) et \(b\in\intervie{a}{\pinf}\).

Alors l'intégrale \(\int_a^{\pinf}f\) converge ssi \(\int_b^{\pinf}f\) converge.

Dans ce cas, on a \(\int_a^{\pinf}f=\int_a^bf+\int_b^{\pinf}f\).
\end{prop}

\begin{dem}
Soit \(x\geq a\).

Par relation de Chasles sur les intégrales classiques, on a \[\int_a^xf=\underbrace{\int_a^bf}_{\in\K}+\int_b^xf.\]

Donc \(\int_a^xf\) a une limite finie quand \(x\to\pinf\) ssi \(\int_b^xf\) a une limite finie quand \(x\to\pinf\) par opération sur les limites.

Dans ce cas, on a \[\begin{aligned}
\int_a^{\pinf}f&=\lim_{x\to\pinf}\int_a^xf \\
&=\lim_{x\to\pinf}\paren{\int_a^bf+\int_b^xf} \\
&=\int_a^bf+\lim_{x\to\pinf}\int_b^xf \\
&=\int_a^bf+\int_b^{\pinf}f.
\end{aligned}\]
\end{dem}

Dans le cas convergent, on retrouve la linéarité.

\begin{prop}
Soient \(\paren{f,g}\in\contm[\intervie{a}{\pinf}]^2\) et \(\paren{\lambda,\mu}\in\K^2\).

Si les intégrales \(\int_a^{\pinf}f\) et \(\int_a^{\pinf}g\) convergent, alors \(\int_a^{\pinf}\paren{\lambda f+\mu g}\) converge.

Dans ce cas, on a \(\int_a^{\pinf}\paren{\lambda f+\mu g}=\lambda\int_a^{\pinf}f+\mu\int_a^{\pinf}g\).
\end{prop}

\begin{dem}
Pour \(x\geq0\), on a \[\int_a^x\paren{\lambda f+\mu g}=\lambda\int_a^xf+\mu\int_a^xg\] par linéarité de l'intégrale classique.

Puis, d'après les théorèmes d'opérations sur les limites, si \(\int_a^{\pinf}f\) et \(\int_a^{\pinf}g\) convergent, les deux fonctions \(x\mapsto\int_a^xf\) et \(x\mapsto\int_a^xg\) ont des limites finies en \(\pinf\) donc \(x\mapsto\lambda\int_a^xf+\mu\int_a^xg\) en a une aussi, \ie \(\int_a^{\pinf}\paren{\lambda f+\mu g}\) converge.

Dans ce cas, \(\int_a^{\pinf}\paren{\lambda f+\mu g}=\lambda\int_a^{\pinf}f+\mu\int_a^{\pinf}g\).
\end{dem}

\begin{rem}
\begin{itemize}
    \item \guillemets{La somme d'une intégrale convergente et d'une divergente est divergente}. \\
    \item Il n'y a rien à dire \textit{a priori} sur la \guillemets{somme de deux intégrales divergentes}.
\end{itemize}
\end{rem}

\begin{dem}~\\
Montrons que si \(\int_a^{\pinf}f\) converge et \(\int_a^{\pinf}g\) diverge, alors \(\int_a^{\pinf}\paren{f+g}\) diverge.

Par l'absurde, si \(\int_a^{\pinf}\paren{f+g}\) converge alors comme \(g=\paren{f+g}-f\), d'après la proposition précédente, \(\int_a^{\pinf}g\) converge : contradiction.
\end{dem}

\subsection{Cas des fonctions réelles positives}

Quand une fonction \(f\) est positive et continue par morceaux, dans le cas où l'intégrale \(\int_a^{\pinf}f\) diverge, on pose par convention \(\int_a^{\pinf}f=\pinf\), ce qui permet de donner un sens à toutes les intégrales de fonctions positives.

\begin{prop}
Soit \(f\in\contm[\intervie{a}{\pinf}][\R]\).

Si \(f\geq0\), alors \(\int_a^{\pinf}f\geq0\).

Si, de plus, \(f\) est continue et prend au moins une valeur strictement positive, alors \(\int_a^{\pinf}f>0\).

Ceci est vrai en particulier quand \(f\) est continue et strictement positive sur \(\intervie{a}{\pinf}\).
\end{prop}

\begin{dem}
\begin{itemize}
    \item Si \(f\geq0\) sur \(\intervie{a}{\pinf}\), alors \(\quantifs{\tpt x\in\intervie{a}{\pinf}}f\geq0\) sur \(\intervii{a}{x}\). \\\\ Donc par positivité de l'intégrale classique, \(\int_a^xf\geq0\). \\\\ Donc si \(\int_a^{\pinf}f\) diverge, alors \[\int_a^{\pinf}f=\pinf\geq0\] et si \(\int_a^{\pinf}f\) converge, on a \[\int_a^{\pinf}f\geq0\] par passage à la limite quand \(x\to\pinf\). \\
    \item Si \(f\) est continue, positive et prend au moins une valeur strictement positive en \(x_0\), pour \(x\geq x_0+1\), on a \[\int_a^xf=\int_a^{x_0+1}f+\int_{x_0+1}^xf.\] D'après le théorème de stricte positivité de l'intégrale classique, comme \(f\) est continue et positive sur \(\intervii{a}{x_0+1}\), \(f\paren{x_0}>0\) et \(x_0\in\intervii{a}{x_0+1}\), on a \(\int_a^{x_0+1}f>0\). \\\\ Or \(\int_{x_0+1}^xf\geq0\) donc \(\int_a^xf\geq\int_a^{x_0+1}f\). \\\\ Par passage à la limite quand \(x\to\pinf\), on obtient \[\int_a^{\pinf}f\geq\int_a^{x_0+1}f>0.\]
\end{itemize}
\end{dem}

\begin{rem}[Contraposée]
Si \(f\) est continue et positive sur \(\intervie{a}{\pinf}\) et \(\int_a^{\pinf}f=0\), alors \(f=0\) sur \(\intervie{a}{\pinf}\).
\end{rem}

On en déduit la propriété de croissance des intégrales.

\begin{prop}
Soit \(\paren{f,g}\in\contm[\intervie{a}{\pinf}][\R]^2\) tel que \(f\) et \(g\) soient positives.

Si \(f\leq g\), alors \(\int_a^{\pinf}f\leq\int_a^{\pinf}g\).
\end{prop}

\subsection{Théorème de comparaison entre fonctions positives}

D'abord une condition nécessaire et suffisante de convergence dans le cas d'une fonction positive.

\begin{prop}\thlabel{prop6.6}
Soit \(f\in\contm[\intervie{a}{\pinf}][\R]\).

Si \(f\geq0\), alors l'intégrale \(\int_a^{\pinf}f\) converge ssi la fonction \(x\mapsto\int_a^xf\) est majorée.
\end{prop}

\begin{dem}
Si \(f\geq0\) sur \(\intervie{a}{\pinf}\), alors \(x\mapsto\int_a^xf\) est croissante sur \(\intervie{a}{\pinf}\).

Donc pour \(\paren{x,y}\in\intervie{a}{\pinf}^2\) tel que \(x\geq y\), on a \[\int_a^xf=\int_a^yf+\int_y^xf\geq\int_a^yf.\]

Donc \(\int_a^{\pinf}f\) converge ssi la fonction croissante \(x\mapsto\int_a^xf\) a une limite finie en \(\pinf\) ssi la fonction croissante \(x\mapsto\int_a^xf\) est majorée, par théorème de la limite monotone.
\end{dem}

On en déduit un théorème de comparaison du même type que celui sur les séries.

\begin{theo}
Soit \(\paren{f,g}\in\contm[\intervie{a}{\pinf}][\R]^2\) tel que \(f\) et \(g\) soient positives.

\begin{itemize}
    \item Si \(f\leq g\) et \(\int_a^{\pinf}g\) converge, alors \(\int_a^{\pinf}f\) converge. \\
    \item Si \(f\leq g\) et \(\int_a^{\pinf}f\) diverge, alors \(\int_a^{\pinf}g\) diverge. \\
    \item Si \(f\simqd{\pinf}g\) alors les intégrales \(\int_a^{\pinf}f\) et \(\int_a^{\pinf}g\) sont de même nature : l'une converge ssi l'autre converge.
\end{itemize}
\end{theo}

\begin{dem}
\begin{itemize}
    \item Si \(0\leq f\leq g\) sur \(\intervie{a}{\pinf}\), alors \(\quantifs{\tpt x\in\intervie{a}{\pinf}}\int_a^xf\leq\int_a^xg\). \\\\ Donc si \(\int_a^{\pinf}g\) converge, alors \(x\mapsto\int_a^xg\) est majorée d'après la \thref{prop6.6}. \\\\ Donc \(x\mapsto\int_a^xf\) est majorée. \\\\ Donc \(\int_a^{\pinf}f\) converge d'après la \thref{prop6.6}. \\
    \item Par contraposée, si \(\int_a^{\pinf}f\) diverge, alors \(\int_a^{\pinf}g\) diverge. \\
    \item Si \(f\simqd{\pinf}g\), il existe une fonction \(\phi\) définie au voisinage de \(\pinf\) telle que \(f=g\phi\) et \(\phi\tendqd{\pinf}1\). \\\\ Comme \(\phi\tendqd{\pinf}1\), au voisinage de \(\pinf\), on a \(\dfrac{1}{2}\leq\phi\leq\dfrac{3}{2}\). \\\\ Donc, au voisinage de \(\pinf\), on a \(0\leq\dfrac{1}{2}g\leq\phi g=f\leq\dfrac{3}{2}g\). \\\\ D'après ce qui précède, si \(\int_a^{\pinf}f\) converge, alors \(\int_a^{\pinf}\dfrac{1}{2}g\) converge donc \(\int_a^{\pinf}g=2\int_a^{\pinf}\dfrac{1}{2}g\) converge et si \(\int_a^{\pinf}f\) diverge, alors \(\int_a^{\pinf}\dfrac{3}{2}g\) diverge donc \(\int_a^{\pinf}g=\dfrac{2}{3}\int_a^{\pinf}\dfrac{3}{2}g\) diverge.
\end{itemize}
\end{dem}

\begin{rem}
\begin{itemize}
    \item Dans ce théorème, il suffit que les inégalités soient vraies au voisinage de \(\pinf\) seulement. \\
    \item Si les fonctions sont à valeurs négatives, on se ramène à ce théorème en travaillant avec les fonctions opposées. Ce qui compte est donc qu'elles soient de signe constant. \\
    \item Avec des fonctions dont le signe n'est pas constant, ce théorème est faux. Il faut donc bien s'assurer et mettre en valeur que les fonctions sont positives (ou négatives). \\
    \item On compare les fonctions, pas les intégrales ! N'écrivez pas des symboles \(\int_a^{\pinf}\cdots\) partout.
\end{itemize}
\end{rem}

\begin{exo}~\\
Montrez que \(\int_1^{\pinf}\dfrac{\e{-t}}{t}\odif{t}\) converge.
\end{exo}

\begin{corr}
Les fonctions \(t\mapsto\dfrac{\e{-t}}{t}\) et \(t\mapsto\e{-t}\) sont continues sur \(\intervie{1}{\pinf}\).

\(\quantifs{\Tpt t\geq1}0\leq\dfrac{\e{-t}}{t}\leq\e{-t}\) or \(\int_1^{\pinf}\e{-t}\odif{t}\) converge donc par comparaison de fonctions positives, \(\int_1^{\pinf}\dfrac{\e{-t}}{t}\odif{t}\) converge.
\end{corr}

\begin{exo}~\\
Montrez que \(\int_2^{\pinf}\dfrac{1}{\ln t}\odif{t}\) diverge.
\end{exo}

\begin{corr}
La fonction \(t\mapsto\dfrac{1}{\ln t}\) est continue sur \(\intervie{2}{\pinf}\).

Pour \(t\geq2\), on a \(\dfrac{1}{\ln t}\geq\dfrac{1}{t}\geq0\) car \(\quantifs{\forall t>0}\ln t\leq t-1\leq t\) car \(\ln\) est concave.

Or \(\int_2^{\pinf}\dfrac{1}{t}\odif{t}\) diverge donc par comparaison de fonctions positives, \(\int_2^{\pinf}\dfrac{1}{\ln t}\odif{t}\) diverge.
\end{corr}

\begin{exo}
Montrez que \(\quantifs{\tpt\alpha>1}\int_1^{\pinf}\dfrac{\ln t}{t^\alpha}\odif{t}\) converge.
\end{exo}

\begin{corr}
\(\quantifs{\Tpt\alpha>1}t\mapsto\dfrac{\ln t}{t^\alpha}\) est continue sur \(\intervie{1}{\pinf}\).

\begin{itemize}
    \item Si \(\alpha>2\), comme on a \(0\leq\dfrac{\ln t}{t^\alpha}\leq\dfrac{t}{t^\alpha}=\dfrac{1}{t^{\alpha-1}}\) pour \(t\in\intervie{1}{\pinf}\) et \(\int_1^{\pinf}\dfrac{1}{t^{\alpha-1}}\odif{t}\) converge (car \(\alpha-1>1\)), on en déduit que \(\int_1^{\pinf}\dfrac{\ln t}{t^\alpha}\odif{t}\) converge. \\
    \item Il existe \(a>1\) tel que \(\quantifs{\tpt t\in\intervie{a}{\pinf}}0\leq\ln t\leq t^{\nicefrac{\alpha-1}{2}}\) car \(\ln t\egqd{t\to\pinf}\o{t^{\nicefrac{\alpha-1}{2}}}\) car \(\dfrac{\alpha-1}{2}>0\). \\\\ Donc \(\quantifs{\tpt t\in\intervie{a}{\pinf}}0\leq\dfrac{\ln t}{t^\alpha}\leq\dfrac{t^{\nicefrac{\alpha-1}{2}}}{t^\alpha}=\dfrac{1}{t^{\nicefrac{\alpha+1}{2}}}\). \\\\ Comme \(\alpha>1\), \(\dfrac{\alpha+1}{2}>1\) donc \(\int_a^{\pinf}\dfrac{1}{t^{\nicefrac{\alpha+1}{2}}}\odif{t}\) converge. \\\\ Donc \(\int_a^{\pinf}\dfrac{\ln t}{t^\alpha}\odif{t}\) converge \ie \(\int_1^{\pinf}\dfrac{\ln t}{t^\alpha}\odif{t}\) converge.
\end{itemize}
\end{corr}

\subsection{Lien avec les séries}

Le théorème de comparaison série-intégrale peut se réécrire comme suit.

\begin{prop}
Soit \(f\in\contm[\Rp][\R]\) positive et décroissante.

La série \(\sum f\paren{n}\) et l'intégrale \(\int_0^{\pinf}f\) sont de même nature.
\end{prop}

Dans le même genre, on peut étudier la convergence d'une intégrale d'une fonction positive par l'intermédiaire d'une série.

\begin{prop}\thlabel{prop6.8}
Soient \(f\in\contm[\Rp][\R]\) positive et \(u\) une suite positive et strictement croissante qui diverge vers \(\pinf\).

L'intégrale \(\int_0^{\pinf}f\) converge ssi la série \(\sum_{n\geq0}\int_{u_n}^{u_{n+1}}f\) converge.
\end{prop}

\begin{dem}
\impdir

Si \(f\geq0\) sur \(\intervie{0}{\pinf}\), alors \(F:x\mapsto\int_0^xf\) est croissante.

Donc si \(\int_0^{\pinf}f\) converge, alors \(\int_0^xf\) a une limite réelle \(l\) quand \(x\to\pinf\).

Donc comme \(u_n\tendqd{n\to\pinf}\pinf\), on a \(\int_0^{u_n}f\tendqd{n\to\pinf}l\).

Or \(\int_0^{u_n}f=\int_0^{u_0}f+\int_{u_0}^{u_1}f+\dots+\int_{u_{n-1}}^{u_n}f\) donc \[\sum_{k=0}^{n-1}\int_{u_k}^{u_{k+1}}f\tendqd{n\to\pinf}l-\int_0^{u_0}f\] \ie la série \(\sum_n\int_{u_n}^{u_{n+1}}f\) converge.

\imprec

Si la série \(\sum_n\int_{u_n}^{u_{n+1}}f\) converge, \(\sum_{k=0}^{n-1}\int_{u_k}^{u_{k+1}}f=F\paren{u_n}-F\paren{u_0}\) a une limite réelle quand \(n\to\pinf\) \ie \(\paren{F\paren{u_n}}\) a une limite réelle.

Or \(F\) est croissante donc d'après le théorème de la limite monotone, \(F\) a une limite \(L\in\R\union\accol{\pinf}\) en \(\pinf\).

D'après la caractérisation séquentielle de la limite, on a \(F\paren{u_n}\tendqd{n\to\pinf}L\) car \(u_n\tendqd{n\to\pinf}\pinf\).

Donc \(L\in\R\). Donc \(\int_0^{\pinf}f\) converge.
\end{dem}

\begin{rem}
L'hypothèse de positivité de \(f\) est essentielle.

Contre-exemple si on ne suppose rien sur le signe de \(f\) :

\(\quantifs{\Tpt n\in\N}\int_{2n\pi}^{2\paren{n+1}\pi}\sin=0\) donc \(\sum\int_{2n\pi}^{2\paren{n+1}\pi}\sin\) converge mais \(\int_0^x\sin=1-\cos x\) n'a pas de limite quand \(x\to\pinf\) \ie \(\int_0^{\pinf}\sin\) diverge.
\end{rem}

\begin{exo}
Montrez que l'intégrale \(\int_1^{\pinf}\dfrac{\abs{\sin t}}{t}\odif{t}\) diverge.
\end{exo}

\begin{corr}
La fonction \(t\mapsto\dfrac{\abs{\sin t}}{t}\) est continue sur \(\intervie{1}{\pinf}\).

D'après la \thref{prop6.8}, \(\int_1^{\pinf}\dfrac{\abs{\sin t}}{t}\odif{t}\) converge ssi la série \(\sum_n\int_{u_n}^{u_{n+1}}\dfrac{\abs{\sin t}}{t}\odif{t}\) converge, où \(\paren{u_n}\) est une suite positive et strictement croissante.

On prend \(u_n=n\pi\) pour \(n\geq1\).

Pour \(n\in\Ns\), on a \(\int_{n\pi}^{\paren{n+1}\pi}\dfrac{\abs{\sin t}}{t}\odif{t}\geq\int_{n\pi}^{\paren{n+1}\pi}\dfrac{\abs{\sin t}}{\paren{n+1}\pi}\odif{t}\) car \[\begin{aligned}
\quantifs{\forall t\in\intervii{n\pi}{\paren{n+1}\pi}}n\pi\leq t\leq\paren{n+1}\pi&\text{ donc }\dfrac{1}{\paren{n+1}\pi}\leq\dfrac{1}{t}\leq\dfrac{1}{n\pi} \\
&\text{ donc }\dfrac{\abs{\sin t}}{\paren{n+1}\pi}\leq\dfrac{\abs{\sin t}}{t}\leq\dfrac{\abs{\sin t}}{n\pi}.
\end{aligned}\]

Donc on a \[\begin{WithArrows}
\int_{n\pi}^{\paren{n+1}\pi}\dfrac{\abs{\sin t}}{t}\odif{t}&\geq\dfrac{1}{\paren{n+1}\pi}\int_{n\pi}^{\paren{n+1}\pi}\abs{\sin t}\odif{t} \Arrow{\(t\mapsto\abs{\sin t}\) est \(\pi\)-périodique} \\
&=\dfrac{1}{\paren{n+1}\pi}\int_0^\pi\abs{\sin t}\odif{t} \\
&=\dfrac{2}{\paren{n+1}\pi}.
\end{WithArrows}\]

Or \(\sum_n\dfrac{2}{\paren{n+1}\pi}\) diverge donc \(\int_1^{\pinf}\dfrac{\abs{\sin t}}{t}\odif{t}\) diverge.
\end{corr}

\begin{exo}
En utilisant l'inégalité \(\sin t\geq\dfrac{2}{\pi}t\), valable pour tout \(t\in\intervii{0}{\dfrac{\pi}{2}}\), montrez que l'intégrale \(\int_0^{\pinf}\dfrac{1}{1+\e{x}\abs{\sin x}}\odif{x}\) converge.
\end{exo}

\begin{corr}
La fonction \(x\mapsto\dfrac{1}{1+\e{x}\abs{\sin x}}\) est continue et positive sur \(\intervie{0}{\pinf}\) donc \(\int_0^{\pinf}\dfrac{1}{1+\e{x}\abs{\sin x}}\odif{x}\) converge ssi la série \(\sum\int_{n\pi}^{\paren{n+1}\pi}\dfrac{1}{1+\e{x}\abs{\sin x}}\odif{x}\) converge.

Pour \(n\in\N\), on a \[\begin{WithArrows}
0&\leq\int_{n\pi}^{\paren{n+1}\pi}\dfrac{1}{1+\e{x}\abs{\sin x}}\odif{x} \Arrow{\(\begin{dcases}x=t+n\pi \\ \odif{x}=\odif{x}\end{dcases}\)} \\
&=\int_0^\pi\dfrac{1}{1+\e{t+n\pi}\abs{\sin t}}\odif{t}.
\end{WithArrows}\]

Or, comme on a \(\quantifs{\forall t\in\intervii{0}{\dfrac{\pi}{2}}}\dfrac{2}{\pi}t\leq\sin t\), on a \[1+\e{n\pi}\dfrac{2}{\pi}t\leq1+\e{t+n\pi}\dfrac{2}{\pi}t\leq1+\e{t+n\pi}\sin t\] donc \[\begin{aligned}
\int_0^{\nicefrac{\pi}{2}}\dfrac{1}{1+\e{t+n\pi}\abs{\sin t}}\odif{t}&\leq\int_0^{\nicefrac{\pi}{2}}\dfrac{1}{1+\e{t+n\pi}\dfrac{2}{\pi}t}\odif{t} \\
&\leq\int_0^{\nicefrac{\pi}{2}}\dfrac{1}{1+\e{n\pi}\dfrac{2}{\pi}t}\odif{t} \\
&=\croch{\dfrac{\ln\paren{1+\e{n\pi}\dfrac{2}{\pi}t}}{\e{n\pi}\dfrac{2}{\pi}}}_0^{\nicefrac{\pi}{2}} \\
&=\dfrac{\pi}{2}\dfrac{\ln\paren{1+\e{n\pi}}}{\e{n\pi}}.
\end{aligned}\]

De même, on a \(\quantifs{\forall t\in\intervii{\dfrac{\pi}{2}}{\pi}}\dfrac{2}{\pi}\paren{\pi-t}\leq\sin t\) donc \[\begin{aligned}
\int_{\nicefrac{\pi}{2}}^\pi\dfrac{1}{1+\e{t+n\pi}\sin t}\odif{t}&\leq\int_{\nicefrac{\pi}{2}}^\pi\dfrac{1}{1+\e{n\pi}\dfrac{2}{\pi}\paren{\pi-t}}\odif{t} \\
&=\croch{\dfrac{\ln\paren{1+\e{n\pi}\dfrac{2}{\pi}\paren{\pi-t}}}{-\e{n\pi}\dfrac{2}{\pi}}}_{\nicefrac{\pi}{2}}^\pi \\
&=\dfrac{\pi}{2}\dfrac{\ln\paren{1+\e{n\pi}}}{\e{n\pi}}.
\end{aligned}\]

Ainsi, pour tout \(n\in\N\), on a \[0\leq\int_{n\pi}^{\paren{n+1}\pi}\dfrac{1}{1+\e{x}\abs{\sin x}}\odif{x}\leq\pi\dfrac{\ln\paren{1+\e{n\pi}}}{\e{n\pi}}=a_n.\]

Or \(1+\e{n\pi}\simqd{n\to\pinf}\e{n\pi}\tendqd{n\to\pinf}\pinf\) donc \(\ln\paren{1+\e{n\pi}}\simqd{n\to\pinf}\ln\paren{\e{n\pi}}=n\pi\).

Donc \(a_n\simqd{n\to\pinf}\pi^2\dfrac{n}{\e{n\pi}}\egqd{n\to\pinf}\o{\dfrac{1}{n^2}}\).

Or \(\sum\dfrac{1}{n^2}\) converge donc \(\sum a_n\) converge.

Donc \(\sum\int_{n\pi}^{\paren{n+1}\pi}\dfrac{1}{1+\e{x}\abs{\sin x}}\odif{x}\) converge.

Donc \(\int_0^{\pinf}\dfrac{1}{1+\e{x}\abs{\sin x}}\odif{x}\) converge.
\end{corr}

\section{Intégrales généralisées sur d'autres types d'intervalles}

\subsection{Intégrales généralisées sur \(\intervie{a}{b}\)}

Dans cette partie, \(a\) est un réel et \(b\) est un réel ou \(\pinf\), de sorte que \(a<b\).

\begin{defi}
Soit \(f\in\contm[\intervie{a}{b}]\).

On dit que l'intégrale \(\int_a^bf\) converge quand \(\int_a^xf\) a une limite finie quand \(x\) tend vers \(b^-\).

Dans ce cas, on pose \(\int_a^bf=\lim_{x\to b^-}\int_a^xf\).

Dans le cas contraire, on dit que l'intégrale \(\int_a^bf\) diverge (ou qu'elle est divergente).
\end{defi}

\subsection{Intégrales généralisées sur \(\intervei{a}{b}\)}

Dans cette partie, \(a\) est un réel ou \(\minf\) et \(b\) un réel, de sorte que \(a<b\).

\begin{defi}
Soit \(f\in\contm[\intervei{a}{b}]\).

On dit que l'intégrale \(\int_a^bf\) converge quand \(\int_x^bf\) a une limite finie quand \(x\) tend vers \(a^+\).

Dans ce cas, on pose \(\int_a^bf=\lim_{x\to a^+}\int_x^bf\).

Dans le cas contraire, on dit que l'intégrale \(\int_a^bf\) diverge (ou qu'elle est divergente).
\end{defi}

Les résultats suivants sont à connaître.

\begin{ex}~\\
\begin{itemize}
    \item \(\int_0^1\ln\paren{t}\odif{t}\) converge. \\
    \item Soit \(\alpha\) un réel. L'intégrale \(\int_0^1\dfrac{1}{t^\alpha}\odif{t}\) converge ssi \(\alpha<1\) (intégrale dite de Riemann).
\end{itemize}
\end{ex}

\begin{dem}
\begin{itemize}
    \item La fonction \(\ln\) est continue sur \(\intervei{0}{1}\). \\\\ Pour \(x\in\intervei{0}{1}\), on a \[\int_x^1\ln\paren{t}\odif{t}=\croch{t\ln t-t}_x^1=-1-x\ln x+x.\] Or \(\lim_{x\to0}x\ln x=0\) donc \(\int_x^1\ln\paren{t}\odif{t}\tendqd{x\to0}-1\). \\\\ Donc \(\int_0^1\ln\paren{t}\odif{t}\) converge et vaut \(-1\). \\
    \item La fonction \(t\mapsto\dfrac{1}{t^\alpha}\) est continue sur \(\intervei{0}{1}\). \\\\ Pour \(x\in\intervei{0}{1}\), on a \[\int_x^1\dfrac{1}{t^\alpha}\odif{t}=\int_x^1t^{-\alpha}\odif{t}=\croch{\dfrac{1}{1-\alpha}t^{1-\alpha}}_x^1=\dfrac{1}{1-\alpha}\paren{1-x^{1-\alpha}}.\] Donc \[\begin{dcases}
        \int_x^1\dfrac{1}{t^\alpha}\odif{t}\tendqd{x\to0}\dfrac{1}{1-\alpha} &\text{si }\alpha<1 \\
        \int_x^1\dfrac{1}{t^\alpha}\odif{t}\tendqd{x\to0}\pinf &\text{si }\alpha>1 \\
        \int_x^1\dfrac{1}{t^\alpha}\odif{t}=-\ln x\tendqd{x\to0}\pinf &\text{si }\alpha=1
    \end{dcases}\] Donc \(\int_0^1\dfrac{1}{t^\alpha}\odif{t}\) converge ssi \(\alpha<1\) et vaut \(\dfrac{1}{1-\alpha}\).
\end{itemize}
\end{dem}

\begin{exo}
Montrez que l'intégrale \(\int_1^2\dfrac{1}{\ln t}\odif{t}\) diverge.

À quelle condition sur \(\alpha\) l'intégrale \(\int_1^2\dfrac{\paren{t-1}^\alpha}{\ln t}\odif{t}\) converge-t-elle ?
\end{exo}

\begin{corr}
\begin{itemize}
    \item La fonction \(t\mapsto\dfrac{1}{\ln t}\) est continue sur \(\intervei{1}{2}\). \\\\ On veut montrer que \(\int_1^2\dfrac{1}{\ln t}\odif{t}\) diverge \ie \(\int_x^2\dfrac{1}{\ln t}\odif{t}\) n'a pas de limite réelle quand \(x\to1^+\). \\\\ On effectue le changement de variable \(\begin{dcases}
        s=\ln t \\
        \odif{s}=\dfrac{\odif{t}}{t}
    \end{dcases}\ssi\begin{dcases}
        t=\e{s} \\
        \odif{t}=\e{s}\odif{s}
    \end{dcases}\) \\\\ On a ramené le problème à la non-existence d'une limite réelle de \(\int_X^{\ln2}\dfrac{\e{s}}{s}\odif{s}\) quand \(X\to0\) \ie à la divergence de \(\int_0^{\ln2}\dfrac{\e{s}}{s}\odif{s}\). \\\\ Or \(\quantifs{\tpt s>0}\dfrac{\e{s}}{s}\geq\dfrac{1}{s}\geq0\) et \(\int_0^{\ln2}\dfrac{1}{s}\odif{s}\) diverge. \\\\ Donc par comparaison de fonctions positives, \(\int_0^{\ln2}\dfrac{\e{s}}{s}\odif{s}\) diverge \ie \(\int_1^2\dfrac{1}{\ln t}\odif{t}\) diverge. \\
    \item Autre méthode : \\\\ \(\quantifs{\Tpt t>1}\ln t\leq t-1\) donc \(0\leq\dfrac{1}{t-1}\leq\dfrac{1}{\ln t}\). \\\\ Or \(\int_1^2\dfrac{1}{t-1}\odif{t}=\int_0^1\dfrac{1}{t}\odif{t}\) diverge. \\\\ Donc par comparaison de fonctions positives, \(\int_1^2\dfrac{1}{\ln t}\odif{t}\) diverge. \\
    \item La fonction \(t\mapsto\dfrac{\paren{t-1}^\alpha}{\ln t}\) est continue et positive sur \(\intervei{1}{2}\). \\\\ On a \(\dfrac{\paren{t-1}^\alpha}{\ln t}\simqd{t\to1}\paren{t-1}^{\alpha-1}=\dfrac{1}{\paren{t-1}^{1-\alpha}}\) car \(\ln t\simqd{t\to1}t-1\). \\\\ Donc \(\int_1^2\dfrac{\paren{t-1}^\alpha}{\ln t}\odif{t}\) converge ssi \(1-\alpha<1\) \ie \(\alpha>0\).
\end{itemize}
\end{corr}

On peut remarquer que par changement de variable \(x\mapsto-x\), l'intégrale \(\int_a^bf\paren{t}\odif{t}\) est de même nature (et, dans le cas convergent, a la même valeur) que \(\int_{-b}^{-a}f\paren{-u}\odif{u}\). Les résultats valables en un point réel ne dépendent donc pas du côté du point où on se place.

\begin{ex}
\begin{itemize}
    \item Si \(a\) est un réel, alors \(\int_a^{a+1}\dfrac{1}{\paren{t-a}^\alpha}\odif{t}\) converge ssi \(\alpha<1\). \\
    \item Si \(a\) est un réel, alors \(\int_{a-1}^a\dfrac{1}{\paren{a-t}^\alpha}\odif{t}\) converge ssi \(\alpha<1\). \\
    \item \(\int_0^1\ln\paren{1-t}\odif{t}\) converge.
\end{itemize}
\end{ex}

\subsection{Intégrales généralisées sur \(\intervee{a}{b}\)}

Dans cette partie, \(a\) et \(b\) sont des réels ou des infinis tels que \(a<b\).

\begin{defi}
Soit \(f\in\contm[\intervee{a}{b}]\).

On dit que l'intégrale \(\int_a^bf\) converge quand il existe \(c\in\intervee{a}{b}\) tel que \(\int_a^cf\) et \(\int_c^bf\) convergent.

Dans ce cas, on pose \(\int_a^bf=\int_a^cf+\int_c^bf=\lim_{x\to a^+}\int_x^cf+\lim_{y\to b^-}\int_c^yf\).

Dans le cas contraire, on dit que l'intégrale \(\int_a^bf\) diverge (ou qu'elle est divergente).
\end{defi}

\begin{rem}
Grâce à la relation de Chasles, on constate que la valeur de \(c\) n'est finalement pas importante : si ça marche pour un certain réel \(c\in\intervee{a}{b}\), alors ça marche pour toute autre valeur prise dans \(\intervee{a}{b}\).
\end{rem}

\begin{exo}
Montrez que l'intégrale de Gauss \(\int_{\minf}^{\pinf}\e{-t^2}\odif{t}\) converge.
\end{exo}

\begin{corr}
La fonction \(t\mapsto\e{-t^2}\) est continue et positive sur \(\intervee{\minf}{\pinf}\).

\(\int_1^{\pinf}\e{-t^2}\odif{t}\) converge car \(\quantifs{\tpt t\in\intervie{1}{\pinf}}t^2\geq t\) donc \(0\leq\e{-t^2}\leq\e{-t}\) et \(\int_1^{\pinf}\e{-t}\odif{t}\) converge.

Or \(t\mapsto\e{-t^2}\) est paire donc \(\int_{\minf}^{-1}\e{-t^2}\odif{t}\) est de même nature que \(\int_1^{\pinf}\e{-t^2}\) donc elle converge.

Finalement, \(\int_{\minf}^{\pinf}\e{-t^2}\odif{t}\) converge.
\end{corr}

\begin{exo}
Même chose avec l'intégrale \(\int_0^1\dfrac{1}{\sqrt{t\paren{1-t}}}\odif{t}\).
\end{exo}

\begin{corr}
La fonction \(t\mapsto\dfrac{1}{\sqrt{t\paren{1-t}}}\) est continue et positive sur \(\intervee{0}{1}\).

\(\quantifs{\Tpt t\in\intervei{0}{\dfrac{1}{2}}}0\leq\dfrac{1}{\sqrt{t\paren{1-t}}}\leq\dfrac{2}{\sqrt{t}}=\dfrac{2}{t^{\nicefrac{1}{2}}}\).

Or \(\int_0^{\nicefrac{1}{2}}\dfrac{1}{t^{\nicefrac{1}{2}}}\odif{t}\) converge donc \(\int_0^{\nicefrac{1}{2}}\dfrac{1}{\sqrt{t\paren{1-t}}}\odif{t}\) converge.

\(\quantifs{\Tpt t\in\intervie{\dfrac{1}{2}}{1}}0\leq\dfrac{1}{\sqrt{t\paren{1-t}}}\leq\dfrac{2}{\sqrt{1-t}}=\dfrac{2}{\paren{1-t}^{\nicefrac{1}{2}}}\).

Or \(\int_{\nicefrac{1}{2}}^1\dfrac{1}{\paren{1-t}^{\nicefrac{1}{2}}}\odif{t}\) converge donc \(\int_{\nicefrac{1}{2}}^1\dfrac{1}{\sqrt{t\paren{1-t}}}\odif{t}\) converge.

Finalement, \(\int_0^1\dfrac{1}{\sqrt{t\paren{1-t}}}\odif{t}\) converge.
\end{corr}

\begin{exo}
Même chose avec l'intégrale \(\int_0^{\pinf}\dfrac{\e{-t}}{\sqrt{t}}\odif{t}\).
\end{exo}

\begin{corr}
La fonction \(t\mapsto\dfrac{\e{-t}}{\sqrt{t}}\) est continue et positive sur \(\intervee{0}{\pinf}\).

\(\quantifs{\Tpt t\geq1}0\leq\dfrac{\e{-t}}{\sqrt{t}}\leq\e{-t}\) et \(\int_1^{\pinf}\e{-t}\odif{t}\) converge donc \(\int_1^{\pinf}\dfrac{\e{-t}}{\sqrt{t}}\odif{t}\) converge.

On a \(\dfrac{\e{-t}}{\sqrt{t}}\simqd{t\to0}\dfrac{1}{\sqrt{t}}\) et \(\int_0^1\dfrac{1}{\sqrt{t}}\odif{t}\) converge donc \(\int_0^1\dfrac{\e{-t}}{\sqrt{t}}\odif{t}\) converge.

Finalement, \(\int_0^{\pinf}\dfrac{\e{-t}}{\sqrt{t}}\odif{t}\) converge.
\end{corr}

\begin{exo}
Montrez que \(\quantifs{\tpt\alpha>0}\int_0^{\pinf}\dfrac{1}{t^\alpha}\odif{t}\) diverge.
\end{exo}

\begin{corr}
Clair.
\end{corr}

\subsection{Propriétés communes à toutes ces intégrales}

Toutes les propriétés vues dans la première section sont préservées : linéarité, positivité, croissance, relation de Chasles.

En voici trois autres.

\subsubsection{Changement de variable}

\begin{prop}
Soient \(a,b,\alpha,\beta\) des réels ou des infinis tels que \(a<b\) et \(\alpha<\beta\) et \(f\in\contm[\intervee{a}{b}]\).

Si \(\phi\) est une bijection de classe \(\classe{1}\) et strictement croissante de \(\intervee{\alpha}{\beta}\) dans \(\intervee{a}{b}\), alors les intégrales \(\int_a^bf\paren{t}\odif{t}\) et \(\int_\alpha^\beta f\rond\phi\paren{u}\phi\prim\paren{u}\odif{u}\) sont de même nature et, si elles sont convergentes, sont égales.
\end{prop}

\begin{dem}
Soit \(\paren{x,y}\in\R^2\) tel que \(a<x\leq y<b\).

D'après la formule classique du changement de variable, on a \[\int_x^yf\paren{t}\odif{t}=\int_{\phi\inv\paren{x}}^{\phi\inv\paren{y}}f\paren{\phi\paren{u}}\phi\prim\paren{u}\odif{u}.\]

On a \(\phi\inv\paren{x}\tendqd{x\to a}\alpha\) et \(\phi\inv\paren{y}\tendqd{y\to b}\beta\).

Donc \[\begin{aligned}
\int_a^bf\text{ converge}&\ssi\int_x^yf\text{ a une limite finie quand }x\to a^+\text{ et }y\to b^- \\
&\ssi\int_{\phi\inv\paren{x}}^{\phi\inv\paren{y}}f\paren{\phi\paren{u}}\phi\prim\paren{u}\odif{u}\text{ a une limite finie quand }x\to a^+\text{ et }y\to b^- \\
&\ssi\int_\lambda^\mu f\paren{\phi\paren{u}}\phi\prim\paren{u}\odif{u}\text{ a une limite finie quand }\lambda\to\alpha^+\text{ et }\mu\to\beta^- \\
&\ssi\int_\alpha^\beta f\paren{\phi\paren{u}}\phi\prim\paren{u}\odif{u}\text{ converge}.
\end{aligned}\]
\end{dem}

\begin{rem}
On a évidemment un résultat analogue avec un changement de variable strictement décroissant et des bornes inversées.

Comme une bijection de classe \(\classe{1}\) entre deux intervalles est forcément strictement monotone, l'hypothèse de stricte monotonie est redondante ; mais comme elle est explicitement dans le programme de MPI, il vaut mieux la préciser (de toute façon, elle sera évidente dans les cas pratiques et ne nécessitera pas de longues preuves).
\end{rem}

\begin{exo}~\\
Montrez que \(\int_0^{\nicefrac{\pi}{2}}\sqrt{\tan t}\odif{t}\) converge et qu'on a l'égalité \(\int_0^{\nicefrac{\pi}{2}}\sqrt{\tan t}\odif{t}=\int_0^{\pinf}\dfrac{\sqrt{u}}{1+u^2}\odif{u}\).
\end{exo}

\begin{corr}
On effectue le changement de variable \(u=\tan t\).

Sous réserve de convergence, on a \[\int_0^{\nicefrac{\pi}{2}}\sqrt{\tan t}\odif{t}=\int_0^{\pinf}\dfrac{\sqrt{u}}{1+u^2}\odif{u}.\]

Or, comme \(u\mapsto\dfrac{\sqrt{u}}{1+u^2}\) est continue et positive sur \(\intervie{0}{\pinf}\) et \(\dfrac{\sqrt{u}}{1+u^2}\simqd{u\to\pinf}\dfrac{1}{u^{\nicefrac{3}{2}}}\) et \(\dfrac{3}{2}>1\), \(\int_0^{\pinf}\dfrac{\sqrt{u}}{1+u^2}\odif{u}\) converge.
\end{corr}

\begin{exo}~\\
Montrez que \(\int_0^{\pinf}\dfrac{\ln t}{1+t^2}\odif{t}\) converge et vaut \(0\).

Déduisez-en la valeur de \(\int_0^{\pinf}\dfrac{\ln t}{a^2+t^2}\odif{t}\) où \(a>0\).
\end{exo}

\begin{corr}
\begin{itemize}
    \item La fonction \(\phi:t\mapsto\dfrac{\ln t}{1+t^2}\) est continue sur \(\intervee{0}{\pinf}\). \\\\ Sur \(\intervei{0}{1}\), \(\phi\) est négative et sur \(\intervie{1}{\pinf}\), \(\phi\) est positive. \\\\ On a \(\phi\paren{t}\simqd{t\to0}\ln t\) et \(\int_0^1\ln\paren{t}\odif{t}\) converge donc \(\int_0^1\phi\paren{t}\odif{t}\) converge par comparaison de fonctions à signe constant. \\\\ On a \(\ln t\egqd{t\to\pinf}\o{\sqrt{t}}\) donc il existe \(A>0\) tel que \(\quantifs{\tpt t\geq A}0\leq\ln t\leq\sqrt{t}\) donc \[0\leq\dfrac{\ln t}{1+t^2}\leq\dfrac{\sqrt{t}}{1+t^2}\leq\dfrac{\sqrt{t}}{t^2}=\dfrac{1}{t^{\nicefrac{3}{2}}}\]. \\\\ Or \(\dfrac{3}{2}>1\) donc \(\int_A^{\pinf}\dfrac{1}{t^{\nicefrac{3}{2}}}\odif{t}\) converge donc \(\int_A^{\pinf}\dfrac{\ln t}{1+t^2}\odif{t}\) converge et donc \(\int_1^{\pinf}\dfrac{\ln t}{1+t^2}\odif{t}\) converge. \\\\ On veut montrer que \(\int_0^1\dfrac{\ln t}{1+t^2}\odif{t}=-\int_1^{\pinf}\dfrac{\ln t}{1+t^2}\odif{t}\) de sorte que \(\int_0^{\pinf}\dfrac{\ln t}{1+t^2}\odif{t}=0\). \\\\ En effectuant le changement de variable \(\begin{dcases}
        t=\dfrac{1}{u} \\
        \odif{t}=\dfrac{-1}{u^2}\odif{u}
    \end{dcases}\) (où \(t\mapsto\dfrac{1}{t}\) est une bijection \(\classe{1}\) strictement décroissante de \(\intervei{0}{1}\) dans \(\intervie{1}{\pinf}\)), on obtient \[\int_0^1\dfrac{\ln t}{1+t^2}\odif{t}=\int_{\pinf}^1\dfrac{\ln\nicefrac{1}{u}}{1+\paren{\nicefrac{1}{u}}^2}\paren{\dfrac{-1}{u^2}}\odif{u}=-\int_1^{\pinf}\dfrac{\ln u}{1+u^2}\odif{u}.\] D'où l'égalité. \\\\ Donc \(\int_0^{\pinf}\dfrac{\ln t}{1+t^2}\odif{t}=0\). \\
    \item Soit \(a>0\). \\\\ \(\int_0^{\pinf}\dfrac{\ln t}{a^2+t^2}\odif{t}\) converge (idem). \\\\ On a \[\begin{WithArrows}
        \int_0^{\pinf}\dfrac{\ln t}{a^2+t^2}\odif{t}&=\dfrac{1}{a^2}\int_0^{\pinf}\dfrac{\ln t}{1+\paren{\nicefrac{t}{a}}^2}\odif{t} \Arrow{\(\begin{dcases}t=au \\ \odif{t}=a\odif{u}\end{dcases}\)} \\
        &=\dfrac{1}{a^2}\int_0^{\pinf}\dfrac{a\ln\paren{au}}{1+u^2}\odif{u} \\
        &=\dfrac{1}{a}\int_0^{\pinf}\dfrac{\ln a}{1+u^2}\odif{u}+\underbrace{\dfrac{1}{a}\int_0^{\pinf}\dfrac{\ln u}{1+u^2}\odif{u}}_{=0} \\
        &=\dfrac{\pi\ln a}{2a}.
    \end{WithArrows}\]
\end{itemize}
\end{corr}

\subsubsection{Intégration par parties}

Si \(f\) est une fonction définie sur \(\intervee{a}{b}\) et a des limites finies en \(a^+\) et en \(b^-\), on note \(\croch{f}_a^b=\lim_{b^-}f-\lim_{a^+}f\).

\begin{prop}
Soient \(a,b\) des réels ou des infinis tels que \(a<b\) et \(\paren{f,g}\in\ensclasse{1}{\intervee{a}{b}}{\K}\).

Si parmi les trois quantités suivantes \[\int_a^bf\prim g\qquad\int_a^bfg\prim\qquad\croch{fg}_a^b\] deux existent, alors la troisième existe aussi et, dans ce cas, on a l'égalité habituelle \[\int_a^bf\prim g=\croch{fg}_a^b-\int_a^bfg\prim\].
\end{prop}

\begin{dem}
Soit \(\paren{x,y}\in\R^2\) tel que \(a<x\leq y<b\).

On a \(\int_x^yfg\prim=\croch{fg}_x^y-\int_x^yf\prim g\).

Si parmi les quantités \(\int_x^yfg\prim\), \(\int_x^yf\prim g\) et \(\croch{fg}_x^y=f\paren{y}g\paren{y}-f\paren{x}g\paren{x}\) deux ont des limites finies quand \(x\to a\) et \(y\to b\), alors la troisième aussi (par opération sur les limites).

Dans ce cas, les limites sont égales : \[\int_a^bfg\prim=\croch{fg}_a^b-\int_a^bf\prim g.\]
\end{dem}

En pratique, pour éviter d'écrire des choses qui n'ont pas de sens, il vaut mieux revenir à une vraie intégration par parties sur un segment \(\intervii{x}{y}\subset\intervee{a}{b}\), s'assurer qu'on peut faire tendre \(x\) vers \(a\) et \(y\) vers \(b\), puis le faire effectivement pour obtenir la relation entre les intégrales.

\begin{exo}
Pour \(n\in\N\), on pose \(u_n=\int_0^{\pinf}t^n\e{-t}\odif{t}\).

Montrez que les intégrales \(u_n\) convergent, donnez une relation de récurrence simple entre \(u_n\) et \(u_{n+1}\), puis donnez la valeur de \(u_n\) en fonction de \(n\).
\end{exo}

\begin{corr}
Pour \(n\in\N\), la fonction \(t\mapsto t^n\e{-t}\) est continue sur \(\intervie{0}{\pinf}\).

Sous réserve de converge, on pose \(u_n=\int_0^{\pinf}t^n\e{-t}\odif{t}\) pour \(n\in\N\).

On a \(t^{n+2}\egqd{t\to\pinf}\o{\e{t}}\) donc \(t^n\e{-t}\egqd{t\to\pinf}\o{\dfrac{1}{t^2}}\).

Donc il existe \(a>0\) tel que \(\quantifs{\forall t\geq a}0\leq t^n\e{-t}\leq\dfrac{1}{t^2}\).

Or \(\int_a^{\pinf}\dfrac{1}{t^2}\odif{t}\) converge donc \(\int_a^{\pinf}t^n\e{-t}\odif{t}\) converge et donc \(\int_0^{\pinf}t^n\e{-t}\odif{t}\) converge.

Sous réserve de converge, par intégration par parties, on a \[\int_0^{\pinf}t^{n+1}\e{-t}\odif{t}=\croch{-t^{n+1}\e{-t}}_0^{\pinf}+\int_0^{\pinf}\paren{n+1}t^n\e{-t}.\]

Les deux intégrales convergent d'après ce qui précède donc l'intégration par parties est licite.

Ainsi, comme \(\croch{-t^{n+1}\e{-t}}_0^{\pinf}=0\), on a \(u_{n+1}=\paren{n+1}u_n\).

Or \(u_0=\int_0^{\pinf}\e{-t}\odif{t}=1\).

Donc \(\quantifs{\forall n\in\N}u_n=n!\) (récurrence immédiate).
\end{corr}

\begin{exo}
Montrez que l'intégrale \(\int_0^{\pinf}\dfrac{1-\cos t}{t^2}\odif{t}\) converge, puis déduisez-en que l'intégrale \(\int_0^{\pinf}\dfrac{\sin t}{t}\odif{t}\) converge.
\end{exo}

\begin{corr}
\begin{itemize}
    \item La fonction \(t\mapsto\dfrac{1-\cos t}{t^2}\) est continue sur \(\intervee{0}{\pinf}\). \\\\ \(\quantifs{\Tpt t>0}-1\leq\cos t\leq1\) donc \(0\leq1-\cos t\leq2\) donc \(0\leq\dfrac{1-\cos t}{t^2}\leq\dfrac{2}{t^2}\). \\\\ Or \(\int_1^{\pinf}\dfrac{1}{t^2}\odif{t}\) converge donc \(\int_1^{\pinf}\dfrac{1-\cos t}{t^2}\odif{t}\) converge. \\\\ On a \(\cos t\egqd{t\to0}1-\dfrac{t^2}{2}+\o{t^2}\) donc \(1-\cos t\simqd{t\to0}\dfrac{t^2}{2}\) donc \(\dfrac{1-\cos t}{t^2}\simqd{t\to0}\dfrac{1}{2}\). \\\\ Or \(\int_0^1\dfrac{1}{2}\odif{t}\) converge donc \(\int_0^1\dfrac{1-\cos t}{t^2}\odif{t}\) converge. \\\\ Finalement, \(\int_0^{\pinf}\dfrac{1-\cos t}{t^2}\) converge. \\
    \item Soit \(\paren{x,y}\in\R^2\) tel que \(0<x\leq y\). \\\\ On a \[\int_x^y\dfrac{1-\cos t}{t^2}\odif{t}=\croch{-\dfrac{1-\cos t}{t}}_x^y+\int_x^y\dfrac{\sin t}{t}\odif{t}.\] On sait déjà que \(\int_x^y\dfrac{1-\cos t}{t^2}\odif{t}\) a une limite réelle quand \(x\to0\) et \(y\to\pinf\). \\\\ De plus, on a \(\croch{-\dfrac{1-\cos t}{t}}_x^y=\dfrac{1-\cos x}{x}-\dfrac{1-\cos y}{y}\). \\\\ Or \(\dfrac{1-\cos x}{x}\simqd{x\to0}\dfrac{x}{2}\) donc \(\lim_{x\to0}\dfrac{1-\cos x}{x}=0\) et \(\dfrac{1-\cos y}{y}\egqd{y\to\pinf}\O{\dfrac{1}{y}}\) donc \(\lim_{y\to\pinf}\dfrac{1-\cos y}{y}=0\) par encadrement. \\\\ Donc \(\int_x^y\dfrac{\sin t}{t}\odif{t}\) a une limite réelle quand \(x\to0\) et \(y\to\pinf\) \ie \(\int_0^{\pinf}\dfrac{\sin t}{t}\odif{t}\) converge (NB : elle se nomme intégrale de Dirichlet). \\
    \item On peut aussi montrer que \(\int_0^{\pinf}\dfrac{\sin t}{t}\odif{t}=\int_0^{\pinf}\dfrac{\sin^2t}{t^2}\odif{t}\). \\\\ En effet, on a \[\begin{WithArrows}
        \int_0^{\pinf}\dfrac{\sin t}{t}\odif{t}&=\int_0^{\pinf}\dfrac{1-\cos t}{t^2}\odif{t} \\
        &=\int_0^{\pinf}\dfrac{2\sin^2\nicefrac{t}{2}}{t^2}\odif{t} \Arrow{\(\begin{dcases}t=2u \\ \odif{t}=2\odif{u}\end{dcases}\)} \\
        &=\int_0^{\pinf}\dfrac{2\sin^2u}{4u^2}\odif{u} \\
        &=\int_0^{\pinf}\dfrac{\sin^2u}{u^2}\odif{u}.
    \end{WithArrows}\]
\end{itemize}
\end{corr}

\subsubsection{Primitives}

\begin{prop}
Soient \(f\in\contm[\intervee{a}{b}]\) et \(c\in\intervee{a}{b}\) tel que l'intégrale \(\int_a^cf\) converge.

Alors la fonction \(x\mapsto\int_a^xf\) est l'unique primitive de \(f\) sur \(\intervee{a}{b}\) qui a pour limite \(0\) en \(a^+\).
\end{prop}

\begin{dem}
La fonction \(f\) est continue sur \(\intervee{a}{b}\) donc elle possède une primitive \(F\) sur \(\intervee{a}{b}\). Par exemple \(F:x\mapsto\int_c^xf\) où \(x\in\intervee{a}{b}\).

\(\int_a^cf\) converge donc \(F\) a une limite finie en \(a^+\).

Donc \[\begin{aligned}
\quantifs{\forall x\in\intervee{a}{b}}\int_a^xf&=\lim_{y\to a^+}\int_y^xf \\
&=\lim_{y\to a^+}\paren{F\paren{x}-F\paren{y}} \\
&=F\paren{x}-\underbrace{\lim_{y\to a^+}F\paren{y}.}_{\text{constante}}
\end{aligned}\]

Donc \(x\mapsto\int_a^xf\) est une primitive de \(f\) sur \(\intervee{a}{b}\).

De plus, \(\lim_{x\to a^+}\int_a^xf=\lim_{x\to a^+}F\paren{x}-\lim_{y\to a^+}F\paren{y}=0\).
\end{dem}

\begin{prop}
Soient \(f\in\contm[\intervee{a}{b}]\) et \(c\in\intervee{a}{b}\) tel que \(\int_c^bf\) converge.

Alors la fonction \(x\mapsto\int_x^bf\) est l'opposée de l'unique primitive de \(f\) sur \(\intervee{a}{b}\) qui a pour limite \(0\) en \(b^-\).
\end{prop}

\begin{ex}
\begin{itemize}
    \item La fonction \(x\mapsto\int_0^x\ln\paren{t}\odif{t}\) est la primitive de \(\ln\) qui a pour limite \(0\) en \(0\). \\
    \item La fonction \(x\mapsto\int_x^{\pinf}\dfrac{\e{\i t}}{t^2}\odif{t}\) est définie sur \(\intervee{0}{\pinf}\), de classe \(\classe{1}\) sur cet intervalle, et sa dérivée est la fonction \(x\mapsto-\dfrac{\e{\i x}}{x^2}\).
\end{itemize}
\end{ex}

\section{Résumé pour étudier la convergence d'une intégrale}

On veut savoir si une intégrale \(\int_a^bf\) existe, où \(a\) et \(b\) sont des réels ou des infinis tels que \(a<b\).

D'abord, on détermine le plus grand sous-ensemble de \(\intervii{a}{b}\) sur lequel \(f\) est continue par morceaux :

\begin{itemize}
    \item si c'est \(\intervii{a}{b}\), alors il n'y a aucun problème d'existence de l'intégrale : c'est une bête intégrale classique ; \\
    \item si c'est \(\intervie{a}{b}\) (avec \(a\) réel) ou \(\intervei{a}{b}\) (avec \(b\) réel), alors il faut étudier le comportement de \(f\) au voisinage du point ouvert ; \\
    \item si c'est \(\intervee{a}{b}\), alors on choisit arbitrairement un point \(c\in\intervee{a}{b}\) et on se ramène deux fois au cas précédent.
\end{itemize}

Un petit résultat qui supprime parfois le problème en un point ouvert réel : pensez à étudier la limite de la fonction : si elle est réelle, c'est réglé. On dit qu'on a une fausse singularité en ce point réel.

\begin{prop}
Soient \(a\) et \(b\) deux réels tels que \(a<b\) et \(f\in\contm[\intervei{a}{b}]\).

Si \(f\) a une limite réelle en \(a\) par valeurs supérieures, alors on peut prolonger \(f\) par continuité en \(a\), le prolongement \(\conj{f}\) est une fonction continue par morceaux sur \(\intervii{a}{b}\) et l'intégrale \(\int_a^bf\) converge et vaut \(\int_a^b\conj{f}\).
\end{prop}

\begin{rem}
Si \(f\) est continue par morceaux sur \(\intervie{0}{\pinf}\) et si \(f\paren{x}\tendqd{x\to\pinf}l\), alors si \(l\not=0\), \(\int_0^{\pinf}f\) diverge (grossièrement).
\end{rem}

\begin{exo}
Justifiez que l'intégrale \(\int_0^{\pinf}\dfrac{\sin^2t}{t^2}\odif{t}\) converge.
\end{exo}

\begin{corr}
La fonction \(t\mapsto\dfrac{\sin^2t}{t^2}\) est continue sur \(\intervee{0}{\pinf}\).

On a \(\sin t\simqd{t\to0}t\) donc \(\dfrac{\sin^2t}{t^2}\tendqd{t\to0}1\) donc \(\int_0^1\dfrac{\sin^2t}{t^2}\odif{t}\) converge.

\(\quantifs{\Tpt t\geq1}0\leq\dfrac{\sin^2t}{t^2}\leq\dfrac{1}{t^2}\) or \(\int_1^{\pinf}\dfrac{1}{t^2}\odif{t}\) converge donc \(\int_1^{\pinf}\dfrac{\sin^2t}{t^2}\odif{t}\) converge.

Finalement, \(\int_0^{\pinf}\dfrac{\sin^2t}{t^2}\odif{t}\) converge.
\end{corr}

\begin{exo}
Montrez que l'intégrale \(\int_0^{\pinf}\dfrac{t\ln t}{1+t^3}\odif{t}\) converge.
\end{exo}

\section{Fonctions intégrables sur un intervalle}

Dans cette section, \(a\) et \(b\) sont des réels ou des infinis tels que \(a<b\). On note \(I=\intervee{a}{b}\).

\subsection{Intégrales absolument convergentes}

\begin{defi}
Soit \(f\in\contm[I]\).

On dit que l'intégrale \(\int_a^bf\) converge absolument (ou est absolument convergente) quand l'intégrale \(\int_a^b\abs{f}\) converge.
\end{defi}

Le théorème suivant est primordial pour la suite du cours.

\begin{theo}
Soit \(f\in\contm[I]\).

Si l'intégrale \(\int_a^bf\) converge absolument, alors l'intégrale \(\int_a^bf\) converge.

Dans ce cas, on a \(\abs{\int_a^bf}\leq\int_a^b\abs{f}\).
\end{theo}

\begin{dem}
\begin{itemize}
    \item Si \(f\) est à valeurs réelles, on a \(0\leq f+\abs{f}\leq2\abs{f}\). \\\\ Donc si \(\int_a^b\abs{f}\) converge, alors par comparaison de fonctions positives, \(\int_a^b\paren{f+\abs{f}}\) converge. \\\\ Donc \(\int_a^bf=\int_a^b\paren{f+\abs{f}}-\int_a^b\abs{f}\) converge. \\
    \item Si \(f\) est à valeurs complexes, on a \(\abs{\Re f}\leq\abs{f}\) et \(\abs{\Im f}\leq\abs{f}\). \\\\ Donc si \(\int_a^b\abs{f}\) converge, \(\int_a^b\abs{\Re f}\) et \(\int_a^b\abs{\Im f}\) convergent par comparaison de fonctions positives. \\\\ Donc d'après ce qui précède, \(\int_a^b\Re f\) et \(\int_a^b\Im f\) convergent. \\\\ Donc \(\int_a^bf=\int_a^b\Re f+\i\int_a^b\Im f\) converge.
\end{itemize}
\end{dem}

\begin{exo}
Montrez que l'intégrale \(\int_0^{\pinf}\dfrac{\sin^3t}{t^3}\odif{t}\) est absolument convergente et donc convergente.
\end{exo}

\begin{corr}
La fonction \(t\mapsto\dfrac{\sin^3t}{t^3}\) est continue sur \(\intervee{0}{\pinf}\).

On a \(\abs{\dfrac{\sin^3t}{t^3}}\simqd{t\to0}\abs{\dfrac{t^3}{t^3}}\tendqd{t\to0}1\) : fausse singularité en \(0\), donc \(\int_0^1\abs{\dfrac{\sin^3t}{t^3}}\odif{t}\) converge.

\(\quantifs{\Tpt t\geq1}\abs{\dfrac{\sin^3t}{t^3}}\leq\dfrac{1}{t^3}\) or \(\int_1^{\pinf}\dfrac{1}{t^3}\odif{t}\) converge donc \(\int_1^{\pinf}\dfrac{\sin^3t}{t^3}\odif{t}\) converge absolument par comparaison de fonctions positives.

Finalement, \(\int_0^{\pinf}\dfrac{\sin^3t}{t^3}\odif{t}\) converge absolument et donc converge.
\end{corr}

\begin{exo}
Montrez que si \(m\) est un complexe de partie réelle strictement positive, alors l'intégrale \(\int_0^{\pinf}\e{-mt}\odif{t}\) converge et donnez sa valeur.

Déduisez-en l'existence et la valeur des intégrales \(\int_0^{\pinf}\sin\paren{t}\e{-t}\odif{t}\) et \(\int_0^{\pinf}\cos\paren{t}\e{-t}\odif{t}\).
\end{exo}

\begin{corr}
\begin{itemize}
    \item On note \(m=a+\i b\) où \(\paren{a,b}\in\Rps\times\R\). \\\\ Pour \(t\in\intervie{0}{\pinf}\), on a \(\e{-mt}=\e{-at}\e{-\i bt}\) donc \(\abs{\e{-mt}}=\e{-at}\). \\\\ Or \(a>0\) donc \(\int_0^{\pinf}\abs{\e{-mt}}\odif{t}\) converge donc \(\int_0^{\pinf}\e{-mt}\odif{t}\) converge. \\
    \item Pour \(X\geq0\), on a \(\int_0^X\e{-mt}\odif{t}=\croch{\dfrac{-\e{-mt}}{m}}_0^X=\dfrac{1}{m}\paren{1-\e{-mX}}\). \\\\ Or \(\abs{\e{-mX}}=\e{-aX}\tendqd{X\to\pinf}0\). \\\\ Donc \(\int_0^{\pinf}\e{-mt}\odif{t}=\dfrac{1}{m}\). \\
    \item Pour \(t\geq0\), on a \(\begin{dcases}
        \abs{\sin\paren{t}\e{-t}}\leq\e{-t} \\
        \abs{\cos\paren{t}\e{-t}}\leq\e{-t}
    \end{dcases}\) et \(\int_0^{\pinf}\e{-t}\odif{t}\) converge donc \(A=\int_0^{\pinf}\cos\paren{t}\e{-t}\odif{t}\) et \(B=\int_0^{\pinf}\sin\paren{t}\e{-t}\odif{t}\) convergent (absolument). \\\\ On a \[\begin{aligned}
        A+\i B&=\int_0^{\pinf}\paren{\cos t+\i\sin t}\e{-t}\odif{t} \\
        &=\int_0^{\pinf}\e{\i t}\e{-t}\odif{t} \\
        &=\int_0^{\pinf}\e{-\paren{1-\i}t}\odif{t} \\
        &=\dfrac{1}{1-i} \\
        &=\dfrac{1+\i}{2}.
    \end{aligned}\] Donc \(A=\Re\dfrac{1+\i}{2}=\dfrac{1}{2}\) et \(B=\Im\dfrac{1+\i}{2}=\dfrac{1}{2}\).
\end{itemize}
\end{corr}

\begin{rem}
La réciproque est fausse ! On a montré que \(\int_0^{\pinf}\dfrac{\sin t}{t}\odif{t}\) converge et qu'elle ne converge pas absolument.
\end{rem}

\subsection{Fonctions intégrables}

\begin{defi}
Soit \(f\in\contm[I]\).

On dit que \(f\) est intégrable sur \(I\) quand l'intégrale \(\int_a^bf\) converge absolument.

On note alors aussi \(\int_If=\int_If\paren{t}\odif{t}=\int_a^bf\).
\end{defi}

L'ensemble des fonctions intégrables sur \(I\) est souvent noté \(\integ{1}{I}{\K}\), L comme Lebesgue, mathématicien français de la fin du 19ème et début du 20ème siècle. Par abus de notation, on écrit parfois \guillemets{\(f\) est \(\integ{1}{}{}\)} pour \guillemets{\(f\) est intégrable sur \(I\)}.

\begin{ex}
Soient \(a\in\R\) et \(\epsilon\in\Rps\).

\begin{itemize}
    \item La fonction \(t\mapsto\e{-t}\) est intégrable sur \(\intervie{\epsilon}{\pinf}\). \\ Plus généralement, la fonction \(t\mapsto t^a\e{-t}\) est intégrable sur \(\intervie{\epsilon}{\pinf}\) (voire \(\intervie{0}{\pinf}\) si \(a>-1\)). \\
    \item La fonction \(t\mapsto\dfrac{1}{t^\alpha}\) est intégrable sur \(\intervie{\epsilon}{\pinf}\) ssi \(\alpha>1\) et sur \(\intervei{0}{\epsilon}\) ssi \(\alpha<1\). \\
    \item La fonction \(\ln\) est intégrable sur \(\intervei{0}{\epsilon}\).
\end{itemize}
\end{ex}

\begin{prop}
\(\integ{1}{I}{\K}\) est un \(\K\)-espace vectoriel.
\end{prop}

\begin{prop}[Stricte positivité de l'intégrale]~\\
Si \(f\) est continue, intégrable sur \(I\) et \(\int_I\abs{f}=0\), alors \(f=0\).

Par contraposée, si \(f\) est continue, intégrable sur \(I\) et \(f\not=0\), alors \(\int_I\abs{f}>0\).
\end{prop}

\subsection{Théorème de comparaison des fonctions intégrables}

\begin{rappel}
Soient \(f\) et \(g\) deux fonctions définies au voisinage d'un point \(p\in\R\union\accol{\minf,\pinf}\).

\begin{itemize}
    \item \(f=\O{g}\) au voisinage de \(p\) signifie qu'il existe \(K>0\) et \(V\) un voisinage de \(p\) tels que \[\quantifs{\forall x\in V}\abs{f\paren{x}}\leq K\abs{g\paren{x}}.\]
    \item \(f=\o{g}\) au voisinage de \(p\) signifie qu'il existe une fonction \(\epsilon>0\) et \(V\) un voisinage de \(p\) tels que \[\quantifs{\forall x\in V}\abs{f\paren{x}}\leq\epsilon\paren{x}\abs{g\paren{x}}\qquad\text{et}\qquad\lim_{x\to p}\epsilon\paren{x}=0.\]
\end{itemize}

Dans le cas où \(g\) ne s'annule pas (ce qui, en pratique, est toujours le cas) :

\begin{itemize}
    \item \(f=\O{g}\) au voisinage de \(p\) signifie que \(\dfrac{f}{g}\) est bornée au voisinage de \(p\). \\
    \item \(f=\o{g}\) au voisinage de \(p\) signifie que \(\dfrac{f}{g}\) a pour limite \(0\) en \(p\).
\end{itemize}
\end{rappel}

\begin{theo}
Soit \(\paren{f,g}\in\contm[I]^2\).

\begin{itemize}
    \item Si \(\abs{f}\leq\abs{g}\) sur \(I\) et si \(g\) est intégrable, alors \(f\) est intégrable. \\
    \item Si \(f=\O{g}\) au voisinage des bornes ouvertes de \(I\) et \(g\) est intégrable, alors \(f\) est intégrable. C'est vrai en particulier si \(f=\o{g}\). \\
    \item Si \(f\sim g\) au voisinage des bornes ouvertes de \(I\), alors il y a équivalence entre l'intégrabilité de \(f\) et l'intégrabilité de \(g\).
\end{itemize}
\end{theo}

\begin{exo}
Montrez que la fonction \(t\mapsto\dfrac{\cos t}{\sqrt{t}}\e{-t}\) est intégrable sur \(\intervee{0}{\pinf}\).
\end{exo}

\begin{corr}
La fonction \(f:t\mapsto\dfrac{\cos t}{\sqrt{t}}\e{-t}\) est continue sur \(\intervee{0}{\pinf}\).

On a \(\dfrac{\cos t}{\sqrt{t}}\e{-t}\simqd{t\to0}\dfrac{1}{\sqrt{t}}\) et \(t\mapsto\dfrac{1}{\sqrt{t}}\) est intégrable sur \(\intervei{0}{1}\) donc \(f\) aussi.

\(\quantifs{\Tpt t\geq1}\abs{\dfrac{\cos t}{\sqrt{t}}\e{-t}}\leq\e{-t}\) et \(t\mapsto\e{-t}\) est intégrable sur \(\intervie{1}{\pinf}\) donc \(f\) aussi.

Finalement, \(f\) est intégrable sur \(\intervee{0}{\pinf}\).
\end{corr}

\begin{exo}
La fonction \(t\mapsto\dfrac{\ln t}{\cos t+t^2}\) est-elle intégrable sur \(\intervee{0}{\pinf}\) ?
\end{exo}

\begin{corr}
\(\quantifs{\Tpt t>1}t^2>1\geq-\cos t\) donc \(t^2+\cos t>0\).

\(\quantifs{\Tpt t\in\intervei{0}{1}}t^2>0\) et \(\cos t>0\) donc \(t^2+\cos t>0\).

Donc \(f:t\mapsto\dfrac{\ln t}{\cos t+t^2}\) est continue sur \(\intervee{0}{\pinf}\).

On a \(\dfrac{\ln t}{\cos t+t^2}\simqd{t\to0}\ln t\) et \(\ln\) est intégrable sur \(\intervei{0}{1}\) donc \(f\) aussi.

On a \(\cos t+t^2\simqd{t\to\pinf}t^2\) donc \(f\paren{t}\simqd{t\to\pinf}\dfrac{\ln t}{t^2}\egqd{t\to\pinf}\o{\dfrac{1}{t^{\nicefrac{3}{2}}}}\).

Or \(t\mapsto\dfrac{1}{t^{\nicefrac{3}{2}}}\) est intégrable sur \(\intervie{1}{\pinf}\) donc \(f\) aussi.

Finalement, \(f\) est intégrable sur \(\intervee{0}{\pinf}\).
\end{corr}

\section{Intégration des relations de comparaison}

Les résultats présentés portent sur des fonctions intégrables sur \(\intervie{a}{\pinf}\). On obtient évidemment des résultats analogues sur les autres types d'intervalles.

\subsection{Théorème de comparaison par domination}

Dans le cas convergent d'abord, les \guillemets{restes partiels} suivent la même relation de comparaison.

\begin{theo}\thlabel{theo6.4}
Soient \(f,g\) deux fonctions définies sur \(\intervie{a}{\pinf}\) avec \(g\) à valeurs dans \(\Rp\).

Si \(f=\O{g}\) et \(g\) est intégrable sur \(\intervie{a}{\pinf}\), alors \(f\) est intégrable sur \(\intervie{a}{\pinf}\).

De plus, \(\int_x^{\pinf}f=\O{\int_x^{\pinf}g}\) quand \(x\to\pinf\).

Si \(f=\o{g}\) et \(g\) est intégrable sur \(\intervie{a}{\pinf}\), alors \(f\) est intégrable sur \(\intervie{a}{\pinf}\).

De plus, \(\int_x^{\pinf}f=\o{\int_x^{\pinf}g}\) quand \(x\to\pinf\).
\end{theo}

\begin{dem}
\begin{itemize}
    \item Si \(f=\O{g}\), alors il existe \(K>0\) et \(A\geq a\) tels que \(\quantifs{\forall x\geq A}\abs{f\paren{x}}\leq Kg\paren{x}\). \\\\ Donc si \(g\) est \(\integ{1}{}{}\) alors \(\abs{f}\) est \(\integ{1}{}{}\) et donc \(f\) est \(\integ{1}{}{}\). \\\\ De plus, \[\begin{aligned}
        \quantifs{\forall x\geq A}&\quantifs{\forall t\geq x}\abs{f\paren{t}}\leq Kg\paren{t} \\
        &\text{donc }\int_x^{\pinf}\abs{f\paren{t}}\odif{t}\leq K\int_x^{\pinf}g\paren{t}\odif{t} \\
        &\text{donc }\abs{\int_x^{\pinf}f\paren{t}\odif{t}}\leq\int_x^{\pinf}\abs{f\paren{t}}\odif{t}\leq K\int_x^{\pinf}g\paren{t}\odif{t} \\
        &\text{\ie }\int_x^{\pinf}f\paren{t}\odif{t}=\O{\int_x^{\pinf}g\paren{t}\odif{t}}.
    \end{aligned}\]
    \item Idem avec un \guillemets{pour tout} devant \(K\) et un \(o\) à la fin pour le cas \(f=\o{g}\).
\end{itemize}
\end{dem}

Dans le cas divergent ensuite, les \guillemets{intégrales partielles} suivent aussi la même relation de comparaison.

\begin{theo}\thlabel{theo6.5}
Soient \(f,g\) deux fonctions définies sur \(\intervie{a}{\pinf}\) avec \(g\) à valeurs dans \(\Rp\).

Si \(f=\O{g}\) et \(f\) n'est pas intégrable sur \(\intervie{a}{\pinf}\), alors \(g\) n'est pas intégrable sur \(\intervie{a}{\pinf}\).

De plus, \(\int_a^xf=\O{\int_a^xg}\) quand \(x\to\pinf\).

Si \(f=\o{g}\) et \(f\) n'est pas intégrable sur \(\intervie{a}{\pinf}\), alors \(g\) n'est pas intégrable sur \(\intervie{a}{\pinf}\).

De plus, \(\int_a^xf=\o{\int_a^xg}\) quand \(x\to\pinf\).
\end{theo}

\begin{dem}
\begin{itemize}
    \item Si \(f=\O{g}\) alors il existe \(K>0\) et \(A\geq a\) tels que \(\quantifs{\forall x\geq A}\abs{f\paren{x}}\leq Kg\paren{x}\). \\\\ Donc si \(f\) n'est pas \(\integ{1}{}{}\), \(\int_a^{\pinf}\abs{f}\) diverge donc \(\int_a^{\pinf}g\) diverge. \\\\ Pour \(x\geq A\), pour \(t\in\intervii{A}{x}\), on a \(\int_A^x\abs{f}\leq K\int_A^xg\) donc \[\int_a^x\abs{f}=\int_a^A\abs{f}+\int_A^x\abs{f}\leq\int_a^A\abs{f}+K\int_A^xg.\] \\\\ Or \(\int_A^xg\tendqd{x\to\pinf}\pinf\). \\\\ Donc il existe \(B\geq a\) tel que \(\quantifs{\forall x\geq B}\int_a^A\abs{f}\leq K\int_A^xg\). \\\\ Pour \(x\geq\max\paren{A,B}\), on a \(\int_a^x\abs{f}\leq2K\int_A^xg\leq2K\int_a^xg\). \\\\ Donc \(\abs{\int_a^xf}\leq\int_a^x\abs{f}\leq2K\int_a^xg\). \\\\ Donc \(\int_a^xf\egqd{x\to\pinf}\O{\int_a^xg}\). \\
    \item Idem avec un \guillemets{pour tout} devant le \(K\) et un \(o\) à la fin dans le cas \(f=\o{g}\).
\end{itemize}
\end{dem}

\subsection{Théorème de comparaison par équivalence}

\begin{theo}
Soient \(f,g\) deux fonctions définies sur \(\intervie{a}{\pinf}\) avec \(g\) à valeurs dans \(\Rp\).

Si \(f\sim g\), alors l'intégrabilité de \(f\) est équivalente à l'intégrabilité de \(g\).

De plus :

\begin{itemize}
    \item si les fonctions sont intégrables, alors les restes partiels sont équivalents : \[\int_x^{\pinf}f\simqd{x\to\pinf}\int_x^{\pinf}g\]
    \item si les fonctions ne sont pas intégrables, alors les intégrales partielles divergent et sont équivalentes : \[\int_a^xf\simqd{x\to\pinf}\int_a^xg.\]
\end{itemize}
\end{theo}

\begin{dem}
On a \(f\simqd{\pinf}g\ssi f-g\egqd{\pinf}\o{g}\).

Si \(g\) est \(\integ{1}{}{}\) alors d'après le \thref{theo6.4}, on a \[\begin{aligned}
\int_x^{\pinf}\paren{f-g}&\egqd{x\to\pinf}\o{\int_x^{\pinf}g} \\
\int_x^{\pinf}f-\int_x^{\pinf}g&\egqd{x\to\pinf}\o{\int_x^{\pinf}g} \\
\int_x^{\pinf}f&\simqd{x\to\pinf}\int_x^{\pinf}g.
\end{aligned}\]

Si \(g\) n'est pas \(\integ{1}{}{}\), d'après le \thref{theo6.5}, on a \[\begin{aligned}
\int_a^x\paren{f-g}&\egqd{x\to\pinf}\o{\int_a^xg} \\
\int_a^xf-\int_a^xg&\egqd{x\to\pinf}\o{\int_a^xg} \\
\int_a^xf&\simqd{x\to\pinf}\int_a^xg.
\end{aligned}\]
\end{dem}


\chapter{Intégrales à paramètre}

\minitoc

Dans ce chapitre, \(\K\) désigne l'ensemble \(\R\) ou \(\C\). Les fonctions dans ce chapitre sont à valeurs dans \(\K\).

On considère dans ce chapitre des intégrales de la forme \(\int_a^bf\paren{p,t}\odif{t}\) où \(f\paren{p,t}\) est une expression qui dépend de deux variables \(p\) et \(t\), \(p\) pouvant être de n'importe quel type mais \(t\) bien sûr réelle. Par habitude, on distingue dans le vocabulaire ces deux variables : \(t\) est appelée la variable d'intégration (notez le \(\odif{t}\) qui le signale) et \(p\) est appelée le paramètre.

L'intégrale \(\int_a^bf\paren{p,t}\odif{t}\) est donc une intégrale qui dépend du paramètre \(p\) (mais qui ne dépend bien entendu pas de \(t\)) et l'objet de ce chapitre est d'étudier des résultats concernant cette dépendance vis-à-vis de \(p\) ; en somme, d'étudier des propriétés de l'application \(p\mapsto\int_a^bf\paren{p,t}\odif{t}\).

Les sections 2 et 3 étudient surtout le cas où \(p\) est un paramètre entier naturel, les suivantes le cas où \(p\) est un paramètre réel.

\section{Introduction}

Pour commencer, un exercice d'intervertion de symboles, qui marque le début de l'étude de ce problème général et qui va nous occuper durant quelques chapitres.

\begin{exo}
Pour \(n\in\Ns\), on pose \(f_n\paren{x}=n^2\e{-nx}\paren{1-\e{-x}}\).

Pour \(x\geq0\), que vaut \(\lim_{n\to\pinf}f_n\paren{x}\) ?

Montrez la convergence de l'intégrale \(\int_0^{\pinf}f_n\paren{x}\odif{x}\) et donnez sa valeur en fonction de \(n\).

Comparez \(\lim_{n\to\pinf}\int_0^{\pinf}f_n\paren{x}\odif{x}\) et \(\int_0^{\pinf}\paren{\lim_{n\to\pinf}f_n\paren{x}}\odif{x}\).

Quelle conclusion peut-on en tirer ?
\end{exo}

\begin{corr}
\begin{itemize}
    \item Si \(x=0\), alors \(f_n\paren{x}=0\tendqd{n\to\pinf}0\). \\\\ Si \(x>0\), alors \(0<\e{-x}<1\) donc \(\e{-nx}=\paren{\e{-x}}^n\tendqd{n\to\pinf}0\). \\\\ Donc \(f_n\paren{x}\tendqd{n\to\pinf}0\) car \(n^2\egqd{n\to\pinf}\o{\paren{\e{x}}^n}\). \\\\ Finalement, \(\quantifs{\forall x\geq0}f_n\paren{x}\tendqd{n\to\pinf}0\). \\
    \item \(f_n\) est \(\classe{0}\) sur \(\intervie{0}{\pinf}\). \\\\ Pour \(x\geq0\), \(0\leq f_n\paren{x}\leq n^2\e{-nx}\) or \(x\mapsto\e{-nx}\) est \(\integ{1}{}{}\) sur \(\intervie{0}{\pinf}\) donc \(f_n\) aussi. \\\\ On a \[\begin{aligned}
        \int_0^{\pinf}n^2\e{-nx}\paren{1-\e{-x}}\odif{x}&=n^2\paren{\int_0^{\pinf}\e{-nx}\odif{x}-\int_0^{\pinf}\e{-\paren{n+1}x}\odif{x}} \\
        &=n^2\paren{\dfrac{1}{n}-\dfrac{1}{n+1}} \\
        &=\dfrac{n}{n+1}.
    \end{aligned}\]
    \item On a \(\lim_{n\to\pinf}\int_0^{\pinf}f_n\paren{x}\odif{x}=1\) et \(\int_0^{\pinf}\paren{\lim_{n\to\pinf}f_n\paren{x}}\odif{x}=\int_0^{\pinf}0\odif{x}=0\). \\\\ Or \(1\not=0\). \\
    \item Conclusion : on ne peut pas intervertir les symboles \(\lim\) et \(\int\) \textit{a priori}.
\end{itemize}
\end{corr}

\section{Convergence simple}

\subsection{Convergence simple d'une suite de fonctions}

\begin{defi}
Soient \(A\) une partie de \(\R\) et \(\paren{f_n}\) une suite de fonctions définies sur \(A\).

On dit que la suite \(\paren{f_n}\) converge simplement sur \(A\) quand pour tout \(x\in A\), la suite numérique \(\paren{f_n\paren{x}}\) converge.

Dans ce cas, on peut définir une fonction \(f\) sur \(A\) en posant, \(\quantifs{\tpt x\in A}f\paren{x}=\lim_{n\to\pinf}f_n\paren{x}\).

La fonction \(f\) est alors appelée limite simple sur \(A\) de la suite \(\paren{f_n}\) et on dit que la suite \(\paren{f_n}\) converge simplement vers \(f\) sur \(A\).
\end{defi}

\begin{exo}
Étudiez, pour \(n\in\Ns\), la convergence simple de la suite de fonctions \(f_n:x\mapsto\dfrac{n\e{-x}+x^2}{n+x}\) sur \(\intervie{0}{\pinf}\)
\end{exo}

\begin{corr}
Les fonctions \(f_n\) sont toutes définies sur \(\intervie{0}{\pinf}\).

Pour \(x\geq0\), on a \(f_n\paren{x}=\dfrac{n\e{-x}+x^2}{n+x}\simqd{n\to\pinf}\dfrac{n\e{-x}}{n}=\e{-x}\tendqd{n\to\pinf}\e{-x}\).

Donc \(\paren{f_n}\) converge simplement vers \(x\mapsto\e{-x}\) sur \(\intervie{0}{\pinf}\).
\end{corr}

\begin{exo}
Même question avec la suite de fonctions \(f_n:x\mapsto\dfrac{x^n}{1+x^n}\) sur \(\intervie{0}{\pinf}\).
\end{exo}

\begin{corr}
Les fonctions \(f_n\) sont toutes définies sur \(\intervie{0}{\pinf}\).

Pour \(x\geq0\), on a \(f_n\paren{x}=\dfrac{x^n}{1+x^n}\).

Si \(x\in\intervie{0}{1}\), on a \(f_n\paren{x}\tendqd{n\to\pinf}0\).

Si \(x=1\), on a \(f_n\paren{x}=\dfrac{1}{2}\tendqd{n\to\pinf}\dfrac{1}{2}\).

Si \(x\in\intervee{1}{\pinf}\), on a \(f_n\paren{x}\simqd{n\to\pinf}1\tendqd{n\to\pinf}1\).

Donc \(\paren{f_n}\) converge simplement vers \(x\mapsto\begin{cases}
    0 &\text{si }0\leq x<1 \\
    \nicefrac{1}{2} &\text{si }x=1 \\
    1 &\text{sinon}
\end{cases}\) sur \(\intervie{0}{\pinf}\).
\end{corr}

\begin{exo}
Même question avec la suite de fonctions \(f_n:x\mapsto n^\alpha x^n\paren{1-x}\) où \(\alpha\) est un réel strictement positif.
\end{exo}

\begin{corr}
Si \(x=0\) ou \(x=1\), on a \(f_n\paren{x}=0\tendqd{n\to\pinf}0\).

Si \(x\in\intervee{0}{1}\), on a \(n^\alpha\egqd{n\to\pinf}\o{\paren{\dfrac{1}{x}}^n}\) donc \(n^\alpha x^n\tendqd{n\to\pinf}0\) donc \(f_n\paren{x}\tendqd{n\to\pinf}0\).

Donc la suite \(\paren{f_n}\) converge simplement vers la fonction nulle sur \(\intervii{0}{1}\).
\end{corr}

\subsection{Convergence simple d'une série de fonctions}

\begin{defi}
Soient \(A\) une partie de \(\R\) et \(\paren{f_n}\) une suite de fonctions définies sur \(A\).

On dit que la série \(\sum_{n\geq0}f_n\) converge simplement sur \(A\) quand pour tout \(x\in A\), la série numérique \(\sum_{n\geq0}f_n\paren{x}\) converge.

Autrement dit, la série de fonctions \(\sum_{n\geq0}f_n\) converge simplement sur \(A\) quand la suite des sommes partielles \(\paren{\sum_{k=0}^nf_k}\) converge simplement sur \(A\).

Dans ce cas, on peut définir une fonction \(f\) sur \(A\) en posant, \(\quantifs{\tpt x\in A}f\paren{x}=\sum_{n=0}^{\pinf}f_n\paren{x}\).

La fonction \(f\) est alors appelée (fonction) somme sur \(A\) de la série \(\sum_{n\geq0}f_n\).
\end{defi}

\begin{exo}
Étudiez, pour \(n\in\Ns\), la convergence simple de la série de fonctions \(f_n:x\mapsto\dfrac{nx^2}{n^3+x^2}\) sur \(\intervie{0}{\pinf}\).
\end{exo}

\begin{corr}
Si \(x=0\), on a \(f_n\paren{x}=0\) donc \(\sum f_n\paren{0}\) converge.

Si \(x>0\), on a \(f_n\paren{x}\simqd{n\to\pinf}\dfrac{nx^2}{n^3}=\dfrac{x^2}{n^2}\).

Donc par théorème de comparaison des séries à termes positifs, \(\sum_{n\geq1}f_n\) converge simplement sur \(\intervie{0}{\pinf}\).
\end{corr}

\begin{exo}
Même question avec la série de fonctions \(f_n:x\mapsto\dfrac{x^n}{1+x^n}\) sur \(\intervie{0}{\pinf}\).
\end{exo}

\begin{corr}
Si \(x\geq1\), la suite \(\paren{f_n\paren{x}}_n\) converge vers \(\dfrac{1}{2}\) ou \(1\).

Si \(0\leq x<1\), \(f_n\paren{x}\simqd{n\to\pinf}x^n\) or \(\sum x^n\) converge donc par théorème de comparaison des séries à termes positifs, \(\sum_{n\geq1}f_n\paren{x}\) converge.

Donc \(\sum_{n\geq1}f_n\) converge simplement sur \(\intervie{0}{1}\).
\end{corr}

\begin{exo}
Même question avec la série de fonctions \(f_n:x\mapsto\dfrac{\sin\paren{nx}}{n^3+x^3}\) sur \(\intervie{0}{\pinf}\).
\end{exo}

\begin{corr}
Pour \(x\geq0\), on a \(\abs{f_n\paren{x}}\leq\dfrac{1}{n^3}\) donc \(\sum_{n\geq1}f_n\paren{x}\) converge (absolument) sur \(\intervie{0}{\pinf}\).
\end{corr}

\section{Suites et séries de fonctions intégrables}

Dans cette section, tous les théorèmes sont admis (démonstrations très difficiles !).

\subsection{Théorème de convergence dominée}

\begin{theo}
Soient \(I\) un intervalle et \(\paren{f_n}\) une suite de fonctions continues par morceaux sur \(I\).

Si

\begin{itemize}
    \item la suite \(\paren{f_n}\) converge simplement sur \(I\) vers une fonction \(f\) qui est continue par morceaux sur \(I\) \\
    \item il existe une fonction \(\phi\), intégrable sur \(I\) et à valeurs positives, telle que \[\quantifs{\tpt n\in\N}\abs{f_n}\leq\phi\text{ sur \(I\) (hypothèse de domination)}\]
\end{itemize}

alors les fonctions \(f\) et \(f_n\) sont toutes intégrables sur \(I\) et \(\int_If_n\tendqd{n\to\pinf}\int_If\).
\end{theo}

L'hypothèse de domination est essentielle ! Il s'agit donc de trouver une fonction \(\phi\) (dont on dit qu'elle domine la suite \(\paren{f_n}\)) intégrable et, surtout, qui ne dépend pas de \(n\) !

\begin{exo}
Montrez que la suite d'intégrales \(\paren{\int_0^{\pinf}\cos\paren{t}\e{-nt^2}\odif{t}}_{n\geq1}\) est bien définie et qu'elle converge vers \(0\).
\end{exo}

\begin{corr}
On pose \(f_n:t\mapsto\cos\paren{t}\e{-nt^2}\) sur \(\intervie{0}{\pinf}\).

Pour \(n\in\Ns\), \(f_n\) est \(\classe{0}\) sur \(\intervie{0}{\pinf}\).

Si \(t=0\), \(f_n\paren{t}=1\tendqd{n\to\pinf}1\).

Sinon, \(f_n\paren{t}\tendqd{n\to\pinf}0\).

Donc \(\paren{f_n}\) converge simplement vers \(f:t\mapsto\begin{dcases}
    1 &\text{si }t=0 \\
    0 &\text{sinon}
\end{dcases}\) sur \(\intervie{0}{\pinf}\).

Pour \(n\in\Ns\) et \(t\in\intervie{0}{\pinf}\), on a \[\abs{f_n\paren{t}}\leq\e{-t^2}.\]

Or \(t\mapsto\e{-t^2}\) est \(\integ{1}{}{}\) sur \(\intervie{0}{\pinf}\) donc d'après le théorème de convergence dominée, pour tout \(n\in\Ns\), \(f_n\) est \(\integ{1}{}{}\) sur \(\intervie{0}{\pinf}\) et \[\int_0^{\pinf}f_n\tendqd{n\to\pinf}\int_0^{\pinf}f=0.\]
\end{corr}

\begin{exo}
Montrez que la suite d'intégrales \(\paren{\int_0^{\nicefrac{\pi}{2}}\sin^nt\odif{t}}\) converge vers \(0\).
\end{exo}

\begin{corr}
Pour \(n\in\N\), on pose \(f_n:t\mapsto\sin^nt\) sur \(\intervii{0}{\nicefrac{\pi}{2}}\).

\(\paren{f_n}\) est une suite de fonctions \(\classe{0}\) sur le segment \(\intervii{0}{\nicefrac{\pi}{2}}\) donc \(\integ{1}{}{}\) sur \(\intervii{0}{\nicefrac{\pi}{2}}\).

Pour \(t\in\intervii{0}{\nicefrac{\pi}{2}}\), \(f_n\paren{t}\tendqd{n\to\pinf}0\) car \(0\leq\sin^nt<1\).

Si \(t=\dfrac{\pi}{2}\), \(f_n\paren{t}\tendqd{n\to\pinf}1\).

Donc \(\paren{f_n}\) converge simplement vers \(f:t\mapsto\begin{dcases}
    0 &\text{si }t\in\intervie{0}{\nicefrac{\pi}{2}} \\
    1 &\text{sinon}
\end{dcases}\) sur \(\intervii{0}{\nicefrac{\pi}{2}}\).

Pour \(n\in\N\) et \(t\in\intervii{0}{\nicefrac{\pi}{2}}\), on a \[\abs{f_n\paren{t}}\leq1.\]

Or \(t\mapsto1\) est \(\integ{1}{}{}\) sur \(\intervii{0}{\nicefrac{\pi}{2}}\).

Donc d'après le théorème de convergence dominée, on a \[\int_0^{\nicefrac{\pi}{2}}f_n\tendqd{n\to\pinf}\int_0^{\nicefrac{\pi}{2}}f=0.\]
\end{corr}

\begin{exo}
Montrez que \(\quantifs{\tpt n\geq2}t\mapsto\dfrac{1}{1+t^n}\) est intégrable sur \(\intervie{0}{\pinf}\), puis donnez la limite des intégrales quand \(n\to\pinf\).
\end{exo}

\begin{corr}
Pour \(n\in\Ns\excluant\accol{1}\), on pose \(f_n:t\mapsto\dfrac{1}{1+t^n}\) sur \(\intervie{0}{\pinf}\).

Pour \(0\leq t<1\), \(f_n\paren{t}\tendqd{n\to\pinf}1\).

Pour \(t=1\), \(f_n\paren{t}=\dfrac{1}{2}\tendqd{n\to\pinf}\dfrac{1}{2}\).

Pour \(t>1\), \(f_n\paren{t}\tendqd{n\to\pinf}0\).

Donc \(\paren{f_n}\) converge simplement vers \(f:t\mapsto\begin{dcases}
    1 &\text{si }0\leq t<1 \\
    \nicefrac{1}{2} &\text{si }t=1 \\
    0 &\text{sinon}
\end{dcases}\) sur \(\intervie{0}{\pinf}\).

Pour \(n\geq2\) et \(t\in\intervie{0}{\pinf}\), on a \[\abs{f_n\paren{t}}\leq\begin{dcases}
    1 &\text{si }t\leq1 \\
    \nicefrac{1}{t^2} &\text{sinon}
\end{dcases}\]

Or \(t\mapsto\begin{dcases}
    1 &\text{si }t\leq1 \\
    \nicefrac{1}{t^2} &\text{sinon}
\end{dcases}\) est \(\integ{1}{}{}\) sur \(\intervie{0}{\pinf}\) donc par théorème de convergence dominée, \(\paren{f_n}_{n\geq2}\) est une suite de fonctions \(\integ{1}{}{}\) sur \(\intervie{0}{\pinf}\) et \[\int_0^{\pinf}f_n\tendqd{n\to\pinf}\int_0^{\pinf}f=1.\]
\end{corr}

Quitte à utiliser la caractérisation séquentielle de la limite, on peut étendre le théorème précédent à des fonctions paramétrées par un réel.

\begin{theo}\thlabel{theo7.2}
Soient \(I,A\) deux intervalles, \(\alpha\in\conj{A}\) et \(\paren{f_a}_{a\in A}\) une famille de fonctions continues par morceaux sur \(I\).

Si

\begin{itemize}
    \item \(\quantifs{\tpt x\in I}f_a\paren{x}\tendqd{a\to\alpha}f\paren{x}\) où \(f\) est une fonction continue par morceaux sur \(I\) \\
    \item il existe une fonction \(\phi\) intégrable sur \(I\) et à valeurs positives, telle que \[\quantifs{\tpt a\in A}\abs{f_a}\leq\phi\text{ sur \(I\) (hypothèse de domination)}\]
\end{itemize}

alors les fonctions \(f\) et \(f_a\) sont toutes intégrables sur \(I\) et \(\int_If_a\tendqd{a\to\alpha}\int_If\).
\end{theo}

\begin{dem}
Si pour \(x\in I\), \(f_a\paren{x}\tendqd{a\to\alpha}f\paren{x}\), d'après la caractérisation séquentielle de la limite, pour toute suite \(\paren{a_n}\in A^\N\) convergeant vers \(\alpha\), on a \(f_{a_n}\paren{x}\tendqd{n\to\pinf}f\paren{x}\).

La suite de fonctions \(\paren{f_{a_n}}_{n\in\N}\) converge simplement vers \(f\).

Or, pour \(x\in I\) et \(n\in\N\), \(\abs{f_{a_n}\paren{x}}\leq\phi\paren{x}\) et \(\phi\) est \(\integ{1}{}{}\) sur \(I\).

Donc d'après le théorème de convergence dominée, \(\int_If_{a_n}\tendqd{n\to\pinf}\int_If\).

Ceci est vrai pour toute suite \(\paren{a_n}\in A^\N\) qui converge vers \(\alpha\) donc d'après la caractérisation séquentielle de la limite, \(\int_If_a\tendqd{a\to\alpha}\int_If\).
\end{dem}

\subsection{Théorème d'intégration terme à terme}

\begin{theo}
Soient \(I\) un intervalle et \(\paren{f_n}\) une suite de fonctions continues par morceaux sur \(I\).

Si

\begin{itemize}
    \item la série \(\sum_{n\geq0}f_n\) converge simplement sur \(I\) vers une fonction continue par morceaux sur \(I\) \\
    \item \(\quantifs{\tpt n\in\N}f_n\) est intégrable sur \(I\) \\
    \item la série numérique \(\sum_{n\geq0}\int_I\abs{f_n}\) converge
\end{itemize}

alors la fonction somme \(\sum_{n=0}^{\pinf}f_n\) est intégrable sur \(I\) et \(\int_I\sum_{n=0}^{\pinf}f_n=\sum_{n=0}^{\pinf}\int_If_n\).
\end{theo}

L'hypothèse de convergence de la série des intégrales est essentielle, mais hélas très contraignante. Il arrive souvent qu'il soit plus facile d'utiliser le théorème de convergence dominée sur les sommes partielles de la série de fonctions.

\begin{exo}
Justifiez l'existence et calculez \(\int_0^{\pinf}\dfrac{t}{\e{t}-1}\odif{t}\).
\end{exo}

\begin{corr}
On pose \(f:t\mapsto\dfrac{t}{\e{t}-1}\) sur \(\intervee{0}{\pinf}\).

On a \(\e{t}-1\simqd{t\to0}t\) donc \(f\paren{t}\simqd{t\to0}1\tendqd{t\to0}1\) : fausse singularité, donc \(\int_0^1f\) converge.

On a \(f\paren{t}\simqd{t\to\pinf}\dfrac{t}{\e{t}}=\o{\dfrac{1}{t^2}}\) donc \(\int_1^{\pinf}f\) converge.

Donc \(\int_0^{\pinf}f\) converge.

Pour \(t>0\), \(f\paren{t}=\sum_{n=0}^{\pinf}?\)

On a \(f\paren{t}=\dfrac{t\e{-t}}{\e{-t}\paren{\e{t}-1}}=\dfrac{t\e{-t}}{1-\e{-t}}\).

Or \(0<\e{-t}<1\) donc \(f\paren{t}=t\e{-t}\sum_{n=0}^{\pinf}\paren{\e{-t}}^n=\sum_{n=0}^{\pinf}t\e{-\paren{n+1}t}\).

On pose \(f_n:t\mapsto t\e{-\paren{n+1}t}\) pour \(n\in\N\).

La série de fonctions \(\sum f_n\) converge simplement sur \(\intervee{0}{\pinf}\) et \(\sum_{n=0}^{\pinf}f_n=f\).

Donc \(\int_0^{\pinf}f=\int_0^{\pinf}\sum_{n=0}^{\pinf}f_n\).

Pour \(n\in\N\), \(f_n\) est \(\classe{0}\) sur \(\intervie{0}{\pinf}\) et \(f_n\paren{t}\egqd{t\to\pinf}\o{\dfrac{1}{t^2}}\).

Donc pour \(n\in\N\), \(f_n\) est \(\integ{1}{}{}\) sur \(\intervie{0}{\pinf}\) donc sur \(\intervee{0}{\pinf}\).

Étudions la convergence de la série \(\sum_{n\geq0}\int_0^{\pinf}\abs{f_n}\).

On a \[\int_0^{\pinf}\abs{f_n}=\int_0^{\pinf}t\e{-\paren{n+1}t}\odif{t}=\croch{\dfrac{-1}{n+1}t\e{-\paren{n+1}t}}_0^{\pinf}-\dfrac{-1}{n+1}\int_0^{\pinf}\e{-\paren{n+1}t}\odif{t}=\dfrac{1}{\paren{n+1}^2}.\]

Donc la série \(\sum_{n\geq0}\int_0^{\pinf}\abs{f_n}\) converge.

D'après le théorème d'intégration terme à terme, on a donc \[\int_0^{\pinf}f=\sum_{n=0}^{\pinf}\int_0^{\pinf}f_n=\sum_{n=0}^{\pinf}\dfrac{1}{\paren{n+1}^2}=\dfrac{\pi^2}{6}.\]
\end{corr}

\begin{rem}
Si on ajoute l'hypothèse \guillemets{les fonctions \(f_n\) sont positives} alors \(f\) est intégrable sur \(I\) ssi \(\sum\int_If_n\) converge et, dans ce cas, \(\int_If=\sum_{n=0}^{\pinf}\int_If_n\).
\end{rem}

\section{Fonctions définies par une intégrale à paramètre}

On s'intéresse aux propriétés des fonctions définies par des intégrales du type \(x\mapsto\int_If\paren{x,t}\odif{t}\). On dit que \(x\) est un paramètre de l'intégrale \(\int_If\paren{x,t}\odif{t}\).

\subsection{Continuité}

\begin{theo}[Théorème de continuité sous le signe intégrale]\thlabel{theo7.4}
Soient \(A,I\) deux intervalles de \(\R\) et \(f:\paren{x,t}\mapsto f\paren{x,t}\) une fonction définie sur \(A\times I\).

Si

\begin{itemize}
    \item \(\quantifs{\tpt x\in A}t\mapsto f\paren{x,t}\) est continue par morceaux sur \(I\) \\
    \item \(\quantifs{\tpt t\in I}x\mapsto f\paren{x,t}\) est continue sur \(A\) \\
    \item il existe une fonction \(\phi\) intégrable sur \(I\) et à valeurs positives, telle que \[\quantifs{\tpt\paren{x,t}\in A\times I}\abs{f\paren{x,t}}\leq\phi\paren{t}\text{ (hypothèse de domination)}\]
\end{itemize}

alors \(\quantifs{\tpt x\in A}t\mapsto f\paren{x,t}\) est intégrable sur \(I\) et la fonction \(x\mapsto\int_If\paren{x,t}\odif{t}\) est continue sur \(A\).
\end{theo}

\begin{dem}
Soit \(x_0\in A\).

On a \(\lim_{x\to x_0}f\paren{x,t}=f\paren{x_0,t}\) pour tout \(t\in I\).

De plus, on a \(\abs{f\paren{x,t}}\leq\phi\paren{t}\) pour tout \(t\in I\), \(x\in A\).

Donc d'après le \thref{theo7.2}, on a \[\lim_{x\to x_0}\int_If\paren{x,t}\odif{t}=\int_I\paren{\lim_{x\to x_0}f\paren{x,t}}\odif{t}=\int_If\paren{x_0,t}\odif{t}.\]

Donc \(x\mapsto\int_If\paren{x,t}\odif{t}\) est continue en \(x_0\).
\end{dem}

\begin{exo}
Montrez que la fonction \(g:x\mapsto\int_0^{\pinf}\dfrac{\cos\paren{xt^2}}{1+t^2}\odif{t}\) est définie et continue sur \(\R\).
\end{exo}

\begin{corr}
On pose \(f:\paren{x,t}\mapsto\dfrac{\cos\paren{xt^2}}{1+t^2}\) sur \(\R\times\intervie{0}{\pinf}\).

\(f\) est continue sur \(\R\times\intervie{0}{\pinf}\) donc pour tout \(x\in\R\), \(t\mapsto f\paren{x,t}\) est \(\classe{0}\) sur \(\intervie{0}{\pinf}\) et pour tout \(t\in\intervie{0}{\pinf}\), \(x\mapsto f\paren{x,t}\) est \(\classe{0}\) sur \(\R\).

Pour \(x\in\R\), \(t\in\intervie{0}{\pinf}\), on a \(\abs{f\paren{x,t}}\leq\dfrac{1}{1+t^2}\).

Or \(t\mapsto\dfrac{1}{1+t^2}\) est \(\integ{1}{}{}\) sur \(\intervie{0}{\pinf}\).

Donc d'après le théorème de continuité sous le signe intégrale, \(g\) est continue sur \(\R\) et \(t\mapsto f\paren{x,t}\) est \(\integ{1}{}{}\) sur \(\intervie{0}{\pinf}\).
\end{corr}

\begin{exo}
Montrez que la fonction \(h:u\mapsto\int_0^1\Arctan\paren{u+x\ln x}\odif{x}\) est définie et continue sur \(\R\).
\end{exo}

\begin{corr}
On pose \(f:\paren{u,x}\mapsto\Arctan\paren{u+x\ln x}\) sur \(\R\times\intervei{0}{1}\).

\(f\) est continue sur \(\R\times\intervei{0}{1}\).

Pour \(u\in\R\), \(x\in\intervei{0}{1}\), on a \(\abs{f\paren{u,x}}\leq\dfrac{\pi}{2}\).

Or \(x\mapsto\dfrac{\pi}{2}\) est \(\integ{1}{}{}\) sur \(\intervei{0}{1}\).

Donc d'après le théorème de continuité sous le signe intégrale, \(x\mapsto f\paren{u,x}\) est \(\integ{1}{}{}\) sur \(\intervei{0}{1}\) et \(h\) est continue sur \(\R\).
\end{corr}

\subsection{Dérivabilité}

\begin{theo}[Théorème de dérivation sous le signe intégrale]
Soient \(A,I\) deux intervalles de \(\R\) et \(f:\paren{x,t}\mapsto f\paren{x,t}\) une fonction définie sur \(A\times I\).

Si

\begin{itemize}
    \item \(\quantifs{\tpt x\in A}t\mapsto f\paren{x,t}\) est continue par morceaux et intégrable sur \(I\) \\
    \item \(\quantifs{\tpt t\in I}x\mapsto f\paren{x,t}\) est de classe \(\classe{1}\) sur \(A\) \\
    \item \(\quantifs{\tpt x\in A}t\mapsto\pdv{f}{x}\paren{x,t}\) est continue par morceaux sur \(I\) \\
    \item il existe une fonction \(\phi\) intégrable et à valeurs positives, telle que \[\quantifs{\tpt\paren{x,t}\in A\times I}\abs{\pdv{f}{x}\paren{x,t}}\leq\phi\paren{t}\text{ (hypothèse de domination)}\]
\end{itemize}

alors la fonction \(g:x\mapsto\int_If\paren{x,t}\odif{t}\) est de classe \(\classe{1}\) sur \(A\) et \(\quantifs{\tpt x\in A}g\prim\paren{x}=\int_I\pdv{f}{x}\paren{x,t}\odif{t}\).
\end{theo}

\begin{dem}
La fonction \(g:x\mapsto\int_If\paren{x,t}\odif{t}\) est bien définie sur \(A\) d'après la première hypothèse.

Soient \(x_0\in A\) et \(x\not=x_0\).

On veut montrer que \(g\) est dérivable en \(x_0\) et \(g\prim\paren{x_0}=\int_I\pdv{f}{x}\paren{x_0,t}\odif{t}\), \ie \[\lim_{x\to x_0}\dfrac{g\paren{x}-g\paren{x_0}}{x-x_0}=\int_I\pdv{f}{x}\paren{x_0,t}\odif{t}.\]

L'hypothèse de domination justifie la convergence de cette intégrale.

On a \(\dfrac{g\paren{x}-g\paren{x_0}}{x-x_0}=\int_I\dfrac{f\paren{x,t}-f\paren{x_0,t}}{x-x_0}\odif{t}\).

On veut intervertir \(\lim_{x\to x_0}\) et \(\int_I\).

Pour \(t\in I\), \(x\mapsto f\paren{x,t}\) est \(\classe{1}\) donc d'après le théorème des accroissements finis, il existe \(c_{x,t}\in\intervee{x_0}{x}\) tel que \[\dfrac{f\paren{x,t}-f\paren{x_0,t}}{x-x_0}=\pdv{f}{x}\paren{c_{x,t},t}.\]

Par théorème d'encadrement, pour \(t\in I\), \(\lim_{x\to x_0}c_{x,t}=x_0\).

Or \(x\mapsto\pdv{f}{x}\paren{x,t}\) est continue sur \(A\) donc par composition des limites, \[\lim_{x\to x_0}\pdv{f}{x}\paren{c_{x,t},t}=\pdv{f}{x}\paren{x_0,t}.\]

Or par hypothèse de domination, \(\abs{\pdv{f}{x}\paren{c_{x,t},t}}\leq\phi\paren{t}\).

Donc d'après le théorème de convergence dominée, on a \[\lim_{x\to x_0}\dfrac{g\paren{x}-g\paren{x_0}}{x-x_0}=\lim_{x\to x_0}\int_I\pdv{f}{x}\paren{c_{x,t},t}\odif{t}=\int_I\paren{\lim_{x\to x_0}\pdv{f}{x}\paren{c_{x,t},t}}\odif{t}=\int_I\pdv{f}{x}\paren{x_0,t}\odif{t}.\]

On a donc montré que \(g\) est dérivable sur \(A\) et \[\quantifs{\forall x\in A}g\prim\paren{x}=\int_I\pdv{f}{x}\paren{x,t}\odif{t}.\]

Avec les hypothèses \(x\mapsto\pdv{f}{x}\paren{x,t}\) continue sur \(A\) et \(\abs{\pdv{f}{x}\paren{c_{x,t},t}}\leq\phi\paren{t}\) et le \thref{theo7.4}, \(g\prim\) est en plus continue sur \(A\), \ie \(g\) est de classe \(\classe{1}\) sur \(A\).
\end{dem}

\begin{exo}
Montrez que la fonction \(g:x\mapsto\int_0^\pi\cos\paren{x\sin t}\odif{t}\) est définie et de classe \(\classe{1}\) sur \(\R\).
\end{exo}

\begin{corr}
On pose \(f:\paren{x,t}\mapsto\cos\paren{x\sin t}\) sur \(\R\times\intervii{0}{\pi}\).

\(f\) est de classe \(\classe{1}\) sur \(\R\times\intervii{0}{\pi}\) (par opérations et compositions de fonctions \(\classe{1}\)).

Pour \(x\in\R\), \(t\mapsto\cos\paren{x\sin t}\) est \(\classe{0}\) sur le segment \(\intervii{0}{\pi}\) donc y est intégrable.

On a \(\pdv{f}{x}\paren{x,t}=-\sin\paren{t}\sin\paren{x\sin t}\).

Pour \(x\in\R\), \(t\in\intervii{0}{\pi}\), on a \(\abs{\pdv{f}{x}\paren{x,t}}\leq1\).

Or \(t\mapsto1\) est \(\integ{1}{}{}\) sur \(\intervii{0}{\pi}\).

D'après le théorème de dérivation sous le signe intégrale, \(g\) est de classe \(\classe{1}\) sur \(\R\) et, pour \(x\in\R\), \(g\prim\paren{x}=-\int_0^\pi\sin\paren{t}\sin\paren{x\sin t}\odif{t}\).
\end{corr}

\begin{exo}
Montrez que la fonction \(g:x\mapsto\int_0^{\pinf}\cos\paren{xt^2}\e{-t}\odif{t}\) est définie et de classe \(\classe{1}\) sur \(\R\).
\end{exo}

\begin{corr}\thlabel{corr7.15}
On pose \(f:\paren{x,t}\mapsto\cos\paren{xt^2}\e{-t}\) sur \(\R\times\intervie{0}{\pinf}\).

\(f\) est de classe \(\classe{1}\) sur \(\R\times\intervie{0}{\pinf}\).

Pour \(x\in\R\), \(t\in\intervie{0}{\pinf}\), \(\abs{f\paren{x,t}}\leq\e{-t}\) or \(t\mapsto\e{-t}\) est \(\integ{1}{}{}\) sur \(\intervie{0}{\pinf}\) donc \(t\mapsto f\paren{x,t}\) l'est aussi.

On a \(\pdv{f}{x}\paren{x,t}=-t^2\e{-t}\sin\paren{xt^2}\).

Donc \(\abs{\pdv{f}{x}\paren{x,t}}\leq t^2\e{-t}\).

Or \(t\mapsto t^2\e{-t}\) est \(\integ{1}{}{}\) sur \(\intervie{0}{\pinf}\) (car \(t^2\e{-t}\egqd{t\to\pinf}\o{\dfrac{1}{t^{72}}}\)).

D'après le théorème de dérivation sous le signe intégrale, \(g\) est de classe \(\classe{1}\) sur \(\R\) et, pour \(x\in\R\), \(g\prim\paren{x}=-\int_0^{\pinf}t^2\e{-t}\sin\paren{xt^2}\odif{t}\).
\end{corr}

Ce théorème est généralisable pour des dérivations d'ordre plus élevé.

\begin{theo}
Soient \(A,I\) deux intervalles de \(\R\) et \(f:\paren{x,t}\mapsto f\paren{x,t}\) une fonction définie sur \(A\times I\).

Si

\begin{itemize}
    \item \(\quantifs{\tpt x\in A}t\mapsto f\paren{x,t}\) est continue par morceaux et intégrable sur \(I\) \\
    \item \(\quantifs{\tpt t\in I}x\mapsto f\paren{x,t}\) est de classe \(\classe{k}\) sur \(A\) \\
    \item \(\quantifs{\tpt x\in A;\tpt j\in\interventierii{1}{k-1}}t\mapsto\pdv[order=j]{f}{x}\paren{x,t}\) est continue par morceaux et intégrable sur \(I\) \\
    \item il existe une fonction \(\phi\) intégrable sur \(I\) et à valeurs positives, telle que \[\quantifs{\tpt\paren{x,t}\in A\times I}\abs{\pdv[order=k]{f}{x}\paren{x,t}}\leq\phi\paren{t}\text{ (hypothèse de domination)}\]
\end{itemize}

alors la fonction \(g:x\mapsto\int_If\paren{x,t}\odif{t}\) est de classe \(\classe{k}\) sur \(A\) et \(\quantifs{\tpt x\in A;\tpt j\in\interventierii{1}{k}}g\deriv{j}\paren{x}=\int_I\pdv[order=j]{f}{x}\paren{x,t}\odif{t}\).
\end{theo}

\begin{exo}
Montrez que la fonction \(g:x\mapsto\int_0^{\pinf}\cos\paren{xt^2}\e{-t}\odif{t}\) est définie et de classe \(\classe{\infty}\) sur \(\R\).
\end{exo}

\begin{corr}
On reprend les notations de la \thref{corr7.15}.

\(f\) est \(\classe{\infty}\) sur \(\R\times\intervie{0}{\pinf}\).

Pour montrer que \(g\) est \(\classe{\infty}\) sur \(\R\), il suffit de donner une domination pour toutes les dérivées partielles : pour \(k\in\N\), \(x\in\R\), \(t\in\intervie{0}{\pinf}\), \[\abs{\pdv[order=k]{f}{x}\paren{x,t}}\leq\phi_k\paren{t}\] où \(\phi_k\) est \(\integ{1}{}{}\) sur \(\intervie{0}{\pinf}\).

On a \(\e{\i x}=\cos x+\i\sin x\) donc \[\odv*[order=k]{\e{\i x}}{x}=\cos\deriv{k}x+\i\sin\deriv{k}x=\i^k\e{\i x}\] donc \[\cos\deriv{k}x=\Re\paren{\e{\i k\nicefrac{\pi}{2}}\e{\i x}}=\cos\paren{x+k\dfrac{\pi}{2}}\] donc \[\pdv[order=k]{f}{x}\paren{x,t}=t^{2k}\e{-t}\cos\paren{xt^2+k\dfrac{\pi}{2}}.\]

Donc \(\abs{\pdv[order=k]{f}{x}\paren{x,t}}\leq t^{2k}\e{-t}\) et \(t\mapsto t^{2k}\e{-t}\) est \(\integ{1}{}{}\) sur \(\intervie{0}{\pinf}\).

Donc d'après le théorème de dérivation sous le signe intégrale, \(g\) est \(\classe{\infty}\) sur \(\R\) et, pour \(x\in\R\), \(k\in\N\), \(g\deriv{k}\paren{x}=\int_0^{\pinf}t^{2k}\e{-t}\cos\paren{xt^2+k\dfrac{\pi}{2}}\odif{t}\).
\end{corr}

\section{Domination sur des sous-intervalles}

La continuité étant une propriété locale, il est souvent inutile d'avoir une domination globale sur \(A\) pour conclure. En général, on peut se contenter de domination sur des parties plus petites que \(A\), en général les segments inclus dans \(A\), ou toute famille recouvrante de parties de \(A\).

\begin{defi}
Soit \(A\) un intervalle.

Une famille \(\fami{F}\) de parties de \(A\) est dite recouvrante quand sa réunion est \(A\) : \(A=\bigunion_{X\in\fami{F}}X\).
\end{defi}

\begin{prop}
Soit \(A\) un intervalle.

La famille des segments inclus dans \(A\) est recouvrante : \(A=\bigunion_{\paren{a,b}\in A^2}\intervii{a}{b}\).
\end{prop}

\begin{dem}
Tout intervalle est convexe.

Si \(x\in A\), il existe \(\paren{a,b}\in A^2\) tel que \(a\leq x\leq b\).

Donc \(x\in\intervii{a}{b}\).

Donc \(x\in\bigunion_{\paren{a,b}\in A^2}\intervii{a}{b}\).

Donc \(A\subset\bigunion_{\paren{a,b}\in A^2}\intervii{a}{b}\subset A\).
\end{dem}

On en déduit alors le théorème suivant, dont il vaut mieux à mon avis, sur chaque exercice, présenter le détail des idées.

\begin{theo}
Soient \(A,I\) deux intervalles de \(\R\) et \(f:\paren{x,t}\mapsto f\paren{x,t}\) une fonction définie sur \(A\times I\).

Soit \(\fami{F}\) une famille recouvrante de parties de \(A\).

Si

\begin{itemize}
    \item \(\quantifs{\tpt x\in A}t\mapsto f\paren{x,t}\) est continue par morceaux sur \(I\) \\
    \item \(\quantifs{\tpt t\in I}x\mapsto f\paren{x,t}\) est continue sur \(A\) \\
    \item pour toute partie \(F\) de \(\fami{F}\), il existe une fonction \(\phi_F\) intégrable sur \(I\) et à valeurs positives, telle que \[\quantifs{\tpt\paren{x,t}\in F\times I}\abs{f\paren{x,t}}\leq\phi_F\paren{t}\text{ (hypothèse de domination)}\]
\end{itemize}

alors \(\quantifs{\tpt x\in A}t\mapsto f\paren{x,t}\) est intégrable sur \(I\) et la fonction \(x\mapsto\int_If\paren{x,t}\odif{t}\) est continue sur \(A\).
\end{theo}

On a de même une version locale des théorèmes de dérivation sous le signe intégrale.

Autrement dit, au lieu de chercher à appliquer les théorèmes précédents directement sur \(A\), on trouve une famille recouvrante de sous-intervalles sur chacun desquels on peut appliquer les théorèmes précédents, conclure à la continuité ou dérivabilité sur chaque sous-intervalle, puis signaler que par réunion, la propriété reste valable sur \(A\).

\begin{exo}
Montrez que la fonction \(f:x\mapsto\int_0^{\pinf}\dfrac{\e{-xt}}{\sqrt{t}}\odif{t}\) est définie et continue sur \(\intervee{0}{\pinf}\).
\end{exo}

\begin{corr}
On pose \(\phi:\paren{x,t}\mapsto\dfrac{\e{-xt}}{\sqrt{t}}\) sur \(\intervee{0}{\pinf}\times\intervee{0}{\pinf}\).

\(\phi\) est continue sur \(\intervee{0}{\pinf}^2\).

Pour \(t\in\intervee{0}{\pinf}\), \(x\in\intervee{0}{\pinf}\), \(\abs{\phi\paren{x,t}}\leq\dfrac{1}{\sqrt{t}}\) est la meilleure majoration possible valable pour \(x>0\).

Soit \(a>0\).

Pour \(t>0\), \(x\in\intervie{a}{\pinf}\), \(\abs{\phi\paren{x,t}}\leq\dfrac{\e{-at}}{\sqrt{t}}\).

Or \(\dfrac{\e{-at}}{\sqrt{t}}\simqd{t\to0}\dfrac{1}{t^{\nicefrac{1}{2}}}\) donc \(t\mapsto\dfrac{\e{-at}}{\sqrt{t}}\) est \(\integ{1}{}{}\) sur \(\intervei{0}{1}\).

De plus, \(\dfrac{\e{-at}}{\sqrt{t}}\egqd{t\to\pinf}\o{\dfrac{1}{t^2}}\) donc \(t\mapsto\dfrac{\e{-at}}{\sqrt{t}}\) est \(\integ{1}{}{}\) sur \(\intervie{1}{\pinf}\).

D'après le théorème de continuité sous le signe intégrale, \(f\) est continue sur \(\intervie{a}{\pinf}\).

Donc \(f\) est continue sur \(\intervee{0}{\pinf}=\bigunion_{a>0}\intervie{a}{\pinf}\).
\end{corr}

\begin{exo}
Montrez que la fonction \(f:x\mapsto\int_1^{\pinf}\dfrac{\ln\paren{1+xt}}{t^2}\odif{t}\) est définie et continue sur \(\intervie{0}{\pinf}\) et qu'elle est de classe \(\classe{1}\) sur \(\intervee{0}{\pinf}\). Donnez une expression simple de \(f\prim\paren{x}\) pour tout \(x>0\).
\end{exo}

\begin{corr}
On pose \(\phi:\paren{x,t}\mapsto\dfrac{\ln\paren{1+xt}}{t^2}\) sur \(\intervie{0}{\pinf}\times\intervie{1}{\pinf}\).

\(\phi\) est de classe \(\classe{\infty}\) sur \(\intervie{0}{\pinf}\times\intervie{1}{\pinf}\).

Soit \(a>0\).

Pour \(x\in\intervii{0}{a}\), \(t\in\intervie{1}{\pinf}\), \(\abs{\dfrac{\ln\paren{1+xt}}{t^2}}\leq\dfrac{\ln\paren{1+at}}{t^2}\).

Or \(t\mapsto\dfrac{\ln\paren{1+at}}{t^2}\) est continue sur \(\intervie{1}{\pinf}\) et \(\dfrac{\ln\paren{1+at}}{t^2}\simqd{t\to\pinf}\dfrac{\ln t}{t^2}\egqd{t\to\pinf}\o{\dfrac{1}{t^{\nicefrac{3}{2}}}}\).

Donc \(t\mapsto\dfrac{\ln\paren{1+at}}{t^2}\) est \(\integ{1}{}{}\) sur \(\intervie{1}{\pinf}\).

Donc \(f\) est continue sur \(\intervii{0}{a}\) et donc sur \(\intervie{0}{\pinf}=\bigunion_{a>0}\intervii{0}{a}\).

On a \(\pdv{\phi}{x}\paren{x,t}=\dfrac{t}{t^2\paren{1+xt}}=\dfrac{1}{t\paren{1+xt}}\).

Soit \(a>0\).

Pour \(x\in\intervie{a}{\pinf}\), \(t\in\intervie{1}{\pinf}\), \(\abs{\pdv{\phi}{x}\paren{x,t}}\leq\dfrac{1}{at^2}\).

Or \(t\mapsto\dfrac{1}{at^2}\) est \(\integ{1}{}{}\) sur \(\intervie{1}{\pinf}\).

Donc \(f\) est \(\classe{1}\) sur \(\intervie{a}{\pinf}\) et donc sur \(\intervee{0}{\pinf}=\bigunion_{a>0}\intervie{a}{\pinf}\).

Pour \(x>0\), calculons \(f\prim\paren{x}=\int_1^{\pinf}\dfrac{1}{t\paren{1+xt}}\odif{t}\).

On a \(\dfrac{1}{t\paren{1+xt}}=\dfrac{1}{t}-\dfrac{x}{1+xt}\).

Donc \[\begin{aligned}
\int_1^X\dfrac{1}{t\paren{1+xt}}\odif{t}&=\int_1^X\dfrac{1}{t}\odif{t}-x\int_1^X\dfrac{1}{1+xt}\odif{t} \\
&=\ln X-\ln\paren{1+xX}+\ln\paren{1+x} \\
&=\ln X-\ln X-\ln\paren{x+\dfrac{1}{x}}+\ln\paren{1+x} \\
&=\ln\dfrac{1+x}{x}.
\end{aligned}\]

Donc \[f\prim\paren{x}=\int_1^{\pinf}\dfrac{1}{t\paren{1+xt}}\odif{t}=\lim_{X\to\pinf}\dfrac{1+x}{x}=\dfrac{1+x}{x}.\]
\end{corr}

\section{Complément : la fonction \(\Gamma\) d'Euler}

Pour \(x\in\R\), on pose, quand cela a un sens \[\Gamma\paren{x}=\int_0^{\pinf}t^{x-1}\e{-t}\odif{t}.\]

Cette fonction très courante a les propriétés suivantes :

\begin{itemize}
    \item \(\Gamma\) est définie sur \(\intervee{0}{\pinf}\) \\
    \item \(\Gamma\) est de classe \(\classe{\infty}\) sur \(\intervee{0}{\pinf}\) \\
    \item \(\quantifs{\tpt x>0}\Gamma\paren{x+1}=x\Gamma\paren{x}\) \\
    \item \(\quantifs{\tpt n\in\Ns}\Gamma\paren{n}=\paren{n-1}!\) \\
    \item \(\Gamma\paren{\dfrac{1}{2}}=\sqrt{\pi}\) \\
    \item il existe un unique \(\alpha\in\intervee{1}{2}\) tel que \(\Gamma\prim\paren{\alpha}=0\) et \(\Gamma\) est strictement décroissante sur \(\intervei{0}{\alpha}\) et strictement croissante sur \(\intervee{\alpha}{\pinf}\) \\
    \item \(\Gamma\) est convexe sur \(\intervee{0}{\pinf}\) \\
    \item \(\Gamma\) a des limites infinies en \(0\) et en \(\pinf\).
\end{itemize}

\begin{dem}
On pose \(f:\paren{x,t}\mapsto\e{\paren{x-1}\ln t}\e{-t}\) sur \(\intervee{0}{\pinf}\times\intervee{0}{\pinf}\).

\(f\) est \(\classe{\infty}\) sur \(\intervee{0}{\pinf}^2\).

Soient \(b>a>0\).

Pour \(n\in\N\), \(x\in\intervii{a}{b}\), \(t\in\intervee{0}{\pinf}\), on a \[\pdv[order=n]{f}{x}\paren{x,t}=\ln^n\paren{t}\e{\paren{x-1}\ln t}\e{-t}=\ln^n\paren{t}\dfrac{1}{t^{1-x}}\e{-t}.\]

Donc \(\abs{\pdv[order=n]{f}{x}\paren{x,t}}=\abs{\ln^nt}\e{\paren{x-1}\ln t}\e{-t}\).

On a \(a-1\leq x-1\leq b-1\).

Donc \(\begin{dcases}
\paren{x-1}\ln t\leq\paren{b-1}\ln t &\text{si }t\geq1 \\
\paren{x-1}\ln t\leq\paren{a-1}\ln t &\text{sinon}
\end{dcases}\)

Donc \(\abs{\pdv[order=n]{f}{x}\paren{x,t}}\leq\phi\paren{t}\) où \(\phi:t\mapsto\begin{dcases}
    \abs{\ln^nt}t^{b-1}\e{-t} &\text{si }t\geq1 \\
    \abs{\ln^nt}t^{a-1}\e{-t} &\text{sinon}
\end{dcases}\)

Quand \(t\to\pinf\), \(\abs{\ln^nt}t^{b-1}\e{-t}=\o{\dfrac{1}{t^2}}\) donc \(\phi\) est \(\integ{1}{}{}\) sur \(\intervie{1}{\pinf}\).

Quand \(t\to0\), \(\phi\paren{t}\sim\dfrac{\abs{\ln^nt}}{t^{1-a}}=\o{\dfrac{1}{t^\alpha}}\) où \(1-a<\alpha<1\) donc \(\phi\) est \(\integ{1}{}{}\) sur \(\intervei{0}{1}\).

Donc \(\phi\) est \(\integ{1}{}{}\) sur \(\intervee{0}{\pinf}\).

Donc \(\Gamma\) est \(\classe{\infty}\) sur \(\intervii{a}{b}\).

Donc \(\Gamma\) est \(\classe{\infty}\) sur \(\intervee{0}{\pinf}=\bigunion_{0<a<b}\intervii{a}{b}\).
\end{dem}


\chapter{Espaces préhilbertiens réels}

\minitoc

\note{À venir}


\chapter{Endomorphismes dans un espace euclidien}

\minitoc

Dans tout ce chapitre, \(E\) désigne un espace euclidien de dimension \(n\), muni du produit scalaire \(\ps{}{}\).

\section{Adjoint d'un endomorphisme}

\subsection{Représentation des formes linéaires}

Le théorème suivant est parfois appelé théorème de représentation de Riesz.

\begin{prop}\thlabel{prop9.1}
Soit \(\phi\) une forme linéaire sur \(E\).

Il existe un unique vecteur \(v\in E\) tel que \(\quantifs{\tpt x\in E}\phi\paren{x}=\ps{v}{x}\).
\end{prop}

\begin{dem}
On choisit \(\fami{B}=\paren{e_1,\dots,e_n}\) une base orthonormée de \(E\).

Soit \(x\in E\) tel que \(x\paren{x_1,\dots,x_n}_{\fami{B}}\).

On a \(\phi\paren{x}=\phi\paren{\sum_{i=1}^nx_ie_i}=\sum_{i=1}^nx_i\phi\paren{e_i}\).

On pose alors \(v\paren{\phi\paren{e_1},\dots,\phi\paren{e_n}}_{\fami{B}}\) de sorte que \(\phi\paren{x}=\ps{v}{x}\).

On a unicité par unicité des coordonnées d'un vecteur.
\end{dem}

\subsection{Adjoint}

\begin{lem}\thlabel{lem9.1}
Pour \(a,b\in E\), on a \[a=b\ssi\quantifs{\forall x\in E}\ps{a}{x}=\ps{b}{x}.\]
\end{lem}

\begin{dem}
\imprec

Pour \(x\in E\), on a \(\ps{a-b}{x}=0\) donc \(\ps{a-b}{a-b}=0\) donc \(a-b=0\) donc \(a=b\).

\impdir

Clair.
\end{dem}

\begin{prop}
Soit \(f\in\Lendo{E}\).

Il existe un unique endomorphisme \(g\in\Lendo{E}\) tel que \(\quantifs{\tpt\paren{x,y}\in E^2}\ps{f\paren{x}}{y}=\ps{x}{g\paren{y}}\).
\end{prop}

\begin{dem}
Pour tout \(y\in E\), l'application \(\fonctionlambda{E}{\R}{x}{\ps{f\paren{x}}{y}}\) est linéaire.

D'après la \thref{prop9.1}, il existe un unique vecteur \(g\paren{y}\) tel que \[\quantifs{\forall x\in E}\ps{f\paren{x}}{y}=\ps{g\paren{y}}{x}.\]

On a donc construit une application \(g:E\to E\).

Soient \(\paren{y,z}\in E^2\) et \(\lambda\in\R\).

Pour tout \(x\in E\), on a \[\begin{aligned}
\ps{g\paren{\lambda y+z}}{x}&=\ps{f\paren{x}}{\lambda y+z} \\
&=\lambda\ps{f\paren{x}}{y}+\ps{f\paren{x}}{z} \\
&=\lambda\ps{g\paren{y}}{x}+\ps{g\paren{z}}{x} \\
&=\ps{\lambda g\paren{y}+g\paren{z}}{x}.
\end{aligned}\]

On en déduit \(g\paren{\lambda y+z}=\lambda g\paren{y}+g\paren{z}\) d'après le \thref{lem9.1}.
\end{dem}

\begin{defi}
L'endomorphisme \(g\) est appelé l'adjoint de \(f\) est est noté \(f\adj\).
\end{defi}

Par bilinéarité et symétrie du produit scalaire, on en déduit les propriétés élémentaires de l'adjonction.

\begin{prop}
\begin{enumerate}
    \item L'application \(f\mapsto f\adj\) est linéaire. \\
    \item \(\quantifs{\Tpt f\in\Lendo{E}}f\adjadj=f\). \\
    \item \(\quantifs{\Tpt\paren{f,g}\in\Lendo{E}^2}\paren{f\rond g}\adj=g\adj\rond f\adj\).
\end{enumerate}
\end{prop}

\begin{dem}[2]
Pour \(\paren{x,y}\in E^2\), on a \[\ps{f\paren{x}}{y}=\ps{x}{f\adj\paren{y}}=\ps{f\adjadj\paren{x}}{y}.\]

D'après le \thref{lem9.1}, on a \(f\paren{x}=f\adjadj\paren{x}\).

Donc \(f=f\adjadj\).
\end{dem}

\begin{dem}[3]
Pour tout \(\paren{x,y}\in E^2\), on a \[\begin{aligned}
\ps{f\rond g\paren{x}}{y}&=\ps{f\paren{g\paren{x}}}{y} \\
&=\ps{g\paren{x}}{f\adj\paren{y}} \\
&=\ps{x}{g\adj\paren{f\adj\paren{x}}} \\
&=\ps{x}{g\adj\rond f\adj\paren{y}}.
\end{aligned}\]

Par unicité de l'adjoint, \(\paren{f\rond g}\adj=g\adj\rond f\adj\).
\end{dem}

\begin{dem}[1]
Soient \(\paren{f,g}\in\Lendo{E}^2\) et \(\lambda\in\R\).

Pour tout \(\paren{x,y}\in E^2\), on a \[\begin{aligned}
\ps{\paren{\lambda f+g}\paren{x}}{y}&=\ps{\lambda f\paren{x}+g\paren{x}}{y} \\
&=\lambda\ps{f\paren{x}}{y}+\ps{g\paren{x}}{y} \\
&=\lambda\ps{x}{f\adj\paren{y}}+\ps{x}{g\adj\paren{y}} \\
&=\ps{x}{\lambda f\adj\paren{y}+g\adj\paren{y}} \\
&=\ps{x}{\paren{\lambda f\adj+g\adj}\paren{y}}.
\end{aligned}\]

Par unicité de l'adjoint, \(\paren{\lambda f+g}\adj=\lambda f\adj+g\adj\).
\end{dem}

\begin{exo}\thlabel{exo9.1}
Montrez que si \(f\) est un projecteur orthogonal, alors \(f\adj=f\).
\end{exo}

\begin{corr}
Soit \(f\) un projecteur orthogonal.

Comme \(E\) est de dimension finie, on a \(E=\Im f\operp\ker f\).

On choisit une base orthonormée de \(\ker f\) et de \(\Im f\) : en les concaténant, on obtient une base orthonormée de \(E\) \[\fami{B}=\paren{e_1,\dots,e_r,e_{r+1},\dots,e_n}.\]

Soit \(\paren{x,y}\in E^2\). On a \[f\paren{x}=\sum_{i=1}^r\ps{x}{e_i}e_i\qquad\text{et}\qquad f\paren{y}=\sum_{i=1}^r\ps{y}{e_i}e_i.\]

Alors \[\begin{aligned}
\ps{f\paren{x}}{y}&=\ps{\sum_{i=1}^r\ps{x}{e_i}e_i}{y} \\
&=\sum_{i=1}^r\ps{x}{e_i}\ps{e_i}{y} \\
&=\ps{x}{\sum_{i=1}^r\ps{e_i}{y}e_i} \\
&=\ps{x}{f\paren{y}}.
\end{aligned}\]

Donc \(f=f\adj\).
\end{corr}

\begin{exo}
Premièrement,

Soit \(f\in\Lendo{E}\).

Montrez que \(\Im f\adj=\paren{\ker f}\ortho\) et \(\ker f\adj=\paren{\Im f}\ortho\).

Comparez \(\rg f\) et \(\rg f\adj\).
\end{exo}

\begin{corr}
Soit \(y\in\Im f\adj\). Il existe \(a\in E\) tel que \(f\adj\paren{a}=y\).

Alors pour \(x\in\ker f\), on a \[\ps{x}{y}=\ps{x}{f\adj\paren{a}}=\ps{a}{f\paren{x}}=0.\]

Donc \(x\perp y\) et donc \(y\in\paren{\ker f}\ortho\).

Donc \(\Im f\adj\subset\paren{\ker f}\ortho\).

Deuxièmement,

Soient \(x\in\ker f\adj\) et \(y\in\Im f\).

Il existe \(a\in E\) tel que \(f\paren{a}=y\).

Donc \[\ps{x}{y}=\ps{x}{f\paren{a}}=\ps{f\adj\paren{x}}{a}=0.\]

Donc \(x\perp y\), donc \(x\in\paren{\Im f}\ortho\) et donc \(\ker f\adj\subset\paren{\Im f}\ortho\).

De plus, on a \[\dim\Im f\adj\leq\dim\paren{\ker f}\ortho=n-\dim\ker f=\dim\Im f\] et \[\dim\ker f\adj\leq\dim\paren{\Im f}\ortho=n-\dim\Im f=\dim\ker f.\]

Or \(\dim\ker f\adj=n-\dim\Im f\adj\leq n-\dim\Im f\) donc \[\dim\Im f\adj\geq\dim\Im f.\]

D'où \[\dim\Im f\adj=\dim\Im f=n-\dim\ker f=\dim\paren{\ker f}\ortho\] et \[\dim\ker f\adj=\dim\ker f=n-\dim\Im f=\dim\paren{\Im f}\ortho.\]

Donc on a \[\Im f\adj=\paren{\ker f}\ortho\] et \[\ker f\adj=\paren{\Im f}\ortho\] et \[\rg f=\rg f\adj.\]
\end{corr}

\begin{exo}
Soit \(f\in\Lendo{E}\).

Montrez que \(\rg f=\rg\paren{f\adj\rond f}\).
\end{exo}

\begin{corr}
Pour tout \(\paren{f,g}\in\Lendo{E}^2\), on a \[\begin{dcases}
\rg\paren{g\rond f}\leq\rg f &\text{(1)} \\
\rg\paren{f\rond g}\leq\rg f &\text{(2)}
\end{dcases}\] En effet, on a \(\Im\paren{f\rond g}\subset\Im f\) donc (2) et on a \(\ker f\subset\ker\paren{f\rond g}\) et le théorème du rang donc (1).

Pour avoir \(\rg\paren{f\adj\rond f}=\rg f\), il suffit donc de montrer que \(\ker f=\ker\paren{f\adj\rond f}\).

On a clairement \(\ker f\subset\ker\paren{f\adj\rond f}\).

Soit \(x\in\ker\paren{f\adj\rond f}\).

On a \(f\adj\rond f\paren{x}=0\) donc \(\ps{f\adj\rond f\paren{x}}{x}=0\) donc \(\ps{f\paren{x}}{f\paren{x}}=0\) donc \(\norme{f\paren{x}}=0\) donc \(f\paren{x}=0\).

Donc \(x\in\ker f\) et donc \(\ker\paren{f\adj\rond f}\subset\ker f\).
\end{corr}

\subsection{Matrice de l'adjoint}

\begin{prop}
Soient \(\fami{B}\) une base orthonormée de \(E\) et \(\paren{f,g}\in\Lendo{E}^2\).

On a \(g=f\adj\ssi\Mat{g}=\trans{\paren{\Mat{f}}}\).
\end{prop}

\begin{dem}
Soit \(\fami{B}\) une base orthonormée de \(E\).

Soient \(X=\tcoords{x_1}{\vdots}{x_n}\) et \(Y=\tcoords{y_1}{\vdots}{y_n}\), et \(x,y\) de coordonnées \(X,Y\) dans \(\fami{B}\).

On pose \(A=\Mat{f}\).

\(f\paren{x}\) a pour coordonnées \(AX\) dans \(\fami{B}\).

Donc \[\begin{aligned}
\ps{f\paren{x}}{y}&=\trans{\paren{AX}}Y \\
&=\trans{X}\trans{A}Y \\
&=\ps{x}{f\adj\paren{y}} \\
&=\trans{X}\paren{BY}
\end{aligned}\] où \(B=\Mat{f\adj}\).

Donc \(\quantifs{\forall\paren{X,Y}\in\M{n,1}[\R]^2}\trans{X}\trans{A}Y=\trans{X}BY\).

Donc \(B=\trans{A}\).

Et réciproquement.
\end{dem}

\begin{rem}
Attention, ceci n'est valable qu'en base orthonormée. En base quelconque, c'est plus compliqué.
\end{rem}

\begin{exo}
Soit \(f\in\Lendo{E}\) diagonalisable.

Montrez l'équivalence \[f\adj=f^2\ssi f\text{ est un projecteur orthogonal}.\]
\end{exo}

\begin{corr}
\imprec Si \(f\) est un projecteur orthogonal, alors \(f=f\adj\) (\cf \thref{exo9.1}) et \(f=f^2\) donc \(f\adj=f^2\).

\impdir

En base orthonormée, la matrice \(A\) de \(f\) vérifie \(\trans{A}=A^2\).

Donc \(A=\trans{\paren{A^2}}=\paren{\trans{A}}^2=\paren{A^2}^2=A^4\).

Donc \(X^4-X\) est un polynôme annulateur de \(A\).

Donc \(\mu_A\), polynôme minimal de \(A\), divise \(X^4-X\) et est scindé à racines simples dans \(\poly[\R]\).

Or \(X^4-X=X\paren{X-1}\paren{X^2+X+1}\).

Donc \(\mu_A=X\) ou \(\mu_A=X-1\) ou \(\mu_A=X\paren{X-1}\) \ie \(f=0\) ou \(f=\id{E}\) ou \(f^2=f\).

Dans tous les cas, \(f\) est projecteur.
\end{corr}

\subsection{Stabilité de sous-espaces vectoriels}

Une propriété remarquable et utile pour la suite du cours.

\begin{prop}\thlabel{prop9.5}
Soient \(f\in\Lendo{E}\) et \(F\) un sous-espace vectoriel de \(E\).

Si \(F\) est stable par \(f\), alors \(F\ortho\) est stable par \(f\adj\).
\end{prop}

\begin{dem}
On veut montrer que \(\quantifs{\tpt x\in F\ortho}f\adj\paren{x}\in F\ortho\).

Soient \(x\in F\ortho\) et \(y\in F\).

On a \(\ps{f\adj\paren{x}}{y}=\ps{x}{f\paren{y}}\).

Or \(y\in F\) et \(F\) est stable par \(f\) donc \(f\paren{y}\in F\).

Or \(x\in F\ortho\) donc \(\ps{x}{f\paren{y}}=0\).

Donc \(f\adj\paren{x}\perp y\) donc \(f\adj\paren{x}\in F\ortho\).
\end{dem}

\section{Orientation d'un \(\R\)-espace vectoriel de dimension finie}

Soit \(E\) un \(\R\)-espace vectoriel de dimension finie \(n\geq1\).

\begin{defi}
On dit que deux bases \(\fami{B},\fami{B}\prim\) de \(E\) ont la même orientation quand \(\detb\fami{B}\prim>0\), sinon on dit qu'elles sont d'orientations contraires.
\end{defi}

Orienter \(E\), c'est choisir une base de référence et déclarer directes toutes les bases qui ont la même orientation que cette base de référence. Les bases de l'autre classe d'équivalence sont dites indirectes (ou rétrogrades).

En géométrie classique, dans le plan ou l'espace, on convient systématiquement d'une orientation.

Dans toute la suite, \(E\) désigne un espace euclidien de dimension \(n\). On suppose aussi que \(E\) est orienté.

\section{Isométries vectorielles}

\begin{defi}
On appelle isométrie vectorielle (ou automorphisme orthogonal) tout endomorphisme de \(E\) qui conserve la norme : \(\quantifs{\tpt x\in E}\norme{f\paren{x}}=\norme{x}\).
\end{defi}

\begin{rem}
L'appellation \guillemets{automorphisme} n'est pas usurpée.
\end{rem}

\begin{dem}
Si \(f\) est une isométrie vectorielle, alors \(\ker f=\accol{0}\) car si \(f\paren{x}=0\), alors \(\norme{f\paren{x}}=0\) donc \(\norme{x}=0\) donc \(x=0\).

Comme \(E\) est de dimension finie, \(f\) est un automorphisme.
\end{dem}

L'ensemble des isométries vectorielles de \(E\) est noté \(\Orth{}[E]\).

\begin{prop}
\(\Orth{}[E]\) est un sous-groupe de \(\groupe{\GL{}[E]}[\rond]\).
\end{prop}

\begin{dem}
On vient de montrer que \(\Orth{}[E]\subset\GL{}[E]\).

De plus, \(\id{E}\in\Orth{}[E]\).

Enfin, pour \(\paren{f,g}\in\Orth{}[E]\) et \(x\in E\), on a \[\begin{WithArrows}
\norme{g\rond f\paren{x}}&=\norme{g\paren{f\paren{x}}} \Arrow{\(g\) conserve la norme} \\
&=\norme{f\paren{x}} \Arrow{\(f\) conserve la norme} \\
&=\norme{x}.
\end{WithArrows}\]

Donc \(g\rond f\in\Orth{}[E]\).

De plus, pour \(f\in\Orth{}[E]\), on a \[\norme{f\inv\paren{x}}=\norme{f\paren{f\inv\paren{x}}}=\norme{x}\] donc \(f\inv\in\Orth{}[E]\).

Donc \(\Orth{}[E]\) est un sous-groupe de \(\groupe{\GL{}[E]}[\rond]\).
\end{dem}

Les symétries orthogonales sont des isométries vectorielles. Parmi celles-ci, on distingue les symétries orthogonales par rapport à un hyperplan : on les appelle les réflexions.

On peut caractériser les isométries vectorielles de diverses façons.

\begin{prop}
Soit \(f\in\Lendo{E}\). Les propositions suivantes sont équivalentes :

\begin{enumerate}
    \item \(f\) est une isométrie vectorielle \\
    \item \(f\) conserve le produit scalaire : \(\quantifs{\tpt\paren{x,y}\in E^2}\ps{f\paren{x}}{f\paren{y}}=\ps{x}{y}\) \\
    \item \(f\) transforme toute base orthonormée en base orthonormée \\
    \item \(f\) est un automorphisme et \(f\adj=f\inv\), ou, ce qui revient au même : \(f\adj\rond f=\id{E}\).
\end{enumerate}
\end{prop}

\begin{dem}[(1) \(\imp\) (2)]
Soient \(f\in\Orth{}[E]\) et \(\paren{x,y}\in E^2\).

On a, d'après Al-Kashi \[\begin{WithArrows}
\ps{f\paren{x}}{f\paren{y}}&=\dfrac{\norme{f\paren{x}+f\paren{y}}^2-\norme{f\paren{x}}^2-\norme{f\paren{y}}^2}{2} \Arrow{\(f\) conserve la norme} \\
&=\dfrac{\norme{x+y}^2-\norme{x}^2-\norme{y}^2}{2} \Arrow{Al-Kashi} \\
&=\ps{x}{y}.
\end{WithArrows}\]
\end{dem}

\begin{dem}[(2) \(\imp\) (1)]
On a \(\quantifs{\forall\paren{x,y}\in E^2}\ps{f\paren{x}}{f\paren{y}}=\ps{x}{y}\).

Donc \(\quantifs{\forall x\in E}\ps{f\paren{x}}{f\paren{x}}=\ps{x}{x}\).

Donc \(\quantifs{\forall x\in E}\norme{f\paren{x}}^2=\norme{x}^2\).

Donc \(\quantifs{\forall x\in E}\norme{f\paren{x}}=\norme{x}\).

Donc \(f\) est une isométrie vectorielle.
\end{dem}

\begin{dem}[(1) \(\imp\) (3)]
Soient \(f\in\Orth{}[E]\) et \(\paren{e_1,\dots,e_n}\) une base orthonormée de \(E\).

\(f\) conserve la norme donc \(\quantifs{\tpt i\in\interventierii{1}{n}}\norme{f\paren{e_i}}=\norme{e_i}=1\).

\(f\) conserve le produit scalaire donc pour tout \(\paren{i,j}\in\interventierii{1}{n}^2\), si \(i\not=j\), alors \(\ps{f\paren{e_i}}{f\paren{e_j}}=\ps{e_i}{e_j}=0\).

Donc \(\paren{f\paren{e_1},\dots,f\paren{e_n}}\) est une famille orthonormée de \(E\), espace de dimension \(n\), donc une base orthonormée de \(E\).
\end{dem}

\begin{dem}[(3) \(\imp\) (1)]
Soient \(f\in\Lendo{E}\) et \(\fami{B}=\paren{e_1,\dots,e_n}\) une base orthonormée de \(E\) telle que \(\fami{B}\prim=\paren{f\paren{e_1},\dots,f\paren{e_n}}\) soit aussi une base orthonormée de \(E\).

Soit \(x\tcoords{x_1}{\vdots}{x_n}_{\fami{B}}\). On a \(\norme{x}=\sqrt{\sum_{i=1}^nx_i^2}\) car \(\fami{B}\) est orthonormée.

Or \(x=\sum_{i=1}^nx_ie_i\) donc \(f\paren{x}=\sum_{i=1}^nx_if\paren{e_i}\).

Donc \(f\paren{x}\tcoords{x_1}{\vdots}{x_n}_{\fami{B}\prim}\).

Or \(\fami{B}\prim\) est orthonormée donc \[\norme{f\paren{x}}=\sqrt{\sum_{i=1}^nx_i^2}=\norme{x}.\]

Donc \(f\in\Orth{}[E]\).
\end{dem}

\begin{dem}[(1) \(\ssi\) (4)]
On a \[\begin{WithArrows}
f\in\Orth{}[E]&\ssi\quantifs{\forall\paren{x,y}\in E^2}\ps{f\paren{x}}{f\paren{y}}=\ps{x}{y} \\
&\ssi\quantifs{\forall\paren{x,y}\in E^2}\ps{x}{f\adj\rond f\paren{y}}=\ps{x}{y} \Arrow{\thref{lem9.1}} \\
&\ssi\quantifs{\forall y\in E}f\adj\rond f\paren{y}=y \\
&\ssi f\adj\rond f=\id{E} \\
&\ssi f\adj=f\inv.
\end{WithArrows}\]
\end{dem}

\begin{exo}
Soient \(E\) un espace euclidien, \(a\in E\excluant\accol{0}\) et \(k\in\R\). On pose \(f:x\mapsto x+k\ps{x}{a}a\).

Montrez que \(f\) est linéaire, puis déterminez les conditions sur \(a\) et \(k\) pour que \(f\) soit une isométrie vectorielle.

Dans ce cas, reconnaissez-la.
\end{exo}

\begin{corr}
Soient \(\paren{x,y}\in E^2\) et \(\lambda\in\R\).

On a \[\begin{aligned}
f\paren{\lambda x+y}&=\lambda x+y+k\ps{\lambda x+y}{a}a \\
&=\lambda\paren{x+k\ps{x}{a}a}+y+k\ps{y}{a}a \\
&=\lambda f\paren{x}+f\paren{y}.
\end{aligned}\]

Si \(k=0\), alors \(f=\id{E}\in\Orth{}[E]\).

Supposons \(k\not=0\).

Pour tout \(x\in a\ortho\), on a \(f\paren{x}=x\).

On a \(f\paren{a}=a+k\norme{a}^2a=\paren{1+k\norme{a}^2}a\).

Si \(f\) conserve la norme, alors \(\norme{f\paren{a}}=\norme{a}\) donc \(1+k\norme{a}^2=-1\) donc \(k=\dfrac{-2}{\norme{a}^2}\).

Réciproquement, si \(k=\dfrac{-2}{\norme{a}^2}\) alors \(\begin{dcases}
\quantifs{\forall x\in a\ortho}f\paren{x}=x \\
f\paren{a}=-a
\end{dcases}\) donc \(f\) est la réflexion par rapport à \(a\ortho\).
\end{corr}

\section{Matrices orthogonales}

\begin{prop}
Soient \(\fami{B}\) une base orthonormée de \(E\) et \(f\in\Lendo{E}\). On pose \(A=\Mat{f}\).

On a \(f\in\Orth{}[E]\ssi\trans{A}A=I_n\).
\end{prop}

Attention ! Ceci n'est valable que si la base \(\fami{B}\) est orthonormée.

\begin{defi}
Une matrice carrée \(A\) est dite orthogonale quand \(\trans{A}A=I_n\), ce qui est équivalent à \(A\trans{A}=I_n\) ou \(A\) est inversible et \(A\inv=\trans{A}\).
\end{defi}

\begin{prop}
Une matrice de \(\M{n}[\R]\) est orthogonale quand ses colonnes sont de norme \(1\) et deux à deux orthogonales pour le produit scalaire canonique de \(\M{n\,1}[\R]\).
\end{prop}

Cela est également valable pour les lignes de la matrice.

\begin{dem}
Soit \(A=\paren{a_{i,j}}\in\M{n}[\R]\).

On pose \(\trans{A}A=\paren{b_{i,j}}\), où \(\quantifs{\forall\paren{i,j}\in\interventierii{1}{n}^2}b_{i,j}=\sum_{k=1}^na_{k,i}b_{k,j}\).

Alors \(A\) est orthogonale ssi \(\trans{A}A=I_n\), \ie \[\quantifs{\forall\paren{i,j}\in\interventierii{1}{n}^2}b_{i,j}=\begin{dcases}
0 &\text{si }i\not=j \\
1 &\text{sinon}
\end{dcases}\]

Or \[\begin{aligned}
b_{i,i}=1&\ssi\sum_{k=1}^na_{k,i}^2=1 \\
&\ssi\text{la norme de la }i\text{-ème colonne est }1.
\end{aligned}\]

Pour \(i\not=j\), \(b_{i,j}=0\) ssi le produit scalaire canonique des colonnes \(i\) et \(j\) est nul.
\end{dem}

\begin{exo}
Vérifiez que la matrice \(M=\dfrac{1}{7}\begin{pmatrix}
-2 & 6 & -3 \\
6 & 3 & 2 \\
-3 & 2 & 6
\end{pmatrix}\) est une matrice orthogonale, puis montrez qu'elle est la matrice d'une symétrie orthogonale donc vous préciserez les éléments caractéristiques.
\end{exo}

\begin{corr}
On a \[\dfrac{1}{7}\norme{\tcoords{-2}{6}{-3}}=1\qquad\dfrac{1}{7}\norme{\tcoords{6}{3}{2}}=1\qquad\dfrac{1}{7}\norme{\tcoords{-3}{2}{6}}=1\] et \[\ps{\tcoords{-2}{6}{-3}}{\tcoords{6}{3}{2}}=-12+18-6=0\qquad\ps{\tcoords{-2}{6}{-3}}{\tcoords{-3}{2}{6}}=6+12-18=0\qquad\ps{\tcoords{6}{3}{2}}{\tcoords{-3}{2}{6}}=-18+6+12=0.\]

Donc \(M\) est orthogonale : on a \(\trans{M}=M\inv\).

On remarque que \(M\) est symétrique : \(\trans{M}=M\).

Donc \(M=M\inv\) \ie \(M^2=I_3\).

Si on note \(f\) l'endomorphisme de matrice \(M\) dans une base orthonormée, \(f\) est une isométrie vectorielle et une symétrie, donc \(f\) est une symétrie orthogonale.

Les valeurs propres de \(f\) sont \(1\) et \(-1\).

\(f\) est la symétrie orthogonale par rapport à \(\sep{f}{1}\).

Or \(n=3\) et \(\tr f=1\) donc \(\Mat{f}=\begin{pmatrix}
1 & 0 & 0 \\
0 & 1 & 0 \\
0 & 0 & -1
\end{pmatrix}\).

\(f\) est donc une réflexion par rapport au plan d'équation \(3x-2y+z=0\).
\end{corr}

\begin{exo}
Déterminez les réels \(a\) et \(b\) tels que la matrice \(A=\begin{pmatrix}
a & b & b \\
b & a & b \\
b & b & a
\end{pmatrix}\) soit orthogonale. Reconnaissez la nature de l'isométrie vectorielle de matrice \(A\) dans une base orthonormée \(\fami{B}\).
\end{exo}

\begin{corr}
La matrice \(A\) est orthogonale ssi \[\begin{dcases}
a^2+2b^2=1 \\
2ab+b^2=0
\end{dcases}\ssi\begin{dcases}
a^2+2b^2=1 \\
b\paren{b+2a}=0
\end{dcases}\]

Les solutions sont les couples \(\paren{1,0}\), \(\paren{-1,0}\), \(\paren{\nicefrac{1}{3},\nicefrac{-2}{3}}\) et \(\paren{\nicefrac{-1}{3},\nicefrac{2}{3}}\).

On a donc \(A=I_3\) ou \(A=-I_3\) ou (1) \(A=\dfrac{1}{3}\begin{pmatrix}
1 & -2 & -2 \\
-2 & 1 & -2 \\
-2 & -2 & 1
\end{pmatrix}\) ou (2) \(A=\dfrac{-1}{3}\begin{pmatrix}
1 & -2 & -2 \\
-2 & 1 & -2 \\
-2 & -2 & 1
\end{pmatrix}\).

Soit \(X=\tcoords{x}{y}{z}\).

Dans le cas (1), on a \(AX=X\ssi x+y+z=0\) donc \(A\) est la matrice de la réflexion par rapport au plan d'équation \(x+y+z=0\).

Dans le cas (2), on a \(AX=X\ssi\begin{dcases}
y=x \\
z=x
\end{dcases}\) donc \(A\) est la matrice de la symétrie orthogonale par rapport à la droite \(\Vect{\paren{1,1,1}}\).
\end{corr}

L'ensemble des matrices orthogonales est noté \(\Orth{n}[\R]\).

\begin{prop}
\(\Orth{n}[\R]\) est un sous-groupe compact de \(\groupe{\GL{n}[\R]}[\times]\).
\end{prop}

\begin{dem}
Soient \(E\) un espace euclidien de dimension \(n\), \(\fami{B}\) une base orthonormée de \(E\) et \(f\in\Lendo{E}\) de matrice \(A\in\M{n}[\R]\) dans la base \(\fami{B}\).

On a \(A\in\Orth{n}[\R]\ssi f\in\Orth{}[E]\).

Comme \(\Orth{}[E]\) est un groupe pour \(\rond\), \(\Orth{n}[\R]\) est un groupe pour \(\times\) via l'isomorphisme de groupes \[\isomorphismelambda{\groupe{\Orth{}[E]}[\rond]}{\groupe{\Orth{n}[\R]}[\times]}{f}{A}\]

Si \(A=\paren{a_{i,j}}\in\Orth{n}[\R]\), on a \[\quantifs{\forall j\in\interventierii{1}{n}}\sum_{i=1}^na_{i,j}^2=1\] donc \[\quantifs{\forall\paren{i,j}\in\interventierii{1}{n}^2}0\leq a_{i,j}^2\leq1\text{ \ie }\abs{a_{i,j}}\leq1\] donc \(\norme{A}_\infty\leq1\), donc \(\Orth{n}[\R]\) est borné.

De plus, on a \(\Orth{n}[\R]=\accol{M\in\M{n}[\R]\tq\trans{M}M=I_n}\).

On pose \(\phi:M\mapsto\trans{M}M\) continue sur \(\M{n}[\R]\).

Or \(\Orth{n}[\R]=\phi\inv\paren{\accol{I_n}}\) et \(\accol{I_n}\) fermé donc \(\Orth{n}[\R]\) fermé.

Enfin, comme \(\M{n}[\R]\) est un \(\R\)-espace vectoriel de dimension finie, \(\Orth{n}[\R]\) est un compact de \(\M{n}[\R]\).
\end{dem}

\subsection{Déterminant d'une isométrie vectorielle}

\begin{prop}
Si \(f\in\Orth{}[E]\), alors \(\det f\in\accol{-1,1}\).
\end{prop}

\begin{dem}
Soit \(f\in\Orth{}[E]\).

Dans une base orthonormée \(\fami{B}\) de \(E\), \(A=\Mat{f}\) est orthogonale.

Donc \(\trans{A}A=I_n\).

Or \(\det\trans{A}=\det A\) donc \[\det\paren{\trans{A}A}=\det\trans{A}\times\det A=\paren{\det A}^2=1.\]

Donc \(\det A\in\accol{-1,1}\).
\end{dem}

La réciproque est bien sûr fausse.

Les isométries vectorielles de déterminant \(1\) sont celles qui conservent l'orientation : les transforment les bases orthonormées directes en bases orthonormées directes. On les appelle les isométries vectorielles directes ou positives.

On note \(\SO{n}[\R]\) l'ensemble des matrices orthogonales de déterminant \(1\) et \(\SO{}[E]\) l'ensemble des isométries vectorielles positives.

\begin{prop}
\(\SO{}[E]\) est un sous-groupe de \(\Orth{}[E]\), appelé groupe spécial orthogonal de \(E\).

\(\SO{n}[\R]\) est un sous-groupe de \(\Orth{n}[\R]\), appelé groupe spécial orthogonal d'ordre \(n\).
\end{prop}

Les réflexions sont des isométries négatives.

\subsection{Changements de bases orthonormées}

\begin{prop}
Soient \(\fami{B}\) et \(\fami{B}\prim\) deux bases orthonormées de \(E\).

La matrice de passage de \(\fami{B}\) à \(\fami{B}\prim\) est une matrice orthogonale.
\end{prop}

\begin{dem}
On note \(\fami{B}=\paren{e_1,\dots,e_n}\) et \(\fami{B}\prim=\paren{e_1\prim,\dots,e_n\prim}\).

On note \(\pass{\fami{B}}{\fami{B}\prim}=\paren{p_{i,j}}\).

Pour tout \(j\in\interventierii{1}{n}\), on a \(e_j\prim\tcoords{p_{1,j}}{\vdots}{p_{n,j}}_{\fami{B}}\).

Les bases \(\fami{B}\) et \(\fami{B}\prim\) étant orthonormées, on a \[\quantifs{\forall j\in\interventierii{1}{n}}\norme{e_j\prim}=\sqrt{\sum_{i=1}^np_{i,j}^2}=1\] et \[\quantifs{\forall\paren{j,k}\in\interventierii{1}{n}^2}j\not=k\imp\ps{e_j\prim}{e_k\prim}=\sum_{i=1}^np_{i,j}p_{i,k}=0.\]

Donc \(\pass{\fami{B}}{\fami{B}\prim}\in\Orth{n}[\R]\).
\end{dem}

\begin{rem}\thlabel{rem9.13}
De plus, si \(\fami{B}\) et \(\fami{B}\prim\) ont la même orientation, alors \(\det\pass{\fami{B}}{\fami{B}\prim}>0\) donc \(\det\pass{\fami{B}}{\fami{B}\prim}=1\).

Si elles sont de sens contraires, alors \(\det\pass{\fami{B}}{\fami{B}\prim}<0\) donc \(\det\pass{\fami{B}}{\fami{B}\prim}=-1\).
\end{rem}

L'intérêt de ce genre de changement de bases est que la difficulté liée au calcul de l'inverse de la matrice de passage disparaît :

\(X=PX\prim\) est équivalent à \(X\prim=\trans{P}X\) donc \(A\prim=P\inv AP\) devient \(A\prim=\trans{P}AP\).

\subsection{Produit mixte}

\begin{prop}
Soit \(\paren{v_1,\dots,v_n}\) une famille de \(n\) vecteurs de \(E\).

Le déterminant de \(\paren{v_1,\dots,v_n}\) dans une base orthonormée directe ne dépend pas du choix de cette base.
\end{prop}

\begin{dem}
Soient \(\fami{B}\) et \(\fami{B}\prim\) deux bases orthonormées directes de \(E\).

D'après la \thref{rem9.13}, on a \(\det_{\fami{B}}\fami{B}\prim=1\).

Donc \[\begin{aligned}
\det_{\fami{B}}\paren{v_1,\dots,v_n}&=\det_{\fami{B}}\fami{B}\prim\times\det_{\fami{B}\prim}\paren{v_1,\dots,v_n} \\
&=\det_{\fami{B}\prim}\paren{v_1,\dots,v_n}.
\end{aligned}\]
\end{dem}

Dans ce cas, on appelle produit mixte de \(\paren{v_1,\dots,v_n}\) le déterminant de cette famille dans n'importe quelle base orthonormée directe : il est noté habituellement \(\Det\paren{v_1,\dots,v_n}\) ou \(\croch{v_1,\dots,v_n}\).

Une conséquence directe de la définition du produit mixte est la caractérisation des bases directes.

\begin{prop}
Soit \(\paren{v_1,\dots,v_n}\) une famille de \(n\) vecteurs de \(E\).

La famille \(\paren{v_1,\dots,v_n}\) est une base directe de \(E\) ssi \(\croch{v_1,\dots,v_n}>0\).
\end{prop}

\subsection{Produit vectoriel en dimension 3}

Dans ce paragraphe, \(n=3\).

\begin{prop}
Soit \(\paren{u,v}\in E^2\).

Il existe un unique vecteur \(w\in E\) tel que \(\quantifs{\tpt x\in E}\croch{u,v,x}=\ps{w}{x}\).

Ce vecteur est appelé le produit vectoriel de \(u\) et \(v\) et est noté \(u\vecto v\) ou \(u\times v\).
\end{prop}

\begin{dem}
Pour \(\paren{u,v}\in E^2\), l'application \(x\mapsto\croch{u,v,x}\) est une forme linéaire sur \(E\).

Ainsi, d'après le théorème de représentation de Riesz, il existe un unique vecteur \(w\) tel que \[\quantifs{\forall x\in E}\croch{u,v,x}=\ps{w}{x}.\]
\end{dem}

En base orthonormée directe, les coordonnées du produit vectoriel se calculent facilement. En base quelconque, c'est beaucoup plus pénible.

Notons quelques propriétés algébriques et géométriques du produit vectoriel.

\begin{prop}
\begin{itemize}
    \item L'application \(\vecto\) est bilinéaire et antisymétrique. \\
    \item \(u\vecto v=0\) ssi \(u\) et \(v\) sont colinéaires. \\
    \item Si \(u\) et \(v\) ne sont pas colinéaires, alors \(u\vecto v\) est un vecteur normal au plan \(\Vect{u,v}\) et la famille \(\paren{u,v,u\vecto v}\) est une base directe de \(E\). \\
    \item Si \(u\) et \(v\) sont unitaires et orthogonaux, alors la famille \(\paren{u,v,u\vecto v}\) est une base orthonormée directe de \(E\).
\end{itemize}
\end{prop}

\begin{dem}
On reprend les mêmes notations.

\begin{itemize}
    \item L'application \(\vecto\) est bilinéaire antisymétrique car le produit mixte est trilinéaire alterné. \\
    \item Si \(u\) et \(v\) sont colinéaires, on a \(\quantifs{\forall x\in E}\croch{u,v,x}=0\) donc \(u\vecto v=0\). \\\\ Réciproquement, si \(u\) et \(v\) ne sont pas colinéaires alors \(\Vect{u,v}\) est un plan. \\\\ On choisit \(x\in E\) normal à ce plan et on obtient une base de \(E\) : \(\paren{u,v,x}\). \\\\ Alors \(\croch{u,v,x}\not=0\), donc \(\ps{u\vecto v}{x}\not=0\) donc \(u\vecto v\not=0\). \\
    \item Si \(u\) et \(v\) ne sont pas colinéaires, on a \[\ps{u\vecto v}{u}=\croch{u,v,u}=0 \text{ donc } u\vecto v\perp u\] et \[\ps{u\vecto v}{v}=\croch{u,v,v}=0 \text{ donc }u\vecto v\perp v\] donc \(u\vecto v\perp\Vect{u,v}\). \\\\ De plus, on a \[\croch{u,v,u\vecto v}=\ps{u\vecto v}{u\vecto v}=\norme{u\vecto v}^2>0\] donc \(\paren{u,v,u\vecto v}\) est une base directe de \(E\). \\
    \item \(\quantifs{\Tpt\paren{u,v}\in E^2}\norme{u\vecto v}^2+\ps{u}{v}^2=\norme{u}^2\norme{v}^2\). \\\\ On choisit une base orthonormée directe \(\fami{B}\) de \(E\). \\\\ On note \(u\tcoords{x}{y}{z}_{\fami{B}}\) et \(v\tcoords{x\prim}{y\prim}{z\prim}_{\fami{B}}\). \\\\ En posant \(X=\begin{vmatrix}y & y\prim \\ z & z\prim\end{vmatrix}\), \(Y=-\begin{vmatrix}x & x\prim \\ z & z\prim\end{vmatrix}\) et \(Z=\begin{vmatrix}x & x\prim \\ y & y\prim\end{vmatrix}\), on a alors \(u\vecto v\tcoords{X}{Y}{Z}_{\fami{B}}\) et \[\begin{aligned}
    \norme{u\vecto v}^2+\ps{u}{v}^2&=\paren{yz\prim-y\prim z}^2+\paren{xz\prim-x\prim z}^2+\paren{xy\prim-x\prim y}^2+\paren{xx\prim+yy\prim+zz\prim}^2 \\
    &=\paren{x^2+y^2+z^2}\paren{{x\prim}^2+{y\prim}^2+{z\prim}^2}.
    \end{aligned}\]
\end{itemize}
\end{dem}

\section{Étude en dimension 2}

\begin{prop}\thlabel{prop9.18}
\(\Orth{2}[\R]\) contient exclusivement les matrices suivantes :

\begin{itemize}
    \item les matrices de rotation \(R\paren{\theta}=\begin{pmatrix}
        \cos\theta & -\sin\theta \\
        \sin\theta & \cos\theta
    \end{pmatrix}\) \\
    \item les matrices de réflexions \(\begin{pmatrix}
        \cos\theta & \sin\theta \\
        \sin\theta & -\cos\theta
    \end{pmatrix}\)
\end{itemize}

où \(\theta\) est un réel quelconque.
\end{prop}

\begin{dem}~\\
Soit \(A=\begin{pmatrix}
a & c \\
b & d
\end{pmatrix}\in\Orth{2}[\R]\) de déterminant \(1\). On a \[\begin{dcases}
a^2+b^2=1 \\
c^2+d^2=1 \\
ac+bd=0 \\
ad-bc=1
\end{dcases}\] donc il existe \(\theta,\alpha\in\R\) tels que \[\begin{dcases}
a=\cos\theta \\
b=\sin\theta
\end{dcases}\qquad\text{et}\qquad\begin{dcases}
c=\cos\alpha \\
d=\sin\alpha
\end{dcases}\]

On a alors \[\begin{dcases}
ac+bd=0=\cos\paren{\alpha-\theta} \\
ad-bc=1=\sin\paren{\alpha-\theta}
\end{dcases}\]

Donc \(\alpha-\theta\equiv\dfrac{-\pi}{2}\croch{2\pi}\) \ie \(\alpha\equiv\dfrac{-\pi}{2}+\theta\croch{2\pi}\), d'où \[\begin{dcases}
c=-\sin\theta \\
d=\cos\theta
\end{dcases}\] \ie \(A=\begin{pmatrix}
\cos\theta & -\sin\theta \\
\sin\theta & \cos\theta
\end{pmatrix}\).

De même, si \(\det A=-1\), on obtient \(A=\begin{pmatrix}
\cos\theta & \sin\theta \\
\sin\theta & -\cos\theta
\end{pmatrix}\).

Et réciproquement.
\end{dem}

L'ensemble des matrices de rotation forme le sous-groupe \(\SO{2}[\R]\) : c'est l'ensemble des matrices orthogonales de déterminant \(1\).

Il est remarquable que ce groupe est commutatif, car en dimension \(n\geq3\), ce n'est plus le cas. En effet, il est facile de constater que l'application \(\theta\mapsto R\paren{\theta}\) est un morphisme surjectif de groupes de \(\groupe{\R}\) dans \(\groupe{\SO{2}[\R]}[\times]\) (dont le noyau est le sous-groupe \(2\pi\Z\) de \(\groupe{\R}\)).

Autrement dit, l'application \(\fonctionlambda{\U}{\SO{2}[\R]}{\e{\i\theta}}{R\paren{\theta}}\) est un isomorphisme de groupes.

\begin{dem}
On a donc bien \[\SO{2}[\R]=\accol{R\paren{\theta}=\begin{pmatrix}\cos\theta & -\sin\theta \\ \sin\theta & \cos\theta\end{pmatrix}\tq\theta\in\R}\] et on remarque \[\begin{dcases}
R\paren{\theta+\theta\prim}=R\paren{\theta}R\paren{\theta\prim}=R\paren{\theta\prim}R\paren{\theta} \\
R\paren{-\theta}=R\paren{\theta}\inv
\end{dcases}\]
\end{dem}

\begin{prop}
En dimension \(2\), les isométries vectorielles sont :

\begin{itemize}
    \item les rotations vectorielles \\
    \item les réflexions vectorielles.
\end{itemize}
\end{prop}

\begin{rem}
On pose, pour \(\theta\in\R\), \(S\paren{\theta}=\begin{pmatrix}
\cos\theta & \sin\theta \\
\sin\theta & -\cos\theta
\end{pmatrix}\).

\(S\paren{\theta}\) est orthogonale et symétrique donc \(S\paren{\theta}^2=I_2\), donc \(S\paren{\theta}\) est la matrice d'une symétrie orthogonale.

Donc comme \(S\paren{\theta}\) n'est ni \(I_2\) ni \(-I_2\), il s'agit de la matrice d'une réflexion.

Or on a \[\begin{pmatrix}
\cos\theta & \sin\theta \\
\sin\theta & -\cos\theta
\end{pmatrix}\dcoords{\cos\nicefrac{\theta}{2}}{\sin\nicefrac{\theta}{2}}=\dcoords{\cos\paren{\theta-\nicefrac{\theta}{2}}}{\sin\paren{\theta-\nicefrac{\theta}{2}}}=\dcoords{\nicefrac{\paren{\cos\theta}}{2}}{\nicefrac{\paren{\sin\theta}}{2}}.\]
\end{rem}

\section{Réduction des isométries vectorielles ou des matrices orthogonales}

\subsection{Réduction des isométries vectorielles}

D'abord, deux résultats généraux sur les isométries vectorielles.

\begin{prop}\thlabel{prop9.20}
Soit \(f\in\Orth{}[E]\). On a :

\begin{itemize}
    \item \(\Sp{f}\subset\accol{-1,1}\) \\
    \item Si \(F\) est un sous-espace vectoriel de \(E\) stable par \(f\), alors \(F\ortho\) est aussi un sous-espace vectoriel de \(E\) stable par \(f\).
\end{itemize}
\end{prop}

\begin{dem}
\begin{itemize}
    \item Soient \(\lambda\) une valeur propre de \(f\) (s'il en existe) et \(x\) un vecteur propre associé. \\\\ On a \(f\paren{x}=\lambda x\) donc \[\norme{f\paren{x}}=\abs{\lambda}\norme{x}=\norme{x}\not=0.\] Donc \(\abs{\lambda}=1\) \ie \(\lambda\in\accol{-1,1}\). \\
    \item Soit \(F\) un sous-espace vectoriel de \(E\) stable par \(f\). \\\\ D'après la \thref{prop9.5}, \(F\ortho\) est stable par \(f\etoile\). \\\\ Or \(f\in\Orth{}[E]\) donc \(f\etoile=f\inv\), donc \(f\inv\paren{F\ortho}\subset F\ortho\). \\\\ Or \(f\inv\) est un automorphisme de \(E\) donc \(f\inv\) conserve la dimension, donc \(\dim f\inv\paren{F\ortho}=\dim F\ortho\). \\\\ D'où \(f\inv\paren{F\ortho}=F\ortho\) et donc \(F\ortho=f\paren{F\ortho}\).
\end{itemize}
\end{dem}

\begin{lem}\thlabel{lemtheo9.1}
Si \(E\) est un \(\R\)-espace vectoriel de dimension finie \(n\) et \(f\in\Lendo{E}\), alors il existe une droite ou un plan de \(E\) stable par \(f\).
\end{lem}

\begin{dem}
Si \(f\) possède une valeur propre réelle, la droite vectorielle dirigée par n'importe quel vecteur propre associé est stable par \(f\).

Sinon, on choisit une base \(\fami{B}\) de \(E\) et on pose \(A=\Mat{f}\in\M{n}[\R]\).

\(A\) ne possède aucune valeur propre réelle mais a au moins une valeur propre complexe \(\lambda\in\C\excluant\R\).

On choisit \(Z\in\M{n,1}[\C]\) un vecteur propre de \(A\) associé à \(\lambda\).

On a \(Z=X+\i Y\) où \(\paren{X,Y}\in\M{n,1}[\R]^2\).

On pose \(x,y\in E\) tels que \(x\) et \(y\) aient pour coordonnées \(X\) et \(Y\) dans la base \(\fami{B}\).

On veut montrer que \(\Vect{x,y}\) est stable par \(f\) et est un plan de \(E\).

On note \(\lambda=a+\i b\) où \(\paren{a,b}\in\R\times\Rs\).

On a \(AZ=\lambda Z\) donc \[\begin{aligned}
A\paren{X+\i Y}&=\paren{a +\i b}\paren{X+\i Y} \\
AX+\i AY&=aX-bY+\i\paren{aY+bX}
\end{aligned}\] donc \[\begin{dcases}
AX=aX-bY \\
AY=bX+aY
\end{dcases}\] \ie \[\begin{dcases}
f\paren{x}=ax-by\in\Vect{x,y} \\
f\paren{y}=bx+ay\in\Vect{x,y}
\end{dcases}\]

Donc \(\Vect{x,y}\) est stable par \(f\).

Supposons maintenant par l'absurde que \(x\) et \(y\) sont colinéaires.

Comme \(Z\not=0\), \(x\) ou \(y\) est non-nul.

Supposons, par exemple, que \(y=\alpha x\) où \(\alpha\in\R\) et \(x\not=0\).

On a \(AZ=\lambda Z\) donc \[\begin{aligned}
A\paren{X+\i Y}&=A\paren{X+\i\alpha X} \\
&=\paren{a+\i b}\paren{X+\i\alpha X}
\end{aligned}\] donc \[\begin{aligned}
\paren{1+\i\alpha}AX&=\paren{1+\i\alpha}\paren{a+\i b}X \\
AX&=\paren{a+\i b}X
\end{aligned}\] donc \(AX+0\i=AX+\i bX\) donc \(\begin{dcases}
AX=aX \\
0=bX
\end{dcases}\)

Or \(b\not=0\) et \(x\not=0\) : contradiction.

De même si \(x=\alpha y\) avec \(y\not=0\).

Donc \(x\) et \(y\) ne sont pas colinéaires \ie \(\Vect{x,y}\) est un plan.
\end{dem}

De ces propriétés découlent le théorème suivant.

\begin{theo}\thlabel{theo9.1}
Soit \(f\in\Orth{}[E]\).

Il existe une base orthonormée de \(E\) dans laquelle la matrice de \(f\) est diagonale par blocs, les blocs étant des scalaires \(1\) ou \(-1\) ou des matrices \(\paren{2,2}\) de rotation.
\end{theo}

Les matrices diagonales par blocs sont donc du type suivant : \[D=\begin{pmatrix}
1 &  &  &  &  &  &  &  &  &  &  \\
& \ddots &  &  &  &  &  &  &  &  &  \\
&  & 1 &  &  &  &  &  &  &  &  \\
&  &  & -1 &  &  &  &  &  &  &  \\
&  &  &  & \ddots &  &  &  &  &  &  \\
&  &  &  &  & -1 &  &  &  &  &  \\
&  &  &  &  &  & \cos\theta_1 & -\sin\theta_1 &  &  &  \\
&  &  &  &  &  & \sin\theta_1 & \cos\theta_1 &  &  &  \\
&  &  &  &  &  &  &  & \ddots &  &  \\
&  &  &  &  &  &  &  &  & \cos\theta_k & -\sin\theta_k \\
&  &  &  &  &  &  &  &  & \sin\theta_k & \cos\theta_k \\
\end{pmatrix}.\]

\begin{dem}
On raisonne par récurrence sur \(n=\dim E\).

On pose \(\P{n}\) : \guillemets{si \(E\) est de dimension \(n\) et si \(f\in\Orth{}[E]\), alors il existe une base orthonormée ...}

\begin{itemize}
    \item \(n=1\) \\\\ Si \(f\in\Orth{}[E]\) alors \(f=\id{E}\) ou \(f=-\id{E}\) donc dans n'importe quelle base \(\fami{B}\), \(\Mat{f}=\begin{pmatrix}1\end{pmatrix}\) ou \(\begin{pmatrix}-1\end{pmatrix}\). \\\\ Donc \(\P{1}\) est vraie. \\
    \item \(n=2\) \\\\ Si \(f\in\Orth{}[E]\), d'après la \thref{prop9.18}, \(f\) est une rotation et il existe \(\theta\in\R\) tel que dans n'importe quelle base orthonormée directe, \(\Mat{f}=R\paren{\theta}\) ; ou \(f\) est une réflexion, \ie une symétrie orthogonale par rapport à une droite donc il existe une base orthonormée dans laquelle \(\Mat{f}=\begin{pmatrix}
        1 & 0 \\
        0 & -1
    \end{pmatrix}\). \\\\ Donc \(\P{2}\) est vraie. \\
    \item Pour \(n\geq3\), supposons que \(\P{n-2}\) et \(\P{n-1}\) sont vraies, que \(\dim E=n\) et que \(f\in\Orth{}[E]\). \\\\ D'après le \thref{lemtheo9.1}, \(f\) possède une droite stable ou un plan stable et d'après la \thref{prop9.20}, l'orthogonal de ce sous-espace vectoriel stable est aussi stable par \(f\). \\ \begin{itemize}
        \item Si \(f\) possède une droite stable \(D\) : \\\\ Alors \(D\) est une droite propre associée à une valeur propre \(1\) ou \(-1\) de \(f\). \\\\ \(g\), l'endomorphisme induit par \(f\) dans l'hyperplan \(D\ortho\), est une isométrie vectorielle de \(D\ortho\). \\\\ Or \(\dim D\ortho=n-1\) donc d'après \(\P{n-1}\), il existe une base orthonormée \(\fami{B}\ortho\) de \(D\ortho\) dans laquelle \[\Mat[\fami{B}\ortho]{g}=\begin{pmatrix}
            1 &        &   &    &        &    &                   &        &  \\
              & \ddots &   &    &        &    &                   &        &  \\
              &        & 1 &    &        &    &                   &        &  \\
              &        &   & -1 &        &    &                   &        &  \\
              &        &   &    & \ddots &    &                   &        &  \\
              &        &   &    &        & -1 &                   &        &  \\
              &        &   &    &        &    & R\paren{\theta_1} &        &  \\
              &        &   &    &        &    &                   & \ddots &  \\
              &        &   &    &        &    &                   &        & R\paren{\theta_k}
        \end{pmatrix}.\] En choisissant un vecteur directeur unitaire \(u\) de \(D\), on obtient \(\fami{B}=\paren{u,\fami{B}\ortho}\) une base orthonormée de \(E\) telle que \[\Mat{f}=\begin{pmatrix}
            \pm1 &   &        &   &    &        &    &                   &        &       \\
                 & 1 &        &   &    &        &    &                   &        &  \\
                 &   & \ddots &   &    &        &    &                   &        &  \\
                 &   &        & 1 &    &        &    &                   &        &  \\
                 &   &        &   & -1 &        &    &                   &        &  \\
                 &   &        &   &    & \ddots &    &                   &        &  \\
                 &   &        &   &    &        & -1 &                   &        &  \\
                 &   &        &   &    &        &    & R\paren{\theta_1} &        &  \\
                 &   &        &   &    &        &    &                   & \ddots &  \\
                 &   &        &   &    &        &    &                   &        & R\paren{\theta_k}
        \end{pmatrix}.\]
        \item Si \(f\) possède un plan stable \(P\) : \\\\ Alors \(P\ortho\) est stable par \(f\). \\\\ L'endomorphisme induit par \(f\) dans \(P\) est une isométrie vectorielle de \(P\). \\\\ Or \(\dim P=2\) donc il existe une base orthonormée de \(P\) dans laquelle l'endomorphisme induit par \(f\) dans \(P\) a pour matrice \(R\paren{\theta}\) ou \(\begin{pmatrix}
            1 & 0 \\
            0 & -1
        \end{pmatrix}\). \\\\ L'endomorphisme induit par \(f\) dans \(P\ortho\) est une isométrie vectorielle de \(P\ortho\), or \(\dim P\ortho=n-2\). \\\\ D'après \(\P{n-2}\), ... \\\\ En concaténant deux bases orthonormées de \(P\) et \(P\ortho\), on obtient une base \(\fami{B}\) dans laquelle \[\Mat{f}=\begin{pmatrix}
            R\paren{\theta} &   &        &   &    &        &    &                   &        &       \\
                     & 1 &        &   &    &        &    &                   &        &  \\
                     &   & \ddots &   &    &        &    &                   &        &  \\
                     &   &        & 1 &    &        &    &                   &        &  \\
                     &   &        &   & -1 &        &    &                   &        &  \\
                     &   &        &   &    & \ddots &    &                   &        &  \\
                     &   &        &   &    &        & -1 &                   &        &  \\
                     &   &        &   &    &        &    & R\paren{\theta_1} &        &  \\
                     &   &        &   &    &        &    &                   & \ddots &  \\
                     &   &        &   &    &        &    &                   &        & R\paren{\theta_k}
        \end{pmatrix}\] ou \[\Mat{f}=\begin{pmatrix}
            1 &    &        &   &    &        &    &                   &        &       \\
              & -1 &&&&&&&& \\
              &    & 1 &        &   &    &        &    &                   &        &  \\
              &&   & \ddots &   &    &        &    &                   &        &  \\
              &&   &        & 1 &    &        &    &                   &        &  \\
              &&   &        &   & -1 &        &    &                   &        &  \\
              &&   &        &   &    & \ddots &    &                   &        &  \\
              &&   &        &   &    &        & -1 &                   &        &  \\
              &&   &        &   &    &        &    & R\paren{\theta_1} &        &  \\
              &&   &        &   &    &        &    &                   & \ddots &  \\
              &&   &        &   &    &        &    &                   &        & R\paren{\theta_k}
        \end{pmatrix}\] Dans les deux cas, \(\P{n}\) est vraie.
    \end{itemize}
\end{itemize}
\end{dem}

\subsection{Réduction des matrices orthogonales}

\begin{defi}
Soient \(A,B\) deux matrices de \(\M{n}[\R]\).

On dit que \(A\) et \(B\) sont orthogonalement semblables (ou orthosemblables) quand il existe \(P\in\Orth{n}[\R]\) telle que \(B=P\inv AP=\trans{P}AP\).
\end{defi}

Deux matrices sont orthosemblables quand elles représentent le même endomorphisme dans des bases orthonormées différentes.

Le théorème de réduction précédent a une traduction matricielle.

\begin{theo}
Toute matrice orthogonale est orthosemblable à une matrice diagonale par blocs du type ci-dessus.

Pour tout \(A\in\Orth{n}[\R]\), il existe \(P\in\Orth{n}[\R]\) et \(D\) diagonale par blocs comme ci-dessus telles que \(A=\trans{P}DP\).
\end{theo}

\subsection{Étude en dimension 3}

À l'aide de ce résultat, on peut classifier les isométries vectorielles de \(E\) en dimension \(3\). Seule la réduction des rotations est au programme.

Dans la suite de cette section, \(E\) est un espace euclidien de dimension \(3\) et orienté.

\begin{prop}
Soit \(f\in\Orth{}[E]\). On pose \(F=\ker\paren{f-\id{E}}\). Alors

\begin{itemize}
    \item si \(\dim F=3\), alors \(f=\id{E}\) \\
    \item si \(\dim F=2\), alors \(f\) est la réflexion par rapport à \(F\) \\
    \item si \(\dim F=1\), alors \(f\) est une rotation d'axe \(F\) \\
    \item si \(\dim F=0\), alors \(f\) est une antirotation, \cad la composée d'une rotation et d'une réflexion dont l'axe et le plan de base respectifs sont orthogonaux.
\end{itemize}
\end{prop}

\begin{dem}
Si \(\dim F=1\), d'après le \thref{theo9.1}, il existe une base orthonormée \(\fami{B}\) telle que \[\Mat{f}=\begin{pmatrix}
1 & 0 \\
0 & ?
\end{pmatrix}\] où le bloc \(?\) est \(\begin{pmatrix}
-1 & 0 \\
0 & -1
\end{pmatrix}=R\paren{\pi}\) ou \(\begin{pmatrix}
\cos\theta & -\sin\theta \\
\sin\theta & \cos\theta
\end{pmatrix}=R\paren{\theta}\).

On pose \(\fami{B}=\paren{e_1,e_2,e_3}\).

On a \(f\paren{e_1}=e_1\).

Dans le plan \(F\ortho\), stable par \(f\), l'endomorphisme induit par \(f\) est une rotation \guillemets{d'angle \(\theta\)} (ou \(-\theta\) selon l'orientation choisie de \(F\ortho\)).

On choisit l'orientation de \(F\ortho\) de sorte que \(\fami{B}\) soit directe.

Ce qui caractérise \(f\), c'est le couple \(\paren{e_1,\theta}\) : \(e_1\) dirige \(F\) et \(\theta\) est l'angle associé.

Pour déterminer \(e_1\), on résout \(f\paren{e_1}=e_1\) et on choisit une solution de norme \(1\).

Pour déterminer l'angle \(\theta\) associé, on choisit un vecteur \(v\not=0\) orthogonal à \(e_1\) puis on calcule \(f\paren{v}\) et \(e_1\vecto v\).

Comme \(\norme{e_1}=1\), on a \(\norme{e_1\vecto v}=\norme{v}=\norme{f\paren{v}}\).

Alors \(f\paren{v}=\cos\paren{\theta}v+\sin\paren{\theta}\paren{e_1\vecto v}\).

On en déduit \(\cos\theta\) et \(\sin\theta\) et enfin \(\theta\) (modulo \(2\pi\)).
\end{dem}

En étudiant les différents cas, on constate un lien entre le type de \(f\) et son déterminant.

\begin{prop}
Soit \(f\in\Orth{}[E]\) telle que \(f\not=\id{E}\).

\(f\) est une rotation ssi \(\det f=1\).
\end{prop}

Dans le cas où \(\det f=-1\), cette information ne suffit pas à connaître le type de \(f\). Cependant, si on connaît la matrice \(A\) de \(f\) dans une base orthonormée, alors on peut distinguer les cas 1 et 3.

\begin{prop}
Soient \(f\in\Lendo{E}\) et \(A\) la matrice de \(f\) dans une base orthonormée.

Si \(A\) est une matrice orthogonale et symétrique, alors \(f\) est une symétrie orthogonale.

\begin{itemize}
    \item Si \(\det f=1\), alors \(A\) est un demi-tour (une rotation d'angle \(\pi\)). \\
    \item Si \(\det f=-1\), alors \(A\) est une réflexion.
\end{itemize}
\end{prop}

Donc, si \(A\) est orthogonale de déterminant \(-1\) et non-symétrique, alors \(f\) est une antirotation.

\begin{exo}
Reconnaissez la nature de l'endomorphisme dont la matrice dans une base orthonormée \(\fami{B}\) est \(\dfrac{1}{15}\begin{pmatrix}
-11 & 10 & 2 \\
-2 & -5 & 14 \\
10 & 10 & 5
\end{pmatrix}\) et précisez ses éléments caractéristiques.
\end{exo}

\begin{corr}
On note \(A\) la matrice donnée et \(C_1,C_2,C_3\) ses colonnes.

On a \[\norme{C_1}^2=\dfrac{1}{15^2}\paren{\paren{-11}^2+\paren{-2}^2+10^2}=1=\norme{C_2}^2=\norme{C_3}^2\] et \[\ps{C_1}{C_2}=\ps{C_1}{C_3}=\ps{C_2}{C_3}=0\] donc \(A\in\Orth{3}[\R]\).

Or \(\fami{B}\) est orthonormée donc \(f\in\Orth{}[E]\).

On note \(u\tcoords{x}{y}{z}_{\fami{B}}\in E\) et on a \[\begin{aligned}
f\paren{u}=u&\ssi\begin{dcases}
-11x+10y+2z=15x \\
-2x-5y+14z=15y \\
10x+10y+5z=15z
\end{dcases} \\
&\ssi\begin{dcases}
-13x+5y+z=0 \\
-x-10y+7z=0 \\
x+y-z=0
\end{dcases} \\
&\ssi\begin{dcases}
x+y-z=0 \\
-9y+6z=0 \\
18y-12z=0
\end{dcases} \\
&\ssi\begin{dcases}
x=\dfrac{1}{3}z \\
y=\dfrac{2}{3}z
\end{dcases}
\end{aligned}\]

Donc \(\ker\paren{f-\id{E}}\) est la droite vectorielle dirigée par \(e_1\tcoords{\nicefrac{1}{\sqrt{14}}}{\nicefrac{2}{\sqrt{14}}}{\nicefrac{3}{\sqrt{14}}}_{\fami{B}}\) de norme \(1\), donc \(f\) est une rotation.

On pose \(v\tcoords{-2}{1}{0}_{\fami{B}}\) de sorte que \(v\perp e_1\).

On a \(f\paren{v}\tcoords{\nicefrac{32}{15}}{\nicefrac{-1}{15}}{\nicefrac{-10}{15}}\) et \(e_1\vecto v\tcoords{\nicefrac{-3}{\sqrt{14}}}{\nicefrac{-6}{\sqrt{14}}}{\nicefrac{5}{\sqrt{14}}}_{\fami{B}}\).

Alors \[\begin{aligned}
f\paren{v}=\cos\paren{\theta}v+\sin\paren{\theta}\paren{e_1\vecto v}&\ssi\begin{dcases}
-2\cos\theta-\dfrac{3}{\sqrt{14}}\sin\theta=\dfrac{32}{15} \\
\cos\theta-\dfrac{6}{\sqrt{14}}\sin\theta=\dfrac{-1}{15} \\
\dfrac{5}{\sqrt{14}}\sin\theta=\dfrac{-10}{15}
\end{dcases} \\
&\ssi\begin{dcases}
\sin\theta=\dfrac{-2\sqrt{14}}{15} \\
\cos\theta=\dfrac{-1}{15}+\dfrac{6}{\sqrt{14}}\sin\theta=\dfrac{-13}{15}
\end{dcases}
\end{aligned}\]

Enfin, on obtient \[\theta\equiv\pi+\Arctan\dfrac{2\sqrt{14}}{15}\croch{2\pi}.\]
\end{corr}

\begin{exo}
Même exercice avec la matrice \(\dfrac{1}{7}\begin{pmatrix}
6 & -2 & -3 \\
3 & 6 & 2 \\
2 & -3 & 6
\end{pmatrix}\).
\end{exo}

\begin{corr}
On note \(A\) la matrice donnée et on vérifie aisément que \(A\) est orthogonale.

On a \(\det A=1\) donc \(f\) est une rotation.

On pose \(u\tcoords{x}{y}{z}_{\fami{B}}\) et on a \[\begin{aligned}
f\paren{u}=u&\ssi\begin{dcases}
6x-2y-3z=7x \\
3x+6y+2z=7y \\
2x-3y+6z=7z
\end{dcases} \\
&\ssi\begin{dcases}
-x-2y-3z=0 \\
3x-y+2z=0 \\
2x-3y-z=0
\end{dcases} \\
&\ssi\begin{dcases}
x+2y+3z=0 \\
-7y-7z=0 \\
-7y-7z=0
\end{dcases} \\
&\ssi\begin{dcases}
y=-z \\
x=y
\end{dcases}
\end{aligned}\]

Donc \(\ker\paren{f-\id{E}}=\Vect{\tcoords{1}{1}{-1}}=\Vect{\tcoords{\nicefrac{1}{\sqrt{3}}}{\nicefrac{1}{\sqrt{3}}}{\nicefrac{-1}{\sqrt{3}}}_{\fami{B}}}=\Vect{e_1}\).

On pose \(v\tcoords{0}{1}{1}_{\fami{B}}\perp e_1\) et on a \[f\paren{v}\tcoords{\nicefrac{-5}{7}}{\nicefrac{8}{7}}{\nicefrac{3}{7}}_{\fami{B}}\qquad\text{et}\qquad e_1\vecto v\tcoords{\nicefrac{2}{\sqrt{3}}}{\nicefrac{-1}{\sqrt{3}}}{\nicefrac{1}{\sqrt{3}}}_{\fami{B}}.\]

Alors \[\begin{aligned}
f\paren{v}=\cos\paren{\theta}v+\sin\paren{\theta}\paren{e_1\vecto v}&\ssi\begin{dcases}
\dfrac{2}{\sqrt{3}}\sin\theta=\dfrac{-5}{7} \\
\cos\theta-\dfrac{1}{\sqrt{3}}\sin\theta=\dfrac{8}{7} \\
\cos\theta+\dfrac{1}{\sqrt{3}}\sin\theta=\dfrac{3}{7}
\end{dcases} \\
&\ssi\begin{dcases}
\sin\theta=\dfrac{-5\sqrt{3}}{14} \\
\cos\theta=\dfrac{8}{7}+\dfrac{1}{\sqrt{3}}\paren{\dfrac{-5\sqrt{3}}{14}}=\dfrac{8}{7}-\dfrac{5}{14}=\dfrac{11}{14}
\end{dcases}
\end{aligned}\]

Donc \(f\) est la rotation d'axe orienté par \(e_1\) et d'angle associé \(\theta=\Arctan\dfrac{-5\sqrt{3}}{11}\).
\end{corr}

\begin{exo}
Même exercice avec la matrice \(\dfrac{1}{7}\begin{pmatrix}
6 & -3 & -2 \\
3 & 2 & 6 \\
2 & 6 & -3
\end{pmatrix}\).
\end{exo}

\begin{corr}
On note \(A\) la matrice donnée. Comme ses colonnes sont de norme 1 et leurs produits scalaires deux à deux tous nuls, \(A\) est orthogonale et donc \(f\) est une isométrie vectorielle.

On a \(\det A=-1\) donc \(f\) est une antirotation.
\end{corr}

\begin{exo}
Soit \(\fami{B}=\paren{i,j,k}\) une base orthonormée directe de \(E\) de dimension \(3\).

Déterminez la matrice dans la base \(\fami{B}\) de la rotation d'axe orienté par \(i+j+k\) et d'angle \(\dfrac{\pi}{3}\).
\end{exo}

\begin{corr}
\(f\) est la rotation d'axe orienté par \(i+j+k\) et d'angle \(\dfrac{\pi}{3}\) donc il existe une base orthonormée \(\fami{B}\prim=\paren{e_1,e_2,e_3}\) telle que \[A\prim=\Mat[\fami{B}\prim]{f}=\begin{pmatrix}
1 & 0 & 0 \\
0 & \cos\nicefrac{\pi}{3} & -\sin\nicefrac{\pi}{3} \\
0 & \sin\nicefrac{\pi}{3} & \cos\nicefrac{\pi}{3}
\end{pmatrix}=\begin{pmatrix}
1 & 0 & 0 \\
0 & \nicefrac{1}{2} & -\nicefrac{\sqrt{3}}{2} \\
0 & \nicefrac{\sqrt{3}}{2} & \nicefrac{1}{2}
\end{pmatrix}.\]

En notant \(P=\pass{\fami{B}}{\fami{B}\prim}\), on a \(A\prim=P\inv AP\) donc \[A=PA\prim P\inv=PA\prim\trans{P}.\]

On a \(e_1=\dfrac{i+j+k}{\sqrt{3}}\tcoords{\nicefrac{1}{\sqrt{3}}}{\nicefrac{1}{\sqrt{3}}}{\nicefrac{1}{\sqrt{3}}}\) donc \(P=\begin{pmatrix}
\nicefrac{1}{\sqrt{3}} & ? & ? \\
\nicefrac{1}{\sqrt{3}} & ? & ? \\
\nicefrac{1}{\sqrt{3}} & ? & ?
\end{pmatrix}\).

On a \(e_2\perp e_1\) et \(\norme{e_2}=1\), par exemple \(e_2\tcoords{\nicefrac{1}{\sqrt{2}}}{\nicefrac{-1}{\sqrt{2}}}{0}\), donc \(P=\begin{pmatrix}
\nicefrac{1}{\sqrt{3}} & \nicefrac{1}{\sqrt{2}} & ? \\
\nicefrac{1}{\sqrt{3}} & \nicefrac{-1}{\sqrt{2}} & ? \\
\nicefrac{1}{\sqrt{3}} & 0 & ?
\end{pmatrix}\).

Puis \(e_3=e_1\vecto e_2\tcoords{\nicefrac{1}{\sqrt{6}}}{\nicefrac{1}{\sqrt{6}}}{\nicefrac{-2}{\sqrt{6}}}\) donc \(P=\begin{pmatrix}
\nicefrac{1}{\sqrt{3}} & \nicefrac{1}{\sqrt{2}} & \nicefrac{1}{\sqrt{6}} \\
\nicefrac{1}{\sqrt{3}} & \nicefrac{-1}{\sqrt{2}} & \nicefrac{1}{\sqrt{6}} \\
\nicefrac{1}{\sqrt{3}} & 0 & \nicefrac{-2}{\sqrt{6}}
\end{pmatrix}\).

On en déduit \(A\).
\end{corr}

\section{Endomorphismes auto-adjoints}

\subsection{Définition et propriétés}

\begin{defi}
On dit qu'un endomorphisme \(f\) de \(E\) est auto-adjoint quand \(f\adj=f\), autrement dit quand \[\quantifs{\forall\paren{x,y}\in E^2}\ps{f\paren{x}}{y}=\ps{x}{f\paren{y}}.\]
\end{defi}

On rencontre encore très souvent le mot \guillemets{symétrique} pour \guillemets{auto-adjoint}.

\begin{ex}
\begin{itemize}
    \item Les projecteurs orthogonaux sont des endomorphismes auto-adjoints (mais pas des endomorphismes orthogonaux !). \\
    \item Les symétries orthogonales sont aussi des endomorphismes auto-adjoints.
\end{itemize}
\end{ex}

\begin{dem}
Soit \(s\) une symétrie orthogonale. On lui associe un projecteur \(p\) tel que \(s=2p-\id{E}\).

\(p\) est un projecteur orthogonal donc auto-adjoint.

Pour tout \(\paren{x,y}\in E^2\), on a \[\begin{aligned}
\ps{s\paren{x}}{y}&=\ps{2p\paren{x}-x}{y} \\
&=2\ps{p\paren{x}}{y}-\ps{x}{y} \\
&=2\ps{x}{p\paren{y}}-\ps{x}{y} \\
&=\ps{x}{2p\paren{y}-y} \\
&=\ps{x}{s\paren{y}}.
\end{aligned}\]

Donc \(s\) est auto-adjoint.
\end{dem}

\begin{prop}
Soient \(\fami{B}\) une base orthonormée de \(E\) et \(f\in\Lendo{E}\).

\(f\) est un endomorphisme auto-adjoint ssi sa matrice dans la base \(\fami{B}\) est symétrique.
\end{prop}

\begin{cor}
L'ensemble des endomorphismes auto-adjoints est un sous-espace vectoriel de \(\Lendo{E}\), de dimension \(\dfrac{n\paren{n+1}}{2}\).
\end{cor}

Il est noté \(\sym{}[E]\).

\subsection{Théorème spectral}

Il y a essentiellement un seul résultat à connaître sur les endomorphismes auto-adjoints ! On commence par deux lemmes.

\begin{lem}\thlabel{lem9.2}
Le polynôme caractéristique d'un endomorphisme auto-adjoint est scindé sur \(\R\).
\end{lem}

\begin{dem}
Soient \(f\in\sym{}[E]\) et \(\fami{B}\) une base orthonormée de \(E\).

Comme \(f\) est auto-adjoint, on a \(A=\Mat{f}\in\sym{n}[\R]\).

\(\chi_A\) est scindé sur \(\C\) donc pour montrer que \(\chi_A\) est scindé sur \(\R\), on montre que toutes les valeurs propres complexes de \(A\) sont en fait réelles.

Soient \(\lambda\in\Sp[\C]{A}\) et \(Z\) un vecteur propre associé.

Comme \(\trans{A}=A\), on a \(\trans{\paren{AZ}}=\trans{Z}\trans{A}=\trans{Z}A=\trans{\paren{\lambda Z}}=\lambda\trans{Z}\).

Alors \[\begin{aligned}
\trans{\conj{Z}}AZ&=\paren{\trans{\conj{Z}}A}Z \\
&=\conj{\trans{Z}A}Z \\
&=\conj{\lambda\trans{Z}}Z \\
&=\conj{\lambda}\times\conj{\trans{Z}}Z
\end{aligned}\] et \(\trans{\conj{Z}}AZ=\trans{\conj{Z}}\paren{AZ}=\lambda\trans{\conj{Z}}Z\) donc \[\lambda\trans{\conj{Z}}Z=\conj{\lambda}\times\trans{\conj{Z}}Z.\]

Or, avec \(Z=\tcoords{z_1}{\vdots}{z_n}\), on a \(\conj{\trans{Z}}Z=\sum_{i=1}^nz_i\conj{z_i}=\sum_{i=1}^n\abs{z_i}^2>0\).

Donc \(\trans{\conj{Z}}Z\not=0\) donc \(\lambda=\conj{\lambda}\) donc \(\lambda\in\R\).
\end{dem}

\begin{lem}\thlabel{lem9.3}
Si un sous-espace vectoriel \(F\) est stable par un endomorphisme auto-adjoint, alors \(F\ortho\) l'est aussi.
\end{lem}

\begin{theo}\thlabel{theo9.3}
Les sous-espaces propres d'un endomorphisme auto-adjoint sont deux à deux orthogonaux et leur somme directe est \(E\).

Autrement dit, tout endomorphisme auto-adjoint est diagonalisable en base orthonormée, \cad qu'il existe une base orthonormée de vecteurs propres.
\end{theo}

\begin{dem}
On pose \(\P{n}\) : l'énoncé du \thref{theo9.3} avec \(\dim E=n\).

\begin{itemize}
    \item \(n=1\) \\\\ On a \(\P{1}\) car tout endomorphisme est auto-adjoint et tout vecteur non-nul est propre. \\
    \item Supposons \(\P{n-1}\) et soit \(f\in\sym{}[E]\) où \(\dim E=n\). \\\\ D'après le \thref{lem9.2}, \(f\) possède au moins une valeur propre \(\lambda\) et un vecteur propre associé \(u\). \\\\ \(D=\Vect{u}\) est stable par \(f\) (c'est une droite propre) donc d'après le \thref{lem9.3}, \(D\ortho\) est stable par \(f\). \\\\ Or \(\dim D\ortho=n-1\) donc par hypothèse de récurrence, l'endomorphisme induit par \(f\) dans \(D\ortho\) est auto-adjoint et on peut donc trouver dans \(D\ortho\) une base orthonormée de vecteurs propres pour cet endomorphisme : \(\paren{e_2,\dots,e_n}\). \\\\ \(\fami{B}=\paren{\dfrac{u}{\norme{u}},e_2,\dots,e_n}\) est alors une base orthonormée de \(E\) dont tous les vecteurs sont propres pour \(f\). \\\\ Donc \(\P{n}\) est vraie.
\end{itemize}
\end{dem}

On dit que les endomorphismes auto-adjoints sont orthodiagonalisables.

\begin{rem}
La réciproque est vraie et presque évidente : si un endomorphisme est orthodiagonalisable, alors il est auto-adjoint.
\end{rem}

\begin{exo}[Un grand classique à savoir refaire]
Soit \(u\) un endomorphisme auto-adjoint de \(E\), \(B\) la boule-unité fermée de \(E\) et \(S\) la sphère-unité de \(E\).

On pose \(\alpha\) la plus petite des valeurs propres de \(u\) et \(\beta\) la plus grande.

Montrez que \(\inf_{x\in S}\ps{x}{u\paren{x}}=\alpha\) et \(\sup_{x\in B}\ps{x}{u\paren{x}}=\sup_{x\in S}\ps{x}{u\paren{x}}=\beta\).
\end{exo}

\begin{corr}
\(u\) est auto-adjoint donc orthodiagonalisable : il existe une base orthonormée \(\fami{B}=\paren{e_1,\dots,e_n}\) de \(E\) telle que \(\Mat{u}\) soit diagonale.

On pose \(D=\Mat{u}=\diag{\lambda_1,\dots,\lambda_n}\).

On peut choisir d'ordonner les valeurs propres par ordre croissant : \[\alpha=\lambda_1\leq\dots\leq\lambda_n=\beta.\]

Soit \(x\in E\) de coordonnées \(X=\tcoords{x_1}{\vdots}{x_n}_{\fami{B}}\).

On a \(\ps{x}{u\paren{x}}=\trans{X}DX=\sum_{i=1}^n\lambda_ix_i^2\).

Si \(x\in B\), alors \(\norme{x}\leq1\).

Pour tout \(i\in\interventierii{1}{n}\), on a \(\lambda_i\leq\beta\) donc \(\lambda_ix_i^2\leq\beta x_i^2\), et donc \[\begin{aligned}
\ps{x}{u\paren{x}}&=\sum_{i=1}^n\lambda_ix_i^2 \\
&\leq\sum_{i=1}^n\beta x_i^2 \\
&=\beta\norme{x}^2 \\
&\leq\beta.
\end{aligned}\]

On a montré \(\quantifs{\forall x\in B}\ps{x}{u\paren{x}}\leq\beta\).

Pour \(x=e_n\), on a \(u\paren{e_n}=\lambda_ne_n=\beta e_n\) donc \[\ps{e_n}{u\paren{e_n}}=\ps{e_n}{\beta e_n}=\beta\norme{e_n}^2=\beta.\] Ceci prouve \(\max_{x\in B}\ps{x}{u\paren{x}}=\beta\).

Si on remplace \(B\) par \(S\), idem.

De même, soit \(x\in S\).

Pour tout \(i\in\interventierii{1}{n}\), on a \(\lambda_i\geq\alpha\) donc \(\lambda_ix_i^2\geq\alpha x_i^2\).

Donc \[\ps{x}{u\paren{x}}=\sum_{i=1}^n\lambda_ix_i^2\geq\alpha\sum_{i=1}^nx_i^2=\alpha\norme{x}^2=\alpha.\]

On a montré \(\quantifs{\forall x\in S}\ps{x}{u\paren{x}}\geq\alpha\), avec égalité en \(x=e_1\).

Donc \(\alpha=\min_{x\in S}\ps{x}{u\paren{x}}\).
\end{corr}

\begin{exo}[Un prolongement de l'exercice précédent]
Montrez que l'application \(N:\sym{}[E]\to\Rp\) définie par \(N\paren{u}=\sup_{x\in B}\abs{\ps{x}{u\paren{x}}}\) est une norme sur \(\sym{}[E]\).
\end{exo}

Le théorème précédent a une version matricielle.

\begin{theo}
Une matrice réelle est orthosemblable à une matrice diagonale ssi elle est symétrique.
\end{theo}

On dit que les matrices symétriques réelles sont orthodiagonalisables.

\begin{exo}
Orthodiagonalisez la matrice suivante : \[A=\begin{pmatrix}
0 & 1 & -1 \\
1 & 0 & -1 \\
-1 & -1 & 0
\end{pmatrix}.\]
\end{exo}

\begin{rem}
La condition \guillemets{réelle} est indispensable dans le théorème spectral !
\end{rem}

\section{Endomorphismes auto-adjoints positifs, définis-positifs}

\subsection{Endomorphismes auto-adjoints positifs}

\begin{defi}
Soit \(f\in\sym{}[E]\).

On dit que \(f\) est un endomorphisme auto-adjoint positif quand \(\quantifs{\tpt x\in E}\ps{f\paren{x}}{x}\geq0\).

On dit que \(f\) est un endomorphisme auto-adjoint défini-positif quand \(\quantifs{\tpt x\in E\excluant\accol{0}}\ps{f\paren{x}}{x}>0\).
\end{defi}

On note \(\sympos{}[E]\) l'ensemble des endomorphismes auto-adjoints positifs et \(\symdefpos{}[E]\) celui des endomorphismes auto-adjoints définis positifs. Attention, ces deux ensembles ne sont pas des espaces vectoriels et ne sont pas stables par composition.

Ces endomorphismes sont couramment présents dans les théories physiques et sont l'objet de propriétés spécifiques.

\begin{rem}
Si \(f\in\symdefpos{}[E]\), \(\paren{x,y}\mapsto\ps{x}{f\paren{y}}\) est aussi un produit scalaire.
\end{rem}

On donne par exemple une caractérisation simple à l'aide de valeurs propres.

\begin{prop}
Soit \(f\in\sym{}[E]\).

On a \(f\in\sympos{}[E]\) ssi les valeurs propres de \(f\) sont positives.

De même, \(f\in\symdefpos{}[E]\) ssi les valeurs propres de \(f\) sont strictement positives.
\end{prop}

\begin{dem}
Supposons \(f\in\sympos{}[E]\).

Soient \(\lambda\in\Sp{f}\) et \(x\) un vecteur propre associé.

Alors \(\ps{x}{f\paren{x}}=\ps{x}{\lambda x}=\lambda\norme{x}^2\geq0\).

Or \(\norme{x}^2>0\) donc \(\lambda\geq0\).

Réciproquement, si \(f\in\sym{}[E]\) et \(\Sp{f}\subset\intervie{0}{\pinf}\), on choisit une base \(\fami{B}\) orthonormée de diagonalisation de \(f\) telle que \[\Mat{f}=\diag{\lambda_1,\dots,\lambda_n}.\]

Avec \(x\tcoords{x_1}{\vdots}{x_n}_{\fami{B}}\), on a \(\ps{x}{f\paren{x}}=\sum_{i=1}^n\lambda_ix_i^2\geq0\).

Idem pour \(\symdefpos{}[E]\) en mettant des inégalités strictes.
\end{dem}

En particulier, \(\symdefpos{}[E]=\sympos{}[E]\inter\GL{}[E]\).

\subsection{Matrices symétriques positives}

\begin{defi}
Soit \(A\in\sym{n}[\R]\).

On dit que \(A\) est une matrice symétrique positive quand \(\quantifs{\tpt X\in\R^n}\trans{X}AX\geq0\).

On dit que \(A\) est une matrice symétrique définie-positive quand \(\quantifs{\tpt X\in\R^n\excluant\accol{0}}\trans{X}AX>0\).
\end{defi}

Les matrices symétriques positives (respectivement définies-positives) sont donc les matrices dans des bases orthonormées des endomorphismes auto-adjoints positifs (respectivement définis-positifs).

On note \(\sympos{n}[\R]\) l'ensemble des matrices symétriques positives et \(\symdefpos{n}[\R]\) celui des matrices symétriques définies-positives. Attention, ces deux ensembles ne sont pas des espaces vectoriels et ne sont pas stables par produit.

\begin{prop}
Soit \(A\in\sym{n}[\R]\).

On a \(A\in\sympos{n}[\R]\) ssi les valeurs propres de \(A\) sont positives.

De même, \(A\in\symdefpos{n}[\R]\) ssi les valeurs propres de \(A\) sont strictement positives.
\end{prop}


\include{cours/chap10_fonctions_vectorielles/chap10_fonctions_vectorielles_notes}

\part{Exercices}

\setcounter{chapter}{0}

\chapter{Espaces vectoriels normés}

\legendeexercices

\begin{exopss}[Exercice 1]
Soit \(u\) une suite réelle bornée. Pour \(n\in\N\), on pose \(U_n=\accol{u_p\tq p\geq n}\), \(a_n=\inf U_n\) et \(b_n=\sup U_n\).

\begin{enumerate}
    \item Justifiez l'existence des suites \(a\) et \(b\) et étudiez leur monotonie, ainsi que leur convergence. \\
    \item Montrez que \(u\) converge ssi \(a\) et \(b\) ont la même limite.
\end{enumerate}

Note culturelle : la limite de \(a\) s'appelle la limite inférieure de \(u\), notée \(\underline{\lim}\,u\) et celle de \(b\) est la limite supérieure, notée \(\overline{\lim}\,u\).
\end{exopss}

\begin{corr}
\note{À venir}
\end{corr}

\begin{exops}[Exercice 2]\label{exo:1.2}
Montrez que les applications \(N\) introduites ci-dessous sont des normes :

\begin{enumerate}
    \item si \(A\in\GL{n}[\R]\), l'application \(N\) définie sur \(\R^n\) par \(N\paren{X}=\norme{AX}_2\) ; \\
    \item sur \(E=\ensclasse{1}{\intervii{a}{b}}{\K}\), \(N\paren{f}=\abs{f\paren{a}}+\int_a^b\abs{f\prim\paren{t}}\odif{t}\) ; \\
    \item pour des réels \(\alpha_0<\dots<\alpha_n\) fixés, l'application \(N\) définie sur \(\polydeg[\R]{n}\) par \(N\paren{P}=\max_{k\in\interventierii{0}{n}}\abs{P\paren{\alpha_k}}\).
\end{enumerate}
\end{exops}

\begin{corr}
\note{À venir}
\end{corr}

\begin{exos}[Exercice 3]
Sur \(E=\R^2\), on définit \[\norme{\paren{x,y}}=\max\paren{\abs{x},\abs{x+2y}}.\]

Démontrez que \(\norme{\cdot}\) définit une norme sur \(\R^2\). Représentez graphiquement la boule-unité.
\end{exos}

\begin{corr}
\note{À venir}
\end{corr}

\begin{exopss}[Exercice 4]
Les normes définies dans l'\hyperref[exo:1.2]{exercice 2} sont-elles équivalentes à la norme \(\norme{\cdot}_\infty\), la norme \(\norme{\cdot}_1\), la norme \(\norme{\cdot}_2\) ?
\end{exopss}

\begin{corr}
\note{À venir}
\end{corr}

\begin{exops}[Exercice 5]
Soit \(\theta\not\equiv0\croch{\pi}\). Montrez que \(\paren{\sin\paren{n\theta}}\) et \(\paren{\cos\paren{n\theta}}\) divergent.
\end{exops}

\begin{corr}
\note{À venir}
\end{corr}

\begin{exops}[Exercice 6]
Soit \(A\in\M{n}\). On suppose que la suite \(\paren{A^k}\) converge. Montrez que sa limite est la matrice d'un projecteur.
\end{exops}

\begin{corr}
\note{À venir}
\end{corr}

\begin{exoss}[Exercice 7]
Cet exercice prolonge le premier.

Soit \(u\) une suite vérifiant la propriété \[\quantifs{\forall\epsilon>0;\exists N\in\N;\forall\paren{n,p}\in\N^2}p\geq n\geq N\imp\abs{u_n-u_p}\leq\epsilon.\]

Montrez que \(u\) est bornée, puis en vous servant des suites \(a\) et \(b\) définies comme ci-dessus, montrez que \(u\) converge.

Note culturelle : on dit que \(u\) est une suite de Cauchy et on a donc montré que toute suite de Cauchy converge.
\end{exoss}

\begin{corr}
\note{À venir}
\end{corr}

\begin{exopss}[Exercice 8]
On note \(l_1\) l'ensemble des suites réelles \(\paren{u_n}\) telles que la série \(\sum u_k\) soit absolument convergente, et on pose : \[\quantifs{\forall u=\paren{u_k}\in l_1}N\paren{u}=\sum_{k=0}^{\pinf}\abs{u_k}.\]

\begin{enumerate}
    \item Justifiez que \(l_1\) est un \(\R\)-espace vectoriel. \\
    \item Montrez que \(N\) est une norme sur \(l_1\). On notera désormais \(\norme{u}_1\) pour \(N\paren{u}\). \\
    \item Justifiez que, si \(\paren{u\deriv{n}}_{n\in\N}\) est une suite d'éléments de \(l_1\) convergeant vers la suite \(a\in l_1\) pour la norme 1, alors \[\quantifs{\forall k\in\N}u_k\deriv{n}\tendqd{n\to\pinf}a_k.\]
    \item Montrez que la réciproque est fausse. \\ \textit{Indication :} on pourra étudier l'exemple où, \(\quantifs{\tpt\paren{n,k}\in\N^2}u_k\deriv{n}=\exp\paren{\dfrac{-k}{n+1}}\).
\end{enumerate}
\end{exopss}

\begin{corr}
\note{À venir}
\end{corr}

\begin{exops}[Exercice 9]
Soit \(E\) le \(\C\)-espace vectoriel des suites bornées muni de la norme infinie.

Montrez que les applications \(u\mapsto\paren{u_{n+1}-u_n}\) et \(u\mapsto\dfrac{u_0+\dots+u_n}{n+1}\) sont des applications continues de \(E\) dans \(E\).
\end{exops}

\begin{corr}
\note{À venir}
\end{corr}

\begin{exopss}[Exercice 10]
Soit \(E=\M{n}\). Pour \(A=\paren{a_{ij}}\) on pose \(\norme{A}=\max_{1\leq i\leq n}\sum_{j=1}^n\abs{a_{ij}}\).

\begin{enumerate}
    \item Montrez que \(\norme{\cdot}\) est une norme sur \(E\). \\
    \item Montrez que l'application \(A\mapsto\trans{A}\) est un endomorphisme continu et déterminez sa norme subordonnée.
\end{enumerate}
\end{exopss}

\begin{corr}
\note{À venir}
\end{corr}

\begin{exoss}[Exercice 11]
Soit \(E=\ensclasse{0}{\intervii{0}{1}}{\R}\). On pose \(A=\accol{f\in E\tq f\geq0}\) et pour \(f\in E\), \(\norme{f}_1=\int_0^1\abs{f}\).

\begin{enumerate}
    \item Montrez que \(\norme{\cdot}_1\) est une norme sur \(E\). \\
    \item Déterminez \(\interieur{A}\) dans \(\groupe{E}[\norme{\cdot}_\infty]\), puis dans \(\groupe{E}[\norme{\cdot}_1]\). \\
    \item On pose \(D=\fami{D}^1\paren{\intervii{0}{1},\R}\) le sous-espace des fonctions dérivables et \(P\) le sous-espace des fonctions polynômes. \\ Déterminez les intérieurs de \(P\) et \(D\) dans \(\groupe{E}[\norme{\cdot}_\infty]\).
\end{enumerate}
\end{exoss}

\begin{corr}
\note{À venir}
\end{corr}

\begin{exoss}[Exercice 12]
Cet exercice prolonge le précédent, les notations sont reprises.

Soit \(u:E\to E\) qui à toute fonction \(f\) de \(E\) associe sa primitive qui s'annule en \(0\). Vérifiez que \(u\) est un endomorphisme de \(E\).

Est-il continu de \(\groupe{E}[\norme{\cdot}_?]\) dans \(\groupe{E}[\norme{\cdot}_?]\) (vous étudierez les quatre possibilités) ? Quand c'est le cas, déterminez la norme subordonnée de \(u\).
\end{exoss}

\begin{corr}
\note{À venir}
\end{corr}

\begin{exopss}[Exercice 13]
Soit \(E=\accol{f\in\ensclasse{1}{\intervii{0}{1}}{\R}\tq f\paren{0}=f\paren{1}=0}\). Pour \(f\in E\), on pose \(\norme{f}=\sup_{\intervii{0}{1}}\abs{f\prim}\).

\begin{enumerate}
    \item Montrez que \(\norme{\cdot}\) est une norme sur \(E\). \\
    \item Soit \(\phi:E\to\R\) définie par \(\phi\paren{f}=\int_0^1f\). Montrez que \(\phi\) est continue et déterminez \(\normesub{f}\).
\end{enumerate}
\end{exopss}

\begin{corr}
\note{À venir}
\end{corr}

\begin{exopss}[Exercice 14]
Soient \(E\) un espace vectoriel normé et \(F\) un sous-espace vectoriel de \(E\).

\begin{enumerate}
    \item Montrez que si \(F\not=E\), alors \(\interieur{F}=\ensvide\) et \(\conj{F}\) est un sous-espace vectoriel de \(E\). \\
    \item Montrez que si \(F\) est un hyperplan, alors \(F\) est fermé ou dense dans \(E\).
\end{enumerate}
\end{exopss}

\begin{corr}
\note{À venir}
\end{corr}

\begin{exops}[Exercice 15]
Soient \(E\) un espace vectoriel normé et \(A\) une partie de \(E\).

Montrez que \(\interieur{A}\) est le plus grand ouvert inclus dans \(A\) et \(\conj{A}\) est le plus petit fermé contenant \(A\). Montrez que la frontière de \(A\) est un fermé d'intérieur vide.
\end{exops}

\begin{corr}
\note{À venir}
\end{corr}

\begin{exos}[Exercice 16]
Une intersection d'ouverts est-elle toujours un ouvert ? Une réunion de fermés est-elle toujours un fermé ?
\end{exos}

\begin{corr}
\note{À venir}
\end{corr}

\begin{exops}[Exercice 17]
Montrez que si \(A\) est une partie convexe d'un espace vectoriel normé \(E\), alors il en est de même pour \(\conj{A}\) et \(\interieur{A}\).
\end{exops}

\begin{corr}
\note{À venir}
\end{corr}

\begin{exoss}[Exercice 18]
Soient \(E\) un espace vectoriel normé, \(A\) une partie de \(E\) et \(x\in E\). On dit que \(x\) est un point d'accumulation de \(A\) quand il existe une suite injective de \(A^\N\) qui converge vers \(x\). On dit que \(x\) est un point isolé de \(A\) quand il existe \(r>0\) tel que \(\bouleo{x}{r}\inter A=\accol{x}\).

\begin{enumerate}
    \item Exemples. On pose \(A=\accol{\dfrac{1}{n}\tq n\in\Ns}\) dans \(\R\) : montrez que tous les points de \(A\) sont isolés, que le seul point d'accumulation de \(A\) est \(0\) et que \(A\) n'est pas fermé. On pose \(B=\accol{\dfrac{1}{n}+\dfrac{1}{p}\tq\paren{n,p}\in\paren{\Ns}^2}\) : quels sont les points d'accumulation de \(B\) ? \\
    \item Montrez que \(x\) est un point d'accumulation ssi \(\quantifs{\tpt r>0}\bouleo{x}{r}\inter A\text{ est un ensemble infini}\). \\
    \item On note \(A\prim\) l'ensemble des points d'accumulation de \(A\) et \(A^d\) l'ensemble des points isolés dans \(A\). Montrez que \(\conj{A}=A\prim\sqcup A^d\). \\
    \item Montrez que \(A\prim\) est un fermé.
\end{enumerate}
\end{exoss}

\begin{corr}
\note{À venir}
\end{corr}

\begin{exoss}[Exercice 19]
Soient \(E,F\) deux espaces vectoriels normés et \(f:E\to F\). Montrez l'équivalence entre les propositions :

\begin{enumerate}
    \item \(f\) est continue \\
    \item \(\quantifs{\forall A\in\P{E}}f\paren{\conj{A}}\subset\conj{f\paren{A}}\) \\
    \item \(\quantifs{\forall B\in\P{F}}f\inv\paren{\interieur{B}}\subset\interieur{\overbrace{f\inv\paren{B}}}\)
\end{enumerate}
\end{exoss}

\begin{corr}
\note{À venir}
\end{corr}

\begin{exopss}[Exercice 20]
Soient \(A,B\) deux fermés disjoints d'un espace vectoriel normé \(E\).

\begin{enumerate}
    \item Montrez que \(\accol{x\in E\tq d\paren{x,A}>d\paren{x,B}}\) est un ouvert. \\
    \item Montrez qu'il existe deux ouverts disjoints \(U\) et \(V\) tels que \(A\subset U\) et \(B\subset V\).
\end{enumerate}
\end{exopss}

\begin{corr}
\note{À venir}
\end{corr}

\begin{exopss}[Exercice 21]
Soit \(A\) une partie d'un espace vectoriel normé \(E\). Pour \(r>0\), on pose \(V\paren{A,r}=\accol{x\in E\tq d\paren{x,A}<r}\).

Montrez que \(V\paren{A,r}\) est un ouvert de \(E\) et \(\biginter_{r>0}V\paren{A,r}=\conj{A}\).
\end{exopss}

\begin{corr}
\note{À venir}
\end{corr}

\begin{exopss}[Exercice 22]
Soient \(E\) un espace vectoriel normé, \(K\) un compact de \(E\), \(k\in\intervie{0}{1}\) et \(f:K\to K\) telle que \[\quantifs{\forall\paren{x,y}\in K^2}\norme{f\paren{x}-f\paren{y}}\leq k\norme{x-y}.\]

Soit \(u\) la suite définie par \(u_0\) quelconque dans \(K\) et \(u_{n+1}=f\paren{u_n}\).

Montrez que \(u\) converge et que sa limite est l'unique point fixe de \(f\).
\end{exopss}

\begin{corr}
\note{À venir}
\end{corr}

\begin{exoss}[Exercice 23]
Soient \(E\) un espace vectoriel normé et \(A\) une partie non-vide de \(E\). On appelle diamètre de \(A\), noté \(\delta\paren{A}\), la borne supérieure dans \(\Rp\union\accol{\pinf}\) des \(\norme{x-y}\) quand \(\paren{x,y}\in A^2\).

\begin{enumerate}
    \item Montrez que \(\delta\paren{A}<\pinf\) ssi \(A\) est bornée. \\
    \item Quel est le diamètre d'une boule ? \\
    \item Montrez que si \(A\) est compacte, alors il existe \(\paren{a,b}\in A^2\) tel que \(\delta\paren{A}=\norme{a-b}\). Est-ce encore vrai si on suppose seulement \(A\) bornée ? \(A\) fermée ?
\end{enumerate}
\end{exoss}

\begin{corr}
\note{À venir}
\end{corr}

\begin{exopss}[Exercice 24]
Soient \(E\) un espace vectoriel normé et \(A,B\) deux parties non-vides de \(E\). On pose \(d\paren{A,B}=\inf_{\paren{a,b}\in A\times B}\norme{a-b}\), appelé distance de \(A\) à \(B\).

\begin{enumerate}
    \item Montrez que si \(d\paren{A,B}>0\), alors \(A\) et \(B\) sont disjointes, mais que la réciproque est fausse. \\
    \item Montrez que si \(A\) et \(B\) sont compactes, alors \(d\paren{A,B}\) est en fait un minimum plutôt qu'une borne inférieure. \\
    \item Montrez que ce résultat reste vrai si \(E\) est de dimension finie, l'une des deux parties est compacte et l'autre fermée. \\
    \item Est-ce encore vrai si on suppose seulement \(A\) et \(B\) fermées ?
\end{enumerate}
\end{exopss}

\begin{corr}
\note{À venir}
\end{corr}

\begin{exopss}[Exercice 25]
Soient \(E\) un espace vectoriel normé de dimension finie et \(\paren{B_n=\boulef{a_n}{r_n}}\) une suite de boules fermées, décroissantes pour l'inclusion, telle que \(r_n\to0\).

\begin{enumerate}
    \item Montrez que la suite \(\paren{a_n}\) admet une sous-suite convergeant vers un vecteur \(a\). \\
    \item Montrez que \(a_n\to a\). \\
    \item Montrez que \(\biginter_{n\in\N}B_n=\accol{a}\).
\end{enumerate}
\end{exopss}

\begin{corr}
\note{À venir}
\end{corr}

\begin{exoss}[Exercice 26]
Soit \(E\) un \(\K\)-espace vectoriel de dimension finie. Montrez que l'ensemble des projecteurs est fermé dans \(\Lendo{E}\). Est-il borné ? Compact ? Connexe par arcs ?
\end{exoss}

\begin{corr}
\note{À venir}
\end{corr}

\begin{exoss}[Exercice 27]
Soient \(f:\R\to\R\) continue et \(\epsilon>0\). On pose \(E=\accol{\paren{x,y}\in\R^2\tq\abs{f\paren{x}-y}\leq\epsilon}\).

\begin{enumerate}
    \item Montrez que \(E\) est connexe par arcs. \\
    \item Montrez que si \(f\) est une fonction affine, alors \(E\) est une partie convexe. \\
    \item Montrez que la réciproque est vraie.
\end{enumerate}
\end{exoss}

\begin{corr}
\note{À venir}
\end{corr}

\begin{exoss}[Exercice 28]
Soient \(E\) un \(\R\)-espace vectoriel de dimension finie et \(H\) un hyperplan de \(E\).

\begin{enumerate}
    \item Montrez que \(E\excluant H\) possède deux composantes connexes par arcs qui sont ouvertes. \\
    \item Soit \(B\) une partie de \(H\) telle que \(H\not=B\). Montrez que \(E\excluant B\) est connexe par arcs.
\end{enumerate}
\end{exoss}

\begin{corr}
\note{À venir}
\end{corr}

\begin{exoss}[Exercice 29]
Soient \(E\) un espace vectoriel normé et \(A,B\) deux parties de \(E\) telles que \(B\) est connexe par arcs et \(B\) rencontre à la fois \(A\) et \(E\excluant A\).

Montrez que \(B\) rencontre la frontière de \(A\).
\end{exoss}

\begin{corr}
\note{À venir}
\end{corr}

\begin{exoss}[Exercice 30]
Deux parties d'un espace vectoriel normé sont dites homéomorphes quand il existe une bijection continue de l'une dans l'autre telle que la réciproque soit aussi continue.

\begin{enumerate}
    \item Montrez que tout intervalle ouvert est homéomorphe à \(\R\). \\
    \item Montrez qu'un intervalle qui contient l'une de ses bornes réelles ne peut pas être homéomorphe à \(\R\). \\
    \item Montrez que toute boule ouverte d'un espace vectoriel normé \(E\) est homéomorphe à \(E\). \\
    \item Montrez qu'aucune boule fermée de \(E\) n'est homéomorphe à \(E\).
\end{enumerate}
\end{exoss}

\begin{corr}
\note{À venir}
\end{corr}

\begin{exosss}[Exercice 31]
Soit \(G\) un sous-groupe de \(\groupe{\R}\), autre que \(\accol{0}\).

\begin{enumerate}
    \item Montrez que \(a=\inf G\inter\Rps\) existe. \\
    \item Montrez que \(G=a\Z\) si \(a>0\) ou \(G\) est dense dans \(\R\) si \(a=0\). \\
    \item On pose \(G=\Z+\sqrt{2}\Z\) (\(G\) est le sous-groupe de \(\groupe{\R}\) engendré par \(1\) et \(\sqrt{2}\)) et \(r=\sqrt{2}-1\). En considérant la suite \(\paren{r^n}\), montrez que \(G\) est dense dans \(\R\). \\ Soit \(f:\R\to\R\) continue et périodique de périodes \(1\) et \(\sqrt{2}\). Que peut-on dire de \(f\) ? \\
    \item Soient \(a,b\) deux réels distincts et non-nuls, on pose \(G=a\Z+b\Z\). Montrez que \(G\) est un sous-groupe de \(\groupe{\R}\), puis que \(G\) est dense dans \(\R\) ssi \(\dfrac{a}{b}\) est un rationnel. \\ Application : montrez que les ensembles \(\accol{\cos n\tq n\in\N}\) et \(\accol{\sin n\tq n\in\N}\) sont denses dans \(\intervii{-1}{1}\).
\end{enumerate}
\end{exosss}

\begin{corr}
\note{À venir}
\end{corr}

\begin{exosss}[Exercice 32]
Soit \(E=\ensclasse{0}{\intervii{0}{1}}{\R}\) muni de la norme infinie \(\norme{f}=\sup_{\intervii{0}{1}}\abs{f}\). On note \(\boulef{0}{1}\) la boule-unité fermée.

Soit \(\paren{t_n}\) une suite injective à valeurs dans \(\intervii{0}{1}\). Pour \(f\in E\), on pose \(L\paren{f}=\sum_{n=1}^{\pinf}\paren{-1}^n\dfrac{f\paren{t_n}}{2^n}\).

\begin{enumerate}
    \item Montrez que \(L\) est une forme linéaire continue sur \(E\). \\
    \item Déterminez \(K=\sup_{f\in\boulef{0}{1}}\abs{L\paren{f}}\). \\
    \item Montrez que si la suite \(\paren{t_n}\) converge ou si elle est dense dans \(\intervii{0}{1}\), \(K\) n'est pas atteinte. \\
    \item Donnez un exemple de suite \(\paren{t_n}\) pour laquelle \(K\) est atteinte. Donnez une condition nécessaire et suffisante sur la suite \(t\) pour que \(K\) soit atteinte.
\end{enumerate}
\end{exosss}

\begin{corr}
\note{À venir}
\end{corr}

\begin{exosss}[Exercice 33]
Soient \(E\) un espace vectoriel normé et \(u\) une forme linéaire non-nulle et continue sur \(E\). On pose \(H=\ker u\) et \(K=\sup_{x\in E\excluant\accol{0}}\dfrac{\abs{u\paren{x}}}{\norme{x}}\).

\begin{enumerate}
    \item Justifiez l'existence de \(K\). \\
    \item Montrez que \(\quantifs{\tpt a\in E}d\paren{a,H}=\dfrac{\abs{u\paren{a}}}{K}\).
\end{enumerate}
\end{exosss}

\begin{corr}
\note{À venir}
\end{corr}

\begin{exo}[Oral CCMP, 1]
Soient \(E\) un espace vectoriel normé réel et \(B\) sa boule-unité ouverte. Montrez que \(E\) et \(B\) sont homéomorphes (\ie il existe une bijection de \(E\) dans \(B\) qui est continue et dont la réciproque est aussi continue).
\end{exo}

\begin{corr}
\note{À venir}
\end{corr}

\begin{exo}[Oral CCMP, 2]
Soient \(E\) un espace vectoriel normé réel et \(C,D\) deux parties de \(E\) telles que \(C\subset D\subset\conj{C}\) et \(C\) convexe. Montrez que \(D\) est connexe par arcs.
\end{exo}

\begin{corr}
\note{À venir}
\end{corr}

\begin{exo}[Oral CCMP, 3]
Soient \(E\) un espace vectoriel normé réel, \(K\) un compact de \(E\) et \(f:K\to K\) telle que \[\quantifs{\forall\paren{x,y}\in K^2}x\not=y\imp\norme{f\paren{x}-f\paren{y}}<\norme{x-y}.\]

\begin{enumerate}
    \item Montrez que \(f\) possède un unique point fixe. \\
    \item Soit \(u\) la suite définie par \(u_0\) quelconque dans \(K\) et \(u_{n+1}=f\paren{u_n}\). Montrez que \(u\) converge vers le point fixe de \(f\).
\end{enumerate}
\end{exo}

\begin{corr}
\note{À venir}
\end{corr}

\begin{exo}[Oral CCMP, 4]
Soit \(u\) une suite réelle bornée telle que \(u_{n+1}-u_n\tendqd{n\to\pinf}0\). Montrez que l'ensemble de ses valeurs d'adhérence est un intervalle.
\end{exo}

\begin{corr}
\note{À venir}
\end{corr}

\begin{exo}[Oral Centrale, 5]
Soit \(G\) un sous-groupe de \(\groupe{\Cs}[\times]\) tel que pour tout \(g\in G\) il existe un voisinage \(V\) de \(g\) tel que \(G\inter V=\accol{g}\).

\begin{enumerate}
    \item Montrez que pour tout compact \(K\) de \(\Cs\), \(G\inter K\) est fini. \\
    \item Montrez que \(G\inter\U\) est un groupe cyclique. \\
    \item On suppose que \(G\) n'est pas contenu dans \(\U\). Soit \(A=\accol{\abs{x}\tq x\in G\text{ et }\abs{x}>1}\). Montrez que \(A\) possède un plus petit élément. Déduisez-en \(G\).
\end{enumerate}
\end{exo}

\begin{corr}
\note{À venir}
\end{corr}

\begin{exo}[Oral Centrale, 6]
Soient \(E,F\) deux espaces vectoriels normés de dimensions finies et \(f:E\to F\). On dit que \(f\) est propre quand pour tout compact \(K\) de \(F\), \(f\inv\paren{K}\) est un compact de \(E\).

\begin{enumerate}
    \item Montrez que si \(f\) est propre, alors l'image d'un fermé de \(E\) est un fermé de \(F\). \\
    \item Montrez que \(f\) est propre ssi \(\norme{f\paren{x}}\tendqd{\norme{x}\to\pinf}\pinf\).
\end{enumerate}
\end{exo}

\begin{corr}
\note{À venir}
\end{corr}

\begin{exo}[Oral X, 7]
Soient \(n\in\Ns\), \(U_n\) l'ensemble des polynômes réels unitaires de degré \(n\) et \(A_n\) l'ensemble des polynômes de \(U_n\) qui sont simplement scindés (\ie ayant \(n\) racines réelles distinctes).

\begin{enumerate}
    \item Montrez que \(A_n\) est un ouvert de \(U_n\). \\
    \item Déterminez l'adhérence de \(A_n\).
\end{enumerate}
\end{exo}

\begin{corr}
\note{À venir}
\end{corr}

\chapter{Séries numériques et vectorielles : révisions et compléments}

\legendeexercices

\begin{exos}[Exercice 1]
\begin{enumerate}
    \item Montrez que la série de terme général \(u_n=\Arctan\dfrac{1}{2n^2}\) est convergente. \\
    \item Montrez que \(\quantifs{\tpt n\geq1}u_n=\Arctan\dfrac{n}{n+1}-\Arctan\dfrac{n-1}{n}\). \\
    \item Déduisez-en la somme \(\sum_{n=1}^{\pinf}u_n\).
\end{enumerate}
\end{exos}

\begin{corr}
\note{À venir}
\end{corr}

\begin{exos}[Exercice 2]~\\
Justifiez que la série \(\sum_{n\geq2}\dfrac{2n-1}{n^3-n}\) converge et déterminez sa somme (indication : décomposition en éléments simples).
\end{exos}

\begin{corr}
\note{À venir}
\end{corr}

\begin{exoss}[Exercice 3]
Donnez la nature des séries suivantes (\(\alpha\) désigne une constante strictement positive, \(x\) un réel dans \(\intervee{-1}{1}\)) :

\begin{enumerate}
    \item \(\sum_{n\geq2}\dfrac{\ln^nn}{n^{\ln n}}\) \\
    \item \(\sum_{n\geq2}\dfrac{1}{\paren{\ln n}^{\ln n}}\) \\
    \item \(\sum\dfrac{\paren{-1}^n}{n^\alpha}\) \\
    \item \(\sum_{n\geq1}\dfrac{\paren{-1}^nn^\alpha}{n!}\) \\
    \item \(\sum\dfrac{\paren{-1}^n\ln n}{n^\alpha}\) \\
    \item \(\sum\ln\paren{1+x^n}\) \\
    \item \(\sum\dfrac{\sin n}{2^n}\) \\
    \item \(\sum2\ln\paren{n^3+1}-3\ln\paren{n^2+1}\) \\
    \item \(\sum\dfrac{\sin n\sin\frac{1}{n}}{n}\) \\
    \item \(\sum_{n\geq2}\dfrac{\paren{-1}^n}{\paren{-1}^n+n}\) \\
    \item \(\sum_{n\geq0}\sin\paren{\pi\sqrt{n^2+1}}\) \\
    \item \(\sum\sqrt[n+1]{n+1}-\sqrt[n]{n}\) \\
    \item \(\sum_{n\geq1}\dfrac{\paren{-1}^n}{\ln n+\sin\frac{2\pi n}{3}}\) \\
    \item \(\sum_{n\geq1}\paren{\paren{1+\dfrac{1}{n}}^n-\e{}}\) \\
    \item \(\sum_{n\geq1}\dfrac{1}{\ln n\ln\paren{\ch n}}\) \\
    \item \(\sum\int_0^{\nicefrac{\pi}{2}}\dfrac{\cos^2t}{n^2+\cos^2t}\odif{t}\)
\end{enumerate}
\end{exoss}

\begin{corr}
\note{À venir}
\end{corr}

\begin{exoss}[Exercice 4]
Soit \(\paren{a,b}\in\R^2\).

\begin{enumerate}
    \item Déterminez \(a\) et \(b\) pour que la série de terme général \(\ln n+a\ln\paren{n+1}+b\ln\paren{n+2}\) converge. Dans ce cas, donnez la valeur de sa somme. \\
    \item Faites de même avec la série de terme général \(\sqrt{n}+a\sqrt{n+1}+b\sqrt{n+2}\).
\end{enumerate}
\end{exoss}

\begin{corr}
\note{À venir}
\end{corr}

\begin{exoss}[Exercice 5]
Pour quelles valeurs de \(\alpha>0\) la série de terme général \(u_n=\ch^\alpha n-\sh^\alpha n\) converge-t-elle ? Dans ce cas, donnez un équivalent de \(\sum_{k=n}^{\pinf}u_k\).
\end{exoss}

\begin{corr}
\note{À venir}
\end{corr}

\begin{exoss}[Exercice 6]~\\
On pose \(u_n=\dfrac{1}{n^\alpha}\sum_{k=1}^nk^{\nicefrac{3}{2}}\). Pour quelles valeurs de \(\alpha>0\) la série \(\sum_{n\geq1}u_n\) converge-t-elle ? Dans ce cas, donnez un équivalent de \(\sum_{k=n+1}^{\pinf}u_k\).
\end{exoss}

\begin{corr}
\note{À venir}
\end{corr}

\begin{exoss}[Exercice 7, séries associées à des suites définies par récurrence]~\\
\begin{enumerate}
    \item Soit \(u\) la suite définie par récurrence par \(u_1\in\R\) et \(\quantifs{\tpt n\geq1}u_{n+1}=\dfrac{1}{n\e{u_n}}\). \\ Quelle est la nature de la série \(\sum_{n\geq1}u_n\) ? \\
    \item Soit \(u\) la suite définie par récurrence par \(u_0\in\Rps\) et \(\quantifs{\tpt n\geq0}u_{n+1}=\ln\paren{1+u_n}\). \\ Quelle est la nature de la série \(\sum\paren{\dfrac{1}{u_{n+1}}-\dfrac{1}{u_n}}\) ? Puis celle de \(\sum u_n\) ? Donnez un équivalent de \(\sum_{k=0}^nu_k\). \\
    \item Soit \(u\) la suite définie par récurrence par \(u_0\in\intervee{0}{\pi}\) et \(\quantifs{\tpt n\geq0}u_{n+1}=\sin u_n\). \\ Quelle est la nature de la série \(\sum\paren{\dfrac{1}{u_{n+1}^2}-\dfrac{1}{u_n^2}}\) ? Puis celle de \(\sum u_n\) ? Donnez un équivalent de \(\sum_{k=0}^n\).
\end{enumerate}
\end{exoss}

\begin{corr}
\note{À venir}
\end{corr}

\begin{exoss}[Exercice 8]
Soit \(P\in\poly[\R]\) tel que \(\quantifs{\tpt x\in\Rp}P\paren{x}\geq0\). Pour \(n\in\Ns\), on pose \(u_n=\sum_{k=n}^{\pinf}\dfrac{\paren{-1}^k}{\e{k}+P\paren{k}}\).

\begin{enumerate}
    \item Justifiez l'existence de \(u_n\). \\
    \item Montrez que la série \(\sum_{n\geq1}u_n\) converge.
\end{enumerate}
\end{exoss}

\begin{corr}
\note{À venir}
\end{corr}

\begin{exoss}[Exercice 9]~\\
Pour \(n\geq1\), on pose \(u_n=\sum_{k=n}^{\pinf}\dfrac{\paren{-1}^k}{\sqrt{k}}\).

\begin{enumerate}
    \item Justifiez l'existence de \(u_n\). \\
    \item Montrez que \(\dfrac{u_n+u_{n+1}}{2}\) est le reste d'une série alternée absolument convergente. \\
    \item Déduisez-en la nature de la série \(\sum u_n\).
\end{enumerate}
\end{exoss}

\begin{corr}
\note{À venir}
\end{corr}

\begin{exoss}[Exercice 10, utilisation de développements limités ou asymptotiques]~\\
\begin{enumerate}
    \item Montrez que la série \(\sum_{n\geq2}\dfrac{\paren{-1}^n}{n-\sqrt{n}}\) converge. \\
    \item Montrez que la série \(\sum_{n\geq2}\ln\paren{1+\dfrac{\paren{-1}^n}{n}}\) converge. \\
    \item Montrez que la série \(\sum_{n\geq2}\dfrac{1}{1+\paren{-1}^nn}\) converge. \\
    \item Montrez que la série \(\sum_{n\geq0}\sin\paren{\pi\sqrt{n^2+1}}\) converge. \\
    \item Déterminez la nature de la série \(\sum_{n\geq2}\paren{-1}^n\dfrac{\ln\paren{n+\paren{-1}^n\sqrt{n}}}{n}\). \\
    \item Déterminez la nature de la série \(\sum_{n\geq2}\dfrac{1}{1+\paren{-1}^n\sqrt{n}}\).
\end{enumerate}
\end{exoss}

\begin{corr}
\note{À venir}
\end{corr}

\begin{exoss}[Exercice 11]
Pour quelles valeurs de \(\alpha>0\) la série \(\sum_{n\geq2}\dfrac{\paren{-1}^n}{\sqrt{\paren{-1}^n+n^\alpha}}\) converge-t-elle ?
\end{exoss}

\begin{corr}
\note{À venir}
\end{corr}

\begin{exoss}[Exercice 12, formule de Stirling]~\\
Montrez que la suite de terme général \(\dfrac{n!}{\sqrt{n}}\paren{\dfrac{\e{}}{n}}^n\) converge vers un réel strictement positif \(L\) (indication : passer au logarithme et penser à une série).

Soit \(u_n=\int_0^{\nicefrac{\pi}{2}}\cos^nt\odif{t}\). On montre que \(u_n\simqd{n\to\pinf}\sqrt{\dfrac{\pi}{2n}}\) et que \(u_{2n}=\dfrac{\pi\paren{2n}!}{2^{2n+1}\paren{n!}^2}\). En admettant ces résultats, montrez la formule de Stirling : \[n!\simqd{n\to\pinf}\sqrt{2\pi n}\paren{\dfrac{n}{\e{}}}^n.\]
\end{exoss}

\begin{corr}
\note{À venir}
\end{corr}

\begin{exoss}[Exercice 13]~\\
Soient \(a\in\R\) et \(u_n=\dfrac{\paren{-1}^{\frac{n\paren{n+1}}{2}}}{n^a}\).

\begin{enumerate}
    \item Dans le cas où \(a\leq0\) ou \(a>1\), quelle est la nature de la série \(\sum_{n\geq1}u_n\) ? \\
    \item On suppose désormais que \(0<a\leq1\) et on pose \(v_n=u_{2n-1}+u_{2n}\). Montrez que la série \(\sum v_n\) converge. Déduisez-en la nature de la série \(\sum u_n\).
\end{enumerate}
\end{exoss}

\begin{corr}
\note{À venir}
\end{corr}

\begin{exoss}[Exercice 14]
Soient \(u\) une suite strictement positive et \(\alpha>0\).

Montrez que les séries de termes généraux \(u_n\), \(v_n=\dfrac{u_n}{1+u_n}\), \(w_n=\ln\paren{1+u_n}\) et \(x_n=\int_0^{u_n}\dfrac{1}{1+x^\alpha}\odif{x}\) sont de même nature.
\end{exoss}

\begin{corr}
\note{À venir}
\end{corr}

\begin{exoss}[Exercice 15]
Soit \(u\) une suite réelle qui ne s'annule pas telle que \(\dfrac{u_{2n+1}}{u_{2n}}\tendqd{n\to\pinf}a\) et \(\dfrac{u_{2n}}{u_{2n-1}}\tendqd{n\to\pinf}b\).

Montrez que si \(\abs{ab}<1\), alors la série \(\sum u_n\) converge.
\end{exoss}

\begin{corr}
\note{À venir}
\end{corr}

\begin{exoss}[Exercice 16]
Soit \(u\) une suite réelle positive décroissante.

Montrez que si \(\sum u_n\) converge, alors \(u_n=\o{\dfrac{1}{n}}\) quand \(n\) tend vers \(\pinf\).

La réciproque est-elle vraie ?
\end{exoss}

\begin{corr}
\note{À venir}
\end{corr}

\begin{exoss}[Exercice 17]
Soit \(\paren{u_n}_{n\geq1}\) une suite réelle strictement positive et bornée telle que la série \(\sum_{n\geq1}u_n\) diverge. Pour \(n\in\Ns\), on pose \(S_n\) la somme partielle d'indice \(n\) de la série \(\sum_{n\geq1}u_n\).

\begin{enumerate}
    \item Montrez que \(\dfrac{u_n}{S_n}\sim\ln\dfrac{S_n}{S_{n+1}}\). Déduisez-en la nature de la série \(\sum_{n\geq1}\dfrac{u_n}{S_n}\). \\
    \item Étudiez la série \(\sum_{n\geq1}\dfrac{u_n}{S_n^\alpha}\) quand \(\alpha\in\intervee{0}{1}\). \\
    \item Soit \(\alpha>1\). Montrez que \(\dfrac{u_n}{S_n^\alpha}\leq\int_{S_{n-1}}^{S_n}\dfrac{1}{x^\alpha}\odif{x}\). Déduisez-en la nature de la série \(\sum_{n\geq1}\dfrac{u_n}{S_n^\alpha}\).
\end{enumerate}
\end{exoss}

\begin{corr}
\note{À venir}
\end{corr}

\begin{exoss}[Exercice 18]
Soit \(u\) une suite strictement positive. Pour \(n\in\N\), on pose \(s_n=\sum_{k=0}^nu_k\).

On suppose que \(u_ns_n\) tend vers \(1\) quand \(n\) tend vers \(\pinf\). Déterminez un équivalent simple de \(u_n\).
\end{exoss}

\begin{corr}
\note{À venir}
\end{corr}

\begin{exoss}[Exercice 19]
Soit \(u\) la suite définie par récurrence par \(u_0>0\) et \(\quantifs{\tpt n\in\N}u_{n+1}=u_n+u_n^2\). On pose \(v_n=\dfrac{1}{2^n}\ln u_n\), puis \(w_n=v_{n+1}-v_n\).

\begin{enumerate}
    \item Montrez que la suite \(\paren{u_n}\) diverge vers \(\pinf\). \\
    \item Montrez que la suite \(\paren{v_n}\) converge vers un réel \(l>0\). On pose alors \(A=\e{l}>1\). \\
    \item Montrez que \(u_n\sim A^{2^n}\).
\end{enumerate}
\end{exoss}

\begin{corr}
\note{À venir}
\end{corr}

\begin{exosss}[Exercice 20, transformation d'Abel]
Soient \(u\) une suite réelle et \(v\) une suite complexe. Pour \(n\in\N\), on pose \(V_n=\sum_{k=0}^nv_k\).

On suppose que la suite \(u\) est positive et décroissante de limite nulle et que la suite \(V\) est bornée.

\begin{enumerate}[series=transformationAbel]
    \item Montrez que \(\quantifs{\tpt n\in\N}\sum_{k=0}^nu_kv_k=u_nV_n-\sum_{k=0}^{n-1}\paren{u_{k+1}-u_k}V_k\). \\
    \item Déduisez-en que la série \(\sum u_nv_n\) converge.
\end{enumerate}

Applications :

\begin{enumerate}[resume=transformationAbel]
    \item Soit \(w\) une suite complexe telle que \(\sum w_n\) converge. Montrez que \(\quantifs{\tpt a>0}\sum\dfrac{w_n}{n^a}\) converge aussi. \\
    \item Soient \(a>0\) et \(\theta\in\R\). Donnez la nature des séries \(\sum\dfrac{\e{\i n\theta}}{n^a}\), \(\sum\dfrac{\cos\paren{n\theta}}{n^a}\) et \(\sum\dfrac{\sin\paren{n\theta}}{n^a}\). \\
    \item Montrez que \(\quantifs{\tpt x\in\R}\abs{\sin x}\geq\dfrac{1-\cos\paren{2x}}{2}\). Déterminez la nature des séries \(\sum\dfrac{\abs{\cos\paren{n\theta}}}{n^a}\) et \(\sum\dfrac{\abs{\sin\paren{n\theta}}}{n^a}\).
\end{enumerate}
\end{exosss}

\begin{corr}
\note{À venir}
\end{corr}

\begin{exosss}[Exercice 21]
Soit \(u\) une suite positive de limite nulle. On appelle \(U_n\) la somme partielle d'indice \(n\) de la série \(\sum u_n\) et on suppose qu'il existe une constante \(M>0\) telle que \(\quantifs{\tpt n\in\N}\abs{U_n-nu_n}\leq M\).

\begin{enumerate}
    \item Montrez que \(\quantifs{\tpt n\geq2}\abs{\dfrac{U_n}{n}-\dfrac{U_{n-1}}{n-1}}\leq M\paren{\dfrac{1}{n-1}-\dfrac{1}{n}}\). \\
    \item Montrez que la série \(\sum u_n\) converge.
\end{enumerate}
\end{exosss}

\begin{corr}
\note{À venir}
\end{corr}

\begin{exosss}[Exercice 22]
Soit \(\sum_{n\geq1}u_n\) une série convergente à termes positifs.

\begin{enumerate}
    \item Montrez que \(\dfrac{u_1+2u_2+\dots+nu_n}{n}\) tend vers \(0\) quand \(n\) tend vers \(\pinf\). \\
    \item Montrez que la série \(\sum\dfrac{u_1+2u_2+\dots+nu_n}{n\paren{n+1}}\) converge et montrez que sa somme est la même que celle de la série \(\sum_{n\geq1}u_n\).
\end{enumerate}
\end{exosss}

\begin{corr}
\note{À venir}
\end{corr}

\begin{exo}[Oral Saint-Cyr, 1]~\\
Pour \(n\in\Ns\), on pose \(u_n=\dfrac{\ln n}{n}\). Déterminez la nature de la série \(\sum u_n\). Donnez un équivalent de \(\sum_{k=1}^nu_k\) quand \(n\to\pinf\).
\end{exo}

\begin{corr}
\note{À venir}
\end{corr}

\begin{exo}[Oral IMT, 2]
Soit \(\alpha>0\). Donnez un équivalent de \(\sum_{k=1}^n\ln^\alpha k\) quand \(n\to\pinf\).
\end{exo}

\begin{corr}
\note{À venir}
\end{corr}

\begin{exo}[Oral CCINP, 3]
Soit \(u\) la suite définie par récurrence par \(u_0>0\) et \(\quantifs{\tpt n\in\N}u_{n+1}=u_n\e{-u_n}\).

\begin{enumerate}
    \item Montrez que \(\paren{u_n}\) converge et déterminez sa limite. \\
    \item Déterminez la limite de \(\dfrac{1}{u_{n+1}}-\dfrac{1}{u_n}\). Déduisez-en un équivalent de \(u_n\).
\end{enumerate}
\end{exo}

\begin{corr}
\note{À venir}
\end{corr}

\begin{exo}[Oral CCINP, 4]
Montrez que pour \(n\geq1\), l'équation \(x^n+\sqrt{n}x-1=0\) admet une unique solution \(x_n\) dans \(\intervii{0}{1}\).

Étudiez la suite \(\paren{x_n}\) et montrez qu'elle converge vers \(0\).

Trouvez un équivalent de \(x_n\) et étudiez la convergence de la série \(\sum_{n\geq1}x_n\).
\end{exo}

\begin{corr}
\note{À venir}
\end{corr}

\begin{exo}[Oral CCINP, 5]
Montrez que la suite \(\paren{u_n}\) définie par \(u_0\in\intervie{0}{1}\) et \(u_{n+1}=\dfrac{1}{2}\paren{u_n+u_n^2}\) converge vers \(0\) et donnez la nature de la série \(\sum u_n\).
\end{exo}

\begin{corr}
\note{À venir}
\end{corr}

\begin{exo}[Oral CCINP, 6]
Quelle est la nature de la série de terme général \(\ln\paren{1+\dfrac{\paren{-1}^n}{\sqrt{n\paren{n+1}}}}\) ?
\end{exo}

\begin{corr}
\note{À venir}
\end{corr}

\begin{exo}[Oral CCINP, 7]
Soient \(x,y>0\). Représentez graphiquement l'ensemble des couples \(\paren{x,y}\) tels que la série \(\sum\dfrac{x^n}{y^n+n^x}\) converge.
\end{exo}

\begin{corr}
\note{À venir}
\end{corr}

\begin{exo}[Oral CCMP, 8]
Étudiez la convergence de la suite \(\paren{a_n}\) définie par \(a_0>0\) et \(a_{n+1}=1-\e{-a_n}\).

Déterminez la nature des séries \(\sum\paren{-1}^na_n\) et \(\sum a_n^2\).

Déterminez la nature de la série \(\sum a_n\) (on pourra étudier la série \(\sum\ln\dfrac{a_{n+1}}{a_n}\)).
\end{exo}

\begin{corr}
\note{À venir}
\end{corr}

\begin{exo}[Oral CCINP, 9]
Soient \(\paren{a_n}\) une suite positive et \(\paren{u_n}\) définie par \(u_0>0\) et \(\quantifs{\tpt n\in\N}u_{n+1}=\dfrac{u_n+\sqrt{u_n^2+a_n^2}}{2}\).

\begin{enumerate}
    \item Montrez que \(\quantifs{\tpt n\in\N}u_{n+1}-u_n\leq\dfrac{a_n}{2}\). \\
    \item Montrez que si la série \(\sum a_n\) converge, alors la suite \(\paren{u_n}\) converge. \\
    \item La réciproque est-elle vraie ? Indication : considérer \(u_n=\dfrac{n}{n+1}\).
\end{enumerate}
\end{exo}

\begin{corr}
\note{À venir}
\end{corr}

\begin{exo}[Oral IMT, 10]
Soit \(\paren{u_n}\) la suite définie par \(u_0>0\) et \(\quantifs{\tpt n\in\N}u_{n+1}=\dfrac{n+1}{n+3}u_n\). On pose \(\paren{v_n}=\paren{n^2u_n}\).

\begin{enumerate}
    \item Déterminez la nature de la série de terme général \(\ln\dfrac{v_{n+1}}{v_n}\). \\
    \item Déduisez-en la nature de la série de terme général \(u_n\).
\end{enumerate}
\end{exo}

\begin{corr}
\note{À venir}
\end{corr}

\begin{exo}[Oral Centrale, 11]~\\
Montrez que la série \(\sum_{n\geq2}\dfrac{1}{n^2-1}\) converge et donnez la valeur de sa somme.

Montrez que la série \(\sum_{n\geq2}\dfrac{n^2}{\paren{n^2+1}^2}\) converge et donnez une valeur de \(n\) pour que sa somme partielle soit une valeur approchée de sa somme à \(10^{-4}\) près.
\end{exo}

\begin{corr}
\note{À venir}
\end{corr}

\begin{exo}[Oral Centrale, 12]~\\
\begin{enumerate}
    \item Pour \(n\in\Ns\), on pose \(H_n=\sum_{k=1}^n\dfrac{1}{k}\). Montrez que la suite \(\paren{H_n-\ln n}_{n\geq1}\) converge. \\
    \item Déduisez de la question précédente la valeur de \(\sum_{k=1}^{\pinf}\dfrac{\paren{-1}^{k+1}}{k}\). \\
    \item Pour \(s>1\), on pose \(\zeta\paren{s}=\sum_{k=1}^{\pinf}\dfrac{1}{k^s}\). Calculez \(\sum_{n=2}^{\pinf}\dfrac{\zeta\paren{n}-1}{n}\).
\end{enumerate}
\end{exo}

\begin{corr}
\note{À venir}
\end{corr}

\begin{exo}[Oral Centrale, 13]
Si \(\paren{u_n}\) est une suite réelle telle que \(u_0=0\), on pose alors \(\paren{v_n}_{n\geq1}\) la suite définie par : \(\quantifs{\tpt n\in\Ns}v_n=n\paren{u_n-u_{n-1}}\).

On note \(P_1\) la propriété \guillemets{la série \(\sum_{n\geq1}v_n\) converge} et \(P_2\) la propriété \guillemets{il existe \(l\) tel que \(u_n\to l\) et \(\sum\paren{l-u_n}\) converge}.

\begin{enumerate}
    \item Si \(\alpha\in\R\) et \(u_n=\Arctan n^\alpha\) pour \(n\geq1\), étudiez la véracité des propositions \(P_1\) et \(P_2\). \\
    \item Soit \(\paren{a_n}_{n\geq1}\) une suite réelle telle que \(\sum a_n\) converge. Montrez que \(\sum\dfrac{a_n}{n}\) converge et que \(\sum_{k=n}^{\pinf}\dfrac{a_k}{k}=\o{\dfrac{1}{n}}\). \\
    \item Comparez les propriétés \(P_1\) et \(P_2\).
\end{enumerate}
\end{exo}

\begin{corr}
\note{À venir}
\end{corr}

\begin{exo}[Oral CCMP, 14]~\\
Soit \(\alpha\in\Rps\excluant\accol{\dfrac{1}{2}}\). La série \(\sum_{n\geq2}\dfrac{\paren{-1}^n}{\sqrt{n}+\paren{-1}^nn^\alpha}\) converge-t-elle ?
\end{exo}

\begin{corr}
\note{À venir}
\end{corr}

\begin{exo}[Oral CCMP, 15]~\\
Soient \(\alpha>0\) et \(R_n=\sum_{k=n+1}^{\pinf}\dfrac{\paren{-1}^k}{k^\alpha}\).

\begin{enumerate}
    \item On suppose \(\alpha>1\). Montrez que \(\sum_{k=1}^nR_k=\paren{n+1}R_n+\sum_{k=1}^n\dfrac{\paren{-1}^k}{k^{\alpha-1}}\). Déduisez-en la convergence de la série \(\sum R_n\). \\
    \item Étudiez le cas \(\alpha\leq1\).
\end{enumerate}
\end{exo}

\begin{corr}
\note{À venir}
\end{corr}

\begin{exo}[Oral CCMP, 16]~\\
Soit \(\paren{a_n}\in\paren{\Rps}^{\Ns}\).

\begin{enumerate}
    \item On suppose que la série \(\sum a_n^{1-\nicefrac{1}{n}}\) converge. Montrez que la série \(\sum a_n\) converge. \\
    \item On suppose que la série \(\sum a_n\) converge. Montrez que la série \(\sum a_n^{1-\nicefrac{1}{n}}\) converge. \\ Vous introduirez, pour \(\lambda>1\), l'ensemble \(\accol{n\in\Ns\tq a_n^{1-\nicefrac{1}{n}}\leq\lambda a_n}\) et son complémentaire. \\
    \item Généralisez en remplaçant \(a_n^{1-\nicefrac{1}{n}}\) par \(a_n^{1-b_n}\) avec une hypothèse adéquate sur la suite \(\paren{b_n}\).
\end{enumerate}
\end{exo}

\begin{corr}
\note{À venir}
\end{corr}

\begin{exo}[Oral CCMP, 17]
Soit \(f\) une permutation de \(\Ns\). On pose \(E\paren{f}=\accol{\alpha\in\R\tq\sum\dfrac{f\paren{n}}{n^\alpha}\text{ converge}}\).

\begin{enumerate}
    \item Montrez que \(E\paren{f}\) peut être vide. Montrez dans le cas contraire que \(E\paren{f}\) est un intervalle minoré par \(2\) et non-majoré. \\
    \item Soit \(B\geq2\). Montrez l'existence de \(f\) telle que \(E\paren{f}=\intervee{B}{\pinf}\).
\end{enumerate}
\end{exo}

\begin{corr}
\note{À venir}
\end{corr}

\begin{exo}[Oral X, 18]~\\
Pour \(n\in\Ns\), on pose \(u_n=\sum_{k=1}^n\dfrac{1}{\sqrt{k}}-2\sqrt{n}\).

\begin{enumerate}
    \item Montrez que la suite \(\paren{u_n}_{n\geq1}\) converge. On note \(l\) sa limite. \\
    \item Montrez que \(l=-\paren{1+\sqrt{2}}\sum_{k=1}^{\pinf}\dfrac{\paren{-1}^{k-1}}{\sqrt{k}}\).
\end{enumerate}
\end{exo}

\begin{corr}
\note{À venir}
\end{corr}

\begin{exo}[Oral X, 19]
Soit \(\sum x_n\) une série absolument convergente de réels.

\begin{enumerate}
    \item Montrez que pour tout réel \(p\geq1\), la série \(\sum\abs{x_n}^p\) converge. \\
    \item Déterminez la limite de \(\sum_{n=0}^{\pinf}\abs{x_n}^p\) quand \(p\to\pinf\).
\end{enumerate}
\end{exo}

\begin{corr}
\note{À venir}
\end{corr}

\begin{exo}[Oral X, 20]
Soient \(\alpha>0\) et \(\paren{u_n}\) une suite de réels strictement positifs telle que \(\dfrac{u_{n+1}}{u_n}=1-\dfrac{\alpha}{n}+\O{\dfrac{1}{n^2}}\).

Étudiez la convergence de la série \(\sum u_n\).
\end{exo}

\begin{corr}
\note{À venir}
\end{corr}


\include{exercices/chap3_familles_sommables/chap3_familles_sommables.tex}

\include{exercices/chap4_algebre_lineaire/chap4_algebre_lineaire.tex}

\chapter{Réduction des endomorphismes}

\minitoc

\section*{Exercices}
\addcontentsline{toc}{section}{Exercices}

\legendeexercices

\begin{exoss}[Exercice 1]
\begin{enumerate}
    \item Soit \(P\in\poly[\C]\) tel que \(P\) ne soit pas constant. On note \(a_1,\dots,a_m\) les racines distinctes de \(P\) et \(\alpha_1,\dots,\alpha_m\) leurs ordres de multiplicité. Rappelez la décomposition en éléments simples de \(\dfrac{P\prim}{P}\). \\
    \item Application : déterminez les éléments propres de l'endomorphisme de \(\poly[\C]\) défini par \(f\paren{P}=\paren{2X+1}P-\paren{X^2-1}P\prim\). On rappelle que la décomposition en éléments simples est unique. \\
    \item Même question avec \(f\paren{P}=\paren{X^3+X}P\prim-\paren{3X^2-1}P\).
\end{enumerate}
\end{exoss}

\begin{corr}
\note{À venir}
\end{corr}

\begin{exoss}[Exercice 2]
Soit \(E=\ensclasse{0}{\intervii{0}{1}}{\R}\). Pour \(f\in E\), on définit l'application \(\phi\paren{f}:\intervii{0}{1}\to\R\) de la façon suivante : \(\phi\paren{f}\paren{0}=f\paren{0}\) et \(\quantifs{\tpt x\in\intervei{0}{1}}\phi\paren{f}\paren{x}=\dfrac{1}{x}\int_0^xf\paren{t}\odif{t}\).

\begin{enumerate}
    \item Montrez que \(\phi\) est un endomorphisme de \(E\). \\
    \item Montrez que \(0\) n'est pas valeur propre de \(\phi\). \\
    \item Montrez que \(1\) est valeur propre de \(\phi\) et donnez le sous-espace propre associé. \\
    \item Déterminez les autres valeurs propres de \(\phi\).
\end{enumerate}
\end{exoss}

\begin{corr}
\note{À venir}
\end{corr}

\begin{exoss}[Exercice 3]
Soient \(n\in\Ns\), \(A\in\M{n}\excluant\accol{0}\) et \(\lambda\in\K\). Pour \(M\in\M{n}\), on pose \(f\paren{M}=\lambda M+\tr\paren{M}A\).

\begin{enumerate}
    \item Justifiez rapidement que \(f\) est un endomorphisme de \(\M{n}\). \\
    \item Déterminez un polynôme annulateur de \(f\) de degré \(2\). \\
    \item Déduisez-en les éléments propres de \(f\). \\
    \item À quelle condition l'endomorphisme \(f\) est-il inversible ?
\end{enumerate}
\end{exoss}

\begin{corr}
\note{À venir}
\end{corr}

\begin{exoss}[Exercice 4]
Soit \(A\in\M{n}\). Pour \(M\in\M{n}\), on pose \(f\paren{M}=AM\).

Montrez que \(\Sp{f}=\Sp{A}\). Pour tout \(\lambda\) valeur propre de \(f\), donnez une relation entre les dimensions des sous-espaces propres \(\sep{f}{\lambda}\) et \(\sep{A}{\lambda}\).
\end{exoss}

\begin{corr}
\note{À venir}
\end{corr}

\begin{exoss}[Exercice 5, matrices stochastiques]
Soit \(A=\paren{a_{i\,j}}\in\M{n}[\R]\) une matrice stochastique, \cad vérifiant : \[\quantifs{\forall\paren{i,j}\in\interventierii{1}{n}^2}a_{i\,j}\geq0\text{ et }\quantifs{\forall i\in\interventierii{1}{n}}\sum_{j=1}^na_{i\,j}=1.\]

\begin{enumerate}
    \item Montrez que \(1\) est valeur propre de \(A\). \\
    \item Montrez que toutes les valeurs propres complexes de \(A\) sont de module inférieur ou égal à \(1\). \\
    \item On suppose que \(A\) est strictement stochastique, \cad que \(A\) est stochastique et que ses coefficients sont strictement positifs. Montrez que \(1\) est la seule valeur propre de \(A\) de module \(1\).
\end{enumerate}
\end{exoss}

\begin{corr}
\note{À venir}
\end{corr}

\begin{exoss}[Exercice 6]
Soient \(\paren{a,b}\in\R^2\) tel que \(a\not=b\) et \(A=\begin{pmatrix}
0 & a & \dots & a \\
b & \ddots & \ddots & \vdots \\
\vdots & \ddots & \ddots & a \\
b & \dots & b & 0
\end{pmatrix}\).

\begin{enumerate}
    \item On pose \(J\) la matrice remplie de \(1\). Montrez que \(t\mapsto\det\paren{XI_n-A-tJ}\) est une fonction polynôme en \(t\) de degré au plus \(1\), puis calculez \(\det\paren{XI_n-A-tJ}\) en fonction de \(t\). \\
    \item Montrez que le polynôme caractéristique de \(A\) est \(\dfrac{1}{b-a}\paren{b\paren{X+a}^n-a\paren{X+b}^n}\). \\
    \item Montrez que si \(b=-a\), alors les images dans \(\C\) des valeurs propres de \(A\) sont sur une droite que vous préciserez, sinon elles sont sur un cercle.
\end{enumerate}
\end{exoss}

\begin{corr}
\note{À venir}
\end{corr}

\begin{exoss}[Exercice 7]
Soient \(n\geq2\), \(U\in\M{n\,1}[\R]\) et \(A=U\trans{U}\). On note \(\trans{U}U=\begin{pmatrix}
s
\end{pmatrix}\) qui est une matrice carrée à un seul élément.

\begin{enumerate}
    \item Montrez que le polynôme \(X^2-sX\) est annulateur de \(A\). Déduisez-en que \(A\) a au plus deux valeurs propres. \\
    \item Quel est le rang de \(A\) ? Précisez les valeurs propres de \(A\). \\
    \item Déterminez le polynôme caractéristique de \(A\). \\
    \item Déterminez les sous-espaces propres de \(A\).
\end{enumerate}
\end{exoss}

\begin{corr}
\note{À venir}
\end{corr}

\begin{exos}[Exercice 8]~\\
Soit \(A=\begin{pmatrix}
-1 & 2 & -1 \\
-3 & 4 & -3 \\
-4 & 4 & -4
\end{pmatrix}\).

\begin{enumerate}
    \item Déterminez le polynôme caractéristique de \(A\). \\
    \item Déduisez-en sans calcul supplémentaire que \(A\) est diagonalisable. \\
    \item Déterminez les sous-espaces propres de \(A\). \\
    \item Diagonalisez \(A\), \cad explicitez une matrice \(P\) inversible et une matrice \(D\) diagonale telles que \(A=PDP\inv\). \\
    \item Calculez \(A^n\) pour tout \(n\in\N\).
\end{enumerate}
\end{exos}

\begin{corr}
\note{À venir}
\end{corr}

\begin{exos}[Exercice 9, réduction en élégance]~\\
Soit \(B=\begin{pmatrix}
1 & 2 & -3 \\
2 & 4 & -6 \\
4 & 8 & -12
\end{pmatrix}\).

\begin{enumerate}
    \item Calculez efficacement le rang de \(B\). Déduisez-en une valeur propre de \(B\) ainsi que la dimension du sous-espace propre associé. \\
    \item Démontrez sans calculer le polynôme caractéristique que \(B\) admet une autre valeur propre et qu'elle est simple. \\
    \item Déduisez-en le polynôme caractéristique de \(B\). Vérifiez le résultat en le calculant grâce à sa définition. \\
    \item La matrice \(B\) est-elle diagonalisable ?
\end{enumerate}
\end{exos}

\begin{corr}
\note{À venir}
\end{corr}

\begin{exos}[Exercice 10]
Les matrices suivantes sont-elles diagonalisables ?

\[A=\begin{pmatrix}
0 & 1 & 0 \\
0 & 0 & 1 \\
1 & 0 & 0
\end{pmatrix}\qquad B=\begin{pmatrix}
1 & 0 & 1 \\
-1 & 3 & 1 \\
1 & -1 & 2
\end{pmatrix}\qquad C=\begin{pmatrix}
1 & 1 & 1 & 1 \\
1 & 0 & 0 & 1 \\
1 & 0 & 0 & 1 \\
1 & 1 & 1 & 1
\end{pmatrix}\qquad D=\begin{pmatrix}
1 & 1 & 1 & 1 \\
0 & 0 & 1 & 0 \\
0 & 1 & 0 & 0 \\
1 & 1 & 1 & 1
\end{pmatrix}\qquad E=\begin{pmatrix}
5 & 1 & 1 & 1 \\
2 & 6 & 2 & 2 \\
3 & 3 & 7 & 3 \\
4 & 4 & 4 & 8
\end{pmatrix}\]
\end{exos}

\begin{corr}
\note{À venir}
\end{corr}

\begin{exos}[Exercice 11]
Les matrices suivantes sont-elles diagonalisables ?

\[A=\begin{pmatrix}
0 & 0 & 4 \\
1 & 0 & -8 \\
0 & 1 & 5
\end{pmatrix}\qquad B=\begin{pmatrix}
2 & 1 & 1 \\
0 & 0 & -2 \\
0 & 1 & 3
\end{pmatrix}\]
\end{exos}

\begin{corr}
\note{À venir}
\end{corr}

\begin{exos}[Exercice 12]
Déterminez une condition nécessaire et suffisante sur le triplet \(\paren{a,b,c}\in\R^3\) pour que la matrice \(M=\begin{pmatrix}
1 & a & 1 \\
0 & 1 & b \\
0 & 0 & c
\end{pmatrix}\) soit diagonalisable.
\end{exos}

\begin{corr}
\note{À venir}
\end{corr}

\begin{exos}[Exercice 13]~\\
Soit \(A=\begin{pmatrix}
a & b \\
c & d
\end{pmatrix}\) une matrice de \(\M{2}[\R]\) qui n'est pas de la forme \(\lambda I_2\).

Montrez que \(A\) est diagonalisable dans \(\M{2}[\C]\) ssi \(\paren{a-d}^2+4bc\not=0\).

À quelle condition \(A\) est-elle diagonalisable dans \(\M{2}[\R]\) ?
\end{exos}

\begin{corr}
\note{À venir}
\end{corr}

\begin{exoss}[Exercice 14]~\\
Soit \(A=\begin{pmatrix}
11 & -5 & -5 \\
-5 & 3 & 3 \\
-5 & 3 & 3
\end{pmatrix}\). On recherche les éventuelles racines carrées de \(A\), \cad les matrices \(R\in\M{3}[\R]\) telles que \(R^2=A\).

\begin{enumerate}
    \item Montrez que \(A\) est diagonalisable et déterminez une matrice \(D\) diagonale semblable à \(A\) avec le moins de calculs possible. \\
    \item Soit \(S\in\M{3}[\R]\) une racine carrée de \(D\). Montrez que \(S\) et \(D\) commutent puis montrez que \(S\) est diagonale. \\
    \item Déterminez les racines carrées \(S\) de \(D\). \\
    \item Déduisez-en toutes les racines carrées \(R\) de \(A\). Combien y en a-t-il ? Pourquoi ? \\
    \item Énoncez des conjectures quant au nombre de racines carrées d'une matrice \(A\in\M{n}[\C]\) plus générale, en discutant selon la nature de ses éléments propres.
\end{enumerate}
\end{exoss}

\begin{corr}
\note{À venir}
\end{corr}

\begin{exoss}[Exercice 15]
Soit \(u\) l'endomorphisme canoniquement associé à la matrice \(A=\begin{pmatrix}
0 & 1 & -1 \\
2 & 1 & 1 \\
-2 & -1 & -1
\end{pmatrix}\).

\begin{enumerate}
    \item Diagonalisez la matrice \(A\). \\
    \item On suppose que \(F\) est un sous-espace de \(\R^3\) stable par \(u\). Montrez que \(F\) est engendré par une famille de vecteurs propres de \(u\). \\
    \item Déterminez tous les sous-espaces de \(\R^3\) stables par \(u\).
\end{enumerate}
\end{exoss}

\begin{corr}
\note{À venir}
\end{corr}

\begin{exos}[Exercice 16]~\\
Même exercice avec \(A=\begin{pmatrix}
-1 & -6 & -2 \\
2 & 7 & 2 \\
-6 & -18 & -5
\end{pmatrix}\).
\end{exos}

\begin{corr}
\note{À venir}
\end{corr}

\begin{exos}[Exercice 17]
Montrez que les suites \(\paren{u_n}\in\R^\N\) vérifiant la relation de récurrence suivante sont combinaisons linéaires de trois suites géométriques réelles \(\paren{\alpha^n}\), \(\paren{\beta^n}\) et \(\paren{\gamma^n}\) : \(\quantifs{\forall n\in\N}u_{n+3}=-\dfrac{5}{12}u_{n+2}+\dfrac{1}{2}u_{n+1}-\dfrac{1}{12}u_n\).

Déterminez à quelle condition sur \(\paren{u_0,u_1,u_2}\) ces suites sont convergentes.
\end{exos}

\begin{corr}
\note{À venir}
\end{corr}

\begin{exos}[Exercice 18]
Même exercice avec \(\quantifs{\forall n\in\N}u_{n+3}=\dfrac{3}{2}u_{n+2}-\dfrac{3}{4}u_{n+1}+\dfrac{1}{4}u_n\) en prenant des suites géométriques complexes.

Montrez que les suites \(\paren{u_n}\) sont combinaisons linéaires de trois suites réelles simples que l'on précisera. Déterminez à quelle condition sur \(\paren{u_0,u_1,u_2}\) ces suites sont convergentes.
\end{exos}

\begin{corr}
\note{À venir}
\end{corr}

\begin{exos}[Exercice 19]
Les matrices suivantes sont-elles trigonalisables ? Donnez alors, quand c'est possible, une matrice triangulaire supérieure semblable.

\[A=\begin{pmatrix}
1 & 4 & -2 \\
0 & 6 & -3 \\
-1 & 4 & 0
\end{pmatrix}\qquad B=\begin{pmatrix}
2 & -1 & -1 \\
2 & 1 & -2 \\
3 & -1 & -2
\end{pmatrix}\]
\end{exos}

\begin{corr}
\note{À venir}
\end{corr}

\begin{exos}[Exercice 20]~\\
Soient \(C=\begin{pmatrix}
2 & -3 & -1 \\
1 & -2 & -1 \\
-2 & 6 & 3
\end{pmatrix}\) et \(T=\begin{pmatrix}
1 & 0 & -2 \\
0 & 1 & 1 \\
0 & 0 & 1
\end{pmatrix}\).

\begin{enumerate}
    \item Déterminez le polynôme caractéristique de \(C\). Est-elle diagonalisable ? Trigonalisable ? \\
    \item Déterminez les éléments propres de \(C\). \\
    \item Montrez que la matrice \(C\) est semblable à la matrice \(T\). On pourra considérer l'endomorphisme \(u\) canoniquement associé à \(C\) et construire, par analyse-synthèse, une base \(\fami{B}\) où la matrice de \(u\) est \(T\). \\
    \item Montrez que \(T\) peut s'écrire \(D+N\) où \(D\) est diagonale, \(N\) est nilpotente et \(D\) et \(N\) commutent. Déduisez-en, pour tout \(n\in\N\), \(T^n\) puis \(C^n\). \\
    \item On considère trois suites \(\paren{x_n},\paren{y_n},\paren{z_n}\in\R^\N\) telles que \[\quantifs{\forall n\in\N}\begin{dcases}
        x_{n+1}=2x_n-3y_n-z_n \\
        y_{n+1}=x_n-2y_n-z_n \\
        z_{n+1}=-2x_n+6y_n+3z_n
    \end{dcases}\] Explicitez \(x_n,y_n,z_n\) en fonction de \(n\) et de \(x_0,y_0,z_0\).
\end{enumerate}
\end{exos}

\begin{corr}
\note{À venir}
\end{corr}

\begin{exoss}[Exercice 21]
Soit \(u\in\Lendo{\R^n}\) tel que \(u^2+u+\id{\R^n}=0\).

\begin{enumerate}
    \item Soient \(F\) un sous-espace stable par \(u\) et \(x\not\in F\). Montrez que \(\Pi_x=\Vect{x,u\paren{x}}\) est un plan, qu'il est stable par \(u\) et qu'il est en somme directe avec \(F\). \\
    \item Montrez qu'il existe une base de \(\R^n\) dans laquelle la matrice de \(u\) est diagonale par blocs, de blocs \(R=\begin{pmatrix}
        0 & -1 \\
        1 & -1
    \end{pmatrix}\). Qu'en déduisez-vous concernant \(n\) ? \\
    \item Soit \(A\in\M{n}[\R]\) la matrice de \(u\) dans la base canonique. Réduisez \(A\) dans \(\C\). \\
    \item Retrouvez que \(A\) est \(\R\)-semblable à une matrice diagonale par blocs, de blocs \(R\).
\end{enumerate}
\end{exoss}

\begin{corr}
\note{À venir}
\end{corr}

\begin{exoss}[Exercice 22]
Soient \(p\in\Ns\), \(E\) un \(\C\)-espace vectoriel de dimension finie et \(f\in\GL{}[E]\).

Montrez que \(f\) est diagonalisable ssi \(f^p\) est diagonalisable.

Est-ce encore vrai si on remplace \(\C\) par \(\R\) ?
\end{exoss}

\begin{corr}
\note{À venir}
\end{corr}

\begin{exoss}[Exercice 23]
Soient \(E\) un \(\C\)-espace vectoriel de dimension finie et \(f\in\Lendo{E}\).

\begin{enumerate}
    \item Montrez que si \(f\) est diagonalisable, alors \(f^2\) est diagonalisable et \(\rg f^2=\rg f\). \\
    \item Montrez que si \(\lambda\) est une valeur propre de \(f^2\) non-nulle et \(\mu\) est une racine carrée de \(\lambda\), alors \(\sep{f^2}{\lambda}=\ker\paren{f-\mu\id{}}\oplus\ker\paren{f+\mu\id{}}\). \\
    \item Montrez que la réciproque de la proposition de la question (1) est vraie.
\end{enumerate}
\end{exoss}

\begin{corr}
\note{À venir}
\end{corr}

\begin{exoss}[Exercice 24, spectre du polynôme d'une matrice]
Soient \(M\in\M{n}\) diagonalisable et \(Q\in\poly\).

\begin{enumerate}
    \item Montrez que la matrice \(Q\paren{M}\) est également diagonalisable. Exprimez le spectre de \(Q\paren{M}\) en fonction des valeurs propres de \(M\). \\
    \item Précisez les sous-espaces propres de \(Q\paren{M}\). \\
    \item Ces résultats restent-ils valables si \(M\) est seulement trigonalisable ? Sans hypothèse sur \(M\) ?
\end{enumerate}
\end{exoss}

\begin{corr}
\note{À venir}
\end{corr}

\begin{exosss}[Exercice 25]
Soit \(A\in\M{2}[\Z]\). On suppose qu'il existe \(p\in\Ns\) tel que \(A^p=I_2\). Montrez que \(A^{12}=I_2\).
\end{exosss}

\begin{corr}
\note{À venir}
\end{corr}

\begin{exoss}[Exercice 26]
Soit \(A\in\M{n}[\C]\). Montrez que \(A\) est nilpotente ssi \(\quantifs{\tpt k\in\Ns}\tr A^k=0\).
\end{exoss}

\begin{corr}
\note{À venir}
\end{corr}

\begin{exoss}[Exercice 27]
Soit \(f\) un endomorphisme diagonalisable d'un \(\K\)-espace vectoriel \(E\) de dimension finie. On note \(C\paren{f}\) le sous-espace vectoriel des endomorphismes de \(E\) qui commutent avec \(f\).

\begin{enumerate}
    \item Démontrez que \(g\in C\paren{f}\) ssi les sous-espaces propres de \(f\) sont stables par \(g\). \\
    \item Déduisez-en que \(\dim C\paren{f}=\sum_{\lambda\in\Sp{f}}\omega_\lambda^2\) où \(\omega_\lambda\) désigne la multiplicité de la valeur propre \(\lambda\). \\
    \item On suppose que les valeurs propres de \(f\) sont simples. Démontrez que \(\paren{\id{},f,\dots,f^{n-1}}\) est une base de \(C\paren{f}\).
\end{enumerate}
\end{exoss}

\begin{corr}
\note{À venir}
\end{corr}

\begin{exoss}[Exercice 28]
\begin{enumerate}
    \item Montrez que \(\GL{n}\) est un ouvert dense de \(\M{n}\). \\
    \item Montrez que \(\GL{n}[\C]\) est connexe par arcs et que \(\GL{n}[\R]\) ne l'est pas. Déterminez alors les composantes connexes par arcs de \(\GL{n}[\R]\). \\
    \item Montrez que l'ensemble \(D_n\paren{\C}\) des matrices diagonalisables de \(\M{n}[\C]\) est dense dans \(\M{n}[\C]\). Montrez la même chose avec l'ensemble \(D_n^+\paren{\C}\) des matrices de \(\M{n}[\C]\) dont toutes les valeurs propres sont distinctes. Montrez que l'intérieur de \(D_n\paren{\C}\) est \(D_n^+\paren{\C}\). \\
    \item Montrez que \(D_2\paren{\R}\) n'est pas dense dans \(\M{2}[\R]\). \\
    \item Montrez que l'ensemble des matrices trigonalisables de \(\M{n}[\R]\) est un fermé et qu'il est l'adhérence de \(D_n\paren{\R}\). \\
    \item Soit \(r\in\interventierii{1}{n}\). Montrez que l'ensemble des matrices de rang inférieur à \(r\) est un fermé et qu'il est l'adhérence de l'ensemble des matrices de rang \(r\).
\end{enumerate}
\end{exoss}

\begin{corr}
\note{À venir}
\end{corr}

\begin{exosss}[Exercice 29]
Soient \(A,B\in\M{n}\).

\begin{enumerate}
    \item Montrez que si l'une des deux matrices est inversible, alors \(AB\) et \(BA\) ont les mêmes valeurs propres avec les mêmes ordres de multiplicité. \\
    \item Montrez que ce résultat reste vrai même sans suppose l'une des deux matrices inversible.
\end{enumerate}
\end{exosss}

\begin{corr}
\note{À venir}
\end{corr}

\begin{exo}[Oral IMT, 1]
Soient \(u,v,w\) trois suites vérifiant \[\quantifs{\forall n\in\N}\begin{dcases}
u_{n+1}=4u_n-3v_n-3w_n \\
v_{n+1}=3u_n-2v_n-3w_n \\
w_{n+1}=3u_n-3v_n-2w_n
\end{dcases}\]

Exprimez \(u_n,v_n,w_n\) en fonction de \(n,u_0,v_0,w_0\).
\end{exo}

\begin{corr}
\note{À venir}
\end{corr}

\begin{exo}[Oral IMT, 2]
Déterminez les matrices \(M\in\M{3}[\R]\) telles que \(M^2=\begin{pmatrix}
0 & 1 & 1 \\
1 & 0 & 1 \\
1 & 1 & 0
\end{pmatrix}\).
\end{exo}

\begin{corr}
\note{À venir}
\end{corr}

\begin{exo}[Oral St-Cyr, 3]~\\
Montrez que la matrice \(A=\begin{pmatrix}
a & 0 & b \\
0 & a+b & 0 \\
b & 0 & a
\end{pmatrix}\) est diagonalisable et donnez ses éléments propres.
\end{exo}

\begin{corr}
\note{À venir}
\end{corr}

\begin{exo}[Oral TPE, 4]~\\
Soient \(\paren{a,b,c}\in\R^3\) et \(A=\begin{pmatrix}
0 & -a & -b \\
a & 0 & -c \\
b & c & 0
\end{pmatrix}\).

\begin{enumerate}
    \item Déterminez un polynôme annulateur de degré \(3\) de \(A\). \\
    \item La matrice \(A\) est-elle inversible ? \\
    \item Est-elle diagonalisable ? \\
    \item Montrez que les valeurs propres de \(A^2\) sont négatives ou nulles.
\end{enumerate}
\end{exo}

\begin{corr}
\note{À venir}
\end{corr}

\begin{exo}[Oral TPE, 5]
Montrez de deux façons différentes que \(\begin{pmatrix}
0 & 1 & 2 \\
1 & 0 & 1 \\
1 & 0 & 0
\end{pmatrix}\) et \(\begin{pmatrix}
0 & 1 & 1 \\
1 & 0 & 2 \\
0 & 1 & 0
\end{pmatrix}\) sont semblables.
\end{exo}

\begin{corr}
\note{À venir}
\end{corr}

\begin{exo}[Oral IMT, 6]
Soit \(\alpha\in\R\). Déterminez la limite quand \(n\to\pinf\) de \(\begin{pmatrix}
\cos\nicefrac{\alpha}{n} & \sin\nicefrac{\alpha}{n} \\
\sin\nicefrac{\alpha}{n} & \cos\nicefrac{\alpha}{n}
\end{pmatrix}^n\).
\end{exo}

\begin{corr}
\note{À venir}
\end{corr}

\begin{exo}[Oral IMT, 7]
Soit \(A\in\M{n}[\C]\) telle que \(A^2-2A\) soit diagonalisable et \(1\) n'est pas valeur propre de \(A\).

Montrez que \(A\) est diagonalisable.
\end{exo}

\begin{corr}
\note{À venir}
\end{corr}

\begin{exo}[Oral CCINP, 8]
\begin{enumerate}
    \item Soient \(M\in\M{n}[\C]\) et \(\lambda_1,\dots,\lambda_n\) ses valeurs propres. Montrez que \(\tr M^2=\sum_{i=1}^n\lambda_i^2\). \\
    \item Pour \(n\geq3\), on pose \(A\) la matrice dont tous les coefficients sont nuls, sauf ceux situés sur les quatre bords, égaux à \(1\). Déterminez les valeurs propres et les vecteurs propres de \(A\).
\end{enumerate}
\end{exo}

\begin{corr}
\note{À venir}
\end{corr}

\begin{exo}[Oral CCINP, 9]
Soient \(\alpha\in\C\) et \(A=\paren{\alpha^{i+j-2}}_{1\leq i,j\leq n}\in\M{n}[\C]\).

\begin{enumerate}
    \item Calculez le rang de \(A\). Déduisez-en ses valeurs propres. \\
    \item À quelle condition sur \(\alpha\) la matrice \(A\) est-elle diagonalisable ?
\end{enumerate}
\end{exo}

\begin{corr}
\note{À venir}
\end{corr}

\begin{exo}[Oral CCINP, 10]
Soient \(n\in\Ns\) et \(f\) l'endomorphisme de \(\M{n}[\R]\) tel que \(f\paren{M}=M+2\trans{M}\).

\begin{enumerate}
    \item Déterminez les valeurs et vecteurs propres de \(f\). \\
    \item L'endomorphisme \(f\) est-il diagonalisable ? Calculez sa trace et son déterminant.
\end{enumerate}
\end{exo}

\begin{corr}
\note{À venir}
\end{corr}

\begin{exo}[Oral Navale, 11]
Soit \(M\in\M{n}[\C]\). Montrez que \(1\) est la seule valeur propre de \(M\) ssi \(\quantifs{\tpt k\in\Ns}\tr M^k=n\).
\end{exo}

\begin{corr}
\note{À venir}
\end{corr}

\begin{exo}[Oral Navale, 12]~\\
\begin{enumerate}
    \item Diagonalisez la matrice \(\begin{pmatrix}
        1 & 4 \\
        1 & 1
    \end{pmatrix}\). \\
    \item Soient \(A\in\M{n}[\C]\), \(B=\begin{pmatrix}
        A & 4A \\
        A & A
    \end{pmatrix}\) et \(C=\begin{pmatrix}
        -A & 0 \\
        0 & 3A
    \end{pmatrix}\). Montrez que \(B\) et \(C\) sont semblables.
\end{enumerate}
\end{exo}

\begin{corr}
\note{À venir}
\end{corr}

\begin{exo}[Oral CCINP, 13]~\\
Soient \(A\in\M{n}[\C]\) et \(B=\begin{pmatrix}
0 & A \\
I_n & 0
\end{pmatrix}\).

\begin{enumerate}
    \item Exprimez le rang de \(B\) en fonction du rang de \(A\). \\
    \item Trouvez une relation entre \(\chi_A\) et \(\chi_B\). Déduisez-en le spectre de \(B\) en fonction de celui de \(A\). \\
    \item Déterminez les dimensions des sous-espaces propres de \(B\) en fonction de celles des sous-espaces propres de \(A\). \\
    \item Montrez que \(B\) est diagonalisable ssi \(A\) est diagonalisable et inversible.
\end{enumerate}
\end{exo}

\begin{corr}
\note{À venir}
\end{corr}

\begin{exo}[Oral IMT, 14]
Soient \(E\) un espace muni d'une base \(\paren{e_1,\dots,e_n}\), \(v\) un vecteur de \(E\) et \(f\) l'endomorphisme de \(E\) tel que \(f\paren{e_1}=\dots=f\paren{e_n}=v\).

\begin{enumerate}
    \item Quel est le rang de \(f\) ? \\
    \item Discutez de la diagonalisabilité de \(f\) en fonction du vecteur \(v\).
\end{enumerate}
\end{exo}

\begin{corr}
\note{À venir}
\end{corr}

\begin{exo}[Oral CCINP, 15]
\begin{enumerate}[series=oral5.15]
    \item Montrez que si deux matrices \(U\) et \(V\) sont semblables, alors pour tout polynôme \(R\), \(R\paren{U}\) et \(R\paren{V}\) sont semblables.
\end{enumerate}

Soient \(A,B\) deux matrices de \(\M{n}[\C]\) telles que \(AB=BA\). On pose \(M=\begin{pmatrix}
A & B \\
0 & A
\end{pmatrix}\).

\begin{enumerate}[resume=oral5.15]
    \item Pour \(P\in\poly[\C]\), exprimez \(P\paren{M}\) en fonction de \(P\paren{A}\), \(P\prim\paren{A}\) et \(B\). \\
    \item Montrez que si \(A\) est diagonalisable et \(B\) est nulle, alors \(M\) est diagonalisable. \\
    \item Montrez la réciproque.
\end{enumerate}
\end{exo}

\begin{corr}
\note{À venir}
\end{corr}

\begin{exo}[Oral CCMP, 16]
\begin{enumerate}
    \item Soient \(A\) et \(B\) deux matrices de \(\M{n}[\R]\) telles que \(A^2=B^2=0\) et \(\rg A=\rg B\). Montrez que \(A\) et \(B\) sont semblables. \\
    \item Le résultat subsiste-t-il avec les hypothèses \(A^3=B^3=0\) et \(\rg A=\rg B\) ?
\end{enumerate}
\end{exo}

\begin{corr}
\note{À venir}
\end{corr}

\begin{exo}[Oral CCMP, 17]
Soient \(n\in\Ns\) et \(A,B\in\M{n}[\R]\) tels que \(\paren{AB}^n=0\). Montrez que \(\paren{BA}^n=0\).
\end{exo}

\begin{corr}
\note{À venir}
\end{corr}

\begin{exo}[Oral CCMP, 18]
Soient \(E\) un \(\K\)-espace vectoriel de dimension finie et \(u,v\in\Lendo{E}\).

\begin{enumerate}
    \item On suppose que \(\Vect{u,v}\) contient un élément inversible. Montrez que \(\ker u\inter\ker v=\accol{0}\). \\
    \item Montrez que la réciproque est fausse. \\
    \item Montrez que si \(u\) et \(v\) commutent, alors la réciproque est vraie.
\end{enumerate}
\end{exo}

\begin{corr}
\note{À venir}
\end{corr}

\begin{exo}[Oral CCMP, 19]
Soient \(u\in\Lendo{\R^3}\) non-nul tel que \(u^3=u^2\) et \(C\paren{u}=\accol{v\in\Lendo{\R^3}\tq u\rond v=v\rond u}\).

Montrez que \(C\paren{u}\) est un sous-espace vectoriel de \(\Lendo{\R^3}\) et déterminez sa dimension.
\end{exo}

\begin{corr}
\note{À venir}
\end{corr}

\begin{exo}[Oral CCMP, 20]
Soient \(A\in\M{n}[\C]\) et \(\phi\) l'endomorphisme de \(\M{n}[\C]\) défini par \(\phi:M\mapsto M+\tr\paren{AM}A\).

\begin{enumerate}
    \item Étudiez la diagonalisabilité de \(\phi\). \\
    \item Calculez \(\tr\phi\) et \(\det\phi\).
\end{enumerate}
\end{exo}

\begin{corr}
\note{À venir}
\end{corr}

\begin{exo}[Oral CCMP, 21]~\\
Pour \(c\in\R\), on pose \(A\paren{c}=\begin{pmatrix}
-c & -1 & c \\
-1 & 1-c & 1 \\
c & -1 & -c
\end{pmatrix}\).

\begin{enumerate}
    \item Déterminez les réels \(c\) tels que \(A\paren{c}\) ne pas diagonalisable. \\
    \item Soit \(d\) la plus petite de ces valeurs. Trouvez \(P\) inversible telle que \(P\inv A\paren{d}P\) soit triangulaire.
\end{enumerate}
\end{exo}

\begin{corr}
\note{À venir}
\end{corr}

\begin{exo}[Oral CCMP, 22]~\\
\begin{enumerate}
    \item Soit \(x=\cos\dfrac{2\pi}{5}\). Déterminez une équation du second degré à coefficients rationnels dont \(x\) est racine, puis donnez les valeurs de \(x\) et \(\cos\dfrac{4\pi}{5}\). \\
    \item Soit \(A\in\M{n}[\R]\) telle que \(A^4+A^3+A^2+A+I_n=0\). On suppose que la trace de \(A\) est un rationnel. Montrez que \(4\) divise \(n\).
\end{enumerate}
\end{exo}

\begin{corr}
\note{À venir}
\end{corr}

\begin{exo}[Oral CCMP, 23]
Soit \(A\in\M{n}[\R]\) telle que \(A^3=A+I_n\). Montrez que \(\det A>0\).
\end{exo}

\begin{corr}
\note{À venir}
\end{corr}

\begin{exo}[Oral CCMP, 24]~\\
Soient \(A\in\M{n}\) et \(B=\begin{pmatrix}
A & A \\
0 & A
\end{pmatrix}\). Déterminez une condition nécessaire et suffisante sur \(A\) pour que \(B\) soit diagonalisable.
\end{exo}

\begin{corr}
\note{À venir}
\end{corr}

\begin{exo}[Oral CCMP, 25]~\\
Soient \(A\in\M{n}\) et \(B=\begin{pmatrix}
I_n & 0 \\
A & A
\end{pmatrix}\). On suppose que \(B\) est diagonalisable. Montrez que \(A\) est diagonalisable et que \(I_n-A\) est inversible.
\end{exo}

\begin{corr}
\note{À venir}
\end{corr}

\begin{exo}[Oral CCMP, 26]
Soit \(A\in\M{n}[\C]\). On suppose qu'il existe \(P\in\poly[\C]\) tel que \(P\paren{A}\) est diagonalisable et \(P\prim\paren{A}\) est inversible. Montrez que \(A\) est diagonalisable.
\end{exo}

\begin{corr}
\note{À venir}
\end{corr}

\begin{exo}[Oral CCMP, 27]
Soit \(A=\paren{a_{i\,j}}\in\M{n}[\R]\) définie par \(a_{i\,i}=0\) et \(a_{i\,j}=i\) si \(i\not=j\).

\begin{enumerate}
    \item Montrez qu'un réel \(\lambda\) est valeur propre de \(A\) ssi \(\sum_{k=1}^n\dfrac{k}{k+\lambda}=1\). \\
    \item Montrez que \(A\) est diagonalisable. Listez ses valeurs propres avec un encadrement le plus précis possible. \\
    \item Déterminez la somme des valeurs propres de \(A\). On note \(\mu_n\) la plus grande d'entre elles. Trouvez \(C\in\R\) tel que \(\mu_n\sim Cn^2\) quand \(n\) tend vers l'infini.
\end{enumerate}
\end{exo}

\begin{corr}
\note{À venir}
\end{corr}

\begin{exo}[Oral Centrale, 28]~\\
À quelles conditions \(A=\begin{pmatrix}
1 & 0 & 0 & 0 \\
a & 1 & 0 & 0 \\
b & c & 2 & 0 \\
d & e & f & 2
\end{pmatrix}\) est-elle diagonalisable ? Le cas échéant, diagonalisez \(A\).
\end{exo}

\begin{corr}
\note{À venir}
\end{corr}

\begin{exo}[Oral Centrale, 29]~\\
Soient \(a,b,c\in\R\) et \(A=\begin{pmatrix}
0 & a & b & c \\
a & 0 & c & b \\
b & c & 0 & a \\
c & b & a & 0
\end{pmatrix}\).

\begin{enumerate}
    \item Donnez un vecteur propre évident de \(A\). \\
    \item Calculez le polynôme caractéristique de \(A\) et donnez son spectre. Justifiez que \(A\) est diagonalisable. \\
    \item Exprimez, lorsqu'elle existe, la matrice inverse \(A\inv\) en fonction de \(I_4\), \(A\), \(A^2\) et \(A^3\).
\end{enumerate}
\end{exo}

\begin{corr}
\note{À venir}
\end{corr}

\begin{exo}[Oral Centrale, 30]
Soient \(A,B\in\M{n}[\C]\) telles que \(\quantifs{\tpt M\in\M{n}[\C]}\chi_{AM+B}=\chi_{AM}\).

\begin{enumerate}
    \item Montrez qu'il existe un entier \(p\in\Ns\) tel que \(B^p=0\). \\
    \item Montrez que \(BA=0\). \\
    \item Réciproquement, soient \(A,B\in\M{n}[\C]\) telles que \(B^n=BA=0\). Montrez que \(\quantifs{\tpt M\in\M{n}[\C]}\chi_{AM+B}=\chi_{AM}\).
\end{enumerate}
\end{exo}

\begin{corr}
\note{À venir}
\end{corr}

\begin{exo}[Oral Centrale, 31]
Soient \(\theta_1,\dots,\theta_p\) des réels distincts modulo \(2\pi\) et \(m_1,\dots,m_p\) des complexes non-nuls. Le but de l'exercice est de montrer que la suite \(\paren{m_1\e{\i\theta_1n}+\dots+m_p\e{\i\theta_pn}}\) ne converge pas vers \(0\).

Par l'absurde, on suppose que \(m_1\e{\i\theta_1n}+\dots+m_p\e{\i\theta_pn}\tendqd{n\to\pinf}0\).

\begin{enumerate}
    \item On note \(M_n=\begin{pmatrix}
        \e{\i\theta_1n} & \dots & \e{\i\theta_pn} \\
        \vdots &  & \vdots \\
        \e{\i\theta_1\paren{n+p-1}} & \dots & \e{\i\theta_p\paren{n+p-1}}
    \end{pmatrix}\). Montrez que \(Y_n=M_n\tcoords{m_1}{\vdots}{m_p}\tendqd{n\to\pinf}0\). \\
    \item Montrez que \(\abs{\det M_n}\) est une constante non-nulle. \\
    \item À l'aide du théorème de Cayley-Hamilton, exprimez \(M\inv\) et trouvez une contradiction.
\end{enumerate}
\end{exo}

\begin{corr}
\note{À venir}
\end{corr}

\begin{exo}[Oral ENS, 32]
On considère un automorphisme \(\alpha\) de \(\M{n}[\C]\) qui conserve le produit matriciel, \cad tel que \(\quantifs{\tpt A,B\in\M{n}[\C]}\alpha\paren{AB}=\alpha\paren{A}\alpha\paren{B}\).

\begin{enumerate}
    \item Montrez que \(\alpha\paren{I_n}=I_n\). \\
    \item Soit \(A\in\M{n}[\C]\) diagonalisable. Montrez que \(\alpha\paren{A}\) l'est aussi. \\
    \item On suppose que \(A\) est semblable à une matrice diagonale \(D\) à coefficients diagonaux tous distincts. Montrez que \(\alpha\paren{A}\) est, elle aussi, semblable à \(D\). \\
    \item Justifiez l'existence de \(P\in\GL{n}[\C]\) telle que \(\alpha\paren{D}=PDP\inv\) puis montrez que \(\alpha\paren{E}=PEP\inv\) pour toute matrice diagonale \(E\). \\
    \item Déterminer \(\alpha\).
\end{enumerate}
\end{exo}

\begin{corr}
\note{À venir}
\end{corr}

\section*{Problème -- Matrices réelles sans valeur propre réelle}
\addcontentsline{toc}{section}{Problème -- Matrices réelles sans valeur propre réelle}

Dans ce problème, \(n\) désigne un entier naturel non-nul. Pour alléger les notations, on pose \(E_n=\M{n\,1}[\C]\). La lettre \(X\) est utilisée ici pour désigner une matrice-colonne, on évitera donc de l'utiliser pour désigner l'indéterminée des polynômes qui sera pour une fois notée \(x\).

Si \(\paren{a,b}\in\R^2\), on pose \(S\paren{a,b}=\begin{pmatrix}
a & -b \\
b & a
\end{pmatrix}\).

Toute matrice \(Z\) à coefficients complexes de \(\M{p\,q}[\C]\) peut s'écrire de manière unique sous la forme \(X+\i Y\) où \(X,Y\) sont deux matrices réelles de \(\M{p\,q}[\R]\). On appelle conjuguée de \(Z\) la matrice \(\conj{Z}=X-\i Y\).

D'après les propriétés de la conjugaison dans \(\C\), on en déduit que les règles de calcul sont les mêmes : \(\conj{Z_1+Z_2}=\conj{Z_1}+\conj{Z_2}\), \(\conj{\lambda Z}=\lambda\conj{Z}\) et \(\conj{Z_1.Z_2}=\conj{Z_1}.\conj{Z_2}\).

Soit \(M\in\M{n}[\R]\). On veut montrer l'équivalence entre les propositions suivante :

\begin{enumerate}
    \item[\(\paren{\alpha}\)] il existe \(\paren{a_1,b_1,\dots,a_m,b_m}\in\R^{2m}\) tel que \(M\) soit semblable dans \(\M{n}[\R]\) à la matrice diagonale par blocs \[\diag{S\paren{a_1,b_1},\dots,S\paren{a_m,b_m}}=\begin{pmatrix}
        S\paren{a_1,b_1} & 0 & \dots & 0 \\
        0 & \ddots & \ddots & \vdots \\
        \vdots & \ddots & \ddots & 0 \\
        0 & \dots & 0 & S\paren{a_m,b_m}
    \end{pmatrix}\]
    \item[\(\paren{\beta}\)] il existe un polynôme de \(\poly[\R][x]\) annulateur de \(M\) à racines simples non-réelles.
\end{enumerate}

\subsection*{Un cas particulier simple}
\addcontentsline{toc}{subsection}{Un cas particulier simple}

Soit \(\omega\) un complexe non-réel. On pose \(f:\C\to\C\) définie par \(f\paren{z}=\omega z\).

Vérifiez que \(f\) est un endomorphisme du \(\R\)-espace vectoriel \(\C\), que \(f\) a un polynôme annulateur réel à racines simples non-réelles et que sa matrice dans une base de \(\C\) bien choisie est une matrice \(S\paren{a,b}\).

\begin{corr}
\note{À venir}
\end{corr}

\subsection*{\(\paren{\beta}\imp\paren{\alpha}\)}
\addcontentsline{toc}{subsection}{\(\paren{\beta}\imp\paren{\alpha}\)}

\begin{q}
Soit \(\paren{a,b}\in\R^2\) tel que \(b\not=0\). Montrez qu'il existe un polynôme de \(\poly[\R][x]\) annulateur de \(S\paren{a,b}\) à racines simples non-réelles.
\end{q}

\begin{corr}
\note{À venir}
\end{corr}

\begin{q}
Montrez l'implication \(\paren{\beta}\imp\paren{\alpha}\).
\end{q}

\begin{corr}
\note{À venir}
\end{corr}

\subsection*{\(\paren{\alpha}\imp\paren{\beta}\)}
\addcontentsline{toc}{subsection}{\(\paren{\alpha}\imp\paren{\beta}\)}

Dans cette partie, \(M\) est une matrice de \(\M{n}[\R]\) qui annule un polynôme réel à racines simples non-réelles. On appelle \(f\) l'endomorphisme de \(E_n\) de matrice \(M\) dans la base canonique de \(E_n\) (autrement dit, \(\quantifs{\tpt Z\in E_n}f\paren{Z}=MZ\)).

\begin{q}
Soient \(z\in\Cs\) et \(\lambda\in\C\) tel que \(\conj{z}=\lambda z\). Montrez qu'il existe \(\theta\in\R\) tel que \(\lambda=\e{2\i\theta}\) et \(\e{\i\theta}z\) est un réel.
\end{q}

\begin{corr}
\note{À venir}
\end{corr}

\begin{q}
Montrez que \(n\) est pair. On note \(n=2m\).
\end{q}

\begin{corr}
\note{À venir}
\end{corr}

\begin{q}
Montrez que si \(\lambda\) est une valeur propre de \(M\), alors \(\conj{\lambda}\) en est une aussi, puis que les sous-espaces propres \(\sep{M}{\lambda}\) et \(\sep{M}{\conj{\lambda}}\) ont la même dimension.
\end{q}

\begin{corr}
\note{À venir}
\end{corr}

\begin{q}
Soient \(\lambda\in\C\) une valeur propre de \(M\) et \(Z\) un vecteur propre de \(M\).

\begin{enumerate}
    \item[(a)] Montrez que \(\Vect{Z,\conj{Z}}\) est un plan stable par \(f\). \\
    \item[(b)] On pose \(Z=X+\i Y\) où \(X,Y\) sont des matrices réelles. Montrez que \(\paren{X,Y}\) est une base de ce plan. \\
    \item[(c)] Montrez qu'il existe \(\paren{a,b}\in\R^2\) tel que \(b\not=0\) et l'endomorphisme induit par \(f\) dans ce plan a pour matrice \(S\paren{a,b}\) dans la base \(\paren{X,Y}\).
\end{enumerate}
\end{q}

\begin{corr}
\note{À venir}
\end{corr}

\begin{q}
Soit \(\lambda\in\C\) une valeur propre de \(M\). On choisit une base \(\paren{Z_1,\dots,Z_p}\) de \(\sep{M}{\lambda}\) et on écrit chaque vecteur sous la forme \(Z_k=X_k+\i Y_k\) où \(X_k,Y_k\) sont réelles.

Montrez que les plans \(\paren{\Vect{X_k,Y_k}}_{1\leq k\leq p}\) sont en somme directe et que \(\sep{M}{\lambda}\oplus\sep{M}{\conj{\lambda}}=\bigoplus_{k=1}^p\Vect{X_k,Y_k}\).
\end{q}

\begin{corr}
\note{À venir}
\end{corr}

\begin{q}
Montrez qu'il existe une base de \(E_n\) constituée de vecteurs réels dans laquelle la matrice de \(f\) est égale à une matrice diagonale par blocs \(D=\diag{S\paren{a_1,b_1},\dots,S\paren{a_m,b_m}}\).

Montrez enfin qu'il existe une matrice \(P\in\GL{n}[\R]\) telle que \(M=PDP\inv\).
\end{q}

\begin{corr}
\note{À venir}
\end{corr}

\subsection*{Un exemple}
\addcontentsline{toc}{subsection}{Un exemple}

On pose \(M=\begin{pmatrix}
-1 & -2 & 4 & 0 \\
1 & -3 & 0 & 4 \\
-2 & 0 & 5 & -2 \\
0 & -2 & 1 & 3
\end{pmatrix}\).

\begin{q}
On écrit \(xI_4-M\) par blocs \(\paren{2,2}\) : \(xI_4-M=\begin{pmatrix}
A & -4I_2 \\
2I_2 & B
\end{pmatrix}\).

\begin{enumerate}
    \item Calculez le produit par blocs \(\begin{pmatrix}-4I_2 & A \\ B & 2I_2\end{pmatrix}\begin{pmatrix}-2I_2 & 0 \\ B & I_2\end{pmatrix}\). Déduisez-en que \(\det\paren{xI_4-M}=\det\paren{AB+8I_2}\). \\
    \item Explicitez le polynôme caractéristique de \(M\). Déterminez ses racines imaginaires pures, puis les autres racines.
\end{enumerate}
\end{q}

\begin{corr}
\note{À venir}
\end{corr}

\begin{q}
Montrez que \(M\) annule un polynôme réel à racines simples non-réelles.

Vérifiez que les vecteurs-colonnes \(\begin{pmatrix}2 \\ 1+\i \\ 2 \\ 1+\i\end{pmatrix}\) et \(\begin{pmatrix}4 \\ 2-2\i \\ 2 \\ 1-\i\end{pmatrix}\) sont des vecteurs propres de \(M\), puis donnez une matrice \(P\in\GL{4}[\R]\) et une matrice réelle diagonale par blocs \(D\) telle que \(M=PDP\inv\).
\end{q}

\begin{corr}
\note{À venir}
\end{corr}

\subsection*{Application}
\addcontentsline{toc}{subsection}{Application}

Dans cette partie, on suppose que \(M\in\M{n}[\R]\) annule un polynôme de \(\poly[\R][x]\) à racines simples dans \(\C\). On garde les notations introduites précédemment.

On note \(\Sp[\R]{M}\) l'ensemble des valeurs propres réelles et \(\Sp[\C\excluant\R]{M}\) l'ensemble des valeurs propres complexes non-réelles de \(M\).

\begin{q}
On pose \(G=\bigoplus_{\lambda\in\Sp[\R]{M}}\sep{M}{\lambda}\) et \(H=\bigoplus_{\lambda\in\Sp[\C\excluant\R]{M}}\sep{M}{\lambda}\). Par convention, si \(M\) n'a pas de valeur propre réelle, alors \(G=\accol{0}\) et si \(M\) n'a que des valeurs propres réelles, alors \(H=\accol{0}\).

Montrez que \(G\) et \(H\) sont supplémentaires dans \(E_n\) et qu'ils sont stables par \(f\).

On note \(g\) et \(h\) les endomorphismes induits par \(f\) dans \(G\) et \(H\) respectivement.
\end{q}

\begin{corr}
\note{À venir}
\end{corr}

\begin{q}
Montrez qu'il existe une base \(\fami{B}\) de \(E_n\) telle que \(\Mat{f}=R=\begin{pmatrix}D & 0 \\ 0 & S\end{pmatrix}\) où \(D\) est une matrice diagonale réelle et \(S\) une matrice diagonale par blocs \(\diag{S\paren{a_1,b_1},\dots,S\paren{a_m,b_m}}\). On remarque que \(R\) est une matrice réelle.
\end{q}

\begin{corr}
\note{À venir}
\end{corr}

\begin{q}
Un exercice classique : soit \(\paren{U,V}\in\M{n}[\R]^2\).

\begin{enumerate}
    \item Justifiez que la fonction \(x\mapsto\det\paren{U+xV}\) est une fonction polynôme. \\
    \item Montrez que s'il existe \(z\in\C\) tel que \(\det\paren{U+zV}\not=0\), alors il existe \(x\in\R\) tel que \(\det\paren{U+xV}\not=0\).
\end{enumerate}

Application : soit \(\paren{A,B}\in\M{n}[\R]^2\) tel que \(A\) et \(B\) soient semblables dans \(\M{n}[\C]\) ; montrez qu'elles sont alors semblables dans \(\M{n}[\R]\).
\end{q}

\begin{corr}
\note{À venir}
\end{corr}

\begin{q}
Montrez finalement que \(M\) et \(R\) sont semblables dans \(\M{n}[\R]\).
\end{q}

\begin{corr}
\note{À venir}
\end{corr}


\include{exercices/chap6_integrales_generalisees/chap6_integrales_generalisees}

\include{exercices/chap7_integrales_parametre/chap7_integrales_parametre}

\include{exercices/chap8_espaces_prehilbertiens_reels/chap8_espaces_prehilbertiens_reels}

\chapter{Endomorphismes dans un espace euclidien}

\minitoc

\note{À venir}


\include{exercices/chap10_fonctions_vectorielles/chap10_fonctions_vectorielles}
\end{document}