\chapter{Suites et séries de fonctions}

\minitoc

\section*{Exercices}
\addcontentsline{toc}{section}{Exercices}

\legendeexercices

\begin{exoss}
Pour chacune des suites de fonctions \(\paren{f_n}\) suivantes, étudiez leur convergence simple puis leur convergence uniforme sur l'intervalle \(I\) spécifié, puis, à défaut, leur convergence uniforme sur tout segment inclus dans \(I\) :

\begin{enumerate}
    \item \(f_n:x\mapsto\dfrac{x}{n\paren{1+x^n}}\) sur \(I=\intervie{0}{\pinf}\) \\
    \item \(f_n:x\mapsto\ln\paren{x+\dfrac{1}{n}}\) sur \(I=\intervie{1}{\pinf}\) \\
    \item \(f_n:x\mapsto nx^2\e{-nx}\) sur \(I=\intervie{0}{\pinf}\) \\
    \item \(f_n:x\mapsto\dfrac{nx}{1+n^3x^2}\) sur \(I=\intervie{0}{\pinf}\) \\
    \item \(f_n:x\mapsto x^2\sin\dfrac{1}{nx}\) sur \(I=\intervee{0}{\pinf}\) \\
    \item \(f_n:x\mapsto\dfrac{nx+n^2x^3}{1+n^2x^2}\) sur \(I=\R\) \\
    \item \(f_n:x\mapsto\paren{\dfrac{x}{n}}^{nx}\) sur \(I=\intervee{0}{\pinf}\) \\
    \item \(f_n:x\mapsto nx^n\sin\paren{\pi x}\) sur \(I=\intervii{0}{1}\) \\
    \item \(f_n:x\mapsto\dfrac{n^3x}{n^4+x^4}\) sur \(I=\R\) \\
    \item \(f_n:x\mapsto\dfrac{x^n}{1+x^{2n}}\) sur \(I=\intervie{0}{\pinf}\) \\
    \item \(f_n:x\mapsto\cos\dfrac{nx}{n+1}\) sur \(I=\R\).
\end{enumerate}
\end{exoss}

\begin{exoss}
On pose \(f_n:x\mapsto\dfrac{\sin\paren{nx}}{n\sqrt{x}}\) sur \(\intervee{0}{\pinf}\).

Montrez que la suite \(\paren{f_n}_{n\geq1}\) converge simplement et uniformément sur \(\intervee{0}{\pinf}\).
\end{exoss}

\begin{exoss}
Soit \(f:\R\to\R\) de classe \(\classe{2}\) telle que \(f\seconde\) soit bornée.

Pour \(n\in\N\), on pose \(f_n:x\mapsto n\paren{f\paren{x+\dfrac{1}{n}}-f\paren{x}}\).

\begin{enumerate}
    \item Montrez que la suite \(\paren{f_n}_{n\geq1}\) converge simplement sur \(\R\). \\
    \item Montrez que la convergence est uniforme.
\end{enumerate}
\end{exoss}

\begin{exoss}
Soit \(f\) une fonction lipschitzienne sur \(\intervii{0}{1}\).

\begin{enumerate}
    \item Montrez que \(\quantifs{\tpt n\in\Ns\text{ et }x\in\intervii{0}{1}}x+\dfrac{x\paren{1-x}}{n}\in\intervii{0}{1}\). \\
    \item On pose \(f_n:x\mapsto f\paren{x+\dfrac{x\paren{1-x}}{n}}\). Montrez que la suite de fonctions \(\paren{f_n}\) converge uniformément sur \(\intervii{0}{1}\).
\end{enumerate}
\end{exoss}

\begin{exoss}
Pour \(x\geq0\), on pose \(u_0\paren{x}=0\) et \(u_{n+1}\paren{x}=\sqrt{x+u_n\paren{x}}\).

\begin{enumerate}
    \item Montrez que la suite de fonctions \(\paren{u_n}\) converge simplement vers une fonction \(f\) que vous expliciterez. \\
    \item On pose \(\phi_x:t\mapsto\sqrt{x+t}\). Vérifiez que \(\phi_x\) est \(\dfrac{1}{2\sqrt{x}}\)-lipschitzienne sur \(\intervie{0}{\pinf}\). \\
    \item Montrez que pour tout \(b>a>\dfrac{1}{4}\), la convergence de la suite \(\paren{u_n}\) est uniforme sur \(\intervii{a}{b}\).
\end{enumerate}
\end{exoss}

\begin{exoss}
Étudiez la nature des séries de fonctions \(\sum_{n\geq1}u_n\) (convergence simple, convergence normale, convergence uniforme) sur \(I\) ou, à défaut, sur tout segment inclus dans \(I\) :

\begin{enumerate}
    \item \(u_n:x\mapsto\dfrac{n+x}{n^3+x^2}\) sur \(I=\Rp\) \\
    \item \(u_n:x\mapsto\dfrac{1}{1+x^n}\) sur \(I=\Rp\) \\
    \item \(u_n:x\mapsto\dfrac{x}{1+n^2x^2}\) sur \(I=\R\) \\
    \item \(u_n:x\mapsto\dfrac{nx}{1+n^3x^2}\) sur \(I=\R\) \\
    \item \(u_n:x\mapsto\dfrac{\paren{-1}^nnx}{1+n^2x^2}\) sur \(I=\R\) \\
    \item \(u_n:x\mapsto\dfrac{n+x^2}{n^3+x^3}\) sur \(I=\Rp\) \\
    \item \(u_n:x\mapsto\dfrac{nx}{1+n^4x^2}\) sur \(I=\R\) \\
    \item \(u_n:x\mapsto\cos^n\paren{x}\sin\paren{x}\) sur \(I=\intervii{0}{\pi}\) \\
    \item \(u_n:x\mapsto\dfrac{x^n}{1+nx^{2n}}\) sur \(I=\Rp\) \\
    \item \(u_n:x\mapsto\Arctan\paren{n+x}-\Arctan n\) sur \(I=\R\) \\
    \item \(u_n:x\mapsto\dfrac{\paren{-1}^n}{nx+1}\) sur \(I=\Rp\).
\end{enumerate}
\end{exoss}

\begin{exoss}
Pour \(n\geq2\), on pose \(u_n:x\mapsto\dfrac{x\e{-nx}}{\ln n}\) sur \(\intervie{0}{\pinf}\).

\begin{enumerate}
    \item Montrez que la série de fonctions \(\sum_{n\geq2}u_n\) converge simplement sur \(\intervie{0}{\pinf}\). \\
    \item Montrez qu'elle ne converge pas normalement sur \(\intervie{0}{\pinf}\). \\
    \item Montrez qu'elle converge uniformément sur \(\intervie{0}{\pinf}\).
\end{enumerate}
\end{exoss}

\begin{exoss}
Même exercice avec \(u_n:x\mapsto\dfrac{1}{n+\paren{n\paren{x-n}}^2}\) pour \(n\geq1\).
\end{exoss}

\begin{exoss}
Pour \(n\in\Ns\), on pose \(u_n:x\mapsto\paren{-1}^nx^{2n}\ln x\) sur \(\intervei{0}{1}\) et \(u_n\paren{0}=0\).

\begin{enumerate}
    \item Montrez que la série \(\sum_{n\geq1}u_n\) converge simplement sur \(\intervii{0}{1}\). On note \(f\) sa somme, que vous calculerez explicitement. \\
    \item Montrez que la série \(\sum_{n\geq1}u_n\) converge uniformément sur \(\intervii{0}{1}\). \\
    \item Justifiez l'égalité \(\int_0^1\dfrac{x^2\ln x}{1+x^2}\odif{x}=\sum_{n=1}^{\pinf}\dfrac{\paren{-1}^n}{\paren{2n+1}^2}\). \\\\ Montrez la convergence de l'intégrale \(\int_0^1\dfrac{\ln x}{1+x^2}\odif{x}\) et donnez sa valeur sous forme d'une somme de série.
\end{enumerate}
\end{exoss}

\begin{exoss}
Pour \(n\in\Ns\), on pose \(u_n:x\mapsto\dfrac{1}{n}-\dfrac{1}{n+x}\) sur \(\intervee{-1}{\pinf}\).

\begin{enumerate}
    \item Montrez que la série \(\sum_{n\geq1}u_n\) converge simplement sur \(\intervee{-1}{\pinf}\). On note \(f\) sa somme. \\
    \item Montrez que pour tout \(a>-1\), la série \(\sum_{n\geq1}u_n\) converge uniformément sur \(\intervei{-1}{a}\). Déduisez-en que \(f\) est continue sur \(\intervee{-1}{\pinf}\). \\
    \item Précisez sa monotonie. Justifiez l'existence de limites pour \(f\) en \(-1\) et en \(\pinf\). Précisez la valeur de \(\lim_{x\to-1}f\paren{x}\). \\
    \item Montrez que \(\quantifs{\tpt N\in\Ns;\tpt x\in\intervie{0}{\pinf}}f\paren{x}\geq\sum_{n=1}^Nu_n\paren{x}\). Déduisez-en que \(\lim_{\pinf}f\geq\sum_{n=1}^N\dfrac{1}{n}\), puis précisez la valeur de \(\lim_{\pinf}f\). \\
    \item Donnez une écriture de \(\dfrac{f\paren{x}}{x}\) sous forme de série, justifiez que cette série converge uniformément sur \(\intervie{0}{\pinf}\) et déduisez-en \(\lim_{x\to\pinf}\dfrac{f\paren{x}}{x}\). \\
    \item Donnez l'allure de la courbe de \(f\).
\end{enumerate}
\end{exoss}

\begin{exoss}~\\
On pose \(f:x\mapsto\sum_{n=1}^{\pinf}\dfrac{\sqrt{x}\ln n}{1+n^2x}\).

\begin{enumerate}
    \item Montrez que \(f\) est définie sur \(\intervie{0}{\pinf}\). \\
    \item Montrez que \(f\) est continue sur \(\intervee{0}{\pinf}\) mais n'est pas continue en \(0\).
\end{enumerate}
\end{exoss}

\begin{exoss}
Reprenez les deux exercices précédents et étudiez la dérivabilité de la fonction somme.
\end{exoss}

\begin{exoss}
Pour \(n\in\Ns\), on pose \(u_n:x\mapsto\dfrac{\paren{-1}^n}{n^2x+n}\) sur \(\intervie{0}{\pinf}\).

\begin{enumerate}
    \item Montrez que la série \(\sum_{n\geq1}u_n\) converge simplement sur \(\intervie{0}{\pinf}\). On note \(f\) sa somme. \\
    \item Montrez que la série \(\sum_{n\geq1}u_n\) converge uniformément sur \(\intervie{0}{\pinf}\). \\
    \item Montrez que \(f\) est continue sur \(\intervie{0}{\pinf}\) et possède une limite réelle en \(\pinf\) que vous préciserez. \\
    \item Montrez que \(f\) est de classe \(\classe{1}\) sur \(\intervee{0}{\pinf}\).
\end{enumerate}
\end{exoss}

\begin{exoss}
Pour \(n\in\Ns\), on pose \(u_n:x\mapsto\dfrac{\Arctan\paren{nx}}{n^2}\) sur \(\R\).

\begin{enumerate}
    \item Montrez que la série \(\sum_{n\geq1}u_n\) converge simplement sur \(\R\). On note \(f\) sa somme. Précisez sa monotonie et montrez que \(f\) possède une limite réelle en \(\pinf\). \\
    \item Montrez que la série \(\sum_{n\geq1}u_n\) converge uniformément sur \(\R\). \\
    \item Montrez que \(f\) est continue sur \(\R\). \\
    \item Calculez la limite réelle en \(\pinf\). On rappelle que \(\sum_{k=1}^{\pinf}\dfrac{1}{k^2}=\dfrac{\pi^2}{6}\). (Une piste possible : utiliser \(\Arctan t+\Arctan\dfrac{1}{t}=\dfrac{\pi}{2}\) quand \(t>0\)). \\
    \item Montrez que \(f\) est de classe \(\classe{2}\) sur \(\Rs\).
\end{enumerate}
\end{exoss}

\begin{exoss}
Pour \(n\in\Ns\), on pose \(u_n:x\mapsto\dfrac{1}{n^2}\cos\dfrac{x}{n+x}\) sur \(\Rp\).

\begin{enumerate}
    \item Montrez que la série \(\sum_{n\geq1}u_n\) converge simplement sur \(\Rp\). On note \(f\) sa somme. \\
    \item Montrez que la série \(\sum_{n\geq1}u_n\) converge uniformément sur \(\Rp\). \\
    \item Montrez que \(f\) est continue sur \(\Rp\) et possède une limite réelle en \(\pinf\) que vous préciserez. \\
    \item Montrez que \(f\) est de classe \(\classe{1}\) sur \(\Rp\).
\end{enumerate}
\end{exoss}

\begin{exoss}
Pour \(x>0\), on pose \(f\paren{x}=\sum_{n=1}^{\pinf}\dfrac{1}{1+n^2x^2}\).

\begin{enumerate}
    \item Justifiez la bonne définition de \(f\). \\
    \item Montrez que \(f\) est continue sur \(\intervee{0}{\pinf}\) et possède une limite réelle en \(\pinf\) que vous préciserez. \\
    \item En étudiant la convergence de la série \(\sum_{n\geq1}\dfrac{x^2}{1+n^2x^2}\), donnez un équivalent de \(f\paren{x}\) quand \(x\) tend vers \(\pinf\). \\
    \item Précisez la monotonie de \(f\) et justifiez l'existence d'une limite \(l\) en \(0\). \\
    \item Montrez que \(\quantifs{\tpt N\in\Ns;\tpt x>0}f\paren{x}\geq\sum_{n=1}^N\dfrac{1}{1+n^2x^2}\). Déduisez-en que \(l=\pinf\). \\
    \item Montrez que \(f\) est de classe \(\classe{1}\) sur \(\intervee{0}{\pinf}\).
\end{enumerate}
\end{exoss}

\begin{exoss}
Pour \(x\geq0\), on pose \(f\paren{x}=\sum_{n=0}^{\pinf}\dfrac{\paren{-1}^n\e{-nx}}{n+1}\).

\begin{enumerate}
    \item Montrez que \(f\) est bien définie et continue sur \(\intervie{0}{\pinf}\). \\
    \item Montrez que \(f\) est de classe \(\classe{1}\) sur \(\intervee{0}{\pinf}\). \\
    \item Montrez que \(\quantifs{\tpt x>0}f\prim\paren{x}=f\paren{x}-\dfrac{1}{1+\e{-x}}\). \\\\ Déduisez-en que \(f\) est de classe \(\classe{1}\) sur \(\intervie{0}{\pinf}\). \\
    \item Montrez que \(g=f-1\) est intégrable sur \(\intervie{0}{\pinf}\). \\\\ Montrez que \(\int_0^{\pinf}g\paren{x}\odif{x}=\sum_{n=1}^{\pinf}\dfrac{\paren{-1}^n}{n\paren{n+1}}=1-2\ln 2\) (on admettra que \(\sum_{k=1}^{\pinf}\dfrac{\paren{-1}^{k-1}}{k}=\ln 2\) : démonstration ultérieure).
\end{enumerate}
\end{exoss}

\begin{exoss}
Pour \(x>0\), on pose \(f\paren{x}=\sum_{n=0}^{\pinf}\dfrac{\paren{-1}^n}{nx+1}\).

\begin{enumerate}
    \item Justifiez la bonne définition de \(f\). \\
    \item Montrez que \(f\) est continue sur \(\intervee{0}{\pinf}\) et possède une limite réelle en \(\pinf\) que vous préciserez. \\
    \item Montrez que \(f\) est de classe \(\classe{1}\) sur \(\intervee{0}{\pinf}\).
\end{enumerate}
\end{exoss}

\begin{exoss}~\\
On pose \(f\paren{x}=\sum_{n=0}^{\pinf}\ln\paren{1+x^n}\).

\begin{enumerate}
    \item Quel est l'ensemble de définition de \(f\) ? \\
    \item Montrez que \(f\) est continue sur \(\intervie{0}{1}\). \\
    \item Quelle est la limite de \(f\) en \(1^-\) ? \\
    \item Montrez que \(f\) est de classe \(\classe{1}\) sur \(\intervee{-1}{1}\).
\end{enumerate}
\end{exoss}

\begin{exoss}~\\
On pose \(f\paren{x}=\sum_{n=1}^{\pinf}\dfrac{\e{-x\sqrt{n}}}{n^{\nicefrac{3}{2}}}\).

\begin{enumerate}
    \item Montrez que \(f\) est bien définie sur \(\intervie{0}{\pinf}\). \\
    \item Montrez que \(f\) est de classe \(\classe{1}\) sur \(\intervee{0}{\pinf}\) et calculez \(f\prim\paren{x}\) pour \(x>0\). \\
    \item Montrez que \(f\) n'est pas dérivable en \(0\). \\
    \item Donnez un équivalent de \(f\paren{x}\) quand \(x\to\pinf\).
\end{enumerate}
\end{exoss}

\begin{exosss}
Étudiez les propriétés de la fonction \(f:x\mapsto\sum_{n=1}^{\pinf}\dfrac{\paren{-1}^nx^n}{n\paren{1+x^n}}\).

\textit{Indication : trouver une relation entre \(f\paren{\dfrac{1}{x}}\) et \(f\paren{x}\).}
\end{exosss}

\begin{exo}[Oral CCINP, 1]
Pour \(n\in\N\), on pose \(f_n:x\in\intervii{-1}{1}\mapsto\sin\paren{nx\e{-nx^2}}\).

\begin{enumerate}
    \item Montrez que \(\paren{f_n}\) converge simplement vers une fonction \(f\) sur \(\intervee{-1}{1}\). \\
    \item Montrez qu'il y a convergence uniforme sur \(\intervii{a}{1}\) pour tout \(a>0\). \\
    \item Y a-t-il convergence uniforme sur \(\intervii{-1}{1}\) ?
\end{enumerate}
\end{exo}

\begin{exo}[Oral CCINP, 2]~\\
\begin{enumerate}
    \item Montrez que \(\quantifs{\tpt t\in\intervii{-\dfrac{1}{2}}{\dfrac{1}{2}}}\abs{\ln\paren{1+t}-t}\leq2t^2\). \\
    \item Étudiez la convergence simple et uniforme sur \(\R\) de la série de fonctions \(\sum x\mapsto\ln\paren{1+\dfrac{\paren{-1}^nx}{n\paren{1+x^2}}}\).
\end{enumerate}
\end{exo}

\begin{exo}[Oral IMT, 3]
Soit \(\paren{a_n}\) une suite décroissante de réels positifs. Pour \(n\in\Ns\) et \(x>0\), on pose \(f_n\paren{x}=a_nx^n\paren{1-x}\).

\begin{enumerate}
    \item Montrez que la série \(\sum f_n\) converge simplement sur \(\intervii{0}{1}\). \\
    \item Montrez que la série \(\sum f_n\) converge normalement sur \(\intervii{0}{1}\) ssi \(\sum\dfrac{a_n}{n}\) converge. \\
    \item Montrez que la série \(\sum f_n\) converge uniformément sur \(\intervii{0}{1}\) ssi \(\paren{a_n}\) converge vers \(0\).
\end{enumerate}
\end{exo}

\begin{exo}[Oral TPE, 4]~\\
Soit \(f\paren{x}=\sum_{n=0}^{\pinf}\dfrac{1}{n!\;\!\paren{n+x}}\).

\begin{enumerate}
    \item Étudiez la convergence de la série sur \(\intervee{0}{\pinf}\). \\
    \item Calculez \(f\paren{1}\). \\
    \item La fonction \(f\) est-elle de classe \(\classe{1}\) ? \\
    \item Exprimez \(f\paren{x+1}\) en fonction de \(f\paren{x}\).
\end{enumerate}
\end{exo}

\begin{exo}[Oral IMT, 5]~\\
Soit \(f\paren{x}=\sum_{n=0}^{\pinf}\dfrac{\e{-nx}}{n^2+1}\).

\begin{enumerate}
    \item Déterminez le domaine de définition de \(f\). \\
    \item Étudiez la continuité de \(f\). \\
    \item Montrez que \(f\) est de classe \(\classe{\infty}\) sur \(\intervee{0}{\pinf}\). \\
    \item Déterminez la limite \(l\) de \(f\) en \(\pinf\) et donnez un équivalent de \(f\paren{x}-l\). \\
    \item Étudiez les variations de \(f\).
\end{enumerate}
\end{exo}

\begin{exo}[Oral CCINP, 6]
Pour \(x\in\R\) et \(n\in\Ns\), on pose \(u_n\paren{x}=\paren{-1}^n\dfrac{\e{-nx}}{n}\).

\begin{enumerate}
    \item Étudiez la convergence de \(\sum u_n\). On note \(S=\sum_{n=1}^{\pinf}u_n\). \\
    \item Montrez que \(S\) est continue sur \(\Rp\). \\
    \item Montrez que \(S\) est de classe \(\classe{1}\) sur \(\Rps\). \\
    \item Calculez \(S\).
\end{enumerate}
\end{exo}

\begin{exo}[Oral CCINP, 7]
Pour \(x>0\) et \(n\geq2\), on pose \(u_n\paren{x}=\dfrac{\ln x}{x^n\ln n}\).

\begin{enumerate}
    \item Déterminez le domaine \(D\) de convergence de \(\sum u_n\). \\
    \item Montrez que la série ne converge pas normalement. \\
    \item Montrez que le reste d'ordre \(n\) de la série vérifie \(\abs{R_n\paren{x}}\leq\dfrac{1}{\ln\paren{n+1}}\). \\
    \item Étudiez la continuité de \(S=\sum_{n=2}^{\pinf}u_n\) sur \(D\). \\
    \item Montrez que \(S\) est intégrable sur \(D\).
\end{enumerate}
\end{exo}

\begin{exo}[Oral IMT, 8]~\\
Soit \(f_n:x\mapsto\paren{x^2+1}\dfrac{n\e{x}+x\e{-x}}{n+x}\).

\begin{enumerate}
    \item Montrez que la suite \(\paren{f_n}\) converge uniformément sur \(\intervii{0}{1}\) vers une fonction \(f\) à déterminer. \\
    \item Calculez \(\lim_{n\to\pinf}\int_0^1f_n\).
\end{enumerate}
\end{exo}

\section*{Problème 1 -- Une série de fonctions}
\addcontentsline{toc}{section}{Problème 1 -- Une série de fonctions}
\setcounter{q}{0}

Soit \(\alpha\) un réel strictement positif. Pour \(n\) un entier naturel non-nul, on considère l'application \(u_n\) de \(\intervie{0}{\pinf}\) vers \(\R\) définie par \[u_n\paren{x}=\dfrac{x}{n^\alpha\paren{1+nx^2}}.\]

\begin{q}[Étude des modes de convergence de la série de fonctions \(\sum u_n\)]
\begin{enumerate}
    \item Montrer que la série \(\sum u_n\) converge simplement sur \(\intervie{0}{\pinf}\). \\
    \item Soient \(a\) et \(b\) deux réels tels que \(0<a<b\). Prouver que la série \(\sum u_n\) converge normalement sur \(\intervii{a}{b}\). \\
    \item Démontrer que la série \(\sum u_n\) converge normalement sur \(\intervie{0}{\pinf}\) ssi \(\alpha>\dfrac{1}{2}\). \\
    \item Montrer que, pour tout \(a>0\), la série \(\sum u_n\) converge normalement sur \(\intervie{a}{\pinf}\).
\end{enumerate}
\end{q}

\begin{q}
On suppose dans cette question que \(\alpha\leq\dfrac{1}{2}\).

Pour \(x\in\intervee{0}{\pinf}\), on pose \(R_n\paren{x}=\sum_{k=n+1}^{\pinf}u_k\paren{x}\).

\begin{enumerate}
    \item Établir l'inégalité \[R_n\paren{x}\geq\sum_{k=n+1}^{2n}\dfrac{x}{\sqrt{2n}\paren{1+kx^2}}.\]
    \item En procédant par comparaison série-intégrale, montrer que \[R_n\paren{x}\geq\dfrac{1}{\sqrt{2n}x}\ln\dfrac{1+\paren{2n+1}x^2}{1+\paren{n+1}x^2}.\]
    \item En déduire que la série \(\sum u_n\) n'est pas uniformément convergente sur \(\intervii{0}{a}\), où \(a\) est un réel strictement positif.
\end{enumerate}
\end{q}

On note \(S\) l'application de \(\intervie{0}{\pinf}\) dans \(\R\) définie par \(S\paren{x}=\sum_{n=1}^{\pinf}u_n\paren{x}\).

Étudions la régularité de \(S\).

\begin{q}
Montrer que si \(\alpha>\dfrac{1}{2}\), alors la fonction \(S\) est continue sur \(\intervie{0}{\pinf}\).

Quand \(0<\alpha\leq\dfrac{1}{2}\), sur quel intervalle peut-on démontrer sans calcul supplémentaire que la fonction \(S\) est continue ?
\end{q}

\begin{q}
Déterminer \(\lim_{x\to\pinf}S\paren{x}\).
\end{q}

\begin{q}
Montrer que pour tout \(\alpha>0\), la fonction \(S\) est de classe \(\classe{1}\) sur \(\intervee{0}{\pinf}\).
\end{q}

\begin{q}
On suppose que \(0<\alpha\leq\dfrac{1}{2}\). Soit \(x\) un réel strictement positif.

\begin{enumerate}
    \item Calculer \(\lim_{n\to\pinf}\int_1^n\dfrac{x}{\sqrt{t}\paren{1+tx^2}}\odif{t}\). \\
    \item Montrer alors \(\pi-2\Arctan x\leq S\paren{x}\). \\
    \item En déduire que \(S\) n'est pas continue en \(0\).
\end{enumerate}
\end{q}

\section*{Problème 2 -- Étude aux extrémités}
\addcontentsline{toc}{section}{Problème 2 -- Étude aux extrémités}
\setcounter{q}{0}

D'après CCP PC 2013.

\begin{q}[Un premier exemple]
\begin{enumerate}
    \item Pour tout \(x\in\intervee{-1}{1}\), calculer \(F\paren{x}=\sum_{n=1}^{\pinf}x^n\) ainsi que \(F\prim\paren{x}\). \\
    \item Déterminer les limites suivantes : \[\lim_{x\to1^-}F\paren{x}\qquad\lim_{x\to1^-}\paren{1-x}F\paren{x}\qquad\lim_{x\to1^-}\paren{1-x}F\prim\paren{x}\qquad\lim_{x\to1^-}\paren{1-x}^2F\prim\paren{x}.\]
\end{enumerate}
\end{q}

\begin{q}[Un deuxième exemple]
Dans cette question, pour tout \(x\in\intervee{-1}{1}\), on pose cette fois \(F\paren{x}=\sum_{n=1}^{\pinf}\dfrac{x^n}{1-x^n}\).

\begin{enumerate}
    \item Soit \(a\in\intervee{0}{1}\). Prouver la convergence normale de cette série de fonctions sur le segment \(\intervii{-a}{a}\). En déduire que \(F\) est définie et continue sur \(\intervee{-1}{1}\). \\
    \item Montrer que \(\quantifs{\tpt x\in\intervee{0}{1};\tpt n\in\Ns}\dfrac{1-x^n}{1-x}\leq n\). \\\\ En déduire \(\lim_{x\to1^-}F\paren{x}\) et \(\lim_{x\to1^-}\paren{1-x}F\paren{x}\).
\end{enumerate}
\end{q}

\begin{q}[Cas général]
Dans cette question, \(f\) est une application réelle continue et croissante sur \(\intervee{0}{1}\) avec \(f\paren{0}=0\) et telle que \(u\mapsto\dfrac{f\paren{u}}{u}\) soit intégrable sur \(\intervee{0}{1}\). Soit \(x\in\intervee{0}{1}\).

\begin{enumerate}
    \item Justifier l'existence de \(G\paren{x}=\int_0^{\pinf}f\paren{x^t}\odif{t}\) et l'égalité \[G\paren{x}=-\dfrac{1}{\ln x}\int_0^1\dfrac{f\paren{u}}{u}\odif{u}.\]
    \item Pour tout \(n\in\Ns\), justifier l'encadrement \[\int_n^{n+1}f\paren{x^t}\odif{t}\leq f\paren{x^n}\leq\int_{n-1}^nf\paren{x^t}\odif{t}.\]
    \item En déduire l'existence de \(F\paren{x}=\sum_{n=1}^{\pinf}f\paren{x^n}\), ainsi qu'un encadrement de \(F\paren{x}\) par deux intégrales dépendant de \(x\). \\
    \item Conclure avec soin que \[\lim_{x\to1^-}\paren{1-x}F\paren{x}=\int_0^1\dfrac{f\paren{u}}{u}\odif{u}.\]
\end{enumerate}
\end{q}

\begin{q}[Calcul d'une intégrale]
On cherche à déterminer la valeur de l'intégrale généralisée \(I=\int_0^1\dfrac{\ln\paren{1-t}}{t}\odif{t}\).

\begin{enumerate}
    \item Justifier que la fonction \(t\mapsto\dfrac{\ln\paren{1-t}}{t}\) est prolongeable en une fonction \(\psi\) continue sur \(\intervie{0}{1}\). \\\\ On admettra que \(\quantifs{\tpt t\in\intervei{-1}{1}}\ln\paren{1+t}=\sum_{k=1}^{\pinf}\dfrac{\paren{-1}^{k-1}t^k}{k}\). \\
    \item Soit \(x\in\intervie{0}{1}\). Démontrer que \[\int_0^x\psi\paren{t}\odif{t}=-\sum_{n=1}^{\pinf}\dfrac{x^n}{n^2}.\]
    \item En déduire soigneusement que l'intégrale \(I\) est convergente et préciser sa valeur.
\end{enumerate}
\end{q}

\begin{q}[Un dernier exemple]
Pour tout \(x\in\intervee{-1}{1}\), on pose cette fois \(F\paren{x}=-\sum_{n=1}^{\pinf}\ln\paren{1-x^n}\).

\begin{enumerate}
    \item Montrer que \(F\) est définie et de classe \(\classe{1}\) sur \(\intervee{-1}{1}\) et exprimer sa dérivée sous la forme d'une série de fonctions. \\
    \item Grâce à la fonction 3d, montrer que \[\lim_{x\to1^-}\paren{1-x}F\paren{x}=\int_0^1\dfrac{\ln u}{u-1}\odif{u}=I.\]
    \item Par une méthode similaire à celle de la question 3, montrer que \[\lim_{x\to1^-}\paren{1-x}^2\sum_{n=1}^{\pinf}\dfrac{nx^n}{1-x^n}=I.\] En déduire \(\lim_{x\to1^-}\paren{1-x}^2F\prim\paren{x}\).
\end{enumerate}
\end{q}
