\chapter{Endomorphismes dans un espace euclidien}\label{exos_ee}

\minitoc

\section*{Exercices}
\addcontentsline{toc}{section}{Exercices}

\legendeexercices

\begin{exos}
Soit \(E=\polydeg[\R]{n}\). On définit \(\ps{P}{Q}=\sum_{k=0}^nP\paren{k}Q\paren{k}\).

\begin{enumerate}
    \item Montrez que \(\ps{}{}\) est un produit scalaire. \\
    \item Dans le cas où \(n=3\), donnez une base orthonormée de \(\polydeg[\R]{2}\). \\
    \item Calculez la distance de \(X^3\) à \(\polydeg[\R]{2}\).
\end{enumerate}
\end{exos}

\begin{exos}
Soit \(E=\M{n}[\R]\) muni du produit scalaire classique \(\paren{A,B}\mapsto\tr\paren{\trans{A}B}\).

\begin{enumerate}
    \item Montrez que si \(A\) est symétrique et \(B\) est antisymétrique, alors \(A\) et \(B\) sont orthogonales pour ce produit scalaire. \\
    \item Déterminez \(\sym{n}[\R]\ortho\). \\
    \item Dans le cas où \(n=2\), calculez la distance de la matrice \(A=\begin{pmatrix}
        1 & 0 \\
        -1 & 2
    \end{pmatrix}\) au sous-espace vectoriel des matrices triangulaires supérieures.
\end{enumerate}
\end{exos}

\begin{exoss}
Soit \(E=\poly[\R]\). On définit \(\ps{P}{Q}=\sum_{k=0}^{\pinf}\dfrac{P\paren{k}Q\paren{k}}{2^k}\).

\begin{enumerate}
    \item Montrez que \(\ps{}{}\) est un produit scalaire. \\
    \item Donnez une base orthonormée du sous-espace \(\polydeg[\R]{2}\). \\
    \item Calculez la distance de \(X^3\) à \(\Vect{1,X,X^2}\).
\end{enumerate}
\end{exoss}

\begin{exoss}
Même exercice avec \(\ps{P}{Q}=\int_0^{\pinf}P\paren{t}Q\paren{t}\e{-t}\odif{t}\).
\end{exoss}

\begin{exoss}
\begin{enumerate}
    \item Montrez que pour tout \(n\in\N\), il existe un unique polynôme \(T_n\) tel que \(\quantifs{\tpt t\in\R}T_n\paren{\cos t}=\cos\paren{nt}\). \\
    \item Calculez \(T_0\) et \(T_1\) puis montrez que \(\quantifs{\tpt n\in\N}T_{n+2}=2XT_{n+1}-T_n\). En déduire le degré et le coefficient dominant de \(T_n\). \\
    \item Montrez que l'application \(\paren{P,Q}\mapsto\int_{-1}^1\dfrac{P\paren{t}Q\paren{t}}{\sqrt{1-t^2}}\odif{t}\) est un produit scalaire sur \(\poly[\R]\). \\
    \item Montrez que la famille \(\paren{T_n}\) est orthogonale pour ce produit scalaire.
\end{enumerate}
\end{exoss}

\begin{exoss}
Soit \(E\) l'ensemble des suites réelles \(x\) telles que la série \(\sum_nx_n^2\) converge.

\begin{enumerate}
    \item Vérifiez rapidement que \(\quantifs{\tpt\paren{a,b}\in\R^2}\abs{ab}\leq\dfrac{a^2+b^2}{2}\). \\
    \item Soient \(x,y\in E\). Montrez que la série \(\sum_nx_ny_n\) est absolument convergente. \\
    \item Montrez que \(E\) est un \(\R\)-espace vectoriel. \\
    \item On pose \(\ps{x}{y}=\sum_{n=0}^{\pinf}x_ny_n\). Montrez qu'on définit ainsi un produit scalaire sur \(E\).
\end{enumerate}
\end{exoss}

\begin{exos}
Soit \(E\) un espace euclidien de dimension \(3\) muni d'une base orthonormée dans laquelle toutes les coordonnées sont exprimées. Soient \(u\) de coordonnées \(\paren{1,3,1}\), \(v\) de coordonnées \(\paren{5,2,-1}\) et \(w\) de coordonnées \(\paren{1,1,1}\).

\begin{enumerate}
    \item Donnez un système d'équation et une base d'un supplémentaire de chacun des sous-espaces vectoriels \(\Vect{u}\) et \(\Vect{u,v}\). \\
    \item Déterminez la projection orthogonale de \(v\) sur la droite \(\Vect{u}\). \\
    \item Déterminez la projection orthogonale de \(w\) sur le plan \(\Vect{u,v}\). \\
    \item Déterminez la matrice du projecteur orthogonal sur le plan précédent.
\end{enumerate}
\end{exos}

\begin{exos}
Soient \(E\) un espace euclidien de dimension \(4\) muni d'une base orthonormée dans laquelle toutes les coordonnées sont exprimées et \(F=\accol{x\in E\tq x_1+x_2+x_3+x_4=0\text{ et }x_1-x_2+x_3-x_4=0}\).

\begin{enumerate}
    \item Justifiez que \(F\) est un sous-espace vectoriel de \(E\). Donnez une base orthonormée de \(F\). \\
    \item Soit \(v\) de coordonnées \(\paren{2,3,1,-1}\). Déterminez sa projection orthogonale sur \(F\). \\
    \item Déterminez la matrice de la symétrie orthogonale par rapport à \(F\).
\end{enumerate}
\end{exos}

\begin{exoss}
Soit \(E=\Vect{\id{\intervii{0}{\pi}},\sin,\cos}\) muni du produit scalaire \(\ps{f}{g}=\int_0^\pi f\paren{t}g\paren{t}\odif{t}\).

Déterminez \(\inf_{\paren{a,b}\in\R^2}\int_0^\pi\paren{x-\paren{a\cos x+b\sin x}}^2\odif{x}\).
\end{exoss}

\begin{exoss}
\begin{enumerate}
    \item Déterminez \(\inf_{x\in\R}\paren{\paren{2x-b_1}^2+\paren{3x-b_2}^2+\paren{4x-b_3}^2}\) où \(b_1,b_2,b_3\) sont fixés dans \(\R\). \\
    \item Déterminez \(\inf_{\paren{a,b}\in\R^2}\sum_{k=1}^n\paren{k^2-ak-b}^2\). On admet \(\sum_{k=1}^nk^3=\dfrac{n^2\paren{n+1}^2}{4}\) et \\ \(\sum_{k=1}^nk^4=\dfrac{n\paren{n+1}\paren{2n+1}\paren{3n^2+3n-1}}{30}\). \\
    \item Déterminez \(\inf_{\paren{a,b}\in\R^2}\int_0^1\paren{x^2-ax-b}^2x^2\odif{x}\). \\
    \item Déterminez \(\inf_{\paren{a,b,c}\in\R^3}\int_{-\pi}^\pi\paren{t-a\sin t-b\cos t-c}^2\odif{t}\).
\end{enumerate}
\end{exoss}

\begin{exoss}
Soient \(E\) un espace euclidien et \(\paren{e_1,\dots,e_n}\in E^n\) des vecteurs unitaires tels que \[\quantifs{\forall x\in E}\norme{x}^2=\sum_{k=1}^n\ps{x}{e_k}^2.\]

Montrez que \(\paren{e_1,\dots,e_n}\) est une base orthonormale de \(E\).
\end{exoss}

\begin{exoss}
Soient \(E\) un espace euclidien et \(F,G\) deux sous-espaces vectoriels de \(E\).

Montrez que \(\paren{F\ortho}\ortho=F\), \(\paren{F+G}\ortho=F\ortho\inter G\ortho\) et \(\paren{F\inter G}\ortho=F\ortho+G\ortho\).
\end{exoss}

\begin{exosss}
On munit \(E=\ensclasse{0}{\intervii{0}{1}}{\R}\) du produit scalaire \(\ps{f}{g}=\int_0^1f\paren{t}g\paren{t}\odif{t}\).

Soit \(F=\accol{f\in E\tq f\paren{0}=0}\).

A-t-on \(E=F\oplus F\ortho\) ? A-t-on \(\paren{F\ortho}\ortho=F\) ?

\textit{Indication : si \(f\in F\ortho\), considérez la fonction \(t\mapsto tf\paren{t}\).}
\end{exosss}

\begin{exoss}
Soient \(E\) un espace euclidien et \(p\in\Lendo{E}\) un projecteur.

On sait d'après le cours que si \(p\) est un projecteur orthogonal, alors \(\quantifs{\tpt x\in E}\norme{p\paren{x}}\leq\norme{x}\).

Montrez que la réciproque est vraie, en utilisant le vecteur \(y+\lambda x\), où \(x\in\Im p\), \(y\in\ker p\) et \(\lambda\in\R\).
\end{exoss}

\begin{exoss}
Soient \(E\) un espace euclidien de dimension \(3\) rapporté à une base orthonormée \(\fami{B}\) et \(f\in\Lendo{E}\) tel que \(\Mat{f}=A=\dfrac{1}{15}\begin{pmatrix}
8 & 6 & -10 \\
-10 & 5 & 0 \\
6 & -8 & 5
\end{pmatrix}\).

\begin{enumerate}
    \item Donnez une base du noyau et de l'image de \(f\). Vérifiez que ces deux sous-espaces de \(E\) sont supplémentaires orthogonaux. \\
    \item Donnez une base orthonormée directe \(\fami{B}\prim\) de \(E\) dont les vecteurs sont choisis dans le noyau ou l'image de \(f\) (dans cet ordre). \\
    \item Montrez que la matrice de \(f\) dans cette nouvelle base est de la forme \(\begin{pmatrix}
        0 & 0 & 0 \\
        0 & \cos\alpha & -\sin\alpha \\
        0 & \sin\alpha & \cos\alpha
    \end{pmatrix}\). \\
    \item Donnez une interprétation géométrique de l'application \(f\).
\end{enumerate}
\end{exoss}

\begin{exos}
Soit \(E\) un espace euclidien orienté de dimension \(3\) rapporté à une base orthonormée directe.

Déterminez la nature et les éléments caractéristiques de chacune des applications linéaires dont les matrices sont : \[A=\dfrac{1}{3}\begin{pmatrix}
-2 & -1 & 2 \\
2 & -2 & 1 \\
1 & 2 & 2
\end{pmatrix}\qquad B=\dfrac{1}{9}\begin{pmatrix}
7 & 4 & 4 \\
-4 & 8 & -1 \\
4 & 1 & -8
\end{pmatrix}\qquad C=\dfrac{1}{9}\begin{pmatrix}
-8 & 4 & 1 \\
4 & 7 & 4 \\
1 & 4 & -8
\end{pmatrix}\]

\[D=\dfrac{1}{4}\begin{pmatrix}
3 & 1 & \sqrt{6} \\
1 & 3 & -\sqrt{6} \\
-\sqrt{6} & \sqrt{6} & 2
\end{pmatrix}\qquad E=\dfrac{1}{4}\begin{pmatrix}
3 & 1 & -\sqrt{6} \\
1 & 3 & \sqrt{6} \\
-\sqrt{6} & \sqrt{6} & -2
\end{pmatrix}.\]
\end{exos}

\begin{exoss}
Soit \(E\) un espace euclidien orienté de dimension \(3\) rapporté à une base orthonormée directe.

Déterminez la matrice de la rotation d'axe orienté par \(u\) de coordonnées \(\paren{1,1,-1}\) et d'angle \(\dfrac{\pi}{2}\).
\end{exoss}

\begin{exoss}
Soient \(E\) un espace euclidien orienté de dimension \(3\) rapporté à une base orthonormée directe, \(u\) un vecteur unitaire, \(\theta\) un réel et \(f\) la rotation d'axe orienté par \(u\) et d'angle \(\theta\).

Montrez que \(\quantifs{\tpt x\in E}f\paren{x}=\paren{1-\cos\theta}\ps{x}{u}u+\cos\paren{\theta}x+\sin\paren{\theta}u\vecto x\).
\end{exoss}

\begin{exoss}
Déterminez \(\Card\paren{\Orth{n}[\R]\inter\M{n}[\Z]}\).
\end{exoss}

\begin{exoss}
Soit \(E\) un espace euclidien et \(u\) un automorphisme orthogonal de \(E\). On pose \(v=\id{E}-u\).

\begin{enumerate}
    \item Montrez que \(\ker v=\paren{\Im v}\ortho\). \\
    \item Soit \(p\) la projection orthogonale sur \(\ker v\). Montrez que \[\quantifs{\forall x\in E}\dfrac{1}{n}\sum_{k=0}^{n-1}u^k\paren{x}\tendqd{n\to\pinf}p\paren{x}.\]
\end{enumerate}
\end{exoss}

\begin{exos}
Soit \(A\in\M{n}[\R]\) telle que \(\trans{A}=A\).

\begin{enumerate}
    \item Montrez que s'il existe \(k\in\N\) tel que \(A^k=I_n\), alors \(A^2=I_n\). Que peut-on dire de mieux si \(k\) est impair ? \\
    \item Montrez que si \(A\) est nilpotente, alors \(A=0\).
\end{enumerate}
\end{exos}

\begin{exos}
Soient \(S=\paren{s_{i\,j}}\) une matrice symétrique de \(\M{n}[\R]\) et \(\lambda_1,\dots,\lambda_n\) ses valeurs propres.

Montrez que \(\sum_{1\leq i,j\leq n}s_{i\,j}^2=\sum_{i=1}^n\lambda_i^2\).
\end{exos}

\begin{exoss}
Soient \(n\geq3\) et \(A=\begin{pmatrix}
1 & 1 & \dots & 1 & 1 \\
1 & 0 & \dots & 0 & 1 \\
\vdots & \vdots & 0 & \vdots & \vdots \\
1 & 0 & \dots & 0 & 1 \\
1 & 1 & \dots & 1 & 1
\end{pmatrix}\in\M{n}[\R]\).

\begin{enumerate}
    \item Justifiez que \(A\) est diagonalisable, puis que \(A\) possède au plus trois valeurs propres distinctes dont l'une est \(0\). \\
    \item En notant \(\lambda\) et \(\mu\) les deux autres valeurs propres (pouvant être égales), donnez deux équations reliant \(\lambda\) et \(\mu\), et déduisez-en le spectre de \(A\).
\end{enumerate}
\end{exoss}

\begin{exoss}
Soit \(A\in\M{n}[\R]\).

Montrez que \(A\trans{A}\) et \(\trans{A}A\) sont orthosemblables.
\end{exoss}

\begin{exoss}
Déterminez toutes les matrices \(A\in\M{n}[\R]\) telles que \[A\trans{A}A=I_n.\]
\end{exoss}

\begin{exoss}
Soit \(n\in\Ns\). On munit \(E=\polydeg[\R]{n}\) du produit scalaire \[\ps{P}{Q}=\int_{-1}^1P\paren{t}Q\paren{t}\odif{t}.\]

\begin{enumerate}
    \item Montrez que l'application \(u:P\mapsto\paren{1-X^2}P\seconde-2XP\prim\) est un endomorphisme auto-adjoint de \(E\). \\
    \item Montrez que les valeurs propres de \(u\) sont négatives.
\end{enumerate}
\end{exoss}

\begin{exoss}
Soient \(E\) un espace euclidien, \(u_1,\dots,u_p\in E\) et \(f\) l'application de \(E\) dans \(E\) telle que \[f:x\mapsto\sum_{k=1}^p\ps{u_k}{x}u_k.\]

\begin{enumerate}
    \item Quelle est l'application \(f\) lorsque \(\paren{u_1,\dots,u_p}\) est une famille orthonormée ? \\
    \item Dans le cas général, montrez que \(f\) est un endomorphisme auto-adjoint positif de \(E\). \\
    \item Déterminez \(\ker f\) et \(\Im f\).
\end{enumerate}
\end{exoss}

\begin{exoss}
Rappel : une matrice symétrique \(A\in\sym{n}[\R]\) est dite positive quand \(\quantifs{\forall X\in\M{n\,1}[\R]}\trans{X}AX\geq0\).

Soit \(A\in\sym{n}[\R]\). Montrez l'équivalence des propriétés suivantes :

\begin{enumerate}
    \item[(\(\alpha\))] \(A\) est positive \\
    \item[(\(\beta\))] toutes les valeurs propres de \(A\) sont positives \\
    \item[(\(\gamma\))] il existe \(M\in\M{n}[\R]\) telle que \(A=\trans{M}M\) \\
    \item[(\(\delta\))] il existe \(S\in\sym{n}[\R]\) telle que \(A=S^2\).
\end{enumerate}

Modifiez les propositions précédentes pour caractériser les matrices \(A\) définies-positives.
\end{exoss}

\begin{exosss}[Racine carrée]
\begin{enumerate}
    \item Montrez que \(\fonction{r}{\symdefpos{n}[\R]}{\symdefpos{n}[\R]}{M}{M^2}\) est une surjection. \\
    \item Soit \(A\in\symdefpos{n}[\R]\) et \(M\) un antécédent de \(A\) par \(r\). Montrez qu'il existe un polynôme \(P\in\poly[\R]\) tel que \(M=P\paren{A}\). \\
    \item Soient \(A\in\symdefpos{n}[\R]\) et \(M,N\) deux antécédents de \(A\) par \(r\). Montrez que \(\paren{M-N}\paren{M+N}=0\) et que \(M+N\) est inversible. Déduisez-en que l'application \(r\) est injective.
\end{enumerate}
\end{exosss}

\begin{exosss}[Décomposition polaire]
Montrez que pour tout \(A\in\GL{n}[\R]\), il existe un unique couple \(\paren{Q,S}\in\Orth{n}[\R]\times\symdefpos{n}[\R]\) tel que \(A=QS\).
\end{exosss}

\begin{exoss}
Soient \(f\) et \(g\) deux endomorphismes auto-adjoints d'un espace euclidien.

Montrez que \(f\rond g\) est auto-adjoint ssi \(f\) et \(g\) commutent.
\end{exoss}

\begin{exoss}
\begin{enumerate}
    \item Soit \(S\in\sym{n}[\R]\). Montrez qu'il existe des réels \(\lambda_1,\dots,\lambda_n\) et des matrices colonnes \(U_1,\dots,U_n\) de \(\M{n\,1}[\R]\) tels que \[\quantifs{\forall\paren{i,j}\in\interventierii{1}{n}^2}\trans{U_i}U_j=\delta_{i\,j}\qquad\text{et}\qquad S=\sum_{i=1}^n\lambda_iU_i\trans{U_i}.\] \\
    \item Soit \(u\) un endomorphisme auto-adjoint d'un espace euclidien \(E\). Montrez que \(u\) est combinaison linéaire de projections orthogonales sur des droites et que ces projections commutent entre elles.
\end{enumerate}
\end{exoss}

\begin{exoss}
Une matrice \(M\in\M{n}[\R]\) vérifie \(M^2+4I_n=0\) et \(\trans{M}M=M\trans{M}=S\).

\begin{enumerate}
    \item Trouvez un polynôme annulateur de \(S\) de degré \(2\). \\
    \item Déduisez-en que \(\dfrac{1}{2}M\) est orthogonale. \\
    \item Dans le cas \(n=2\), déterminez toutes les matrices de \(\M{2}[\R]\) qui vérifient les conditions de l'exercice.
\end{enumerate}
\end{exoss}

\begin{exoss}
Soient \(E\) un espace euclidien orienté de dimension \(3\) et \(u\) un vecteur non-nul de \(E\).

On pose \(f\) l'endomorphisme de \(E\) défini par \(f\paren{x}=u\vecto\paren{u\vecto x}\).

Montrez que \(f\) est diagonalisable et diagonalisez-le.
\end{exoss}

\begin{exoss}
Soient \(A\in\M{n}[\R]\), \(S=\dfrac{1}{2}\paren{A+\trans{A}}\) et \(a,b\) les plus petite et plus grande valeurs propres de \(S\).

Montrez que toute valeur propre réelle de \(A\) est comprise entre \(a\) et \(b\).
\end{exoss}

\begin{exoss}
Soit \(E\) un espace euclidien. On dit qu'un endomorphisme \(f\) de \(E\) est anti-adjoint quand \[\quantifs{\forall u,v\in E}\ps{f\paren{u}}{v}=-\ps{u}{f\paren{v}}.\]

Soit \(f\) un endomorphisme anti-adjoint de \(E\).

\begin{enumerate}
    \item Montrez que \(\quantifs{\forall u\in E}f\paren{u}\perp u\). \\
    \item Montrez que la seule valeur propre possible de \(f\) est \(0\). L'endomorphisme \(f\) peut-il être diagonalisable ? \\
    \item Montrez que \(f\rond f\) est un endomorphisme auto-adjoint de \(E\) et que ses valeurs propres sont négatives ou nulles. \\
    \item Soit \(x\) un vecteur propre de \(f\rond f\) pour la valeur propre \(-\lambda\not=0\). \\ \begin{enumerate}
        \item Montrez que \(P_x=\Vect{x,f\paren{x}}\) est un plan vectoriel stable par \(f\). \\
        \item Montrez qu'il existe une base orthonormée \(\paren{a,b}\) de \(P_x\) où la matrice de l'endomorphisme induit \(f_{P_x}\) est de la forme \[R_\mu=\begin{pmatrix}
            0 & -\mu \\
            \mu & 0
        \end{pmatrix},\] où \(\mu\) est un réel strictement positif que l'on précisera. \\
    \end{enumerate}
    \item Montrez qu'il existe une base orthonormée de \(E\) où la matrice de \(f\) est diagonale par blocs, avec des blocs de la forme \(R_\mu\) d'abord puis des blocs \(\paren{0}\) de taille \(1\times1\).
\end{enumerate}
\end{exoss}

\begin{exo}[Oral CCINP, 1]~\\
\begin{enumerate}
    \item Calculez, pour \(n\in\N\), \(I_n=\dfrac{1}{\sqrt{\pi}}\int_{\minf}^{\pinf}\e{-t^2}t^n\odif{t}\) (distinguer les cas \(n\) pair et \(n\) impair ; on donne de plus \(I_0=1\)). \\
    \item Montrez que l'application \(\paren{P,Q}\mapsto\int_{\minf}^{\pinf}P\paren{t}Q\paren{t}\e{-t^2}\odif{t}\) est un produit scalaire sur \(\poly[\R]\). \\
    \item Calculez la distance de \(X^3\) à \(\polydeg[\R]{2}\).
\end{enumerate}
\end{exo}

\begin{exo}[Oral CCINP, 2]
Soit \(E=\ensclasse{0}{\intervii{0}{1}}{\R}\). On pose pour \(\paren{f,g}\in E^2\) : \(\ps{f}{g}=\int_0^1f\paren{t}g\paren{t}t^2\odif{t}\).

\begin{enumerate}
    \item Montrez que \(\ps{}{}\) est un produit scalaire. \\
    \item Calculez \(\int_0^1t^n\ln\paren{t}\odif{t}\) pour tout \(n\in\N\). \\
    \item Soit \(F=\accol{x\mapsto ax+b\tq\paren{a,b}\in\R^2}\) et \(u\in E\) telle que \(u\paren{x}=x\ln x\) pour tout \(x\in\intervei{0}{1}\). Déterminez le projeté orthogonal de \(u\) sur \(F\). \\
    \item Déterminez \(\inf_{\paren{a,b}\in\R^2}\int_0^1\paren{at+b-t\ln t}^2t^2\odif{t}\).
\end{enumerate}
\end{exo}

\begin{exo}[Oral CCINP, 3]
Soit \(E=\ensclasse{2}{\intervii{0}{1}}{\R}\). On pose pour \(\paren{f,g}\in E^2\) : \(\ps{f}{g}=\int_0^1\paren{fg+f\prim g\prim}\).

On considère les sous-ensembles \[V=\accol{f\in E\tq f\seconde=f}\qquad G=\accol{f\in E\tq f\paren{0}=f\paren{1}=0}\qquad H=\accol{f\in E\tq f\paren{0}=\ch1\text{ et }f\paren{1}=1}.\]

\begin{enumerate}
    \item Montrez que la famille \(\paren{\ch,\sh}\) est une base de \(V\). \\
    \item Soient \(f\in V\) et \(g\in E\). Montrez que \(\ps{f}{g}=f\prim\paren{1}g\paren{1}-f\prim\paren{0}g\paren{0}\). Calculez \(\ps{\ch}{\sh}\), \(\norme{\ch}^2\) et \(\norme{\sh}^2\). \\
    \item Soient \(f\in V\) et \(g\in G\). Montrez que \(\ps{f}{g}=0\). \\
    \item Soit \(f\in H\). Calculez \(\ps{f}{\ch}\) et \(\ps{f}{\sh}\). Déduisez-en le projeté orthogonal de \(f\) sur \(V\). \\
    \item Calculez \(\inf_{f\in H}\int_0^1\paren{f^2+{f\prim}^2}\).
\end{enumerate}
\end{exo}

\begin{exo}[Oral CCINP, 4]
Soient \(E\) un espace euclidien, \(\paren{e_1,\dots,e_n}\) une base orthonormée de \(E\) et \(\paren{u_1,\dots,u_n}\) une famille de vecteurs de \(E\).

\begin{enumerate}
    \item Soit \(\paren{\lambda_1,\dots,\lambda_n}\in\R^n\). Montrez que \(\norme{\sum_{i=1}^n\lambda_iu_i}^2\leq\sum_{i=1}^n\lambda_i^2\times\sum_{i=1}^n\norme{u_i}^2\). \\
    \item Déduisez-en que si \(\sum_{i=1}^n\norme{u_i}^2<1\), alors la famille \(\paren{e_i+u_i}_{1\leq i\leq n}\) est une base de \(E\).
\end{enumerate}
\end{exo}

\begin{exo}[Oral CCINP, 5]
Soient \(E\) un espace euclidien et \(\fami{B}=\paren{e_1,\dots,e_n}\) une base orthonormée de \(E\). On note \(D\) la droite engendrée par le vecteur \(u=\sum_{k=1}^nke_k\).

\begin{enumerate}
    \item Donnez la matrice dans la base \(\fami{B}\) du projecteur orthogonal sur \(D\), noté \(p\). \\
    \item Donnez le polynôme caractéristique et le spectre de \(p\). \\
    \item Calculez la distance de \(v=\sum_{k=1}^ne_k\) à \(D\).
\end{enumerate}
\end{exo}

\begin{exo}[Oral CCINP, 6]
Soient \(A\in\sym{n}[\R]\) une matrice définie-positive et \(B\in\M{n\,m}[\R]\) une matrice de rang \(m\).

\begin{enumerate}
    \item Montrez que \(n\geq m\). \\
    \item Montrez que \(C=\begin{pmatrix}
        A & B \\
        \trans{B} & 0
    \end{pmatrix}\) est inversible.
\end{enumerate}
\end{exo}

\begin{exo}[Oral IMT, 7]
Soient \(E\) un espace euclidien et \(f\) un endomorphisme auto-adjoint défini-positif de \(E\).

\begin{enumerate}
    \item Montrez que \(\paren{x,y}\mapsto\ps{f\paren{x}}{y}\) est un produit scalaire sur \(E\). \\
    \item Montrez qu'il existe un endomorphisme auto-adjoint défini-positif \(g\) tel que \(g^2=f\).
\end{enumerate}
\end{exo}

\begin{exo}[Oral IMT, 8]
Soient \(A\in\M{n}[\R]\) telle que \(A\trans{A}=\trans{A}A\) et \(p\geq2\) tel que \(A^p=0\).

Montrez que \(A=0\).
\end{exo}

\begin{exo}[Oral CCINP, 9]
Soient \(A,B\in\Orth{n}[\R]\) telles que \(M=\dfrac{1}{3}\paren{2A+B}\in\Orth{n}[\R]\).

\begin{enumerate}
    \item Calculez \(A\trans{B}+B\trans{A}\) en calculant \(M\trans{M}\). \\
    \item Montrez que \(A=B\).
\end{enumerate}
\end{exo}

\begin{exo}[Oral CCINP, 10]~\\
Soient \(C\in\M{n\,1}[\R]\) et \(M=\begin{pmatrix}
1 & -\trans{C} \\
C & I_n
\end{pmatrix}\in\M{n+1}[\R]\).

\begin{enumerate}
    \item Calculez \(\trans{M}M\), déduisez-en que \(M\) est inversible. \\
    \item Montrez que \(M\inv\trans{M}\) est une matrice orthogonale.
\end{enumerate}
\end{exo}

\begin{exo}[Oral CCINP, 11]
Soit \(M\in\M{n}[\R]\) telle que \(M\not=I_n\), \(M^3=I_n\) et \(M\trans{M}=\trans{M}M\).

\begin{enumerate}
    \item Montrez que \(M\in\Orth{n}[\R]\). \\
    \item Dans le cas où \(n=3\), déterminez les matrices \(M\) vérifiant les conditions de l'énoncé.
\end{enumerate}
\end{exo}

\begin{exo}[Oral CCMP, 12]~\\
\begin{enumerate}
    \item Montrez que \(\dim\sym{n}[\R]=\dfrac{n\paren{n+1}}{2}\). \\
    \item Est-ce que l'ensemble des matrices nilpotentes de \(\M{n}[\R]\) est un espace vectoriel ? \\
    \item Soit \(\fami{N}\) un sous-espace de \(\M{n}[\R]\) constitué de matrices nilpotentes. Montrez que \(\dim\fami{N}\leq\dfrac{n\paren{n-1}}{2}\). \\
    \item Peut-on avoir \(\dim\fami{N}=\dfrac{n\paren{n-1}}{2}\) ?
\end{enumerate}
\end{exo}

\begin{exo}[Oral Centrale, 13]
Soit \(A\in\M{n}[\R]\). On dit que \(A\) est une matrice normale quand \(A\trans{A}=\trans{A}A\).

\begin{enumerate}
    \item Déterminez les matrices normales de \(\M{2}[\R]\). \\
    \item Montrez que toute matrice normale stabilise un espace de dimension \(1\) ou \(2\). \\
    \item Montrez que si \(A\) est normale, alors il existe \(P\in\Orth{n}[\R]\) telle que \(\trans{P}AP\) soit diagonale par blocs, chaque bloc diagonal étant soit un bloc \(\paren{a}\), soit un bloc \(\begin{pmatrix}
        a & c \\
        -c & b
    \end{pmatrix}\).
\end{enumerate}
\end{exo}

\begin{exo}[Oral Centrale, 14]
Soit \(A\in\M{n\,p}[\R]\). On munit \(\R^n\) et \(\R^p\) de leurs produits scalaires canoniques.

\begin{enumerate}
    \item Montrez que \(\ker A=\ker\trans{A}A\). Déduisez-en que \(\rg A=\rg\trans{A}A\) (noté \(r\) dans la suite). \\
    \item Montrez qu'il existe une famille orthonormée \(\paren{y_1,\dots,y_r}\) de \(\R^r\) telle que la matrice \(Y\) de colonnes \(y_1,\dots,y_r\) vérifie \(\trans{Y}\trans{A}AY=D\), où \(D\) est diagonale à coefficients diagonaux strictement positifs. \\
    \item Montrez qu'il existe \(U,V\) orthogonales et \(\Delta\) diagonale telles que \(A=U\Delta V\).
\end{enumerate}
\end{exo}

\begin{exo}[Oral CCMP, 15]
Soit \(A=\paren{a_{i\,j}}\in\symdefpos{n}[\R]\).

\begin{enumerate}
    \item Montrez que \(\quantifs{\tpt\paren{i,j}\in\interventierii{1}{n}^2}a_{i\,j}>0\text{ et }a_{i\,j}^2\leq a_{i\,i}a_{j\,j}\). \\
    \item Montrez que \(\max_{\paren{i,j}\in\interventierii{1}{n}^2}\abs{a_{i\,j}}=\max_{k\in\interventierii{1}{n}}a_{k\,k}\).
\end{enumerate}
\end{exo}

\begin{exo}[Oral CCMP, 16]
\begin{enumerate}
    \item Montrez que si \(A,B\in\sympos{n}[\R]\), alors \(\tr\paren{AB}\geq0\). \\
    \item Déterminez les matrices \(M\in\M{n}[\R]\) telles que \(\quantifs{\forall S\in\sympos{n}[\R]}\tr\paren{MS}\geq0\).
\end{enumerate}
\end{exo}

\begin{exo}[Oral CCMP, 17]
Soient \(A\in\M{n}[\R]\) et \(k\geq2\) tels que \(A^k=\trans{A}\).

\begin{enumerate}
    \item Montrez que \(\ker A\) et \(\Im A\) sont supplémentaires orthogonaux dans \(\R^n\). \\
    \item Montrez que \(B=A^{k+1}\) est une matrice de projecteur orthogonal. \\
    \item Montrez que \(A\) induit une isométrie sur \(\Im A\). \\
    \item Déduisez-en \(A\).
\end{enumerate}
\end{exo}

\begin{exo}[Oral CCMP, 18]
Soient \(S\in\symdefpos{n}[\R]\) et \(A\in\antisym{n}[\R]\).

\begin{enumerate}
    \item Montrez que le spectre complexe de \(A\) est inclus dans \(\iR\). \\
    \item Montrez qu'il existe \(S\prim\in\symdefpos{n}[\R]\) telle que \({S\prim}^2=S\). \\
    \item Montrez que \(\det S\leq\det\paren{S+A}\).
\end{enumerate}
\end{exo}

\section*{Problème}
\addcontentsline{toc}{section}{Problème}

\subsection*{Notations}
\addcontentsline{toc}{subsection}{Notations}

Pour ce problème, on désigne par \(n\) un entier naturel non-nul.

On note \(\sym{n}[\R]=\accol{A\in\M{n}[\R]\tq\trans{A}=A}\) le sous-espace vectoriel de \(\M{n}[\R]\) formé des matrices symétriques.

Un vecteur de \(\R^n\) est noté \(x=\paren{x_j}_{1\leq j\leq n}=\tcoords{x_1}{\vdots}{x_n}\) en identifiant les matrices-colonnes de \(\M{n\,1}[\R]\) avec les éléments de \(\R^n\).

Une matrice \(A\) de \(\M{n}[\R]\) est notée \(A=\paren{a_{i\,j}}_{1\leq i,j\leq n}\) où \(a_{i\,j}\) est le coefficient de \(A\) situé en ligne \(i\) et colonne \(j\).

\(\Sp{A}\) est le spectre de \(A\), ensemble de ses valeurs propres.

L'espace vectoriel \(\R^n\) est muni du produit scalaire canonique défini par \[\quantifs{\forall\paren{x,y}\in\R^n\times\R^n}\ps{x}{y}=\trans{x}y=\sum_{k=1}^nx_ky_k\] et \(x\mapsto\norme{x}=\sqrt{\ps{x}{x}}\) est la norme euclidienne associée.

La sphère unité de \(R^n\) est notée \(\Omega_n=\accol{x\in\R^n\tq\norme{x}=1}\).

À toute matrice \(A\in\M{n}[\R]\), on associe la fonction \(q_A:\R^n\to\R\) définie par \[\quantifs{\forall x\in\R^n}q_A\paren{x}=\ps{Ax}{x}.\]

\subsection*{Une norme sur \(\sym{n}[\R]\)}
\addcontentsline{toc}{subsection}{Une norme sur \(\sym{n}[\R]\)}

\setcounter{q}{0}

\begin{q}
Soit \(A\in\M{n}[\R]\).

\begin{enumerate}
    \item Montrez que la fonction \(q_A\) est bornée et atteint ses bornes sur la sphère unité \(\Omega_n\). On note alors \[m_A=\min_{x\in\Omega_n}q_A\paren{x}\qquad\text{et}\qquad M_A=\max_{x\in\Omega_n}q_A\paren{x}.\] \\
    \item Démontrez que toute valeur propre réelle de \(A\) se trouve dans l'intervalle \(\intervii{m_A}{M_A}\). \\
    \item Explicitez \(\Sp{A}\), \(m_A\) et \(M_A\) pour la matrice \[A=\begin{pmatrix}
        2 & -1 \\
        0 & 2
    \end{pmatrix}.\]
\end{enumerate}
\end{q}

\begin{q}
Soit \(A\in\M{n}[\R]\). On suppose que \(q_A\paren{x}=0\) pour tout \(x\in\Omega_n\).

\begin{enumerate}
    \item Montrez que \(q_A\paren{y}=0\) pour tout \(y\in\R^n\). \\
    \item Si \(\paren{y,z}\in\R^n\times\R^n\), exprimez \(q_A\paren{y+z}\) (qui est nul d'après la question 2a) en fonction de \(\ps{Ay}{z}\) et \(\ps{Az}{y}\). \\
    \item Montrez que la matrice \(A\) est anti-symétrique (c'est à dire que \(\trans{A}=-A\)).
\end{enumerate}
\end{q}

\begin{q}
Soit \(A\in\sym{n}[\R]\). Montrez que \[\croch{\quantifs{\forall x\in\Omega_n}q_A\paren{x}=0}\ssi A=0_n.\]
\end{q}

\begin{q}
Montrez que l'application \(N:\sym{n}[\R]\to\R\) définie par \[\quantifs{\forall A\in\sym{n}[\R]}N\paren{A}=\sup_{x\in\Omega_n}\abs{q_A\paren{x}}\] est une norme.
\end{q}

\begin{q}
On considère \(A\in\sym{n}[\R]\) et \(u\) l'endomorphisme de \(\R^n\) canoniquement associé à \(A\).

\begin{enumerate}
    \item Justifiez qu'il existe \(n\) nombres réels \[\lambda_1\leq\dots\leq\lambda_n\] et une base orthonormée \(\paren{e_k}_{1\leq k\leq n}\) de \(\R^n\) tels que \[\quantifs{\forall k\in\interventierii{1}{n}}u\paren{e_k}=Ae_k=\lambda_ke_k.\] \\
    \item Précisez \(q_A\paren{e_k}\) pour tout \(k\in\interventierii{1}{n}\). \\
    \item Soit \(x=\sum_{k=1}^nx_k\prim e_k\in\Omega_n\). Exprimez \(q_A\paren{x}\) en fonction des valeurs propres \(\lambda_k\) de \(A\) et des composantes \(x_k\prim\) de \(x\). \\
    \item Redémontrez le résultat de la question 1 : la fonction \(q_A\) possède un minimum \(m_A\) et un maximum \(M_A\) sur la sphère unité \(\Omega_n\). Explicitez \(m_A\) et \(M_A\) en fonction des valeurs propres de \(A\). \\
    \item Montrez que \[N\paren{A}=\sup_{x\in\Omega_n}\abs{q_A\paren{x}}=\max_{\lambda\in\Sp{A}}\abs{\lambda}.\] Établissez une inégalité entre \(\abs{\det A}\) et \(\paren{N\paren{A}}^n\). \\
    \item Exemple : si \(A=\begin{pmatrix}
        1 & \nicefrac{1}{2} \\
        \nicefrac{1}{2} & \nicefrac{1}{3}
    \end{pmatrix}\), calculez \(\det A\) et \(N\paren{A}\).
\end{enumerate}
\end{q}

Dans toute la suite du problème, pour tout entier \(n\geq2\), on désigne par \(H_n\) la matrice de Hilbert d'ordre \(n\) définie par \[H_n=\paren{\dfrac{1}{i+j-1}}_{1\leq i,j\leq n}=\begin{pmatrix}
1 & \nicefrac{1}{2} & \dots & \nicefrac{1}{n} \\
\nicefrac{1}{2} & \nicefrac{1}{3} & \dots & \nicefrac{1}{n+1} \\
\vdots & \vdots &  & \vdots \\
\nicefrac{1}{n} & \nicefrac{1}{n+1} & \dots & \nicefrac{1}{2n-1}
\end{pmatrix}\] ou encore \(H_n=\paren{a_{i\,j}}_{1\leq i,j\leq n}\) avec \(a_{i\,j}=\dfrac{1}{i+j-1}\).

Pour simplifier, on notera \(q_n\) la fonction \(q_{H_n}:\R^n\to\R\) : \(\quantifs{\forall x\in\R^n}q_n\paren{x}=q_{H_n}\paren{x}=\ps{H_nx}{x}\).

\subsection*{Sur les valeurs propres de \(H_n\)}
\addcontentsline{toc}{subsection}{Sur les valeurs propres de \(H_n\)}

\begin{q}
Soit \(x=\paren{x_j}_{1\leq j\leq n}\in\R^n\).

\begin{enumerate}
    \item Montrez que \[q_n\paren{x}=\ps{H_nx}{x}=\sum_{i=1}^n\paren{\sum_{j=1}^n\dfrac{x_j}{i+j-1}}x_i=\sum_{1\leq i,j\leq n}\dfrac{x_ix_j}{i+j-1}.\] \\
    \item Développez \(\paren{\sum_{i=1}^nx_it^{i-1}}\paren{\sum_{j=1}^nx_jt^{j-1}}\) où \(t\) est une variable réelle. \\
    \item Montrez que \(q_n\paren{x}=\int_0^1\paren{\sum_{j=1}^nx_jt^{j-1}}^2\odif{t}\). \\
    \item Montrez que \[\quantifs{\forall x\in\R^n}q_n\paren{x}\geq0\] et \[\quantifs{\forall x\in\R^n}q_n\paren{x}=0\ssi x=0.\] Que peut-on en déduire concernant les valeurs propres de \(H_n\) ?
\end{enumerate}
\end{q}

\begin{q}
\begin{enumerate}
    \item Soit \(P\paren{t}=\sum_{k=0}^ma_kt^k\) un polynôme à coefficients complexes. \\ Montrez que \(\int_{-1}^1P\paren{t}\odif{t}=-\i\int_0^\pi P\paren{\e{\i\theta}}\e{\i\theta}\odif{\theta}\). \\ \textit{On pourra expliciter \(\int_{-1}^1t^k\odif{t}\) et \(-\i\int_0^\pi\e{\i k\theta}\e{\i\theta}\odif{\theta}\).} \\
    \item En gardant les notations introduites dans la question 6 et en notant \(Q\paren{t}=\sum_{k=1}^nx_kt^{k-1}\), montrez que pour tout \(x\in\R^n\), on a \[0\leq q_n\paren{x}=\int_0^1Q^2\paren{t}\odif{t}\leq\int_0^\pi\abs{\sum_{k=1}^nx_k\e{\i\paren{k-1}\theta}}^2\odif{\theta}.\]
\end{enumerate}
\end{q}

\begin{q}
\begin{enumerate}
    \item Soit \(f\) une fonction continue sur \(\intervii{0}{\pi}\) à valeurs complexes telle que \(\int_0^\pi\abs{f\paren{\theta}}\odif{\theta}=\abs{\int_0^\pi f\paren{\theta}\odif{\theta}}\). \\ Montrez que \(f\) est à valeurs dans \(\Rp\). \\
    \item Soit \(P\paren{t}=\sum_{k=0}^ma_kt^k\) un polynôme à coefficients complexes. \\ Calculez, pour \(k\in\interventierii{0}{n}\), \(\int_0^{2\pi}P\paren{\e{\i\theta}}\e{-\i k\theta}\odif{\theta}\). \\ Déduisez-en que s'il existe \(\alpha\in\R\) tel que \(\quantifs{\tpt\theta}P\paren{\e{\i\theta}}\e{-\i\alpha}\in\Rp\), alors \(P\) est constant. \\
    \item Montrez que les inégalités de la question 7b sont strictes pour \(x\not=0\).
\end{enumerate}
\end{q}

\begin{q}
Montrez que \[\quantifs{\forall x\in\R^n}0\leq q_n\paren{x}\leq\pi\norme{x}^2,\] les inégalités étant strictes pour \(x\not=0\).
\end{q}

\begin{q}
Pour tout entier \(n\geq2\), on note \[\mu_n=\min\Sp{H_n}\qquad\text{et}\qquad\rho_n=\max\Sp{H_n}.\]

\begin{enumerate}
    \item Explicitez \(\mu_2\) et \(\rho_2\). Montrez que pour tout \(n\geq2\), on a \[0<\mu_n<\rho_n<\pi.\] \\
    \item Montrez que \(q_n\paren{\Omega_n}=\intervii{\mu_n}{\rho_n}\). \\
    \item Calculez \(\ps{H_n\epsilon_n}{\epsilon_n}\) où \(\epsilon_n\) désigne le dernier vecteur de la base canonique de \(\M{n\,1}[\R]\) : \(\epsilon_n=\begin{pmatrix}
        0 \\
        \vdots \\
        0 \\
        1
    \end{pmatrix}\). \\ Déduisez-en la limite de \(\mu_n\) lorsque \(n\to\pinf\).
\end{enumerate}
\end{q}

\subsection*{Sur le déterminant de \(H_n\)}
\addcontentsline{toc}{subsection}{Sur le déterminant de \(H_n\)}

\(H_n\) désigne toujours la matrice de Hilbert d'ordre \(n\), pour \(n\geq2\).

\begin{q}[Une fraction rationnelle]
On considère la fraction rationnelle \[R_n\paren{x}=\dfrac{\ds\prod_{k=1}^n\paren{x-k}}{\ds\prod_{k=0}^n\paren{x+k}}.\]

On admettra qu'il existe des réels \(\lambda_{0\,n},\dots,\lambda_{n\,n}\) tels que \[\quantifs{\forall x\in\R\excluant\accol{0,\dots,-n}}R_n\paren{x}=\sum_{k=0}^n\dfrac{\lambda_{k\,n}}{x+k},\] cette décomposition (en éléments simples) de \(R_n\) étant unique.

Exprimez le coefficient \(\lambda_{n\,n}\) de \(\dfrac{1}{x+n}\) à l'aide de \(\paren{2n}!\) et de \(n!\).
\end{q}

\begin{q}[Matrice \(A_n\)]
Pour \(n\geq2\), on considère la matrice \(A_n\) définie par \(A_n=\paren{a_{i\,j}}_{1\leq i,j\leq n}\) avec \[a_{i\,j}=\begin{dcases}
\dfrac{1}{i+j-1} &\text{pour }1\leq i\leq n-1\text{ et }1\leq j\leq n \\
R_{n-1}\paren{j} &\text{pour }i=n\text{ et }1\leq j\leq n
\end{dcases}\] où les \(R_p\) ont été définis plus haut.

\begin{enumerate}
    \item Montrez que pour \(1\leq i\leq n\), on a \[R_{n-1}\paren{i}=\sum_{j=1}^n\lambda_{j-1\,n-1}h_{i\,j}\] puis déduisez-en que \(\det A_n=\binom{n-1}{2\paren{n-1}}\det H_n\). \\
    \item Montrez que \[\det A_n=\dfrac{\det H_{n-1}}{\paren{2n-1}\binom{n-1}{2\paren{n-1}}}.\] Déduisez-en l'expression de \(\det H_n\) en fonction de \(\det H_{n-1}\). \\
    \item Montrez que pour tout \(n\geq2\), \[\det H_n\not=0,\] puis que \[\dfrac{1}{\det H_n}\in\Ns.\]
\end{enumerate}
\end{q}

\begin{q}[Calcul de \(\det H_n\)]
En notant, pour tout \(n\in\Ns\) : \(\Phi_n=\prod_{k=1}^nk!\), montrez que \[\quantifs{\forall n\geq2}\det H_n=\dfrac{\Phi_{n-1}^4}{\Phi_{2n-1}}.\]
\end{q}
