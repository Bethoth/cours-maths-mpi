\chapter{Équations différentielles linéaires}

\minitoc

\legendeexercices

\begin{exos}
Résolvez les équations différentielles suivantes sur les intervalles indiqués :

\begin{enumerate}
    \item \(xy\prim+y=\sin x\) sur \(\Rps\) \\
    \item \(ty\prim-2y=t\) sur \(\Rms\) \\
    \item \(xy\prim+\paren{x-1}y=x^2\) sur \(\Rps\) \\
    \item \(\paren{1+t^2}y\prim+ty=1+2t^2\) sur \(\R\) \\
    \item \(\paren{x^2-1}y\prim+xy=x^3\) sur \(\intervee{-1}{1}\) \\
    \item \(\paren{x-1}y\prim+xy=\sin x\) sur \(\intervee{\minf}{1}\).
\end{enumerate}
\end{exos}

\begin{exos}
Même exercice :

\begin{enumerate}
    \item \(\paren{x+5}\paren{x-3}y\prim+\paren{x+13}y=1+\dfrac{5}{x}\) sur \(\intervee{3}{\pinf}\) \\
    \item \(\paren{x+1}^2y\prim-\paren{3x+4}y=3\paren{x+1}^3\) sur \(\intervee{\minf}{-1}\) \\
    \item \(\paren{x^2-2x+1}y\prim+xy=\exp\paren{\dfrac{1}{x-1}}\) sur \(\intervie{1}{\pinf}\) \\
    \item \(x\paren{x-1}y\prim-\paren{3x-1}y=-x^2\paren{x^2+1}\) sur \(\intervee{0}{1}\).
\end{enumerate}
\end{exos}

\begin{exoss}
Déterminez, s'il en existe, les solutions sur \(\R\) des équations différentielles suivantes (équations avec raccordements) :

\begin{enumerate}
    \item \(xy\prim-2y=-x\) \\
    \item \(xy\prim+y=\sin x\) \\
    \item \(xy\prim-y=\dfrac{x}{1+x^2}\) \\
    \item \(\paren{1-x^2}y\prim+2y=\paren{x-1}^2\) \\
    \item \(\paren{x-1}\paren{x-2}y\prim-y=x\).
\end{enumerate}
\end{exoss}

\begin{exos}
Résolvez les systèmes différentiels suivants :

\begin{enumerate}
    \item \(\begin{dcases}
        x\prim=4x-2y \\
        y\prim=x+y
    \end{dcases}\) \\
    \item \(\begin{dcases}
        x\prim=2x+y \\
        y\prim=-x \\
        z\prim=x+y+z
    \end{dcases}\) et \(\begin{dcases}
        x\paren{0}=0 \\
        y\paren{0}=1 \\
        z\paren{0}=-1
    \end{dcases}\) \\
    \item \(\begin{dcases}
        x\prim=x-y+\dfrac{1}{\cos t} \\
        y\prim=2x-y
    \end{dcases}\) sur \(\intervee{-\dfrac{\pi}{2}}{\dfrac{\pi}{2}}\) \\
    \item \(\begin{dcases}
        x\prim=tx-y \\
        y\prim=\paren{2-t}x+y
    \end{dcases}\) \textit{(indication : solutions polynomiales).}
\end{enumerate}
\end{exos}

\begin{exoss}
Résolvez les systèmes différentiels suivants :

\begin{enumerate}
    \item \(\begin{dcases}
        x\prim=4x+y+2y \\
        y\prim=4x+2y+2z \\
        z\prim=-4x-2y-2z
    \end{dcases}\) \\
    \item \(\begin{dcases}
        x\prim=-x+y+z \\
        y\prim=x-y-z \\
        z\prim=x-y-3z
    \end{dcases}\) \\
    \item \(\begin{dcases}
        x\prim=-4x+8y+22z \\
        y\prim=-2x+3y+4z \\
        z\prim=-x+2y+7z
    \end{dcases}\) \\
    \item \(\begin{dcases}
        x\prim=y+2z \\
        y\prim=2x+y-z \\
        z\prim=-2x-y-z
    \end{dcases}\) \\
    \item \(\begin{dcases}
        x\prim=2y+2z \\
        y\prim=-x+2y+2z \\
        z\prim=-x+y+3z
    \end{dcases}\) \\
    \item \(\begin{dcases}
        x\prim=2x+z \\
        y\prim=x-y-z \\
        z\prim=-x+2y+2z
    \end{dcases}\)
\end{enumerate}
\end{exoss}

\begin{exoss}
Résolvez le système différentiel homogène à coefficients variables : \[\begin{dcases}
x\prim\paren{t}=\paren{t-2}x\paren{t}-\paren{t-1}y\paren{t} \\
y\prim\paren{t}=2\paren{t-1}x\paren{t}-\paren{2t-1}y\paren{t}
\end{dcases}\]

\textit{Indication : on pourra présenter le système différentiel sous forme d'équation différentielle vectorielle et réduire la matrice \(A\paren{t}\) de ses coefficients.}
\end{exoss}

\begin{exos}
Résolvez l'équation différentielle \[\paren{1+x^2}y\seconde-\paren{3x^2-4x+3}y\prim+\paren{2x^2-6x+4}y=0\] en faisant le changement de fonction inconnue \(z:x\mapsto\paren{1+x^2}y\paren{x}\).
\end{exos}

\begin{exoss}
Résolvez l'équation différentielle \[xy\seconde-y\prim-4x^3y=x^3\e{x^2}\] sur l'intervalle \(\intervee{0}{\pinf}\) en effectuant le changement de fonction inconnue \(z\paren{x}=y\paren{\sqrt{x}}\).
\end{exoss}

\begin{exoss}
\begin{enumerate}
    \item Soit \(\paren{a,b,c}\in\R^3\) tel que \(a\not=0\). Montrez que \(y\) est solution de l'équation \(ax^2y\seconde+bxy\prim+cy=0\) sur \(\Rps\) ssi \(z:t\mapsto y\paren{\e{t}}\) est solution d'une équation linéaire du second ordre à coefficients constants. \\
    \item Exemples : résolvez les équations suivantes sur \(\Rps\) : \[x^2y\seconde-xy\prim+y=0\qquad x^2y\seconde+y=0.\]
    \item Déterminez les fonctions \(f\) dérivables sur \(\Rps\) telles que \(\quantifs{\forall x\in\Rps}f\prim\paren{x}=f\paren{\dfrac{1}{x}}\).
\end{enumerate}
\end{exoss}

\begin{exoss}
Résolvez les équations différentielles suivantes en vous aidant des indications fournies :

\begin{enumerate}
    \item \(y\seconde+ty\prim+y=0\) \textit{(cherchez des solutions développables en série entière)} \\
    \item \(y\seconde+ty\prim+3y=0\) \textit{(idem)} \\
    \item \(ty\seconde+2y\prim-ty=0\) sur \(\Rps\) \textit{(idem)} \\
    \item \(4ty\seconde+2y\prim-y=0\) sur \(\Rps\) \textit{(idem)} \\
    \item \(ty\seconde-y\prim-4t^3y=0\) sur \(\Rps\) \textit{(idem)} \\
    \item \(t^2y\seconde+ty\prim+y=0\) sur \(\Rps\) \textit{(changement de variable \(t=\e{s}\))} \\
    \item \(ty\seconde+2\paren{t+1}y\prim+2y=1-\e{-2t}\) sur \(\Rps\) \textit{(chercher une solution \(t\mapsto t^\alpha\) et utiliser la méthode de variation de la constante).}
\end{enumerate}
\end{exoss}

\begin{exoss}[Raccordement de solutions]
Déterminez les solutions sur \(\R\) des équations 3, 4, 5 et 6 de l'exercice précédent.
\end{exoss}

\begin{exoss}
On considère la fonction \(A:t\mapsto\Arcsin^2\paren{t}\).

\begin{enumerate}
    \item Sans chercher à calculer ses coefficients, justifiez que la fonction \(A\) est développable en série entière sur un intervalle \(I\) que l'on précisera. \\
    \item Montrez que la fonction \(A\) est solution de l'équation différentielle \[\quantifs{\forall t\in I}\paren{1-t^2}y\seconde\paren{t}-ty\prim\paren{t}=2.\]
    \item Déduisez-en le développement en série entière de \(A\).
\end{enumerate}
\end{exoss}

\begin{exoss}
On considère l'équation différentielle \[\paren{E}~ty\seconde+y\prim+ty=0.\]

\begin{enumerate}
    \item Trouvez sous forme d'un développement en série entière l'unique solution \(J_0\) de \(\paren{E}\) telle que \(J_0\paren{0}=1\) et \(J_0\prim\paren{0}=0\). \\
    \item \begin{itemize}
        \item Développez \(\cos\paren{t\sin\theta}\) en série entière de \(t\), en l'écrivant sous la forme \[\cos\paren{t\sin\theta}=\sum_{n=0}^{\pinf}u_n\paren{\theta},\] de façon que \(u_n\paren{\theta}\) possède dans son expression une puissance de \(t\). \\
        \item Justifiez l'égalité \[\int_0^\pi\sum_{n=0}^{\pinf}u_n\paren{\theta}\odif{\theta}=\sum_{n=0}^{\pinf}\int_0^\pi u_n\paren{\theta}\odif{\theta}.\]
        \item Déduisez-en que \(J_0\paren{t}=\dfrac{1}{\pi}\int_0^\pi\cos\paren{t\sin\theta}\odif{\theta}\). \\
    \end{itemize}
    \item On veut retrouver d'une autre façon cette expression de \(J_0\). \\ \begin{itemize}
        \item Vérifiez, en utilisant des dérivations sous le signe somme, que \[f:t\mapsto\dfrac{1}{\pi}\int_0^\pi\cos\paren{t\sin\theta}\odif{\theta}\] vérifie bien \(\paren{E}\). \\
        \item Montrez qu'il existe \(\alpha>0\) tel que \(J_0\) ne s'annule pas sur \(\intervii{-\alpha}{\alpha}\), puis montrez que \(K_0:t\mapsto J_0\paren{t}\int_\alpha^t\dfrac{1}{sJ_0\paren{s}^2}\odif{s}\) est aussi solution de \(\paren{E}\) sur \(\intervei{0}{\alpha}\). \\
        \item Justifiez qu'il existe \(\paren{\lambda,\mu}\in\R^2\) tel que \(f=\lambda J_0+\mu K_0\) sur \(\intervei{0}{\alpha}\). \\
        \item Déduisez-en que \(f=J_0\).
    \end{itemize}
\end{enumerate}
\end{exoss}

\begin{exoss}
Soient \(p,q\) deux applications continues de \(\R\) dans \(\R\).

On suppose \(p\) impaire et \(q\) paire, et on considère l'équation différentielle : \(\paren{H}~y\seconde+py\prim+qy=0\).

\begin{enumerate}
    \item Montrez qu'il existe une unique solution \(y_1\) de \(\paren{H}\) telle que \(y_1\paren{0}=1\), \(y_1\prim\paren{0}=0\) et \(y_1\) est paire. \\
    \item Même question pour \(y_2\) solution de \(\paren{H}\) telle que \(y_2\paren{0}=0\), \(y_2\prim\paren{0}=1\) et \(y_2\) est impaire. \\
    \item Montrez que \(\paren{y_1,y_2}\) est une base de \(\fami{S}\paren{H}\) (ensemble des solutions de \(\paren{H}\)).
\end{enumerate}
\end{exoss}

\begin{exoss}
Montrez, sans chercher à les expliciter à l'aide des fonctions usuelles, que toutes les solutions de l'équation différentielle \[\paren{E}~y\seconde-xy=0\] sont développables en série entière sur \(\R\).
\end{exoss}

\begin{exoss}
Soit \(a:\Rp\to\R\) continue et intégrable sur \(\Rp\).

Montrez que les solutions de l'équation différentielle \[y\prim-a\paren{x}y=0\] sont bornées sur \(\Rp\).
\end{exoss}

\begin{exoss}
Déterminez toutes les applications \(f:\R\to\R\) dérivables en \(0\) telles que \[\quantifs{\forall\paren{x,y}\in\R^2}f\paren{x+y}=\e{x}f\paren{y}+\e{y}f\paren{x}.\]
\end{exoss}

\begin{exoss}
Soient \(A\in\M{n}[\R]\) une matrice fixée et \(\paren{H}\) le système différentiel \(X\prim=AX\). On munit \(\M{n,1}[\R]\) de sa structure euclidienne canonique.

\begin{enumerate}
    \item Si \(X\) est une solution de \(\paren{H}\), justifiez que \(\norme{X}^2\) est une fonction dérivable et calculer sa dérivée. \\
    \item Montrez que toute solution du système différentiel \(\paren{H}\) est de norme constante ssi \(A\) est une matrice antisymétrique.
\end{enumerate}
\end{exoss}

\begin{exoss}
Soit \(P:\R\to\antisym{n}[\R]\) continue.

\begin{enumerate}
    \item Montrez qu'il existe une unique fonction \(U:\R\to\M{n}[\R]\) de classe \(\classe{1}\) telle que \(\quantifs{\tpt t\in\R}U\prim\paren{t}=P\paren{t}U\paren{t}\) et \(U\paren{0}=I_n\). \\
    \item Montrez que \(\quantifs{\tpt t\in\R}U\paren{t}\in\Orth{n}[\R]\) et \(\det U\paren{t}=1\).
\end{enumerate}
\end{exoss}

\begin{exoss}
Soient \(P:\R\to\antisym{n}[\R]\) continue et \(S_0\in\sym{n}[\R]\).

\begin{enumerate}
    \item Montrez qu'il existe une unique fonction \(S:\R\to\M{n}[\R]\) de classe \(\classe{1}\) telle que \(S\paren{0}=S_0\) et \(\quantifs{\tpt t\in\R}S\prim\paren{t}=P\paren{t}S\paren{t}-S\paren{t}P\paren{t}\). \\
    \item Montrez que \(\quantifs{\tpt t\in\R}S\paren{t}\in\sym{n}[\R]\). \\
    \item Montrez qu'il existe une application \(U:\R\to\Orth{n}[\R]\) de classe \(\classe{1}\) telle que \(\quantifs{\tpt t\in\R}S\paren{t}=U\paren{t}S_0\trans{U\paren{t}}\) \textit{(indication : utiliser l'exercice précédent).}
\end{enumerate}
\end{exoss}

\begin{exoss}
Soient \(A:\R\to\M{n}[\C]\) dérivable et \(B:\R\to\M{n}[\C]\) continue telles que \(\quantifs{\tpt t\in\R}A\prim\paren{t}=A\paren{t}B\paren{t}-B\paren{t}A\paren{t}\).

\begin{enumerate}
    \item Montrez que \(\quantifs{\tpt k\in\N}\paren{A^k}\prim=A^kB-BA^k\). \\
    \item Montrez que le spectre de \(A\) est constant.
\end{enumerate}
\end{exoss}

\begin{exoss}
Soit \(p:\R\to\Rp\) une fonction continue non-identiquement nulle.

\begin{enumerate}
    \item Montrez que toute solution sur \(\R\) de l'équation différentielle \(y\seconde+p\paren{t}y=0\) s'annule en au moins un point. \\
    \item Soit \(y\) une solution de l'équation différentielle \(y\seconde-p\paren{t}y=0\) autre que la fonction nulle. \\\\ Montrez que \(y\) et \(y\prim\) s'annulent au plus une fois. \\\\ \textit{(Indication : calculez l'intégrale \(\int_a^bp\paren{t}y^2\paren{t}\odif{t}\) et étudiez les conséquences d'une annulation de \(y\) en \(a\) et en \(b\).)}
\end{enumerate}
\end{exoss}

\begin{exoss}
Déterminez toutes les fonctions \(f:\R\to\R\) dérivables sur \(\R\) telles que \(\quantifs{\forall x\in\R}f\prim\paren{x}+f\paren{-x}=\e{x}\).
\end{exoss}

\begin{exoss}
Soit \(k\) un réel.

Déterminez les fonctions \(f:\R\to\R\) dérivables sur \(\R\) telles que \(\quantifs{\forall x\in\R}f\prim\paren{x}=f\paren{k-x}\).
\end{exoss}

\begin{exoss}
Déterminez les fonctions \(f\) continues sur \(\R\) telles que \(\quantifs{\forall x\in\R}f\paren{x}=\int_0^x\paren{x-t}f\paren{t}\odif{t}+\cos x\).
\end{exoss}

\begin{exoss}[Une équation différentielle non-linéaire]
Résolvez l'équation différentielle \[2t\e{y\paren{t}}y\prim\paren{t}+\e{y\paren{t}}-t^2=0\] à l'aide du changement de fonction inconnue \(u=\e{y}\).

\textit{On étudiera avec soin l'ensemble de définition de chaque solution \(y\).}
\end{exoss}
