\chapter{Structures algébriques}

\minitoc

\legendeexercices

Dans un groupe, deux éléments \(x,y\) sont dits conjugués quand il existe \(g\) tel que \(y=gxg\inv\).

~\\

\begin{exos}
Soit \(\groupe{G}[.]\) un groupe de neutre \(e\) tel que \(\quantifs{\tpt x\in G}x^2=e\).

\begin{enumerate}
    \item Montrez que \(G\) est un groupe abélien. \\
    \item Soient \(H\) un sous-groupe propre de \(G\) (\ie \(H\not=G\)) et \(a\not\in H\). Montrez que \(H\union aH\) est un sous-groupe de \(G\).
\end{enumerate}
\end{exos}

\begin{exoss}
Cet exercice prolonge le précédent : on suppose de plus que \(G\) est fini.

Montrez que le cardinal de \(G\) est une puissance de \(2\).
\end{exoss}

\begin{exoss}
Soit \(\groupe{G}[.]\) un groupe. Pour \(g\in G\), on pose \(\fonction{\gamma_g}{G}{G}{x}{gx}\).

\begin{enumerate}
    \item Montrez que \(\quantifs{\tpt g\in G}\gamma_g\) est une permutation de \(G\). À quelle condition \(\gamma_g\) est-elle un automorphisme (de groupes) ? \\
    \item Montrez que l'application \(\fonction{\phi}{G}{\S{G}}{g}{\gamma_g}\) est un morphisme injectif de groupes. \\
    \item Déduisez-en le théorème de Cayley : tout groupe \(G\) est isomorphe à un sous-groupe de \(\S{G}\). Autrement dit, pour connaître tous les types de groupes, il suffit de connaître uniquement les groupes symétriques et leurs sous-groupes.
\end{enumerate}
\end{exoss}

\begin{exoss}
Soient \(\groupe{G}[.]\) un groupe et \(H\) une partie finie de \(G\) stable par multiplication.

Montrez que \(H\) est un sous-groupe de \(G\).

\textit{Indication : considérez les applications \(\gamma_g\) précédentes.}
\end{exoss}

\begin{exoss}
Soient \(G,G\prim\) deux groupes, \(G\) étant fini, et \(f:G\to G\prim\) un morphisme de groupes.

Montrez que \(\Card\paren{\ker f}\times\Card\paren{\Im f}=\Card G\).

\textit{Indication : utilisez la relation d'équivalence vue dans la démonstration du \thref{theo:Lagrange} : \(x\sim y\ssi x\inv y\in\ker f\).}
\end{exoss}

\begin{exoss}
Soit \(n\in\Ns\). Montrez que \(\S{n}\) est engendré par les parties suivantes :

\begin{enumerate}
    \item l'ensemble \(A\) des transpositions \(\cycle{i;i+1}\) pour \(i\in\interventierii{1}{n-1}\) ; \\
    \item l'ensemble \(B\) des transpositions \(\cycle{1;i}\) pour \(i\in\interventierii{2}{n}\) ; \\
    \item la paire \(\accol{\cycle{1;2},\cycle{1;2;\dots;n}}\).
\end{enumerate}
\end{exoss}

\begin{exoss}
Soit \(n\in\Ns\).

\begin{enumerate}
    \item Montrez que dans le groupe \(\S{n}\), deux cycles de même longueur sont conjugués. \\
    \item Montrez que deux permutations de \(\S{n}\) sont conjuguées ssi leurs décompositions en produits de cycles à supports disjoints ont le même nombre d'éléments et des cycles de même longueur.
\end{enumerate}
\end{exoss}

\begin{exoss}
Soit \(f:\S{n}\to\Cs\) un morphisme de groupes.

\begin{enumerate}
    \item Que peut valoir \(f\paren{\tau}\) si \(\tau\) est une transposition ? \\
    \item Montrez que deux permutations conjuguées ont la même image par \(f\). \\
    \item Déduisez-en que \(f\) est la fonction constante égale à \(1\) ou alors la signature.
\end{enumerate}
\end{exoss}

\begin{exoss}
Soient \(H\) un sous-groupe de \(\S{n}\) de cardinal \(\dfrac{n!}{2}\) et \(K=\S{n}\excluant H\).

\begin{enumerate}
    \item Si \(\sigma\in H\), montrez que \(\sigma H=H\) et \(\sigma K=K\). \\
    \item Soit \(\sigma\in\S{n}\). Selon que \(\sigma\) soit dans \(H\) ou dans \(K\), déterminez \(\sigma H\), \(\sigma K\), \(H\sigma\) et \(K\sigma\). \\
    \item Déduisez-en que deux permutations conjuguées sont soit toutes les deux dans \(H\), soit toutes les deux dans \(K\). \\
    \item Montrez que \(H=\frakA{n}\).
\end{enumerate}
\end{exoss}

\begin{exos}
Dans un groupe, \(a\) est un élément d'ordre \(n\). Pour \(u\in\Z\), quel est l'ordre de \(a^u\) ?
\end{exos}

\begin{exos}
Donnez un exemple de groupe infini dans lequel tous les éléments sont d'ordre fini.
\end{exos}

\begin{exoss}
Quels sont les sous-groupes finis de \(\groupe{\Cs}[.]\) ?
\end{exoss}

\begin{exoss}
Soit \(G\) un groupe cyclique d'ordre \(n\).

\begin{enumerate}
    \item Montrez que si \(H\) est un sous-groupe de \(G\), alors \(H\) est cyclique d'ordre \(p\) divisant \(n\). \\
    \item Réciproquement, montrez que pour tout diviseur \(p\) de \(n\), \(G\) possède exactement un seul sous-groupe d'ordre \(p\).
\end{enumerate}
\end{exoss}

\begin{exoss}
Soient \(G\) un groupe fini de cardinal \(n\) et \(p\) un diviseur premier de \(n\).

On pose \(E=\accol{\paren{x_1,\dots,x_p}\in G^p\tq x_1x_2\dots x_p=e}\) et \(\sigma:G^p\to G^p\) telle que \(\sigma\paren{x_1,\dots,x_p}=\paren{x_2,\dots,x_p,x_1}\).

Pour \(\paren{X,Y}\in E^2\), on pose \(X\sim Y\) quand il existe \(k\in\Z\) tel que \(Y=\sigma^k\paren{X}\).

\begin{enumerate}
    \item Montrez que \(\sim\) est une relation d'équivalence sur \(E\) et que ses classes d'équivalence ont pour cardinal \(1\) ou \(p\). \\
    \item On note \(r\) le nombre de classes d'équivalence de cardinal \(1\) et \(s\) celui des classes d'équivalence de cardinal \(p\). Donnez une relation liant \(r\), \(s\) et \(n\). \\
    \item Montrez que \(G\) possède au moins un élément d'ordre \(p\).
\end{enumerate}
\end{exoss}

\begin{exos}
Résolvez dans \(\znz{53}\) le système \(\begin{dcases}
\overline{13}x+\overline{7}y=\overline{22} \\
\overline{17}x+\overline{8}y=\overline{5}
\end{dcases}\)
\end{exos}

\begin{exoss}
Montrez que \(2023\) possède un multiple qui ne s'écrit en base \(10\) qu'avec des \(1\).
\end{exoss}

\begin{exoss}
Pour tout \(n\in\Ns\), on pose \(s\paren{n}\) la somme des diviseurs positifs de \(n\).

\begin{enumerate}
    \item Donnez une condition nécessaire et suffisante simple sur \(s\paren{n}\) pour que \(n\) soit premier. \\
    \item Si \(n=p^\alpha\) où \(p\) est premier et \(\alpha\geq1\), calculez \(s\paren{n}\). \\
    \item Soient \(a,b\) deux entiers premiers entre eux. Montrez que \(s\paren{ab}=s\paren{a}\times s\paren{b}\).
\end{enumerate}
\end{exoss}

\begin{exoss}
Déterminez les polynômes \(P\in\poly[\C]\) tels que \(P\prim\) divise \(P\).
\end{exoss}

\begin{exoss}
Soit \(A\) un anneau. Un élément \(x\) de \(A\) est dit nilpotent quand il existe \(n\in\Ns\) tel que \(x^n=0\).

\begin{enumerate}
    \item Si \(A\) est intègre, quels sont les nilpotents de \(A\) ? \\
    \item Montrez que si \(a\) et \(b\) sont deux nilpotents qui commutent, alors \(a+b\) et \(ab\) sont nilpotents. Est-ce encore vrai s'ils ne commutent pas ? \\
    \item Montrez que si \(a\) est nilpotent, alors \(1-a\) est inversible.
\end{enumerate}
\end{exoss}

\begin{exoss}
Soit \(A\) un anneau. On considère l'application \(\fonction{f}{\Z}{A}{k}{k.1_A}\)

\begin{enumerate}
    \item Vérifiez que \(f\) est un morphisme d'anneaux. Montrez qu'il existe un unique \(p\in\N\) tel que \(\ker f=p\Z\). L'entier \(p\) est appelé la caractéristique de l'anneau \(A\). \\
    \item Quelles sont les caractéristiques des anneaux \(\Z\), \(\poly[\R]\) et \(\znz{p}\) ? \\
    \item Montrez que si \(A\) est intègre, alors \(p\) est un nombre premier. \\
    \item On suppose que \(A\) est un corps fini. Montrez qu'il existe \(p\) premier et \(s\in\Ns\) tels que \(\Card A=p^s\).
\end{enumerate}
\end{exoss}

\begin{exosss}
Soit \(A\) un sous-anneau de \(\R\).

Montrez que \(A\) est dense dans \(\R\) ssi \(A\inter\intervee{0}{1}\not=\ensvide\).
\end{exosss}

\begin{exosss}
Pour \(n\in\Ns\), on note \(\SL{n}=\accol{A\in\M{n}\tq\det A=1}\).

\begin{enumerate}
    \item Montrez que \(\SL{n}\) est un sous-groupe de \(\GL{n}\). \\
    \item Montrez que \(\SL{n}\) est engendré par les matrices de transvections.
\end{enumerate}
\end{exosss}

\begin{exo}[Oral TPE, 1]
Un nombre complexe \(a\) est dit algébrique quand il existe un polynôme non-nul \(P\in\poly[\Z]\) tel que \(P\paren{a}=0\). Soit \(a\) un nombre complexe algébrique.

\begin{enumerate}
    \item Montrez qu'il existe un unique polynôme \(\Pi\in\poly[\Q]\), unitaire et irréductible, tel que \(\Pi\paren{a}=0\). On note \(d\) le degré de \(\Pi\). \\
    \item On pose \(\poly[\Q][a]=\accol{P\paren{a}\tq P\in\poly[\Q]}\). Montrez que \(\poly[\Q][a]=\accol{P\paren{a}\tq P\in\polydeg[\Q]{d-1}}\). \\
    \item Montrez que \(\poly[\Q][a]\) est un corps.
\end{enumerate}
\end{exo}

\begin{exo}[Oral IMT, 2]
Soient \(m,n\in\Ns\).

Montrez que si \(m\) divise \(n\), alors \(X^m-1\) divise \(X^n-1\) et étudiez la réciproque.
\end{exo}

\begin{exo}[Oral IMT, 3]
Soit \(P\in\poly[\R]\) simplement scindé.

\begin{enumerate}
    \item Montrez que \(P\prim\) est simplement scindé. \\
    \item Comparez la moyenne arithmétique des racines de \(P\) et de \(P\prim\).
\end{enumerate}
\end{exo}

\begin{exo}[Oral CCMP, 4]
Soit \(p\) un nombre premier impair.

\begin{enumerate}
    \item Dénombrez les carrés de \(\znz{p}\). \\
    \item On suppose que \(p\equiv1\croch{4}\). En calculant la classe de \(\paren{p-1}!\) modulo \(p\) de deux manières différentes, montrez que \(-1\) est un carré modulo \(p\). \\
    \item Montrez la réciproque.
\end{enumerate}
\end{exo}

\begin{exo}[Oral CCMP, 5]
Soit \(n\in\Ns\).

\begin{enumerate}
    \item Caractérisez et dénombrez les inversibles de l'anneau \(\znz{2^n}\). \\
    \item Soit \(x\) un entier impair. Donnez deux algorithmes de calcul de l'inverse de \(x\) modulo \(2^n\), le premier utilisant la suite \(\paren{\overline{x}^{2^k}}\), où \(\overline{x}\) est la classe modulo \(2^n\) de \(x\), le second résolvant successivement pour \(k\in\interventierii{1}{n}\) les congruences \(y_kx\equiv1\croch{2^k}\).
\end{enumerate}
\end{exo}

\begin{exo}[Oral Centrale, 6]
Pour \(P\in\poly[\C]\), on note \(\rac[]{P}\) l'ensemble des racines de \(P\).

Soient \(P,Q\in\poly[\C]\) non-constants tels que \(\rac[]{P}=\rac[]{Q}\) et \(\rac[]{P-1}=\rac[]{Q-1}\).

Montrez que \(P=Q\).
\end{exo}

\begin{exo}[Oral Centrale, 7]
Soit \(P\in\poly[\R]\) de degré au moins \(2\).

\begin{enumerate}
    \item On suppose \(P\) scindé sur \(\R\) et on considère \(x\in\R\) tel que \(P\prim\paren{x}=0\) et \(P\paren{x}\not=0\). En utilisant \(\dfrac{P\prim}{P}\), montrez que \(P\seconde\paren{x}P\paren{x}<0\). \\
    \item Soient \(x,y\) deux racines consécutives de \(P\). Montrez que \(P\prim\paren{x}P\prim\paren{y}\leq0\). \\
    \item Soient \(a<b\) deux réels tels que \(P-a\) et \(P-b\) soient scindés. Montrez que \(P\prim\) est scindé à racines simples.
\end{enumerate}
\end{exo}

\begin{exo}[Oral Centrale, 8]~\\
Pour un groupe fini \(G\), on pose \(p_G=\dfrac{\abs{\accol{\paren{x,y}\in G^2\tq xy=yx}}}{\abs{G\times G}}\) et pour \(x\in G\), \(p_x=\dfrac{\abs{\accol{y\in G\tq xy=yx}}}{\abs{G}}\).

On admet que la relation de conjugaison de deux éléments est une relation d'équivalence sur \(G\) et on note \(N_G\) le nombre de classes d'équivalence pour cette relation.

\begin{enumerate}
    \item Soit \(c=\cycle{a_1;\dots;a_r}\) un \(r\)-cycle de \(\S{n}\). Pour \(\sigma\in\S{n}\), montrez que \(\sigma c\sigma\inv\) est un \(r\)-cycle à préciser. \\
    \item Codez une fonction donnant une permutation aléatoire de \(\S{n}\). Codez une fonction qui approxime \(p_{\S{n}}\) et tracez \(p_{\S{n}}\) pour \(n\in\interventierii{1}{n}\). \\
    \item Justifiez les valeurs de \(p_{\S{1}}\), \(p_{\S{2}}\) et \(p_{\S{3}}\). \\
    \item Montrez que si \(x\) et \(y\) sont conjugués, alors \(p_x=p_y\). \\\\ Montrez que si \(x\) et \(y\) sont conjugués, alors \(\abs{\accol{s\in G\tq y=sxs\inv}}=\abs{\accol{t\in G\tq tx=xt}}\). \\\\ Montrez que \(p_G=\dfrac{N_G}{\abs{G}}\).
\end{enumerate}
\end{exo}

\begin{exo}[Oral Centrale, 9]
Soit \(\K\) une \(\R\)-algèbre commutative intègre de dimension finie \(n\geq2\). En notant \(e\) sont neutre multiplicatif, on identifie \(\R e\) à \(\R\) et donc \(e\) à \(1\).

\begin{enumerate}
    \item On fixe \(a\) non-nul dans \(\K\). Montrez que \(\phi_a:x\mapsto xa\) est un automorphisme d'espaces vectoriels de \(\K\). Déduisez-en que \(a\) est inversible. \\
    \item On fixe \(a\in\K\excluant\R\). Montrez que la famille \(\paren{1,a}\) est libre, mais pas \(\paren{1,a,a^2}\). \\
    \item Montrez qu'on peut trouver \(i\in\K\) tel que \(i^2=-1\). Déduisez-en que \(n=2\) et que \(\K\) est isomorphe à \(\C\).
\end{enumerate}
\end{exo}

\begin{exo}[Oral X, 10]
Quels sont les morphismes de \(\groupe{\Q}\) dans \(\groupe{\Qs}[\times]\) ?
\end{exo}

\begin{exo}[Oral X, 11]~\\
Résolvez l'équation \(\dfrac{1}{a}+\dfrac{1}{b}=\dfrac{n}{a+b}\) d'inconnue \(\paren{a,b,n}\in\paren{\Ns}^3\).
\end{exo}

\begin{exo}[Oral ENS, 12]
Résolvez l'équation \(y^3=x^2+1\) dans \(\Z^2\).
\end{exo}

\begin{exo}[Oral ENS, 13]
Dans un anneau commutatif \(A\), un idéal \(M\) est dit maximal quand \(M\not=A\) et tout idéal qui contient \(M\) est soit \(M\), soit \(A\).

\begin{enumerate}
    \item Soit \(M\) un idéal de \(A\). Montrez que \(M\) est maximal ssi pour tout \(a\not\in M\), il existe \(x\in M\) et \(u\in A\) tels que \(1=x+ua\). \\
    \item Soit \(B\) un anneau commutatif et \(f:A\to B\) un morphisme surjectif d'anneaux. Montrez que si \(M\) est un idéal maximal de \(A\), alors soit \(f\paren{M}=B\), soit \(f\paren{M}\) est un idéal maximal de \(B\). \\
    \item Si \(\K\) est un corps, quels sont les idéaux maximaux de \(\poly\) ? \\
    \item Soit \(M\) un idéal maximal de \(\poly[\Z]\) tel que \(M\inter\Z\not=\accol{0}\). \\\\ Montrez qu'il existe \(p\) premier tel que \(M\inter\Z=p\Z\). \\\\ Montrez qu'il existe deux éléments irréductibles \(P,Q\) de \(\poly[\Z]\) tels que \(M=\paren{P}+\paren{Q}\).
\end{enumerate}
\end{exo}
