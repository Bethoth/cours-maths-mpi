\chapter{Séries numériques et vectorielles : révisions et compléments}

\legendeexercices

\begin{exos}[Exercice 1]
\begin{enumerate}
    \item Montrez que la série de terme général \(u_n=\Arctan\dfrac{1}{2n^2}\) est convergente. \\
    \item Montrez que \(\quantifs{\tpt n\geq1}u_n=\Arctan\dfrac{n}{n+1}-\Arctan\dfrac{n-1}{n}\). \\
    \item Déduisez-en la somme \(\sum_{n=1}^{\pinf}u_n\).
\end{enumerate}
\end{exos}

\begin{corr}
\note{À venir}
\end{corr}

\begin{exos}[Exercice 2]~\\
Justifiez que la série \(\sum_{n\geq2}\dfrac{2n-1}{n^3-n}\) converge et déterminez sa somme (indication : décomposition en éléments simples).
\end{exos}

\begin{corr}
\note{À venir}
\end{corr}

\begin{exoss}[Exercice 3]
Donnez la nature des séries suivantes (\(\alpha\) désigne une constante strictement positive, \(x\) un réel dans \(\intervee{-1}{1}\)) :

\begin{enumerate}
    \item \(\sum_{n\geq2}\dfrac{\ln^nn}{n^{\ln n}}\) \\
    \item \(\sum_{n\geq2}\dfrac{1}{\paren{\ln n}^{\ln n}}\) \\
    \item \(\sum\dfrac{\paren{-1}^n}{n^\alpha}\) \\
    \item \(\sum_{n\geq1}\dfrac{\paren{-1}^nn^\alpha}{n!}\) \\
    \item \(\sum\dfrac{\paren{-1}^n\ln n}{n^\alpha}\) \\
    \item \(\sum\ln\paren{1+x^n}\) \\
    \item \(\sum\dfrac{\sin n}{2^n}\) \\
    \item \(\sum2\ln\paren{n^3+1}-3\ln\paren{n^2+1}\) \\
    \item \(\sum\dfrac{\sin n\sin\frac{1}{n}}{n}\) \\
    \item \(\sum_{n\geq2}\dfrac{\paren{-1}^n}{\paren{-1}^n+n}\) \\
    \item \(\sum_{n\geq0}\sin\paren{\pi\sqrt{n^2+1}}\) \\
    \item \(\sum\sqrt[n+1]{n+1}-\sqrt[n]{n}\) \\
    \item \(\sum_{n\geq1}\dfrac{\paren{-1}^n}{\ln n+\sin\frac{2\pi n}{3}}\) \\
    \item \(\sum_{n\geq1}\paren{\paren{1+\dfrac{1}{n}}^n-\e{}}\) \\
    \item \(\sum_{n\geq1}\dfrac{1}{\ln n\ln\paren{\ch n}}\) \\
    \item \(\sum\int_0^{\nicefrac{\pi}{2}}\dfrac{\cos^2t}{n^2+\cos^2t}\odif{t}\)
\end{enumerate}
\end{exoss}

\begin{corr}
\note{À venir}
\end{corr}

\begin{exoss}[Exercice 4]
Soit \(\paren{a,b}\in\R^2\).

\begin{enumerate}
    \item Déterminez \(a\) et \(b\) pour que la série de terme général \(\ln n+a\ln\paren{n+1}+b\ln\paren{n+2}\) converge. Dans ce cas, donnez la valeur de sa somme. \\
    \item Faites de même avec la série de terme général \(\sqrt{n}+a\sqrt{n+1}+b\sqrt{n+2}\).
\end{enumerate}
\end{exoss}

\begin{corr}
\note{À venir}
\end{corr}

\begin{exoss}[Exercice 5]
Pour quelles valeurs de \(\alpha>0\) la série de terme général \(u_n=\ch^\alpha n-\sh^\alpha n\) converge-t-elle ? Dans ce cas, donnez un équivalent de \(\sum_{k=n}^{\pinf}u_k\).
\end{exoss}

\begin{corr}
\note{À venir}
\end{corr}

\begin{exoss}[Exercice 6]~\\
On pose \(u_n=\dfrac{1}{n^\alpha}\sum_{k=1}^nk^{\nicefrac{3}{2}}\). Pour quelles valeurs de \(\alpha>0\) la série \(\sum_{n\geq1}u_n\) converge-t-elle ? Dans ce cas, donnez un équivalent de \(\sum_{k=n+1}^{\pinf}u_k\).
\end{exoss}

\begin{corr}
\note{À venir}
\end{corr}

\begin{exoss}[Exercice 7, séries associées à des suites définies par récurrence]~\\
\begin{enumerate}
    \item Soit \(u\) la suite définie par récurrence par \(u_1\in\R\) et \(\quantifs{\tpt n\geq1}u_{n+1}=\dfrac{1}{n\e{u_n}}\). \\ Quelle est la nature de la série \(\sum_{n\geq1}u_n\) ? \\
    \item Soit \(u\) la suite définie par récurrence par \(u_0\in\Rps\) et \(\quantifs{\tpt n\geq0}u_{n+1}=\ln\paren{1+u_n}\). \\ Quelle est la nature de la série \(\sum\paren{\dfrac{1}{u_{n+1}}-\dfrac{1}{u_n}}\) ? Puis celle de \(\sum u_n\) ? Donnez un équivalent de \(\sum_{k=0}^nu_k\). \\
    \item Soit \(u\) la suite définie par récurrence par \(u_0\in\intervee{0}{\pi}\) et \(\quantifs{\tpt n\geq0}u_{n+1}=\sin u_n\). \\ Quelle est la nature de la série \(\sum\paren{\dfrac{1}{u_{n+1}^2}-\dfrac{1}{u_n^2}}\) ? Puis celle de \(\sum u_n\) ? Donnez un équivalent de \(\sum_{k=0}^n\).
\end{enumerate}
\end{exoss}

\begin{corr}
\note{À venir}
\end{corr}

\begin{exoss}[Exercice 8]
Soit \(P\in\poly[\R]\) tel que \(\quantifs{\tpt x\in\Rp}P\paren{x}\geq0\). Pour \(n\in\Ns\), on pose \(u_n=\sum_{k=n}^{\pinf}\dfrac{\paren{-1}^k}{\e{k}+P\paren{k}}\).

\begin{enumerate}
    \item Justifiez l'existence de \(u_n\). \\
    \item Montrez que la série \(\sum_{n\geq1}u_n\) converge.
\end{enumerate}
\end{exoss}

\begin{corr}
\note{À venir}
\end{corr}

\begin{exoss}[Exercice 9]~\\
Pour \(n\geq1\), on pose \(u_n=\sum_{k=n}^{\pinf}\dfrac{\paren{-1}^k}{\sqrt{k}}\).

\begin{enumerate}
    \item Justifiez l'existence de \(u_n\). \\
    \item Montrez que \(\dfrac{u_n+u_{n+1}}{2}\) est le reste d'une série alternée absolument convergente. \\
    \item Déduisez-en la nature de la série \(\sum u_n\).
\end{enumerate}
\end{exoss}

\begin{corr}
\note{À venir}
\end{corr}

\begin{exoss}[Exercice 10, utilisation de développements limités ou asymptotiques]~\\
\begin{enumerate}
    \item Montrez que la série \(\sum_{n\geq2}\dfrac{\paren{-1}^n}{n-\sqrt{n}}\) converge. \\
    \item Montrez que la série \(\sum_{n\geq2}\ln\paren{1+\dfrac{\paren{-1}^n}{n}}\) converge. \\
    \item Montrez que la série \(\sum_{n\geq2}\dfrac{1}{1+\paren{-1}^nn}\) converge. \\
    \item Montrez que la série \(\sum_{n\geq0}\sin\paren{\pi\sqrt{n^2+1}}\) converge. \\
    \item Déterminez la nature de la série \(\sum_{n\geq2}\paren{-1}^n\dfrac{\ln\paren{n+\paren{-1}^n\sqrt{n}}}{n}\). \\
    \item Déterminez la nature de la série \(\sum_{n\geq2}\dfrac{1}{1+\paren{-1}^n\sqrt{n}}\).
\end{enumerate}
\end{exoss}

\begin{corr}
\note{À venir}
\end{corr}

\begin{exoss}[Exercice 11]
Pour quelles valeurs de \(\alpha>0\) la série \(\sum_{n\geq2}\dfrac{\paren{-1}^n}{\sqrt{\paren{-1}^n+n^\alpha}}\) converge-t-elle ?
\end{exoss}

\begin{corr}
\note{À venir}
\end{corr}

\begin{exoss}[Exercice 12, formule de Stirling]~\\
Montrez que la suite de terme général \(\dfrac{n!}{\sqrt{n}}\paren{\dfrac{\e{}}{n}}^n\) converge vers un réel strictement positif \(L\) (indication : passer au logarithme et penser à une série).

Soit \(u_n=\int_0^{\nicefrac{\pi}{2}}\cos^nt\odif{t}\). On montre que \(u_n\simqd{n\to\pinf}\sqrt{\dfrac{\pi}{2n}}\) et que \(u_{2n}=\dfrac{\pi\paren{2n}!}{2^{2n+1}\paren{n!}^2}\). En admettant ces résultats, montrez la formule de Stirling : \[n!\simqd{n\to\pinf}\sqrt{2\pi n}\paren{\dfrac{n}{\e{}}}^n.\]
\end{exoss}

\begin{corr}
\note{À venir}
\end{corr}

\begin{exoss}[Exercice 13]~\\
Soient \(a\in\R\) et \(u_n=\dfrac{\paren{-1}^{\frac{n\paren{n+1}}{2}}}{n^a}\).

\begin{enumerate}
    \item Dans le cas où \(a\leq0\) ou \(a>1\), quelle est la nature de la série \(\sum_{n\geq1}u_n\) ? \\
    \item On suppose désormais que \(0<a\leq1\) et on pose \(v_n=u_{2n-1}+u_{2n}\). Montrez que la série \(\sum v_n\) converge. Déduisez-en la nature de la série \(\sum u_n\).
\end{enumerate}
\end{exoss}

\begin{corr}
\note{À venir}
\end{corr}

\begin{exoss}[Exercice 14]
Soient \(u\) une suite strictement positive et \(\alpha>0\).

Montrez que les séries de termes généraux \(u_n\), \(v_n=\dfrac{u_n}{1+u_n}\), \(w_n=\ln\paren{1+u_n}\) et \(x_n=\int_0^{u_n}\dfrac{1}{1+x^\alpha}\odif{x}\) sont de même nature.
\end{exoss}

\begin{corr}
\note{À venir}
\end{corr}

\begin{exoss}[Exercice 15]
Soit \(u\) une suite réelle qui ne s'annule pas telle que \(\dfrac{u_{2n+1}}{u_{2n}}\tendqd{n\to\pinf}a\) et \(\dfrac{u_{2n}}{u_{2n-1}}\tendqd{n\to\pinf}b\).

Montrez que si \(\abs{ab}<1\), alors la série \(\sum u_n\) converge.
\end{exoss}

\begin{corr}
\note{À venir}
\end{corr}

\begin{exoss}[Exercice 16]
Soit \(u\) une suite réelle positive décroissante.

Montrez que si \(\sum u_n\) converge, alors \(u_n=\o{\dfrac{1}{n}}\) quand \(n\) tend vers \(\pinf\).

La réciproque est-elle vraie ?
\end{exoss}

\begin{corr}
\note{À venir}
\end{corr}

\begin{exoss}[Exercice 17]
Soit \(\paren{u_n}_{n\geq1}\) une suite réelle strictement positive et bornée telle que la série \(\sum_{n\geq1}u_n\) diverge. Pour \(n\in\Ns\), on pose \(S_n\) la somme partielle d'indice \(n\) de la série \(\sum_{n\geq1}u_n\).

\begin{enumerate}
    \item Montrez que \(\dfrac{u_n}{S_n}\sim\ln\dfrac{S_n}{S_{n+1}}\). Déduisez-en la nature de la série \(\sum_{n\geq1}\dfrac{u_n}{S_n}\). \\
    \item Étudiez la série \(\sum_{n\geq1}\dfrac{u_n}{S_n^\alpha}\) quand \(\alpha\in\intervee{0}{1}\). \\
    \item Soit \(\alpha>1\). Montrez que \(\dfrac{u_n}{S_n^\alpha}\leq\int_{S_{n-1}}^{S_n}\dfrac{1}{x^\alpha}\odif{x}\). Déduisez-en la nature de la série \(\sum_{n\geq1}\dfrac{u_n}{S_n^\alpha}\).
\end{enumerate}
\end{exoss}

\begin{corr}
\note{À venir}
\end{corr}

\begin{exoss}[Exercice 18]
Soit \(u\) une suite strictement positive. Pour \(n\in\N\), on pose \(s_n=\sum_{k=0}^nu_k\).

On suppose que \(u_ns_n\) tend vers \(1\) quand \(n\) tend vers \(\pinf\). Déterminez un équivalent simple de \(u_n\).
\end{exoss}

\begin{corr}
\note{À venir}
\end{corr}

\begin{exoss}[Exercice 19]
Soit \(u\) la suite définie par récurrence par \(u_0>0\) et \(\quantifs{\tpt n\in\N}u_{n+1}=u_n+u_n^2\). On pose \(v_n=\dfrac{1}{2^n}\ln u_n\), puis \(w_n=v_{n+1}-v_n\).

\begin{enumerate}
    \item Montrez que la suite \(\paren{u_n}\) diverge vers \(\pinf\). \\
    \item Montrez que la suite \(\paren{v_n}\) converge vers un réel \(l>0\). On pose alors \(A=\e{l}>1\). \\
    \item Montrez que \(u_n\sim A^{2^n}\).
\end{enumerate}
\end{exoss}

\begin{corr}
\note{À venir}
\end{corr}

\begin{exosss}[Exercice 20, transformation d'Abel]
Soient \(u\) une suite réelle et \(v\) une suite complexe. Pour \(n\in\N\), on pose \(V_n=\sum_{k=0}^nv_k\).

On suppose que la suite \(u\) est positive et décroissante de limite nulle et que la suite \(V\) est bornée.

\begin{enumerate}[series=transformationAbel]
    \item Montrez que \(\quantifs{\tpt n\in\N}\sum_{k=0}^nu_kv_k=u_nV_n-\sum_{k=0}^{n-1}\paren{u_{k+1}-u_k}V_k\). \\
    \item Déduisez-en que la série \(\sum u_nv_n\) converge.
\end{enumerate}

Applications :

\begin{enumerate}[resume=transformationAbel]
    \item Soit \(w\) une suite complexe telle que \(\sum w_n\) converge. Montrez que \(\quantifs{\tpt a>0}\sum\dfrac{w_n}{n^a}\) converge aussi. \\
    \item Soient \(a>0\) et \(\theta\in\R\). Donnez la nature des séries \(\sum\dfrac{\e{\i n\theta}}{n^a}\), \(\sum\dfrac{\cos\paren{n\theta}}{n^a}\) et \(\sum\dfrac{\sin\paren{n\theta}}{n^a}\). \\
    \item Montrez que \(\quantifs{\tpt x\in\R}\abs{\sin x}\geq\dfrac{1-\cos\paren{2x}}{2}\). Déterminez la nature des séries \(\sum\dfrac{\abs{\cos\paren{n\theta}}}{n^a}\) et \(\sum\dfrac{\abs{\sin\paren{n\theta}}}{n^a}\).
\end{enumerate}
\end{exosss}

\begin{corr}
\note{À venir}
\end{corr}

\begin{exosss}[Exercice 21]
Soit \(u\) une suite positive de limite nulle. On appelle \(U_n\) la somme partielle d'indice \(n\) de la série \(\sum u_n\) et on suppose qu'il existe une constante \(M>0\) telle que \(\quantifs{\tpt n\in\N}\abs{U_n-nu_n}\leq M\).

\begin{enumerate}
    \item Montrez que \(\quantifs{\tpt n\geq2}\abs{\dfrac{U_n}{n}-\dfrac{U_{n-1}}{n-1}}\leq M\paren{\dfrac{1}{n-1}-\dfrac{1}{n}}\). \\
    \item Montrez que la série \(\sum u_n\) converge.
\end{enumerate}
\end{exosss}

\begin{corr}
\note{À venir}
\end{corr}

\begin{exosss}[Exercice 22]
Soit \(\sum_{n\geq1}u_n\) une série convergente à termes positifs.

\begin{enumerate}
    \item Montrez que \(\dfrac{u_1+2u_2+\dots+nu_n}{n}\) tend vers \(0\) quand \(n\) tend vers \(\pinf\). \\
    \item Montrez que la série \(\sum\dfrac{u_1+2u_2+\dots+nu_n}{n\paren{n+1}}\) converge et montrez que sa somme est la même que celle de la série \(\sum_{n\geq1}u_n\).
\end{enumerate}
\end{exosss}

\begin{corr}
\note{À venir}
\end{corr}

\begin{exo}[Oral Saint-Cyr, 1]~\\
Pour \(n\in\Ns\), on pose \(u_n=\dfrac{\ln n}{n}\). Déterminez la nature de la série \(\sum u_n\). Donnez un équivalent de \(\sum_{k=1}^nu_k\) quand \(n\to\pinf\).
\end{exo}

\begin{corr}
\note{À venir}
\end{corr}

\begin{exo}[Oral IMT, 2]
Soit \(\alpha>0\). Donnez un équivalent de \(\sum_{k=1}^n\ln^\alpha k\) quand \(n\to\pinf\).
\end{exo}

\begin{corr}
\note{À venir}
\end{corr}

\begin{exo}[Oral CCINP, 3]
Soit \(u\) la suite définie par récurrence par \(u_0>0\) et \(\quantifs{\tpt n\in\N}u_{n+1}=u_n\e{-u_n}\).

\begin{enumerate}
    \item Montrez que \(\paren{u_n}\) converge et déterminez sa limite. \\
    \item Déterminez la limite de \(\dfrac{1}{u_{n+1}}-\dfrac{1}{u_n}\). Déduisez-en un équivalent de \(u_n\).
\end{enumerate}
\end{exo}

\begin{corr}
\note{À venir}
\end{corr}

\begin{exo}[Oral CCINP, 4]
Montrez que pour \(n\geq1\), l'équation \(x^n+\sqrt{n}x-1=0\) admet une unique solution \(x_n\) dans \(\intervii{0}{1}\).

Étudiez la suite \(\paren{x_n}\) et montrez qu'elle converge vers \(0\).

Trouvez un équivalent de \(x_n\) et étudiez la convergence de la série \(\sum_{n\geq1}x_n\).
\end{exo}

\begin{corr}
\note{À venir}
\end{corr}

\begin{exo}[Oral CCINP, 5]
Montrez que la suite \(\paren{u_n}\) définie par \(u_0\in\intervie{0}{1}\) et \(u_{n+1}=\dfrac{1}{2}\paren{u_n+u_n^2}\) converge vers \(0\) et donnez la nature de la série \(\sum u_n\).
\end{exo}

\begin{corr}
\note{À venir}
\end{corr}

\begin{exo}[Oral CCINP, 6]
Quelle est la nature de la série de terme général \(\ln\paren{1+\dfrac{\paren{-1}^n}{\sqrt{n\paren{n+1}}}}\) ?
\end{exo}

\begin{corr}
\note{À venir}
\end{corr}

\begin{exo}[Oral CCINP, 7]
Soient \(x,y>0\). Représentez graphiquement l'ensemble des couples \(\paren{x,y}\) tels que la série \(\sum\dfrac{x^n}{y^n+n^x}\) converge.
\end{exo}

\begin{corr}
\note{À venir}
\end{corr}

\begin{exo}[Oral CCMP, 8]
Étudiez la convergence de la suite \(\paren{a_n}\) définie par \(a_0>0\) et \(a_{n+1}=1-\e{-a_n}\).

Déterminez la nature des séries \(\sum\paren{-1}^na_n\) et \(\sum a_n^2\).

Déterminez la nature de la série \(\sum a_n\) (on pourra étudier la série \(\sum\ln\dfrac{a_{n+1}}{a_n}\)).
\end{exo}

\begin{corr}
\note{À venir}
\end{corr}

\begin{exo}[Oral CCINP, 9]
Soient \(\paren{a_n}\) une suite positive et \(\paren{u_n}\) définie par \(u_0>0\) et \(\quantifs{\tpt n\in\N}u_{n+1}=\dfrac{u_n+\sqrt{u_n^2+a_n^2}}{2}\).

\begin{enumerate}
    \item Montrez que \(\quantifs{\tpt n\in\N}u_{n+1}-u_n\leq\dfrac{a_n}{2}\). \\
    \item Montrez que si la série \(\sum a_n\) converge, alors la suite \(\paren{u_n}\) converge. \\
    \item La réciproque est-elle vraie ? Indication : considérer \(u_n=\dfrac{n}{n+1}\).
\end{enumerate}
\end{exo}

\begin{corr}
\note{À venir}
\end{corr}

\begin{exo}[Oral IMT, 10]
Soit \(\paren{u_n}\) la suite définie par \(u_0>0\) et \(\quantifs{\tpt n\in\N}u_{n+1}=\dfrac{n+1}{n+3}u_n\). On pose \(\paren{v_n}=\paren{n^2u_n}\).

\begin{enumerate}
    \item Déterminez la nature de la série de terme général \(\ln\dfrac{v_{n+1}}{v_n}\). \\
    \item Déduisez-en la nature de la série de terme général \(u_n\).
\end{enumerate}
\end{exo}

\begin{corr}
\note{À venir}
\end{corr}

\begin{exo}[Oral Centrale, 11]~\\
Montrez que la série \(\sum_{n\geq2}\dfrac{1}{n^2-1}\) converge et donnez la valeur de sa somme.

Montrez que la série \(\sum_{n\geq2}\dfrac{n^2}{\paren{n^2+1}^2}\) converge et donnez une valeur de \(n\) pour que sa somme partielle soit une valeur approchée de sa somme à \(10^{-4}\) près.
\end{exo}

\begin{corr}
\note{À venir}
\end{corr}

\begin{exo}[Oral Centrale, 12]~\\
\begin{enumerate}
    \item Pour \(n\in\Ns\), on pose \(H_n=\sum_{k=1}^n\dfrac{1}{k}\). Montrez que la suite \(\paren{H_n-\ln n}_{n\geq1}\) converge. \\
    \item Déduisez de la question précédente la valeur de \(\sum_{k=1}^{\pinf}\dfrac{\paren{-1}^{k+1}}{k}\). \\
    \item Pour \(s>1\), on pose \(\zeta\paren{s}=\sum_{k=1}^{\pinf}\dfrac{1}{k^s}\). Calculez \(\sum_{n=2}^{\pinf}\dfrac{\zeta\paren{n}-1}{n}\).
\end{enumerate}
\end{exo}

\begin{corr}
\note{À venir}
\end{corr}

\begin{exo}[Oral Centrale, 13]
Si \(\paren{u_n}\) est une suite réelle telle que \(u_0=0\), on pose alors \(\paren{v_n}_{n\geq1}\) la suite définie par : \(\quantifs{\tpt n\in\Ns}v_n=n\paren{u_n-u_{n-1}}\).

On note \(P_1\) la propriété \guillemets{la série \(\sum_{n\geq1}v_n\) converge} et \(P_2\) la propriété \guillemets{il existe \(l\) tel que \(u_n\to l\) et \(\sum\paren{l-u_n}\) converge}.

\begin{enumerate}
    \item Si \(\alpha\in\R\) et \(u_n=\Arctan n^\alpha\) pour \(n\geq1\), étudiez la véracité des propositions \(P_1\) et \(P_2\). \\
    \item Soit \(\paren{a_n}_{n\geq1}\) une suite réelle telle que \(\sum a_n\) converge. Montrez que \(\sum\dfrac{a_n}{n}\) converge et que \(\sum_{k=n}^{\pinf}\dfrac{a_k}{k}=\o{\dfrac{1}{n}}\). \\
    \item Comparez les propriétés \(P_1\) et \(P_2\).
\end{enumerate}
\end{exo}

\begin{corr}
\note{À venir}
\end{corr}

\begin{exo}[Oral CCMP, 14]~\\
Soit \(\alpha\in\Rps\excluant\accol{\dfrac{1}{2}}\). La série \(\sum_{n\geq2}\dfrac{\paren{-1}^n}{\sqrt{n}+\paren{-1}^nn^\alpha}\) converge-t-elle ?
\end{exo}

\begin{corr}
\note{À venir}
\end{corr}

\begin{exo}[Oral CCMP, 15]~\\
Soient \(\alpha>0\) et \(R_n=\sum_{k=n+1}^{\pinf}\dfrac{\paren{-1}^k}{k^\alpha}\).

\begin{enumerate}
    \item On suppose \(\alpha>1\). Montrez que \(\sum_{k=1}^nR_k=\paren{n+1}R_n+\sum_{k=1}^n\dfrac{\paren{-1}^k}{k^{\alpha-1}}\). Déduisez-en la convergence de la série \(\sum R_n\). \\
    \item Étudiez le cas \(\alpha\leq1\).
\end{enumerate}
\end{exo}

\begin{corr}
\note{À venir}
\end{corr}

\begin{exo}[Oral CCMP, 16]~\\
Soit \(\paren{a_n}\in\paren{\Rps}^{\Ns}\).

\begin{enumerate}
    \item On suppose que la série \(\sum a_n^{1-\nicefrac{1}{n}}\) converge. Montrez que la série \(\sum a_n\) converge. \\
    \item On suppose que la série \(\sum a_n\) converge. Montrez que la série \(\sum a_n^{1-\nicefrac{1}{n}}\) converge. \\ Vous introduirez, pour \(\lambda>1\), l'ensemble \(\accol{n\in\Ns\tq a_n^{1-\nicefrac{1}{n}}\leq\lambda a_n}\) et son complémentaire. \\
    \item Généralisez en remplaçant \(a_n^{1-\nicefrac{1}{n}}\) par \(a_n^{1-b_n}\) avec une hypothèse adéquate sur la suite \(\paren{b_n}\).
\end{enumerate}
\end{exo}

\begin{corr}
\note{À venir}
\end{corr}

\begin{exo}[Oral CCMP, 17]
Soit \(f\) une permutation de \(\Ns\). On pose \(E\paren{f}=\accol{\alpha\in\R\tq\sum\dfrac{f\paren{n}}{n^\alpha}\text{ converge}}\).

\begin{enumerate}
    \item Montrez que \(E\paren{f}\) peut être vide. Montrez dans le cas contraire que \(E\paren{f}\) est un intervalle minoré par \(2\) et non-majoré. \\
    \item Soit \(B\geq2\). Montrez l'existence de \(f\) telle que \(E\paren{f}=\intervee{B}{\pinf}\).
\end{enumerate}
\end{exo}

\begin{corr}
\note{À venir}
\end{corr}

\begin{exo}[Oral X, 18]~\\
Pour \(n\in\Ns\), on pose \(u_n=\sum_{k=1}^n\dfrac{1}{\sqrt{k}}-2\sqrt{n}\).

\begin{enumerate}
    \item Montrez que la suite \(\paren{u_n}_{n\geq1}\) converge. On note \(l\) sa limite. \\
    \item Montrez que \(l=-\paren{1+\sqrt{2}}\sum_{k=1}^{\pinf}\dfrac{\paren{-1}^{k-1}}{\sqrt{k}}\).
\end{enumerate}
\end{exo}

\begin{corr}
\note{À venir}
\end{corr}

\begin{exo}[Oral X, 19]
Soit \(\sum x_n\) une série absolument convergente de réels.

\begin{enumerate}
    \item Montrez que pour tout réel \(p\geq1\), la série \(\sum\abs{x_n}^p\) converge. \\
    \item Déterminez la limite de \(\sum_{n=0}^{\pinf}\abs{x_n}^p\) quand \(p\to\pinf\).
\end{enumerate}
\end{exo}

\begin{corr}
\note{À venir}
\end{corr}

\begin{exo}[Oral X, 20]
Soient \(\alpha>0\) et \(\paren{u_n}\) une suite de réels strictement positifs telle que \(\dfrac{u_{n+1}}{u_n}=1-\dfrac{\alpha}{n}+\O{\dfrac{1}{n^2}}\).

Étudiez la convergence de la série \(\sum u_n\).
\end{exo}

\begin{corr}
\note{À venir}
\end{corr}
